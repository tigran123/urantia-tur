\upaper{32}{Yerel Evrenler’in Evrimi}
\vs p032 0:1 Bir yerel evren, Mikâil’in Cennet düzeyine ait olan bir Yaratan Evladı’nın kendi eseridir. Bu türden bir evren, her biri yerleşik dünyaların yüz sistemini içine alan yüz takımyıldızından meydana gelmektedir. Her sistem nihai olarak, yaklaşık bin yerleşik âlemi içinde barındıracaktır.
\vs p032 0:2 Zaman ve mekânın bu evrenlerinin tümü evrimseldir. Cennet Mikâilleri’nin yaratıcı tasarımı her zaman; bu türden bir yerel evreni meydana getiren âlemlerin değişen düzeylerin içinde barındığı çok katmanlı olan yaratılmışların fiziksel, ussal ve ruhsal olan doğaları ve yetkinliklerinin düzenli bir biçimde olan evriminin ve ilerleyici gelişmesinin doğrultusu boyunca ilerler.
\vs p032 0:3 Urantia; Nasıralı İsa ve Salvington’un Mikâil’i olarak, Nebadon’un Tanrı\hyp{}insanı’nın egemen olduğu yerel bir evrene aittir. Buna ek olarak Mikâil’in bu yerel evren için sahip olduğu tasarılarının tümü, mekânın yüce serüvenine başlamasından önce Cennet Kutsal Üçlemesi tarafından bütünüyle kabul edilmiştir.
\vs p032 0:4 Tanrı’nın Evlatları, yaratan eylemlerinin âlemlerini seçebilirler; fakat bu maddi yaratımlar özgün olarak Üstün Evren’in Cennet Mimarları tarafından tasarlanmış ve arzu edilen bir biçimde hayata geçirilmişlerdir.
\usection{1.\bibnobreakspace Evrenlerin Fiziksel Ortaya Çıkışı}
\vs p032 1:1 Mekân\hyp{}kuvvet ve kökensel enerjilerin evren öncesi düzenlenmeleri, Cennet Üstün Kuvvet Düzenleyicileri’nin görevidir; fakat aşkın\hyp{}evren nüfuz alanları içinde ortaya çıkan enerji yerel veya doğrusal olan çekime karşılık gösterir bir hale geldiğinde onlar, ilgili aşkın\hyp{}evrenin güç yöneticilerinin yararına görevlerinden ayrılır.
\vs p032 1:2 Bu güç yöneticileri, yerel bir evren yaratımının madde\hyp{}öncesi ve kuvvet\hyp{}sonrası fazları içinde yalnız başına faaliyet gösterir. Bir Yaratan Evlat’ın evren düzenlemesine başlaması; güç yöneticilerinin ortaya çıkan evrene, bağımlı güneşler ve maddi âlemler biçimindeki maddi bir oluşumu yeterli bir biçimde sağlamak için mekân\hyp{}enerjilerini harekete geçirdiği ana kadar mümkün değildir.
\vs p032 1:3 Her ne kadar onlar fiziksel boyutlar bakımından büyük bir ölçüde farklılık gösterseler de ve zaman zaman dışsal\hyp{}maddi içerik bakımından değişiklik arz etseler de, yerel evrenlerin tümü yaklaşık olarak aynı enerji potansiyelinin bir parçasıdır. Yerel bir evrenin güç etkisi ve potansiyel\hyp{}maddi edinimi; Yaratan Evlat’ın etkinliklerine ek olarak güç yöneticileri ve onlardan önce gelen unsurların düzenlemeleri ve onun yaratıcı birlikteliğinin elinde bulundurduğu içkin fiziksel denetimin edinimi tarafından belirlenmektedir.
\vs p032 1:4 Bir yerel evrenin enerji etkisi yaklaşık olarak, ona ait olan aşkın evrenin kuvvet ediniminin yüz binde biridir. Yerel evreniniz olan Nebadon’un durumunda ise kütlenin maddeleşmesi, önemsiz derece olan bir biçimde daha azdır. Fiziksel olarak bahsedecek olursak Nebadon, Orvonton yerel yaratımlarının herhangi biri içinde bulunabilecek olan enerji ve maddenin fiziksel edinimlerinin tümünü elinde bulundurmaktadır. Nebadon evreninin gelişimsel büyümesi üzerinde olan tek fiziksel kısıtlama; bir araya gelen evren işleyiş düzenin birliktelik halindeki güçleri ve kişiliklerinin çekim düzenleyicileri tarafından alıkonulan mekân\hyp{}enerjisinin niceliksel olan etkisinden meydana gelmektedir.
\vs p032 1:5 Enerji\hyp{}maddesi, kütlenin maddeleşmesi sürecinde mevcut bir düzeye eriştiğinde; bir Cennet Yaratan Evladı, Sınırsız Ruhaniyet’in bir Yaratıcı Kız Evladı tarafından eşlik edilen bir biçimde bu durumda ortaya çıkar. Yaratan Evlat’ın varışıyla birlikte eş zamanlı olarak bahse konu görev, tasarlanan yerel evrenin yönetim merkezi dünyaları haline gelecek olan mimari âlem üzerinde başlar. Takımyıldız yönetim merkezleri ve sistem başkentleri olarak hizmet etmek için bulunan mimari dünyaların yaratımlarının görevi devam ederken; uzun çağlar boyunca bu türden yerel bir yaratım evrimleşmekte, güneşler sabitleşmekte, gezegenler yörüngeleri oluşturmakta onlar etrafında dönmektedir.
\usection{2.\bibnobreakspace Evrenin İşleyişsel Düzenlenişi}
\vs p032 2:1 Evrenin işleyişsel düzenlenişi içinde Yaratan Evlatlar, Üçüncü Kaynak ve Merkez’den kökenini alan güç yöneticileri ve diğer varlıkları takip ederler. Yaratan Evlat’ınız olan Mikâil böylelikle, önceden işleyişsel olarak düzenlenmiş olan mekânın enerjilerinden Nebadon’a ait olan evrenin yerleşik âlemlerini oluşturmuştur; ve bahse konu bu zamandan itibaren kendisi, onların idaresine titizlikle atanmış bir halde bulunmaktadır. Mevcudiyet\hyp{}öncesi enerjiden bahse konu bu kutsal Evlatlar; görülebilen maddeyi oluşturmakta, yaşayan yaratımları sağlamakta ve Sınırsız Ruhaniyet’in evren mevcudiyetinin eş güdümüyle birlikte ruhaniyet kişiliklerinin çeşitlilik arz eden bir maiyetini yaratmaktadır.
\vs p032 2:2 Evrenin işleyişsel düzenlemesinin başlangıçsal fiziksel görevi içinde Yaratan Evlat’dan çok önce görev yapan bu güç yöneticileri ve enerji düzenleyicileri; Yaratan Evlat’ın mevcudiyeti sonrasında Evren Evladı ile muhteşem bir birliktelik içinde, onların özgün olarak düzenledikleri ve döngüleştirdikleri bu enerjilerin birliktelik haline getirilmiş olan denetiminde sonsuza kadar kalmaya devam eder. Salvington üzerinde şu an, bu yerel evrenin özgün oluşumu içinde Yaratan Evlat’ınız ile birlikte eş güdümde bulunan bahse konu bu yüz güç merkezi faaliyet halinde bulunmaktadır.
\vs p032 2:3 Nebadon içindeki fiziksel yaratımın ilk tamamlanmış eylemi; Salvington’un mimari âlemi biçimindeki yönetim merkezi dünyasının uyduları ile birlikte olan işleyişsel düzenlenişinden oluşmuştur. Güç merkezi ve fiziksel düzenleyicilerin öncül devinimlerinin gerçekleştiği andan Salvington’un tamamlanmış âlemleri üzerinde yaşayan görevlilerin varışına kadarki olan bu ara süreç, mevcut olan gezegensel zamanınızın bir milyar yılından biraz daha fazla sürmüştür. Salvington’un inşası eş zamanlı olarak; tasarlanan takımyıldızlarının yüz yönetim merkezi dünyasına ek olarak gezegensel denetimin ve idarenin tasarlanan yerel sistemlerinin mimari uyduları ile birlikte yaratımlarını izlemiştir. Bu türden mimari dünyalar; varlığın ara düzeyi olan morontia veya geçiş aşamalarına ek olarak, fiziksel ve ruhsal varlıkların ikisine birden yerleşke alanı sağlamak için tasarlanmıştır.
\vs p032 2:4 Nebadon’un yönetim merkezi olan Salvington, yerel evrenin tam olarak enerji\hyp{}kütle merkezinde konumlandırılmıştır. Fakat sizin yerel evreniniz tek bir gökbilimsel sistem değildir, gerçekte daha büyük bir sistem onun fiziksel merkezinde mevcut bir durumda bulunmaktadır.
\vs p032 2:5 Salvington, Nebadonlu Mikâil’e ait olan kişisel yönetim merkezidir; fakat o her zaman burada bulunmamaktadır. Her ne kadar yerel evreniniz, başkent âleminde Yaratan Evlat’ın sabit mevcudiyetine ihtiyaç duymasa da; bu durum, fiziksel düzenlenmenin önceki çağları için geçerli bulunmamaktaydı. Karşılıklı maddi çekim tarafından birbirlerini dengelemek için çeşitli döngüleri ve sistemleri etkin hale getirmek amacıyla yeterli enerjinin maddileştirilmesi boyunca âlemin çekimsel sabitleştirilmesinin harekete geçirildiği bu türden bir zamana kadar bir Yaratan Evlat, kendi yönetim merkezinden ayrılmaya yetkin değildir.
\vs p032 2:6 Mevcut an içerisinde bir âlemin fiziksel tasarımı tamamlanmış olup; Yaratıcı Ruhaniyet ile birliktelik içerisindeki Yaratan Evlat, tasarımını ve yaşam yaratımını hayata geçirmektedir. Bu durumun hemen sonrasında Sınırsız Ruhaniyet’in bu temsili, farklı bir yaratıcı kişiliği olarak kendi evren faaliyetine başlamaktadır. İlk yaratıcı eylem oluşturulduğunda ve uygulandığında, kutsallığın kimliğine ve nihai amacına ait olan bahse konu bu öncül kavramsallaşmasının bireysel hale gelişi biçimindeki Berrak ve Sabah Yıldızı orada mevcudiyetine kavuşmaktadır. Her ne kadar kutsallığın nitelikleri bakımından önemli bir ölçüde kısıtlı olan, Yaratan Evlat’ın tüm nitelikleri ve karakterleri göz önüne alındığında ona benzeyen onun kişilik birlikteliği biçimindeki bu unsur evrenin baş yöneticisidir.
\vs p032 2:7 Yaratan Evlat’ın sağ kolu olan yardımcısı ve baş yöneticisi böylelikle sağlanırken orada, çeşitlilik içerisinde bulunan yaratılmışların çok geniş ve muazzam bir kapsamdaki varoluşu gerçekleşir. Yerel evrenlerin erkek ve kız evlatları burada belirecek olup, bu durumun hemen sonrasında; evrenin yüce kurullarından, irade sahibi yaratılmışların çeşitli fani ırklarının evleri haline gelmesi için peşi sıra tasarlanan bu dünyaların bir araya toplanması şeklindeki yerel sistemlerin takımyıldızlarının ve egemenlerinin yaratıcılarına kadar uzanan bu türden bir yaratımın hükümeti sağlanacaktır; ve bu dünyaların her biri, bir Gezegensel Prens tarafından idare edilecektir.
\vs p032 2:8 Ve bu durumun sonrasında bu türden bir evren oldukça bütüncül bir biçimde işleyişsel olarak düzenlendiğinde, ve oldukça tamamlayıcı bir şekilde güçlendirildiğinde; Yaratan Evlat, onların kutsal görüntüsü içinde fani insanı yaratmak için Yaratıcı’nın niyetini gerçekleştirme yolunda adım atmış olur.
\vs p032 2:9 Gezegensel yerleşkelerin işleyişsel düzenlenmesine dair süreç, Nebadon içinde hala devam etmektedir; çünkü bu evren gerçekten de, Orvonton’un yıldızsal ve gezegensel âlemleri içinde henüz genç kümelenme şeklindedir. Son kayıtlara göre Nebadon içerisinde 3.840.101 yerleşik gezegen bulunmaktadır; buna ek olarak dünyanızın ait olduğu yerel sistem olan Satania, diğer sistemlere oldukça benzer bir niteliktedir.
\vs p032 2:10 Satania, yalnız bir gökbilimsel birim veya işleyişsel düzenleme şeklinde bulunan bir biçimde tek\hyp{}tip bir fiziksel sistem değildir. Onun 619 yerleşik dünyası, beş yüzü aşkın farklı fiziksel sistem içerisinde konumlandırılmıştır. Orada iki yerleşik dünyaya sahip olan kırk altı tane sistem bulunurken, onlardan sadece beşi, iki yerleşik dünyadan daha fazlasına sahip olup; bu beşi arasında yalnızca biri dört yerleşik gezegene sahiptir.
\vs p032 2:11 Yerleşik dünyaların Satania sistemi; Uversa’dan, ve yedinci aşkın evrenin fiziksel veya gökbilimsel olan merkezi biçiminde faaliyet gösteren büyük güneş kümelenmesinden oldukça uzak bir şekilde konumlandırılmıştır. Satania’nın yönetim merkezi Jerusem’den, Samanyolu’nun oldukça geniş çapı içerisinde çok uzak bir konumda bulunan Orvonton’un aşkın evreninin fiziksel merkezine olan uzaklık iki yüz bin ışık yılından fazladır. Satania, yerel evrenin çevresel kısmı üzerinde bulunmaktadır; ve Nebadon şu anda Orvonton’un sınırlarına yönelen bir biçimde oldukça dışa doğru bir konumda bulunmaktadır. Yerleşik dünyaların en dışta bulunan sisteminden aşkın evrenin merkezine doğru olan uzaklık, iki yüz elli bin ışık yılından biraz daha azdır.
\vs p032 2:12 Nebadon’a ait olan evren mevcut an içinde, Orvonton’un aşkın evren döngüsü içinde güney ve doğu istikametine çok daha yakın olan bir konumda dönüşünü gerçekleşmektedir. Ona en yakın olan komşu evrenler: Avalon, Henselon, Sanselon, Portalon, Wolvering, Fanoving ve Alvoring’dir.
\vs p032 2:13 Fakat yerel bir evrenin evrimi, uzun bir anlatımı gerektirecek içeriğe sahiptir. Aşkın evren ile ilgili olan makaleler bu konuya giriş yapmakta, yerel yaratımları kaleme alan bu kısım ise bahse konu bu anlatımı devam ettirmekte, ve son olarak bu makaleyi takip eden biçimdeki Urantia’nın tarihi ve nihai sonuna değinen makaleler ise bu hikâyeyi tamamlamaktadır. Fakat siz böyle bir yerel yaratıma ait olan fanilerin nihai sonunu ancak; evrimsel dünyanız üzerinde fani bedeni sureti içinde insan yaşamını bir kez deneyimlemiş olarak, Yaratan Evlat’ın yaşamının ve öğretilerinin anlatımlarının irdelenmesiyle yeterli bir biçimde kavrayabilirsiniz.
\usection{3.\bibnobreakspace Evrimselliğin Düşüncesi}
\vs p032 3:1 Kusursuz bir biçimde oluşturulmuş tek yaratım, Kâinatın Yaratıcısı’nın düşüncesi ve Ebedi Evlat’ın emri ile doğrudan bir biçimde oluşturulmuş olan merkezi evren biçimindeki Havona’dır. Havona; her şeyin merkezi olan ebedi İlahiyatlar’ın çevreleyici evi biçimindeki deneyimsel, kusursuz ve tamamlanmış olan bir evrendir. Yedi aşkın evrenin yaratımları; sınırlı, evrimsel ve tutarlı bir biçimde ilerleyicidir.
\vs p032 3:2 Zaman ve mekânın fiziksel sistemlerinin tümü kökeni bakımından evrimseldir. Onlar; kendilerine ait olan aşkın evrenlerin oluşturulmuş döngüleri etrafında dönüşlerine gerçekleştirinceye kadar, fiziksel olarak bile sabitleştirilmiş bir konumda bulunmamaktadır. Bir yerel evrene ait olan gelişmenin ve büyümenin fiziksel olanakları gerçekleştirilinceye, ve onun yerleşik dünyalarının tümünün ruhsal düzeyi sonsuza kadar oluşturuluncaya ve sabitleştirilinceye kadar bahse konu bu yerel evren; ışık ve yaşam içinde sabit bir biçimde konumlandırılmış niteliğe sahip değildir.
\vs p032 3:3 Merkezi evren içindeki niteliğinin dışında kusursuzluk, ilerleyici bir erişimdir. Yerel yaratım içinde biz, kusursuzluğun bir yaratım biçimine sahip bulunmaktayız; fakat tüm diğer âlemler, bu belirli dünyalar veya evrenlerin ilerleyişi için oluşturulmuş olan yöntemler aracılığıyla bahse konu bu kusursuzluğa erişmek zorundadırlar. Buna ek olarak neredeyse sınırsız olan çeşitlilik; Yaratan Evlatlar’ın ilgili yerel evrenlerinin düzenlemesi, evirilmesi, yetiştirilmesi ve oluşturulmasına dair olan tasarıları simgelemektedir.
\vs p032 3:4 Yaratıcı’nın ilahiyat mevcudiyetinin dışında her yerel evren belirli bir biçimde, merkezi veya doğum biçimindeki yaratımın idari işleyişsel düzenlenmesinin bir suretidir. Her ne kadar Kâinatın Yaratıcısı, kişisel olarak yerleşkesel âlem içinde mevcut bir biçimde bulunuyorsa da; o, zaman ve mekânın fanilerinin ruhlarıyla birlikte gerçek anlamda ikamet ederken bahse konu bu evren içinden kökenini alan varlıkların akıllarında yerleşik değildir. Uçsuz bucaksız olan yaratımın ruhsal olaylarının düzenlenmesi ve idare edilmesi içinde bütünüyle ussal olan bir telafinin var olduğu gözlenmektedir. Merkezi evren içinde Yaratıcı bu haliyle mevcut olurken, yine de bahse konu kusursuz yaratıma ait olan evlatların akıllarında yer almamaktadır. Mekânın evrenleri içinde Yaratıcı, kendisine ait olan Egemen Evlatlar tarafından temsil edildiği biçimde kişisel olarak ikamet etmezken; bahse konu irade sahibi yaratılmışlarının akıllarında barınan Gizem Görüntüleyicileri’nin birey öncesi mevcudiyeti tarafından ruhsal olarak temsil edilerek, fani çocuklarının akıllarında çok yakın bir biçimde var olmaktadır.
\vs p032 3:5 Yerel bir evrenin yönetim merkezi dünyaları üzerinde; Kâinatın Yaratıcısı’nın kişisel mevcudiyeti dışında, kendinden müstakil yönetim yetkisini ve idari özerkliği temsil eden bahse konu yaratan ve yaratıcı kişiliklerin tümü ikamet etmektedir. Yerel evren üzerinde merkezi evren içinde Kâinatın Yaratıcısı dışında, mevcut olan ussal varlıkların neredeyse her sınıfına ait olan ortak bir takım varlık ve her bireyine ait olan ortak bir takım şey bulunmaktadır. Her ne kadar Kâinatın Yaratıcısı kişisel olarak yerel evren içinde var olmasa da, Tanrı’nın ileride gerçekleşecek olan vekili ve bunun hemen sonrasında kendinden müstakil bir biçimde yüce ve egemen iradecisi olacak biçimdeki onun Yaratan Evladı tarafından kişisel olarak temsil edilir.
\vs p032 3:6 Yaşamın daha derininde bulunan derecelerine doğru gittiğimizde, görünmez olan Yaratıcı’ya inancın gözü ile konumlandırmak daha zor bir hale gelmektedir. Alt düzeyde bulunan yaratılmışlar için --- hata zaman zaman yüksek kişilikler için bile --- Yaratan Evlatlar içerisinde Kâinatın Yaratıcısı’nı tahayyül etmek her zaman zor bir durumdur. Ve bu zorluk sonucunda, gelişmenin kusursuzluğu onların Tanrı’yı bizzat görmelerini yetkin hale getireceği an olarak ruhsal yüceltilmelerinin zamanı gerçekleşinceye kadar onlar; ilerlemeden yorgun düşüp, ruhsal kuşkuları barındırıp, kafa karışıklığına uğrayıp bu sebepten dolayı kendilerini zamanlarının ve evrenlerinin ilerleyici ruhsal gelişiminin dışında bırakırlar. Böylelikle onlar, Yaratıcı Evlat’a dikkatli bir biçimde bakarken onun bünyesinde Yaratıcı’yı görmenin yetkinliğini kaybederler. İçkin koşulların böyle bir erişimi imkânsız kıldığı anlar olan bu türden zamanlar süresince, yaratılmış için Yaratıcı’ya ulaşmanın uzun uğraşları boyunca en güvenilir olan koruyucu; Yaratıcı’nın Evlatları bünyesinde onun mevcudiyetine ait doğru\hyp{}olan\hyp{}gerçekliğe sımsıkı sarılmaktadır. Gerçek ve mecazi anlamıyla, ruhsal ve kişisel olarak, Yaratıcı ve Evlatlar bir bütündür. “Bir Yaratıcı Evlat’ı gören Tanrı’yı görmüştür yargısı” başlı başına bir gerçektir.
\vs p032 3:7 Herhangi bir evrenin kişilikleri; ilk olarak yalnızca İlahiyat ile olan bağının derecesiyle iniltili bir biçimde yerleşik bir konumda olup, güvenilir bir niteliğe sahiptir. Yaratılmış kökeni, oldukça farklı bir doğrultu içerisinde kökensel ve kutsal Kaynaklar’dan uzaklaştığında; ister Tanrı’nın Evlatları veya ister Sınırsız Ruhaniyet’e ait olan hizmetin yaratılmışları olsun fark etmeyen bir biçimde, orada kötülük olarak bulunan uyumsuzluğun, kafa karışıklığının ve zaman zaman başkaldırının bir artışı var olmaktadır.
\vs p032 3:8 İlahiyat kökenine ait olan kusursuz varlıklar haricinde, aşkın evrenler içindeki irade sahibi yaratılmışlarının tümü; gerçekte içe doğru olan yolculuk biçimindeki, alt düzeyden başlayıp sürekli bir biçimde yukarı düzeylere doğru çıkarak gerçekleşen evrimsel doğanın bir parçasıdır. Her yüksek ruhsal kişilik; bir hayattan diğer bir hayata ve bir âlemden bir diğerine gerçekleşen biçimde, ilerleyici dönüşümler aracılığıyla yaşamın dereceleri içinde yukarı doğru olan yükselişlerine devam etmektedir. Ve Gizem Görüntüleyicileri’nden faydalananlar hususunda ise, onların ruhsal yükselişleri ve evren erişimlerinin olası sınırları için gerçek anlamda herhangi bir kısıtlama bulunmamaktadır.
\vs p032 3:9 Zamanın yaratılmışlarının kusursuzluğu nihai olarak erişildiğinde, tüm içtenliğiyle elde bulundurulan bir kişilik sahipliği biçiminde bütüncül bir kazanımdır. İnayetin unsurları özgür bir biçimde seçilen bir şekilde bir araya getirilirken, yaratılmış erişimleri yine de var olan çevreye karşı kişiliğin tepkisi biçimindeki bireysel çaba ve mevcut yaşamın sonuçlarıdır.
\vs p032 3:10 Hayvansal eviriliş kökenine dair gerçeklik; sınırlı bir biçimde ussal olan irade sahibi yaratılmışlarının iki temel türünden bir tanesini ortaya çıkaran ayrıcalıklı yöntem olarak, evrenin görünümü içindeki herhangi bir kişilik için utanç verici bir anlam taşımamaktadır. Kusursuzluk ve ebediyetin yüksek olan düzeylerine erişildiği zaman daha fazla onur; hayatın merdivenlerinin en altından basamak basamak memnuniyetle çıkanların, ve yüceliğin en yüksek düzeyine ulaştıklarında en alttan en yüksek düzeye kadar yaşamın her fazına dair mevcut olan bir bilgiyi bünyesinde taşıyan kişisel bir deneyim kazanacakların tümüne aittir.
\vs p032 3:11 Bahse konu bu kazanımların hepsi Yaratanlar’ın bilgeliğinde gösterilmiştir. Kâinatın Yaratıcısı için, kutsal emri vasıtasıyla kusursuzluğun aktarımı biçiminde tüm fanileri kutsal varlıklar haline getirmek çok kolay bir durum olacaktı. Fakat böyle bir girişim; yaşam mevcudiyetinin en altından başlayacak kadar uğurlu olan sadece bu varlıklar tarafından elde edilecek bir deneyim biçimindeki, uzun ve aşamalı olarak gerçekleşen içsel doğru yükselişle birlikte bütünleşen serüvenin ve eğitimin muhteşem bir deneyiminden onların mahrum kalmasına neden olacaktı.
\vs p032 3:12 Havona’yı çevreleyen evrenler içinde; yaşamın evrimsel ölçeği boyunca yükselmekte olan unsurlar için, yaratım biçimi öğretmen rehberlerinin ihtiyaçlarını karşılamak amacıyla kusursuz yaratılmışların sadece yeterli bir sayıdaki nüfusu sağlanmıştır. Kişiliğin evrimsel türüne ait olan deneyimsel doğa, Cennet\hyp{}Havona yaratılmışlarının başından beri kusursuz olan doğalarının doğal olan kâinatsal tamamlayıcısıdır. Gerçekte kusursuz ve kusursuzlaştırılmış olan yaratılmışlar, sınırlı bütünlük bakımından tamamlanmamış bir halde bulunmaktadır. Fakat evrimsel evrenlerden yükselen deneyimsel olarak kusursuzlaştırılmış kesinliğe ulaşacak olan unsurlarla birlikte, Cennet\hyp{}Havona sisteminin deneyimsel olarak kusursuz yaratılmışlarının tamamlayıcı birliktelikleri içinde iki tür de; içkin olan kısıtlamalardan kurtulmanın olanağına sahiptir; ve böylelikle onlar, yaratılmış düzeyinin nihayetine ait olan ulvi yüksek seviyelere birlikte erişme girişiminde bulunabilirler.
\vs p032 3:13 Bu yaratılmış etkileşimleri, Yedi Katmanlı Olan İlahiyat içindeki etki ve tepkilerin evren sonuçlarıdır. Burada Cennet Kutsal Üçlemesi’nin ebedi kutsallığı; Yüce Varlık’ın güç\hyp{}etkinleştirici İlahiyatı içinde, onun aracılığıyla ve onun tarafından zaman\hyp{}mekân evrenlerinin Yüce Yaratanları’nın evrimleşen kutsallığı ile birlikte bütünleşir.
\vs p032 3:14 Kutsal bir biçimde kusursuz olan yaratılmış ve evrimsel bir biçimde kusursuzlaştırılmış yaratılmış, kutsallık potansiyelinin düzeyi bakımından eşittir; fakat onlar türleri bakımından farklılık göstermektedir. Evrimsel aşkın evrenler, yükseliş vatandaşlarına verilecek nihai eğitimin sağlaması için kusursuz Havona’ya ihtiyaç duymaktadır; ancak kusursuz merkezi evren, alçalan sakinlerinin bütüncül gelişimini sağlamak için kusursuzlaştırıcı aşkın evrenlerin mevcudiyetine ihtiyaç duymaktadır.
\vs p032 3:15 İster kişilikler için veya ister evrenler için olsun sınırlı gerçekliğin içkin kusursuzluk ve evrimleşmiş kusursuzluk biçimindeki iki temel dışavurumu; eş güdüm halinde, bağlı ve birbirleriyle bütünlük içerisindedir. Her biri; faaliyetin, hizmetin ve nihai sonun tamamlanışına erişmek için bir diğerine ihtiyaç duymaktadır.
\usection{4.\bibnobreakspace Tanrı’nın Yerel Bir Evrenle İlişkisi}
\vs p032 4:1 Kendisine ait birçok niteliği ve gücü, temsil etmesi için diğerlerine aktarması sebebiyle; Kâinatın Yaratıcısı hakkında, İlahiyat birlikteliğinin sessiz veya etkisiz bir üyesi olduğu şeklinde bir düşünceyi sakın aklınızdan geçirmeyin. Kişilik nüfuz alanları ve Düzenleyici’nin bahşedilmesi dışında o; kendisine ait olan İlahiyat eş güdüm sağlayıcılarının, Evlatlar’ın ve sayısız ölçekte bulunan yaratılmış ussal varlıkların kendi ebedi amacının yerine getirilmesinde oldukça tatmin edici bir biçimde görev yapmalarına izin vererek, bariz bir biçimde Cennet İlahiyatları arasında en az etkin olan üyedir. Onun, yaratıcı üçlemeye ait olan sessiz bir üyesi oluşu sadece; eş güdümde bulunduğu unsurların veya emri altındaki birlikteliklerin herhangi birinin yapabileceği bir etkinliğe zerre kadar bile müdahalede bulunmamasından kaynaklanmaktadır.
\vs p032 4:2 Tanrı, işlev ve deneyim için her ussal yaratılmışın ihtiyacı hususunda bütüncül bir anlayışa sahiptir. Ve bu sebepten dolayı ister bir evrenin nihai sonu ile ilgili olsun isterse de yaratılmışların en alçak gönüllü olanının refahı için olsun, hiçbir koşul altında fark etmeyen bir biçimde Tanrı; kendisi ve herhangi bir evren durumu veya yaratıcı etkinlik arasında içkin olarak müdahil olan, yaratıcı ve Yaratan kişiliklerinin bütüncül topluluğu yararına eylemine son verir. Fakat sınırsız eş güdümün bu biçimde temsili şeklindeki bahse konu bu faaliyetten çekiliş gerçekleşmiş olsa da; bahse konu bu emredilen topluluklar ve kişilikler tarafından ve onlar aracılığıyla bu etkinlikler içinde Tanrı’nın rol aldığı mevcut, tam anlamıyla gerçek ve kişisel katılım bulunmaktadır. Yaratıcı, kendisine ait olan uçsuz bucaksız yaratımın refahı için tüm bu kanallar içinde ve onun vasıtasıyla görev yapmaktadır.
\vs p032 4:3 Yerel bir evrenin siyasaları, davranışları ve idaresi ile ilgili olarak Kâinatın Yaratıcısı, Yaratan Evlat’ın kişiliği içinde hareket etmektedir. Tanrı’nın Evlatları’na ait olan karşılıklı ilişkiler içerisinde, Üçüncül Kaynak ve Merkez’in kökenine ait olan kişiliklerin topluluk birlikteliklerinde, veya insan varlıkları olarak herhangi bir diğer yaratılmışlar arasındaki ilişkilerde --- bu tür birlikteliklerin bütününü kapsayan genel bir biçimde --- Kâinatın Yaratıcı hiçbir zaman müdahalede bulunmamaktadır. İlgili olan evren için hükmedilen siyasalar ve usuller biçiminde; Yaratan Evlat’ın yasası, Takımyıldız Yaratıcıları’nın idaresi, Sistem Egemenleri ve Gezegensel Prensler her zaman etkin bir konumda üstünlüklerini sürdürmektedir. Ne yetkide bir bölünme var olmakta, ne de kutsal güç ve amacının çelişkili bir görevi mevcut bulunmaktadır. İlahiyatlar, kusursuzluk ve ebedi görüş birliği içerisindedir.
\vs p032 4:4 Yaratan Evlat; herhangi bir topluluk içindeki yaratılmışların herhangi bir bölümünün, yaratılmışların ya da iki veya daha fazla bireyin herhangi bir diğer sınıfı ile olan ilişkileri biçimindeki etik birlikteliklerinin olayları içinde yüceliğin yönetimini sağlamaktadır. Fakat bu türden bir tasarımın; Kâinatın Yaratıcısı’nın arzuladığı şekliyle müdahil olamayacağı, ve bireyin mevcut durumu veya gelecek olanaklarına ek olarak Yaratıcı’nın ebedi tasarımı ve sınırsız amacı ile ilgili olan bir biçimde, herhangi bir \bibemph{bireysel yaratım} ile kutsal aklı yaratımın tümü boyunca tatmin eden zerre kadar bir etkinlikte bulunmadığı anlamına gelmemektedir.
\vs p032 4:5 İrade sahibi fani yaratılmışları içinde Yaratıcı; kendisinin birey öncesi ruhaniyetinin bir nüvesi biçimindeki, mevcut olarak ikamet eden Düzenleyici içerisinde varoluş içerisindedir; buna ek olarak Yaratıcı aynı zamanda, bu türden bir irade sahibi fani yaratılmış içerisinde kişiliğin kaynağıdır.
\vs p032 4:6 Kâinatın Yaratıcısı’nın bahşedilmişleri olarak bahse konu bu Düşünce Düzenleyicileri, göreceli olarak yalıtılmış bir durumdadır; onlar insan akıllarında ikamet etmektedirler, fakat yerel bir yaratımın etik olayları ile birlikte algılanabilen hiçbir ilişkiye sahip değildir. Onlar; ne doğrudan bir biçimde yüksek meleksel hizmetle, ne de sistemlerin, takımyıldızların veya bir yerel evrenin idaresi ile birlikte eş güdüm haline getirilmiştir; hatta onlar, iradesinin kendisine ait olan evrenin yüce kanunu olduğu bir Yaratan Evlat’ın yönetimiyle birlikte bile ortak bir biçimde hareket etmemektedir.
\vs p032 4:7 İkamet eden Düzenleyiciler; Tanrı’nın neredeyse sınırsız nitelikteki yaratılmışları ile birlikte, ayrımsallaşmış bir konumda bulunan fakat yine de bütünleşmiş özelliğe sahip olan ilişkisinin türlerinden biridir. Fani insan için görünmez bir biçimde olan o, mevcudiyetini bu şekilde ortaya çıkarır; o kendisini diğer biçimlerle de bize göstermeye yetkindir, fakat bu türden ilave bir açığa çıkarış kutsal olarak mümkün değildir.
\vs p032 4:8 Evlatlar’ın kendilerine ait olan yetki alanlarının evrenleri ile ilgili memnuniyetle yaşadıkları içten ve bütüncül bilgiyi elde ettikleri işleyiş biçimini görüyor ve onu anlıyoruz. Fakat her ne kadar biz, Kâinatın Yaratıcısı’nın engin yaratılmışlarıyla ilgili bilgi aldığı ve kendi mevcudiyetini onlar için dışa vurduğu biçime en azından aşina olsak da; Tanrı’nın oldukça bütüncül ve fazlasıyla kişisel olarak kâinatın âlemlerinin en ufak ayrıntılarına aşina olduğu yöntemleri etraflı bir biçimde kavrayamayız. Kişilik döngüsü vasıtasıyla bireysel olarak sahip olduğu bilgi biçiminde Yaratıcı, yaratımın tümüne ait olan evrenlerin bütününün sistemlerinin hepsi içindeki tüm varlıkların düşünceleri ve eylemlerinin tamamından haberdardır. Her ne kadar biz, Tanrı’nın kendi evlatları ile olan bütünlüğünün bu işleyiş biçimini bütüncül olarak kavrayamasak da; “Koruyucu evlatlarını bilir,” ve her birimiz için ifade edildiği biçimiyle “o, nerede doğduğumuzun kaydını tutmaktadır” şeklindeki güvenceler vasıtasıyla güçlendirilmiş bir yetkinliğe sahip olabiliriz.
\vs p032 4:9 Ruhsal bir ifadeyle; merkezi yerleşkenin Yedi Üstün Ruhaniyeti’nden biri, ve özellikle fani aklın derinlikleri içinde yaşayan, görev yapan ve bekleyen kutsal Düzenleyici vasıtasıyla, Kâinatın Yaratıcısı evreninizde ve kalbiniz içinde var olmaktadır.
\vs p032 4:10 Tanrı birey\hyp{}merkezci bir kişilik değildir; Yaratıcı özgür bir biçimde kendisini, yaratımına ve yaratılmışlarına bahşetmektedir. O sadece İlahiyatlar içerisinde değil, aynı zamanda kutsal bir biçimde gerçekleştirmeleri mümkün olan her şeye dair faaliyeti onlara öğrettiği Evlatları içinde yaşar ve hareket eder. Kâinatın Yaratıcısı, diğer varlığın gerçekleştirmesi mümkün olan her faaliyeti kendisinden alıp bu varlıklara aktarmıştır. Ve bu durum, yerel bir evrenin yönetim merkezi üzerinde Tanrı’nın yerinde idarede bulunan Yaratan Evlat’a ek olarak fani insan içinde aynı gerçekliği taşımaktadır. Bu nedenden dolayı biz, Kâinatın Yaratıcısı’nın nihai ve sınırsız sevgisinin işleyişine dikkatlice bakmaktayız.
\vs p032 4:11 Kendisinin bu kâinatsal bahşedişi içerisinde biz, Yaratıcı’nın kutsal doğasının ölçeği ve yüceliğinin oldukça fazla olan kanıtına sahip bulunmaktayız. Eğer Tanrı, kâinatsal yaratımdan kendisinin bir zerresini bile saklı tutmuş olsaydı; böyle bir durumun sonucunda, sonsuza kadar devam eden yaşam için fani adaylar içinde fazlasıyla sabırlı bir biçimde ikamet eden zamanın Gizem Görüntüleyicileri biçimindeki âlemlerin fanileri üzerinde Düşünce Denetleyicileri’nin oldukça zengin olan cömert bahşedilmişliği içerisinde onun mevcudiyetine dair geride kalan her şey var olmuş olacaktı.
\vs p032 4:12 Kâinatın Yaratıcısı kendisini, kişilik sahipliği ve potansiyel ruhsal erişim içinde tüm yaratımları zengin kılmak için adeta bahşetmiştir. Tanrı, bizim kendisi gibi olabilmemizin ihtimali için kendisini bize bahşetmiş olup; gücünü ve ihtişamını yalnızca, her şeye kendisini böylelikle bahşedişine ait olan sevgisi için bu varlıkların idaresine özgü gerekli koşullar uğruna saklı tutmuştur.
\usection{5.\bibnobreakspace Ebedi ve Kutsal Amaç}
\vs p032 5:1 Mekân boyunca evrenlerin ilerleyişinde çok büyük ve yüce olan bir amaç bulunmaktadır. Sizin fani çabalarınızın hiçbiri amaçsız değildir. Bizim hepimiz, devasa olan bir girişim biçimindeki engin bir tasarımın parçasıdır; ve bu tasarımın yürütülmesine ait olan uçsuz bucaksızlık, herhangi bir yaşam boyunca ve belirli bir an içinde bu gayeye dair birçok şeyi görmemizi imkânsız kılmaktadır. Bizim hepimiz, Tanrılar’ın yüksek denetimde bulundukları ve işlettikleri bir ebedi tasarımın birer parçasıyız. Bahse konu bu olağanüstü ve kâinatsal olan işleyiş düzeni, İlk Büyük Kaynak ve Merkez’in sınırsız düşüncesinin ve ebedi amacının ölçülü ahengi içerisinde mekân boyunca tüm ihtişamıyla hareket etmektedir.
\vs p032 5:2 Ebedi Tanrı’nın ebedi amacı, yüksek bir ruhsal nihai gayedir. Zamanın etkinlikleri ve maddi mevcudiyetin çabaları ancak; ruhsal gerçekliğin ve göksel mevcudiyetin vaat edilmiş âlemi biçimindeki diğer tarafa bağlayacak köprü niteliğindeki geçici oluşumlardır. Tabiidir ki siz faniler, ebedi amacın nihai gayesini kavramakta zorluk çekmektesiniz; ancak gerçekte siz, başlangıcı ve sonu olmayan bir şey biçimindeki ebediyetin düşüncesini algılamakta yetkin bir durumda bulunmamaktasınız. Siz aşina olduğunuz her şeyin bir sonu bulunmaktadır.
\vs p032 5:3 Bir bireysel yaşam, bir âlemin süreci veya birbiriyle ilişkili olan sıralı etkinliklerin zaman dizini ile ilgili olarak bizim sadece zamanın yalıtılmış olan bir kapsamı ile karşı karşıya olduğumuza dair bir görünüş ortaya çıkabilir; her şeyin bir başlangıcı ve sonu varmış gibi gözlenmektedir. Buna ek olarak bu tür deneyimlerin, yaşamların, çağların veya devirlerin bir sırası birbirlerini takip eden sıralı bir şekilde bir araya getirildiğinde, zamanın yalıtılmış bir olayının ebediyetin sınırsız yüzeyi boyunca anlık bir biçimde parlaması şeklinde düz bir doğrultuyu oluşturuyormuşçasına bir görünüme sahiptir. Fakat biz bu olayların tümüne perde arkasından baktığımızda, daha derin olan bir bakış açısı ve daha tamamlayıcı bir anlayış; bu türden bir açıklamanın, ebediyetin altında yatan amaçlarla ve temel tepkimelerle birlikte zamanın etkileşimlerini yeterli bir biçimde açıklamak ve diğer bir taraftan onunla ilişkilendirmek için yetersiz, ilgisiz ve bütünüyle elverişsiz olduğunu göstermektedir.
\vs p032 5:4 Ebediyetin bir döngüsünün bir şekilde zamanın geçici olan maddi döngüleriyle uyumlu olduğu bir biçimde, ebediyeti bir döngü ve ebedi amacı sonu olmayan bir daire olarak algılamanın fani akıl için açıklama amacıyla daha uygun olduğunu düşünmekteyim. Zamanın bölümlerinin bağlı olduğu ve onun bir parçasını oluşturduğu ebediyetin döngüsü ile ilgili biz, bu tür geçici çağların tıpkı zamanın geçici varlıkları gibi doğduğu, yaşadığı ve öldüğünün farkındalığına zorlanmaktayız. Birçok insan varlığı ölmektedir; çünkü Düzenleyici ile bütünleşmenin ruhani düzeyine erişmede başarısız olarak ölümün başkalaşımı, onların maddi yaratımın sınırları ve zamanın kısıtlılıklarından kaçarak ebediyetin ilerleyici süreci ile ruhsal aşamayı gerçekleştirmeye yetkin hale geldikleri tek mümkün işleyiş biçimini oluşturmaktadır. Zamanın ve maddi mevcudiyetin sınayış yaşamında varlığınızı sürdürdükten sonra; ebedi çağların dairesi etrafından mekânın dünyalarıyla birlikte dönen bir biçimde, ebediyet ile ilişki halinde onun bir parçası bile olmayı sürdürmeniz mümkün hale gelmektedir.
\vs p032 5:5 Zamanın bölümleri, geçici biçimdeki kişiliğin parıltıları gibidir; onlar bir süreliğine görünüp, daha sonra ise insan gözünden kaybolurlar; onlar, ebedi döngü etrafından sonsuz bir biçimdeki dönüşün yüksek yaşamları içinde yeni etkenler ve devam eden etmenler olarak yeniden ortaya çıkarlar. Sınırlandırılmış bir kâinat içinde Kâinatın Yaratıcısı’nın merkezi yerleşik konumu etrafında çok uzun olan geniş bir daire üzerindeki dönüşe dair bizim görüşlerimize göre ebediyet, doğrusal bir doğrultu olarak algılanamaz.
\vs p032 5:6 Doğruyu söylemek gerekirse ebediyet, zamanın sınırlı aklı için kavranamaz niteliktedir. Yalın bir değişle sizin, onu algılamaya ve kavrayamaya dair bir yetkinliğiniz bulunmamaktadır. Ben ebediyeti, bütüncül bir biçimde tahayyül edememekteyim; eğer bunu gerçekleştirmiş olsaydım bile, bunu insan aklı için taşımaya çalışmak benim için imkânsız bir nitelikte olacaktı. Yine de ben; ebedi olan şeylere dair anlayışımızın bir türünü sizlere aktarma biçiminde, bizim görüşlerimizin bazılarını tasvir etmek için elimizden gelenin en iyisini yapmış bulunmaktayım. Ben, sınırsız doğaya ve ebedi aktarıma ait olan bu değerler ile ilgili sizin görüşlerinizin belirginleşmesine yardım etmek için çabalamaktayım.
\vs p032 5:7 Tanrı’nın aklında, geniş nüfuz alanlarının tümüne ait olan her yaratım ile bütünleşen bir tasarı bulunmaktadır; ve bu tasarı sonsuz olanağın, sınırsız ilerlemenin ve sonu olmayan yaşamın ebedi bir amacıdır. Buna ek olarak, bu türden bir benzersiz sürecin sınırsız hazineleri onu arzulamanız için sizlere aittir.
\vs p032 5:8 Ebediyetin hedefi önünüzde sizi beklemektedir! Kutsallık erişiminin serüveni nihai sonunuz olarak sizlere aittir! Kusursuzluğun yarışı başlamıştır! Bu yarışa girecek olanlar arasında kesin olan zafer; bedenin tümü üzerinde özgür bir biçimde bahşedilmiş olan bu yolun her aşamasındaki ikamet eden Düzenleyici’nin yönlendirmesine ve Evren Evladı’nın iyi ruhaniyetinin rehberliğine bağlı olarak, inancın ve güvenin yarışını koşan her insan varlığının çabasını taçlandıracaktır.
\vs p032 5:9 [Nebadon’un Yüce Kurulu’na geçici bir biçimde bağlanmış ve Salvingtonlu Cebrail tarafından bu göreve atanmış olan bir Kudretli İletici tarafından sunulmuştur.]
