\upaper{79}{Doğu’daki And Genişlemesi}
\vs p079 0:1 Asya, insan ırkının anavatanıdır. Andon ve Fonta’nın doğduğu yer bu kıtanın bir doğu yarımadası üzerindeydi; bugünün Afganistan olan bölgesindeki yükseltilerde onların soyu Badonan, bir buçuk milyon yıldan daha fazla süreden beri varlığını sürdüren kültürün ilkel bir merkezinin temellerini attı. Burada insan ırkının bu doğu odağında, Sangik toplulukları Andon ırk kolundan ayrıştı; ve Asya onların ilk evi, ilk av alanı, ilk savaş meydanıydı. Güneybatı Asya; Dalamatia, Âdem ve And unsurlarının birbirlerini takip eden medeniyetlerine tanık oldu; ve bu bölgelerden çağdaş medeniyetin tohumları dünyaya yayıldı.
\usection{1.\bibnobreakspace Türkistan Andları}
\vs p079 1:1 Yirmi beş bin yıldan fazla bir süre boyunca, neredeyse M.Ö. 2000’li yıllara gelinceye kadar, Avrasya’nın kalbi başat bir biçimde, her ne kadar giderek azalma gösterse de, And kökenine aitti. Türkistan’ın düzlüklerinde And unsurları iç göller etrafında dönerek batı yönünde Avrupa’ya yönelirken, bu bölgenin yükseltilerinden onlar doğu yönüne doğru nüfuz ettiler. Doğu Türkistan (Sincan Uygur Özerk Bölgesi) ve, daha az bir ölçüde, Tibet Mezopotamya’nın bu insanlarının sarı ırkın kuzey yerleşkelerine doğru dağlara yöneldiği tarihi geçiş noktalarından biriydi. Hindistan’a olan And nüfuzu, Türkistan yükseltilerinden Punjab ve İran otlak arazilerinden Baluşistan boyunca ilerledi. Bu öncül göçler anlamsızca gerçekleşen fetihler değillerdi; gerçekte onlar, And kabilelerinin batı Hindistan ve Çin’e olan devamlı kayışlarıydı.
\vs p079 1:2 Neredeyse on beş bin yıl boyunca karma And kültür merkezleri Doğu Türkistan’daki Tarim Irmağı havzasına ek olarak buranın güneyindeki And ve Andon unsurlarının geniş ölçüde birbirine çoktan karışmış olduğu Tibet’in yüksek düzlüklerinde varlığını sürdürmeye devam etmişti. Tarım vadisi öz And kültürünün doğu sınırındaki yerleşkeydi. Burada onlar yerleşkelerini inşa edip, doğudaki gelişme gösteren Çinliler’e ek olarak kuzeydeki Andon unsurlarıyla birlikte ticaret ilişkilerine girdiler. Bu dönemlerde Tarım bölgesi, verimli bir araziydi; yağmurlar boldu. Doğuda bulunan Gobi, sürü sahiplerinin kademeli olarak tarıma yöneldikleri açık bir otlaktı. Bu medeniyet, yağmur rüzgârları güneydoğuya kaydığı zaman yok oldu; ancak en parlak dönemlerinde Mezopotamya’ya rakip oldu.
\vs p079 1:3 M.Ö. 8000’li yıllarda merkezi Asya’nın yüksek bölgelerinde yaşanan yavaş bir biçimde artış gösteren kuraklık And unsurlarını nehir tabanlarına ve deniz kıyılarına itmeye başladı. Bu artan kuraklık sadece onları Nil, Fırat, İndus ve Sarı nehirlerine sürüklemedi, aynı zamanda And medeniyeti içinde yeni bir gelişimi açığa çıkardı. Tüccarlar biçiminde insanların yeni bir sınıfı geniş sayılar halinde ortaya çıkmaya başladı.
\vs p079 1:4 İklim koşulları göç etmekte olan And unsurları için avcılığı elverişsiz kılınca, sürü sahipleri haline gelen bir biçimde eski ırkların evrimsel gidişatını takip etmediler. Ticaret ve şehir yaşamı ortaya çıkmaya başladı. Mısır’dan başlayarak Mezopotamya ve Türkistan boyunca Çin ve Hindistan’ın nehirlerine kadar daha yüksek bir biçimde medenileşmiş kabileler, imalat ve ticarete adanan şehirlerde bir araya gelmeye başladılar. Adonia, bugünün Aşkabat şehrinin yakınında konumlanan bir biçimde, merkezi Asya’nın ticari merkezi haline gelmişti.
\vs p079 1:5 Ancak sürekli artış gösteren kuraklık kademeli olarak Hazar Denizi’nin güneyi ve doğusundaki yerleşkelerden yapılan büyük And göçünü beraberinde getirdi. Bu göç dalgası kuzeyden güneye doğru yön değiştirmeye başladı, ve Babil atlıları Mezopotamya kapılarını zorladıkları sürece giriş yaptılar.
\vs p079 1:6 Merkezi Asya’daki artan kuraklık ilave bir biçimde, nüfusun azalmasına ve bu insan topluluklarının daha az savaşçıl hale gelmesine yol açtı; ve kuzeyde azalan yağmurlar göçebe Andon unsurlarını güneye doğru itince, Türkistan’dan devasa bir And göçü gerçekleşmiş oldu. Bu durum, Ari ırk unsurları olarak adlandırdığınız toplulukların Levant ve Hindistan’a yaptıkları göçe karşılık gelmektedir. Bu göçlerin bütünü, Âdem’in melez soylarının uzun süren dağılımları boyunca her Asyalı topluluğun ve Büyük Okyanus ada inanlarının çoğunun bir ölçüde bu üstün ırklar tarafından gelişmesine yol açmıştır.
\vs p079 1:7 Her ne kadar Doğu Yarıküre’nin tamamına yayılmış olsalar da And unsurları Mezopotamya ve Türkistan’daki anavatanlarından böylece mahrum bırakılmışlardı; bu durumun nedeni, merkezi Asya’da seyrelmiş And unsurlarını neredeyse yok olma düzeyine kadar indiren onların güneye doğru yapmış oldukları bu geniş çaplı göçtü.
\vs p079 1:8 Ancak İsa’dan sonra yirminci yüzyılda bile, Turan ve Tibet insanları arasında And kanının izleri mevcuttur; bu durum, bahse konu bölgelerde zaman zaman bulunan kan türlerinde gözlemlenmektedir. Öncül Çin yıllıkları, Sarı Nehir’in barışçıl yerleşkelerinin kuzeyinde kırmızı saçlı göçebelerin mevcudiyetini belirtmektedir; ve burada hala, uzun zaman öncesinin Tarım havzası içinde sarışın And unsurları ve kumral Mongol türlerinin mevcudiyetini tam olarak gösteren çizimler varlığını sürdürmektedir.
\vs p079 1:9 Merkezi Asyalı And unsurlarının sular altında kalan askeri dehasına dair en son büyük dışavurum; Cengiz Kağan yönetimi altındaki Mongol topluluklarının Asya kıtasının büyük bir kısmını ele geçirmeye başladığı M.S. 1200 yılıdır. Ve eskilerin And unsurları gibi bu kahramanlar “cennet içindeki tek bir Tanrı’yı” ilan etmişlerdir. Onların imparatorluğunun erken parçalanışı; Batı ve Doğu arasındaki kültürel iletişimi uzun bir süre geciktirmiş, Asya’daki tek tanrılı din kavramının gelişmesini fazlasıyla engellemiştir.
\usection{2.\bibnobreakspace Andlar’ın Hindistan’ı Fethi}
\vs p079 2:1 Hindistan, Urantia ırklarının tümünün birbirine karıştığı yer olan tek mekândır; And istilası kendilerinin ırk kolunu ekleyerek bu bütünlüğü tamamlamıştır. Hindistan’ın kuzeybatısındaki yükseltilerde Sangik ırkları ortaya çıkmış ve istinasız bu öncül dönemlerde Hindistan alt\hyp{}kıtasına giren her topluluğun üyeleri arkalarında Urantia üzerindeki en ayrışık ırk karışımını bırakmıştır. Tarihi Hindistan, göç eden ırklar için bir toplama noktası olarak görev yapmıştı. Yarımadanın giriş kısmı eskiden, şimdikine kıyasla biraz daha dardı; Ganj ve İndus deltalarının çoğu son elli bin yıldaki oluşumların neticesinde ortaya çıkmıştır.
\vs p079 2:2 Hindistan içerisindeki en öncül ırk karışımları, özgün Andon unsurları ile göç halindeki kırmızı ve sarı ırkların bütünleşmeleriydi. Bu topluluk daha sonra, turuncu ırkın geniş sayıdaki üyelerine ek olarak soyları tükenen doğudaki yeşil insanların daha büyük bir oranı tarafından onların baskınlığında karışmaları sonucu zayıflamıştı; bu insanlar her ne kadar mavi insanlar ile gerçekleştirdikleri sınırlı bir karışım sonucunda çok az bir ölçüde gelişseler de, çivit ırkının geniş sayıdaki üyelerine onların baskınlığında karışmaları sonucunda geniş çaplı gerileme yaşamışlardı. Ancak Hindistan’ın özgün insanları olarak tanımladığınız topluluklar bu öncül insanları temsil etmektedirler; bunun yerine onlar, öncül Andon topluluklarının veya onların daha sonra ortaya çıkan Ari kuzenlerinin hiçbir zaman bütünüyle karışmadığı en alt düzeydeki güney ve doğu azınlık insanlarıdır.
\vs p079 2:3 M.Ö. 20.000’li yıllarda batı Hindistan nüfusu çoktan Âdem kanı ile karışmış bir hale gelmişti; ve Urantia tarihi içinde hiçbir zaman bir topluluk bu kadar fazla ırkı bünyesinde barındırmamıştı. Ancak ikincil Sangik ırk kollarının baskın gelmesi talihsiz bir durumdu; buna ek olara mavi ve kırmızı insanların eskinin bu kaynaşma noktasından çok geniş çaplı eksiklikleri gerçek bir faciaydı; birincil Sangik ırk kollarının çoğu, ortaya çıkabilecek daha da büyük bir medeniyetin ilerlemesine oldukça fazla katkı sağlayabilirdi. Sonuç olarak; kırmızı insanlar Amerika kıtalarında kendilerini yok etmekte, mavi insanlar Avrupa’da kendilerini oyalamakta, ve Âdem’in öncül soyları (ve daha sonrakilerin çoğu) ister Hindistan, ister Afrika veya herhangi bir yer olsun daha koyu tenli insanlar ile karışmakta çok az istek gösterdiler.
\vs p079 2:4 Türkistan ve İran boyunca M.Ö. yaklaşık 15.000’li yıllarda artan nüfus baskısı, Hindistan’a yapılan gerçek anlamıyla ilk geniş çaplı And göçüne sebebiyet verdi. On beş asırdan fazla bir süre boyunca bu üstün insan toplulukları, İndus ve Ganj vadileri üzerinden yayılıp Deccan’a doğru güney doğrultusunda yavaşça hareket ederek Belucistan’ın yükseltileri üzerinde boşaldılar. Kuzeybatıdan gelen bu And baskısı, güneyde ve doğuda bulanan alt düzey toplulukları Burma’ya ve güney Çin’e itmişti; ancak bu itiş, istilacıları ırksal tahribattan kurtaracak kadar etkili düzeyde gerçekleşmemişti.
\vs p079 2:5 Hindistan’ın Avrasya’nın üstünlüğünü elde etmedeki başarısızlığı fazlasıyla bir yeryüzü dağılım sonucunun eseriydi; kuzeyden gelen nüfus baskısı yalnızca, insanların çoğunluğunu güneye doğru, deniz tarafından tüm sınırlarının çevrildiği Deccan’ın azalan topraklarına yöneltmiştir. Dışa doğru göç etmek için orada komşu bir kara parçası bulunmuş olsa, her yönde alt düzey insan toplulukları burayı doldurur ve üstün ırk kolları daha üstün bir medeniyete sahip olurdu.
\vs p079 2:6 Hal böyle olunca, bahse konu öncül And fatihleri; kimliklerini korumaya çalışmak için çaresiz bir girişimde bulunup, karşılıklı evlenmeye dair katı kısıtlamaları oluşturarak ırksal girdabın çekimine set çekmeye çabaladılar. Her ne kadar And unsurları M.Ö. 10.000’ler de yok olmuş bir halde olsalar da, onların tamamı bu karışım neticesinde dikkate değer bir biçimde gelişme göstermişti.
\vs p079 2:7 Irkların karışımı, kültürün çok yönlülüğüne yok açması ve ilerleyici bir medeniyeti ortaya çıkarması bakımından her zaman faydalıdır; ancak eğer ırksal kolların alt düzey unsurları baskın olursa, bu türden kazanımlar her zaman kısa süreli bir etkiye sahip olacaktır. Birçok dilli kültür yalnızca; üst düzey ırk kolları kendilerini, alt düzeyler karşısında güvenli bir oranda çoğaltabilirse korunabilir. Alt düzey unsurların kısıtlanmamış çoğalımı, üst düzeydekilerin azalan çoğalımı ile birlikte, kesin bir biçimde kültürel medeniyetin intiharıdır.
\vs p079 2:8 And fatihleri bulunduklarından üç kat fazla bir nüfusta olsalardı, veya onlar melez turuncu\hyp{}yeşil\hyp{}çivit sakinlerin en az düzeyde arzulanan dörtte üçünü dışarı doğru itseler veya yok etselerdi, Hindistan kültürel medeniyette dünyanın başat merkezlerden bir tanesi olur, ve kuşkusuz bir biçimde, Türkistan ve oradan kuzey yönünde Avrupa’ya doğru akan Mezopotamya sakinlerinin son dalgalarının daha fazlasını çekebilirdi.
\usection{3.\bibnobreakspace Dravid Hindistanı}
\vs p079 3:1 Özgün ırk koluyla birlikte karışan Hindistan’ın And fatihleri nihai olarak Dravid olarak adlandırılan melez insan topluluklarını yarattı. Daha öncül ve daha saf olan Dravid unsurları, kültürel kazanım için büyük bir kabiliyeti ellerinde bulundurdular; bu kabiliyet, And kalıtımları gittikçe azalırken düzenli bir biçimde zayıfladı. Ve bu durum, Hindistan’ın tomurcuklanmakta olan medeniyetinin neredeyse on iki bin yıl önceki sonlanışının ta kendisidir. Ancak Âdem kanının bu küçük miktarının nüfuzu bile, toplumsal gelişimde dikkate değer bir hızlanmayı yarattı. Bu karma ırk kolu doğrudan bir biçimde, bu dönemde dünya üzerindeki en çok yönlü medeniyeti açığa çıkardı.
\vs p079 3:2 Hindistan’ın fethinden sonra yakın bir zaman içerisinde Dravid And unsurları, Mezopotamya ile ırksal ve kültürel iletişimlerini kaybettiler; ancak deniz hatlarının ve kervan rotalarının daha sonra açılmasından sonra bu iletişimler yeniden kuruldu; ve son on bin yıl içinde bir kez bile Hindistan, her ne kadar dağ engelleri fazlasıyla batı irtibatını elverişli kılsa da, batıda Mezopotamya ve doğuda Çin ile iletişimini hiçbir şekilde tamamen yitirmedi.
\vs p079 3:3 Hint insan topluluklarının üstün kültürü ve dini eğilimleri, Dravid egemenliğinin öncül dönemlerine dayanmaktadır; ve bunlar kısmen de olsa, Seth din adamlığının birçok üyesinin öncül And ve daha sonraki Ari istilaları içinde Hindistan’a girişinden kaynaklanmaktadır. Hindistan’ın dini tarihi boyunca var olan tek dinlilik akımı böylelikle, ikinci bahçe içerisindeki Âdem unsurlarının öğretilerine dayanmaktadır.
\vs p079 3:4 Daha M.Ö. 16.000’li yıllarda yüz Seth din adamından oluşan bir birlik Hindistan’a girmiş olup, bu çok dilli insanların batı kesimini dini bakımından ele geçirmeyi başarmaya çok yaklaşmışlardı. Ancak onların dinleri varlıklarını sürdürmedi. Beş bin yıl içinde Cennet Kutsal Üçlemesi’ne dair savları, ateş tanrısının üçlü simgesine indirgenmiş bir hale gelmişti.
\vs p079 3:5 Ancak yedi bin yıldan daha fazla bir süre boyunca, And göçlerinin sonuna kadar, Hindistan sakinlerinin dini düzeyi dünyanın büyük bir kısmından çok daha fazla yüksekti. Bu dönemler boyunca Hindistan, dünyanın önde gelen kültürel, dini, felsefi, ve ticari medeniyetini yaratma ümidi verdi. Ancak eğer And unsurlarının güney insanları tarafından tamamiyle kaybedilmesi gerçekleşmeseydi, bu nihai son muhtemel bir biçimde gerçekleşecekti.
\vs p079 3:6 Dravid kültür merkezleri; özellikle İndus ve Ganj olmak üzere nehir vadilerinde ve Doğu Gat boyunca denize akan üç büyük ırmak boyunca Deccan içinde konumlanmıştı. Batı Gat’ın sahil şeridi boyunca mevcut olan yerleşkeler varlıklarını, Sümerliler ile olan denizcilik ilişkilerine borçluydu.
\vs p079 3:7 Dravid unsurları, şehirler inşa eden ve hem kara hem de deniz yoluyla geniş çaplı ithalat ve ihracat işine girişen öncül topluluklar arasındaydı. M.Ö. 7000’li yıllarda deve trenleri uzak Mezopotamya’ya düzenli seyahatlerde bulunmaktaydı; Dravid nakliyesi; Umman Deniz kıyısı boyunca Basra Körfezi’nin Sümer şehirlerine kadar gitmekte olup, Doğu Hint Adaları’na kadar Bengal Körfezi sularına açılmaktaydı. Yazma sanatıyla birlikte bir alfabe Sümer ülkesinden bu denizciler ve tüccarlar tarafından getirilmişti.
\vs p079 3:8 Bu ticari ilişkiler; şehir yaşamının birçok kibarlık ölçütlerine ek olarak şatafatlarının bile öncül ortaya çıkışlarıyla sonuçlanan bir biçimde, birçok uluslu kültürün daha ileri düzeydeki farklılaşmasına fazlasıyla katkıda bulunmuştu. Daha sonra ortaya çıkan Ari unsurları Hindistan’a girdiklerinde, Dravid topluluklarını Sangik ırkları içinde kaybolan And kuzenleri olarak tanımadılar; ancak onlar oldukça gelişmiş bir medeniyeti buldular. Biyolojik kısıtlılıklarına rağmen Dravidler üstün bir medeniyet inşa ettiler. Bu medeniyet Hindistan’ın tümüne çok iyi bir biçimde yayılmış olup, Deccan’ın çağdaş dönemlerine kadar varlığını sürdürmüştür.
\usection{4.\bibnobreakspace Ari Irkın Hindistan’ı İstilası}
\vs p079 4:1 Hindistan’a olan ikinci And nüfuzu, İsa’dan önce üçüncü bin yılın ortasında neredeyse beş yüz yıllık bir süreci kapsayan Ari istilasıydı. Bu göç, And unsurlarının Türkistan’daki anavatanlarından yaptıkları dönemsel büyük bir göçü simgelemektedir.
\vs p079 4:2 Öncül Ari merkezleri, özellikle kuzeybatı olmak üzere Hindistan’ın kuzey tarafının tümüne dağılmıştı. Bu istilacılar ülkenin fethini hiçbir zaman tamamlamamış olup, daha sonra bu umursamazlık içerisinde felaketleriyle yüzleştiler, çünkü onların giderek azalan sayıdaki nüfusu kendilerini, ileride Himalaya bölgeleri dışında yarımadanın tamamını elinde bulunduran güney Dravid topluluklarının baskınlığı altında karışma karşısında savunmasız bir konumda bıraktı.
\vs p079 4:3 Ari unsurları, kuzey bölgeleri dışında Hindistan üzerinde çok küçük bir ırksal etki bıraktı. Deccan’da onların etkisi ırktan ziyade kültürel ve diniydi. Kuzey Hindistan’daki Ari kanı olarak adlandırdığınız soyun daha baskın bir biçimde varlığını sürdürme nedeni sadece, onların bu bölgelerdeki geniş nüfusu değil; aynı zamanda, onların daha sonraki fatihler, tüccarlar ve din elçileri tarafından güçlenmiş olmalarıdır. İsa’dan önceki ilk asra kadar Punjab yerleşkesine doğru devamlı bir Ari soyu nüfuzu bulunmaktaydı; en son nüfuz, Helen dönemi topluluklarının seferlerinde gerçekleşmiştir.
\vs p079 4:4 Ganj düzlükleri üzerinde Ari ve Dravid toplulukları nihai olarak daha yüksek bir kültürü meydana getirecek şekilde birbirlerine karıştı; ve bu merkez daha sonra, Çin’den gelerek kuzeydoğu doğrultusundan uğrayan katkılar tarafından güçlendi.
\vs p079 4:5 Hindistan’da toplumsal örgütlenmelerin birçok türü, Ari unsurların yarı demokratik düzenlerinden zorba ve monarşik hükümetlere varan bir biçimde zaman zaman gelişme gösterdi. Ancak bu toplumun en belirgin özelliği, ırksal kimliği daimi kılma amacındaki bir teşebbüs içerisinde Ari unsurları tarafından kurulmuş büyük toplumsal tabakaların devamlılığıydı. Toplumun tabakalaşmasından oluşan bu detaylı düzen mevcut ana kadar korunmuştur.
\vs p079 4:6 Dört büyük toplum tabakası arasında ilki dışında hepsi, Ari fatihlerinin alt düzey bireyler ile olan ırksal karışımını engellemeye dair nafile çaba içerisinde oluşturulmuştu. Ancak, öğretmen\hyp{}din adamları olarak en üst sınıf Seth unsurları tarafından oluşturulmuştu; İsa’dan sonra yirminci yüzyılın Brahmanları, her ne kadar öğretileri ünlü seleflerininkinden fazlasıyla farklılık gösterse de, ikinci bahçenin doğrudan kültür soylarıdır.
\vs p079 4:7 Ari unsurları Hindistan’a girdikleri zaman, ikinci bahçenin dinine ait uzun süredir mevcut gelenekleri içinde korudukları İlahiyat’a dair kavramlarını beraberlerinde getirmişlerdi. Ancak Brahman din adamları, Ari unsurlarının ırksal tahribatından sonra Deccan’ın alt düzey dinleri ile birlikte birdenbire gerçekleşen iletişimleri sonucunda ortaya çıkan putlara olan inanış devinimine hiçbir zaman engel olamadılar. Böylelikle nüfusun büyük bir çoğunluğu, alt düzey dinlerin köleleştirici hurafeleri altında esir düştü; ve bu nedenle Hindistan, öncül dönemlerde vaat ediği yüksek medeniyeti gerçekleştirmede başarısız oldu.
\vs p079 4:8 İsa’dan önceki ruhsal doğum, Muhammed topluluklarının akınlarından önce bile çoktan ortadan kaybolmuş bir biçimde, Hindistan’da varlığını sürdürmedi. Ancak ileride büyük bir Gotama Buda, yaşayan Tanrı’yı bulma amacı içinde tüm Hindistan’ı yönetmek için doğabilir; ve bunun sonrasında dünya, gelişmeyen bir ruhsal tasavvurun uyuşturan etkisi altında uzun yıllar baygın yatan çok yönlü bir topluluğun kültürel yetkinliklerinin meyvelerini gözlemleyecektir.
\vs p079 4:9 Kültür biyolojik bir temele dayanmamaktadır; ancak tabakalaşmış toplum tek başına Ari kültürünü ebedileştirmeye yeterli olamamıştır; çünkü din, gerçek din, insan kardeşliğine dayanan üstün bir medeniyeti oluşturmak için insanları bir araya yönelten bu yüksek enerjinin hayati kaynağıdır.
\usection{5.\bibnobreakspace Kırmızı ve Sarı Irklar}
\vs p079 5:1 Her ne kadar Hindistan’ın tarihi, And fetihlerinden ve daha eski evrimsel insan topluluklarının nihai olarak ortadan kayboluşundan ibaret olsa da; doğu Asya’nın tarihi daha belirgin bir biçimde, özellikle kırmızı ve sarı insanlar olmak birincil Sangik unsurlarından meydana gelmektedir. Bu iki ırk büyük bir ölçüde, Avrupa’da mavi ırkı oldukça fazla bir biçimde gerileten değersizleşmiş Neanderthal ırk kolu ile karışmaktan kurtarmış ve böylelikle birincil Sangik türünün üstün yetisini korumuştur.
\vs p079 5:2 Öncül Neanderthal unsurları Avrasya’nın tamamına yayıldıklarında doğu kanadı, alt düzeydeki hayvan ırk kollarına daha çok bulaşmış bir haldeydi. Bu alt düzey insan türleri, beşinci buzul döneminde güneye doğru itilmiştir; aynı buz tabakası Sangik ırkların doğu Asya’ya olan göçünü uzun yıllar engellemiştir. Ve kırmızı insanlar Hindistan düzlükleri etrafından kuzeydoğuya doğru hareket ettiklerinde, kuzeydoğu Asya’yı olumlu bir biçimde bu insanlardan yoksun halde buldular. Kırmızı ırkların kabile örgütlenişi, diğer insan topluluklarınkinden daha önce oluşturulmuştu; ve onlar, Sangik toplulukların merkezi Asya odağından göçen ilk unsurlardı. Alt düzey Neanderthal ırk kolları, daha sonra göç eden sarı kabileler tarafından ya yok edildi ya da buradan uzaklaştırıldı. Ancak kırmızı insanlar, sarı kabilelerin varışından önce neredeyse yüz bin yıl boyunca doğu Asya içinde üstünlüklerini korumuşlardı.
\vs p079 5:3 Üç bin yıldan daha fazla bir süre önce sarı ırkların ana bünyesi Çin’e, sahil şeridi göçmenleri olarak güneyden giriş yaptı. Her bin yılda onlar karanın iç kesimlerinin daha da derinlerine gittiler; ancak onlar, görece yakın dönemlere kadar göç halindeki Tibet kardeşleriyle iletişimde bulunmadılar.
\vs p079 5:4 Büyüyen nüfus baskısı, kuzey doğrultusunda hareket eden sarı ırkın kırmızı insanların av alanlarına girmeye başlamasına neden oldu. Doğal ırksal çekişme ile bütünleşen bu haneye tecavüz, artan düşmanlıklar ile sonuçlandı; ve böylece, uzak Asya’nın verimli toprakları için hayati bir mücadele başlamış oldu.
\vs p079 5:5 Kırmızı ve sarı ırklar arasında gerçekleşen bu çağlar süren mücadelenin hikâyesi Urantia tarihinin bir destanıdır. İki yüz bin yıldan daha fazla bir süre boyunca bu iki üstün ırk çetin ve araklıksız savaşlar verdiler. Öncül mücadelelerde kırmızı insanlar genel olarak başarılıydı; onların saldırı toplulukları sarı yerleşkeleri arasında büyük hasarlar bırakmaktaydı. Ancak sarı insanlar savaş sanatında çabuk öğrenen bir öğrenciydi; ve onlar öncül bir biçimde, yurttaşlarıyla beraber dikkate değer bir barışçıl halde yaşama yetisi sergilemişlerdi; Çinliler, birlikten kuvvetin doğduğunu öğrenen ilk topluluktu. Kırmızı kabileler her iki tarafın da sonunu getiren iç çatışmaları sürdürmeye devam etmişlerdi; ve onlar yakın bir zaman içerisinde, kuzeye doğru önlenemez yürüyüşünü sürdüren acımasız Çin topluluklarının saldırgan ellerinde tekrar eden yenilgilerini deneyimlemeye başladılar.
\vs p079 5:6 Yüz bin yıl önce kırmızı ırkın katliama uğramış kabileleri, son buzul hareketinin geri çekilmekteki hareketine sırtlarını verip savaşmaktalardı; ve Bering kara köprüsü üzerinden Batı’ya olan kara irtibatı tekrar geçilebilir olduğunda, bu kabileler Asya kıtasının düşmansı sahillerini terk etme de hiç de yavaş davranmadılar. Saf kırmızı ırkın son üyelerinin Asya kıtasından ayrılışı üzerinden seksen beş bin yıl geçmiştir; ancak bu uzun süreli savaş kalıtımsal damgasını bu galip sarı ırk üzerinde bırakmıştır. Kuzey Çin toplulukları, Andon kökeninden gelen Siberliler ile birlikte, kırmızı ırk kolunun çoğunu ortadan kaldırmış olup böylece kendi adlarına dikkate değer bir yarar sağlamışlardı.
\vs p079 5:7 Kuzey Amerika’lı Kızılderililer, Âdem’in varışından yaklaşık elli bin yıl önce Asya anavatanlarından mahrum bırakılmış bir biçimde, Âdem ve Havva’nın And doğumları ile bile görmemişlerdi. And göçlerinin bu çağı boyunca saf kırmızı ırk kolları, tarımı küçük bir ölçüde uygulayan avcılar halindeki göçebe kabileler olarak Kuzey Amerika’nın tamamı üzerine yayılmaktalardı. Bu ırklar ve kültürel topluluklar, Amerika Kıtalarına olan varışlarından Avrupa’nın beyaz ırkları tarafından keşfedildikleri dönem olan İsa sürecinin ilk bin yılının sonuna kadar dünyanın geri kalanından neredeyse tamamen tecrit edilmiş bir biçimde kaldılar. Bu döneme kadar Eskimolar, kırmızı insanların kuzey kabilelerinin görebildiği beyaz ırka en yakın olan topluluktu.
\vs p079 5:8 Kırmızı ve sarı ırk, And unsurları etkileri olmadan yüksek bir medeniyet düzeyine erişebilmiş tek insan ırklarıdır. En eski Kızılderili kültürü, Kaliforniya içindeki Onamonalonton merkeziydi; ancak burası, M.Ö. 35.000’li yıllarda çoktan yok olmuş bir haldeydi. Merkezi Amerika olarak Meksika’da ve Güney Amerika’nın dağlarında ileriki dönemlere ait ve daha dayanıklı medeniyetler, başat olarak kırmızı insanlardan oluşan ancak dikkate değer bir biçimde sarı, turuncu ve mavi ırkların karışımını taşıyan bir ırk tarafından kurulmuştu.
\vs p079 5:9 Bu medeniyetler, her ne kadar Peru’ya ulaşan And kanı izlerini taşısa da, Sangik topluluklarının evrimsel çabalarıydı. Kuzey Amerika’daki Eskimolar ve Güney Amerika’daki birçok Polinezya’lı And topluluğu dışında Batı Yarımküresi’nin insan toplulukları, İsa’dan sonraki ilk bin yılın sonuna kadar dünyanın geri kalanıyla hiçbir iletişime sahip değildi. Urantia ırklarının gelişimine dair özgün Melçizedek tasarımında Âdem’in saf soyundan gelen bir milyon üyenin Amerika Kıtaları’nın kırmızı insanlarını canlandırmak için gidişleri şartlandırılmıştı.
\usection{6.\bibnobreakspace Çin Medeniyeti’nin Doğuşu}
\vs p079 6:1 Kırmızı insanları Kuzey Amerika’ya itişlerinden kısa bir süre sonra bu genişleyen Çin toplulukları; doğu Asya’nın nehir vadilerinden Andon unsurlarını, kuzeyde Sibirya’ya ve güneyde And unsurlarının üstün kültürü ile yakın zamanda irtibata geçecekleri yer olan Türkistan’a iterek temizlemişlerdi.
\vs p079 6:2 Burma ve Hindiçini yarımadası içinde Çin ve Hindistan kültürleri, bu bölgelerin ilerideki medeniyetlerini yaratmak için bir araya gelmiş ve karışmışlardı. Burada nesli tükenmiş yeşil ırk, dünyanın herhangi bir yerindekinden daha büyük bir oranda varlığını sürdürmüştür.
\vs p079 6:3 Birçok farklı ırk Büyük Okyanus’un adalarını doldurmuştur. Genellikle güneydeki ve bu dönemin daha geniş adaları, yeşil ve çivit kanının daha yüksek bir oranını taşıyan topluluklar tarafından ikamet edilmişti. Kuzey adalar Andon toplulukları tarafından tutulmaktaydı; ve daha sonra buralar, sarı ve kırmızı ırk kollarının büyük oranlarından meydana gelen ırklar tarafından elde edilmişti. Japonlar’ın kökenleri, kuzey Çin kabilelerinin güçlü bir güney\hyp{}sahil göçü tarafından itildikleri M.Ö. 12.000’li yıllara kadar anavatanlarından uzaklaştırılmamışlardı. Onların büyük nihai göçleri, kutsal bir şahsiyet olarak görmeye başladıkları bir kabile liderinin kararından ziyade baskın bir biçimde nüfus baskısı yüzünden gerçekleşmemişti.
\vs p079 6:4 Hindistan ve Levant’ın insanları gibi sarı ırkın muzaffer kabileleri, sahil şeritleri boyunca ve nehirlere kadar öncül merkezlerini kurmuşlardı. Kıyı yerleşkeleri, artan seller ve akım yönü değişen nehirler ova şehirlerini savunulamaz kıldığı için, daha sonraki yıllarda yetersiz yaşam olanaklarını yaratmıştı.
\vs p079 6:5 Yirmi bin yıl önce Çin topluluklarının ataları, özellikle Sarı ve Yangtze nehri boyunca, ilkel kültür ve eğitimin bir düzine güçlü merkezini kurmuş haldelerdi. Ve bu aşamada bahse konu merkezler, Doğu Türkistan ve Tibet’den gelen üstün düzeydeki melez insanların devamlı bir akışının varışı tarafından güçlenmeye başlamıştı. Tibet’den Yangtze vadisine olan göç kuzeydekine kıyasla geniş çaplı bir düzeyde bulunmamaktaydı; buna ek olarak Tibet merkezleri, Tarim havzasındakilere kıyasla o kadar gelişmiş bir konumda değildi. Ancak bu iki göç hareketi de, belirli düzeydeki And kanını doğu yönündeki kuzey yerleşkelerine taşımıştı.
\vs p079 6:6 Tarihi sarı ırkın üstünlüğü dört büyük etkene dayanm
\vs p079 6:7 1.\bibnobreakspace \bibemph{Kalıtımsal}: Avrupa’daki mavi kuzenlerinin aksine kırmızı ve sarı ırklar büyük ölçüde, değersizleşmiş insan ırk kolları ile karışmaktan kurtulmuş bir halde bulunmaktalardı. Üstün kırmızı ve Andon ırk kollarının küçük düzeydeki oranları ile çoktan güçlenmiş haldeki Kuzey Çin toplulukları, yakın bir zamanda And kanının dikkate değer nüfuzundan yarar sağlayacaktı. Güney Çin toplulukları için işler bu hususta iyi gitmedi; ve onlar, yeşil ırkın üstünlüğü altında onlara karışmaktan uzun bir süre zarar görmüş haldelerdi; ileride onlar, Dravid\hyp{}And istilası nedeniyle Hindistan’dan uzaklaştırılan alt düzey insanların birçoğunun nüfuzuyla daha da zayıflayacaklardı. Ve bugün Çin içinde, kuzey ve güney ırkları arasında belirgin bir farklılık bulunmaktadır.
\vs p079 6:8 2.\bibnobreakspace \bibemph{Toplumsal}. Sarı ırk öncül bir biçimde, kendi aralarında sahip oldukları barışın değerini öğrendi. Onların içsel uzlaşabilirliği nüfuslarının artmasına böylelikle o kadar katkıda bulundu ki, milyonlara varan sayıları arasında medeniyetlerinin yayılmasını sağladı. M.Ö. 25.000 ile 5.000’li yıllar arasında Urantia üzerinde en yüksek kitle nüfusu, merkezi ve kuzey Çin’deydi. Sarı insan toplulukları, --- geniş çaplı bir kültürel, toplumsal ve siyasi medeniyete erişen ilk bireyler olarak --- bir ırksal bütünlüğe ulaşan ilk bütünlüktü.
\vs p079 6:9 M.Ö. 15.000’li yılların Çin toplulukları savaşçı askerlerdi; onlar, geçmişe haddinden fazla duyulan bir saygıyla güçsüzleşmemişlerdi; ve on iki milyondan daha az bir nüfusla onlar, ortak bir dilin bütüncül bir birlikteliğini oluşturdular. Bu çağ boyunca, geçmiş dönemdeki siyasi birlikteliklerinden çok daha bütüncül ve uyumlu bir nitelikte, gerçek bir milleti inşa ettiler.
\vs p079 6:10 3.\bibnobreakspace \bibemph{Ruhsal}. And göçleri çağı boyunca Çin toplulukları, dünya üzerindeki daha ruhsal olan birliktelikler arasındaydı. Singlangton’un duyurduğu Tek Gerçek’e ibadetine sergiledikleri uzun süreli bağlılık onları diğer ırkların önünde bir konumda tutmuştu. İlerleyici ve gelişmiş bir dinin etkisi kültürel gelişimde sıklıkla belirleyici bir etkendir; Hindistan cansızlaşırken Çin, içinde gerçekliğin Yüce İlahiyat olarak kutsal bir konumda değerlendirildiği dinin canlandırıcı etkisi içinde sağlam adımlarla ilerlemekteydi.
\vs p079 6:11 Gerçekliğin bu ibadeti, doğa yasalarına ek olarak insan türünün yetileri üzerinde gerçekleştirilen kalıpların dışına çıkan araştırma ve korkusuz keşifti. Altı bin yıl öncesinin dahi Çin toplulukları, hala azimli öğrenciler halinde olup gerçekliğe ulaşma arzusu içinde fazlasıyla girişkenlerdi.
\vs p079 6:12 4.\bibnobreakspace \bibemph{Coğrafi}. Çin, batısındaki dağlar ve doğusundaki Büyük Okyanus tarafından korunmaktadır. Buranın sadece kuzey kısmı saldırıya açıktır; ve kırmızı insan döneminden And unsurlarının daha sonraki soylarının varışına kadar kuzey herhangi bir saldırgan ırk tarafından ikamet edilmemiştir.
\vs p079 6:13 Ve, dağ engelleri ve daha sonra gerçekleşen ruhsal kültürdeki düşüş meydana gelmeseydi sarı ırk; Türkistan’dan gelen And göçlerinin büyük bir kısmını kuşkusuz kendisine doğru çekecek, ve kesinlikle dünya medeniyeti üzerinde üstünlüğünü çabucak ilan edecekti.
\usection{7.\bibnobreakspace And Unsurları’nın Çin’e Girişi}
\vs p079 7:1 Yaklaşık on beş bin yıl önce And toplulukları, dikkate değer bir nüfusla, Ti Tao boyunca ilerlemekte ve Kansu’nun Çin yerleşkeleri arasında Sarı Nehir’in üst vadisine yayılmaktalardı. Yakın bir zaman içerisinde onlar, en ilerleyici yerleşkelerin konumlandığı bölge olan Honan’a doğru doğu yönünde ilerlemişlerdi.
\vs p079 7:2 Sarı Irmak boyunca kuzey kültür merkezleri, Yangtze üzerindeki güney yerleşkelerine kıyasla her zaman daha ilerleyici bir konum içerisindeydi. Bu üstün fanilerin küçük bir nüfusunun bile varışından sonra birkaç bin yıl içerisinde Sarı Irmak boyunca konumlanmış yerleşkeler; Yangtze köylerinin çok ötesine geçmiş olup, bahse konu dönemden bu güne kadar yönettikleri güneydeki kardeşleri üzerinde gelişmiş bir konumu elde etmiş bir halde bulunmaktaydılar.
\vs p079 7:3 Bu durumun nedeni ne birçok And unsurunun gelmiş olması ne de onların kültürünün çok üstün bir konumda bulunması değildi; ancak onların karışımı kendilerini daha çok yönlü bir ırk kolu haline getirdi. Kuzey Çin toplulukları; kendilerinin içkin düzeydeki yetkin akıllarını narince kamçılayacak fakat kuzey beyaz ırkların oldukça belirgin özelliği olan yerinde duramayan keşfedici merakıyla onu yanıp tutuşturmayacak düzeyde And ırk kolunu tam da kararında almışlardı. Bu daha kısıtlı düzeydeki And kalıtım nüfuzu, Sangik türün içkin istikrarı için daha az rahatsız ediciydi.
\vs p079 7:4 And unsurlarının daha sonraki dalgaları kendilerine, Mezopotamya’nın belirli kültürel gelişmelerini beraberinde getirdi; bu durum özellikle batıdan gelen son göç dalgaları için doğruluk taşımaktadır. Onlar fazlasıyla, kuzey Çin topluluklarının ekonomik ve eğitimsel uygulamalarını geliştirmişlerdi; ve sarı ırkın dini kültürü üzerinde etkileri kısa süreli olsa da, onların daha sonraki soyları ileri dönemdeki bir ruhsal uyanışa fazlasıyla katkıda bulunmuştur. Ancak Cennet Bahçesi ve Dalamatia’nın güzelliğine dair And kültürünün tarihi anlatımları, Çin geleneklerini etkilemişti; öncül Çin efsaneleri “tanrıların yerleşkesini” batıda konumlandırmaktadır.
\vs p079 7:5 Çin insanları; Türkistan’daki iklim değişikliklerinin ve daha sonraki And göçmenlerinin varışının ardından, M.Ö. 10.000’li yılların sonuna kadar bile şehirler inşa etmeye ve üretime sürecine giriş yapmaya başlamamışlardı. Bu yeni kanın nüfuzu, üstün Çin ırk kollarının dışarıdan çok belirgin olmayan ileri ve hızlı gelişim eğilimlerini etkilemesi karşısında sarı ırkın medeniyetine büyük oranda katkı sağlamamıştır. Honan’dan Şensi’ye kadar gelişmiş bir medeniyetin yetileri meyvelerini vermeye başlamaktaydı. Madeni eşyalar yapım ve tüm el işi üretim sanatları bu dönemlerden gelmektedir.
\vs p079 7:6 Zamanı hesaplama, gökbilim ve yönetimsel idareye dair öncül Çin ve Mezopotamya kültürünün yöntemlerinin bazıları arasındaki benzerlikler, birbirinden uzakta konumlanan bu merkezler arasındaki ticaret ilişkilerinden kaynaklanmaktadır. Çin tüccarları, Sümerler döneminde bile Türkistan boyunca Mezopotamya’ya kara güzergâhları üzerinde seyahat etmişlerdi. Ve bu değiş tokuş tek taraflı değildi --- Fırat vadisi, Ganj düzlükleri insanlarının yarar sağladığı gibi ciddi bir biçimde bu bireylerden faydalanmıştı. İsa’dan önceki üçüncü bin yılın iklim değişiklikleri ve göçebe istilaları, merkezi Asya’da kervan yolları üzerinde geçiş halindeki ticaret hacmini fazlasıyla azalttı.
\usection{8.\bibnobreakspace Daha Sonraki Çin Medeniyeti}
\vs p079 8:1 Kırmızı insanlar savaşlardan fazlasıyla olumsuz etkilenirken, Çin toplulukları arasında devlet kurumunun gelişiminin onların Asya’yı fethedişindeki titizliği nedeniyle gecikmiş olduğunu söylemek tamamiyle yanlış olmaz. Onlar ırksal bütünlüğü elde etmeye dair büyük bir olanağa sahiptiler; ancak onlar bunu yerinde bir biçimde geliştirmekte, dışsal düşmanlığın sürekli mevcut tehlikesinin devamlılıkla yönlendiren etkisine sahip olmamalarından dolayı başarısız oldular.
\vs p079 8:2 Doğu Asya’nın fethedilişinin tamamlanması ile birlikte tarihi askeri devlet kademeli olarak parçalandı --- eskiden verilmiş savaşlar unutuldu. Kırmızı ırk ile giriştikleri destansı mücadeleden geriye sadece, okçu insanlarla gerçekleştirilen belirsiz bir tarihi yarışma geleneği kaldı. Çin toplulukları öncül bir biçimde, barışçıl eğilimlerine daha fazla katkı sağlayan tarım uğraşlarına geride döndü; bunun karşısında, tarım için toprak insan oranının çok altında bulunan bir nüfus daha fazla bir biçimde ülkede büyüyen barışa katkıda bulundu.
\vs p079 8:3 Fazlasıyla baskın bir tarım toplumunun muhafazakârlığı olarak, geçmiş kazanımlara dair bilinç (her ne kadar bugün bir ölçüde azalmış olsa da) ve oldukça gelişmiş bir aile yaşamı; ibadet sınırına yaklaşan bir düzeyde geçmişin insanlarını onurlandırma geleneğine yol açarak atalara duyulan saygının doğuşunu beraberinde getirdi. Oldukça benzer bir tutum, Greko\hyp{}Romen medeniyetinin bozulmasından sonra yaklaşık beş yüz yıl boyunca beyaz ırklar arasında varlığını sürdürmeye devam etmiştir.
\vs p079 8:4 Singlangton tarafından öğretildiği biçimiyle “Tek Gerçek’e” duyulan inanç ve ona yapılan ibadet hiçbir zaman bütünüyle ortadan kaybolmadı; ancak zaman ilerledikçe yeni ve daha yüksek gerçeğin arayışı, hali hazırda oluşturulana derinden saygı duyma biçimindeki büyüyen bir eğilim tarafından gölgede bırakılmış hale geldi. Yavaş yavaş sarı ırkın usu, bilinenin korunmasından bilinmeyen uğraşına doğru kayan bir konuma geldi. Ve bu durum, bir zamanların dünyanın en hızlı gelişen medeniyetinin durağanlığının sebebi olmuştur.
\vs p079 8:5 M.Ö. 4000 ile 5000’li yıllar arasında sarı ırkın siyasi düzeydeki yeniden bütünlenişi tamamlandı; ancak Yangtze ve Sarı ırmak merkezlerinin kültürel birlikteliği çoktan gerçekleşmiş bir haldeydi. Daha sonraki kabile topluluklarının bu siyasi düzeydeki yeniden birlikteliği savaşsız ortaya çıkmamıştı; ancak savaşa dair toplumsal görüş alt düzeyde kalmaya devam etti; atalara olan ibadet, artan lehçeler ve binlerce yıl boyunca hiçbir savaş çağrısında yapılmayışı bu toplulukları olası en yüksek barışçıl nitelikte kılmıştı.
\vs p079 8:6 Gelişmiş bir devlet düzeninin öncül bir gelişim vaadini yerine getirmedeki başarısızlığa rağmen sarı ırk ilerleyen bir biçimde, özellikle tarım ve bahçecilik alanında medeniyet sanatlarının gerçekleştirilmesinde yol kat etti. Şensi ve Honan’da tarımla uğraşan kişiler tarafından karşılaşılan sulama sorunları, bu sorunların çözümünde topluluk işbirliğini gerekli kıldı. Bu türden sulama ve toprak korunum zorlukları, çiftçi toplulukları arasında barışın sonuçsal desteklenişi ile birlikte karşılıklı dayanışmanın gelişmesine hiç de az katkı sağlamadı.
\vs p079 8:7 Yakın zaman içerisinde yazıdaki gelişmeler, okulların kurulması ile birlikte, bilginin dağılımına geçmişle karşılaştırılamayacak bir düzeyde katkı sağladı. Ancak kavramsal yazı düzeninin oldukça zor olan yapısı, matbaanın öncül bir biçimde ortaya çıkışına rağmen, eğitilmiş sınıflar üzerinde sayısal bir sınırı beraberinde getirdi. Atalara olan derin saygıya dayanan dini gelişme, doğaya olan ibadeti içine alan bir hurafeler seli tarafından daha da karmaşık bir hal aldı; ancak Tanrı’ya dair gerçek bir kavramın hala varlığını sürdüren kalıntıları, Şang\hyp{}ti’nin imparatorluk ibadeti içinde korumaya devam etti.
\vs p079 8:8 Atalara duyulan derin saygıdan kaynaklanan büyük zaaf, sürekli geriye bakan bir felsefeyi destekliyor olmasıdır. Geçmişten bilgelikler elde etmek bilgece olsa da, geçmişi gerçekliğin tek kaynağı olarak görmek akılsızlıktır. Gerçek göreceli ve genişlemektedir; o her zaman, insanların her neslinde --- hatta her insan yaşamında --- yeni bir dışavurumu elde ederek şimdiki zamanla birlikte \bibemph{yaşamaktadır}.
\vs p079 8:9 Atalara duyulan derin bir saygıdaki fazlasıyla güçlü taraf, aileye verilen bu önemdeki değerdir. Çin kültürünün muhteşem istikrarı ve devamlılığı, aileye atfedilmiş olası en yüksek konumun bir sonucudur; çünkü medeniyet doğrudan bir biçimde, ailenin etkin bir biçimdeki faaliyetine dayanmaktadır; ve Çin’de aile toplumsal bir öneme erişmiş, hatta birkaç diğer topluluk tarafından düşünüldüğü biçimiyle dini bir değeri bile elde etmiştir.
\vs p079 8:10 Atalara olan ibadetin büyümekteki inancının mecbur kıldığı çocuklara olan bağlılık ve aile sadakati, üstün aile ilişkilerinin ve uzun ömürlü aile topluluklarının inşa edilmesini sağladı; şu etkenlerin tümü medeniyetin korunmasını kolaylaştırmıştır:
\vs p079 8:11 1.\bibnobreakspace Özel mülkiyet ve servetin korunumu.
\vs p079 8:12 2.\bibnobreakspace Bir nesilden fazlasının sahip olduğu deneyimi bir araya getirme.
\vs p079 8:13 3.\bibnobreakspace Geçmişin sanatları ve bilimlerinde çocukların etkin bir biçimde eğitilm
\vs p079 8:14 4.\bibnobreakspace Güçlü bir görev duygusunun gelişmesi, ahlakın derinleştirilmes
\vs p079 8:15 And unsurlarının gelişi ile başlayan Çin medeniyetinin oluşum dönemi, İsa’dan önce altıncı asrın büyük çaplı etik, ahlaki ve yarı\hyp{}dini uyanışına kadar devam etmektedir. Ve Çin geleneği, evrimsel geçmişin çok belirgin olmayan bir kaydını muhafaza etmektedir; ana\hyp{}erkil aileden baba\hyp{}erkile geçiş, tarımın oluşturulması, mimarinin gelişmesi ve üretimin başlaması olarak --- tüm bu gelişmeler başarılı bir biçimde anlatılmıştır. Ve bu hikâye, diğer herhangi benzer bir anlatıma kıyasla daha büyük bir doğrulukla, barbarlığın düzeylerinden üstün bir insan topluluğunun muhteşem yükselişine dair resmi temsil etmektedir. Bu dönem boyunca onlar; ilkel bir tarım toplumundan geçerek şehirlerden, el işleri imalatından, madeni eşyalar yapımından, ticari alışverişten, hükümetten, matematikten, sanattan, bilimden ve matbaacılıktan meydana gelen yüksek bir toplum örgütlenişine doğru ilerlemişlerdir.
\vs p079 8:16 Ve böylelikle sarı ırkın tarihi medeniyeti, çağlar boyunca varlığını sürdüregelmiştir. Çin topluluklarının kültürü içinde ilk önemli gelişmelerin gerçekleştirilmesinden bu yana neredeyse kırk bin yıl geçmiştir; ve her ne kadar orada birçok gerileme yaşanmış olsa da, Han evlatlarının medeniyeti yirminci yüzyıl dönemine kadar düzenli ilerlemenin bütüncül bir resmini yansıtan en yakın örnektir. Beyaz ırkların teknik ve dini gelişimleri yüksek bir düzeye ait olagelmiştir; ancak onlar hiçbir zaman Çin topluluklarını sadakat, topluluk etiği veya kişisel ahlakta geçememişlerdir.
\vs p079 8:17 Bu eski kültür, insan mutluluğuna fazlasıyla katkıda bulunmuştur; milyonlarca insan, kazanımlarıyla kutsanmış bir biçimde yaşayıp ölmüşlerdir. Çağlar boyunca bu büyük medeniyet, geçmişin şöhretine dayanmıştır; ancak şimdilerde bile, fani mevcudiyetin aşkın amaçlarını yeniden tasavvur etmek için tekrar doğmakta, ve bir kez daha sonu gelmez bir ilerleme için yılmak bilmeyen mücadeleyi vermeye başlamaktadır.
\vs p079 8:18 [Nebadon’un bir Başmelek unsuru tarafından sunulmuştur.]
