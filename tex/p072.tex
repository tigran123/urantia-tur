\upaper{72}{Komşu Bir Gezegen Üzerindeki Hükümet}
\vs p072 0:1 Lanaforge’nin izni ve Edentia’nın En Yüksek Unsurları’nın onayı ile ben, Satania sistemine ait çokta uzakta bulunmayan bir gezegen üzerinde yaşayan en ileri insan ırkının toplumsal, ahlaki ve siyasi yaşamına dair birtakım şeyleri anlatmak için yetkinlendirildim.
\vs p072 0:2 Lucifer isyanına katılımları dolayısıyla tecrit altına alınmış Satania dünyalarının tümü içerisinde bu gezegen, Urantia’nınkine oldukça benzer bir tarihi deneyimlemiştir. Bu iki âlemin benzerliği kuşkusuz bir biçimde, bahse konu bu olağandışı sunumun yapılmasına neden izin verildiğini açıklamaktadır; çünkü sistem yöneticilerinin bir gezegenin olaylarının diğeri üzerinde anlatılmasına izin vermeleri en olağandışı durumdur.
\vs p072 0:3 Urantia gibi bu gezegen, Lucifer isyanı ile ilgili olarak sahip olduğu Gezegensel Prensi’nin sadakatsizliği tarafından doğru düzenden uzaklaşmıştı. Âdem’in Urantia’ya gelmesinden kısa bir süre sonra bir Maddi Evlat’ı almıştı; ve bu Evlat’da aynı zamanda bu âlemin tecrit altına alınmasına sebebiyet veren bir biçimde görevini yerine getirmede başarısız olmuştu; çünkü bir Hakimane Evlat, fani ırklarına hiçbir zaman bahşedilmemiştir.
\usection{1.\bibnobreakspace Kıtasal Ülke}
\vs p072 1:1 Bahse konu bu gezegensel engellerin tümüne rağmen oldukça üstün bir medeniyet, Avustralya’nın büyüklüğüne yaklaşık bir alana sahip olan tecrit altına alınmış bir kıta üzerinde evirilmektedir. Bu ülkenin nüfusu yaklaşık olarak 140 milyondur. Onun insanları, tarafınızdan adlandırılmakta olan Urantia’nın beyaz ırkı ve bu insanlardan biraz daha fazla olan eflatun ırka ek olarak başlıca mavi ve sarı ırklardan meydana gelen bir biçimde karma bir ırktır. Bu farklı ırklar henüz birbirlerine bütünüyle karışmamışlardır; ancak onlar oldukça makul bir biçimde birbirleriyle bütünleşmiş ve toplumsallaşmıştır. Bu kıta üzerinde ortalama insan yaşamı mevcut an içerisinde; gezegen üzerindeki herhangi bir diğer topluluktan yüzde on beş daha fazla olan bir biçimde, doksan yıldır.
\vs p072 1:2 Bu ülkenin üretim işleyiş düzeni, kıtanın benzersiz bir arazi dağılımından kaynağını alan belirli bir büyük üstünlük yaratan farklılığı memnuniyetle deneyimlemektedir. Şiddetli yağmurların yılda sekiz ay düştüğü yüksek dağlar ülkenin en merkez bölgesinde konumlanmıştır. Bu doğal oluşum su gücünün kullanılması için elverişlilik yaratmakta ve kıtanın daha kurak olan batı bölgesinin sulanmasını büyük ölçüde kolaylaştırmaktadır.
\vs p072 1:3 Bu topluluk kendi yaşamlarını tek başlarına idame ettiren bir birlikteliktir; onlar, etrafındaki ülkelerden herhangi bir şey ithal etmeden sonsuza kadar yaşayabilir. Onların doğal kaynakları oldukça fazladır; ve bilimsel yöntemler vasıtasıyla onlar, yaşamın temel gereksinimleri arasında yoksun oldukları şeyleri nasıl telafi etmeleri gerektiğini öğrenmişlerdir. Onlar canlı bir iç ticareti memnuniyetle deneyimlemektedirler; ancak onlar, daha az gelişmiş olan komşularına takındıkları ortak düşmanlık sebebiyle çok az bir dış ticarete sahiptirler.
\vs p072 1:4 Bu kıtasal ülke genel olarak, kabile döneminden binlerce yıllık zaman zarfını kaplayan güçlü yöneticiler ve kralların ortaya çıktığı sürece doğru gelişerek gezegenin evrimsel eğilimini takip etmişti. Mutlak hükümdarları, başarısız cumhuriyetlerin, özel mülkiyetin müşterek olduğu devletlerin ve diktatörlerin sayısız kere gelip gittiği bir biçimde, hükümetin birçok farklı düzeni takip etmiştir. Bu büyüme; devleti mutlak bir biçimde yöneten üçlü idareciden birinin değişeceği döneme kadar olan, bir siyasi mayalanma süreci boyunca yaklaşık beş yüz yıl sürmüştür. Bahse konu bu yönetici, yönetimin ortağı diğer iki hükümdarın birinin de hükümranlığından feragat etmesi koşuluyla tahtından çekilmeye gönüllü olmuştur. Böylelikle kıtanın egemenliği tek bir yöneticinin elinde toplanmıştır. Bütünleşen devlet, özgürlüğün üstün bir sözleşmesinin geliştiği süreç olan, güçlü krallık idaresi altında bir yüzyıl daha ilerlemiştir.
\vs p072 1:5 Krallık yönetiminden hükümetin temsili bir türüne olan bir sonraki geçiş, kralların yalnızca toplumsal veya duygusal nitelikte simgesel önderler olarak kaldığı ve nihai olarak hükümranlığın erkek kolunun tükenmesiyle ortadan kalktığı bir biçimde, kademeli olmuştur. Mevcut cumhuriyet bu zaman zarfı içerisinde yalnızca iki yüz yıl yaşındadır; bu süreç içerisinde orada, üretim ve siyasi alanda yapılan değişikliklerin geçmiş on yılda hayata geçirilmiş olduğu hükümetsel yöntemler alanında, anlatılacak olan, sürekli bir gelişme söz konusu olmuştur.
\usection{2.\bibnobreakspace Siyasi Düzen}
\vs p072 2:1 Kıtasal ülke şu an, ülke başkentinde merkezi olarak konumlanan bir başkent ile birlikte temsili bir hükümete sahiptir. Merkezi hükümet, göreceli bağımsız yüz eyaletten meydana gelen güçlü bir federasyondan oluşmaktadır. Bu eyaletler valilerini ve meclis üyelerini on yıllığına seçmekte olup, hiçbirinin tekrar seçilmeye hakkı bulunmamaktadır. Eyalet yargıçları valiler tarafından yaşam boyu hizmet vermek üzere atanmakta olup, bu atamalar her birinin yüz bin vatandaşı temsil ettiği milletvekilleri tarafından onaylanır.
\vs p072 2:2 Orada şehrin büyüklüğüne bağlı olarak beş farklı anakent hükümeti bulunmaktadır; ancak hiçbir şehrin bir milyondan fazla sakini barındırmasına izin verilmemektedir. Bütünü itibariyle bahse konu bu belediye yönetim düzenleri oldukça basit, doğrudan ve ekonomiktir. Şehir idaresinin çok az sayıdaki makamı, vatandaşların en yüksek türleri tarafından oldukça kararlı bir biçimde arzulanmaktadır.
\vs p072 2:3 Federal hükümet yönetim, yasama ve yargı erkleri biçiminde üç eş\hyp{}güdüm kolundan meydana gelir. Federal yönetimin baş sorumlusu her altı yılda bir herkesin kendi bölgesinden özgür olarak katıldığı oylarla seçilmektedir. Bu sorumlu, bağlı bulundukları eyalet valileriyle hemfikir en az yetmiş beş eyalet temsilcisinin talebi dışında tekrar seçilememektedirler; böyle bir durumda ise baş sorumluların görevi en fazla bir seçim dönemliğine kadar uzatılır. Bu kişi, yaşayan tüm eski devlet başkanlarından oluşan üstün bir bakanlar kurulunun tavsiyelerini alır.
\vs p072 2:4 Yasama erki üç kamaradan meydana gelir:
\vs p072 2:5 1.\bibnobreakspace \bibemph{Üst meclis}; ekonomik faaliyet uyarınca tercih edilen bir biçimde sanayi, meslek, tarım ve diğer topluluk çalışanlarından seçilmektedir.
\vs p072 2:6 2.\bibnobreakspace \bibemph{Alt meclis}; üretim ve meslek çalışanlarını içine almayan bir biçimde toplumsal, siyasi ve felsefi topluluklardan oluşan belirli toplumsal birlikteliklerden seçilmektedir. İtibar sahibi her vatandaş temsilcilerin bu iki sınıfı için de seçilebilir; ancak onlar, üst veya alt meclisin seçimine bağlı olarak farklı bir biçimde birliktelik kazandırılmışlardır.
\vs p072 2:7 3.\bibnobreakspace \bibemph{Üçüncü meclis}, kıdemli devlet adamları; devlet hizmeti emektarlarından meydana gelip, devlet başkanı, bölgesel (alt\hyp{}federal) yöneticiler, yüce mahkemenin başkanı ve diğer iki yasama meclislerinden birinin başkanı tarafından aday gösterilen birçok seçkin kişiyi içine alır. Bu topluluk yüz kişi ile sınırlandırılmış olup, onun üyeleri kıdemli devlet adamlarının oy çokluğuyla seçilmektedir. Üyelik yaşam boyu olup, boş üyelikler ortaya çıktığında aday listesi arasında en yüksek oyu alan kişi böylelikle yönetmeliğe uygun olarak seçilir. Bu bünyenin kapsamı tamamiyle tavsiye niteliğindedir; ancak bu birim kamuoyunun kudretli bir belirleyicisi olup, hükümetin tüm kolları üzerinde güçlü bir etki yaratmaktadır.
\vs p072 2:8 Federal idari sorumluluğun çok büyük bir kısmı, her biri on eyaletin birlikteliğinden meydana gelmiş on bölgesel (alt\hyp{}federal) yönetim tarafından yerine getirilir. Bu bölgesel birimler, hiçbir şekilde ne yasama ne de yargı faaliyetlerine sahip olan bir biçimde tamamiyle yönetimsel ve idaridir. On bölge yöneticisi federal baş sorumlunun kişisel olarak atadığı bireylerdir; onların görev süresi devlet başkanınki ile uyumlu bir biçimde altı yıldır. Federal düzeydeki yüce mahkeme, bu on bölge yöneticisinin atamasını onaylamaktadır; ve onların yeniden atanmadıkları durumlarda, emekli olan yönetici kendiliğinden yerine geçen kişinin yardımcısı ve danışmanı olur. Bunun dışında kalan hallerde ise bahse konu bölge sorumluları, idari görevlilerden oluşan kendi bakanlar kurulunu seçer.
\vs p072 2:9 Bu ülke, hukuk ve sosyo\hyp{}ekonomik mahkemeler olarak iki ana yargı düzeni tarafından yargılanmaktadır. Hukuk mahkemeleri şu üç ana düzeyde faaliyet göstermektedir:
\vs p072 2:10 1.\bibnobreakspace \bibemph{İlk derece mahkemeleri}, kentsel ve yerel yargıya ait, kararlarına yüksek eyalet mahkemelerinde itiraz edilebilen mahkemelerdir.
\vs p072 2:11 2.\bibnobreakspace \bibemph{Eyalet yüksek mahkemeleri}, federal hükümetin nüfuz alanına girmeyen ve vatandaşlık hakları ve özgürlüklerini tehlikeye atmayan tüm durumlarda kararları nihai olan mahkemelerdir. Bölgesel yöneticiler, federal yüce mahkemeye herhangi bir davayı bir kez getirme gücü ile donatılmışlardır.
\vs p072 2:12 3.\bibnobreakspace \bibemph{Federal yüce mahkeme} --- devlet düzeyindeki anlaşmazlıkların ve eyalet mahkemelerinden gelen temyiz davalarının yargısı için yüksek mahkemedir. Bu yüce mahkeme, bir eyalet mahkemesinde iki veya daha fazla yıl hizmet vermiş olan kırk yaşın üstünde ve yetmiş yaşın altındaki on iki üyeden meydana gelmektedir; bu üyeler bahse konu bu yüksek mevkie, devletin baş sorumlusu tarafından üstün bakanlar kurulu ve üçüncü yasama meclisinin oy çokluğu ile atanmaktadırlar. Bu yüce yargı bünyesinin tüm kararları en az üçte iki oy çokluğu ile kabul edilmektedir.
\vs p072 2:13 Sosyo\hyp{}ekonomik mahkemeler şu üç düzeyde faaliyet göstermektedir:
\vs p072 2:14 1.\bibnobreakspace \bibemph{Aile mahkemeleri}, ev ve toplumsal düzenin yasama ve yürütme birimleri ile ilgili olan mahkemelerdir.
\vs p072 2:15 2.\bibnobreakspace \bibemph{Eğitim mahkemeleri} --- eyalet ve bölge okul düzenlerine ek olarak eğitimsel idare işleyişinin yürütme ve yasama birimleri ile ilgili yargı bünyeleridir.
\vs p072 2:16 3.\bibnobreakspace \bibemph{Üretim mahkemeleri} --- ekonomik anlaşmazlıkların tümünün giderilmesi için tam yetkiyle donatılmış karar mahkemeleridir.
\vs p072 2:17 Federal yüce mahkeme, kıdemli devlet adamlar meclisi olan milli hükümete ait üçüncü yasama kolunun dörtte üçünün oyu alınmadan sosyo\hyp{}ekonomik konularda alınan kararları görüşmemektedir. Bunun durum dışında aile, eğitim ve üretim yüksek mahkemelerinin tüm kararları nihaidir.
\usection{3.\bibnobreakspace Ev Yaşamı}
\vs p072 3:1 Bu kıta üzerinde iki ailenin aynı çatı altında yaşaması kanuna aykırıdır. Ve toplu ikamet yasaklanmış olduğundan apartman tipi binaların çoğu yıkılmıştır. Ancak evli olmayan bireyler hala derneklerde, otellerde ve diğer topluluk meskenlerinde yaşayabilmektedirler. En küçük ev yerleşkesi en az dört bin altı yüz elli metre kare alanı sağlamak zorundadır. Ev amacıyla kullanılan tüm araziler ve diğer taşınmaz mallar en düşük ev yerleşke alanının on katına kadar vergiden muaftır.
\vs p072 3:2 Bu insanların ev yaşamı geçen yüzyıl boyunca büyük ölçüde gelişme göstermiştir. Babalar ve anneler olarak ebeveynlerin çocuk yetiştirilimine dair ebeveyn okullarına katılması zorunludur. Küçük şehir yerleşkelerinde ikamet eden çiftçiler bile bu görevi, her iki hafta içerisinde bir kez, on günde bir sözlü eğitim için en yakında bulunan merkezlere giderek, yazışma yoluyla gerçekleştirmektedir; onların bir haftası beş gündür.
\vs p072 3:3 Her ailede ortalama çocuk sayısı beştir; ve onlar ebeveynlerinin bütüncül denetimi altındadır; ebeveynlerin biri veya ikisi de ölürse, aile mahkemeleri tarafından atanan gözetimcilere vesayetleri emanet edilir. Tamamiyle öksüz bir çocuğun gözetimi ile ödüllendirilmek her bir aile tarafından büyük bir onur olarak değerlendirilir. Ebeveynler arasında çekişmeli sınavlar düzenlenir, ve öksüz çocuk en iyi ebeveynsel nitelikleri sergileyenlerin evine ödül olarak gönderilir.
\vs p072 3:4 Bu insanlar evi, medeniyetin temel kurumu olarak görmektedirler. Bir çocuğun eğitimi ve karakterinin en değerli kısmının ebeveynleri tarafından ve onun evinde teminat altına alınması beklenmektedir; ve babalar çocukların yetiştirilmesinde, neredeyse annelerin gösterdiğine eşit bir ilgiyi göstermektedir
\vs p072 3:5 Cinsel eğitimin tümü evde ebeveynler veya yasal vasiler tarafından verilir. Ahlaki eğitim, okul atölyelerinde dinlenme dönemleri boyunca eğitmenler tarafından verilmektedir; ancak bu durum dinsel eğitim için farklı bir şekilde işlemektedir; din, ev yaşamının temel bir parçası olarak görülmektedir. Tümüyle dini eğitim yalnızca felsefe mabetlerinde halka açık olarak verilmektedir; Urantia kiliseleri gibi din kurumları için özellikle ayrılmış binalar bu insanların yaşantılarında gelişmemiştir. Felsefelerinde din, Tanrı’yı tanımak ve başkaları için hizmet ederek akranlarına duydukları sevgiyi dışa vurmaktadır; ancak onların bu felsefesi, bu gezegendeki diğer milletlerin dini düzeylerine benzer bir nitelikte bulunmamaktadır. Din bu insanlar arasında bütünüyle bir aile meselesi halindedir ki özellikle din birlikteliği için ayrılmış kamuya açık hiçbir yerleşke bulunmamaktadır. Siyasi olarak din ve devlet, Urantia unsurlarının söylem alışkanlarında dışa vurulduğu gibi, tamamiyle ayrıdır; ancak orada, din ve felsefe arasında garip bir örtüşme söz konusudur.
\vs p072 3:6 Ebeveynleri tarafından yerinde bir biçimde eğitildiklerinden emin olmak için çocuklarını sınava tabi tutmak amacıyla her aileyi ziyaret eden ruhsal eğitmenler (Urantia papazlarına benzer bireyler), hükümetin yüksek denetimi altındaydı. Bu ruhsal danışmanlar ve müfettişler şu an, gönüllü bağışlar ile desteklenen bir kurum olarak yeni oluşturulmuş Ruhsal İlerleme Kuruluşu’nun yönlendirmesi altındadır. Muhtemel bir biçimde bu kurum, bir Cennet Hakimane Evladı’nın varışına kadar daha fazla evrimleşmeyecektir.
\vs p072 3:7 Çocuklar, yurttaşlık sorumluluğuna ilk adımlarının gerçekleştiği on beş yaşına kadar yasal bir biçimde ebeveynlerine tabi olmaya devam ederler. Bunun sonrasında her beş senede bir ilerleyen beş aşamalı dönemler halinde kamu faaliyetleri bu türden yaş toplulukları tarafından yerine getirilirken, ebeveynlerine olan taabiyetleri azalır; yine bu dönemde devlete olan yeni yurttaşlık ve toplum sorumlulukları kendileri tarafından üstlenilir. Oy kullanma hakkı yirmi yaşında verilirken, ebeveynlerin onayına ihtiyaç duyulmadan gerçekleştirilecek evlenme hakkı yirmi beş yaşına kadar onlara verilmemektedir; ve çocuklar otuz yaşına ulaştıklarında evden ayrılmakla yükümlüdür.
\vs p072 3:8 Evlenme ve boşanma yasaları ülke bütününde ortaktır. Oy kullanma yaşı olan yirmiden önce evliliğe izin verilmemektedir. Evlenme izni yalnızca; evlilik başvurusundan bir yıl sonra ve gelin ve damadın evlilik yaşamının sorumlulukları ile ilgili ebeveyn okullarında yerinde bir biçimde eğitim gördüklerine dair sahip oldukları belgeleri sunmalarından sonra verilmektedir.
\vs p072 3:9 Boşanma yasaları bir ölçüde gevşektir; ancak ayrılık kararları aile mahkemeleri tarafından verilir; bu kararlar, başvurunun kayıt altına alındığı zaman zarfından bir yıl sonrasına kadar verilmeyebilir; ve bu gezegen üzerinde bir yıllık süreç Urantia’dakinden oldukça fazladır. Her ne kadar sıkı olmayan boşanma kanunlarına sahip olsalar da, mevcut boşanma oranları Urantia’nın medeni ırklarınınkinin sadece yüzde onudur.
\usection{4.\bibnobreakspace Eğitim Düzeni}
\vs p072 4:1 Bu milletin eğitim düzeni, öğrencilerin beş yaşından on sekiz yaşına kadar eğitim gördükleri üniversite öncesi okullarda zorunlu ve karmadır. Bu okullar Urantia’nın emsallerine kıyasla oldukça farklıdır. Bu okullarda hiçbir sınıf bulunmamaktadır; eğitim gününde sadece tek bir ders işlenir; ve ilk üç yıldan sonra öğrencilerin tümü, altlarında bulunan öğrencilere ders veren bir biçimde yardımcı öğretmenler haline gelirler. Kitaplar yalnızca, okul atölyelerinde ve okul çiftliklerinde ortaya çıkan sorunların çözülmesinde yardımcı olmak için saklanan bilgiler için kullanılır. İcatların ve makineleşmenin bu büyük çağı olan bir dönemde, kıtada kullanılan mobilyanın çoğu ve birçok mekanik düzenek bu atölyelerde üretilmektedir. Her atölyenin yanında, öğrencilerin gerekli kaynak kitaplara başvurabileceği bir çalışma kütüphanesi bulunmaktadır. Tarım ve bahçecilik aynı zamanda, her yerel okulun yanında bulunan geniş tarlalar üzerinde bütün eğitim dönemi boyunca öğretilmektedir.
\vs p072 4:2 Akli bakımdan zayıf olan bireyler yalnızca tarım ve hayvancılıkta eğitilmektedirler; ve onlar yaşam boyu, olağan düzeyin altında bulunan tüm bireyler için yasaklanan ebeveynliğe girişlerini engellemek için cinsiyetlerine göre ayrıldıkları özel gözetim topluluklarına bağlanırlar. Bu kısıtlayıcı önlemler yetmiş beş yıldır faaliyettedir; bahse konu bağlanma hükümleri, aile mahkemeleri tarafından verilir.
\vs p072 4:3 Herkes bir aylığına tatile çıkmaktadır. Üniversite öncesi okullar on aylık yılın dokuz ayında faaliyet gösterir; bahse konu tatil dönemleri ebeveynler veya arkadaşlar ile gerçekleştirilen seyahatlerde harcanır. Bu seyahat; erişkin\hyp{}eğitim programının bir parçası olup, yaşam boyunca devam eder; bu türden seyahatleri karşılamak için gerekli kaynaklar, emeklilik sigortası yöntemlerinde kullanılan aynı işleyiş biçimleriyle birikmektedir.
\vs p072 4:4 Okul zamanının dörtte biri, rekabete dayalı atletizm olarak oyuna adanmıştır; öğrenciler bu yarışmalara yerel oyunlardan başlayarak eyalet ve bölge boyunca ülke düzeyine kadar gerçekleştirilen kabiliyet ve güç yarışlarında ilerlerler. Benzer bir biçimde hitabet ve müziğe ek olarak bilim ve felsefe alanında gerçekleştirilen yarışmalar, alt düzey yerel seviyelerden ülkesel çapta ödüller için gerçekleştirilen mücadelelere kadar öğrencilerin ilgisini çekmektedir.
\vs p072 4:5 Okul hükümeti, bağlı üç kademeli ülke hükümetinin bir nüshasıdır; öğretmen kadrosu, bu okul hükümetinin üçüncü veya diğer bir değişle tavsiye nitelikli yasama bölümü olarak faaliyet göstermektedir. Bu kıta üzerinde eğitimin temel amacı, her öğrenciyi kendi kendisine yeten bir vatandaş haline getirmektir.
\vs p072 4:6 On sekiz yaşında üniversite öncesi okul düzeninden mezun olan her çocuk kabiliyetli birer zanaatkâr haline gelir. Bu aşamadan sonra ya erişkin okullarında veya üniversitelerde kitaplardan çalışma ve özel bilgiyi elde etme süreci başlar. Mükemmel bir öğrenci çalışmasını tasarlanan çizelgeden önce tamamlarsa, kendisine bir çabukluk ödülü verilir; bu çabukluk ödülüyle birlikte bahse konu öğrenci, kendisinin ilgiyle tasarlamış olduğu bir yaratıcılık çalışmasını hayata geçirebilir. Eğitim düzeninin bütünü, bireyi yetkin bir biçimde hazırlamak için tasarlanmıştır.
\usection{5.\bibnobreakspace Üretim Düzeni}
\vs p072 5:1 Bu insanlar arasında üretimin mevcut durumu, nihai hedeflerinden oldukça uzaktır; sermaye ve iş gücü hala kendisine özel sorunlara sahiptir; ancak onların ikisi de, içten iş birliği tasarımına uyumlu hale gelmektedir. Bu benzersiz kıta üzerinde işçiler artan bir biçimde, tüm üretim sorunlarında birer söz sahibi haline gelmektedirler; her us sahibi işçi yavaş bir biçimde küçük bir sermaye sahibine dönüşmektedir.
\vs p072 5:2 Toplumsal düşmanlıklar azalmakta, iyi niyet hızlı bir biçimde büyümektedir. Köleliğin kaldırılması sonucunda (yaklaşık yüz yıl önce) büyük ölçekli hiçbir sorun yaşanmamıştır; çünkü bu uygulama, her yıl kölelerin yüzde ikilik bir kısmının özgürleştirilmesi vasıtasıyla kademeli olarak gerçekleştirilmiştir. Akılsal, ahlaki ve fiziksel sınavları başarıyla geçen kölelere vatandaşlık hakkı verilmişti; bu üstün kölelerin birçoğu, savaş esirleri veya bu esirlerin çocuklarıydılar. Elli yıl önce onlar, alt düzeyde bulunan kölelerinin geride kalan son unsurlarını sınır dışı etmişlerdi; ve daha da yakın bir zaman içinde onlar, bayağı ve kötü niyetli olan toplumsal sınıfların nüfusunu düşürme görevine kendilerini adamışlardır.
\vs p072 5:3 Bu insanlar yakın zaman içerisinde, üretim anlaşmazlıklarının giderilmesine ek olarak bu tür sorunların çözümü için var olan eski yöntemler üzerinde ciddi ölçekli iyileştirmeleri getiren ekonomik sömürünün düzeltilmesi için yeni işleyiş biçimleri geliştirdiler. Kişisel veya üretimsel farklılıkların uyumlu hale getirilmesinde şiddet bir işleyiş biçimi olarak yasa dışı kabul edilmiştir. İş gücü ücretleri, karlar ve diğer ekonomik sorunlar keskin hatlar dâhilinde düzenlenmemiştir; ancak onlar genellikle, üretimden sorumlu yasama organı tarafından denetlenmektedir; bununla birlikte üretimden doğan anlaşmazlıklar, üretim mahkemeleri tarafından giderilmektedir.
\vs p072 5:4 Üretim mahkemeleri yalnızca otuz yaşındadır; ancak bu mahkemeler oldukça başarılı bir biçimde faaliyet göstermektedir. En yeni gelişme, üretim mahkemelerinin bundan böyle şu üç sınıfa düşen yasal tazminatı tanıyacak olması değişikliğini sağlamaktadır:
\vs p072 5:5 1.\bibnobreakspace Üzerinde yatırım yapılmış sermayeden elde edilen yasal faiz oranları.
\vs p072 5:6 2.\bibnobreakspace Üretim faaliyetlerinde çalıştırılan maharet için kabul edilebilir ücret.
\vs p072 5:7 3.\bibnobreakspace İş gücü için adil ve hakkani ücretler.
\vs p072 5:8 Adil ve hakkani ücret koşulları ilk önce imzalanan sözleşme uyarınca yerine getirilmeli, veya azalan gelirler karşısında ise kayıplar ücretler üzerinden geçici kısıntıya gidilerek paylaşılmalıdır. Ve belirli giderler düşüldükten sonra gelirlerin tümü kar payı olarak değerlendirilmeli, sermaye, maharet ve iş gücü kollarının tümüne ağırlıklarına göre orantılı bir biçimde dağıtılmalıdır.
\vs p072 5:9 Her on yılda bir bölgesel yöneticiler, günlük kazanç sağlayan emeğin yassal olarak kaç saat olması gerektiğini düzenleyip bu konuda hüküm verirler. Üretim mevcut an içerisinde, haftada beş gün içinde çalışmanın dört eğlencenin bir gün olduğu düzende faaliyet göstermektedir. Bu insanlar her çalışma günü altı saat emeklerini sergileyip, öğrenciler gibi on ayın dokuzunda faaliyet gösterirler. Tatiller genellikle seyahatlerde harcanmakta olup, ulaşımın yeni yöntemleri oldukça yakın bir zaman içerisinde gelişme göstermiştir; ülkenin tamamı ulaşılabilir konumdadır. İklim yılda sekiz ay ulaşımı elverişli kılmakta olup, bu insanlar olanaklarının büyük bir kısmını değerlendirmektedirler.
\vs p072 5:10 İki yüz yıl önce kar gayesi üretimde tamamen baskın bir konumda bulunmaktaydı; ancak bugün kar amacının yerini hızlı bir biçimde daha yüksek diğer amaçlar almaktadır. Rekabet bu kıta üzerinde varlığını tüm gücüyle sürdürmektedir; ancak bu rekabetin büyük bir kısmı üretimden oyun, maharet, bilimsel başarı ve ussal kazanıma aktarılmıştır. Rekabet en etkin bir biçimde toplumsal hizmet ve hükümete olan bağlılıkta varlığını sergilemektedir. Bu insanlar arasında kamu hizmeti, geleceğe dair arzuların temel gayesi haline gelmektedir. Kıtanın en zengin bireyi sahip olduğu makine atölyesinde günde altı saat çalışmakta, ve bunun sonrasında kamu hizmeti için yetkin hale gelmeyi arzuladığı devlet adamı yetiştiren okulların yerel bir birimine yetişmeye çabalamaktadır.
\vs p072 5:11 Emek bu kıta üzerinde daha onursal bir hale gelmektedir; on sekiz yaşın üstünde olan yetkin bedenlere sahip vatandaşlar üretim yerleşkeleri olarak tanınan belirli bölgeler halindeki evlerde veya çiftliklerde çalışmaktadırlar; geçici olarak işsiz konuma düşen kişiler kamu işlerinde veya madenlerde zorunlu işçi birlikleri içinde emeklerini sarf etmektedirler.
\vs p072 5:12 Bu insanlar aynı zamanda --- tembelliğe ve emekle kazanılmamış servete yönelik --- toplumsal tiksintinin yeni bir türünü yeşertmeye başlamışlardır. Yavaş ama kesin bir biçimde onlar makineleri üzerinde hâkimiyet kurmaktadırlar. Onlar da bir zamanlar önce siyasi bağımsızlık ve daha sonra ekonomik özgürlük için mücadele vermişlerdi. Şimdi onlar; bu iki kazanımı da memnuniyetle deneyimleme safhasına girmekte olup, buna ek olarak artış gösteren bireyin kendisini gerçekleştirme faaliyetine adanabilecek oldukça hak edilmiş boş zaman etkinliklerini takdir etmeye başlamışlardır.
\usection{6.\bibnobreakspace Emeklilik Sigortası}
\vs p072 6:1 Bu ülke, bireyin kendisine olan saygısına zarar veren bağış türünü yaşlılıkta güvence teminatı sağlayan soylu hükümet sigortası ile değiştirmek için kararlı bir çaba sarf etmektedir. Bu ülke her çocuğa eğitim ve her insana bir iş olanağı sağlamaktadır; böylelikle bu işleyiş, çalışamaz durumda ve yaşlı olan bireylerin korunumu için bu türden bir sigorta düzenini başarıyla yürütebilmektedir.
\vs p072 6:2 Bu ülke insanları arasında bireylerin tümü; yetmiş yaşına kadar çalışmaya devam etmeleri için izin sağlayan devlet çalışma müdürlüğünden onay almadıkça, altmış beş yaşında kazançlı çalışmalarından emekli olmak zorundadır. Bu yaş sınırı, hükümet çalışanları veya felsefecilere uygulanmamaktadır. Fiziksel olarak engelli veya kalıcı olarak sakat hale gelmiş bireyler, bölgesel hükümetin emeklilik müdürlüğü ile ortak bir biçimde imzalanan mahkeme emri tarafından her yaşta emekliler listesine alınabilir.
\vs p072 6:3 Emeklilik maaşları için ödenekler şu dört kaynaktan elde edilmektedir:
\vs p072 6:4 1.\bibnobreakspace Her ay içerisinde bir günlük gelire federal hükümet bu amaçla el koymaktadır; ve bu ülkede herkes bir işte çalışır halde bulunmaktadır
\vs p072 6:5 2.\bibnobreakspace Miraslar --- birçok varlıklı vatandaş kaynaklarını bu amaç için ayırmaktadır.
\vs p072 6:6 3.\bibnobreakspace Devlet madenlerinde zorunlu çalışmadan elde edilen gelirler. Zorunlu olarak çalışmaya sevk edilen çalışanlar yaşamlarını idame ettirecekleri kadar gelir kazandıktan ve kendi emeklilik paylarını ayırdıktan sonra, iş güçlerinden elde edilen tüm ilave karlar bu emeklilik ödeneğine aktarılır.
\vs p072 6:7 4.\bibnobreakspace Doğal kaynaklardan elde edilen gelirler. Bu kıta üzerindeki doğal kaynakların tümü, federal hükümet tarafından toplum için kullanılan bir havuzda tutulmaktadır; ve bu üretim kolundan elde edilen gelirler hastalıkların önlenmesi, dâhilerin eğitimi ve devlet görevlilerini yetiştiren okullarda özellikle gelecek vaat eden bireylerin giderleri gibi toplumsal amaçlarda kullanılmaktadır. Doğal kaynaklardan elde edilen gelirlerin yarısı, emeklilik maaşları fonuna gitmektedir.
\vs p072 6:8 Her ne kadar eyalet ve bölge sigorta kurumları koruyucu sigortanın birçok türünü sağlasa da, emeklilik ödemeleri on bölge birimi vasıtasıyla yalnızca federal hükümet tarafından yönetilmektedir.
\vs p072 6:9 Bu hükümet kaynakları uzun bir süreden beri dürüst bir biçimde idare edilmektedir. Vatana ihanet ve cinayetten sonra en ağır cezalar mahkemeler tarafından, kamu güvenini kötüye kullanma suçuna verilmektedir. Toplumsal ve siyasi sadakatsizlik mevcut an içerisinde tüm suçlar arasında en çirkini olarak görülmektedir.
\usection{7.\bibnobreakspace Vergilendirme}
\vs p072 7:1 Federal hükümet yalnızca, emeklilik ödemelerinin uygulanmasına ek olarak dahi ve yaratıcı özgünlüğü destekleme alanında kesin hatları çizilmiş bir yönetim işleyişine sahiptir; eyalet hükümetleri biraz daha fazla bireysel düzeyde vatandaşlar ile ilgiliyken, yerel hükümetler çok daha fazla düzenleyici veya diğer bir değişle toplumcudur. Şehir (ve onun alt yönetimleri) kendi çalışmalarını; sağlık, temizlik, imar düzenlemeleri, güzelleştirme, su arzu, aydınlatma, ısıtma, boş zaman etkinliklerini sağlama, müzik ve iletişim gibi konularda odaklamaktadır.
\vs p072 7:2 Üretimin tümü içerisinde ilk öncelik sağlığa ayrılmaktadır; fiziksel esenlik düzeyinin belirli seviyeleri, üretimsel ve toplumsal öneme sahip olarak görülmektedir; ancak bireysel ve ailevi sağlık sorunları yalnızca kişiyi ilgilendiren durumlardır. Tamamiyle kişisel olan durumların tümünde olduğu gibi sağlık alanında müdahalede bulunmamak hükümetin gittikçe daha çok tercih ettiği tasarruf haline gelmektedir.
\vs p072 7:3 Şehirler hiçbir vergi gücüne sahip değillerdir; buna ek olarak onlar borçlanamamaktadırlar. Onlar; eyalet hazinesinden sahip olduğu her kişi başına ödenek almakta olup, bu ödeneği toplumcu girişimlerinden ve çeşitli ticari etkinliklere onay belgesi çıkararak elde ettiği gelirlerle desteklemek zorundadır.
\vs p072 7:4 Şehir sınırlarını genişletmeyi oldukça işlevsel kılan hızlı taşımacılık imkânları belediye denetimi altındadır. Şehrin yangın birimleri, yangın koruma ve sigorta kurumları tarafından desteklenmektedir; buna ek olarak şehir içinde ve dışında bulunan her bina, yetmiş beş yıldan fazla bir süredir, yanmaz hale getirilmiştir.
\vs p072 7:5 Buralarda belediye tarafından atanmış güvenlik görevlileri bulunmamaktadır; polis kuvvetleri, eyalet hükümetleri tarafından idare edilmektedir. Bu birimin tamamı, yirmi beş ila elli yaş arasında bulunan evlenmemiş bireylerden seçilmiştir. Eyaletlerin çoğu ağır bir bekâret vergisi uygulamaktadır; eyalet polis gücüne katılan bireylerin tümü bu vergiden muaf tutulmaktadır. Ortalama büyüklükteki bir eyalette polis kuvveti şu an içerisinde yalnızca, elli yıl öncesinin onda biri düzeyindedir.
\vs p072 7:6 Kıtanın farklı bölgeleri arasında ekonomik ve diğer şartların büyük ölçüde farklılık göstermesinden dolayı göreceli özgür ve egemen yüz eyaletin sahip olduğu vergi düzenleri arasında neredeyse hiçbir ortak payda bulunmamaktadır. Her eyalet, federal yüce mahkemenin rızası olmadan değiştirilemeyecek on temel anayasa hükmüne sahiptir; ve bu hükümlerin bir tanesi, bir yıllık bir süreç içerisinde herhangi bir taşınmazın bütüncül değerinin yüzde birinden fazlasının vergi olarak almasını önlemektedir; ister şehir içinde ister şehir dışında olsun ev yerleşkelerinin tümü vergiden muaftır.
\vs p072 7:7 Federal hükümet borçlanamamaktadır; ve herhangi bir eyaletin savaş amacı dışında diğer bir eyaletten borç alım talebi dörtte üç oranında halk oyunu gerektirmektedir. Federal hükümet borçlanamamakta olduğu için savaş durumunda Milli Savunma Heyeti, ihtiyaç duyulabilecek insan ve malzemelere ek olarak eyaletlerin para durumlarının bilançosunu çıkarma gücü ile donatılmıştır. Ancak hiçbir borç yirmi beş yıldan daha fazla bir süre boyunca ödenmemiş kalamaz.
\vs p072 7:8 Federal hükümeti desteklemek için gelirler şu beş kaynaktan elde edilmektedir:
\vs p072 7:9 1.\bibnobreakspace \bibemph{İthalat vergileri}. İthalatın tümü, bu kıta üzerinde ortak yaşam düzeyini korumak için tasarlanan bir gümrük vergi düzenine tabidir; bu ülkenin uyguladığı gümrükler gezegen üzerinde diğer herhangi bir ülkenin sahip olduğundan çok daha yüksektir. Bu gümrükler; üretim meclisinin iki kamarasının da, beraber atadıkları ekonomik olaylardan sorumlu baş yöneticinin tavsiyelerini onaylamasından sonra en yüksek üretim mahkemesi tarafından yürürlüğe girer. Üst üretim meclisi çalışanlar tarafından seçilirken, altta bulunan meclis ise sermaye sahipleri tarafından belirlenir.
\vs p072 7:10 2.\bibnobreakspace \bibemph{Telifler}. Federal hükümet; --- sanatçılar, yazarlar ve bilim adamları olarak --- dâhilerin tüm türlerine yardım ederek ve onların ebeveynlerini koruyarak on bölgesel yaratım merkezinde buluşları ve özgün yaratımları teşvik etmektedir. Bunun karşılığında hükümet; ister makineler, kitaplar, sanat ürünleri, bitkiler ve ister hayvanlar ile ilgili olsun, bu türden icatlar ve yaratımların tümünden elde edilen gelirlerin yarısını almaktadır.
\vs p072 7:11 3.\bibnobreakspace \bibemph{Miras vergisi}. Federal hükümet, bir taşınmazın büyüklüğüne ek olarak diğer koşullara bağlı bir biçimde yüzde bir ila elli arasında değişen derecelendirilmiş bir miras vergisini uygulamaktadır.
\vs p072 7:12 4.\bibnobreakspace \bibemph{Askeri malzeme}. Hükümet, ticari ve eğlencesel amaçlar için kullanılan kara ve deniz malzemelerinin kiralanmasından yüklü bir gelir elde etmektedir.
\vs p072 7:13 5.\bibnobreakspace \bibemph{Doğal kaynaklar}. Federal devlet yönetim sözleşmesi içinde belirlenen özel amaçlar için bütünüyle gerekmedikçe, doğal kaynaklardan elde edilen gelirler ülke hazinesine aktarılır.
\vs p072 7:14 Savaş kaynaklarının Milli Savunma Heyeti tarafından yaratılması dışında federal el koymalar; alt meclisin onayı ile birlikte üst yasama kurumunda çıkarılıp, devlet başkanı tarafından onaylanıp son olarak yüz üyeden oluşan federal bütçe heyeti tarafından nihai olarak geçerli kılınır. Bu heyetin üyeleri; eyalet valileri tarafından aday gösterilmekte olup, her altı yılda bir dörtte birinin yenilendiği bir biçimde yirmi dört yıllığına eyalet milletvekilleri tarafından seçilirler. Her altı yılda bir bu birim, dörtte üç oy çokluğu ile bir üyesini başkan olarak seçer; ve bu kişi böylelikle federal hazinenin yönetici\hyp{}denetleyicisi haline gelir.
\usection{8.\bibnobreakspace Özel Amaçlı Üniversiteler}
\vs p072 8:1 Beş\hyp{}on sekiz yaş dönemini kapsayan zorunlu temel eğitim düzenine ek olarak özel amaçlı okullar şu alanlarda faaliyetlerini gerçekleştirir:
\vs p072 8:2 1.\bibnobreakspace \bibemph{Devlet\hyp{}adamlığı okulları}. Bu okullar ülkesel, bölgesel ve eyaletsel düzeyde olmak üzere üç sınıftır. Ülkenin kamu kurumları dört birime ayrılmıştır. Kamu yararını gözeten bu oluşumun ilk birimi başlıca olarak ülke idaresi ile ilgilidir; ve bu topluluğun tüm mevki sahipleri, devlet adamı yetiştirme okullarının bölgesel ve ülkesel çaptaki eğitim birimlerinin mezunu olmak durumundadır. Bireyler, devlet\hyp{}adamlığının on bölgesel okulundan bir tanesinden mezun olduktan sonra ikinci birim içerisindeki siyasi, seçimle gelinen veya atanılan mevkileri kabul edebilirler; onların kamu görevleri, bölgesel idare ve eyalet hükümetlerindeki sorumluluklar ile ilgilidir. Üçüncü birim eyalet sorumluluklarını içine almakta olup, bu tür görevlilerin yalnızca eyalet düzeyindeki devlet\hyp{}adamlığı diplomasına sahip olmaları gerekmektedir. Devlet görevlilerin dördüncü ve son biriminin devlet\hyp{}adamlığı diplomasına sahip olmaları gerekmemektedir; bu mevkiler tamamen atama yoluyla belirlenir. Onlar, hükümetsel idari alanlarda faaliyet halindeki çeşitli eğitimsel meslekler tarafından belirlenen yardımcılık, sekreterlik veya teknik görevlerin alt kademelerini temsil eder.
\vs p072 8:3 Yerel ve eyalet mahkeme yargıçları, eyalet devlet\hyp{}adamlığı okullarından alınan diplomalara sahiplerdir. Toplumsal, eğitimsel ve üretimsel alanlardaki yargı mahkemelerinin hâkimleri, bölgesel okullardan alınan diplomalara sahiplerdir. Federal yüce mahkeme yargıçları, devlet adamı yetiştiren okulların tümünü bitirmek zorundadır.
\vs p072 8:4 2.\bibnobreakspace \bibemph{Felsefe okulları}. Bu okullar felsefe mabetleri ile ilişkili olup, bir kamu faaliyeti olarak din ile az çok birliktelik halindedir.
\vs p072 8:5 3.\bibnobreakspace \bibemph{Bilim enstitüleri}. Bu teknik okullar mevcut eğitim düzeni yerine sanayi ile eş güdüm halinde olup, on beş alt birimde idare edilir.
\vs p072 8:6 4.\bibnobreakspace \bibemph{Mesleksel eğitim okulları}. Bu özel kurumlar, sayıca on iki olan çeşitli eğitimsel meslekler için tekniksel hazırlanışı sağlamaktadır.
\vs p072 8:7 5.\bibemph{ Kara ve deniz askeri okulları}. Ülke yönetim merkezi yakınında ve yirmi beş kıyı askeri merkezde, on sekiz yaş ila otuz yaş arasında bulunan gönüllü vatandaşların askeri eğitimine ayrılan bu kurumlar idare edilmektedir. Yirmi beş yaşın altında bulunan bireylerin bu okullara kabul edilebilmesi için ailelerinin onayını almış olması gerekmektedir.
\usection{9.\bibnobreakspace Genel Oy Hakkı Tasarımı}
\vs p072 9:1 Her ne kadar kamu görev adaylarının tümü eyalet, bölgesel veya devlet\hyp{}adamlığı okullarının mezunlarından seçilen bir biçimde sınırlandırılmış olsa da, bu ülkenin ilerici önderleri genel oy tasarımlarında ciddi bir zaaf noktası keşfettiler; ve yaklaşık elli yıl önce şu niteliklerden oluşan değiştirilmiş bir oy verme düzeni için anayasal hükmü yürürlüğe koydular:
\vs p072 9:2 1.\bibnobreakspace Yirmi yaş ve üstünde olan her erkek ve kadın bir oya sahiptir. Bu yaşa erişilmesiyle birlikte vatandaşların tümü iki oy verme topluluğundan birine olan üyeliği kabul etmek durumundalardır. Onlar üretimsel, mesleksel, tarımsal veya ticaretsel olarak ekonomik faaliyetleri uyarınca ya ilk topluluğa katılacak, ya da siyasi, felsefi ve toplumsal eğilimleri uyarınca ikinci topluluğun üyesi olacaklardır. Çalışanların tümü böylelikle birtakım ekonomik oy topluluğuna ait olup, bu localar, ekonomik olmayan birliktelikler gibi, milli hükümetin gücün üç katmanlı erki tarafından denetlenmesine oldukça benzer bir biçimde, idare edilmektedir. Bu topluluklara gerçekleştirilen üyelik kayıtları on iki yılda bir değiştirilebilir.
\vs p072 9:3 2.\bibnobreakspace Eyalet valileri veya bölgesel yöneticilerin aday göstermesiyle ve bölgesel yüce heyetlerin emri ile, toplum için büyük hizmetlerde bulunma görevi emanet edilmiş bireylere veya hükümet hizmeti içinde olağanüstü bilgeliği sergilemiş olanlara; her seferinde beş yıllık bir süreden daha fazla olmayacak ve dokuz üstün oyu geçmeyecek bir biçimde ek oy kullanma hakkı verilebilir. Aynı şekilde bilim adamlarına, mucitlere, öğretmenlere, felsefecilere ve ruhsal önderlere de bu ilave siyasi güç verilebilmekte ve onlar bu güçle onurlandırılabilmektedir. Bu gelişmiş vatandaşsal ayrıcalıklar, özel amaçlı okulların bitirme belgeleri vermelerine oldukça benzer bir biçimde, eyalet ve bölgesel yüce heyetler tarafından sağlanmaktadır; ve bu ayrıcalıklara layık görülen bireyler, bu türden kamusal tanınmanın nişanını, sahip olduğu diğer başarı belgeleri ile birlikte, kişisel kazanımlarının listesine eklemekten onur duymaktadırlar.
\vs p072 9:4 3.\bibnobreakspace Madenlerde zorunlu işçi olarak cezalandırılan tüm bireylere ek olarak vergi gelirleri ile hayatlarını idame ettiren hükümet çalışanları bu hizmetleri süresince oy kullanamamaktadırlar. Bu hüküm, altmış beş yaşında emekliliğe hak kazanıp görevlerini bırakan kişiler için geçerli değildir.
\vs p072 9:5 4.\bibnobreakspace Her beş yıllık süreç içerisinde ödenen yıllık ortalama vergileri yansıtan beş ilave oy kullanma dilimi bulunmaktadır. Yüklü miktarda vergi veren bireylerin beşe kadar ilave oy kullanmasına izin verilmektedir. Bu izin diğer idari onurların dışındadır; ancak hiçbir durumda bir kişi ondan fazla oy kullanamaz.
\vs p072 9:6 5.\bibnobreakspace Bu oy kullanma tasarımının yürürlüğe girdiği dönemde, yerleşkesel oy verme sistemi ekonomik veya diğer bir değişle işlevsel sistemle değiştirilmiştir. Şimdi tüm vatandaşlar, ikametlerinden bağımsız bir biçimde üretimsel, toplumsal veya mesleksel topluluk üyeleri olarak oy kullanmaktadırlar. Böylelikle seçmenler, hükümetsel görev ve sorumluluk mevkilerine gelebilecek sadece en iyi üye arkadaşlarını seçen birleşmiş, bütünleşmiş ve ussal topluluklardan meydana gelmektedir. Bu işlevsel veya diğer bir değişle topluluk oy hakkı düzeni içinde bir istisna bulunmaktadır: bir federal devlet başkanının her altı yılda bir gerçekleştirilen seçimi ülke çapında olup, hiçbir vatandaş birden fazla oy verememektedir.
\vs p072 9:7 Böylece devlet başkanının seçimi dışında oy verme hakkı, vatandaşlığın ekonomik, mesleksel, ussal ve toplumsal birliktelikleri tarafından yerine getirilir. Nihai devlet organik olup, vatandaşların özgür ve ussal her bir topluluğu daha büyük hükümet organizması içinde hayati ve faal bir organı temsil eder.
\vs p072 9:8 Devlet\hyp{}adamlığı okulları; yetersiz, tembel, umursamaz veya suçlu herhangi bir bireyin oy hakkının alınması için eyalet mahkemelerinde dava açabilme gücüne sahiptir. Bu insanlar, bir ülkenin yarısının alt düzeyde veya diğer bir değişle yetersiz olup hala oy kullanma hakkına sahip olduğunda, o ülkenin çöküşünün kaçınılmaz olduğunu bilmektedirler. Onlar, sıradanlığın hâkimliğinin herhangi bir ülkenin sonunu hazırladığına inanmaktadırlar. Oy vermek zorunlu olup, oylarını kullanmada başarısız olan herkese yüklü cezalar uygulanmaktadır.
\usection{10.\bibnobreakspace Suçla Mücadele}
\vs p072 10:1 Bu insanların suçla, akıl hastalarıyla ve yozlaşma ile mücadeleleri, her ne kadar bir biçimde memnuniyet verici olsa da, kuşkusuz birçok Urantia unsurunu dehşete düşürecektir. Sıradan suçlular ve yetersizler cinsiyetlerine göre farklı tarım topluluklarına yerleştirilmekte olup, bu yerler kendi yaşamlarını idame ettirme olanağından fazlasını sağlamaktadır. Alışkanlıkla suç işleyen daha ciddi suçlular ve iyileştirilemez bir biçimde akıl hastası olan kişiler, mahkemeler tarafından ölümcül gaz odasına gönderilme cezasına çarptırılırlar. Aynı zamanda cinayetin yanı sıra sayısız birçok suç, hükümetsel göreve olan hıyanet dâhil olmak üzere, ölüm cezası taşımakta olup; adalet kesin ve hızlı bir biçimde yerine getirilir.
\vs p072 10:2 Bu insanlar kısıtlayıcı ve cezalandırıcı hukuktan önleyici hukuka doğru giriş yapmaktadır. Kısa bir süre önce onlar, olası katiller ve büyük suçlular olduklarına inanılanları gözaltı topluluklarında yaşam boyunca mahkûm ederek suçun önlenmesi girişiminde bulunmaya kadar ilerlediler. Eğer bu türden mahkûmlar ilerleyen zamanlarda biraz daha olağan hale geldiklerini gösterirlerse, ya şartla salıverilirler veya affedilirler. Bu kıta üzerinde cinayet oranı, diğer ülkelerdekinin yalnızca yüzde biridir.
\vs p072 10:3 Suçlular ve yetersizleri yaratan koşulları ortadan kaldırma çabaları yüz yıl önce başlamış olup, çoktandır memnuniyet verici sonuçları beraberinde getirmiştir. Akıl hastaları için orada herhangi bir cezaevi veya hastane bulunmamaktadır. Bunun nedenlerinden biri Urantia üzerinde bulunabilecek bu kişilerin yalnızca yaklaşık yüzde onunun bu ülkede mevcut olmasıdır.
\usection{11.\bibnobreakspace Askeri Hazırlılık}
\vs p072 11:1 Federal askeri okul mezunları, Milli Savunma Heyeti’nin başkanı tarafından kabiliyet ve deneyimleri uyarınca yedi rütbe içinde “medeniyetin koruyucuları” olarak görevlendirilebilirler. Bu heyet yirmi beş üyeden meydana gelmekte olup, onların üyeleri en yüksek aile, eğitim ve üretim mahkemeleri tarafından aday gösterilip, federal yüce mahkeme tarafından onaylanır; ve bu heyete resen, idare edilen askeri olaylardan sorumlu görevlilerin başı başkanlık eder.
\vs p072 11:2 Bu türden görevlendirilmiş çalışanlar tarafından takip edilen dersler dört yıl sürmekte olup, her zaman belirli bir zanaatın veya mesleğin ustalığı ile ilgili haldedir. Askeri eğitim, bahse konu birliktelik halindeki üretimsel, bilimsel veya mesleksel hazırlanma olmadan sunulmamaktadır. Askeri eğitim tamamlandığında dört yıllık eğitim süreci boyunca birey, benzer bir biçimde eğitimin dört yıl olduğu özel amaçlı okulların herhangi birinde aktarılan eğitimi askeri okullardaki öğreniminin yarısında alır. Eğitimlerinin yarısı boyunca teknik veya mesleksel bir hazırlanmanın teminat altına alınmasıyla çok sayıdaki insana yaşamlarını idame ettirebilme olanağı sunarak, mesleksel bir askeri sınıfın yaratılmasından böylece kaçınılmış olur.
\vs p072 11:3 Barış dönemlerinde askeri hizmet tamamen gönüllü bir biçimde gerçekleştirilmektedir; ve bu hizmetin tüm kollarındaki görevlendirilmeler dört yıllıktır; bu dönemde her birey, askeri taktiklerdeki ustalığa ek olarak belirli bir özel çalışma kolunda ilerler. Müzik alanındaki eğitim, merkezi askeri okullarının başlıca uğraşlarından biridir; ve bu okulların yirmi beş eğitim yerleşkesi kıtanın sınır bölgeleri etrafına dağıtılmıştır. Üretimsel durgunluk dönemlerinde binlerce işsiz kendiliğinden; kara, deniz ve havada kıtanın askeri savunma kaynaklarını inşa etmede görevlendirilmektedir.
\vs p072 11:4 Her ne kadar bu insanlar; çevreleyen düşmancıl toplulukların saldırılarına karşı bir savunma biçimi olarak güçlü bir savaş kabiliyetini devam ettirseler de; bu askeri kaynakları saldırgan bir savaşta yüz yıldan fazla bir süreden beri kullanmamış olmaları onların bu uygulamalarındaki başarı hanesine yazılabilir. Onlar öyle bir medeniyet seviyesine ulaşmışlardır ki, saldırganlık içinde savaş güçlerini kullanma cezp ediciliğine kapılmaya fırsat vermeden medeniyetlerini oldukça kuvvetli bir biçimde savunabilmektedirler. Birleşik kıta devletinin oluşturulmasından beri orada hiçbir iç savaş yaşanmamıştır; ancak geçmiş iki yüzyıl boyunca bu insanlar, savunma amaçlı dokuz çetin savaşa girmek için çağrılmışlardır; bu dokuz savaştan üçü, dünya güçlerinin çok güçlü ittifaklarına karşı gerçekleştirilmiştir. Her ne kadar bu ülke; düşmancıl komşularının saldırılarına karşı yeterli düzeyde bir savunmayı sürdürse de, çok daha büyük bir ilgiyi devlet çalışanları, bilim adamları ve felsefecilerin eğitimine ayırmaktadır.
\vs p072 11:5 Ülkede barış egemen olduğunda etkin savunma düzen unsurlarının tümü oldukça bütüncül bir biçimde alışveriş, ticaret ve boş zaman etkinliklerinde kullanılmaktadır. Savaş ilan edildiğinde tüm ülke etkin hale getirilmektedir. Düşmanlık dönemi boyunca ordu tüm sanayi kollarını iyeliği altına almakta olup, askeri birimlerin tümünün baş sorumluları devlet başkanının bakanlar kurulunun üyeleri haline gelmektedir.
\usection{12.\bibnobreakspace Diğer Ülkeler}
\vs p072 12:1 Her ne kadar bu benzersiz insanların toplumu ve hükümeti Urantia milletlerinden birçok açıdan üstün olsalar da, diğer kıtalardaki (bu gezegen on bir kıtaya sahiptir) hükümetler Urantia’nın daha gelişmiş milletlerine kıyasla kesin bir biçimde alt düzeyde bulunmaktadır.
\vs p072 12:2 Tam da bu mevcut zaman içerisinde bahse konu üstün hükümet, alt düzeyde bulunan insanlar ile elçiliksel düzeyde temsil ilişkileri kurmayı tasarlamaktadır; ve ilk kez, bu komşu milletlere dini elçiler göndermeyi savunan bir büyük dini önder çıkmıştır. Bizler bu insanların, üstün bir kültürü ve dini diğer ırklar üzerine dayatmaya çabaladıklarında birçoklarının yaptıkları yanlışın aynısını gerçekleştirecek olmalarından korku duymaktayız. Gelişmiş kültürün bu kıtasal ülkesi sadece sınırları dışına çıksa ve komşu topluluklar arasındaki en iyi bireyleri kendi ülkelerine getirseydi, onları eğittikten sonra kültür elçileri olarak karanlıkta kalmış kardeşlerine geri gönderseydi, bu ülke içerisinde ne de mükemmel bir şey gerçekleştirilmiş olurdu! Tabi ki, eğer bir Hakimane Evlat yakın bir zaman içerisinde bu gelişmiş ülkeye gelirse, büyük gelişmeler hızlı bir biçimde bu dünya üzerinde gerçekleşebilir.
\vs p072 12:3 Komşu bir gezegenin olaylarına dair bu anlatım, Urantia üzerinde medeniyeti geliştirme ve hükümetsel evrimi ilerletme amacıyla özel izin tarafından aktarılmıştır. Urantia unsurlarının kuşkusuz ilgisini çekecek ve onları olumlu bir biçimde etkileyecek daha birçok şey anlatılabilirdi, ancak bu anlatımın bütünü sahip olduğumuz izin hükmünün sınırlarını oluşturmaktadır.
\vs p072 12:4 Urantia unsurları, buna rağmen; Satania ailesi içindeki kardeş âlemlerinin, Cennet Evlatları’nın ne hakimane ne de bahşedilme görevlerinden şimdiye kadar yarar sağlamamış olmalarını dikkatle irdelemelidir. Buna ek olarak diğer bir açıdan, bahse konu kıtasal ülkeyi gezegensel akranlarından ayıran bu derece bir kültür uçurumu tarafından Urantia’nın çeşitli toplulukları birbirinden ayrışmamışlardır.
\vs p072 12:5 Gerçeğin Ruhaniyeti’nin bu aktarımı, bahşedilmiş dünyanın insan ırkının sahip oldukları amaçlarda büyük kazanımların gerçekleşmesi için ruhsal bir alt yapıyı sağlamaktadır. Urantia bu anlatımın neticesinde; --- kanunlarla dünya çapındaki barışın çok güçlü bir biçimde oluşturulmasına katkıda bulunabilecek ve, ışık ve yaşamın en nihai çağları için gezegensel eşik dönemi olan bir çağ biçiminde, ruhsal arayışın gerçek bir çağının günün birinde doğuşuyla sonuçlanabilecek --- yasalarıyla, işleyiş düzenleriyle, simgeleriyle, ortak kabulleriyle ve diliyle gezegensel bir hükümetin daha yakın bir zamanda gerçekleşmesi için çok daha iyi bir biçimde hazırlanmış hale getirilmiştir.
\vs p072 12:6 [Nebadon’un bir Melçizedek unsuru tarafından sunulmuştur.]
