\upaper{68}{Medeniyetin Doğuşu}
\vs p068 0:1 Bu anlatım; bir hayvan mevcudiyetinden biraz daha iyi düzeyde bulunan bir seviyeden çağlar boyunca, her ne kadar kusursuz olmasa da gerçek bir medeniyetin insan varlıklarının yüksek ırkları arasında evrimleştiği daha sonraki çağlara kadar ki insan türlerinin çok uzun yıllar süren ilerleyiş mücadelelerine ait serüvenin başlangıcıdır.
\vs p068 0:2 Medeniyet ırksal bir kazanımdır; bu kazanım, içkin bir biçimde biyolojik olarak mevcut bulunmamaktadır; birbirini takip eden her genç nesil eğitimini yeni baştan almak zorundayken tüm çocuklar bu nedenle bir kültür çevresi içinde yetiştirilmekle yükümlüdür. Bilimsel, felsefi ve dinsel olarak medeniyetin üstün nitelikleri bir nesilden diğerine doğrudan kalıtım biçiminde aktarılmamaktadır. Bu kültürel kazanımlar yalnızca, toplumsal mirasın aydınlanmış muhafazası tarafından korunmaktadır.
\vs p068 0:3 İşbirliksel düzenin toplumsal evrimi; Dalamatia eğitmenleri tarafından başlatılmış olup, üç yüz bin yıl boyunca insanlık toplumsal etkinliklerin düşüncesi içerisinde yetiştirilmiştir. Mavi insan, bu öncül toplumsal öğretilerden en fazla yararlanan ırk olmuştur; kırmızı insan bu öğretilerden bir ölçüde yarar sağlamış ve siyah ırk diğer ırkların arasında onlardan en az faydalanan topluluk olmuştur. Daha yakın zamanlar içerisinde sarı ve beyaz ırk, Urantia üzerinde en gelişmiş toplumsal ilerlemeyi sergilemiştir.
\usection{1.\bibnobreakspace Koruyucu Toplumsallaşma}
\vs p068 1:1 Yakın bir biçimde bir araya getirildiğinde insanlar sıklıkla birbirlerini sevmeyi öğrenirler; ancak ilkel insanlar doğal bir biçimde, kardeşsel duygunun ruhaniyeti ve akranlarıyla gerçekleşecek toplumsal iletişimin arzusu ile dolup taşmamaktaydı. Bunun yerine öncül ırklar, “birlikten kuvvet doğar” deneyimini kötü olaylar neticesinde öğrenmişlerdi; ve doğal kardeşsel etkileşimin bu eksikliği, Urantia üzerinde kardeşliğin doğrudan gerçekleşme biçimini engellemektedir.
\vs p068 1:2 Birliktelik ilk başta hayatta kalma mücadelesi açığa çıkmıştır. Yalnız insan; kendisine yapılacak herhangi bir saldırının intikamını kesin bir biçimde alabilecek bir topluluğa üye olduğunu gösteren bir kabile simgesini taşımadığı durumlarda yardıma muhtaç bir haldeydi. Kabil’in döneminde bile, topluluk birlikteliğine ait herhangi bir işareti taşımadan dışarı çıkmak ölümcüldü. Medeniyet şiddet sonucunda gerçekleşen ölüme karşı insanın teminatı haline gelirken, bu teminatın bedeli toplumun sayısız kanun beklentilerine olan bağlılık tarafından ödenmektedir.
\vs p068 1:3 İlkel toplum böylelikle, karşılıklı ihtiyaç ve birlikteliğin gelişmiş güvenliği üzerine inşa edilmiştir. Ve insan toplumu, bu yalnız kalma korkusunun bir sonucu olarak ve gönülsüz işbirliğinin araçları vasıtasıyla yüz yıllar süren dönüşümler içerisinde evrimleşmiştir.
\vs p068 1:4 İlkel insan varlıkları toplulukların, bireylerin tekil bütünlüğünden çok büyük bir biçimde güçlü ve yetkin olduğunu öncül olarak öğrenmişti. Bir araya gelen ve birliktelik içinde çalışan yüz kadar insan, büyük bir kaya parçasını hareket ettirebilir; barışın oldukça iyi eğitilmiş dikkate değer sayıdaki koruyucusu, kızgın bir topluluğu bastırabilir. Bu bağlamda toplum, belirli sayıdaki unsurun yalın bir birlikteliği tarafından değil ussal işbirlikçilere ait \bibemph{düzenin} bir sonucu olarak böylece doğmuştur. Ancak işbirliği insanın doğal bir niteliği değildir; insan ilk başta korku nedeniyle işbirliğinde bulunmayı öğrenmekte olup, daha sonra zamanın zorlukları ile başa çıkmada ve ebediyetin varsayılan tehlikelerine karşı korunmada en yararlı olan deneyim şeklinde işbirliğini keşfetmesi nedeniyle topluluk birlikteliklerini yerine getirmektedir.
\vs p068 1:5 İlkel bir toplum biçiminde kendilerini öncül olarak bu biçimde düzenlemiş insan toplulukları, doğada gerçekleştirdikleri saldırılara ek olarak akranları karşısında sergiledikleri savunmalarda daha başarılı olmuşlardır; onlar daha büyük kurtuluş olanaklarına sahip oldular; böylelikle medeniyet, birçok zararlı yan etkilerine rağmen, Urantia üzerinde doğrusal bir biçimde ilerlemeye devam etmiştir. Ve yalnızca birliktelik içerisinde bulunan kurtuluş değerinin gelişmesi sonucunda insanın birçok yanlışı insan medeniyetinin gelişmesini durdurmada veya bu medeniyeti yok etmede bu nedenle açık bir biçimde başarısız olmuştur.
\vs p068 1:6 Çağdaş kültürel toplumun tarihsel alışılmışlığın dışındaki yakın bir zaman olgusu niteliğine sahip oluşu, Avustralya yerlilerine ek olarak Afrikalı Buşman ve Pigme insanlarını belirleyen bu tür ilkel toplumsal şartların varlığını bugün bile devam ettirişi tarafından oldukça iyi bir biçimde sergilenmiştir. Bu geride kalmış insanlar arasında, ilkel ırkların tümünün çok belirgin ortak nitelikleri olan öncül topluluk düşmanlığına, kişisel şüpheye ve oldukça yüksek düzeyde bulunan toplumsal düzene aykırı özelliklere benzer gözlemler deneyimlenebilir. Tarihi dönemlere ait toplumsal olmayan insanların bu kötü şartlarda yaşayan kalıntıları; insanın doğal niteliğe sahip bireyci eğiliminin, toplumsal ilerleyişin daha yetkin ve güçlü düzenlenişleri ve birliktelikleri ile başarılı bir biçimde yarışamayacağına dair etkili tanıklığı taşımaktadır. Her kırk veya elli milden sonra farklı bir lehçe konuşan bu geri kalmış ve kuşku içinde yaşayan toplumsal düzene aykırı ırklar; Gezegensel Prens’in bedensel görevlilerinin bütünleşmiş öğretilerine ek olarak ırksal canlandırıcılara ait Âdemsel topluluğun daha sonraki çabaları olmadan şu an nasıl bir dünya içinde yaşayabilecek oluşunuzu göstermektedir.
\vs p068 1:7 Çağdaş söylem olan “doğaya dönüş”, bir zamanlar yaşanmış olan hayali “altın çağın” gerçekliğine ait inanç biçiminde cahilliğin bir yanılgısıdır. Altın çağ efsanesine dayanak teşkil eden tek temel kaynak, Dalamatia ve Cennet Bahçesi’nin tarihsel gerçekliğidir. Ancak bu gelişen toplumlar, hayalperest düşlerin gerçekleşmesinden oldukça uzak bir düzeyde bulunmaktaydı.
\usection{2.\bibnobreakspace Toplumsal İlerlemedeki Etkenler}
\vs p068 2:1 Medeni toplum, \bibemph{tecrit} edilmekten hoşlanmayan insanın bu hoşnutsuzluğunun üzerinden gelmesine dair öncül çabalarının sonucunda meydana gelmiştir. Ancak bu genel çaba doğrudan bir biçimde karşılıklı sevgi anlamına gelmemekte olup, belirli ilkel toplulukların bugünkü mevcut çalkantılı düzeyi ilkel kabilelerin hangi aşamalardan geçtiğini oldukça iyi bir biçimde sergilemektedir. Ancak her ne kadar bir medeniyetin bireyleri birbirleri ile ters düşebilse ve aralarında mücadeleler verebilse de, ve medeniyetin kendisi arayış ve mücadelenin tutarsız bir kitlesi olarak görünebilse de, medeniyet ciddi bir uğraşı sergilemekte olup ölü bir tekdüzeliğe sahip duranlık anlamına gelmemektedir.
\vs p068 2:2 Us seviyesi kültürel ilerleyiş hızına ciddi bir biçimde katkıda bulunurken toplum özü itibariyle bireyin yaşam biçimi içindeki tehlike etkenini düşürmek için tasarlanmıştır; ve toplum, yaşam içerisinde acı etkenini azaltıp sevinci arttırmada başarılı oldukça aynı hızda ilerleme göstermiştir. Böylelikle toplumsal birliğin bütünlüğü, --- yok olma veya kurtuluşa erişme biçiminde --- amacının birey\hyp{}korunumu veya birey\hyp{}tatmini olduğu koşullara bağlı olarak nihai sonunda dair gayeye doğru yavaşça ilerler. Birey\hyp{}korunumu toplumun oluşmasına kaynaklık ederken, aşırı birey\hyp{}tatmini medeniyeti yok eder.
\vs p068 2:3 Toplum; birey\hyp{}devamlılığı, birey\hyp{}korunumu ve birey\hyp{}tatmini ile ilgilidir; ancak insana ait olan bireyin\hyp{}kendisini\hyp{}gerçekleştirme niteliği, birçok kültürel topluluğun doğrudan amacı haline gelmeye layıktır.
\vs p068 2:4 İnsan doğasında var olan sürü içgüdüsü, Urantia üzerinde şu an içerisinde mevcut olan türden bir toplumsal düzenin gelişmesini açıklamaya yeterli değildir. Her ne kadar bu içkin olan toplumsal eğilim insan toplumunun kökenine kaynaklık etse de, insanın toplumsal hale gelme kabiliyetinin büyük bir kısmı kendisine ait çabalarla elde ettiği bir kazanımdır. İnsan varlıklarının öncül birlikteliğine katkıda bulunan iki büyük etki yiyecek açlığı ve cinsel ilişki arzusuydu; bu içgüdüsel dürtüleri insan türü hayvan dünyası ile paylaşmaktadır. İnsan varlıklarını bir araya getiren ve onları bir birliktelik halinde \bibemph{tutan} iki diğer duygu, gösteriş ve, özellikle hayalet korkusu biçimindeki, genel korkuydu.
\vs p068 2:5 Tarih başlı başına insanın çağlar boyu süren yiyecek mücadelesinin serüvenidir. \bibemph{İlkel insan yalnızca aç olduğu zaman düşünmeye başlamıştır}; yiyecekler üzerinden tasarrufa gitme, insanın ilk olarak gerçekleştirdiği öz\hyp{}denetim biçimindeki özverisiydi. Çeşitli ihtiyaçların yerine getirilmesi biçimindeki açlığın sayısız farklı türünün hepsi, insan türünün yakın birlikteliğine sebebiyet vermiştir. Ancak bugün toplum, varsayılan insan ihtiyaçlarının gereğinden fazla artışı ile birlikte aşırı bir biçimde yüklü hale gelmiştir. Yirminci yüzyılın batı medeniyeti, şatafatın çok büyük bir fazlalığına ek olarak insan ihtiyaçları ve arzularının düzensiz çoğalımı altında bitkin bir halde inlemektedir. Çağdaş toplum, çok geniş çaplı karşılıklı birlikteliğin ve oldukça karmaşık düzeydeki karşılıklı bağımlılığının en tehlikeli fazlarının bir tanesine ait doğrultuda zorlanmaktadır.
\vs p068 2:6 Açlık, gösteriş ve hayalet korkusu onların toplumsal baskısı içinde sürekli bir biçimde etkindi; ancak cinsel tatmin geçici ve aralıklı dönemlerde kendisini göstermekteydi. Cinsel dürtü tek başına, ilkel erkek ve kadınların ev idaresinin ağır yüklerini yüklenmelerine kaynaklık etmemekteydi. Öncül ev idaresi, düzenli bir biçimde gerçekleşen cinsel tatminin yoksunluğunda erkeğin cinsel huzursuzluğu ve kadının yüksek hayvanların dişileriyle bir ölçüde paylaştığı sadık anne sevgisi üzerine inşa edilmiştir. Yardıma muhtaç bir bebeğin mevcudiyeti, erkek ve kadın etkinliklerinin öncül farklılaşması tarafından belirlenmiştir; kadın, toprağı ekebileceği yer olan yerleşik bir konumu idare etmekle yükümlüydü. Ve ilkel zamanlardan beri kadının ikamet ettiği yer her zaman evin ta kendisi olarak görülmektedir.
\vs p068 2:7 Böylelikle kadın öncül olarak, evrimleşen toplumsal düzen için hayati bir değere sahip olmuştur; bu hayati değer, \bibemph{yiyecek gereksiniminin} bir sonucu oluşuna kıyasla aralıklarla gelip geçen cinsel arzudan kaynağını baskın bir biçimde almamıştır; kadın, birey\hyp{}korunumu bakımından hayati derecede öneme sahip yardımcı halindeydi. Kadın; bir yiyecek sağlayıcı, zorlukların üstesinden gelen güçlü bir canlı ve şiddete başvurmadan büyük kötülüklere karşı mücadele verebilecek bir dosttu; bütün bu arzulanan niteliklerine ek olarak kadın, cinsel tatmini elde etmenin her zaman erişilebilir bir aracıydı.
\vs p068 2:8 Medeniyet içinde değerini uzun yıllar koruyan neredeyse her şey, kökeni aile kurumundan almaktadır. Aile; erkek ve kadının sahip olduğu çatışmasal farklılıkları uyumlu hale getirmeyi öğrenirlerken aynı süreçte çocuklarına temel barış gayelerini öğrettikleri bütünlük olarak, ilk başarılı barış topluluğudur.
\vs p068 2:9 Evrim içerisinde evliliğin işlevi, ırk kurtuluşun teminat altına almaktır; evlilik kurumunun işlevi yalnızca kişisel mutluluğun gerçekleştirilmesi değildir; birey\hyp{}korunumu ve birey\hyp{}devamlılığı, ev oluşumunun gerçek amaçlarıdır. Birey\hyp{}tatmini ikincil düzeyde kendisini gerçekleştirirken, cinsel birleşmeye zemin hazırlayan bir teşviki oluşturması dışında birincil düzeyde öneme sahip değildir. Doğa kurtuluşu zorunlu kılmaktadır; ancak medeniyetin sanatları, evliliğe ait olan hazları ve aile yaşamının tatminlerini arttırmaya devam etmektedir.
\vs p068 2:10 Eğer gösteriş gururu, geleceğe dair arzuyu ve onuru kapsayacak kadar büyük bir hale gelirse, bizler; yalnızca bu eğilimlerin insan birlikteliklerinin oluşmasına nasıl katkıda bulunduklarını değil aynı zamanda onların insanları nasıl bir arada tuttuklarını da algılayabiliriz; çünkü bu türden duygular, kendilerini önünde göstermek için bir izleyici kitlesi olmadan içi boş hislerdir. Yakın zaman içerisinde gösteriş, kendilerini sergileyebilecekleri ve tatmin edebilecekleri yer olan bir toplumsal alana ihtiyaç duyan diğer duygular ve dürtüler ile birlikte bütünsel hale gelmişti. Duyguların bu topluluğu; sanatlar, törenler, spor oyunları ve yarışmalarının tüm biçimlerinin ilkel başlangıçlarına kaynaklık etmiştir.
\vs p068 2:11 Gösteriş, toplumun doğuşuna çok büyük bir ölçüde katkı sağlamıştır; ancak bu açığa çıkarışların gerçekleştiği zamanlarda aşırı derecede gösterişe sahip bir neslin hilekâr amaçları, oldukça yüksek bir biçimde özelleşmiş bir medeniyetin bütüncül karmaşık düzenini batırmakla ve onu tarih sahnesinden silmekle yüz yüze bırakmıştır. Hazzın tatmini uzun bir süreden beri açlığın giderilişinin yerini almıştır; birey\hyp{}korunumunun yerinde toplumsal amaçları hızlı bir biçimde birey tatmininin bayağı ve tehlikeli türlerine dönüşmüştür. Birey\hyp{}korunumu toplumu inşa etmektedir; kısıtlanmamış birey\hyp{}tatmini medeniyet oluşumunu her zaman yok etmektedir.
\usection{3.\bibnobreakspace Hayalet Korkusunun Toplumlaştırıcı Etkisi}
\vs p068 3:1 İlkel arzular, özgün toplumu yaratmıştı; ancak hayalet korkusu, bu toplumu bir arada tutmuş ve onun mevcudiyetine insanüstü bir nitelik kazandırmıştı. Ortak korku; fiziksel acıdan, giderilemeyecek açlıktan ve birtakım dünyevi afetlerden duyulan korku biçiminde köken itibariyle psikolojikti; ancak hayalet korkusu yeni ve yüce türden bir dehşetti.
\vs p068 3:2 İnsan toplumu içerisinde belki de en büyük temel etken, hayalet düşüydü. Her ne kadar rüyaların çoğu ilkel aklı büyük ölçüde tedirgin ettiyse de, hayalet rüyası gerçekte, ruhani dünyanın belirsiz ve görülmeyen hayali tehlikelerine karşı karşılıklı korunma için arzulu ve samimi birliktelik içerisinde bu hurafelere sahip olan hayalperestleri birbirlerinin kollarına iterek öncül insanları gerçekte dehşete düşürmüştür. Hayalet rüyası, aklın hayvan ve insan türleri arasında ortaya çıkan farklılıkların ilkinden biriydi. Hayvanlar, ölümden sonraki varlığı devam ettirişi tahayyül etmemektedirler.
\vs p068 3:3 Bu hayalet etkeni dışında toplumun bütünlüğü, temel ihtiyaçlar ve ana biyolojik dürtüler üzerine inşa edilmişti. Ancak hayalet korkusu; bireyin temel ihtiyaçlarından dışarı doğru uzanan ve topluluğu idame etme çabalarının bile oldukça ötesine geçen bir korku biçiminde, medeniyetin oluşumuna yeni bir etkeni beraberinde getirmiştir. Ölmüş varlığa ait ayrılmış ruhaniyetin korkusu; öncül çağlara ait zayıf toplumsal düzenlerini sıkılaştırarak eski zamanların daha bütüncül bir biçimde düzene girmiş ve daha iyi denetlenmiş ilkel topluluklara onların dönüşmesine katkıda bulunan ürkütücü ve güçlü bir dehşet biçiminde korkunun yeni ve muhteşem bir türünü gözler önüne sermişti. Bazılarının hala varlığını sürdürdüğü bu anlamsız hurafeler; gerçek olmayan ve doğa\hyp{}ötesi kaynaklara ait hurafesel korku vasıtasıyla daha sonraki “Koruyucu’ndan duyulan korkunun bilgeliğin başlangıcı” keşfi için insanların akıllarını hazırlamıştır. Evrime ait bu dayanaksız korkular, açığa çıkarılıştan ilham alınan İlahiyat’a duyulan korku ile bütünleşmiş saygıyla yer değişikliğine uğratılmak için tasarlanmıştır. Hayalet korkusunun öncül simgesi güçlü bir toplumsal birleştirici haline gelmişti; ve bahse konu zaman zarfından bu yana insan türü, ruhaniyetin erişimi için benzer çabada bulunmaya devam etmektedir.
\vs p068 3:4 Açlık ve derin sevgi insanları bir araya getirmiştir; gösteriş ve hayalet korkusu onları bir arada tutmuştur. Ancak bu duygular tek başına, barışı destekleyen açığa çıkarışların etkisi olmadan, insanların karşılıklı birliktelikleri içindeki kuşkular ve sorunlarının ilerleyişine karşı koymaya yetkin değildir. İnsan\hyp{}üstü kaynaklarının yardımı olmadan toplumun ilerleyiş kolu, belirli sınırlara ulaşmasının ardından kopmaktadır; ve açlık, sevgi, gösteriş, ve korku biçiminde toplumsal hareketin bahse konu etkileri insan türünü savaşa ve katliama sürüklemektedir.
\vs p068 3:5 İnsan ırkının barış eğilimi doğal bir kazanım değildir; bu eğilim, ilerleyici ırkların birikmiş deneyimi biçiminde özellikle Barış’ın Prensi olan İsa’nın öğretileri olarak açığa çıkarılmış din öğretilerinden elde edilmiştir.
\usection{4.\bibnobreakspace Örf ve Adetlerin Evrimi}
\vs p068 4:1 Çağdaş toplumsal kurumların tümü, yabansı atalarınızın ilkel geleneklerinin evriminden doğmuştur; bugünün kabulleri, dünün değişikliğe uğramış ve genişlemiş gelenekleridir. Birey için alışkanlık ne ise, toplum için gelenek o anlama gelmektedir; ve toplum gelenekleri, kitlesel kabuller biçiminde halk adetleri veya diğer bir değişle kabilesel törelere doğru gelişmiştir. Bugünkü insan toplumuna ait kurumlarının tümü mütevazı kökenlerini bahse konu öncül başlangıçlardan almıştır.
\vs p068 4:2 Kitle mevcudiyetinin şartlarına topluluk yaşamını uyumlu hale getirmek amacıyla gerçekleştirilen bir çaba içerisinde örf ve adetlerin oluşmuş olduğu akılda tutulmalıdır. Ve bu kabile tepkilerinin tümü, keyif ve gücü memnuniyet ile deneyimlemenin yolları aranırken acı ve aşağılanmadan kaçınma çabasından doğmuştur. Dillerin oluşumuna benzer bir biçimde halk geleneklerinin kökeni her zaman bilinç dışı ve istemsiz bir nitelikte bulunmaktadır; bu nedenle onlar her koşulda gizemli bir hüviyet içerisinde saklı bir biçimde gerçekliğini sürdürmektedir.
\vs p068 4:3 Hayalet korkusu ilkel insanları doğa\hyp{}ötesi güçleri tahayyül etmeye sevk etmişti; ve böylelikle bu korku, toplumun adet ve geleneklerini nesiller boyu sonuçsal olarak bozulmayan bir biçimde muhafaza eden etik ve dinin bahse konu güçlü toplumsal etkileri için temelleri kesin olarak atmıştı. Örf ve adetleri öncül bir biçimde oluşturan ve onları kemikleştiren önemli bir inanış, ölülerin yaşadıkları ve öldükleri biçimlerde kıskançlığa sahip olduklarına dair kanı olmuştu; böylelikle bu topluluklar, beden içinde yaşadıkları dönemlerde onurlandırdıkları yaşam kurallarına aldırış etmeden saygısızlıkta bulunan hayattaki diğer fanileri ölülerin sert bir biçimde cezalandırmak için geri döneceklerine dair inancı barındırmaktaydılar. Bu inanç düzeninin tümü, sarı ırkın ataları için besledikleri bugünkü derin saygı tarafından sergilenmektedir. Daha sonraki gelişen ilkel din, örf ve adetleri kemikleştirme amacı içerisinde hayalet korkusunu büyük ölçüde güçlendirmiştir; ancak ilerleyen medeniyet artan bir biçimde insan türünü, hurafeye dayalı inancın korkusu ve köleliğine ait esaretten kurtarmıştır.
\vs p068 4:4 Dalamatia öğretmenlerine ait özgürleştiren ve bağımsızlaştıran öğretiden önce eski insan, örf ve adetlerin barındırdığı töre uygulamalarının savunmasız bir kurbanı olmuştu; ilkel yabansı varlıklar, bitmek tükenmek bilmeyen bir tören düzeniyle kuşatılmıştı. Uyandığı zamandan gece mağarasında uyuduğu vakte kadar yaptığı her şey --- kabilenin halk adetleri uyarınca --- bu yönde gerçekleştirilmek zorundaydı. İlkel insan, âdetin zorbalığının bir kölesi konumundaydı; onun yaşamı özgür, kendiliğinden gerçekleşen veya özgün hiçbir şeyi barındırmamaktaydı. Orada, daha yüksek bir akli, ahlaki veya toplumsal mevcudiyete doğru doğal bir ilerleme bulunmamaktaydı.
\vs p068 4:5 Öncül insan, oldukça kuvvetli bir biçimde gelenek tarafından tutulmaktaydı; yabansı insan, âdetin gerçek bir kölesiydi; ancak orada zaman zaman, düşüncenin yeni biçimlerini ve yaşamın gelişmiş yöntemlerini hayata geçirme cesaretine sahip olan insan türlerinden kaynaklanan değişik yaşam çeşitleri doğmuştur. Yine de ilkel insanın eylemsizliği, aşırı derecede hızlı ilerleyen bir medeniyetin olumsuz yöndeki yok edici uyumuna olan çöküş karşısında biyolojik emniyet kemeri olmuştur.
\vs p068 4:6 Ancak bu gelenekler, tamamiyle kötü bir niteliğe sahip değildir; onların evrimi devam etmelidir. Bu geleneklerin köklü devrimler sonucunda bütüncül bir biçimde değişikliğe uğratılmasına girişmek medeniyetin devamı için neredeyse ölümcül bir etkiye sahiptir. Gelenek, medeniyeti bir arada tutan devamlılık çizgisidir. İnsan tarihinin geldiği yol, bırakılmış gelenekler ve vadesi dolmuş toplumsal uygulamaların kalıntıları ile kaplıdır; ancak hiçbir gelenek, daha iyi ve daha yerinde geleneklerin eskileri ile değiştirilmesi dışında, örf ve adetlerini terk ederek yaşamını sürdürememiştir.
\vs p068 4:7 Bir toplumun kurtuluşu başlıca olarak örf ve adetlerinin gelişimsel evirilişine bağlıdır. Gelenek evriminin süreci, deneyim arzusundan kaynaklanmaktadır; rekabeti yaratan bir biçimde, yeni fikirler öne sürülmektedir. İlerleyen bir medeniyet, ilerleme fikri ile bütünleşmekte ve varlığını devam ettirmektedir. Ancak bu kanı, insan toplumunun bütünlüğü içerisinde ayrı ve yalıtılmış her değişikliğin iyiye hizmet ettiği anlamına gelmemektedir. Hayır! Gerçekten hayır! Çünkü Urantia medeniyetine ait ileri yönde gerçekleştirilen uzun çabalar içerisinde oldukça fazla gerileme meydana gelmiştir.
\usection{5.\bibnobreakspace Yeryüzü Yaşam Yöntemleri --- Korunum Sanatları}
\vs p068 5:1 Yeryüzü toplum sahnesi, insanlar bu sahnenin oyuncularıdır. Ve insan her zaman, etkin olan yeryüzü koşuluna oyununu uyumlu hale getirerek onu sergilemek zorundadır. Bu durum onun algı güçlüğünün varlığı durumunda bile gerçeklik taşımaktadır. İnsanın yeryüzü yaşama yöntemlerinin veya diğer bir değişle korunum sanatlarının yaşam ölçütleri ile toplamı, örf ve adetler olarak halk geleneklerinin bütününe eşittir. Ve insanın yaşamın ihtiyaçlarına olan uyumunun bütünü, onun kültürel medeniyetine eşittir.
\vs p068 5:2 En öncül insan kültürleri, Doğu Yarımküresi’nin nehirleri boyunca ortaya çıkmıştı; ve orada, medeniyetin ileriye doğru hareketi içinde dört büyük adım atılmıştı. Bu adımlar şunlardı:
\vs p068 5:3 1.\bibemph{ Toplama aşaması}. Açlık biçiminde yiyecek bulma zorunluluğu, ilkel yiyecek toplayıcılığı kuyruğu olarak üretim düzeninin ilk türünün ortaya çıkmasına sebebiyet vermiştir. Zaman zaman açlık yürüyüşünün bu türden bir kuyruğu, toprak üzerinde yiyecekleri toplayarak hareket eden bir biçimde on millik uzunluğa kadar varabilmekteydi. Bu aşama ilkel göçebe kültür dönemi olup, şu an yaşam türü biçiminde Afrikalı Buşmanlar tarafından takip edilmektedir.
\vs p068 5:4 2.\bibnobreakspace Avcılık Aşaması. Silahsal araç gereçlerin icadı; insanı bir avcı olmaya yetkin hale getirip, böylece yiyecek köleliğinden ciddi bir biçimde kurtarmıştır. Sert bir çatışmada yumruğunu kullanması sonucu elini ciddi bir biçimde inciten düşünceli bir Andonsal unsuru, kolu için uzun bir sopaya ek olarak yumruğu için kas kirişlerine dik konumda tutturulmuş sert bir çakmaktaşı parçasını kullanma fikrini yeniden keşfetmiştir. Birçok kabile bu türden bağımsız keşifleri gerçekleştirmiştir; ve çekiçlerin bahse konu bu çeşitli türleri, insan medeniyeti içinde ileri doğru gerçekleştirilen adımlardan birini temsil etmiştir. Bugün bazı Avustralya yerlileri, bu aşamanın biraz ötesine geçmiştir.
\vs p068 5:5 Mavi insanlar, uzman avcılar ve tuzak kurucular haline gelmişti; nehirlere sınırlı alan yaratan çitler çekerek büyük sayıda balık yakalayıp, bu balıkların fazlasını kış için kuruttular. Hünerli tuzak ve kapanların birçok türü yakalama oyununda kullanılmıştır; ancak daha ilkel ırklar büyük hayvanları avlamamışlardır.
\vs p068 5:6 3.\bibnobreakspace \bibemph{Kırsal yaşam aşaması}. Medeniyetin bu fazı, hayvanların evcilleştirilmesiyle mümkün hale gelmiştir. Araplar ve Afrika yerlileri, çok yakın zaman içerisinde kırsal hayata geçmiş topluluklar arasındadır.
\vs p068 5:7 Kırsal yaşam, yiyecek esaretinden ilave bir bağımsızlaşmayı sağlamıştır; insan, sürülerinin artışı biçiminde sahip olduğu sermayenin üretiminden yaşamayı öğrenmiştir; ve bu durum kendisine kültür ve ilerleyiş için daha fazla boş zamanı beraberinde getirmiştir.
\vs p068 5:8 Kırsal yaşam öncesi toplum, erkek ve kadın işbirliğinin gözlendiği topluluklardan bir tanesiydi; ancak hayvan evcilleştirilişinin yayılması kadınları toplumsal kölelik düzeyinin en alt seviyelerine düşürmüştür. Daha önceki zamanlarda hayvandan elde edilecek yiyeceği sağlamak erkeğin görevi iken, kadının işi bitkisel yiyecek kaynakları toplamaktı. Bu nedenle, insan mevcudiyetinin kırsal dönemine giriş yaptığında kadının soylu yeri büyük oranda düşüş göstermiştir. Yaşamın bitkisel ihtiyaçlarını üretmek için kadın hala toprak üzerinde çalışmak zorundayken, bol miktardaki bir hayvan yiyeceğini sağlamak için erkeğin yalnızca sürülerini ziyaret etmesi yetmekteydi. Böylelikle erkek göreceli bir biçimde kadından bağımsız hale gelmişti; kırsal yaşam döneminin tamamı boyunca kadının toplumdaki düzeyi gittikçe azalmıştır. Bu dönemin sonladığı zaman zarfında kadın; tıpkı sürünün hayvanlarından beklenen bir biçimde mücadele vermek ve sürünün gençlerini dünyaya getirmek gibi yalnızca çalışmak ve insan doğumlarını taşımak için gönderilmiş insan görünümlü hayvan düzeyinden çok da üst bir seviyede bulunmayan bir toplumsal konuma sahipti. Kırsal çağların insanları sahip oldukları hayvan sürüleri için büyük bir sevgi beslemektelerdi; bu hayvanlara besledikleri sevgi karşısında kadınları için daha derin sıcak duyguları geliştirememiş olmaları ne kadar acınası bir durumdur.
\vs p068 5:9 4.\bibnobreakspace \bibemph{Tarım Aşaması}. Bu çağ bitkilerin evcilleştirilmesiyle başlamış olup, maddi medeniyetin en yüksek türünü temsil etmektedir. Caligastia ve Âdem’in ikisi de, bahçecilik ve tarımcılığı öğretmeye çabalamışlardır. Âdem ve Havva bahçecilik ile uğraşmaktalardı, hayvanları gütmemektelerdi; ve bahçecilik, bahse konu zamanların gelişmiş bir kültürüydü. Bitkilerin büyümesi, insan türlerinin tüm ırkları üzerinde asilleştirici bir etki bırakmaktadır.
\vs p068 5:10 Tarım, dünyanın kara\hyp{}insan oranını dört katından daha fazla bir düzeye yükseltmişti. Tarım yaşamı, önceki kültürel aşamanın kırsal yaşam amaçlarıyla bütünleşebilmektedir. Bahsi geçen üç aşama bütünleştiğinde, erkekler avcılık yapmakta kadınlar ise toprağı sürmektedir.
\vs p068 5:11 Sürüyü güden insanlar ile toprağı süren bireyler arasında her zaman bir anlaşmazlık durumu mevcut olmuştur. Avcılar ve sürüyü güdenler, savaşçıl bir biçimde saldırgan bireylerdi; tarımla uğraşan birey ise daha çok barışı arzulayan bir türdü. Hayvanlar ile birleşme mücadele ve kuvveti beraberinde getirmektedir; bitkiler ile birleşme sabrı, sakinliği ve barışı yavaş yavaş kanıksatmaktadır. Tarım ve üretim, barışın etkinlikleridir. Ancak toplumsal etkinlikler olarak onların ikisinin de zayıf noktası, heyecan ve serüveni içlerinde barındırmamalarıdır.
\vs p068 5:12 İnsan toplumu, avcılık düzeyinden sürü iradeciliği boyunca tarımın toprak aşamasına doğru evirilme göstermiştir. Ve bu ilerleyici medeniyetin her aşaması, gittikçe azalan sayılarda varlığını sürdüren göçebeliği içinde barındırmıştır; evlerde yaşamaya başlayan insanların sayıları sürekli olarak artış göstermiştir.
\vs p068 5:13 Ve mevcut an içerisinde, artan şehirleşme ve vatandaşlık sınıflarının tarımsal olmayan topluluklarının çoğalımı ile sonuçlanan bir biçimde sanayi tarımı desteklemektedir. Ancak bir sanayi çağını yönlendirenler en yüksek toplumsal gelişmelerin her koşulda güçlü bir tarım temeline dayanmak zorunda oluşunu tanımada başarısız olursalar, bu sanayi çağı hayatta kalmayı ümit dahi edemez.
\usection{6.\bibnobreakspace Kültürün Evrimi}
\vs p068 6:1 İnsan, doğanın bir çocuğu olarak toprağın bir yaratılmıştır; karadan ne kadar kararlı bir biçimde kaçmaya çalışırsa çalışsın, son kertede o kesinlikle başarısızlığa uğrayacaktır. “Topraktan geldiniz ve yine toprağa döneceksiniz” ifadesi, insan türünün hepsi için gerçek anlamıyla doğruluk taşımaktadır. İnsanın temel mücadelesi; geçmişte olduğu, bugün gerçekleştiği ve gelecekte her zaman mevcut olacağı gibi kara için verilecektir. İlkel insan varlıklarının ilk toplumsal birliktelikleri, kara mücadelelerinden zaferle ayrılmak için gerçekleştirilmişti. Kara\hyp{}insan oranı her toplumsal medeniyetin oluşumuna kaynaklık etmektedir.
\vs p068 6:2 Sanat ve bilim araçları vasıtasıyla insanın usu, toprağın verimliliğini arttırmıştır; bu gelişme ile eş zaman içerisinde insan doğumundaki doğal artış bir ölçüde denetim altına alınmış, ve böylece kültürel bir medeniyet inşa etmek için besin ve boş zaman koşulları sağlanmıştır.
\vs p068 6:3 İnsan toplumu; nüfusun toprak sanatları ile doğru, hâkim ortak yaşam koşullarıyla ters orantılı bir biçimde çeşitlilik göstermesi zorunluluğunu emreden bir yasa tarafından denetlenmektedir. Bu öncül çağlar boyunca, hatta mevcut zamana kıyasla daha fazla bir biçimde, insanlar ve onların yaşadıkları toprak ile ilgili arz ve talep denge yasası ikisinin de değerini belirlemişti. Kullanılmayan kara parçası biçiminde toprağın bol olduğu zaman zarfında insanlara duyulan ihtiyaç oldukça fazlaydı; ve bu nedenle insan yaşamının değeri daha fazla yükselmişti; böyle olduğu için yaşamın yitirilmesi daha çok ürkütücü bir niteliğe sahipti. Toprak azlığı ve onunla ilişkili aşırı derecedeki nüfusun var olduğu dönemlerde savaşa, kıtlığa ve salgına daha az önem atfedilecek kadar insan yaşamı göreceli olarak değersizleşmişti.
\vs p068 6:4 Topraktan elde edilen üretim azaldığında veya nüfus artış gösterdiğinde, kaçınılmaz mücadele tekrar ortaya çıkmaktadır; insan doğasının bilinen en kötü nitelikleri gün yüzüne çıkmaktadır. Mekanik sanatların genişlemesi biçiminde toprak üretimindeki ilerleme ve nüfusun azalması bir bütün olarak, insan doğasının daha iyi yönlerinin gelişmesini destekleme eğilimi göstermektedir.
\vs p068 6:5 Sınır toplumu, insanlığın yeteneksiz olan yönünü sivrilmektedir; ruhsal kültür ile birlikte güzel sanatlar ve doğru bilimsel ilerlemenin tümü en iyi biçimde; genel kara\hyp{}insan oranına kıyasla bir parça alt düzeyde bulunan bir tarım ve sanayi toplumu tarafından yaşamın desteklendiği koşullar içerisinde geniş yerleşim merkezlerinde büyüme göstermektedir. Şehirler her zaman, sakinlerinin iyi veya kötü yönde sahip olduğu güçleri çoğaltmaktadır.
\vs p068 6:6 Ailenin büyüklüğü her zaman, ortak yaşam koşullarından etkilenmektedir. Ortak yaşam şartları iyileşme gösterdiği zaman aile, sabit düzeye veya diğer bir değişle kademeli nesil tükeniş seviyesine kadar varan bir doğrultuda küçülme göstermektedir.
\vs p068 6:7 Ortak yaşam koşulları en başından mevcut zamana kadar çağlar boyunca, yalın niceliğe tezat oluşturan bir biçimde yaşam mücadelesi veren bir nüfusun niteliğini belirlemiştir. Yerel sınıfın sahip olduğu ortak yaşam koşulları, yeni örf ve adetler biçiminde yeni toplumsal tabakalaşmanın açığa çıkışına kaynaklık etmektedir. Ortak yaşam koşulları çok karmaşık veya çok şatafatlı olduğunda, insanlar hızla intihara meyilli hale gelmektedir. Toplumsal tabakalaşma, yoğun nüfusun ürettiği çetin rekabete ait yüksek toplumsal baskının doğrudan sonucudur.
\vs p068 6:8 Öncül ırklar, nüfusu kısıtlamak için tasarlanan uygulamalara sıkça başvurmak zorunda kaldılar; ilkel kabilelerin tümü çarpık doğan veya iyileşmeyecek hastalıklara sahip olan çocukları öldürmüştü. Kız bebekler, eş alımı zamanından önce sıkça başvurulan bir biçimde öldürülmekteydi. Çocuklar zaman zaman doğum anında boğulmaktaydı, ancak öldürme uygulaması için gözde yöntem yiyecek ve giyecek olmadan onları doğada açıkta bırakmaydı. İkizlerin babası genellikle çocuklarından birinin öldürülmesinde ısrar etmekteydi; çünkü bir seferde gerçekleşen çoklu doğumların, büyü veya sadakatsizlik soncunda meydana geldiğine inanılmaktaydı. Buna rağmen aynı cinsiyette dünyaya gelmiş ikizlerin yaşamı bir kural olarak bağışlanmaktaydı. İkizler hakkındaki bu tabular her ne kadar bir dönem içerisinde neredeyse evrensel niteliğe sahip olmuşsa da, onlar hiçbir zaman Andonsal adetlerin bir parçası haline gelmemiştir; bu insanlar ikizleri her zaman, iyi şansın habercileri olarak görmüşlerdir.
\vs p068 6:9 Birçok ırk çocuk düşürme yöntemini öğrenmişti; ve bu uygulama, evli olmayan çiftler arasında çocuk doğumuna dair tabunun yerleşmesinden sonra oldukça yaygın hale gelmişti. Evlenmemiş bir kadının çocuğunu öldürmesi uzun süreler boyunca yerine getirilen bir adetti; ancak daha medeni topluluklar arasında yasal olmayan bu çocuklar, doğumu yapan bireyin annesinin vesayetine alınmıştı. Birçok ilkel kavim, çocuk düşürme ve çocuk öldürme uygulaması yüzünden neredeyse tamamen yok olmuştu. Ancak adetlerin dayattığı emirlere rağmen, en başından beri çok az sayıdaki çocuk bir kez emzirildikten sonra öldürülmüştür --- anne sevgisi fazlasıyla güçlüdür.
\vs p068 6:10 Yirminci yüzyılda bile, bu ilkel nüfus denetiminin kalıntıları varlığını sürdürmektedir. Avustralya’da, iki veya üç çocuktan fazlasını yetiştirmeyi reddeden annelere sahip bir kabile bulunmaktadır. Yakın bir zaman önce her doğan beşinci çocuğu yiyen insan etiyle beslenen bir kabile bulunmaktaydı. Madagaskar’da bazı kabileler hala, belirli şansız günlerde doğan çocukların hepsini yok etmektedirler; bu uygulama, doğan tüm bebeklerin yaklaşık olarak yüzde yirmi beşinin ölmesiyle sonuçlanmaktadır.
\vs p068 6:11 Bir dünya bütünlüğünden bakıldığında nüfus fazlalığı geçmişte hiçbir zaman ciddi bir sorun teşkil etmemiştir; ancak savaşlar azaltıldığında ve bilim artan bir biçimde insan hastalıklarını denetim altına aldığında, gelecek zaman içerisinde nüfus fazlalığı ciddi bir sorun haline gelebilir. Bu türden bir zaman zarfında dünya önderliğinin sahip olduğu bilgeliğin vereceği büyük bir sınav kendisini gösterecektir. Urantia yöneticileri; olağanüstü insanların aşırı uçlardaki örneklerini ve olağan düzeyin altında bulunan insanların devasa sayıda artan topluluklarını desteklemek yerine olağan veya diğer bir değişle istikrara kavuşmuş insan varlığını güçlendirme kavrayış ve cesaretine sahip olacaklar mı? Olağan insan desteklenmelidir; olağan insan, medeniyetin omurgası ve ırkın sahip olduğu değişken dâhilerin kaynağıdır. Olağan düzeyin altında bulunan insan, toplumunun denetimi altında tutulmalıdır; bu insanların, hayvan düzeyinin üstünde bir usu gerektiren ancak insan varlıklarının yüksek türleri için gerçek köleliği ve esareti yaratacak bayağı taleplerde bulunan görevler biçiminde üretimin daha alt seviyelerini idare etmekten fazlası üretilmemelidir.
\vs p068 6:12 [Urantia üzerinde belirli bir zaman zarfında konumlanmış olan bir Melçizedek tarafından sunulmuştur.]
