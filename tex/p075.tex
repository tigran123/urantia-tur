\upaper{75}{Âdem ve Havva’nın Yükümlülüklerini Yerine Getirmedeki Başarısızlığı}
\vs p075 0:1 Urantia üzerindeki yüz yıldan fazla süren çabası sonrasında Âdem, Cennet Bahçesi’ni dışında oldukça düşük ölçekte gerçekleşen ilerleyişi görebilmekteydi; dünyanın büyük bir kısmının fazla gelişmediği görünmekteydi. Irk ilerlemesinin gerçekleşmesi, çok uzun bir süre sonra ortaya çıkacağa benziyordu; ve bu durum, asli tasarımlar içinde mevcut olmayan başka şeylerin talep edilmesini gerektirecek kadar ümitsiz göründü. En azından bu düşünce Âdem’in aklından sıkça geçmişti; ve o, birçok kez bu düşüncelerini Havva’ya ifade etti. Âdem ve onun eşi sadıklardı; ancak onlar, türlerinden tecrit edilmiş bir haldelerdi; ve onlar, dünyalarının üzüntü verici talihsiz geleceğinden fazlasıyla endişeye kapılmışlardı.
\usection{1.\bibnobreakspace Urantia Sorunu}
\vs p075 1:1 Deneyimsel nitelikte bulunan, isyanla parçalanmış ve tecrit edilmiş Urantia üzerinde Âdemsel görev ürkütücü bir teşebbüstü. Ve Maddi Erkek ve Kız Evlat, gezegensel görevlerinin zorluğundan ve onun çetrefilliğinden haberdar olmuşlardı. Yine de onlar cesur bir biçimde, çok katmanlı sorunlarını çözme görevlerine koyulmuşlardı. Ancak onlar; insan ırkları arasındaki kusurlu ve bozulmaya uğramış kolları ortadan kaldırmaya yönelik çok önemli görevi üstlendiklerinde, oldukça umutsuzluğa düştüler. Onlar, bu karmaşadan herhangi bir çıkar yolu göremediler; ve onlar, ne Jerusem’de ne de Edentia’da bulunan üstlerinin tavsiyelerini alamadılar. Burada onlar tecrit edilmiş olup, gün be gün bazılarının çözülemez göründüğü birtakım yeni ve çetrefilli çıkmazlar ile karşılaşmaktalardı.
\vs p075 1:2 Olağan koşullar altında, bir Gezegensel Âdem ve Havva’nın ilk görevi, ırkların arasındaki eş güdümü ve onların birbirine olan karışımını sağlamak olurdu. Ancak Urantia üzerinde bu türden bir gaye yalnızca, gerçekleşmesi imkânsızmış bir tasarım gibi göründü; çünkü ırklar her ne kadar biyolojik olarak zinde olsalar da, bu vakte kadar geri ve kusurlu kollarından hiçbir şekilde arındırılmamıştı.
\vs p075 1:3 Âdem ve Havva kendilerini; perişan ruhsal karanlık içinde doğrultusunu bulmaya çalışan ve bir önceki idarenin görevlerini yerine getirmedeki başarısızlığının çok talihsiz bir biçimde şiddetlendirdiği kafa karışıklığının kendisine musallat olduğu bir dünya olarak, insanın kardeşliğinin duyurulması için tamamiyle hazırlıksız bir âlemde bulmuşlardı. Akıl ve ahlak gelişmemiş bir düzeydeydi; ve dinsel bütünlüğü sağlama görevine başlamak yerine onlar, dünya sakinlerini dinsel inanışın en basit türlerine olan inanca çekme görevine yeniden başlamak zorundalardı. Kullanılmaya hazır bir dili bulmak yerine onlar, yerel lehçelerin yüzlercesinden doğan dünya çapındaki karmaşayla karşılaştılar. Gezegensel hizmetin üyesi hiçbir Âdem, daha öncesinde bundan daha zorlu bir dünya üzerine görevlendirilmemişti; zorluklar aşılmaz, sorunlar ise yaratılmışın beraberinde getirebileceği çözümlerin ötesinde göründü.
\vs p075 1:4 Onlar tecrit edilmişti, ve üzerlerine çöken devasa yalnızlık hissi Melçizedek alıcılarının erken ayrılışlarıyla birlikte tamamiyle daha da fazlalaşmıştı. Meleksel düzeylerin araçlarıyla sadece dolaylı bir biçimde onlar, gezegen dışında herhangi bir varlık ile iletişimde bulunabilmekteydiler. Kademeli olarak onların cesareti zayıfladı, hevesleri kırıldı ve zaman zaman inançları neredeyse bocaladı.
\vs p075 1:5 Ve bu anlatım, karşılarına çıkan görevler karşısında düşünceye dalan iki soylu ruhun şaşkınlığının gerçek resmidir. Onların ikisi de kesin bir biçimde, gezegensel görevlerinin yerine getirilmesine ilişkin sahip oldukları çok büyük sorumluluğun farkındaydılar.
\vs p075 1:6 Muhtemelen Nebadon’un Hiçbir Maddi Evladı daha öncesinde, Urantia’nın üzüntü verici talihsiz geleceğinde Âdem ve Havva’nın karşılaştığı bu türden zorlu ve ümitsiz görülen bir görevle yüzleşmemişlerdi. Ancak onlar daha ileriyi görüşlü ve daha \bibemph{sabırlı} olsalardı, bir zaman zarfında başarıyı elde edeceklerdi. Özellikle Havva olmak üzere ikisi de tamamiyle çok sabırsızdı; onlar, çok uzun dayanıklılık sınavını vermeye gönüllü değillerdi. Onlar, bir takım anlık sonuçların ortaya çıktığını görmek istediler; ve onlar bunları gördüler de, ancak onların bu şekilde elde ettiği sonuçlar kendilerine ve dünyalarına olabilecek en zarar verici bir halde ortaya çıktı.
\usection{2.\bibnobreakspace Caligastia’nın Komplosu}
\vs p075 2:1 Caligastia Cennet Bahçesini sıkça ziyaret etmiş olup, Âdem ve Havva ile birlikte birçok görüşme düzenledi; ancak onlar, Caligatia’nın önerdiği tavizsel tasarımların ve kısa\hyp{}yolu tercih eden serüvenlerin tümüne karşı kararlı durmuşlardı. Onlar, bu türden ahlaksız tekliflere karşı etkin bir bağışıklığın yaratılması için isyanın neden olduğu yeterli miktardaki gelişmeyi gözlemlemişlerdi. Âdem’in genç evladı bile, Daligastia’nın tekliflerinden etkilenmemişti. Ve tabii ki ne Caligastia ne de onun yardımcısı, bırakın Âdem’in çocuklarını yanlışa çekebilmek için ikna edebilmeyi, herhangi bir bireyin iradesini etkileyecek bile güce sahip değillerdi.
\vs p075 2:2 Caligastia’nın bu dönemde hala, yerel evrenin yanlış yönlendirilmiş ama yine de yüksek bir Evladı olarak Urantia’nın unvan sahibi Gezegensel Prens’i olduğu hatırlanmalıdır. O, Urantia üzerindeki Hazreti Mikâil’in dönemine kadar nihai olarak görevden alınmamıştı.
\vs p075 2:3 Ancak düşkün Prens azimli ve kararlıydı. O; yakın bir zaman içerisinde Âdem üzerinde emellerini gerçekleştirmekten vazgeçip, kurnaz bir yan saldırıyı Havva üzerinde uygulamaya karar verdi. Bu kötü kişilik, Nodit topluluğunun üst tabakasının üyeleri olan uygun kişiler üzerinde hünerli bir biçimde emellerini gerçekleştirmenin tek yolunun bir dönem bedensel\hyp{}görev yardımcıları olan soylarından geçtiğini anladı. Ve böylelikle tasarımlarını, eflatun ırkının annesini tuzağa düşürecek şekilde gerçekleştirdi.
\vs p075 2:4 Âdem’in tasarımlarını engelleyecek veya eşiyle birlikte gezegensel görevlerini tehlikeye atacak bir şeyi herhangi bir biçimde yapmak Havva’nın niyetine taban tabana zıttı. Kadının sahip olduğu, ileriyi gören bir biçimde daha uzun süreçler içerisinde sonuç verecek şeyleri tasarlama yerine yakın vadedeki sonuçlara odaklanma eğilimi bilerek Melçizedekler ayrılmalarından önce; özellikle Havva’yı, gezegen üzerinde tecrit edilmiş konumlarında onları kuşatan belirli tehlikelere karşı tembihlemiş olup, bilhassa, ortak sorumluluklarını yerine getirmede kişisel veya gizli herhangi bir yöntemi denemeyen bir biçimde eşinin yanından hiçbir zaman ayrılmaması hususunda onu özel olarak uyarmışlardır. Havva, yüz yıldan uzun bir süre boyunca bu yönergeleri en olması gereken titizlikle takip etmiştir; ve o, Serapatatia ismindeki belirli bir Nodit önderine memnuniyetle yaptığı gittikçe özelleşen ve gizlileşen ziyaretlerde herhangi bir tehlikenin yattığını fark etmemişti. Bu olayın bütünlüğü o kadar kademeli ve doğal bir biçimde ilerlemişti ki Havva hiçbir şeyin farkına varmamıştı.
\vs p075 2:5 Cennet Bahçesi sakinleri, Bahçe’nin ilk günlerinden beri Nodit unsurları ile iletişim halindeydiler. Caligastia görevlilerinin doğru düzenden ayrılan unsurlarından gelen bu karma soylardan onlar, oldukça değerli yardım ve işbirliği görmüşlerdi; ve onların vasıtasıyla Cennet Bahçesi düzeni bu aşamada, bütüncül felaketi ve nihai yıkımıyla buluşmaya yakındı.
\usection{3.\bibnobreakspace Havva’nın Cezbedilişi}
\vs p075 3:1 Babasının ölümü üzerine Serapatatia Nodit kabilelerinin batı veya diğer bir değişle Suriye konfederasyonunun başına geldiğinde, Âdem dünya üzerindeki ilk yüz yılını yeni tamamlamıştı. Serapatatia, eskinin mavi ırkının üstün bilge kadınlarından biriyle evlenen Dalamatia’nın sağlık heyetinin bir zamanlar başkanlığını yapmış olan kişinin soyundan gelen parlak bir kişi olarak, buğday tenli birisiydi. Bu döneme kadar gelen bütün çağlar boyunca bu ırk kolu; batı Nodit kabileleri üzerinde yönetimi elinde bulundurmuş olup, onlar üzerinde büyük bir etkiye sahip olmuştu.
\vs p075 3:2 Serapatatia Cennet Bahçesi’ne birkaç ziyarette bulunup, Âdem’in varoluş gayesinin doğruluğundan derin bir biçimde etkilenen bir hale gelmişti. Suriye Nodit unsurlarının önderliğini üstlendiğinden kısa bir süre sonra o, Cennet Bahçesi içinde Âdem ve Havva’nın çalışmalarına destek olacak bir biçimde bir beraberlik kurma isteğini açıkladı. İnsanlarının büyük bir kısmı onunla beraber bu birliktelik tasarımına katıldı; ve komşu kabilelerin tümü içinde en güçlü ve en us sahibi olan topluluğun, neredeyse tek vücut halinde dünyanın gelişmesi için gayesine destek vermek amacıyla harekete geçmiş olduğuna dair haber Âdem’i sevindirmişti; bu hareket kesin bir biçimde cesaretlendiriciydi. Ve bu büyük gelişmeden kısa bir süre sonra Serapatatia ve onun yeni yardımcıları, Âdem ve Havva tarafından evlerinde ağırlandı.
\vs p075 3:3 Serapatatia, Âdem’in kumandanlarının tümü içinde en yetkin ve etkin olanlardan biri haline gelmişti. O, faaliyetlerinin tümü içinde tamamiyle dürüst ve bütünüyle içtendi; o hiçbir zaman uyanık olmamıştı; daha sonraki zamanlarda bile o, oyunbaz Caligastia’nın tesadüfen keşfettiği bir araç olarak kullanılmıştı.
\vs p075 3:4 Yakın bir süre sonra Serapatatia, Cennet Bahçesi kabile ilişkileri heyetinin yardımcı başkanı olmuştu; ve birçok tasarım, uzak kabilelerin Cennet Bahçesi gayesine olan bağlılığını elde etme görevinin daha kararlı bir biçimde uygulanması uyarınca hazırlanmıştı.
\vs p075 3:5 O, Âdem ve --- bilhassa --- Havva ile birçok görüşme düzenledi; ve onlar, yöntemlerini geliştirmek için birçok tasarım üzerinde konuştular. Bir gün Havva ile görüşmesi esnasında Serapatatia; eflatun ırkının geniş sayılardaki üyeleri seçilmek için beklenirken, bu arada, ihtiyaç duyan bekleyiş halindeki kabileleri doğrudan geliştirmek için bir şeylerin yapılmasının oldukça yararlı olabileceğini düşündü. Serapatatia; en ileri ve en işbirlikçi ırk olarak Nodit unsurları eğer bir kökeni eflatun kolundan gelen kendilerinden çıkmış bir öndere sahip olurlarsa, bahse konu önderin bu unsurları Cennet Bahçesi’ne daha yakından bir biçimde bağlayacak güçlü bir bağı oluşturacağını öne sürdü. Ve tüm bunların hepsinin dünya yararına olacağı aklı başında bir biçimde ve içtenlikle düşünülmüştü, çünkü Cennet içinde büyütülecek ve eğitilecek olan bu çocuk babasının insanları üzerinde sonsuza kadar büyük bir etki bırakacaktı.
\vs p075 3:6 Serapatatia’nın önerdiği her şeyde tamamiyle dürüst ve bütünüyle içtenlikle hareket etmiş olduğunun altı tekrar çizilmelidir. O, Caligastia ve Daligastia’nın yararına çalışmakta olduğuna dair kuşkuya bir kez bile kapılmamıştır. Serapatatia, Urantia’nın kafa karışıklığı içindeki topluluklarının dünya çapındaki canlandırılışına girişilmeden önce eflatun ırkının güçlü bir ırk kolu kökenini inşa etme tasarımına sonuna kadar sadıktı. Ancak bu durumun gerçekleşmesi yüzyıllar sürecekti; ve o sabırsızdı; o --- kendi yaşam süresi içinde --- yakın vadede sonuçlanacak birtakım şeyleri görmek istemişti. O, dünyanın ilerletilmesi gayesinde çok az şeyin gerçekleşmiş olması nedeniyle Âdem’in cesaretinin çoğu kez kırılmış olduğunu Havva’ya kesin bir biçimde aktardı.
\vs p075 3:7 Beş yıldan fazla bir süre boyunca bu tasarımlar gizli bir biçimde olgunlaşmıştı. En sonunda bu tasarımlar, dost Nodit unsurlarının yakın kolunun en parlak aklı ve en etkin lideri olan Cano ile gizli bir görüşme yapmaya Havva’nın razı olduğu noktaya kadar gelişti. Cano, Âdemsel düzene oldukça olumlu bakmaktaydı; gerçekte o, Cennet Bahçesi ile dostane ilişkilerin kurulmasını isteyen komşu Nodit unsurlarının samimi ruhsal önderiydi.
\vs p075 3:8 Kader buluşması, Âdem’in evinden çokta uzak olmayan bir yerde, sonbahar akşamı alacakaranlık saatlerinde yapıldı. Havva, güzel ve coşkulu olan Cano ile daha önce hiç tanışmamıştı --- ve o, Prens’in görevlilerinden gelen uzak atalarının sahip olduğu üstün beden ve olağanüstü aklın bu günlere geldiği muhteşem bir örneğiydi. Ve Cano da, Serapatatia tasarımının doğruluğuna bütünüyle inanmıştı. (Cennet Bahçesi dışında birden fazla kişi ile çiftleşme yaygın bir uygulamaydı.)
\vs p075 3:9 Övgüden, coşkudan ve güçlü kişisel iknadan etkilenerek Havva hemen oracıkta; geniş çaplı ve ileriyi gören kutsal tasarıma kendi küçük dünyayı kurtarma katkısını eklemek için, bahse konu oldukça tartışılmış girişime koyulmaya razı oldu. Kendisi neyin meydana gelmekte olduğuna dair bütüncül bir farkındalığa varmadan önce, vahim adım çoktan atılmıştı.
\usection{4.\bibnobreakspace Doğru Düzenden Ayrılışın Gerçekleşmesi}
\vs p075 4:1 Gezegenin göksel yaşamı hareketli bir hal içindeydi. Âdem bir şeylerin yanlış gittiğini anladı, ve Havva’nın Cennet Bahçesi’nde yayına gelmesini istedi. Ve bu aşamada ilk defa Âdem, dünya gelişimini iki doğrultuda hızlandırmak için uzunca düşünülerek beslenen tasarıma dair bütüncül hikâyeyi duydu: bu iki doğrultudaki gelişim, Serapatatia girişiminin uygulanmasıyla eş zamanlı olarak kutsal tasarımın sürdürülmesiydi.
\vs p075 4:2 Ve Maddi Erkek ve Kız Evlat mehtaplı bir Bahçe gecesinde böyle konuşurlarken, “Cennet’in sesi” görevlerine itaatsizlikleri nedeniyle onları kınadı. Ve bu ses, onların Bahçe anlaşmasına ters düştüklerine dair Cennet Bahçesi çiftine karşı yapılmış benim kendi duyurumdan başkası değildi; onların Melçizedekler’in yönergelerine karşı geldiklerine, ve evrenin egemenine olan bağlılık yeminlerinin yerine getirilmesinde yükümlülüklerini yapmadıklarına dair bir bildiriydi.
\vs p075 4:3 Havva, iyiliğe kötülüğü karıştırmaya razı olmuştu. İyilik, kutsal tasarımların yerine getirilmesidir; günah ise, kutsal iradeye karşı kasıtlı bir biçimde karşı gelmektir; kötülük, evren düzensizliğiyle ve gezegensel kafa karışıklığıyla sonuçlanan, tasarımların yanlış uygulanması ve işleyiş biçimlerini olması gereken bir biçimde düzenlememektir.
\vs p075 4:4 Cennet Bahçesi çifti yaşam ağacından ne zaman bir meyve koparsa, iyilik ve kötülüğü bir araya getirerek Caligastia’nın tavsiyelerine uyan sonuçlara sebebiyet vermekten kaçınmaları konusunda sorumlu baş melek tarafından uyarılmışlardı. Onlar böylelikle sert bir biçimde uyarılmışlardı: “İyiliğe kötülüğü karıştırdığınız gün, sizler kesin bir biçimde âlemin fanileri haline geleceksiniz; sizler kesinlikle öleceksiniz.”
\vs p075 4:5 Havva Cano’ya, gizli buluşmalarının kaçınılmaz sonu ile ilgili bu sürekli tekrarlanan uyarıdan bahsetti; ancak Cano, bu türden uyarıların anlamını veya önemini bilmeden, erkek ve kadınların iyi niyetlerle ve doğru amaçlarla kötülük işleyemeyeceklerini söyleyerek ona güvence verdi; o Havva’yı, kesinlikle ölmek yerine dünyayı kutsayacak ve onu istikrara kavuşturacak bir biçimde büyüyecek evladının kişiliği içinde yeniden doğacağına inandırdı.
\vs p075 4:6 Kutsal tasarımı değişikliğe uğratmaya dair bu tasarım; her ne kadar dünyanın refahı ile ilgili en içten ve sadece en yüksek amaçlarla düşünülmüş ve uygulanmış olsa da, kutsal tasarım olan doğru yoldan ayrıldığı için, doğru sonuçları elde etmek için yanlış yollardan gidişi temsil ettiği için kötülüğün kendisi olmuştur.
\vs p075 4:7 Havva’nın Cano’yu iyi görünümlü bulduğu ve kendisini baştan çıkaranın “Âdemsel doğanın kavranılmasına yardımcı bir biçimde insan olaylarının yeni ve artan bilgisine ek olarak insan doğasının hızlandırılmış anlayışı” vaatlerinin tümüne ön ayak olduğu doğrudur.
\vs p075 4:8 Üzücü durumlarda gerçekleştirilmesi benim görevim haline gelmiş bir biçimde, o gece Cennet Bahçesi’nde eflatun ırkının baba ve annesiyle konuştum. Ben, Anne Havva’nın doğru düzenden ayrılışıyla sonuçlanan her şeyin hikâyesini bütünüyle dinledim; ve ben, mevcut durum ile ilgili onlara öğütlerde ve tavsiyelerde bulundum. Bu tavsiyelerin bazılarını onlar dinlediler; bazılarını ise önemsemediler. Bu görüşme yazıtlarınızdaki “Koruyucu Tanrı Âdem ve Havva’yı Cennet’e çağırıp, ‘Nerdesiniz?’ diye sorduğu” anlatımda geçmektedir. İster doğal ister ruhsal olsun olağandışı veya olağanüstü her şeyi Tanrı’ların kişisel müdahalesine atfetmek daha sonraki nesillerin bir uygulamasıydı.
\usection{5.\bibnobreakspace Doğru Düzenden Ayrılmanın Sonuçları}
\vs p075 5:1 Havva’nın gerçekleri görmemesi gerçekten acınası bir durumdu. Âdem vaziyetin durumunu bütünüyle gördü, ve kalbi kırık ve karamsar bir halde hatalı eşi için yalnızca acıma ve anlayış besledi.
\vs p075 5:2 Havva’nın yanlış adımı attığı gün sonrasında Âdem’in, Cennet Bahçesi’nin batı okullarının başı olan parlak Nodit kadını Laotta’nın peşine düşmesi; başarısızlığın farkındalığından doğan umutsuzluk içinde ve Havva’nın düşüncesiz tasarımına daha önceden evet demiş oluşu sonrasında yapmış olduğu bir eylemdi. Ancak yanlış anlamayın; Âdem aldanmamıştı; o, tam da neyle karşılaşmakta olduğunu bilmekteydi; o bilinçli bir biçimde Havva’nın kaderini paylaşmayı tercih etmişti. O, fani\hyp{}üstü bir sevgi ile eşini sevmişti; ve Urantia üzerinde onsuz yalnız bir gece nöbetçisi olarak kalma olasılığını düşünmek katlanabileceğinden çok daha fazlasıydı.
\vs p075 5:3 Ve Havva’ya ne olduğunu öğrendiklerinde Cennet Bahçesi’nin kızgın sakinleri denetlenemez bir duruma geldiler; onlar, yakındaki Nodit yerleşkesine savaş ilan ettiler. Onlar; Cennet Bahçesi kapılarından taşıp, bu hazırlıksız insanların üzerine yürüyüp, --- erkek, kadın veya çocuk ayrımı yapmadan --- onları tamamen yok ettiler. Ve henüz doğmamış Kabil’in babası olan Cano aynı zamanda yok edilmişti.
\vs p075 5:4 Nelerin meydana geldiğinin farkına vardığında Serapatatia; dehşete kapılıp, korku ve pişmanlık içine düşmüştü. Bir sonraki gün kendisini büyük nehrin sularına bırakıp, boğularak intihar etmişti.
\vs p075 5:5 Âdem’in çocukları, babaları otuz gün ortan oraya yalnızlık içerisinde gezerken kendinden geçmiş annelerini teselli etmeye çalıştılar. Ve bu dönemin sonunda kararlılık kendisini göstermiş ve Âdem evine dönüp, gelecekteki faaliyetleri için tasarımlarda bulunmaya başlamıştır.
\vs p075 5:6 Yanlış yönlendirilmiş ebeveynlerin düşüncesizliklerinin sonuçları çoğu zaman masum çocukları tarafından paylaşılmaktadır. Âdem ve Havva’nın dürüst ve soylu çocukları, oldukça anlık ve acımasız bir biçimde üzerlerine düşen akıl almaz facianın tarif edilemez üzüntüsüyle şaşkına dönmüşlerdi. Elli yıl boyunca bu çocukların daha ergin olanları; kendinden geçmiş anneleri hangi konumda olduğuna veya geleceğinin ne olacağına dair tamamiyle bilinçsizlik içerisindeyken babalarının evden uzaklaştığı özellikle o otuz günlük sürecin yarattığı dehşet olmak üzere, bahse konu facia dönemlerinin üzüntüsü ve kederinden kurtulamamışlardı.
\vs p075 5:7 Ve bahse konu bu otuz gün Havva’ya, üzüntü verici ve acı dolu uzun yıllar gibi gelmişti. Bu soylu ruh hiçbir zaman, akli düzeyde acı çektiği ve ruhsal düzeyde üzüntü duyduğu bu dayanılmaz sürecinin yarattığı etkilerden bütünüyle kurtulamamıştı. Onların daha sonraki yoksunlukları ve maddi zorluklarının hiçbiri, Havva’nın hafızasındaki yalnızlık ve dayanılmaz belirsizliğin bu korkunç günleri ve berbat geceleri ile hiçbir zaman karşılaştırılamazdı bile. O Serapatatia’nın sabırsızlıkla ne yaptığını öğrenmişti; ve o, eşinin keder içinde kendisini yok edip etmediğini veya kendisinin neden olduğu yanlış adım sonrası eşinin dünyadan cezalandırmak için alınıp alınmadığını bilmiyordu. Ve Âdem geri döndüğünde Havva, uzun ve zorlu yaşam birlikteliklerinin parçası olan meşakkatli hizmetlerinin hiçbir zaman ortaya çıkmasına engel olamadığı neşe ve minnettarlıktan doğan bir tatmini yaşadı.
\vs p075 5:8 Zaman ilerlemekteydi, ancak Âdem; Melçizedek alıcılarının Urantia’ya döndüğü ve dünya olayları üzerinde yönetimi üstlendiği vakit olan, Havva’nın doğru düzenden ayrılmasından yetmiş gün sonraya kadar sebep oldukları suçun içeriğini bilmemekteydi. Ve bunun sonrasında onlar başarısız olduklarını bilmektelerdi.
\vs p075 5:9 Ancak daha fazla sorun ortaya çıkmaktaydı: Cennet Bahçesi yakındaki Nodit yerleşkesinin yok edilmesine dair haberler Serapatatia’nın ev kabilelerinden kuzeye doğru hızlı bir biçimde yayılmaktaydı; ve yakın bir zaman içinde büyük bir kalabalık Cennet Bahçesi’ne yürümek için toplanmaktaydı. Ve bu durum, Âdem ve Nodit unsurları arasındaki uzun ve çetin bir savaşın başlangıcıydı; çünkü bu düşmanlıklar, Âdem ve onu takip edenlerin Fırat nehri vadesinde ikinci bahçeye yaptıkları göçlerinden çok sonraya kadar devam etmişti. Orada, “Âdem ve Havva’nın tohumları arasında gerçekleşmiş bir biçimde, erkek ve kadın arasında [yoğun ve uzun yıllar süren] düşmanlık” vardı.
\usection{6.\bibnobreakspace Âdem ve Havva’nın Cennet Bahçesi’nden Ayrılışları}
\vs p075 6:1 Âdem Nodit unsurlarının kendilerine gelmekte olduklarını öğrendiğinde, Melçizedekler’in tavsiyesine başvurmaya çalıştı; ancak onlar, yalnızca en iyi düşündüğü şeyi yapmasını ve seçeceği herhangi bir doğrultuda olabildiğince gerçekleştirecekleri dostane iş birlik sözlerini vermesini söyleyerek, ona tavsiyede bulunmayı reddettiler. Melçizedekler’in çok daha öncesinden, Âdem ve Havva’nın kişisel tasarımlarına karışmaları yasaklanmıştı.
\vs p075 6:2 Âdem, kendisi ve Havva’nın başarısız olduğunu bilmekteydi; Melçizedek alıcılarının mevcudiyeti, her ne kadar kişisel düzeylerine veya kendilerini bekleyen kaderlerine dair hiçbir şeyi bu aşamada hala bilmemekte olsa da, başarısız oldukları gerçeğini anlatmaktaydı. O, önderlerini takip etmeye kendisini adayan bin iki yüz sadık takipçisi ile bütün bir gece süren bir görüşme düzenledi; ve bir sonraki gün öğlen vaktinde bu kutsal yolcular, yeni evlerini bulmak için Cennet Bahçesi’nden ayrıldılar. Âdem hiçbir şekilde savaş istemiyordu, ve buna uygun bir biçimde ilk bahçeyi Nodit unsurlarına onlara karşı gelmeden bıraktı.
\vs p075 6:3 Cennet Bahçesi kervanının ilerleyişi, Jerusem’den gelen yüksek melek taşıyıcılarının varışıyla Bahçe’den çıktıkları üçüncü gününde durdu. Ve ilk kez Âdem ve Havva, çocuklarının başına ne geleceğinden haberdar edilmişti. Taşıyıcılar beklerken, (yirmi yıl olan) reşitlik yaşına ulaşmış çocuklara, Urantia’da ebeveynleri ile kalma veya Norlatiadek’in En Yüksek Unsurları’nın vesayetleri altına girme tercihi sunuldu. Bu çocukların üçte ikisi Edentia’ya gitti; üçte biri ise ebeveynleri ile kalmayı tercih etti. Reşitlik öncesi dönemde bulunan çocukların hepsi Edentia’ya götürüldü. Maddi Erkek ve Kız Evlat’ın deneyimlediği üzücü ayrılığa bakmaya hiç kimsenin yüreği kaldırmazdı; ve onların çocukları, emirlere karşı gelmenin sonucunun ağır olduğunu bilmiyorlardı. Âdem ve Havva’nın bu doğumları şu an Edentia’dadır; bizler, kendileri ile ilgili nelerin tasarlanmış olduğuna dair bilgiye sahip değiliz.
\vs p075 6:4 Bu kafile yolculuğuna devam etmek için hazırlanmış çok üzgün bir kervandı. Bu durumdan daha acı dolu bir şey ne olabilirdi ki! Bir dünyaya bu kadar yüksek umutlarla gelmek, bu kadar uğurlu biçimde kabul görmek, daha sonra Cennet Bahçesi’nden utançla ayrılmak, ve üstüne üstlük bir de, yeni bir ikamet yerleşkesi bile bulmadan önce çocuklarının dörtte üçünden fazlasını kaybetmek!
\usection{7.\bibnobreakspace Âdem ve Havva’nın Düzeylerinin Alçaltılması}
\vs p075 7:1 Cennet Bahçesi kervanı durdurulmuşken, Âdem ve Havva suçlarının niteliği hakkında bilgilendirilmiş ve gelecekleri ile aydınlatılmışlardı. Cebrail, kararı açıklamak için görünmüştü. Ve şu verilen karardı: Urantia’nın Gezegensel Âdem ve Havva’sının yükümlülüklerini yerine getirmediklerinin yargısına varılmıştır; onlar, bu yerleşim dünyasının yöneticileri olarak kendilerine emanet edilmiş görev anlaşmasına uymamışlardır.
\vs p075 7:2 Suçluluk duygusunun üzüntüsü duyarken Âdem ve Havva; Salvington’daki hâkimlerin “evren hükümetini aşağılamaya” dair kendilerine yönlendirilen tüm suçlamalardan aklanmalarına hükmetmiş oldukları duyurudan büyük mutluluk duymuşlardı. Onlara isyan suçundan suçlu bulunmamışlardı.
\vs p075 7:3 Cennet Bahçesi çiftine, kendilerini âlemin fani unsurlarının düzeyine indirmiş oldukları bildirilmişti; ve, gelecekleri hakkında bir kanıya varmak için dünya ırklarının geleceklerine bakarak, bundan böyle Urantia’nın erkek ve kadını gibi kendilerini değerlendirerek hareket etmelerinin zorunluluğu onlara iletildi.
\vs p075 7:4 Âdem ve Havva Jerusem’i terk etmeden uzun bir süre önce eğitmenleri, kutsal tasarımlardan yapılacak herhangi bir hayati ayrılığın yaratacağı sonuçlar hakkında onlara her şeyi bütünüyle izah etmişlerdi. Ben kişisel olarak ve sürekli bir biçimde, Urantia’ya gelmelerinden önce ve varışlarından sonra; gezegensel görevlerinin uygulanmasında kesin bir şekilde doğru yoldan ayrılık ile sonuçlanacak eylemin, mutlak ceza halinde fani bedene indirgenmenin kaçınılmaz sonu olacağı hususunda onları uyardım. Ancak evlatlığın maddi düzeyinin sahip olduğu ölümsüzlük niteliğine dair bir kavrayış, Âdem ve Havva’nın doğru yoldan ayrılışının getirdiği sonuçları açık bir biçimde anlamak için temel teşkil etmektedir.
\vs p075 7:5 1.\bibnobreakspace Âdem ve Havva, Jerusem’deki akranları gibi, Ruhaniyet’in akıl\hyp{}çekim döngüsü ile birlikte ussal birliktelik vasıtasıyla ölümsüzlük düzeyini sürdürmüştü. Bu hayati beslenme akılsal kopuş ile kesildikten sonra, yaratılmış mevcudiyetinin ruhsal düzeyinden bağımsız olarak ölümsüzlük niteliği kaybedilir. Fiziksel ayrışmayı takip eden fani düzey, Âdem ve Havva’nın ussal başarısızlığının kaçınılmaz sonucuydu.
\vs p075 7:6 2.\bibnobreakspace Urantia’nın maddi bedeni hüviyetinde aynı zamanda kişilikleştirilmiş olan bu dünyanın Maddi Erkek ve Kız Evladı, ilaveten bir çifte dolaşım sisteminin idaresine bağlıydı: bu sitemlerden biri fiziksel doğalarından, diğeri ise yaşam ağacının meyvesinde depolanan üstün\hyp{}enerjiden kaynaklanmaktaydı. Her durumda baş melek görevlisi, görevlerine riayet etmemenin düzey alçaltılmasıyla sonuçlanacağı ve bu enerji kaynağına olan erişimin bahse konu eylemlerinin hemen sonrasında kendilerinden mahrum bırakılacağı hususunda Âdem ve Havva’yı uyarmıştır.
\vs p075 7:7 Caligastia, Âdem ve Havva’yı tuzağa düşürmede başarılı olmuştur; ancak o, bu çifti evren hükümetine karşı açık bir isyana sürükleme amacını elde edememiştir. Onların yaptığı şey gerçekten kötülüktü; ancak onlar hiçbir zaman, doğruluğun aşağılanması gibi bir suç işlememişlerdi; buna ek olarak onlar, Kâinatın Yaratıcısı’nın ve onun Yaratan Evladı’nın adil idaresi karşısında bilinçli bir biçimde isyan yataklık etmemişlerdi.
\usection{8.\bibnobreakspace Sözde İnsanın Çöküşü}
\vs p075 8:1 Âdem ve Havva, maddi evlatlığın üstün seviyesinden fani insanın alt düzeyine düşmüştür. Ancak bu durum insanın çöküşü değildi. İnsan ırkları, Âdemsel başarısızlığın doğrudan sonuçlarına rağmen üst bir konuma çekilmişti. Her ne kadar Urantia insanlarına eflatun ırkını bahşetmenin kutsal tasarımı yanlış yönetilse de, fani ırklar Âdem ve onun soylarının Urantia ırklarına sağladıkları kısıtlı katkıdan çok büyük kazançlar elde etmişti.
\vs p075 8:2 “İnsanın çöküşü” hiçbir zaman yaşanmamıştır. İnsan ırkının tarihi, ilerleyici evrimden birisidir; ve Âdemsel bahşedilme, bir önceki biyolojik düzeyi üzerinden dünya ırklarını büyük ölçüde ilerletmiş halde bırakmıştır. Urantia’nın daha üstün ırk kolları şu an; Adon, Sangik, Nodit ve Âdem unsurları şekilde dört gibi sayıca fazla ayrı kaynaktan elde edilen kalıtım etkenlerini taşımaktadır.
\vs p075 8:3 Âdem, insan ırkının üstünde bir lanet sebebi olarak görülmemelidir. Kutsal tasarımı ilerletme görevinde başarısız olmasına, İlahiyat ile sahip olduğu anlaşmaya karşı gelmesine, o ve eşi yaratılmış düzeyine kesin bir biçimde indirilmesine rağmen, tüm bunlara rağmen, onların insan ırkına olan katkısı Urantia üzerinde medeniyetin ilerlemesi için çok şeyi gerçekleştirmiştir.
\vs p075 8:4 Dünyanız üzerindeki Âdemsel görevin sonuçlarını tahlil ederken, adalet gezegeninizin sahip olduğu koşulların görmezden gelinmemesini gerektirmektedir. Âdem, güzel eşi ile birlikte Jerusem’den bu karanlık ve kafası karışık gezegene ulaştırıldığında, neredeyse ümitsiz bir görev ile karşılaştı. Ancak onlar Melçizedekler ve onların birlikteliklerinin tavsiyelerini dinleselerdi ve \bibemph{daha sabırlı olsalardı}, nihai olarak başarıyı elde edeceklerdi. Ancak Havva, kişisel bağımsızlık ve gezegensel eylem özgürlüğünün sinsi ilanlarına kulak verdi. O, evlatlığın maddi düzeyine ait olan yaşam plazması üzerinde deneyde bulunmaya çekildi; böyle yaparak o, daha öncesinde Gezegensel Prens’in görevlilerine bir zamanlar verilmiş olan doğum varlıklarınınkiler ile bütünleşmiş olan Yaşam Taşıyıcıları’nın özgün tasarımlarından çıkmış o dönemin karmaşık düzeyi ile, bu yaşam görevinin vaktinden önce karışmasına izin vermiş oldu.
\vs p075 8:5 Cennet’e olan yükselişinizin bütünü içinde siz hiçbir zaman; kusursuzluk, daha fazla kusursuzluk ve en sonunda ebedi kusursuzluk yolunda gelişmek için, kısa yollar, kişisel icatlar veya diğer imkânlar ile kurulu ve kutsal tasarımı sabırsız bir biçimde atlamaya çalışmaktan hiçbir şey elde edemeyeceksiniz.
\vs p075 8:6 Sonuçta, Nebadon’un tümü içerisinde üzerinde bilgeliğin daha cesaret kırıcı bir biçimde kötü idare edildiği bir gezegen muhtemelen daha önce hiç olmamıştı. Ancak, evrimsel evrenlerin olayları içinde bu türden yanlış adamların ortaya çıkması şaşırtıcı değildir. Bizler çok devasa bir yaratımın birer parçasıyız; ve her şeyin kusursuzluk içinde çalışmıyor olması garip değildir; evrenimiz kusursuzluk içinde yaratılmamıştır. Kusursuzluk bizim ebedi gayemizdir, geldiğimiz köken değil.
\vs p075 8:7 Eğer bu evren mekanik bir evren olsaydı, İlk Büyük Kaynak ve Merkez sadece bir kuvvet olup aynı zamanda kişilik olmazdı; eğer yaratımın tümü değişmeyen enerji faaliyetleri tarafından belirlenen bir biçimde kesin yasaların üstünlüğünde fiziksel maddenin çok geniş bir birlikteliği olsaydı, bunun sonrasında kusursuzluk evren düzeyinin tamamlanmamış olmasından bağımsız olarak bile elde edilebilirdi. Böyle bir durumda hiçbir anlaşmazlık yaşanmazdı; hiçbir çatışma gerçekleşmezdi. Ancak göreceli kusurluluğa ve kusursuzluğa sahip olan evrim halindeki evrenimiz içinde bizler, anlaşmazlığın ve yanlış anlaşılmanın mümkün olmasından memnuniyet duymaktayız; çünkü böylelikle kişiliğin evren içindeki gerçekliği ve eylemi kendisini kanıtlamaktadır. Ve eğer bizim yaratımımız kişilik üstünlüğündeki bir mevcudiyetse, bunun sonrasında sizler kişilik kurtuluşu, gelişimi ve kazanımlarına dair imkânlarının mevcut olduğundan emin olabilirsiniz; bizler kişiliğin büyümesi, deneyimi ve serüveninden emin olabiliriz. Yalnızca mekanik veya yalnızca durağan haldeki kusursuz bir dünya yerine, içinde kişiliğin ve ilerleyişin olduğu evren ne de muhteşem bir evrendir!
\vs p075 8:8 [“Bahçe’nin {yüksek} meleksel sesi” olan Solonia tarafından sunulmuştur.]
