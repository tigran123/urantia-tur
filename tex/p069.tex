\upaper{69}{İlkel İnsan Toplulukları}
\vs p069 0:1 Duygusal olarak insan; mizahı, sanatı ve dini takdir edebilme yetisi içinde hayvan atalarını aşan bir düzeyde bulunmaktadır. Toplumsal olarak insan; alet yapıcı, iletişim kurucu ve kurum inşa edici bir varlık olarak üstünlüğünü sergilemektedir.
\vs p069 0:2 İnsan varlıkları uzun süre boyunca toplumsal birliktelikleri idare ettikleri zaman, bu türden birleşmeler her zaman kurumsallaşma ile neticelenen belirli etkinlik eğilimlerinin yaratılmasıyla sonuçlanmıştır. İnsanın sahip olduğu kurumların çoğu, aynı zamanda toplum güvenliğinin gelişmesine bir ölçüde katkıda bulunurken çalışmayı kolaylaştırıcı olmuştur.
\vs p069 0:3 Medeni insan; oluşturduğu kurumların kişiliği, istikrarı ve devamlılığından büyük gurur duymaktadır; ancak insan kurumlarının tümü başlı başına, tabular tarafından muhafaza edilirken ve din aracılığı ile saygınlaştırılırken geçmişin biriktirdiği örf ve adetlerden meydana gelmektedir. Bu türden miraslar gelenek haline gelmekte ve gelenekler ise nihai olarak ortak toplum kabullerine dönüşmektedir.
\usection{1.\bibnobreakspace Temel İnsan Kurumları}
\vs p069 1:1 İnsan kurumlarının tümü; içinde kişiliğin gölgelendiği ve özgür birey girişiminin azaldığı aşırı büyümenin bireyin değerini kendisinden elinden alan etkisine rağmen, geçmişteki veya şimdiki bir takım toplumsal ihtiyaca hizmet vermektedir. İlerleyen medeniyetin bu oluşumlarının kendisi üzerinde üstünlük kurmasına izin vermesi yerine, insan kendi kurumlarını denetlemelidir.
\vs p069 1:2 İnsan kurumları üç sınıfta bulunmaktadır:
\vs p069 1:3 1.\bibnobreakspace \bibemph{Birey\hyp{}korunum kurumları}. Bu kurumlar, yiyecek ihtiyacından doğan uygulamalar ve onlar ile iniltili bireyin varlığını idame ettiriş içgüdülerini içine alır. Bu kurumlar; üretim, emlak, kazanç için savaş ve toplumun tüm idari işleyiş düzenlerini kapsar. Elinde sonunda korku içgüdüsü; tabular, gelenekler ve dinsel cezalandırma araçları vasıtasıyla hayat mücadelesinin bu kurumlarının oluşumunu desteklemektedir. Ancak korku, cehalet ve hurafelere olan inanç, insan kurumlarının tümünün öncül doğuşu ve daha sonraki gelişimi içinde dikkate değer bir rol oynamıştır.
\vs p069 1:4 2.\bibnobreakspace \bibemph{Birey\hyp{}çoğalımının kurumları}. Bu kurumlar; ırklar sahip olduğu cinsel açlık, annelik içgüdüsü ve daha yüksek sıcak duygulardan doğan toplumsal oluşumlardır. Onlar; aile yaşamı, eğitim, etik ve din olarak ev ve okulun toplumsal koruma düzenlerini içine alır. Bu kurumlar evlilik kabullerini, savunma için gerçekleştirilen savaşları ve ev inşasını kapsar.
\vs p069 1:5 3.\bibnobreakspace \bibemph{Birey\hyp{}tatmininin kurumları}. Bu kurumlar gösteriş eğilimleri ve gurur duygularından doğmuş olan uygulamalardır; ve onlar kıyafet ve kişisel süsleniş kabullerini, toplumsal adetleri, ihtişam için verilen savaşları, dansları, eğlenceleri, oyunları ve cinsel tatminin diğer fazlarını içine alır. Ancak medeniyet hiçbir zaman, birey\hyp{}tatmininin farklılaşmış kurumlarını yaratacak bir biçimde evirilmemiştir.
\vs p069 1:6 Toplumsal uygulamaların bu üç topluluğu, çok yakın bir biçimde birbirine bağlı ve belirli ilişkiler ağı içerisinde birbirlerine muhtaçtırlar. Urantia üzerinde onlar, tek bir toplumsal işleyiş biçimi olarak faaliyet gösteren karmaşık bir düzeni temsil etmektedirler.
\usection{2.\bibnobreakspace Üretimin Doğuşu}
\vs p069 2:1 İlkel üretim, kıtlığın yıkımına karşı bir teminat yaratacak düzeye yavaşça ilerleyerek gelmiştir. Tarihsel mevcudiyetinin ilk başlarında insan, kıtlık zamanlarına karşı bol bir hasat süresinde yiyecekleri saklamaları gerektiğine dair dersleri bazı hayvanlardan öğrenmeye başlamıştı.
\vs p069 2:2 Öncül tutumluluk ve ilkel üretimin doğuşundan önce olağan kabile yaşamı yoksulluk ve gerçek sıkıntıyla geçmekteydi. Yiyeceği için öncül insanın hayvan dünyasının bütünüyle rekabet etmesi gerekmekteydi. Rekabet\hyp{}çekimi, insanı her zaman hayvan düzeyine geri çekmektedir; açlık, insanın doğal ve gaddar sermayesidir. Servet, doğal bir hediye değildir; bu nitelik çalışma, bilgi ve düzenleme sonucunda açığa çıkmaktadır.
\vs p069 2:3 İlkel insan, birlikteliğin yararlarını anlamada geç kalmamıştır. Birliktelik düzenlemeyle sonuçlanmıştı; ve düzenlemenin ilk sonucu, zamandan ve kullanılan eşyalardan sağlanan doğrudan tasarruf ile birlikte iş bölümüydü. İşgücü üzerinden gerçekleşen bu özelleşmeler, direnç göstermenin azalışını tercih etme biçiminde baskıya olan uyumla ortaya çıkmıştır. İlkel yabansı bireyler hiçbir zaman gerçek bir görevi mutlulukla veya istençli bir biçimde yerine getirmediler. Toplumsal işbölümüne olan uyum onların durumunda ihtiyaçların baskısı sonucunda gerçekleşmişti.
\vs p069 2:4 İlkel insan ağır çalışmayı sevmemekteydi; ve bu insan, büyük bir tehlike ile karşılaşmadıkça hiçbir şey için çabukluk göstermezdi. Herhangi bir görev için belirli bir zaman sınırı koyma fikri olarak işgücü içerisinde zaman etkeni bütünüyle modern bir kavramdır. İlkel insanlar hiçbir zaman acelecilik sergilememişti. Öncül insanın doğası itibariyle eylemsiz ırkları üretim düzenine çeken şey, yoğun hayatta kalma mücadelesinin ve sürekli gelişen yaşam koşullarının çifte talebiydi.
\vs p069 2:5 Tasarım çabaları olarak işgücü insanı, uğraşları büyük ölçüde içgüdüsel olan hayvandan ayırmaktadır. İşgücüne olan ihtiyaç insanın sahip olduğu en yüksek lütuftur. Prens’in yönetim görevlilerinin tümü fiziksel bir biçimde çalışmıştı; onlar, Urantia üzerinde fiziksel işgücünün soylulaştırılmasına fazlasıyla katkıda bulunmuştu. Âdem bir bahçıvandı; her şeyin yaratanı ve koruyucusu olarak İbraniler’in Tanrı’sı çalışmıştı. İbraniler, üretime çok yüksek bir mükâfat koyan ilk kavimdi; onlar, “çalışmayan yememeli” şeklinde emri koyan ilk insan topluluğuydu. Ancak dünya dinlerinin çoğu, eylemsizliğin öncül nihai amacına geri dönmüştü. Jüpiter bir eğlence düşkünüydü; Buda ise, boş zamanda irdeleyici düşünmenin bir tutkunu haline gelmişti.
\vs p069 2:6 Sangik ırkları, sıcak iklim ormanlarından uzak bir konumda ikamet ettiklerinde oldukça üretimsel topluluklardı. Ancak orada, sihrin tembel tutkunları ile ileriyi düşünenler olarak çalışmanın öncüleri arasında çok uzun yıllar süren bir mücadele mevcuttu.
\vs p069 2:7 İlk insan öngörüsü; ateş, su ve yiyeceğin muhafazası yönüne çevrilmişti. Ancak ilkel insan doğuştan bir kumarbazdı; bu insan her zaman, hiçbir şey vermeden bir şeyler elde etmeyi arzulamıştı; ve sabırlı uygulamalar sonucu doğan başarı, bu öncül zamanlar boyunca fazlasıyla sıklıkla gerçekleşen bir biçimde büyü ile ilişkilendirilmekteydi. Sihrin yerini öngörüye, öz\hyp{}denetime ve üretime bırakması yavaş bir şekilde gerçekleşmişti.
\usection{3.\bibnobreakspace İşgücünün Farklılaşması}
\vs p069 3:1 İlkel toplum içerisinde işlerin bölümlere ayrılışı, ilk olarak doğal ve sonra toplumsal şartlar tarafından belirlenmiştir. İşgücü içinde farklılaşmanın öncül düzeni şuydu:
\vs p069 3:2 1.\bibemph{ Cinsiyet temelli farklılaşma}. Kadının görevi, çocuğun belirleyici mevcudiyeti tarafından belirlenmişti; kadınlar özü itibariyle çocuklarını erkeklerden daha fazla sevmektedir. Böylelikle kadın düzenli çalışan işçi haline gelirken, erkek iş ve istirahatın belirgin farklılaşmasının gözlendiği dönemler boyunca çalışarak avcı ve savaşçı haline gelmiştir.
\vs p069 3:3 En başından başlayarak bugüne kadar çağlar boyunca tabular, kendi yaşam alanı içinde kadını katı bir biçimde tutma yönünde işlemiştir. Erkek kendisi için en uygun görevi olabilecek en bencil biçimde seçmiş ve geride kalan gündelik angarya işleri kadına vermiştir. Erkek her zaman kadının işini yapmaktan utanç duymuştur; ancak kadın, erkeğin görevini yapmada herhangi bir isteksizliği hiçbir zaman göstermemiştir. Ancak garip bir biçimde, erkek ve kadınlar evin inşasında ve onun döşenmesinde her zaman beraber çalışmışlardır.
\vs p069 3:4 2.\bibnobreakspace \bibemph{Belirli Çağa ve hastalık şartlarına göre değişiklik}. Bu türden farklılıklar, işgücünün bir sonraki bölünmesini belirlemiştir. Yaşlı insanlar ve kötürümler ilk olarak alet ve silahların üretilme işine aktarılmıştı. Bu insanlar daha sonra, sulama yapılarının inşası için görevlendirilmişlerdi.
\vs p069 3:5 3.\bibnobreakspace \bibemph{Din temelli farklılaşma}. Sağlıkçılar, fiziksel uğraşlardan muaf tutulan ilk insan varlıkları olmuştu; onlar, öncü uzman sınıf üyeleriydiler. Demir ustaları, büyücüler gibi sağlıkçılar ile rekabet eden küçük bir topluluktu. Bu ustaların metaller ile çalışma yeteneği, insanları onlardan korkar bir hale getirmişti. “Beyaz demir ustaları” ve “siyah demir ustaları, beyaz ve siyah büyüye dair ilk inançlara kaynaklık etmiştir. Ve bu inanç daha sonra, iyi veya kötü ruhaniyetler olarak iyi ve kötü hayaletlere dair hurafeyi içine alır bir hale gelmiştir.
\vs p069 3:6 Demir ustaları, özel ayrıcalıkları memnuniyetle deneyimleyen ilk din\hyp{}dışı topluluktu. Onlar savaş zamanında tarafsız bireyler olarak görülmüştü; ve bu ilave boş zaman, ilkel toplumun siyasetçileri olarak onların bir sınıf haline gelmesine yol açmıştı. Ancak bu ayrıcalıkları ciddi bir biçimde kötüye kullanmaları sonucunda demir ustaları herkes tarafından nefret edilen bir konuma gelmişti; ve sağlıkçılar, rekabet içinde bulunduğu insanlara olan nefreti körüklemede hiç vakit kaybetmediler. Bilim ve din arasındaki ilk mücadelede (hurafeler olarak) din galip gelmiştir. Köylerinden atıldıklarından sonra demir ustaları, yerleşkelerin uç bölgelerinde kamuya açık pansiyonlar olarak ilk konakları idare ettiler.
\vs p069 3:7 4.\bibnobreakspace \bibemph{Efendi ve köle}. İşgücünün bir sonraki farklılaşması, galip ve mağlup ilişkilerinden doğmuştur; ve farklılaşma insan köleliğinin başlangıcı anlamına gelmiştir.
\vs p069 3:8 5.\bibnobreakspace \bibemph{Çeşitli fiziksel ve akılsal kazanımlar temelli farklılaşma}. İşgücünün bir sonraki düzeyde gerçekleşen farklılaşması, insanlar arasındaki içkin farklılıklar temelinde gerçekleşmiştir; insan varlıkların hiçbiri eşit doğmamaktadır.
\vs p069 3:9 Üretimde ilk uzmanlar, çakmaktaşı yontucuları ve taş ustalarıydı; bunların hemen ardından demir ustaları gelmekteydi. Bu özelleşmeleri takiben topluluk uzmanlaşması gelişmiştir; aile ve kavimlerin tamamı, işgücünün belirli kollarına kendilerini adamıştı. Kabilesel sağlıkçılardan ayrı olarak din adamlarının ilk tabakalaşmalarından biri kaynağını, uzman kılıç ustalarının bir ailesinin hurafelerin önemini yüceltmesinden almıştı.
\vs p069 3:10 Üretimdeki ilk topluluk uzmanları, kaya tuzu tüccarları ve çömlekçiler olmuştu. Kadınlar basit çömlekleri yaparken, erkekler onların süslü olanlarını üretmekteydi. Bazı kabileler arasında dikiş işleri ve dokumacılık kadınlar tarafından gerçekleştirilirken, diğerlerinde ise bu iki el işi erkekler tarafından yapılmaktaydı.
\vs p069 3:11 İlk tüccarlar kadınlardı; onlar, ticareti ek iş biçiminde yerine getirerek casuslar olarak kullanılmaktalardı. Kısa bir süre sonra ticaret genişledi ve kadınlar toptancılar olarak aracı tüccarlar biçiminde faaliyet gösterdi. Bu gelişmenin ardından, kar için hizmetleri karşılığında komisyon alan tüccar sınıfı açığa çıktı. Topluluk takasının büyümesi para temelli alım satımın gerçekleştiği düzene doğru gelişti; ve ürün alım satımını, kalifiye işgücünün değiş tokuşu takip etti.
\usection{4.\bibnobreakspace Ticaretin İlk Adımları}
\vs p069 4:1 Zor kullanma ile gerçekleşen evliliği sözleşme ile yapılan evliliğin izleyişi gibi, yağma ile gerçekleşen ürün gaspının yerini takas ticareti aldı. Ancak sessiz takasın öncül uygulamaları ile daha sonraki çağdaş alım satım yöntemleri arasında korsanlığın uzun yıllar süren bir dönemi etkin oldu.
\vs p069 4:2 İlk takas, tarafsız bir bölgeye ürünlerini bırakan silahlı tüccarlar tarafından gerçekleştirilmişti. İlk pazarları kadınlar düzenledi; onlar ilk tüccarlar olup, bu görev onlara ağır sorumluluk taşıyıcıları oldukları için verilmişti; erkekler savaşçılardı. Ticaretin daha ilk başlarında, tüccarların birbirlerine silah çıkarmalarını önleyecek kadar geniş bir duvar biçimde, alım satım tezgâhları gelişmişti.
\vs p069 4:3 Sessiz takas için ürünlerin bir araya geldikleri yerlerde koruma için bekleyen bir korkuluk kullanılmıştır. Bu türden pazar yerleşkeleri, hırsızlığa karşı güvenli bölgelerdi; takas veya alım sonucunda gerçekleşen ürün değişimi dışında bu pazarlardan hiçbir ürün dışarıya çıkmamaktaydı; hazır bekleyen bir korkulukla bu ürünler her zaman güvendeydi. İlk tüccarlar kabileleri içinde yaptığı anlaşmalarda dürüst olmaya özen göstermektelerdi; ancak onlar, uzakta yaşayan yabancıları kandırmayı olağan karşılamaktalardı. Öncül İbraniler bile, kendi dinlerinin mensubu olmayan insanlar ile olan ilişkilerinde ayrı bir etik düzenini tanımışlardı.
\vs p069 4:4 Çağlar boyunca sessiz takas, insanların kutsal ticaret pazarlarına silahsız olarak gelip buluşmasına kadar devam etti. Bahse konu bu pazar merkezleri sığınak olarak kullanılan yerleşkeler haline geldi; bazı ülkelerde bu yerleşimler daha sonra “mülteci şehirleri” olarak adlandırılmaktaydı. Pazarın olduğu yerleşkeler güvenli ve saldırılara karşı korunaklıydı.
\vs p069 4:5 İlk ağırlık birimleri, buğday ve diğer hububat taneleriydi. Alışverişin ilk ortak aracı bir balık veya bir keçiydi. Daha sonra inek takasın bir birimi haline gelmişti.
\vs p069 4:6 Mevcut anda kullanılan yazı, öncül ticaret kayıtlarında doğmuştur; insanın ilk edebiyatı, bir tuz reklamı biçiminde ticaret ile ilgili bir belgeydi. Savaşların ilkinin birçoğu; çakmaktaşı, tuz ve metaller gibi doğal kaynaklar için verilmişti. İlk resmi kabile anlaşması bir tuz kaynağının kabilelerin ortak kullanımına açılışı ile ilgiliydi. Bu gibi anlaşmaların yapıldığı yerler, düşüncelerin dostça ve barışçıl değiş tokuşunun ve çeşitli kabilelerin birbirlerine karışmasının olanağını sağladı.
\vs p069 4:7 Yazı; düğümlenen şeritler, resim yazıcılığı, hiyeroglifler ve boncuk kemerleri biçimindeki “ileti çubuğu” aşamalarından simgesel alfabenin ilk örneklerine kadar ilerledi. İleti gönderimi; ilkel dumandan başlayarak kurye koşucuları, hayvan binicileri, demir yolları ve uçaklara ek olarak telgraf, telefon ve kablosuz iletişime kadar uzanan bir kapsamda evirilmiştir.
\vs p069 4:8 Yeni düşünceler ve daha iyi yöntemler, eski tüccarlar tarafından yerleşik bir ikamet bölgesi etrafında uygulanmıştır. Serüven arzusu ile birlikte alım satım, araştırma ve keşfi beraberinde getirmiştir. Ticaret en başından beri, kültürün karşılıklı etkileşimini sağlayarak çok etkili uygarlaştırıcı olmuştur.
\usection{5.\bibnobreakspace Sermayenin İlk Adımları}
\vs p069 5:1 Sermaye, gelecek için şimdiki zamanın bir tasarrufunda uygulanan emektir. Tasarruf, yaşam idamesi ve hayatta kalma güvencesinin bir türünü temsil eder. Yiyeceklerin saklanması birey\hyp{}denetimini geliştirmiş olup, sermaye ve işgücünün ilk sorunlarını yaratmıştır. Yiyeceğe sahip olan insan, eğer hırsızlardan sahip olduğu ürünleri koruyabilirse, yiyeceği olmayan insanlar üzerinde bariz bir üstünlüğe sahipti.
\vs p069 5:2 İlk bankacı, kabilenin cesur bireyiydi. Bu insan birikmiş topluluk hazinelerini bir arada tutarken, kabilenin tamamı bu üyenin barakasını saldırı anında korurdu. Böylelikle bireysel sermaye ve topluluk malvarlığının birikimi doğrudan bir biçimde askeri düzenlemeye yol açtı. İlk başta bu türden önlemler yabancı yağmacılara karşı özel mülkleri korumak için tasarlanmıştı, ancak daha sonra komşu kabilelerin emlaklarına ve diğer malvarlıklarına gerçekleştirilecek saldırıları başlatmak için askeri oluşumu tutma adet haline gelmişti.
\vs p069 5:3 Sermayenin birikimine yol açan üç temel etken şunlardı:
\vs p069 5:4 1.\bibnobreakspace \bibemph{Öngörü ile ilişkili açlık}. Yiyecek tasarrufu ve onun muhafazası, gelecek ihtiyaçları sağlamak amacıyla belli bir yönde yeterli \bibemph{öngörüye} sahip olanlar için güç ve huzur anlamına gelmekteydi. Yiyecek biriktirme, kıtlık ve felakete karşı yeterli ölçekte güvence sağlamaktaydı. Ve ilkel örf ve adetlerin bütüncül oluşumu gerçekten de, şimdiki zamanı geleceğe bağımlı kılmaya yardım etmek için tasarlanmıştı.
\vs p069 5:5 2.\bibnobreakspace \bibemph{Aile sevgisi} --- aile isteklerini yerine getirme arzusu. Sermaye, bugünün taleplerinin baskısına rağmen geleceğin ihtiyaçlarını sağlamayı güvence altına almak için varlık tasarrufunu temsil etmektedir. Gelecek ihtiyacın bir kısmı, bir bireyin gelecek nesilleri ile ilgili olabilir.
\vs p069 5:6 3.\bibnobreakspace \bibemph{Gösteriş} --- bir bireyin varlık birikimlerini gösterme arzusu. İlave kıyafet, ayırt ediciliğin ilk ölçütlerinden biriydi. Eşya toplayıcılığına ait gösteriş öncül bir biçimde insan gururunu okşadı.
\vs p069 5:7 4.\bibnobreakspace \bibemph{Makam }--- toplumsal ve siyasi saygınlığı satın alma hevesi. Orada; yönetim idaresine karşı birtakım özel hizmet gösterimine bağlı olarak kabul edilen veya açık bir biçimde parayla verilen, ticari kimlik kazanmış bir soyluluk sınıfı türemişti.
\vs p069 5:8 5.\bibnobreakspace \bibemph{Güç} --- üstün olma arzusu. Değerli eşyaları borç olarak verme, bu ilkel zamanlarda yılda yüzde yüz kredi faiz oranına sahip olarak, bir köleleştirme aracı olarak uygulanmaktaydı. Tefeciler, kendilerine borçlu olanlardan daimi bir ordu yaratarak kendilerini kral yapmışlardır. Borçlu olarak hizmet veren işçiler, biriktirilebilecek ilk malvarlığı türlerinden biri olmuştur; ve eski zamanlarda borç köleliği, ölümden sonra bedenin hâkimiyetine bile uzanmıştır.
\vs p069 5:9 6.\bibnobreakspace \bibemph{Ölünün sahip olduğu hayaletlerden korku} --- koruma için din adamları ücreti. İnsanlar öncül olarak, öteki yaşamda ilerleyişlerini gerçekleştirmek için malvarlıklarının kullanılacağına dair bir inanç ile din adamlarına ölüm hediyeleri vermeye başladılar. Din mensuplarının zümresi böylelikle oldukça zengin bir hale geldi; onlar, eski sermaye sahiplerinin en başta gelenlerinden biriydi.
\vs p069 5:10 7.\bibnobreakspace \bibemph{Cinsel ilişki dürtüsü} --- bir veya daha fazla eşi satın alma arzusu. İnsanın ilk ticaret türü kadın alım satımıydı; bu ticaret uzun süreler boyunca at ticaretinden önce varlığını sürdürmüştü. Ancak cinsel ilişki kölelerinde yapılan takas toplumun ilerlemesine hiçbir zaman katkıda bulunmamıştır; bu türden bir değiş tokuş geçmişte, şimdiki olduğu gibi, ırksal bir rezaletti çünkü bu ticaret eş zamanlı olarak aile gelişimini sekteye uğratmış ve üstün ırkların biyolojik zindeliğine zarar vermişti.
\vs p069 5:11 8.\bibnobreakspace \bibemph{Birey\hyp{}tatmininin sayısız türü}. Bazı insanlar gücü elde etmeyi sağladığı için mülkiyete sahip olmanın yollarını aramıştır; diğerleri ise kolaylık sağladığı için malvarlığı mücadelesinde bulunmuştur. Öncül insan (ve daha sonrakileri) kaynaklarını şatafat için harcama eğilimini göstermiştir. İçkiler ve uyuşturucular ilkel ırkların ilgisini çekmiştir.
\vs p069 5:12 Medeniyet geliştikçe insanlar tasarruf için yeni ereklere sahip oldular; yeni istekler, doğuştan gelen yiyecek açlığının yanında hızlı bir biçimde yer etti. Açlık öyle bir şekilde hor görülmekteydi ki, yalnızca zengin insanların öldükleri zaman cennete doğrudan gidebilecek olduklarına inanılmaktaydı. Özel mülkiyet o kadar yüksek bir biçimde değerli görülen bir hale gelmişti ki, gösterişli bir ziyafet vermek insanın lekelenmiş ismini temizleyebilirdi.
\vs p069 5:13 Servet birikimi öncül bir biçimde, toplumsal farklılaşmanın temel özelliği haline geldi. Belirli kabileler içindeki bireyler, bir tatil gününde tamamını yakarak veya kabile üyelerine karşılıksız dağıtarak sadece etkili bir iz yaratma amacıyla yıllarca malvarlıklarını biriktirmekteydi. Bu türden davranışlar onları büyük insanlar haline getirmişti. Çağdaş insanlar bile Noel hediyelerinin şatafatlı dağıtımından keyif alırken, zengin insanlar iyilikseverlik ve öğrenim oluşturulmuş büyük kurumları yaratmak için bağışta bulunmaktadırlar. İnsanların yöntemleri değişkenlik göstermektedir, fakat onların bu yöndeki eğilimleri neredeyse hiçbir değişikliğe uğramadan varlığını devam ettirmektedir.
\vs p069 5:14 Ancak eski zamanların zengin insanlarının servetlerinin büyük bir kısmını varlıklarına göz dikmiş insanlar tarafından öldürülme korkusuyla dağıtmış olduğunun belirtilmesi açık bir biçimde ortaya konulmalıdır. Varlıklı insanlar genellikle, servetlerini önemsemedikleri göstermek için dikkate değer sayıdaki kölelerini kurban etmişlerdi.
\vs p069 5:15 Her ne kadar malvarlığı en başından beri insanı özgürleştirme eğilimi gösterse de, insanın toplumsal ve üretim düzenini büyük ölçüde karmaşıklaştırmıştır. Sermayenin adaletsiz sermayedarlar tarafından kötüye kullanımı, sermayenin çağdaş sanayi toplumunun temeli olduğu gerçeğini ortadan kaldırmamaktadır. Sermaye ve icatlar vasıtasıyla mevcut nesil, dünya üzerinde şimdiye kadar yaşamış nesillerin hepsinden daha yüksek bir özgürlüğü memnuniyetle deneyimlemektedir. Bu gözlem bir gerçek olarak belirtilmekte olup, düşüncesiz ve bencil sorumlular tarafından gerçekleşen sermayenin birçok yanlış kullanımını haklı çıkarmamaktadır.
\usection{6.\bibnobreakspace Medeniyet ile ilgili olarak Ateş}
\vs p069 6:1 İlkel toplum --- üretimsel, idaresel, dinsel ve askeri olarak --- dört bölümü ile birlikte ateşin, hayvanların, kölelerin ve özel mülkiyetin kullanılması ile yükselmişti.
\vs p069 6:2 Tek bir ilerleme aracı olarak ateş yakma insanı hayvandan sonsuza kadar ayırmıştır; ateş temel insan icadı veya keşfidir. Ateş, hayvanların tümünün korkmuş olduğu gece vakti yerde kalmayı insanın gerçekleştirmesini mümkün hale getirmiştir. Ateş, akşamüzeri gerçekleştirilen toplumsal etkileşimi desteklemiştir; ateş insanları yalnızca soğuktan ve vahşi hayvanlardan korumamış, onlar tarafından aynı zamanda hayaletlere karşı bir güvenlik sağlayıcı olarak kullanılmıştır. İlk başta ateş, ısıdan çok ışık için kullanılmıştır; geri kalmış birçok kabile, tüm gece boyunca yanan bir alev parıltısı olmadan uyumayı reddetmiştir.
\vs p069 6:3 Ateş; kendisini mahrumiyet içerisinde bırakmadan insanların yanan kömürleri komşularına vermesini mümkün hale getirerek, hiçbir şey kaybettirmeden toplumsal fedakârlığın ilk araçlarını onlara sunan bir biçimde büyük bir uygarlaştırıcıydı. Evin annesi veya en büyük kızı tarafından idare edilen ev ateşi, dikkat ve güvenilirlik gerektiren bir biçimde ilk eğitmendi. Öncül ev, ailenin kalbi olarak aile üyelerinin etrafında toplandığı ateşten oluşmaktaydı, bir binadan değil. Bir evlat yeni bir ev kurduğu zaman, ailenin kalbinden aldığı bir meşaleyi taşımaktaydı.
\vs p069 6:4 Her ne kadar ateşi keşfetmiş Andon onu bir ibadet aracı olarak görmekten kaçınmış olsa da, kendisinin soyundan gelen birçok unsur aleve tapınacak bir nesne veya ruhaniyet olarak itibar etti. Onlar ateşin sağlık yararlarından faydalanmada başarısız oldular, çünkü onlar çöpleri yakmamaktalardı. İlkel insan ateşten kormuş, onu her zaman hoş tutmaya çalışmış böylece tütsüler yakmıştır. Hiçbir koşul altında ilkel insanlar ateşe tükürmezlerdi; buna ek olarak onlar, yanan bir ateş ile onun karşısında duran kişi arasından geçmezlerdi. Kıvılcım çıkarmak için kullanılan demir piritleri ve çakmaktaşları bile, öncül insan varlıkları tarafından kutsal bir biçimde muhafaza edilmekteydi.
\vs p069 6:5 Bir alevi söndürmek günahtı; eğer bir baraka alev alırsa, onun yanmasına izin verilirdi. Tapınakların ve mabetlerin ateşi kutsal olup, bir takım afetlerden sonra veya her yıl tekrarlanan bir biçimde yeni ateşin yakılmasının adet olduğu uygulamalar dışında bu yapıların ateşlerinin sönmesine hiçbir zaman izin verilmemişti. Kadınlar din mensupları olarak seçilmekteydi, çünkü onlar ev ateşlerinin koruyucularıydılar.
\vs p069 6:6 Ev ateşinin tanrılardan inmiş olduğuna dair öncül efsaneler, ateşin ışık yayması sonucu gerçekleşen gözlemlerden doğmuştur. Ateşin doğaüstü kaynağına dair fikirler doğrudan bir biçimde ateşe yapılan ibadet ile sonuçlanmıştır; ve ateş ibadeti, Musa’nın zamanına kadar gerçekleştirilen bir uygulama olarak “ateşin içinden geçme” âdetine yol açmıştır. Ve orada hala, ölüm sonrasında ateşten geçme düşüncesi varlığını sürdürmektedir. Ateş efsanesi; öncül zamanlarda büyük bir birleştirici niteliğinde bulunmuş olup, Farslı insanların simgesel ifadelerinde varlığını hala devam ettirmektedir.
\vs p069 6:7 Ateş, yemekleri pişirmeye yol açmış olup; “yemekleri çiğ yiyenler” alay etmede kullanılan bir terim haline gelmişti. Ve yiyecekleri pişirme; besinlerin sindiriminde gerekli olan hayati enerji kullanım miktarını azaltmış olup, toplumsal kültürleri üzerinde yoğunlaşmaları için onların bedenlerine kuvvet kattı; bunun karşısında, yiyecekleri sağlamak için gösterilecek olan çabanın hayvanların evcilleştirilmesi vasıtasıyla azalmasıyla, evcilleştirme toplumsal etkinlikler için harcanacak zamanı yarattı.
\vs p069 6:8 Ateşin, metal yapım işlerinin kapılarını açtığı ve bunun sonrasında buhar gücünün keşfine ek olarak elektriğin bugünkü kullanımlarına yol açtığı unutulmamalıdır.
\usection{7.\bibnobreakspace Hayvanların Kullanılması}
\vs p069 7:1 İlk olarak, hayvan dünyasının tamamının insanın düşmanı konumunda değerlendirilmiş olduğu ifade edilmelidir; insan varlıkları, kendilerini vahşi hayvanlardan korumayı öğrenmek zorundalardı. İlk başta insanlar hayvanlarından beslenmişlerdi, ancak daha sonra insanlar hayvanları evcileştirmeyi ve onların kendilerine hizmet vermelerini sağlamayı öğrenmişti.
\vs p069 7:2 Hayvanların evcilleştirilmesi neredeyse şans eseri gerçekleşmişti. Yabansı insanlar, Amerikalı Kızılderililerin yaban öküzlerini avlayışına benzer bir biçimde hayvan sürülerini avlamaktalardı. Hayvan sürüsünün etrafını çevreleyerek onları denetim altına alıp, böylelikle gerektiği zaman onları öldürmeye yetkin hale gelmişlerdi. Daha sonra ağıllar inşa edilmiş ve sürülerin hepsi ulaşılabilir hale gelmişti.
\vs p069 7:3 Birtakım hayvanları evcilleştirmek kolaydı; ancak fil gibi birçok tür esaret altında doğum vermemekteydi. Daha ileri zamanlarda hayvanların belirli türlerinin insan mevcudiyetine girmeyi kabul ettikleri ve esaret alında doğum verdikleri keşfedildi. Hayvanların evcilleştirilmesi böylelikle, Dalamatia zamanlarından beri oldukça büyük ilerleme sağlanmış olan bir sanat biçiminde seçimsel üreme vasıtasıyla sağlanmıştı.
\vs p069 7:4 Köpek evcilleştirilen ilk hayvan olmuştu; bir avcıyı bütün gün boyunca takip eden belirli bir köpeğin nihayeten onun evine akşam geri dönüşüyle birlikte evcilleştirmenin zorlu deneyimini başlamıştı. Çağlar boyunca köpekler; yiyecek, avcılık, taşıma ve dostluk amacıyla kullanılmıştı. İlk başta köpekler sadece ulumaktaydı, ancak daha sonra onlar havlamayı öğrendiler. Köpeğin keskin burnu, onların ruhani unsurları görebildiğine dair bir ancın doğmasına neden oldu; ve böylelikle tapınılacak köpek inanışları doğmuştu. Bekçi köpeklerinin kullanılması kavimin tamamının gece uyuyabilmesini ilk kez mümkün kılmıştı. Bunun sonrasında evleri ruhani unsurlara ek olarak maddi düşmanlara karşı korumak amacıyla bekçi köpekleri kullanmak adet haline gelmişti. Bir köpek havladığı zaman insan veya hayvanın yaklaşmakta olduğuna inanılmaktaydı; ancak aynı köpek uluduğu zaman, ruhani unsurların yakında olduğu düşünülmekteydi. Şimdi bile birçok insan, gece vakti bir köpek ulumasının ölümün işareti olduğuna hala inanmaktadır.
\vs p069 7:5 Erkek bir avcı olduğunda, kadınlara oldukça iyi davranmaktaydı; ancak hayvanların evcilleştirilmesinden sonra, Caligastia kargaşalığının da etkisiyle, birçok kabile kadınlarına utanç verici bir biçimde davranmıştı. Onlar kadınlarını, tıpkı hayvanlarına davrandıkları gibi, bir fazlalık olarak görmekteydiler. Erkeğin bu dönemdeki kadına zalimce davranışı, insanlık tarihinin en karanlık dönemlerinden bir tanesini oluşturmaktadır.
\usection{8.\bibnobreakspace Medeniyetin Oluşumunda bir Etken olarak Kölelik}
\vs p069 8:1 İlkel insan, akranlarını köleleştirmede hiçbir zaman çekince göstermemiştir. Kadın, bir aile kölesi olarak açığa çıkan ilk köleydi. Kırsal yaşam erkeği, kendisinin alt düzeyde bulunan cinsel ilişki eşi olarak kadını köleleştirmişti. Bu türden cinsel ilişki köleliği doğrudan bir biçimde, erkeğin kadına olan bağımlılığının azalmasından kaynaklanmıştı.
\vs p069 8:2 Yakın bir zaman önce kölelik, savaş galibinin dinini kabul etmeyen asker tutukluların kaderiydi. Daha önceki zamanlarda mahkûmlar; ya yenilmekte, ölene kadar işkenceye uğramakta, birbirleriyle dövüştürülmekte, ruhaniyetlere kurban verilmekte ya da köleleştirilmekteydi. Kölelik, katliam ve insan yeme karşısında büyük bir gelişmeydi.
\vs p069 8:3 Kölelik, savaş eserlerine olan bağışlayıcı tutum içerisinde ileri bir aşamaydı. Erkekler, kadınlar ve çocukların tamamının katliamı ile sonuçlanan Ai’nin tuzağı sonrası savaş galibinin gösteriş arzusunu tatmin etmek için kurtarılan tek kral, sözde medeniyetleşmiş insanlar tarafından bile uygulanan vahşi insan katliamının şüphe götürmez bir resmidir. Bashan kralı tarafından Og’a gerçekleştirilen saldırı, bahsi geçmiş olay karşısında eşit derecede korkunç ve büyüktü. İbraniler düşmanlarını, tüm özel eşyalarını ganimet olarak toplayarak düşmanlarını “tamamen yok etmişti.” Onlar, “erkeklerin hepsinin kökünün kazılması” korkusunu yayarak şehirlerinin hepsini zorunlu vergiye bağlamıştı. Ancak daha az kabile bencilliğine sahip olan çağdaş kabilelerin çoğu uzun bir süreden beri üstün esirlerin topluma kazandırılma uygulamasına başlamış haldeydi.
\vs p069 8:4 Amerikalı kırmızı insanlar gibi avcı bireyler, köleleştirme uygulamasını gerçekleştirmemiştir. Bu insanlar, esirleri ya topluluklarının arasına katmış ya da onları öldürmüştür. Kölelik, kırsal yaşama uyum sağlamış insanlar arasında yaygın bir uygulama değildi; çünkü onlar çok az işgücüne ihtiyaç duymaktalardı. Savaş zamanı sürü sahipleri, erkek esirlerin hepsini öldürme ancak yalnızca kadın ve çocukları köleleştirme gibi bir uygulamayı takip etmişlerdi. Musevi yasa, bu kadın savaş esirlerinin ev hanımları yapılmasına dair özel yönergeleri taşımıştı. Şayet eğer evlilikte başarısız olurlarsa bu esirler gönderilebilmekteydi; ancak --- en azından medeniyette bir ilerleme olarak --- İbraniler’in bu türden reddedilmiş eşleri köle biçiminde satmalarına izin verilmemişti. Her ne kadar İbraniler’in toplumsal ortak kabulleri olgunlaşmamış bir düzeyde bulunsa da, bu ortak kabuller çevre kabilelerinkinden oldukça yüksek bir seviyedeydi.
\vs p069 8:5 Sürü sahipleri ilk sermayedarlardı; onların sürüleri sermayeyi temsil etmiş, --- sürülerin doğal yollardan çoğalımı biçiminde --- sermaye üzerinden elde edilen faiz ile yaşamışlardı. Ve onlar, kölelerin veya kadınların bu servetin bakım işiyle ilgilenmelerine güvenme meyli göstermemişlerdir. Ancak daha sonra onlar erkek mahkûmları ayırmış ve onları toprağı ekmeye zorlamışlardır. Bu uygulama, insanın toprağa bağlılığı biçiminde serfliğin öncül kökenidir. Afrikalı insanların toprağı işlemesini öğretmek oldukça kolaydı; bu nedenle onlar büyük bir köle ırkı haline geldi.
\vs p069 8:6 Kölelik, insan medeniyet zinciri içerisinde hayati öneme sahip bir halkaydı. Kölelik, toplumun kargaşa ve tembellikten düzen ve medeni etkinliklere geçtiği bir köprüydü; kölelik, geride kalmış ve tembel insanları çalışmaya itip ve böylece onların üst bireylerinin toplumsal gelişimi için refahı ve boş zamanı yaratmıştır.
\vs p069 8:7 Kölelik kurumu insanı, ilkel toplumun düzenleyici işleyiş biçimlerini icat etmeye itmiştir; bu kurum, hükümetin ilk oluşumlarının meydana gelmesine kaynaklık etmiştir. Kölelik; güçlü bir düzenleyişi gerektirmekte olup, Avrupa’nın Orta Çağları boyunca derebeylerinin köleleri denetleyememesi sonucunda neredeyse tamamen ortadan kalmıştır. İlkel çağların gelişmemiş kabileleri, bugünün Avustralya yerlileri gibi, hiçbir zaman kölelere sahip olmamıştır.
\vs p069 8:8 Bu dönemde köleliğin baskıcı olduğu doğrudur; ancak köleliğin okullarında insanlar üretimi öğrenmişlerdir. Nihai olarak köleler, yaratmaya oldukça gönülsüz bir biçimde katkıda bulundukları yüksek bir toplumun güzelliklerini paylaştılar. Kölelik, kültür ve toplumsal kazanımın bir düzenini yaratmaktadır; ancak bu yaratımdan sonra yakın bir zaman içerisinde kölelik, toplumsal hastalıkların tümü içinde en ağırı olarak topluma içeriden saldırmaktadır.
\vs p069 8:9 Makine alanındaki çağdaş icatlar, köleliği gerici kılan değişimi beraberinde getirdi. Kölelik, çok eşlilik gibi, gerçekliğini kaybetmektedir; çünkü kölelikte emeğin maddi bir takdiri sunulmamaktadır. Ancak köleliğin çok sayıdaki unsurunu ansızın bir biçimde özgür bırakmanın zarar verici olduğu her zaman kanıtlanmıştır; kölelerin kademeli bir biçimde özgürleştirilmesi sağlandığında daha az kargaşa çıkmaktadır.
\vs p069 8:10 Bugün insanlar toplumsal köleler değillerdir; ancak insanların binlercesi onları borç ve yükümlülük altına sokup köleleştirmeyi arzulamaktadır. Bilinçsiz yapılan kölelik, dönüşüme uğraşım üretim uşaklığının yeni ve gelişmiş bir türüyle sonuçlanmıştır.
\vs p069 8:11 Toplumun nihai amacı evrensel özgürlük iken, tembellik hiçbir zaman hoş görülmemelidir. Yetkin bedene sahip olan bireylerin tümü, en azından yaşamlarını idame ettirecek düzeyde bir işi yapmaya zorlanmalıdırlar.
\vs p069 8:12 Çağdaş toplum geriye gitmektedir. Kölelik neredeyse tamamen ortadan kalmıştır; evcilleştirilmiş hayvanlar yok olmaktadır. Medeniyet gücü elde etmek için cansız dünya olarak ateşe geri dönmektedir. İnsan yabansı düzeyinden ateş, hayvanlar ve kölelik aracılığıyla yükselmişti; ancak insanlar bugün, kölelerin yardımı ve hayvanların desteğini bir kenara iterek geldiği konuma geri dönmektedirler; bunun karşısında insanlar, doğanın temel hazinelerinden zenginlik ve gücün yeni sırları ve kaynaklarını zorla elde etmeye çabalamaktadır.
\usection{9.\bibnobreakspace Özel Mülkiyet}
\vs p069 9:1 İlkel toplum neredeyse tamamen ortak paylaşımcı bir halk niteliğine sahipken, ilkel insan komünizmin çağdaş savları uyarınca yaşamamıştı. Bu öncül zamanların paylaşımcı toplumu ne yalın bir kuram, ne de toplumsal öğretiydi; bu toplum, basit ve işlevsel nitelikte kendiliğinden gerçekleşen uyum sonucunda açığa çıkmış bir gelişimdi. Bu dönemin paylaşımcı toplumu yoksulluk ve yoksunluğu önledi; dilencilik ve fuhuş, bu ilkel toplumlar arasında bilinmemekteydi.
\vs p069 9:2 İlkel paylaşımcı toplum bilinç dâhilinde, insanları aynı düzeye indirgeyerek onları eşit bir konuma getirmedi; buna ek olarak o, yetersiz durumlara olan tamahkârlığı yüceltmedi; ancak bu toplum eylemsizlik ve tembelliği mükâfatlandırıp, üretimin gelişmesini engellemeye ek olarak geleceğe dair taşınan umut dolu amaçları yok etti. Paylaşımcı toplum, ilkel toplumun yükselmesinde hayati derecede öneme sahip bir iskeleydi; ancak bu toplum daha yüksek bir toplum düzenin evrimleşmesine yol açtı, çünkü bu paylaşımcı toplum yapısı insanın şu dört güçlü eğilimine karşı gelmekteydi:
\vs p069 9:3 1.\bibnobreakspace \bibemph{Aile}. İnsan sadece malvarlığını arttırmayı arzulamamaktadır; insan sahip olduğu malları nesillerine bırakmayı derinden istemektedir. Ancak öncül paylaşımcı toplumlarda bir insanın sahip olduğu sermaye ölümü üzerine, ya doğrudan bir biçimde harcandı ya da topluluk üyeleri arasında paylaştırıldı. Orada, --- miras vergisinin yüzde yüz olduğu bir biçimde --- özel mülkiyetin mirası bulunmamaktaydı. Daha sonraki sermaye birikimi ve özel mülkiyetin mirasına dair gelenekler, farklılaşmış bir toplumsal ilerlemeydi. Ve bu durum, daha sonra sermayenin kötüye kullanılmasından doğan büyük istismarlara rağmen gerçektir.
\vs p069 9:4 2.\bibnobreakspace \bibemph{Dinsel eğilimler}. İlkel insan aynı zamanda, bir sonraki dünyada yaşamına başlamak için özel mülkiyetini bir çekirdek olarak kurmayı arzuladı. Bu amaç, bir insanın yanında kişisel eşyalarının neden gömüldüğüne dair uzun yıllar süren âdeti açıklamaktadır. İlkel insanlar, sadece zengin insanların doğrudan zevk ve sefaya ek olarak saygınlık içinde yaşamdan sonra varlıklarını devam ettirdiklerine inanmaktalardı. Özellikle Hıristiyan öğretmenler olarak, açığa çıkarılmış dinin eğitmenleri; fakirlerin zenginler ile bir eşit düzeyde kurtuluşa erişebileceklerini duyuran ilk bireylerdi.
\vs p069 9:5 3.\bibemph{ Özgürlük ve boş zamana sahip olma arzusu}. Toplumsal evrimin daha önceki zamanlarında toplum arasında birey kazanımlarının paylaştırılması neredeyse bir kölelik düzeyindeydi; çalışanlar, tembeller için köleleştirilmişti. İsraf eden bireylerin sürekli olarak tutumluların üzerinden geçinmesi paylaşımcı toplumun kendisini ortadan kaldırışıyla sonuçlanacak bir zaaftı. Bugünkü zamanlarda bile savurgan insanlar, (tutumlu olup vergisini ödeyen bireyler biçiminde) kendilerine bakılması için devlete muhtaçlık duymaktadırlar. Sermayeye sahip olmayan insanlar hala diğerlerinden kendilerini doyurmalarını beklemektedirler.
\vs p069 9:6 4.\bibnobreakspace \bibemph{Güvenlik ve gücü elde etme dürtüsü}. Paylaşımcı toplumsal düzen nihai olarak, kabilelerinin beceriksiz olan tembel insanlarına yaptıkları kölelikten bir kaçış uğraşı içinde çeşitli hilelere başvurmak zorunda kalmış ilerleyici ve başarılı bireylerin aldatıcı uygulamalarıyla yıkılmıştır. Ancak ilk başta kişisel eşya biriktirme uygulamalarının tümü gizli bir biçimde yapılmaktaydı; ilkel dönemlerde yaygın güvenlik yoksunluğu, malvarlıklarının görülebilen yerlerde birikmesini engelledi. Ve daha sonraki dönemde bile, sermayenin çok büyük bir miktarını yığmak en tehlikeli uygulamaydı: kral kesin bir biçimde, birtakım suçlamalarla bahaneler yaratarak zengin bir insanın servetine el koyardı; ve varlıklı bir insan öldüğünde, bir miras vergisi olarak ailenin toplum refahına veya krala büyük bir meblağı bağışladığı vakte kadar cenaze töreni bekletilirdi.
\vs p069 9:7 İlk zamanlarda kadınlar paylaşımcı toplumun ortak mülkiyetiydi; ve anne aile kurumunda baskın bir konumdaydı. Öncül toplum önderleri; tüm toprakların sahibi olmuş, ve böylelikle kadınların hepsinin iyeleri haline gelmişlerdir; evlilik, kabile yöneticisinin rızasını gerektirmekteydi. Paylaşımcı toplumun dağılmasıyla birlikte kadınlar bireysel olarak değerlendirilmeye başlamıştı; ve baba kademeli olarak aile denetimini üstlenmişti. Böylelikle ev kurumu öncül oluşumuna başlamıştı; ve varlığını sürdüregelmiş çokeşlilik geleneklerinin yerini kademeli olarak tekeşlilik almıştı. (Çokeşlilik, evlilikte kadının köle olduğu durumun uzantısıdır. Tekeşlilik ise; ev kurulumu, çocuk yetiştirilimi, karşılıklı kültür ve birey gelişiminin seçkin oluşumu içinde bir erkek ve bir kadının benzersiz birlikteliğinin köleliğe dayanmayan en yüksek amacıdır.)
\vs p069 9:8 İlk başta kullanılan araç ve gereçlere ek olarak silahları içine alan bir biçimde özel eşyaların tümü, kabilenin ortak malıydı. Özel eşyalar ilk olarak, kişisel biçimde dokunulan şeylerin tümünü kapsamaktaydı. Eğer bir yabancı bir bardaktan içki içerse, bundan böyle o bardak onun olurdu. Daha sonra, kanın aktığı yerlerin tümü yaralanan kişi veya toplulukların özel mülkiyeti haline gelmiştir.
\vs p069 9:9 Özel mülkiyet böylelikle en başından beri saygı duyulan bir konumdaydı, çünkü bu türden malvarlıklarının sahibinin kişiliğini taşıdığı varsayılmaktaydı. Özel mülkiyete dair dürüstlük, bu türden bir hurafe üzerine güvenilir bir biçimde dayanmaktaydı; bir kişinin sahip olduğu eşyalarını korumak için herhangi bir polise ihtiyaç duyulmamaktaydı. Her ne kadar insanlar diğer kabilelerin mallarını gasp etmede çekince göstermese de, topluluk arasında hırsızlık bulunmamaktaydı. Özel mülkiyet ilişkileri ölümle sona ermemekteydi; ilk başta ölünün sahip olduğu kişisel eşyalar yakılmış, sonra onlarla bir gömülmüş ve daha sonrasında ise hayatta kalan aile veya kabile tarafından miras yoluyla devralınmıştır.
\vs p069 9:10 Kişisel süs eşyaları, büyü gücüne sahip olanlarının giyilmesinden kaynağını almaktadır. Hayalet korkusuna ek olarak gösteriş, özel mülkiyet eşyalarının ihtiyaçlardan daha değerli görülmesi gibi bireyin gözdesi olan sihirli eşyalardan kendisini kurtarmayı amaçlayan her girişime karşı onun direnişine yol açmıştır.
\vs p069 9:11 Bireyin uyuduğu mekân, insanın sahip olduğu ilk özel mülkiyetlerden biriydi. Daha sonra ev yerleşkeleri, gayrimenkullerin tamamını topluluk için elinde barındıran kabile önderleri tarafından yeni sahiplerine verilmekteydi. Bu dönemi takip eden süreçte ateşin bulunduğu bir yerleşke özel mülkiyet kapsamına girmişti; daha sonra ise bir kuyu, komşu toprak parçasına bağlı bir emlak düzeyine kavuştu.
\vs p069 9:12 Su birikintileri ve havuzlar, özel mülkiyet kapsamına giren ilk emlaklar arasında yer teşkil etmiştir. Bütün korkuluk uygulamaları; su birikintilerini, kuyuları, ağaçları, ekinleri ve balları korumak için kullanılmıştı. Korkuluklara olan inancın azalması ile birlikte özel eşyaları korumak için yasalar gelişmişti. Avlanma hakkı biçiminde oyun yasaları, toprak ile ilgili olan kanunlardan önce uzun süreler boyunca var olmuştu. Amerikalı kızıl derililer, toprağın özel mülkiyetini hiçbir zaman anlamamışlardı; bu insanlar, beyaz insanın toprak ile ilgili görüşünü kavrayamamıştı.
\vs p069 9:13 Özel mülkiyet öncül bir biçimde aile nişanı tarafından işaretlenmişti; ve bu uygulama, aile simgelerinin ilk kaynağını teşkil etmiştir. Gayrimenkul aynı zamanda ruhaniyetlerin gözetimine de bırakılabilmekteydi. Din mensupları bir toprak parçasını “kutsayabilir”, böylece buralara inşa edilen sihirli tabuların koruması altında bu emlaklar ikametlerine devam edebilirlerdi. Bu yerleşkelerin sahiplerinin bir “din adamının mülkiyet icazetini” taşımakta olduğu söylenirdi. İbraniler, bu aile emlaklarının taşıdığı sınır taşlarına hâlihazırda büyük bir saygı beslemektelerdi. “Komşusunun sınır taşına dokunanlar lanetlenmelidir” ifadesi bu bağlamda dile getirilmişti. Bu taş işaretleri, ilgili din mensubunun baş harflerini taşımaktaydı. Ağaçlar bile baş harfler ile kazındığında özel mülkiyet düzeyine kavuşmaktaydı.
\vs p069 9:14 İlk zamanlarda yalnızca ekinler özel mülkiyet düzeyindeydi, ancak sürekli ekin veren bitkiler bu nitelikte değerlendirilmekteydi; tarım bu nedenle toprağın özel mülkiyetinin kökeniydi. Bireylere ilk başta yalnızca bir yaşam boyu sahip olacakları mülkiyet hakkı verilmişti; bireyin ölümü halinde onun sahip olduğu toprak kabileye geri dönmekteydi. Kabileler tarafından bireylere verilen ilk toprak mülkiyetleri, ailenin toprağa gömme uygulaması biçiminde, mezarlardı. Daha sonraki zamanlarda toprak onu çitleyenin olmuştu. Ancak şehirler, ortak mera alanları olarak ve savaş zamanlarında kullanılmaları için belirli toprak arazilerini ayırmıştı; bu “ortak alanlar”, toplumsal mülkiyetin ilk türünün varlığını devam ettirmekte olduğunu göstermektedir.
\vs p069 9:15 Nihai olarak devlet, vergilendirme hakkını saklı tutan bir biçimde bireye özel mülkiyet sağlamıştı. Bireylerin sahip oldukları emlaklara dair hakların korunmasıyla emlak sahipleri kiracılarından kira toplayabilmişti; ve toprak, sermaye olarak --- bir gelir kaynağı haline gelmişti. En sonunda toprak parçası; satışlar, devirler, uzun vadeli kredili alımlar ve hacizler ile birlikte üzerinden ticari anlaşmaların yapılabildiği hale gelmişti.
\vs p069 9:16 Özel mülkiyet, artan özgürlüğü ve gelişmiş istikrarı beraberinde getirdi; ancak toprağın özel mülkiyeti sadece, paylaşımcı toplum denetimi ve idaresinin yerine getirilmediği hallerde toplumsal yaptırıma uğramaktaydı; ve bu uygulamayı daha sonra kölelerin, serflerin ve toprağa sahip olmayan toplum sınıfları izlemişti. Ancak tarımda artan makine kullanımı insanları kademeli olarak köleleştirici toprak uğraşlarından özgürleştirmektedir.
\vs p069 9:17 Özel mülkiyet hakkı mutlak değildir; bu hak tamamiyle toplumsaldır. Çağdaş insanlar tarafından memnuniyetle deneyimlenen hükümet kurumları, yasalar, düzenler, vatandaşlık hakları, toplumsal bağımsızlıklar, toplumsal kabuller, barışlar ve mutlulukların tümü özel mülkiyet sahipliği etrafında gelişme göstermiştir.
\vs p069 9:18 Bugünün toplumsal düzeninin --- ilahi veya kutsal biçimde --- mutlak olarak gelmesi gereken yerde bulunduğu yargısına varılamaz; ancak insanlık, bu düzen içinde yapılacak değişikliklerde yavaşça hareket etmede başarılı olacaktır. Sizin sahip olduğunuz toplumsal işleyiş yapısı, atalarınız tarafından bilinen her düzenden çok daha iyi bir düzeyde bulunmaktadır. Toplumsal düzeni değiştirdiğinizde bu değişikliğin daha iyi bir düzeni sağlamak için yapıldığından emin olun. Atalarınızın gözden çıkardığı düzeltme reçetelerini tekrar denemeye razı gelmeyin. İleri doğru gidin, geri değil! Evrimin ilerleyişine destek olun! Geri bir adım atmayın.
\vs p069 9:19 [Nebadon’un bir Melçizedek unsuru tarafından sunulmuştur.]
