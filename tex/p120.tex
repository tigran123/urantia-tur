\upaper{120}{Urantia üzerinde Mikâil’in Bahşedilişi}
\vs p120 0:1 Urantia üzerinde ve fani bedenin suretinde bulunduğu haldeki Mikâil’in sahip olduğu yaşamım tekrar aktarımını denetlemek için Cebrail tarafından görevlendirilmiş bir konumdaki, bu görevle emanet edilen açığa çıkarış heyetinin Melçizedek yöneticisi olarak, ben; Yaratan Evlat’ın, evren bahşediliş deneyiminin sonlandırıcı aşamasına adım atmak için Urantia’ya olan varışının hemen öncesindeki belirli olayların bu anlatımını sunmak için görevlendirmiş durumdayım. Tıpkı kendi yaratımının ussal varlıklarına emrettiği bu türden yaşamların aynısını yaşamak, böylece yaratılmış varlıklara ait sahip olduğu çeşitli düzeylerin suretinde kendisini bahşetmek; her Yaratan Evlat’ın, nesnelerden ve varlıklardan oluşan bireysel olarak yarattığı evrenin bütüncül ve yüce egemenliğini elde etmesi için ödemek zorunda olduğu hesabın bir parçasıdır.
\vs p120 0:2 Birazdan irdeleyeceğin olaylardan önce, Nebadon’un Mikâili kendisini; kendisinin çeşitli yaratımı olan ussal varlıklara ait altı farklı düzeyin suretinde altı kez bahşetmişti. Bunun sonrasında, o; kâinat âlemlerinin tümünün kutsal Cennet Yöneticileri’ne ait emirler uyarınca, evren egemenliğini elde etme piyesinin son sahnesini yerine getirmek için, maddi âlemin bir insanı biçiminde, kendisinin ussal irade yaratımlarının en alt düzeyi olarak fani bedeninin suretinde, Urantia’ya inmeye hazırlanmıştı.
\vs p120 0:3 Bu önceki bahşedilmelerin her birinin süreci boyunca, Mikâil; yalnızca, kendi yaratmış olduğu varlıklarının tek bir topluluğunun sınırlı deneyimini elde etmemişti, aynı zamanda, bağımsız bir biçimde yaratmış olduğu evrenin egemeni olarak kendisini oluşturmaya, tek başına, ileri bir biçimde katkıda bulunacak nitelikteki Cennet eş\hyp{}güdümündeki olmazsa olmaz bir deneyimi de elde etti. Tüm geçmiş yerel evren zamanı boyunca herhangi bir an içerisinde, Mikâil; bir Yaratan Evlat olarak kişisel egemenliğini ilan edebilir, bir Yaratan Evlat olarak kendi tercih ettiği biçimde sahip olduğu evreni yönetebilirdi. Bu türden bir gelişimde, Emanuel ve onun birliktelik içinde bulunduğu Cennet Evlatları, bu evrenden ayrılırdı. Ancak, Mikâil; bir Yaratan Evlat olarak, yalnızca kendisi için tanınmış hak olduğu için Nebadon’u idare etmeyi arzu etmemişti. O; belirli bir zaman içinde Yüce Varlık’ın yüceltilmiş yönetiminin ayırt edici niteliği haline gelecek, kavrayışın ve bilgeliğin uygulanmasının kusursuzluğuyla kendi evrenini yönetmeye ve onun olaylarını idare etmeye yetkin hale geleceği yer olan evren düzeyindeki bu yüksek konuma, Cennet Kutsal Üçlemesi’ne olan yetkisel nitelikteki işbirliksel bağlılığının mevcut deneyimi boyunca yükselmeyi tercih etti. O; bir Yaratan Evlat olarak yönetimin kusursuzluğunu değil, Yüce Varlık’ın kâinat bilgeliği ve kutsal deneyiminin bütünlüksel temsili olarak idarenin yüceliğini amaçladı.
\vs p120 0:4 Mikâil, bu nedenle, sahip olduğu evren yaratılmışlarının çeşitli düzeyleri üzerindeki bu yedi bahşedilişi gerçekleştirmede çifte bir amaca sahipti: İlk olarak, o, bütüncül egemenliği üstlenmeden önce tüm Yaratan Evlatlar’dan istenen yaratılmış anlayışı içindeki gerekli deneyimi tamamlamaktaydı. Herhangi bir zaman zarfı içinde bir Yaratan Evlat, kendi hakkı doğrultusunda kendi evrenini yönetebilir; ancak, o, yalnızca yedi evren\hyp{}yaratılmış bahşedilişi aşamasından geçtikten sonra Cennet Kutsal Üçlemesi’nin yüce temsilcisi olarak yönetimini gerçekleştirebilir. İkinci olarak, o; bir yerel evrenin doğrudan ve kişisel idaresinde uygulanabilecek, Cennet Kutsal Üçlemesi’ne ait en yüksek yönetimi temsil etme ayrıcalığını amaçlamaktaydı. Bunun uyarınca, Mikâil; evren bahşedilişlerinin her birinin deneyimi boyunca, kendisini, Cennet Kutsal Üçlemesi’nin bireylerine ait çeşitli ilişkilemlerin farklı biçimlerde oluşturulmuş iradelerine kendisini başarılı ve uygun görülen bir biçimde gönüllü olarak tabi kılmıştı. Bu, birinci bahşedilişte, kendisinin Yaratıcı, Evlat ve Ruhaniyet’in bir araya gelmiş iradesine bağlı olduğu; ikinci bahşedilişte, Yaratıcı ve Evlat’ın iradesine bağlı olduğu; üçüncü bahşedilişte, Yaratıcı ve Ruhaniyet’in iradesine bağlı olduğu; dördüncü bahşedilişte, Evlat ve Ruhaniyet’in iradesine bağlı olduğu; beşinci bahşedilişte, Sınırsız Ruhaniyet’in iradesine bağlı olduğu; altıncı bahşedilişte, Ebedi Evlat’ın iradesine bağlı olduğu; ve, yedinci ve son bahşediliş boyunca, Urantia üzerinde, Kâinatın Yaratıcısı’nın iradesine bağlı olduğu anlamına gelmektedir.
\vs p120 0:5 Mikâil, bu nedenle; kendi yerel evren yaratılmışlarının duygudaş deneyimiyle birlikte kâinatsal Yaratanlar’ın yedi katmanlı fazlarına ait kutsal iradeyi, kişisel egemenliği için birleştirmektedir. Böylelikle, onun idaresi; her ne kadar tüm keyfi hakları geride bırakmış olsa da, olası en büyük güç ve yönetimin temsilcisi olan hale gelmiştir. Onun gücü, Cennet İlahiyatları ile olan deneyimlenmiş ilişkilemden kökenini aldığı için sınırsızdır; onun yönetim yetkisi, evren yaratılmışlarının suretindeki mevcut deneyim vasıtasıyla elde edildiği için sorgulanamaz niteliktedir; onun egemenliği yücedir, çünkü o, Cennet Kutsal Üçlemesi’nin yedi katmanlı bakış açısıyla zaman ve mekâna dair yaratılmışın bakış açısını aynı anda bünyesinde taşımaktadır.
\vs p120 0:6 Bu nihai bahşedilişin zamanına karar vermiş ve bu olağanüstü olayın üzerinde gerçekleşeceği gezegeni seçmiş olarak, Mikâil; Cebrail ile olağan bir bahşediliş\hyp{}öncesi görüş alış\hyp{}verinde bulunup, bunun sonrasında, Emanuel olarak büyük kardeşinin ve Cennet danışmanının huzuruna çıktı. Cebrail’e daha öncesinde verilmemiş olan evren idaresinin tüm güçleri, bu aşamada, Emanuel’in gözetimi altına verilmişti. Ve, Urantia vücutlaşımı için Mikâil’in ayrılışından hemen önce, Emanuel; Urantia bahşediliş süresi boyunca evrenin gözetimini kabule ederek, âlemin bir fanisi olarak Urantia üzerinde yakın bir zaman içinde büyüdüğünde Mikâil için vücutlaşım rehberi olarak hizmet verecek, bahşedilme önerisini aktarmaya başladı.
\vs p120 0:7 Bu iletişimde, Mikâil’in; Cennet Yaratıcısı’nın iradesine bağlı olarak, fani bedenin suretinde bu bahşedilişi yerine getirmeyi seçmiş konumda bulunduğu hatırlanmalıdır. Yaratan Evlat, evren egemenliğine erişmenin bu tek amacı için bahse konu vücutlaşımı yerine getirmek amacıyla hiç kimseden yönergeler almak zorunda değildi; ancak, o, Cennet İlahiyatları’nın çeşitli iradeleri ile eş güdümsel faaliyet göstermeyi içine alan Yüce’nin açığa çıkarılışının bir tasarımsal işleyiş sürecine girişmiş konumda bulunmaktaydı. Böylece, onun egemenliği, nihai ve kişisel olarak elde edildiği zaman; Yüce içinde sonuçlandığı haliyle, İlahiyat’ın yedi\hyp{}katmanlılığının tamamını kapsayan nitelikte olacaktı. O, bu nedenle, daha öncesinde, çeşitli Cennet İlahiyatları’nın ve onlara ait birlikteliklerin kişisel temsilcileri tarafından altı kez eğitilmiş konumda bulunmaktaydı; ve, bu aşamada o, Kâinatın Yaratıcısı’nın adına hareket etmekte olan, Nebadon yerel evrenindeki Cennet Kutsal Üçlemesi’nin büyükelçisi tarafından eğitilmişti.
\vs p120 0:8 Orada; bu sefer Kâinatın Yaratıcısı’nınkine gerçekleşen bir biçimde, bu kudretli Yaratan Evlat’ın Cennet İlahiyatları’nın iradesine kendisini bir kez daha gönüllü olarak tabi kılışının istekliliğinden doğan, derhal gerçekleşen faydalar ve devasa derecedeki telafiler bulunmaktaydı. Bu tür ilişkilenimsel tabi olmayı yerine getirmenin kararıyla, Mikâil, bu vücutlaşımda; yalnızca fani insanın doğasını değil, aynı zamanda, her şeyin Cennet Yaratıcısı’nın sahip olduğu iradeyi deneyimleyecekti. Ve, buna ek olarak, o; yalnızca, Urantia bahşedilişi sebebiyle görevinden uzak kaldığı süre boyunca sahip olduğu evreninin idaresinde Emanuel’in Cennet Yaratıcısı’nın bütüncül yönetim yetkisini uygulayacak oluşu gerçeğiyle değil, aynı zamanda, aşkın\hyp{}evrene ait Kâinatın Ataları’nın bütüncül bahşedilme süreci boyunca sahip olduğu âleminin güvenliğini emretmiş oluşunun yarattığı rahatlatıcı bilgi niteliğinde, bütüncül güvence ile birlikte bu benzersiz bahşedilmeye başlamıştı.
\vs p120 0:9 Ve, bu, Emanuel yedinci bahşediliş görevini sunduğundaki çok önemli gerçekleşimin koşullarıydı. Ve, Emanuel’in; takip eden bir biçimde Urantia’da Nasıralı İsa (Mesih İsa) haline gelen, evren yöneticisine gerçekleştirdiği bu bahşedilme\hyp{}öncesi görevlendirilişten, şu alıntılarını sunmaya izin verilmiş durumdayım:
\usection{1.\bibnobreakspace Yedinci Bahşediliş Görevlendirilişi}
\vs p120 1:1 “Benim Yaratan kardeşim, ben senin yedinci ve son bahşedilmeni gözlemlemek üzereyim. En sadık ve en kusursuz bir biçimde sen, bundan önceki altı görevlendirmeyi yerine getirdin; ve, ben, sonuçlandırıcı egemenlik bahşedilişin olarak, bunda aynı şekilde utgun olacağından başka hiçbir şey düşünmemekteyim. Buraya kadar sen, tercih ettiğin düzeyin tamamiyle gelişmiş bir varlığı olarak bahşediliş âlemlerinde ortaya çıktın. Şimdi sen; tamamiyle gelişmiş bir fani değil, ancak, yardıma muhtaç bir bebek halinde, tercihin olan bu düzeni bozulmuş ve gelişimi sekteye uğramış gezegen olarak Urantia üzerinde ortaya çıkmak üzeresin. Bu, yol arkadaşım, yeni ve daha öncesinde denenmemiş bir deneyim olacak. Sen; bahşedilişin bütüncül bedelini ödeyerek, bir yaratılmışın sureti içinde bir Yaratan’ın vücutlaşımına ait bütüncül aydınlanmayı deneyimlemek üzeresin.
\vs p120 1:2 “Bundan önceki bahşedilmelerin her biri boyunca, sen gönüllü olarak; kendini üç Cennet İlahiyatı’nın ve onların kutsal karşılıklı\hyp{}ilişkilemlerinin iradesine tabi kılmayı tercih ettin. Yüce’nin iradesine ait yedi faz içinde, sen, bundan önceki bahşedilmelerinde, kendi Cennet Yaratıcı’nın kişisel iradesi dışında hepsine tabi oldun. Şimdi, yedinci bahşedilmen boyunca Yaratıcı’nın iradesine tümüyle tabi olmayı seçmiş bulunarak, bizim Yaratıcımız’ın kişisel temsilcisi olarak ben; vücutlaşımın boyunca sahip olduğun evrenin koşulsuz yönetim yetkisini üstleniyorum.
\vs p120 1:3 “Urantia bahşedilmesi sürecine girerken, sen, kendini; böyle bir durumda sahip olduğun yaratımın herhangi bir yaratılmışı tarafından gerçekleştirebilecek bir biçimde, tüm gezegen\hyp{}ötesi destekten ve özel yardımdan gönüllü olarak mahrum kıldın. Nasıl senin yaratmış olduğun Nebadon evlatları sahip oldukları evren süreçleri boyunca senin güvenli davranışına tamamen bağlıysa, şimdi sen de, senin gerçekleşmekte olan fani sürecinin açığa çıkarılmamış iniş\hyp{}çıkışları boyunca güvenli davranış için Cennet Yaratıcısı’na tamamiyle ve koşulsuz olarak bağlı hale gelmekte zorundasın. Ve, sen; bu bahşedilme deneyimini tamamladığında, tüm yaratılmışlarından, onların yerel evren Yaratanı ve Yaratıcısı olarak seninle gerçekleştirdikleri gönülden ilişkilerinin bir parçası olarak hiç değişmeksizin şart koştuğun bu inanç\hyp{}güveninin bütüncül anlamını ve zengin önemini tüm gerçekliğiyle bilmiş olacaksın.
\vs p120 1:4 “Urantia bahşedilmen boyunca, sen, sen ve senin Cennet Yaratıcın arasında kesintisiz birliktelik olarak sadece tek bir şeyle ilgilenmeye ihtiyaç duyacaksın; ve, bu türden bir ilişkinin kusursuzlaşmasıyla, bahşediliş dünyan, hatta yaratımın tüm evreni, her şeyin Kâinatsal Yaratıcısı olarak senin Yaratıcın ve benim Yaratıcım’ın yeni ve daha anlaşılabilir bir açığa çıkarılışına gözlerini çevirecekler. Senin ilgin bu nedenle yalnızca, Urantia üzerindeki kişisel yaşamınla ilgili olacak. Gönüllü gerçekleştirdiğin yönetim yetkisini bırakış anından Evren Egemeni olarak bizlere geri dönüşüne kadar, ben, sahip olduğun evrenin güvenliğinden ve kesintisiz devam eden idaresinden bütünüyle ve etkili bir biçimde sorumlu olacağım; ve, Cennet’in kabul edişi üzerine, sen, şu an teslim etmekte olduğun vekâlet yönetimini değil, evreninin yüce gücünü ve yönetim yetkisini benim ellerimden geri alacaksın.
\vs p120 1:5 “Ve, (benim, sözümü sadık bir biçimde yerine getireceğime dair tüm Cennet’in güvencesi olduğumu çok bilen bir biçimde) şu an söz vermekte olduğum şeylerin hepsini yapabilme gücüyle donatılmış olduğumu güvenceyle bilebiliyor olsan da, gönüllü bahşedilişinin süreci boyunca Nebadon içinde tüm ruhsal tehlikeyi önleyecek Uversa üzerindeki Zamanın Ataları’nın bir emrinin tarafıma daha yeni iletildiğini sana duyurmak isterim. Fani vücutlaşıma başlaman üzere, bilincini teslim ettiğin andan itibaren, senin kendi yaratımın ve düzenlemen olan bu evrenin yüce ve koşulsuz egemeni olarak bizlere geri dönenmene kadar, dışarıdan hiçbir ciddi etki Nebadon’un hiçbirinde gerçekleşemez. Vücutlaşımının bu ara döneminde, ben; bu bahşedilişin için sen görevinde bulunmazken, Nebadon evreni içinde isyan suçu işleyen veya isyan çıkarmayı tasarlayan her varlığın anlık ve kendiliğinden gerçekleşecek olan yok edilişlerini koşulsuz olarak emreden Zamanın Ataları’nın hükümlerine sahibim. Kardeşim, mevcudiyetim içindeki içkin Cennet yönetimi ve buna eklenmiş Uversa’nın yargısal emri göz önünde bulundurulduğunda, sahip olduğun evren ve onun tüm sadık yaratılmışları bahşedilmen boyunca güvende olacaklar. Görevine tek bir düşünce ile başlayabilirsin --- sahip olduğun evrenin ussal varlıkları için bizim Yaratıcımız’ın gelişmiş açığa çıkarılışı.
\vs p120 1:6 “Daha önceki bahşedilmelerinin her birinde olduğu gibi, kardeş\hyp{}emanetçisi olarak sahip olduğun evrenin yönetim yetkisi almış bulunduğumu sana hatırlatmak isterim. Ben, adına tüm yetkiyi kullanmakta ve tüm gücü elimde bulundurmaktayım. Ben; bizim Cennet Yaratımız nasıl yerine getirirse o şekilde faaliyet göstermekte, ve, açık talebin uyarınca bu şekilde sahip olduğun mevkide faaliyet göstermekteyim. Ve, gerçek böyle olduğu için, bu verilmiş yönetim yetkisinin tümü; geri verilişini uygun görmen üzerine her an uygulayabileceğin biçimde, yeniden senindir. Senin bahşedilişin, başından sonuna kadar, tamamiyle gönüllük üzerine gerçekleştirilmektedir. Âlemde bir fani vücutlaşımı olarak, göksel donatımlarından yoksun bir konumda bulunmaktasın; ancak, bıraktığın tüm güç, evren yönetim yetkisi ile kendini tekrar donatmayı tercih edeceğin her an, tekrardan elde edilebilir. Eğer, güç ve yönetimini yeniden ilan etmeyi tercih edersen, bunun tamamiyle \bibemph{kişisel} nedenlerden gerçekleşeceğini unutma; zira, ben, taşıdığı mevcudiyeti ve sözü senin Yaratıcı’nın iradesi uyarınca sahip olduğun evrenin güvenli idaresini teminat altına alan, yaşayan ve en yüksek güvenceyim. Nebadon içinde bu gibi üç kez gerçekleştiği biçiminde, isyan, bu bahşedilme için Salvington’da ayrı bulunduğun süreçte ortaya çıkamaz. Urantia bahşedilişinin süreci boyunca, Zamanın Ataları; Nebadon içindeki isyanın, kendi kendisini kendiliğinden yok eden tohumu beraberinde taşımasını emretmiştir.
\vs p120 1:7 “Bu son ve olağanüstü bahşedilme için görevinde bulunmadığın sürenin tamamı boyunca, ben (Cebrail’in işbirliği ile birlikte), sahip olduğun evrenin sadık yönetiminin sözünü veriyorum; ve, ben seni, kutsal açığa çıkarılışın bu hizmetinin sorumluluğunu üstlenmek ve kusursuzlaştırılmış insan anlayışının bu deneyim sürecinden geçmek için görevlendirirken, ben, benim Yaratıcım’ın ve senin Yaratıcı’nın adına hareket etmekte, ve, sana, beden içindeki devam eden konukluğunun kutsal görevi ile ilgili ilerleyen bir biçimde öz bilince varır hale gelirken, dünya hayatını yaşamanda seni yönlendirecek şu öneriyi sunmaktayım:
\usection{2.\bibnobreakspace Bahşediliş Kısıtlılıkları}
\vs p120 2:1 “1. Sonarington’un gelenekleri uyarınca ve onun işleyiş biçimine uygun olarak --- Cennet’in Ebedi Evladı’nın emirlerini yerine getiren bir biçimde --- ben; senin tarafından düşünülerek oluşturulmuş ve Cebrail tarafından benim korumama verilmiş olan tasarımlar ile uyumlu olarak bu fani bahşedilişe olan anlık girişin için her şeyi yerine getirmiş durumdayım. Sen; âlemin bir çocuğu olarak Urantia üzerinde büyüyecek, insan eğitimini tamamlayacak --- sürekli olarak senin Cennet Yaratıcı’nın iradesine tabi olacak bir halde --- belirlediğin şekilde Urantia üzerinde yaşayacak, gezegensel konukluğunu sonlandıracak, ve, sahip olduğun evrenin en yüksek egemenliğimi ondan almak amacıyla Yaratıcı’na yükselmeye hazırlanacaksın.
\vs p120 2:2 “2. Dünya görevinin ve evren açığa çıkarılışının dışında, ancak her ikisini de içine alabilecek şekilde, ben; kutsal kimliğine dair yerinde bir biçimde öz bilince sahip olduktan sonra, Satania sistemi içinde Lucifer isyanına tam olarak son vermenin ilave görevini üstlenmeni önermekteyim; ve, bu ise, tüm bunların hepsini \bibemph{İnsanın Evladı} olarak yerine getirmendir; böylece, âlemin bir fani yaratılmışı olarak, senin Yaratıcı’nın iradesine olan inanç\hyp{}bağlanışının sonucu olarak güçlünün zayıf halde geldiği konumda, ben, bu günahkâr ve temelsiz isyanın baş gösterişi döneminde fazlasıyla donanımda bulunduğun güç ve kudretle yerine getirmeyi kendi kararın uyarınca tekrar eden bir biçimde reddetmiş olduğun şeylerin hepsini, alçakgönüllülüğünle elde etmeni öneriyorum. Ben; eğer, İnsanın Evladı’na, Urantia’nın Gezegensel Prensi’ne ek olarak, sahip olduğun evrenin en yüksek egemeni biçiminde Tanrı’nın Evladı olarak bizlere geri dönersen, bunun fani bahşedilişinin yerinde bir doruk noktası olacağını düşünmekteyim. Nebadon içindeki ussal yaratılmışın en alt türü olarak fani bir insan; Caligastia ve Lucifer’in hakaretkâr iddialarıyla karşılaşacak ve onlar hakkında karar varacak, ve, yüklendiğin alçakgönüllü düzey içinde, ışığın bu devrik çocuklarının utanç dolu yanlış temsillerine sonsuza kadar son verecek. Sahip olduğun yaratan ayrıcalıklarını kullanarak bu isyankârları değersizleştirmeye oldukça kararlı bir biçimde karşı koymuş bir konum bulunmuşken, şimdi, kendi yaratımının en alçak düzeyindeki yaratılmışların suretinde, bu devrik Evlatlar’ın ellerinden egemenliği geri alman tam da yerinde olacaktır; ve, böylece sahip olduğun bütün yerel evren, tüm gerçekliğiyle açıkça ve sonsuza kadar, bağışlamanın isteksel yönetim gücü ile yapmada seni uyardığı bu şeyleri fani bedenin rolünde gerçekleştirmene dair adaleti tanıyacaktır. Senin bahşedilişinin, Nebadon’da Yüce’nin egemenliğine dair olasılığı bu şekilde var kılışıyla, sen gerçekte; bu kazanımın gerçekleşimde ne kadar az veya çok zamanın geçeceğinden bağımsız olarak, önceki tüm vücutlaşımların içerdiği sonuca varılmamış olaylara bir sonucu sağlamış olacaksın. Bu eylemle, sahip olduğun evrenin beklemekte olan anlaşmazlıkları, özü itibariyle sonlanacaktır. Ve, sahip olduğun evrene ait en yüksek egemenliğin takip eden bir biçimde sana verilişiyle, yönetim yetkine karşı benzer zorluklar hiçbir zaman, senin kişisel nitelikteki büyük yaratımının hiçbir kısmında ortaya çıkmayacaktır.
\vs p120 2:3 “3. Urantia dönemini sonlandırmada başarılı olduğunda, kuşkusuz sen bunu gerçekleştirecekken, ben sana, nihai bahşedilme deneyimine dair sahip olduğun evren tarafından ebedi tanınma niteliğindeki Cebrail tarafından sunulacak olan ‘Urantia’nın Gezegensel Prensi’ unvanını kabul etmeni öneriyorum; ve, Caligastia ihaneti ve onun sonrasında gerçekleşen Âdemsel başarısızlık tarafından Urantia’da ortaya çıkmış keder ve kafa karışıklığını telafi etmen için, bahşedilişinin amacıyla tutarlı nitelikte, ilave her bir şeyi yapmanı.
\vs p120 2:4 “4. Talebin doğrultusunda, Cebrail ve ilgili olan herkes; âlemin bir yazgı sonu yargısının duyuruluşuyla, bir çağın sonlanışının beraberinde gelişiyle, uyku halindeki fani kurtuluş unsurlarının yeniden dirilişiyle, ve, Gerçeklik’e ait bahşedilmiş Ruhaniyetin yazgı döneminin kuruluşuyla Urantia bahşedilişini sonlandırmak için ifade ettiğin arzun doğrultusunda işbirliğinde bulunacaklardır.
\vs p120 2:5 “5. Bahşedilmenin gezegeniyle ve fani konukluğun döneminde üzerinde yaşayan insanların mevcut bulunduğu nesli ile ilgili olarak, ben sana, büyük ölçüde bir öğretmen rolünde faaliyet göstermeni önermekteyim. İlk olarak ilgini, insanın ruhsal doğasının özgürleşimine ve ona ilham olmaya ver. Bunun sonrasında, karanlıkta kalmış insan usunu aydınlat, insanların ruhlarını iyileştir ve onların akıllarını çağlar kadar eski korkularından kurtar. Ve, bunun sonrasında, fani bilgeliğin uyarınca, beden içindeki kardeşlerinin fiziksel refahına ve maddi rahatlığına hizmet et. Sahip olduğun tüm evren için ilham kaynağı olmak ve onun eğitimini sağlamak amacıyla, ideal dini yaşamı yaşa.
\vs p120 2:6 “6. Bahşedildiğin gezegen üzerinde, isyan\hyp{}tarafından\hyp{}dışlanmış insanları ruhsal olarak özgür kıl. Urantia üzerinde, Yüce’nin egemenliğine ilave bir katkıda bulun, böylece sahip olduğun kişisel yaratıma ait geniş nüfuz alanları boyunca bu egemenliğin oluşumunu genişlet. Beden sureti içindeki maddi bahşedilmen olarak bu bütünlük içinde, senin Cennet Yaratıcın’ın iradesi ile birlikte insanın doğası içerisindeki faaliyette bulunmanın çifte deneyimi olarak, bir zaman\hyp{}mekân Yaratanı’nın nihai aydınlanışını deneyimlemek üzeresin. Zamansal yaşamın içinde, sınırlı yaratılmışın iradesi ve sınırsız Yaratan’ın iradesi; tıpkı onların aynı zamanda Yüce Varlık’ın evrimleşen İlahiyatı içinde bütünleşmekte olduğu gibi, bir tek haline gelecektir. Bahşediliş gezegeninin üzerine, Gerçekliğin Ruhaniyeti’ni akıt; ve böylece, bu tecrit edilmiş âlem üzerindeki tüm olağan fanileri, âlemlerin Düşünce Düzenleyicisi olarak Cennet Yaratıcısı’nın ayrışmış mevcudiyetinin hizmetine derhal ve bütünüyle ulaşabilir kıl.
\vs p120 2:7 “7. Bahşediliş dünyan üzerinde sergileyeceğin her şeyde, sahip olduğun evrenin tümünün eğitimi ve öğretimi için bir yaşamı yaşamakta olduğunu sürekli olarak aklında tut. Sen, Urantia üzerine olan fani vücutlaşımının bu yaşamını \bibemph{bahşetmektesin}; ancak sen, senin idari nüfuz alanına ait çok geniş gökadasının bir parçasını oluşturmuş, oluşturmakta ve ileride oluşturabilecek olan her yerleşik dünya üzerinde yaşamış, yaşamakta ve ileride yaşayabilecek olan her insan ve insan\hyp{}ötesi usun ruhsal ilhamı için bu türden bir yaşamı \bibemph{yaşamak üzeresin}. Fani beden sureti içindeki dünya yaşamın; dünyasal konukluğunun günlerinde Urantia’nın fanileri için, ya da, Urantia üzerindeki veya bir diğer dünya üzerindeki insan varlıklarının daha sonraki herhangi bir nesli için bir \bibemph{örnek} oluşturacak şekilde yaşanılmamalı. Bunun yerine, Urantia üzerinde beden içindeki yaşamın; gelecek çağlar içindeki nesillerin tümü boyunca, Nebadon dünyalarının hepsi üzerindeki yaşamların bütünü için\bibemph{ ilham kaynağı} olmalıdır.
\vs p120 2:8 “8. Fani vücutlaşımda gerçekleştirilmesi ve deneyimlenmesi gereken büyük görevin; senin Cennet Yaratıcın’ın iradesini gerçekleştirmek, böylece beden içinde ve özellikle bedenin yaratılmışları için Yaratıcın halindeki \bibemph{Tanrı’yı açığa çıkarmak}, için oldukça gönülden güdülenen bir yaşamı yaşamaya karar vermenden oluşmaktadır. Bununla birlikte sen aynı zamanda, yeni bir büyüleyişle, tüm Nebadon’un fani\hyp{}ötesi varlıkları için Yaratımız’ı \bibemph{yorumlayacaksın}. Aklın insan ve insan\hyp{}ötesi türü için Cennet Yaratıcısı’nın yeni bir açığa çıkarılışının ve derinleşmiş yorumunun bu hizmetiyle birlikte eşit bir biçimde, Tanrı için insanın yeni bir açığa çıkarılışını gerçekleştirecek şekilde faaliyet göstereceksin. Beden içindeki tek bir kısa yaşamın içinde, tüm Nebadon içinde şimdiye kadar hiçbir şekilde görülmemiş olan, fani mevcudiyetin kusa süreci boyunca bir Tanrı\hyp{}bilen insan tarafından elde edilebilecek aşkın olasılıkları sergile; ve, tüm Nebadon’un ve tüm zamanın insan\hyp{}ötesi uslarının hepsi için, insana ve onun gezegensel yaşamının iniş\hyp{}çıkışlarına dair yeni ve aydınlatıcı bir \bibemph{yorumunda} bulun. Sen, Urantia’ya fani bedenin suretinde inmektesin; ve, dönenim ve neslin içinde bir insan olarak yaşayarak, sen, geniş yaratımına ait olaylara olası en yüksek katılımda şu ideal ve kusursuzlaştırılmış yöntemi tüm evrenine gösterecek biçimde faaliyet göstereceksin: Tanrı’nın insanı buluşunun gerçekleşimi ve insanın Tanrı’yı arayışı ve onu buluşunun olgusu; ve, bunun tümünü karşılıklı tatmin için, hem de beden içindeki bir kısa yaşam süreci boyunca gerçekleştirmek.
\vs p120 2:9 “9. Gerçekte âlemin olağan bir insanı haline gelecek olurken, potansiyel olarak Cennet Yaratıcısı’nın bir Yaratan Evladı olarak mevcudiyetini korumaya devam edecek olmanı sürekli olarak aklında tutman konusunda seni uyarmak isterim. Bu vücutlaşım boyunca, her ne kadar sen, bir İnsan Evladı olarak yaşayıp hareket edecek olsan da; senin kişisel kutsallığının yaratıcı nitelikleri seni Salvington’dan Urantia’ya takip edecek. Düşünce Düzenleyici’nin varışının sonrasında vücutlaşımı herhangi bir an içinde sonlandırmak, sürekli olarak senin iradenin\hyp{}gücü içinde olacak. Düzenleyici’nin varışı ve alınışından önce, senin kişilik bütünlüğün için destekte bulunacağım. Ancak, Düzenleyici’nin varışını takiben, sen; yaratan ayrıcalıklarının, bu niteliklerin kişisel mevcudiyetinden ayrılamaz konumda olduğu için, fani kişiliğin ile ilişkilem içinde bulunmaya devam edeceği gerçeğini göz önünde bulundurarak, herhangi bir insan\hyp{}ötesi erişim\hyp{}iradesi, kazanım veya gücü oluşturmaktan sakınmalısın. Ancak, sen; bilinçli ve istek dâhilinde gerçekleşen iradenin bir eylemi tarafından, tüm\hyp{}kişilik tercihini sonlandıracak olan kesin bir karara varıncaya kadar, Cennet Yaratıcısı’nın iradesi dışında senin dünyasal sürecinde hiçbir insan\hyp{}ötesi oluşum eşlik etmeyecektir.
\usection{3.\bibnobreakspace İlave Öneri ve Tavsiye}
\vs p120 3:1 “Ve, şimdi, kardeşim, sen Urantia için ayrılmaya hazırlanırken sana elveda ederken, ve, bahşedilmenin genel davranışına dair sana öneride bulunduktan sonra, Cebrail ile fikir alış\hyp{}verişimizde ve senin fani yaşamının küçük çaplı fazları ile ilgili vardığımız belirli tavsiyeleri sunmama izin ver. Bizler, ilaveten şunları tavsiye etmekteyiz:
\vs p120 3:2 “1. Fani dünyasal yaşamının idealine olan arayışın içinde, sen aynı zamanda, akran insanların için kullanışlı ve ilk elden yardımcı nitelikteki bazı şeylerin yerine getirilişine ve örneklendirilişine belirli bir düzeyde ilgi göster.
\vs p120 3:3 “2. Aile ilişkileri ilgili olarak, bahşedilişine ait dönemde veya nesilde kurumsallaşmış olarak bulduğun gibi, aile yaşamının kabul edilmiş adetlerine öncelik ver. Aile ve cemiyet yaşamını, arasında ortaya çıkmayı seçtiğin insanların uygulamaları uyarınca yaşa.
\vs p120 3:4 “3. Toplumsal düzen ile ilişkilerinde, bizler; çabalarını fazlasıyla ruhsal yenilenim ve ussal özgürleşim için sınırlamanı tavsiye etmekteyiz. Gününün ekonomik yapısı ve siyasi bağlılıkları ile her türlü ilişkilemden kaçın. Bunun yerine kendini daha fazla, Urantia üzerindeki ideal dini yaşamı yaşamaya ada.
\vs p120 3:5 “4. Hiçbir koşulda ve hatta hiçbir önemsiz görünen detayda bile, Urantia ırklarının olağan ve düzen içinde seyreden ilerleyici evrimine müdahalede bulunmamalısın. Ancak, bu yasak; \bibemph{teşvik edici nitelikteki dini etik kurallarının} dayanıklı ve gelişmiş bir sitemini Urantia üzerinde geride bırakmana dair çabalarını sınırlar biçimde yorumlanmamalıdır. Bir yazgı\hyp{}dönem Evladı olarak sana, dünya insanlarının \bibemph{ruhsal} ve \bibemph{dini} düzeyini ilerletmek ile ilgili belirli ayrıcalıklar sağlanmıştır.
\vs p120 3:6 “5. Gerekli gördüğün takdirde, Urantia üzerinde bulabileceğin mevcut dini ve ruhsal hareketler ile kendini özdeşleştireceksin; ancak, olası her durumda, örgütlenmiş bir inancın, sabitleşmiş bir dinin veya fani varlıkların ayrışmış bir etik topluluğunun resmi oluşumundan kaçın. Senin yaşam ve öğretilerin, tüm dinlerin ve tüm insanların ortak mirası haline gelecektir.
\vs p120 3:7 “6. Urantia dini inanışlarının veya ilerleyici\hyp{}olmayan dini bağlılıklarına ait diğer türlerin ileride gerçekleşecek basmakalıplaşmış sistemlerinin yaratımına istemeden de olsa katkıda bulunmamanın önerisi hususunda şunların da ilave tavsiyesinde bulunmak istiyoruz: Gezegen üzerinde arkanda hiçbir yazı bırakma. Kalıcı maddeler üzerine gerçekleşecek her türlü yazımdan uzak dur; birlikteliklerinin, beden içindeki resimlerini veya başka benzerlerini yapmalarını yasakla. Ayrılış zamanında gezegen üzerinde potansiyel nitelikte putsal bir şey bırakmadığından emin ol.
\vs p120 3:8 “7. Her ne kadar sen, gezegen üzerindeki mevcut olağan ve ortalama toplumsal hayatı yaşayacak olsan da, erkek cinsinin olağan bir bireyi olarak sen muhtemelen, bahşedilmen bakamından tamamiyle onurlu ve onunla tutarlı nitelikteki bir ilişki olacak, evlilik ilişkisine girmeyeceksin; ancak, ben seni, Sonarington vücutlaşım emirlerinden bir tanesinin, Cennet kökenine ait bir bahşedilme Evladı’nın herhangi bir gezegen üzerinde arkasında insan doğumunu bırakışını yasaklamış olduğu hususunda hatırlatmak zorundayım.
\vs p120 3:9 “8. Yaklaşmakta olan bahşedilişine ait tüm diğer detaylarda, bizler; insan rehberliğinin sürekli\hyp{}mevcut kutsal ruhaniyetine ait öğreti olarak ikamet eden Düzenleyici’nin yönlendirişine ve kalıtımsal donatımdan gelen genişleyen aklının nedensellik\hyp{}yargısına bağlı olacağız. Yaratılmış ve Yaratan niteliklerinin bu türden bir ilişkilemi; gezegensel âlemler üzerinde tarafımızca, herhangi bir dünya üzerinde (hele Urantia üzerinde hiç olmamak üzere) herhangi bir nesil içinde tek bir insan tarafından değerlendirebilecek şekilde olmak zorunda bulunmayan ancak sahip olduğun uçsuz bucaksız evrene ait daha yüksek düzeyde kusursuzlaşmış ve kusursuzlaşmakta olan dünyalar üzerinde değerlendirildiği biçimiyle tamamiyle ve en yüksek derecede doygun nitelikteki, kusursuz olan yaşamı yaşamana yetkin kılacak.
\vs p120 3:10 “Ve, şimdi, geçmiş gerçekleştirimlerimizde sürekli olarak bizlere destek olmuş olan senin Yaratıcın ve benim Yaratım; bizlere elveda ettiğin ve kişilik bilincini teslim ettiğin aşamaya eriştiğin andan itibaren, insan bütünlüğü içinde vücutlaşmış olarak kutsal kimliğinin tanınışına kademeli olarak geri dönüşün boyunca, ve bunun da sonrasında, bedenden kurtuluşuna ve Yaratımız’ın egemenlik sağ koluna yükselene kadar, Urantia üzerindeki bahşedilme deneyiminin bütünü boyunca, seni yönlendirsin, seni desteklesin ve seninle birlikte olsun. Salvington üzerinde seni tekrar gördüğümde, kendi yarattığın, ona hizmet verdiğin ve ona dair tamamlanmış anlayışa eriştiğin bu evrenin en yüksek ve koşulsuz egemeni olarak bizlere geri dönüşünü karşılayacağız.
\vs p120 3:11 “Senin koltuğunda şimdi ben yönetimde bulunmaktayım. Ben, Urantia üzerinde senin yedinci ve fani bahşedilişinin ara dönemi boyunca vekâlet halindeki egemen olarak tüm Nebadon’un karar yetkisini üstlenmekteyim. Ve, sana Cebrail; o bana, İnsan Evladı ve Tanrı Evladı olarak yakın zamanda ve güç ve ihtişam içinde dönene kadar, bu olmaya\hyp{}hazırlanan İnsan Evladı’nın korunuşunda bağlıyım. Ve, Cebrail, ben, Mikâil bu şekilde dönene kadar senin egemeninim.”
\separatorline
\vs p120 3:12 Bunun sonrasında, derhal gerçekleşen bir biçimde, bir araya gelmiş tüm Salvington’un mevcudiyetinde, Mikâil aramazdan ayrıldı; ve, biz onu, Urantia üzerindeki bahşediliş sürecinin tamamlanışından sonra, evrenin en yüksek ve kişisel yöneticisi olarak geri dönüşüne kadar alışılmış konumunda onu bir kere daha görmedik.
\usection{4.\bibnobreakspace Vücutlaşım --- İkiyi Bir Yapmak}
\vs p120 4:1 Ve, sahip oldukları Yaratan\hyp{}yaratıyı bencil bir biçimde yönetimi arzulamakla suçlamış, ve, kölesel yaratılmışların elindeki yanlışa düşmüş bir evrenin sorgulamadığı sadakat sayesinde Yaratan Evlat’ın oldukça keyfi ve zorbaca kollandığına dair şüphenin çekiciliğine kapılmış olan Mikâil’in bir takım layıksız çocukları; bu süreçte her zaman “Cennet Yaratıcısı’nın iradesine” bağlı olan bir biçimde --- Tanrı Evladı’nın İnsan Evladı’na bu aşamada girmiş olduğu benliğini unutan hizmetinin yaşamı tarafından sonsuza kadar susturulmuş olacak ve kafaları karışmış ve inanışlarını yitirilmiş konumda bırakılacaklardı.
\vs p120 4:2 Ancak yanlış anlamayın; Mesih İsa, gerçek bir çifte\hyp{}köken varlığı olurken, bir çifte kişilik değildi. O, insanla \bibemph{beraber} ilişkilem içindeki Tanrı değildi; ancak, o bunun yerine, insan içinde \bibemph{vücutlaşmış} Tanrı’ydı. Ve, o her zaman, bu bileşmiş varlık bütünlüğündeydi. Bu tür anlaşılamaz nitelikteki bir ilişki içinde tek gelişimsel biçimde gerçekleşen etken, Tanrı ve insan olmanın bu gerçekliğinin (insan aklı tarafından) gelişimsel biçimde gerçekleşen öz bilinç tarafından fark edilişi ve tanınmasıydı.
\vs p120 4:3 Mesih İsa, ilerleyen bir biçimde Tanrı haline gelmedi. Tanrı, İsa’nın dünyasal yaşamı içindeki bir hayati an içinde, insan haline gelmedi. İsa, her zaman ve sonsuza kadar sürecek nitelikte --- Tanrı \bibemph{ve} insandı. Ve, bu Tanrı ve insan, eskiden, ve şimdi olduğu gibi; nasıl üç varlıktan oluşan Cennet Kutsal Üçlemesi gerçekte \bibemph{bir tek} İlahiyat ise, \bibemph{bir tek} bütünlüktü.
\vs p120 4:4 Mikâil bahşedilişinin en yüksek ruhsal amacının \bibemph{Tanrı’nın açığa çıkarılışını} geliştirmek gerçeği olduğunu hiçbir zaman gözden kaçırmayın.
\vs p120 4:5 Urantia fanileri, mucizeye dair değişiklik gösteren kavramsallaşmalara sahiptirler; ancak, yerel evrenin vatandaşları olarak yaşayan bizler için, birkaç mucize bulunmaktadır; ve, bunların arasında kıyas edilemez biçimde en ilgi çekici olanları Cennet Evlatları’nın vücutlaşımsal bahşedilmeleridir. Bir kutsal Evlat’ın, görünüşte doğal süreçler vasıtasıyla, dünyanız içinde ve üzerinde ortaya çıkışını --- anlayışımızın ötesindeki evrensel yasalarının işleyişi olarak --- bizler bir mucize olarak görmekteyiz. Nasıralı İsa mucizevî bir kişiydi.Bu olağanüstü deneyimin tümü içinde ve onun vasıtasıyla, Yaratıcı Tanrı, her zaman gerçekleştirdiği gibi kendisini; olağan bir biçimde olarak --- kutsal eylemin olağan, doğal ve güvenilir biçiminde sergilemeyi tercih etmiştir.
\vs p120 4:6 Bu olağanüstü deneyimin tümü içinde ve onun vasıtasıyla, Yaratıcı Tanrı, her zaman gerçekleştirdiği gibi kendisini; \bibemph{olağan bir biçimde} olarak --- kutsal eylemin olağan, doğal ve güvenilir biçiminde sergilemeyi tercih etmiştir.
