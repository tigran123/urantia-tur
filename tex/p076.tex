\upaper{76}{İkinci Bahçe}
\vs p076 0:1 Âdem Nodit unsurlarına ilk bahçeyi karşı koymadan bırakmayı tercih ettiği zaman; o ve onun takipçileri batıya doğru gidemediler; çünkü Cennet Bahçesi unsurları, bu türden deniz seyahati için uygun hiçbir tekneye sahip değillerdi. Onlar kuzeye doğru yönelemediler; çünkü kuzey Nodit unsurları çoktan Cennet Bahçesi’ne doğru yürüyüşe geçmişlerdi. Onlar güneye gitmekten korktular; bu bölgenin tepeleri düşman kabileler ile doluydu. Önlerinde elverişli olan açık tek yön doğu doğrultusuydu, ve böylece onlar Dicle ve Fırat nehirleri arasındaki o zamanların güzel bölgelerine doğru doğu yönünde hareket ettiler. Ve arkada bırakılan sakinlerin çoğu daha sonra, yeni vadi evlerinde Âdem unsurlarına katılmak için doğuya doğru hareket etti.
\vs p076 0:2 Kabil ve Sansa’nın ikisi de, kervan Mezopotamya nehirleri arasındaki istikametine ulaşmadan önce doğmuştu. Sansa’nın annesi Laotta, kızının doğumu sırasında hayatını kaybetti; Havva, sahip olduğu üstün kuvveti sayesinde çok acı çekmesine rağmen hayatta kalmayı başardı. Havva, Laotta’nın çocuğu olan Sansa’yı bağrına bastı; ve o, Kabil ile beraber yetiştirildi. Sansa, büyük bir yetkinliğe sahip bir kadın haline gelerek büyüdü. O, kuzey mavi ırkların başı olan Sargan’ın karısı oldu; ve o, bu dönemlerin mavi ırklarının gelişimine katkı sağladı.
\usection{1.\bibnobreakspace Cennet Bahçesi Unsurları’nın Mezopotamya’ya Girişi}
\vs p076 1:1 Âdem’in kervanının Fırat Nehri’ne ulaşması neredeyse tam bir yıl aldı. Bu nehri taşkın olduğu dönemde buldukları için, ikinci bahçeleri haline gelecek nehirler arasındaki yerleşkeye geçebilmek için yollarını bulmadan yaklaşık altı hafta önce, nehrin batı kesimindeki düzlüklerde geçici olarak konakladılar.
\vs p076 1:2 İkinci bahçe yerleşkesi içindeki sakinlere Cennet Bahçesi’nin kralı ve yüksek din adamının onlara doğru yaklaşmakta olduğu haberleri ulaşınca, aceleyle doğu dağlarına doğru kaçtılar. Âdem buraya ulaşınca arzu edilen bölgenin tamamını boşaltılmış olarak buldu. Ve bu yeni yerleşke içerisinde Âdem ve ona yardım edenler kendilerini, yeni evler inşa etmeye ek olarak kültür ve dinin yeni merkez kültürünü oluşturma görevine adadılar.
\vs p076 1:3 Bu yerleşke, Cennet Bahçesi için olası yerleşkelerin seçilmesi amacıyla zamanında görevlendirilmiş heyetin karara vardığı ilk üç tercihten biri olarak Van ve Amadon’un tavsiyesi vasıtasıyla Âdem tarafından bilinmekteydi. Bu iki ırmak, bu dönemlerin iyi birer doğal savunma aracıydı; ve ikinci bahçenin kuzeyindeki dar bir hatta Fırat ve Dicle, ırmaklar arasında ve güneye doğru bölgenin koruması amacıyla inşa edilebilecek doksan kilometreye varan bir koruma duvarına imkân verecek şekilde birbirine yaklaşmaktaydı.
\vs p076 1:4 Yeni Cennet Bahçesi’ne yerleştikten sonra, yaşamın çetin yöntemlerine uyum sağlamak gerekli hale gelmişti; toprağın neredeyse lanetlenmiş olduğu tamamiyle doğruymuş gibi gözüktü. Doğa bir kez daha yönetimi eline geçirmekteydi. Bu aşamada Âdem unsurları, fani mevcudiyetin doğal karşıtlıkları ve uyumsuzluklarının yanı sıra hazırlıksız toprakla başa çıkarak yaşamın gerçeklerinin üstesinden gelmek zorunda bırakılmışlardı. Onlar ilk bahçeyi kendileri için kısmen hazırlanmış olarak bulmuşlardı; ancak ikinci bahçe, kendi bilek güçleriyle ve “alın terleri” ile yaratılmak zorundaydı.
\usection{2.\bibnobreakspace Kabil ve Habil}
\vs p076 2:1 Kabil’in doğumundan sonra iki yılından daha kısa bir süre içinde, Âdem ve Havva’nın ikinci bahçe içindeki ilk çocukları olarak Habil dünyaya gelmişti. Habil on iki yaşına geldiğinde sürülerin sorumluluğunu tercih etmişti; Kabil ise tarımı seçmişti.
\vs p076 2:2 Bu aşamada, bahse konu dönemlerde sahip olunan şeyler içerisinde din adamlığı kurumuna bağışta bulunmak adetti. Sürü güdenler sürü hayvanlarını getirir, çiftçiler ise tarla ürünlerini sunarlardı; ve bu gelenek uyarınca Kabil ve Habil böylece din adamlarına dönemsel bağışlarda bulundular. Bu iki erkek evlat birçok kez seçmiş oldukları işlerin göreceli yararları üzerine tartışmışlardı; ve Habil, ilk tercihin kendisinin sunduğu hayvan kurbanlarına olduğunu anlamakta gecikmemişti. Bilinçsiz bir biçimde Kabil, tarlalardan elde edilen ürünlere yönelik daha önce uygulanmakta olan ilk tercih uygulaması biçimindeki birinci Cennet Bahçesi’nin geleneklerinin gelmesi için itirazda bulundu. Ancak bunun gerçekleşmesine Habil izin vermezdi; ve Habil başarısızlığı sebebiyle abisiyle alay etti.
\vs p076 2:3 İlk Cennet Bahçesi zamanlarında Âdem gerçekten de, Kabil’i itirazlarında haklı çıkaran bir biçimde hayvanların bağış olarak kurban edilişini caydırmaya çalıştı. Ancak, ikinci Cennet Bahçesi’nin dini yaşamını düzenlemek zordu. Âdem inşaat, savunma ve tarım işleri ile ilgili bin bir ayrıntıyla boğuşmaktaydı. Ruhsal olarak bu ölçüde umutsuzluğa düşmüş bir haldeyken o, birinci bahçede aynı yetkinlikler içinde görev yapmış Nodit kökenli topluluk üyelerine ibadet ve eğitimin örgütlenişini emanet etmiştir; ve buna rağmen bile görev halindeki Nodit din adamları kısa bir süre içinde Âdem\hyp{}öncesi dönemlerin ortak ölçütlerine ve idare biçimlerine geri dönmektelerdi.
\vs p076 2:4 Bu iki erkek çocuk hiçbir bir zaman anlaşamadılar; ve bu kurbanlık meselesi, ikisi arasındaki mevcut kinin daha da büyümesine sebep oldu. Habil, Âdem ve Havva’nın oğlu olduğunu bilmekteydi; ve o hiçbir zaman, Âdem’in kendi babası olmadığını Kabil’in başına kalkmaktan geri durmuyordu. Kabil; babası, mavi ve kırmızı ırka ek olarak yerli Andonsal ırk koluyla karışmış Nodit ırkının bir üyesi olduğu için, saf eflatun ırk mensubu değildi. Ve bütün bunların hepsi Kabil’in kavgacı doğasıyla birleşince, küçük kardeşi için sürekli artan bir kini içinde beslemesine sebep olmuştu.
\vs p076 2:5 Aralarındaki gerginlik tamamen sonuçlandığında, bu çocuklar sırasıyla on sekiz ve yirmi yaşındaydılar; bu günde Habil’in sataşmaları kavgacı kardeşi Kabil’i o kadar sinirlendirmişti ki, Kabil bunun karşılığında öfkeye kapılıp onu öldürdü.
\vs p076 2:6 Habil’in davranışı irdelendiğinde, karakter gelişimindeki etkenler olarak çevre ve eğitimin önemi ortaya çıkmaktadır. Habil olası en yüksek bir kalıtıma sahipti, ve kalıtım ise karakter bütünlüğünün temelini oluşturmaktadır; ancak alt düzeyde bulunan bir çevrenin etkisi, bu muhteşem kalıtımı neredeyse tamamen etkisiz hale getirdi. Özellikle daha küçük yaşları boyunca Habil, elverişsiz çevre koşulları tarafından büyük ölçüde etkilenmişti. O, yirmi beş veya otuz yaşına kadar yaşasaydı tamamiyle başka bir kişi olacaktı; onun muhteşem kalıtımı bu zaman zarfında kendisini gösterebilecekti. İyi bir çevre, alt düzeydeki bir kalıtımın ürünü olan karakter yoksunluklarının gerçek anlamda üstesinden gelinmesine yönelik çok katkı sağlayamasa da; kötü bir çevre, muhteşem bir kalıtımı, en azından genç yaşlarında, oldukça etkin bir biçimde bozabilir. İyi toplumsal çevre ve yerinde eğitim, iyi kalıtımı en iyi bir şekilde açığa çıkarmak için hayati derecede önemli hava ve topraktır.
\vs p076 2:7 Habil’in ölümü, köpekleri sahibi olmadan sürüleri eve getirdiğinde ebeveynleri tarafından anlaşıldı. Âdem ve Havva için Kabil hızlı bir biçimde, akılsızlıklarının acımasız bir yadigârı haline gelmekteydi; ve onlar, bahçeden ayrılma kararında Kabil’i desteklediler.
\vs p076 2:8 Kabil’in Mezopotamya’daki yaşamı, doğru düzenden ayrılışın öylesine tuhaf bir biçimde simgesi olduğu için, mutluluk içerisinde geçememişti. Onun birliktelik içerisinde bulunduğu kişiler kendisine kötü davranmamaktalardı, ama Kabil mevcudiyeti karşısında onların bilinçaltında barındırdıkları hınçtan habersizdi. Fakat Kabil, hiçbir kabile simgesi taşımadığı için kendisini şans eseri görecek ilk komşu kabile mensubu tarafından öldürüleceğini bilmekteydi. Korku ve bir parça vicdan azabı kendisini yaptığından dolayı pişman olmaya itti. Kabil’in içinde hiçbir zaman bir Düşünce Düzenleyicisi ikamet etmemişti; ve o her zaman, aile düzeninin karşısında durmuş ve babasının dinini küçümsemişti. Ancak bu aşamada o annesi olan Havva’ya gitmiş ve ondan ruhsal yardım ve yönlendirme talebinde bulunmuştu; ve dürüst bir biçimde kutsal desteğin peşine düştüğünde, bir Düşünce Düzenleyicisi onun içinde ikamet etmeye başlamıştır. Ve içinde ikamet eden ve dikkatle her şeyi gözeten bu Düzenleyici, kendisinden en çok korku duyulan Âdem kabile üyesi biçiminde görülmesine neden olan, farklı bir üstünlük imkânı sağlamıştı.
\vs p076 2:9 Ve böylece Kabil, ikinci Cennet Bahçesi’nin doğusunda bulunan Nod yerleşkesine doğru buradan ayrıldı. O, babasının insanlarına ait bir topluluk içinde büyük bir önder haline geldi; ve bir dereceye kadar, Serapatatia’nın öngörülerini doğruladı; çünkü Kabil, yaşamı boyunca Nod ve Âdem unsurlarının bu bölünüşünün barışla düzeltilmesini sağladı. Kabil, uzak kuzeni olan Remona ile evlendi; ve onun ilk erkek çocuğu olan Enoch, Elam Nod unsurlarının başı haline geldi. Ve yüz yıllar boyunca Elam ve Âdem unsurları barış içinde kalmaya devam etti.
\usection{3.\bibnobreakspace Mezopotamya’da Yaşam}
\vs p076 3:1 İkinci bahçe içerisinde zaman ilerlerken, doğru yoldan ayrılışın sonuçları artarak belirginleşmeye başladı. Âdem ve Havva, Edentia’ya götürülen çocuklarına ek olarak önceki evlerinin güzelliğini ve huzurunu çok aradılar. Bu muhteşem çiftin âlemin ortak bedeni düzeyine indirgenişini gözlemlemek gerçekten de acınası bir durumdu; ancak onlar, alçalan düzeylerini saygıyla ve metanetle karşıladılar.
\vs p076 3:2 Âdem vaktinin büyük bir kısmını bilge bir biçimde, çocuklarını ve onların birliktelik içinde bulunduğu kişileri kamu idaresinde, eğitimsel yöntemlerde ve dini ibadetlerde eğitmekle harcadı. Eğer onun bu öngörüsü olmasaydı, ölümü üzerinde çok ciddi karışıklıklar açığa çıkabilirdi. Böyle olduğu için Âdem’in ölümü, insanlarının yaşam koşullarını idare edişi üzerinde çok az bir değişikliğe neden olmuştur. Ancak vefatlarından çok uzun bir süre önce Âdem ve Havva, Cennet Bahçesi içerisindeki ihtişam dolu günleri çocukları ve takipçilerinin yavaş yavaş unutmaya başladıklarının farkına varmışlardı. Ve takipçilerinin büyük bir çoğunluğu için Cennet Bahçesi’nin yüceliğini unutmak iyi bir şeydi; onlar, daha az elverişli çevrelerinden dolayı gereksiz bir hayal kırıklığına kolay kolay düşmeyeceklerdi.
\vs p076 3:3 Âdem unsurlarının toplum yöneticileri, ilk bahçenin evlatlarından saltanat usulü ile seçilmişlerdi. Âdem’in ilk oğlu Âdemoğlu (Âdem oğlu Âdem), ikinci Cennet Bahçesi’nin kuzeyinde eflatun ırkının yardımcı bir merkez idaresini kurdu. Âdem’in ikici evladı Eveson, üstün bir önder ve idareci haline geldi; o, babasının büyük bir yardımcısıydı. Eveson, Âdem kadar uzun bir süre yaşamamıştı; ve onun en büyük oğlu Jansad, Âdem kabilelerinin başı olarak Âdem’in varisi haline geldi.
\vs p076 3:4 Din adamlığı veya diğer bir değişle dini önderlik, ikinci bahçede dünyaya gelmiş olan Âdem ve Havva’nın hayattaki en büyük oğlu Seth ile doğdu. Seth, Âdem’in Urantia’ya varışından yüz yirmi dokuz yıl sonra doğmuştu. Seth, ikinci bahçenin yeni din adamlık kurumunun başı haline gelerek, babasının insanlarının ruhsal düzeyini geliştirme görevine kendisini adayan bir hale geldi. Onun oğlu Enos, ibadetin yeni düzeninin kurdu; ve onun torunu Kenan, yakında ve uzakta bulunan komşu kabileler için dışa yönelik din yayma hizmetini kurumsallaştırdı.
\vs p076 3:5 Seth din adamlığı; dini, sağlığı ve eğitimi içine alan bir biçimde üç katmanlı bir girişimdi. Bu düzenin din adamları; dini törenleri yönetmek, sağlık uzmanları ve temizlik denetleyicileri olarak faaliyet göstermek ve bahçenin okullarında öğretmenler olarak faaliyet göstermek üzere eğitilmişlerdi.
\vs p076 3:6 Âdem’in kervanı, birinci bahçenin bitkilerinden ve tahıllarından yüzlercesinin tohumunu ve çiçek soğanını beraberlerinde ırmaklar arasındaki bu yerleşkeye getirmişlerdi; onlar aynı zamanda, geniş sürüler ve evcilleştirilmiş her hayvandan bir kaçıyla buraya gelmişlerdi. Bu nedenle onlar, çevre kabileler üzerinde büyük bir üstünlüğe sahip oldular. Onlar, ilk Bahçe’nin geçmiş kültürünün birçok yararından memnuniyetle faydalandılar.
\vs p076 3:7 İlk bahçeden ayrılış vaktine kadar Âdem ve onun ailesi her zaman; meyveler, tahıllar ve yemişler ile beslenmekteydiler. Mezopotamya’ya olan yolculuklarında onlar ilk kez, şifalı bitkiler ve sebzeleri tattılar. Et ile beslenme ikinci bahçeye en erken dönemlerinden itibaren girmişti, ancak Âdem ve Havva eti günlük beslenme biçimlerinin bir parçası haline hiçbir zaman getirmediler. Ne Âdemoğlu ne de Havvaoğlu, ilk bahçenin ilk nesil çocuklarının herhangi biri bile, et ile beslenen konuma gelmemişlerdir.
\vs p076 3:8 Âdem unsurları, kültürel kazanım ve ussal gelişim bakımından çevre toplulukları üzerinde büyük bir üstünlük kurdu. Onlar üçüncü alfabeyi yaratmış olup, diğer bir yandan ise modern sanatın, bilimin ve edebiyatın ilk temsillerinin ortaya çıkması için temelleri attılar. Fırat ve Dicle nehirleri arasındaki bu topraklarda onlar; yazı, madeni eşyalar yapım, çömlek ve dokuma sanatlarını besleyip, bin yıllar boyunca daha iyisinin ortaya çıkmayacağı bir mimari türü yarattılar.
\vs p076 3:9 Eflatun insanlarının ev yaşamı, içlerinde bulundukları dönem ve koşulları için olası en yüksek düzeyde bulunmaktaydı. Çocuklar; tarım, el işleri ve hayvan evcilleştirilmesi hususlarındaki derslere tabilerdi; ve onların dışında kalanlar ise, bir Seth üyeliğinin üç katmanlı görevini yerine getirmek için eğitilmektelerdi: bu görev din adamı, sağlık uzmanı ve öğretmen olmaktı.
\vs p076 3:10 Ve Seth din adamlığını düşünürken, gerçek eğitmenler olarak sağlık ve dinin yüksek akla sahip ve soylu bu öğretmenlerini; daha sonraki kabilelerin ve çevre insan topluluklarının yozlaşmış ve ticari hale gelmiş din adamları ile karıştırmayın. Onların dini evren ve İlahiyat kavramları ileri bir düzeyde olup, neredeyse tamamen doğruydu; onların sağlık emirleri, içinde bulundukları dönem bakımından, mükemmeldi; ve onların eğitim yöntemlerinin üzerine bu dönemden beri hiçbir şekilde geçilmemiştir.
\usection{4.\bibnobreakspace Eflatun Irkı}
\vs p076 4:1 Âdem ve Havva, Urantia üzerinde ortaya çıkan dokuzuncu insan ırkı olarak, eflatun insanları ırkının kurucularıydılar; Âdem ve onun doğumu mavi gözlere sahiplerdi; ve eflatun insan toplulukları, açık ten rengine ek olarak sarı, kırmızı ve kahverengi şeklindeki açık saç renkleriyle ayırt edilmekteydi.
\vs p076 4:2 Havva, doğumda acı çekmemişti; buna ek olarak öncül evrimsel ırkların hiçbiri bu acıyı deneyimlememiştir. Evrimsel insanın Nodit unsurları ve daha sonra Âdem unsurları ile olan birlikteliğinden doğan melez ırklar, doğumun ciddi sancılarını deneyimlediler.
\vs p076 4:3 Jerusem üzerindeki kardeşleri gibi Âdem ve Havva’ya; yiyecek ve ışıktan oluşan ve Urantia üzerinde açığa çıkarılmamış belirli üstün fiziksel enerjilerle takviye edilen çifte besin ile yaşam enerjisi sağlanmaktaydı. Onların Urantia doğumları, ebeveynlerinin sahip olduğu enerji alımı ve ışık dolaşımı kazanımını onlardan kalıtımsal olarak alamadılar. Onlar, insan türüne özgü kan dolaşımıyla beslenme olarak, tek bir dolaşıma sahiplerdi. Onlar, her ne kadar çok uzun yaşam sürelerine sahip olsalar da fani olarak tasarlanmışlardı; her ne kadar bu uzun ömürlülük her bir ilerleyen nesilde daha çok insanların sahip oldukları ölçütlere gerilemiştir.
\vs p076 4:4 Âdem, Havva ve onların ilk nesil çocukları, beslenmeleri için hayvan etini kullanmadılar. Onlar tamamen “ağaçların meyvelerinden” beslendiler. İlk nesilden sonra Âdem’in tüm soyları, mandıra ürünlerinden beslendi; ancak onların çoğu, etsiz yeme alışkanlarına bağlı kalmayı sürdürdü. Daha sonra kendileriyle bütünleşen birçok güney kabile üyesi de et ile beslenmemektelerdi. Daha sonra, sadece sebze ile beslenen bu kabilelerin çoğu doğuya göç etmiş olup, şimdi Hindistan insanları olan bütünlük içinde varlıklarını devam ettirmişlerdir.
\vs p076 4:5 Âdem ve Havva’nın fiziksel ve ruhsal öngörüleri, bugünün insan topluluklarına kıyasla çok daha üstündü. Onların özel duyuları çok daha keskindi; ve onlar yarı\hyp{}ölümlü unsurları, Melçizedekler olan meleksel ev sahiplerini, ve soylu halefi ile birkaç kez görüşmek için gelmiş olan devrik Prens Caligastia’yı görebilmektelerdi. Bu özel duyular çocuklarında bu kadar keskin değildi; ve bu duyular, her yeni nesilde körelme eğilimi gösterdi.
\vs p076 4:6 Âdem çocukları, hepsi kuşkusuz bir biçimde varoluş yetkinliğine sahip oldukları için, genellikle Düzenleyici’yi içlerinde barındırmaktaydı. Onların üstün doğumları, evrimin çocukları gibi fazlasıyla korku hissi duymamaktaydılar. Bugünün Urantia ırkları içinde korku hala fazla bir biçimde var olmaya devam etmektedir; bu durumun nedeni, ırkları fiziksel olarak canlandırma tasarımının erken bir biçimde yanlış idare edilmesi yüzünden atalarınızın Âdem’in yaşam plazmasının çok azından faydalanabilmiş olmasıdır.
\vs p076 4:7 Maddi Evlatlar ve onların doğumlarının sahip oldukları beden hücreleri, gezegenin yerli sakinleri olan evrimsel varlıklarınınkine kıyasla hastalığa çok daha fazla dirençliydi. Yerli ırkların beden hücreleri, âlemin mikroskopla görülen ve hatta ondan daha küçük hastalık\hyp{}yaratıcı yaşayan organizmalar ile aynı türdendir. Bu gerçekler, bilimsel uğraşlar vasıtasıyla birçok fiziksel hastalığa karşı koymak için Urantia insanlarının ne kadar çok çalışmak zorunda olduklarını açıklamaktadır. Eğer ırklarınız Âdemsel yaşamdan daha çok özü barındırabilseydi, sizler kıyaslanamaz bir derecede hastalığa karşı dirençli olurdunuz.
\vs p076 4:8 Fırat Nehri kenarındaki ikinci bahçede yerleşik bir konuma gelince Âdem, ölümü sonrasında dünyaya yarar sağlaması için yaşam plazmasının olabildiğince çok miktarını ardında bırakmayı tercih etmiştir. Bunun sonucunda Havva, ırk gelişiminden sorumlu on iki heyet üyesinden oluşan bir birliğin başına getirilmişti; ve Âdem’in ölümünden önce bu heyet, Urantia üzerinde en yüksek düzeyde bulunan 1.682 kadını seçmişti; ve bu kadınlar, Âdem’in yaşam plazmasından gebe kalmıştı. Onların çocuklarının tümü, 112’si dışında erişkin düzeye kadar büyüdü; dünya bu şekilde, 1.570 üstün ilave erkek ve kadının kendisine tahsis edilmesinden böylece faydalanmış oldu. Her ne kadar bu aday anneler çevre kabilelerin tümünden seçilmiş ve dünya üzerindeki ırkların birçoğunu temsil etmiş olsalar da, onların çoğunluğu Nod unsurlarının en yüksek ırk kollarından belirlenmişti; ve onlar, kudretli And unsurları ırkının öncül temellerini oluşturmuşlardı. Bu çocuklar, dünyaya geldikleri annelerin ait oldukları kabile yerleşkesinde doğup büyütülmüşlerdi.
\usection{5.\bibnobreakspace Âdem ve Havva’nın Ölümü}
\vs p076 5:1 İkinci Cennet Bahçesi’nin kurulmasından sonra yakın bir zaman içerisinde Âdem ve Havva, pişmanlıklarının kabul edildiğine dair gerektiği gibi bilgilendirildi; buna ek olarak onlara, dünyalarına ait fanilerin nihai sonunu paylaşmakla yükümlü olduklarından Urantia’nın uyku halindeki kurtuluş unsurlarının düzeylerine girmeye kesinlikle yetkin oldukları iletildi. Onlar, Melçizedekler’in oldukça etkileyici bir biçimde kendilerine duyurduğu yeniden dirilişe ve iyileşmelerine dair bu müjdeye bütünüyle inandılar. Onların neden oldukları suç, bir karar hatasıydı; o, bilinçli ve kasıtlı isyana ait bir günah değildi.
\vs p076 5:2 Âdem ve Havva, tıpkı Jerusem’in vatandaşları gibi, Düşünce Düzenleyicileri’ne sahip değillerdi; buna ek olarak onlar, ilk bahçe döneminde Urantia üzerinde faaliyet gösterirken Düzenleyici’yi içlerinde barındırmamaktalardı. Ancak fani düzeye indirgenmelerinden kısa bir süre sonra onlar, içlerindeki yeni bir mevcudiyetin farkına vardılar; ve içten pişmanlıkla birleşen insan mevcudiyetinin, Düzenleyicileri içlerinde barındırmalarını mümkün kıldığını hemen kavradılar. Düzenleyiciler’i içlerinde barındırmaya dair bu gerçeklik, Âdem ve Havva’yı yaşamlarının geri kalanı boyunca büyük ölçüde cesaretlendirmiştir; onlar, Satania’nın Maddi Evlatları olarak başarısız olduklarını bilmektelerdi; ancak onlar aynı zamanda, Cennet sürecinin evrenin yükseliş halindeki evlatları olarak kendilerine açık olduğunu bilmektelerdi.
\vs p076 5:3 Âdem, gezegene varışıyla birlikte eş zamanlı olarak gerçekleşmiş yazgı sonu dirilişi hakkında bilgiye sahipti; ve o, evlatlığın yeni düzeyinin varışıyla birlikte kendisi ve eşinin muhtemel bir biçimde yeniden kişilikleştirileceğine inanmaktaydı. O, bu evrenin hâkimi olan Mikâil’in çok yakın bir zaman içerisinde Urantia üzerinde ortaya çıkacağını bilmemekteydi; o, bir sonraki Evlat’ın Avonal düzeyden geleceğini düşünmekteydi. Böyleyken bile, Âdem ve Havva’nın Mikâil’den o zamana kadar aldıkları tek kişisel ileti üzerinde düşünmesi onlar için anlaşılması zor bir şey olmakla birlikte her zaman rahatlatan bir etkiye sahipti. Arkadaşlığın ve tesellinin diğer dışavurumlarına ek olarak bu ileti şunu bildirmiştir: “Ben yükümlülüklerinizi yerine getirmeyişiniz ile ilgili olan koşulları irdeledim, ben kalplerinizin her zaman duyduğu Yaratıcı’nın iradesine sadık kalma arzusunu unutmadım, ve sizler, âlemimin alt sorumlu Evlatları tarafından daha önce çağrılmazsanız fani geniş uykusu sürecinden istenilecekseniz.”
\vs p076 5:4 Ve bu ileti Âdem ve Havva için büyük bir sırdı. Onlar, bu ileti içerisinde olası bir özel dirilişin örtülü sözünü kavrayabiliyorlardı; ve bu türden bir olasılık onları büyük bir neşeye boğdu; ancak onlar, Urantia üzerinde Mikâil’in kişisel görünüm vaktine kadar uyuyabileceklerine dair üstü kapalı bilgilendirmenin anlamını kavrayamıyorlardı. Ve böylece Cennet Bahçesi çifti her zaman, ileride bir vakit Tanrı’nın bir Evladı’nın geleceğini duyurdular; ve onlar, en azından olmasını arzuladıkları bir ümitle, sevdiklerine; hatalarına ve üzüntülerine sahne olan dünyalarının, evrenin yöneticisinin Cennet bahşedilme Evladı olarak üzerinde faaliyet göstermek için tercih edebileceği âlem olma ihtimaline dair inançlarını iletmişlerdi. Bu türden bir ümit gerçekleşmeyecek kadar güzeldi; ancak Âdem, isyanla parçalanmış Urantia’nın sonunda Nebadon’un tümü içindeki en kıskanılan gezegen olarak Satania sisteminin içindeki en talihli dünya haline gelebileceğini aklından geçirmişti.
\vs p076 5:5 Âdem 530 yıl yaşamıştır; o, ileri yaş olarak adlandırılabilecek sebepten dolayı vefat etmiştir. Onun fiziksel bünyesi yalın bir deyimle yıpranmıştı; ayrışma süreci kademeli olarak onarım işleyişinin önüne geçmişti; ve kaçınılmaz son gelmişti. Havva, Âdem’den on dokuz yıl önce zayıflayan kalbi nedeniyle yaşamını yitirmişti. Onların ikisi de, tasarımları uyarınca inşa edilmiş birlikteliklerini çevreleyen duvar tamamlandıktan sonra kutsal mabedin merkezine gömülmüşlerdir. Ve bu eylem, ibadethanelerin altlarına yüksek mertebeli ve dindar erkek ve kadının gömülmesi uygulamasının kökenini oluşturmuştur.
\vs p076 5:6 Melçizedekler’in idaresi altındaki Urantia’nın aşkın\hyp{}maddi yönetimi görevine devam etmişti; ancak evrimsel ırklar ile gerçekleştirilen doğrudan fiziksel iletişime son verildi. Gezegensel Prens’in bedensel görevlilerinin geldiği uzak dönemlerden Van ve Amadon zamanları boyunca Âdem ve Havva’nın gelişine kadar, evren yönetiminin fiziksel temsilcileri gezegen üzerinde konumlanmış bir haldeydi. Ancak Âdem’in görevini yerine getirmedeki başarısızlığı ile birlikte, dört yüz elli bin yılı aşan bir süreci kaplayan, bu düzen sona erdi. Ruhsal alanlarda meleksel yardımcılar, bireyin kurtuluşu için kahramanca çalışır haldeki Düşünce Düzenleyicileri ile birlikte mücadelelerini sürdürdüler; ancak Maçiventa Melçizedek’in varışına kadar dünya refahına dair çok uzun bir dönemi kapsayan etraflıca bir tasarımın varlığı dünya fanilerine duyurulmamıştır; bir Tanrı Evladı’nın sahip olduğu güç, sabır ve yönetim yetkisiyle İbrahim kendi döneminde, talihsiz Urantia’nın ileri canlandırılışı ve ruhsal iyileştirilişi için temelleri atmıştır.
\vs p076 5:7 Talihsizlik buna rağmen Urantia’nın payına düşen tek şey değildi; bu gezegen aynı zamanda Nebadon yerel evreni içinde en talihli olandı. Urantialılar; atalarının apaçık yanlışlarından ve öncül dünya yöneticilerinin hatalarından doğan bu karanlık geçmişin kendisi, cennet içindeki Yaratıcı’nın sevgi dolu kişiliğini üzerinde açığa çıkarmak için esas mekân olarak bu dünyayı tercih ettirtecek kadar Nebadon Mikâili’nin ilgisini çeken bir derecede gezeni bu tür ümitsiz bir düzeye sevk ettiyse, daha da fazlası onu kötülük ve günahla kafa karışıklığına ittiyse, bunların hepsini bir kazanç olarak değerlendirilmelidir. Karmaşık hale gelen olayları düzene sokmak için Urantia bir Yaratan Evlat’a ihtiyaç duymamaktaydı; bunun yerine Urantia üzerinde mevcut olan kötülük ve günah, Cennet Yaratıcısı’nın eşsiz sevgisini, bağışlamasını ve sabrını açığa çıkarmak için Yaratan Evlat’a çok daha etkileyici bir geçmişi sunmuştu.
\usection{6.\bibnobreakspace Âdem ve Havva’nın Kurtuluşu}
\vs p076 6:1 Âdem ve Havva; Urantia üzerindeki eflatun ırkının maddi bedeni içindeki görevlerinden önceki dönemlerde kendilerine tamamen aşina olan dünyalar biçimindeki, malikâne dünyaları üzerindeki yaşamlarına ölüm uykularından ileride kalkıp devam edeceklerine dair Melçizedekler tarafından kendilerine verilen vaatlere derinden inanarak fani istirahatlarına çekildiler.
\vs p076 6:2 Onlar, âlemin fanilerine ait bilinçsiz haldeki uykunun şuursuzluğunda uzun süre istirahat etmediler. Âdem’in ölümünün üçüncü gününde, hürmetkârca düzenlenen naaş töreninden iki gün sonra; Mikâil adına görevde bulunan, Salvington üzerindeki Zamanın Birliktelikleri tarafından onaylanmış ve vekil Edentia’nın En Yüksek Unsurları tarafından gözden geçirilmiş, Urantia üzerinde Âdemsel görev başarısızlığının tüm seçkin kurtuluş unsurlarına özel yoklama çağrısını emreden Lanaforge yönergeleri Cebrail’in ellerine teslim edilmişti. Ve özel dirilişin bu emri uyarınca Urantia zincirlerinin yirmi altıncısı olan Âdem ve Havva, ilk bahçe döneminde kendilerine yardımda bulunan 1.316 yardımcısı ile birlikte Satania’nın malikâne dünyalarına ait diriliş yapılarında yeniden kişilikleştirilip tekrar bir bütün haline getirildiler. Âdem’in varış zamanında, uyuyan kurtuluş unsurlarına ve yaşayan yetkin yükseliş bireylerine ait bir yazgı dönem sonu hükmüyle gerçekleşen olaylar sonucunda birçok diğer sadık ruh çoktandır bu dünyaya aktarılmış halde bulunmaktaydı.
\vs p076 6:3 Âdem ve Havva, ilerleyici yükseliş dünyaları boyunca Jerusem üzerindeki vatandaşlık seviyesine erişene kadar hızlı bir biçimde yükseldiler; onlar bir kez daha kökenleri olan gezegenin vatandaşları haline gelmişlerdi; ancak bu sefer, evren kişiliklerin farklı bir düzeyine ait üyeler olarak bu gezegenin vatandaşlığını ellerinde bulundurmaktalardı. Onlar Jerusem’i --- Tanrı’nın Evlatları olarak --- kalıcı vatandaşlar düzeyinde terk etmişlerdi; onlar buraya insan evlatları olarak --- yükseliş vatandaşları seviyesine geri dönmüşlerdi. Onlar, Urantia’nın mevcut danışma\hyp{}denetim bünyesini oluşturan yirmi dört danışman arasındaki mevkie daha sonra atanan bir biçimde, sistem başkenti üzerinde doğrudan Urantia hizmetine görevlendirilmişlerdi.
\vs p076 6:4 Ve böylelikle Urantia’nın Gezegensel Âdem ve Havvası’nın hikâyesi sona ermektedir; onların hikâyesi bir deneme\hyp{}yanılma, trajedi ve zafer öyküsüdür; bu hikâye en azından, sahip olduğunuz iyi niyetli fakat yanılmış Maddi Erkek ve Kız Evlat’ın kişisel başarısıdır; ve kuşkusuz bir biçimde onların öyküsü sonuç olarak, bu iki bireyin dünyası ve onun isyanla çalkalanmış ve kötülükten bitap düşmüş sakinleri için nihai bir zafer hikâyesidir. Her şey göz önünde tutulduğunda Âdem ve Havva, insan ırkının çabuk medenileşmesine ve hızlandırılmış biyolojik ilerleyişine çok büyük katkı sağlamıştır. Onlar dünya üzerinde büyük bir kültürü gerilerinde bırakmışlardı; ancak bu türden gelişmiş bir medeniyetin kurtuluşu, Âdemsel kalıtımın öncül zayıflayışı ve nihai çöküşü karşısında mümkün değildi. Bir medeniyeti mevcut kılan onu inşa eden insan topluluklarıdır; bu yetkin birey toplulukları olmadan medeniyet kendisini sürdürecek zümreleri yaratamamaktadır.
\vs p076 6:5 [“Bahçe’nin {yüksek} meleksel sesi” olan Solonia tarafından sunulmuştur.]
