\upaper{2}{Tanrı’nın Doğası}
\vs p002 0:1 İnsanin olası en yüksek Tanrı kavramının sınırsız ve özüt bir kişiliğin bünyesinde bütünleştiği ölçüde İlahiyat’ın karakterini oluşturan kutsal doğanın niteliklerini irdelemek mümkün ve aynı zamanda yararlı hale gelecektir. Tanrı’nın doğası Nebadon'a ait olan Mikâil’in çok çeşitli öğretilerinde ve bedensel yaşamının üstün ölümlü hayatında ortaya çıktığı şekliyle Tanrı’nın kendini açığa çıkarmasıyla en iyi bir biçimde anlaşılabilir. Bu kutsal doğa kendisini Tanrı’nın evladı olarak değerlendiren ve Cennet Yaratanı’na gerçek bir ruhani Yaratıcı olarak atfeden insan tarafından da en iyi bir biçimde anlaşılabilir.
\vs p002 0:2 Tanrı’nın doğası yüce düşüncelerin kendini açığa çıkarmasının birinde bile irdelenebilir, bu kutsal karakter ruhani nihai amaçların bir tasviri olarak tahayyül edilebilir; fakat kutsal doğanın tüm kendini açığa çıkarışlarının ruhani öğretileri ve aydınlatışı, Nasıralı İsa’nın kutsallığın bütüncül bilincine varışının öncesini ve sonrasını kapsayan dinsel yaşamının idrakinde bulunur. Mikâil’in ete kemiğe büründürülen yaşamı Tanrı’nın insana bahşettiği gerçeğin açığa çıkarışına temel oluşturursa, Kâinatın Yaratıcısı’nın kişiliği ve karakterinin insani kavramsallaşmasını daha ileri düzeyde aydınlatacak ve bütünleştirecek belirli düşünceleri ve olası en yüksek amaçları insana ait olan kelimeler üzerine inşa etmeye çabalayabiliriz.
\vs p002 0:3 Tanrı’nın insan tarafından kavramsallaşmasını genişletecek ve ruhanileştirecek tüm gayretlerimizde bizler fani bir hayatı olan aklımızın sınırlı kabiliyeti tarafından çok etkili bir biçimde kısıtlanırız. Bizler aynı zamanda üzerimize düşen bu görevi yerine getirmemizde dilimizin sınırlı doğası tarafından ve kutsal değerleri ve ruhani anlamları sınırlı ve fani olan insan aklına sunma çabalarımızda tasvir etmek amacıyla kullanacağımız imgesel açıklamalardaki veya karşılaştırmalarındaki maddi yetersizlik tarafından ciddi bir biçimde engellenmekteyiz. Tanrı’nın insana ait kavramsallaşmasını genişletecek tüm çabalarımız, bahşedilen Kâinatın Yaratıcı’sının Düzenleyicisi’nin ikame ettiği ve Yaratan Evlat’ın Gerçek Ruhaniyeti’nin hüküm sürdüğü ölümlü insan aklının gerçekliği dışında neredeyse tamamen kısırdır. Bunların sonucunda, insan kalbi içerisinde bahsi geçen kutsal ruhların mevcudiyeti üzerine Tanrı kavramsallaştırmasının genişlemesi amacıyla ihtiyaç duyulan destek için; Tanrı’nın doğasının daha kapsamlı bir tasvirini ortaya koyacak çabayı bu bağlamda tarafıma tahsis edilen vazifeyi memnuniyetle uygulayarak göstereceğim.
\usection{1.\bibnobreakspace Tanrı’nın Sınırsızlığı}
\vs p002 1:1 “Sınırsızlıkta hüküm süren Onu biz tam anlamıyla idrak edemeyiz. Onu sınırsızlığa götüren kutsal yol takip edilemezdir.” “Onun sınırsız anlayışı ve onun büyüklüğü irdelenemezdir.” Tanrı’nın mevcudiyetinin göz kamaştırıcı ışığı onun düşük derecede bulunan yaratılanlarının ikame ettiği zifiri karanlığı karşısında kör edici bir aydınlıktır. Onun düşünceleri veya planları yalnızca araştırılamaz olması sebebiyle irdelenemez değildir, aynı zamanda “Onun sayısız muhteşem ve eşsizliklere imzasını atması” bakımından o tamamiyle takip edilemez. “Tanrı muhteşemdir; ne biz onu tam anlamıyla algılayabiliriz, ne de onun yaşamına dair seneler numaralara dökülebilir.” “Tanrı gerçekten dünya üzerinde ikame ediyor mu?” biçiminde yöneltilen sorulara “İşte, Onu cennet (evren) ve cennetlerin tümü (kâinatın âlemlerinin tümü) bile içine sığdıramıyor.” “Onun yargılarının ve geçmişinin izlerinin ne kadar algılanamaz ve irdelenemez olduğunu artık siz hesap edin!”
\vs p002 1:2 “Aynı zamanda inançlı bir Yaratan olarak sınırsız olan Yaratıcı tek bir Tanrı’dır.” “Kutsal Yaratan aynı zamanda ruhların kökeni ve kaderi olan Evrensel Bahşedici’dir. Kendisi Aşkın Ruh, Ezeli Akıl ve tüm yaratılmışların Sınırsız Ruhaniyeti’dir.” “Bahsettiğimiz bu muazzam Düzenleyici hiçbir hataya yer vermez. Kendisi görkeminde ve ihtişamında göz kamaştırıcıdır.” “Yaratan Tanrı tüm korkulardan ve düşmanlıklardan tamamen arınmıştır. Kendisi ölümsüz, ebedi, kendiliğinden var olan, kutsal ve cömerttir.” “Kendisi nasıl da saf ve güzel, derin ve sırrına erişilemez olan her şeyin tanrısal Ata’sıdır!” “Onun kendisinin insanlığa olan takdiminde ve iletişiminde Sınırsız en mükemmel yapısında bulunur. O başlangıç ve sondur, her iyiliğin ve kusursuz niyetin Yaratıcısı’dır.” “Tanrı ile her şey mümkün hale gelir; ebedi Yaratan tüm sebeplerin kaynaklandığı tek kökendir.”
\vs p002 1:3 Yaratıcı’nın ebedi ve evrensel kişiliğinin kendini dışavurumunun göz alıcı görkeminin sınırsızlığı yanı sıra tıpkı tamamiyle kendi kusursuzluğu ve gücünün bilgisine sahip olduğu gibi kendisi koşulsuz bir biçimde kendi sınırsızlığı ve ebediyetinin bilincindedir. O kendisinin kusursuz, düzgün ve tamamlanmış bir değerlendirmesini kendi kutsal yardımcılarından bağımsız bir biçimde deneyimleyen kâinattaki tek varoluştur.
\vs p002 1:4 Yaratıcı sürekli ve hatasız olarak onun asli evreninin farklı birçok bölümünde zaman zaman kendisi için farklılaşan ihtiyaçlara göre değişimleri karşılar. Muhteşem olan Tanrı kendi mevcudiyetinin bütünsel anlamının farkında ve ayırdında; sınırsız bir biçimde kusursuzluğun içerisinde onun asli niteliklerinin bütününün bilincindedir. Ne tanrı kozmik bir tesadüften kaynaklanan bir oluşum, ne de kendisi bir evren tarafından kullanılan bir deneyimleyici değildir. Evren Egemenleri bu tür maceraların içinde olabilirler; Takımyıldız Yaratıcıları deneyimleyebilir; düzenin başlarında bulunanlar uygulayabilirler; fakat Kainatın Yaratıcısı kâinata dair her şeyi zaman bakımından başından sonuna apaçık bir biçimde görür. Bununla birlikte, onun kutsal tasarıları ve ebedi amacı, kendi çok çeşitli nüfuzuna ait her dünyada, düzende ve her evrende kendisi için hizmet veren tüm yardımcılarının deneyimlemelerini ve serüvenlerini gerçekte kapsar ve onları kavrar.
\vs p002 1:5 Tanrı için hiçbir şey yeni değildir, ve hiçbir kozmik durum ona hiçbir zaman bir sürpriz olarak gelmez; çünkü kendisi ebediyetin döngüsünde ikamet eder. O günlerin başlangıcından veya sonundan tamamiyle bağımsızdır. Tanrı’ya göre, ne geçmiş, ne şimdiki zaman veya gelecek bulunmaktadır; bu üçünü kapsayan zamanın kendisi için her an her saniye herhangi bir kısıtlamadan bağımsız, ‘an’a ait olan mevcudiyettir. O harikulade ve o sadece BEN’dir.
\vs p002 1:6 Kâinatın Yaratıcısı kendisinin tüm niteliklerinde herhangi bir koşuldan bağımsız ve mutlak olarak sınırsızdır ve bu gerçeklik kendi içinde ve kendiliğinden sınırlı maddi canlılar ve diğer düşük seviyede yaratılmış akli varlıklar ile olan tüm dolaysız bireysel iletişimden kendisini doğal olarak ayırır.
\vs p002 1:7 Bunun yanı sıra bu durumun bütünsel varlığı, çok çeşitli yaratılmışlarıyla olan bu tür ilişki ve iletişimin düzenlenmesini emredildiği gibi ilk olarak Tanrı’nın Cennet Evlatları’nın kişiliklerinin varlığına olan ihtiyacı doğurur. Bu Evlatlar her ne kadar kutsal olarak kusursuz olursa olsun gezegen ırklarının bedeni ve kanının doğasından sıklıkla gözlenen bir biçimde bir parça paylaşarak, sizden biri gibi olarak sizinle birlikte olmak için, Mikâil’in bahşedilişinin değişik adlarla Tanrı’nın Evladı ve İnsanoğlundan Doğan olarak adlandırıldığı ve onun varlığının fani bir yaşamda gözlenişi gibi bu sebeple Tanrı insan haline gelir. Ve ikincil olarak, alt düzey kökenindeki maddi varlıklara yakınlaşan ve onlara birçok çeşitli biçimlerde yardımcı olan ve hizmet eden yüksek meleksel ev sahipleri ve diğer göksel akli varlıklarının çok çeşitli düzeylerinin oluşturduğu Sınırsız Ruhaniyet’in kişilikleri mevcuttur. Üçüncü ve sonuncu olarak bildirilmeden ve açıklanmadan Urantia’nın içinde ikame eden insanlarına muhteşem Tanrı’nın gerçek bir hediyesi olarak birey dışı Gizem Gözlemleyicileri ve Düşünce Denetleyicileri gönderilmiştir. Bitip tükenmeyen bollukta, onlar ihtişamın yücesinden Tanrı\hyp{}bilinci niteliğine sahip veya bu amacın potansiyeline haiz bu fanilerin alçak gönüllü akıllarında şükranlıkla ikamet etmek için indirilir.
\vs p002 1:8 Bu biçimlerde veya tarafınızdan bilinemeyecek biçimlerdeki veya tamamen sınırlı algılayışın çok ötesindeki yolların vasıtasıyla, Cennet Yaratıcısı sevgiyle ve tüm istenciyle bulunduğu konumdan aşağıya doğru ve doğasının aksine kendi sınırsızlığını kısıtlayarak, sınırlayarak ve kapsamını daraltarak kendi yarattığı çocuklarının sınırlı akıllarına daha fazla yakınlaşabilmek amacındadır. Ve böylelikle, mutlaklığın alt seviyelerinde birbirini izleyen kişilik dağıtımları sürecinde, sınırsız Yaratıcı uçsuz bucaksız kâinatının birçok âlemlerinin birbirinden farklı akıl sahipleriyle yakın ilişki kurabilmesinin memnuniyetine kavuşmuş olur.
\vs p002 1:9 Onun şu ana kadar gerçekleştirmiş, şu an yapmakta ve gelecekte eylemlerini daha fazla bir biçimde sürdürecek olduğu tüm bu faaliyetleri onun sınırsızlığının, ebediyetinin ve üstünlüğünün gerçekliği veya doğruluğunu bir nebze olsun değiştirmez. Buna ek olarak ona ait tüm bu nitelikler her ne kadar onların algılamalarında yaşadıkları zorluğa, bu gizemin görülmeyecek bir biçimde örtülü oluşuna ve Urantia’da ikamet eden yaratılmışların deneyimlediği gibi onlar tarafından tamamen anlaşılmasının imkânsızlığına rağmen tamamen gerçektir.
\vs p002 1:10 Öncül Yaratıcı’nın kendi tasarılarında sınırsız oluşundan ve kendi niyetlerinde ebedi olmasından dolayı herhangi sınırlı bir varlığın bu tasarıları ve niyetleri tam anlamıyla algılaması veya onları kavraması bu durumun doğası gereği mümkün değildir. Fani olan insan Tanrı’nın amaçlarını evren işleyişinin birbirini takip eden seviyelerinde yaratılmışın göğe çıkışının tamamlanmış tasarılarıyla ilişkili açığa çıkarıldığı biçimde anlık bir sınırlı kapsamla algılayabilir. Sınırsız Yaratıcı en yüksek kesinlikle, sevgiyle ve bütünsel bir kavramayla tüm âlemlerdeki tüm çocuklarının sınırlılığıyla bütünleşirken insanın nitelikleri bu derecedeki sınırsızlığı kapsayamaz.
\vs p002 1:11 Yaratıcı’nın daha yüksek derecedeki Cennet varlıklarının büyük bir kısmıyla paylaştığı kutsallık ve ebediyet karşısında biz sınırsızlığın ve bunun sonucundaki evrensel üstünlüğün Cennetin Kutsal Üçlemesi’ndeki eş güdümü sağlayan herhangi bir yardımcısı tarafından tam olarak paylaşılıp saklandığı ve bunun olmadığını sorgularız. Kişiliğin sınırsızlığı kişiliğin tüm sınırlarıyla bütünleşmek zorunda ve durumundadır; bu sebeple kelimenin tam anlamıyla bir doğruluk olarak öğretinin gerçekliği “biz Onun içinde yaşar, hareket eder ve varoluşumuza sahip oluruz” sözünü haykırır. Kâinatın Yaratıcısı’nın saf İlahiyatı’nın fani insanda ikamet eden bu nüvesi Yaratıcıların Yaratanı olan İlk Muhteşem Kaynak ve Merkezin sınırsızlığının bir \bibemph{parçasıdır}.
\usection{2.\bibnobreakspace Yaratıcı’nın Ebedi Kusursuzluğu}
\vs p002 2:1 Geçmiş dönemlerdeki peygamberleriniz bile ebedi, başlangıcı ve sonu olmayan Kâinatın Yaratıcısı’nın döngüsel doğasını anlamışlardı. Tanrı kelimenin tam anlamıyla ebedi bir biçimde kendisinin yarattığı kâinatın âlemlerinin tümünün içinde mevcut bulunmaktadır. Mutlak ihtişamıyla ve ebedi büyüklüğüyle O akıp giden her ‘an’a ikamet eder. “Tanrı kendi içerisinde yaşama sahiptir ve bu yaşam ebedidir.” Ebedi çağlar boyunca “tüm yaşamı bahşeden” Yaratıcı kendisi olmuştur. Kutsal bütünlükte sınırsız kusursuzluk bulunur. “Ben Koruyucunuz’um, ve ben değişmeyenim.” Kâinatın âlemlerinin tümü hakkındaki bilgimiz aydınlığın Yaratıcısı gerçeğini açığa çıkarmaz, fakat aynı zamanda gezegensel arası ilişkiler içerisinde kendi faaliyetlerinde “ne değişkenler ne de değişimin gölgesi mevcuttur.” O “sonu başından itibaren bildirir.” Kendisi “Evladım’ın mevcudiyetinde amaçladığım ebedi niyetin ışığında” “benim tavsiyemin geçerliliğini sürdürmesi için elimden gelen her şeyi yapacağım.” Bu sebeple İlk Kaynak ve Merkez’in tasarıları ve niyetleri tıpkı kendisi gibi ebedi, kusursuz ve sonsuza kadar değişmezdir.
\vs p002 2:2 Yaratıcı’nın hâkimiyeti altında tamamlanmışlığın kusursuzluğu ve bütünlüğünün kesinliği vardır. “Tanrı ne yaparsa yapsın, onun yaptığı sonsuza kadar baki kalacaktır; buna ne bir şey eklenebilir ne de ondan bir şey alınabilir.” Kâinatın Yaratıcısı erdemin ve kusursuzluğun içindeki kendi özgün niyetleri hakkında pişmanlık duymaz veya geri adım atmaz. Onun hareketleri kutsal ve hatasızken kendisinin tasarıları değişmez, tavsiyeleri ise sabittir. “Onun bir bakışıyla bin yıllık bir süreç dün kadar yakındır, ve bu süreç tamamlandığında geçen süre gece içerisinde birkaç saatlik bir nöbet vakti kadardır.” Kutsallığın kusursuzluğu ve ebediyetin büyüklüğünün şiddeti fani insanın sınırlar içerisinde faaliyet gösteren aklının bütüncül algısının sonsuza kadar ötesindedir.
\vs p002 2:3 Değişmeyen bir Tanrı’nın kendi ebedi niyetinin uygulamalarındaki tepkimeleri onun yarattığı akıl sahibi yaratılmışların mantık dönüşümleri ve bakış açılarının değişimiyle birlikte farklılık gösteriyormuş gibi görünür. Bu durum açık ve doğal olmayan bir biçimde farklılığa uğruyormuş gibidir, fakat tüm çevreye doğru olan dışavurumların altında ve onların görünen yüzeyinin aşağısında ebedi Tanrı’nın ezeli tasarısı olan değişmeyen niyetinin varlığı sürekli olarak yatar.
\vs p002 2:4 Yaşadığınız evrenin dışındaki diğer âlemlerin içinde kusursuzluk doğası gereği göreceli bir kavramsallaşmayı tanımlar, fakat merkezi evrende ve özellikle Cennet üzerinde kusursuzluk katışıksız; ve hatta belirli fazlarda ise mutlaktır. Kutsal Üçleme’nin dışavurumları kutsal kusursuzluğun kendisini yansıtmasıyla değişkenlik gösterebilir, fakat bu durum onun bu yöndeki kendi niteliğinden bir şey kaybettirmez.
\vs p002 2:5 Tanrı’nın asli kusursuzluğu varsayılan bir doğruluk üzerinde oluşmaz, bunun yerine onun kutsal doğasının varlığının kendisiden gelen iyiliğinin kusursuzluğunda kendisine bir yer bulur. O nihai, tamamlanmış ve kusursuzdur. Onun doğruluğu simgeleyen karakterinin kusursuzluğu ve güzelliğinden hiçbir şey eksik değildir. Bununla birlikte, yaşayan varlıkların tüm uyumsal düzeni, boşluk üzerindeki dünyalarda tüm idrak sahibi yaratılmışların ruhu temiz tutan kutsal niyetinde Yaratıcı’nın Cennetsel kusursuzluğunun paylaşımcı deneyiminin yüksek nihai sonuna ulaşması etrafında merkezileştirilmiştir. Tanrı ne ben\hyp{}merkezci ne de kimsenin varlığına ihtiyaç duymayacak kadar kibirli bir tavırla kendinden müstakildir; bunun yerine uçsuz bucaksız olan kâinat âlemlerinin tümünün öz bilinç sahibi yaratılmışlarına karşı güzelliklerini bahşetmekten kendisini hiçbir zaman alamaz.
\vs p002 2:6 Tanrı ebedi ve sınırsız olarak kusursuzdur, kendi deneyimlerinde kişisel olarak kusurluluğun ne olduğunu bilemez, fakat Cennetin Yaratan Evlatları’nın evrimsel âlemleri içerisinde kusursuzluğa ulaşmak için çaba gösteren yaratılmışların hepsinin kusursuz olmayan deneyimlerinin bütününün bilincini paylaşır. Tanrı’nın kişisel ve özgürleştirici kusursuzluğunun etkisi ahlaki farkındalığın evrensel seviyesine yükselen tüm bu fani yaratılmışların kalplerini kapsamı içine alır ve onların doğalarını bütünlüğü içerisinde çevreler. Bu bağlamda, kutsal mevcudiyetin ilişki içerisindeki olanların vasıtasıyla birlikte Kâinatın Yaratıcısı tüm kâinatın her ahlaki varlığının evrimleşen gelişiminde onların \bibemph{olgunlaşmamışlıkla} ve kusurlulukla geçen deneyimleme sürecinin içerisindedir.
\vs p002 2:7 İnsana ait sınırlılıklar ve onlara atfedilebilecek olası kötülükler kutsal doğanın bir parçası değildir, fakat fani deneyimlemelerin \bibemph{kötülükle olan etkileşimi} ve tüm insan ilişkileri, Tanrı’nın Cennet’ten çıkmış her Yaratan Evlat tarafından yaratılmış ve evrimleşmiş ahlak sorumluluğuna sahip yaratılanlar olarak zamanın evlatları içerisinde onun ezelden beri genişleyen kendisini gerçekleştirmesinin çok kesin bir parçasıdır.
\usection{3.\bibnobreakspace Adalet ve Doğruluk}
\vs p002 3:1 Tanrı’nın kendisi başlı başına doğruluktur; bu sebeple kendisi adaletin timsalidir. “Koruyucu her bir biçimde doğruluktur.” Koruyucu bu bağlamda “Ben şu ana kadar yaptığım hiçbir şeyi bir sebebe dayanmadan gerçekleştirmedim,” sözünü söyler. Kâinatın Yaratıcısı’nın adaleti onun yaratılmışlarının faaliyetleri veya davranışları tarafından etkilenmez, Koruyucu olan Tanrımız’a atfedilecek hiçbir kötülük ve adaletsizlik yoktur, O ne kimseden kendisine gösterilecek saygıya ne de onların hediyelerine muhtaçtır.”
\vs p002 3:2 Tanrı’ya, onun değişmeyen kurallarına, arkasında derin bir zekâ bulunan doğa yasaların ve adil ruhani çalışanlarının işleyişinden kaynaklanan sonuçları değiştirmek için ona çocukça karşı gelmek ne kadar da gereksiz bir uğraştır! “Kendinizi kandırmayın; Tanrı aldanmaz, ve insan neyi ekerse onu biçer.” Aynı zamanda yanlış olan bir ekimin adalet içerisindeki biçimi bile ılımanlaştırıcı bir bağışlanmayla kendisini gösterir. Sınırsız olan bilgelik hangi durumda olursa olsun uygulanacak adalet ve bağışlanmanın oranını belirleyecek ebedi karar verici ve arabulucudur. Tanrı’nın hükümranlığına karşı gösterilecek kararlı kötülüğün ve kasıtlı isyanın karşılığında kaçınılmaz bir sonuç olan en yüksek derecedeki ceza hükümranlığın altında barınan bir bireysel öznenin mevcudiyetini hiç yaşanmamış gibi kaybetmesidir. Bilinçli olarak yapılan bir günahın nihai sonucu onun hiç oluşmamış gibi ortadan kaldırılmasıdır. Son kertede, böyle günah ile özleşen bireyler ahlak dışılıkla bütünleşerek tamamen gerçek\hyp{}dışı haline gelerek kendilerini yok ederler. Fakat, böyle bir yaratılmışlığın bilgiye dayanan bu ortadan kayboluşu mevcut evrenin tamamen bağlı olduğu adaletin hüküm vereceği karara kadar ertelenir.
\vs p002 3:3 Varlığın kesin bir biçimde sona ermesi genellikle âlemlerin veya âlemin çağsal veya yazgı dönemi yargısında karara varılır. Urantia gibi bir dünya üzerinde bu karar gezegensel bir yagı döneminin sonucunda ortaya çıkar. Varlığın mevcudiyetinin son buluşu gezegensel kuruldan Yaratan Evladı’nın mahkemeleri boyunca Zamanın Ataları’nın karar alıcı adli yapılarına uzanan yargı mahkemelerinin yardımcı faaliyetleri tarafından bu tür zamanlarda karara vardırılır. Yok oluşun emri, suç işleyen bireyin ikamet ettiği âlemde suçlama hakkında değiştirilemez yargıya varılmasının ardından aşkın evrenin daha yüksek seviyedeki mahkemeleri tarafından oluşturulur. Bunun sonucunda ortandan kaldırılmasına karar verilen varoluşun ceza kararı yüksek mahkemede onaylanınca, aşkın âlemin yönetim merkezinden ve orada ikamet ederek faaliyetlerde bulunan hâkimlerin doğrudan eylemleriyle bu yok oluş yerine getirilir.
\vs p002 3:4 Bu karar kesin olarak onaylanınca, suçlanan kötülükle tanımlanan varlık anında sanki hiç yaşamamış gibi yok olur. Böyle bir kaderden tekrar diriliş söz konusu değildir; bu karar sonsuza kadar bağlayıcı ve ebedidir. Kimliğin yaşayan enerji yapıları kozmik potansiyele olan zamanın dönüşümü ve mekânın başkalaşımı tarafından onların ilk olarak ortaya çıktıkları yerden çözülmeye ve yok olmaya başlar. Kötülük sahibi olanın kişiliği hakkında, yaratılmışlığının ebedi yaşamı sağlayacağı tercihleri ve kararları alma konusundaki başarısızlığı tarafından devam etmesi beklenen yaşamından mahrum kalma durumu söz konusudur. Günah ile devam eden bütünleşme onun etkileşimde bulunduğu akıl tarafından kötülük ile bütüncül bir kişilik birleşimine ulaşmasınin ardından; hayatın sona ermesinin kozmik yok oluşunu deneyimleyen böyle bir soyutlanmış kişilik Yüce Varlık’ın evrimleşen deneyiminin bir parçası haline gelerek yaratılanın üst ruhu tarafından içine alınarak soğurulur. Bir daha hiçbir biçimde kişilik olarak tekrar açığa çıkmaz; onun kimliği sanki hiç oluşmamış gibi ortadan çekilir. Düzenleyici\hyp{}ikamesine sahip bir kişiliğin böyle bir durumunda deneyimsel ruhaniyet, yaşamasını sürdüren Düzenleyici’nin gerçekliğin hayatına devam ettirmesini önemser ve bu olanağı ona sağlar.
\vs p002 3:5 Gerçekliğin mevcut düzeyleri arasında herhangi bir evren mücadelesinde, daha yüksek seviyedeki kişilik daha alt düzey kişilikle olan yarışından zaferle ayrılacaktır. Evrenin bu tartışıla gelen engel olunamaz sonucu, herhangi bir irade sahibi yaratılmışın mevcudiyeti veya gerçekliğinin derecesine eşit nitelikli kutsallığın doğru olan bilgisinin doğasında bulunur. Kesinleşmiş bütünlükte bir hata, iradeyle işlenen bir günah ve mutlak bir ahlak dışı hareket olarak katışıksız kötülük doğası gereği ve kendiliğinden olarak intihar vakasıdır. Kozmik gerçek dışılığının bu tür davranışları, sadece evren içerisinde doğruluk yargısının âlem mahkemeleri işleyişi içerisinde adalet\hyp{}sağlayıcı ve hakkaniyet\hyp{}bulucu eylemlerini bekleyen kısa süreli bağışlayıcı\hyp{}merhameti sayesinde barınabilir.
\vs p002 3:6 Yaratan Evlatlar’ın yerel evrenlerdeki yönetimi ruhaniyetin ve yaratılmışın bir örneğidir. Bu Evlatlar fani yükselişin Cennet tasarılarının etkili uygulamalarına ve yanlış ve tarumar edici bir isyankârlıkla düşünenlerin geri kazanımlarına kendilerini adamışlardır, fakat onların tüm bu emekleri kesin ve sonsuza kadar reddedildiği zaman yok oluşun son kararı Zamanın Ataları’nın yetki alanı altında hareket eden kuvvetler tarafından uygulanır.
\usection{4.\bibnobreakspace Kutsal Bağışlama}
\vs p002 4:1 Bağışlama, sınırlı yaratılmışların çevresel kusurlarından ve doğal zayıflılığının bir bütün olarak tanınmasından ve kusursuz bilgiden türeyen adalet temelli bir bilgeliktir. “Tanrımız merhamette, kutsal lütufta ve nezakette, ümit dolu sabırlı bekleyişte sınır tanımaz, ve kendisi bağışlayıcılıkta oldukça cömerttir.” Bu sebeple, “her kim en sonunda Koruyucu’nun yardımına ihtiyaç duyarsa, o kişi onun tarafından kollanacaktır,” “ve bunun için yardıma ihtiyaç duyan kişi fazlasıyla affedilecektir.” “Koruyucu’nun bağışlaması sonsuzluktan geldiği için onun etkisi de sonsuzdur”; bu sebeple kuşkusuzdur ki “onun bağışlaması ebediyete kadar dayanır.” “Ben sevgi\hyp{}dolu iyi niyeti, adaleti ve doğruluğu dünya üzerinde uygulayan ve bunların içinde büyük memnuniyet duyan Koruyucu’yum.” “Ben kasıtlı olarak veya iradem dâhilinde insan evlatlarına ne zarar verebilirim ne de onlara ıstırap çektirebilirim,” bunların karşısında ben “bütüncül bir huzurun Tanrı’sı ve bağışlamanın Yaratıcısı’yım.”
\vs p002 4:2 Tanrı özü gereği sıcak, doğası gereği merhamet sahibi, ve etkisi sonsuza kadar sürecek ölçüde bağışlayıcıdır. Buna ek olarak, Tanrı’nın sevgi\hyp{}dolu sıcaklığını herhangi bir etkiyle harekete geçirmek gibi gereklilik hiçbir zaman söz konusu bile değildir, onun bu sevgi dolu doğası kesinlikle koşulsuzluk ve süreklilik arz eder. Yaratılanın ona olan ihtiyacı Yaratıcı’nın incelikli bağışlaması ve onun kurtarıcı lütfunun eksiksiz deviniminin varlığına sebep teşkil eder. Tanrı kendi çocukları ile ilgili her şeyi bilmesi sebebiyle onun için onları affetmek hiçbir zorluk yaratmaz. Tıpkı, insanın komşusunu daha iyi anladığında onu affetmesi ve hatta ona sevgi beslemesinin önü rahatlıkla açıldığı gibi Yaratan’ın kendi çocuklarını bilmesi için de aynı durum söz konusudur.
\vs p002 4:3 Yalnızca sınırsız bilgeliğin algısı, hangi evren durumunda olursa olsun adil bir Tanrı’yı aynı zamanda hem adalete yardımcı olmak ve hem de bağışlamak için harekete geçirir. Cennetsel Yaratıcı kendi evren çocuklarına karşı birbiriyle çelişen davranışlarla bölünmez; Tanrı kişisel düşmanlıkların hiçbir zaman kurbanı olmaz. Tanrı’nın her şeyi biliyor oluşu, onun ebedi doğasının sınırsız niteliklerini ve kutsal özelliklerinin tümünün isteklerine eşit bir ölçüde anında ve kusursuzca cevap verecek bir evren faaliyetini tercih edecek bir biçimde onun özgür iradesini kusursuz bir biçimde yönlendirir.
\vs p002 4:4 Bağışlama, sevginin ve iyiliğin doğal ve karşı konulamaz bir biçimde doğumudur. Sevgi dolu bir Yaratıcı’nın iyi olan doğası, onun evren çocuklarının herhangi bir biriminin herhangi bir üyesine bağışlamanın akıllıca yardımını göstermemeyi hiçbir olasılık dâhilinde kabul etmez. Ebedi adalet ve kutsal bağışlama insan deneyimlerinde birleşerek \bibemph{adiliyet} adı verilen yapıyı oluştururlar.
\vs p002 4:5 Kutsal bağışlama, kusurluluğun ve kusursuzluğun evren düzeyleri arasındaki adiliyetin düzenlemesinin bir araçsallığını yansıtır. Bağışlama, evrimleşen sınırlılığın durumlarına uyum sağlayan Yücelik’in adaleti ve ebediyetin doğruluğunun zamanın çocuklarının evrensel refahını ve en yüksek seviyedeki amaçlarını karşılamak için değişim geçirmesidir. Bağışlama adaletin içinde onun mevcudiyetine bir tezat oluşturmaz, fakat bunun yerine evrimleşen evrenlerin maddi yaratılmışlarına ve emir altında çalışan ruhani varlıklara adil bir biçimde uygulandığı biçimiyle yüksek yargının isteklerinin yorumlanmasının bir anlayışıdır. Bağışlama, Kâinatın Yaratıcısı ve onun tüm yardımcı Yaratanları’nın egemen özgür iradesi ve bilgiye sahip aklı tarafından belirlenmiş ve kutsal bilgelik tarafından tasarlanmış haliyle, zaman ve mekân yaratılmışlarının çok çeşitli akli yapılarını akıllıca ve sevgiyle ziyaret eden Cennetin Kutsal Üçlemesi’nin adaletidir.
\usection{5.\bibnobreakspace Tanrı’nın Sevgisi}
\vs p002 5:1 “Tanrı derin sevginin ta kendisidir”; bu sebeple onun evren olaylarına karşı kişisel tutumu her zaman kutsal sevgi ve şefkatin bir karşılığıdır. Yaratıcı kendi hayatını bizlere bahşedecek kadar bizleri çok sevmektedir. “Kötülüğün ve iyiliğin üstüne kendi güneşini doğurur, ve adil ve adil olmayanların üstüne yağmuru gönderen O’dur.
\vs p002 5:2 Kendi Evlatları’nın fedakârlıkları yüzünden yâda onun emri altındaki yaratılmışların “Yaratıcı’nın sizi sevmesinden dolayı” sizin adına dua etmesi sebebiyle Tanrı’nın biz çocuklarına dair olan sevgisine dışsal yollardan çekilmesini düşünmek yanlışlık olur. Bunun yerine Tanrı kendi içinden gelen babalık sevgisi karşısında ihtişamlı Düzenleyicileri insan aklına ikamet etmesi için göndermiştir. Tanrı’nın sevgisi evrenseldir; “kim olursa olsun herkes bu sevgiye doğru yönelebilir.” Kendisi “tüm insanların bu gerçeğin bilgisine doğru yönelerek kurtulmasını arzu eder.” O “hiç kimsenin ortadan yok olmasını arzulamaz.”
\vs p002 5:3 Yaratanlar, kutsal kanunların insanlar tarafından mantıksız bir biçimde uyulmamasından doğan zarar verici etkilerden onları koruyamaya ilk elden çabalamakla yükümlüdürler. Tanrı’nın sevgisi doğası gereği koruyucu ve kollayıcı olan bir baba şefkati gibidir; bu sebeple insan bazı zamanlar “bizim doğru yolu bulmamızdan doğan kazancımız için bizi ıslah etki biz de senin kutsallığından nasibini alanlardan olalım.” En şiddetli sıkıntılarınızda bile “tüm ıstırap ve acılarınızı onunda sizinle birlikte hissettiğini” unutmayın.
\vs p002 5:4 Tanrı kutsal bir biçimde günahkârlara bile sıcak ve samimidir. Ona isyan edenler doğruluk yoluna döndüklerinde, “bizim Tanrı’mız bizleri fazlasıyla affedecektir” gerçeğini bağışlanmayla deneyimler. “Ben sizin kurallarımın dışına çıkmanızı bilerek kendim için görmezden gelen ve günahlarınızı hatırlamayacak olan Yaratıcı'yım.” “Yaratıcı’nın nasıl bir anlamda bir sevgiyi bizlere bahşettiğinin işte ayırdında ol ki bizim Tanrı’nın evlatları olarak anılmamız gerektiğini anlayabilesin.”
\vs p002 5:5 Son kertede, onu sevmenin en yüce sebebi ve Tanrı’nın iyiliğinin en mükemmel kanıtı Tanrı’nın sizde ikamet eden hediyesi olan, siz onunla birlikte ebedi haline geldiğiniz saati oldukça sabırlı bir biçimde bekleyen Düzenleyici’dir. Her ne kadar siz Tanrı’yı fiziksel bir biçimde arayarak bulamasanız da, kendinizi bu ikamet eden ruhaniyetin öncülüğüne teslim ettiğiniz zaman; Kâinatın Yaratıcısı’nın Cennet kişiliğinin mevcudiyetinde en sonunda durmaya başlayacağınız vakte kadar âlemler ve çağlar boyunca kusursuz bir biçimde adım adım yaşamın her düzeyinde onun rehberliğinde ilerleyeceksiniz.
\vs p002 5:6 Kısıtlı olan insan doğası sebebiyle ve sizin maddi yaratılmışlığınızdan gelen kusurların onu görmenizi imkânsız hale getirmesinin sizin Tanrı’ya ibadet edemeyeceğiniz gibi bir sonuca varması ne kadar kabul edilemez bir mantıksal yargıdır. Sizinle Tanrı arasında katetmek için fiziksel boşluktan kaynaklanan korkunç bir uzaklık bulunmaktadır. Aynı zamanda buna benzer olarak birbirine bağlanması için aranızda büyük bir ruhani farklılaşmadan kaynaklanan boşluk bulunur, fakat sizi Tanrı’nın Cenneti kişiliğinin mevcudiyetinden fiziksel ve ruhani olarak ayıran bu farklılaşma karşısında durun ve Tanrı’nın sizin içinizde yaşadığınız kutsal bilgisini düşünün. Tanrı kendisine ait olan biçimlerde çoktan bu farklılaşmalardan doğan boşluğu doldurur ve sizi kendisine bağlar. O, kendi mevcudiyetinden bir parça olan kendi ruhunu, sizinle birlikte yaşaması ve zahmetlere sizlerle birlikte göğüs germek için göndermiştir.
\vs p002 5:7 Ben, onun düşük düzeydeki yaratılmışlarının yüceltici yardımına oldukça sevgi dolu bir biçimde adanmış ve aynı zamanda bu kadar mükemmel olan birine ibadet etmeyi çok kolay ve çok memnuniyet verici olarak buluyorum. Ben, yaratmada ve bu sebeple düzenlemede bu derece güçlü olan birini, ve aynı zamanda hiç durmadan bizi kapsamına alan sevgi\hyp{}dolu sıcaklıkta bu kadar iyi ve inanç dolu olan oldukça kusursuz birini doğal olarak çok seviyorum. Düşündüğümde şayet o bu kadar mükemmel ve güçlü olmasaydı fakat böyle iyi ve bağışlayıcı olsa bile Tanrı’yı yine bu kadar severdim. Hepimiz Yaratıcı’yı, onun harikulade güçsel özelliklerinin farkına varmamızdan daha çok onun doğasının büyüleyici güzelliği sebebiyle ona sevgiyle bağlıyız.
\vs p002 5:8 Mekânın evrenlerinin evriminin doğasında olan çeşitli birçok zorluklarla cesurca başa çıkmak için uğraş veren Yaratan Evlatlar ve onların emri altında olan yöneticileri gözlemlediğimde bu âlemlerin yardımcı yönetenlerini büyük ve derin bir sevgiyle karşıladığımı keşfettim. Tüm bunların sonucunda, tüm âlemlerin faniler dâhil hepimiz Kâinatın Yaratıcısı’nı ve kutsal olan veya insani bir hüviyette olan tüm diğer varlıkları çok seviyoruz, çünkü biz bu kişiliklerin bizleri çok sevdiklerini algılıyoruz. Böyle bir sevginin bu deneyimi sevilmenin doğrudan bir karşılığıyla oldukça iniltilidir. Eğer Tanrı mutlak, nihai ve yüce olan tüm özelliklerinden mahrum olsaydı bile Tanrı’nın beni çok sevdiğini bildiğim için onu çok yüce bir biçimde sevmeye devam ederdim.
\vs p002 5:9 Yaratıcı’nın sevgisi bizi şu an ve ebedi çağların bitip tükenmek bilmeyen döngüsü boyunca takip etmeye devam eder. Tanrı’nın sevgi dolu doğasını derin bir biçimde düşündüğünüzde sadece tek bir mantıklı ve doğal kişilik karşılığı olduğunu göreceksiniz. Bu ise sizin artan bir biçimde sizi Yaratan’ı çok seveceğiniz; bir çocuğun dünyevi ebeveynine karşı gösterdiği karşılaştırılabilir bir sevgiyi ve bağlılığı Tanrı’ya karşı açığa çıkaracağınız; ve gerçek, doğru ve olması gerektiği gibi bir babanın çocuklarını sevmesi gibi Kainatın Yaratıcısı’nın sevmesi ve sonsuza kadar kendi yarattığı oğulların ve çocukların refahını arzulaması gerçeğidir.
\vs p002 5:10 Fakat Tanrı’nın sevgisi uzak görüşlü ve mantıksal bir ebeveyn sevgisidir. Kutsal Yaratıcı’nın kusursuz doğasının tüm diğer sınırsız nitelikleri ve kutsal bilgeliğiyle birlikte bütünsel olarak faaliyet gösterir. Tanrı başlı başına bir sevgidir, fakat sevgi tek başına Tanrı değildir. Kutsal sevginin fani varlıklar için en büyük dışavurumu Düşünce Denetleyicileri’nin bahşedilişinde gözlemlenir, fakat Yaratıcı’nın sevgisinin sizin için bahşedilmiş en muhteşem açığa çıkarılışı dünya üzerinde en nihai ruhani hayatı yaşamış olan Mikâil Evladı’nın armağan edilmiş yaşamında görülür. Tanrı’nın sevgisini her insan ruhu için kişiselleştiren içsel olarak barınan Düzenleyici’dir.
\vs p002 5:11 Bir insan kelimesinin sembolik dışavurumu olan \bibemph{sevginin} kullanımı vasıtasıyla cennetsel Yaratıcı’nın kendi kâinat çocukları için beslediği kutsal şefkati tasvir etmeye çalışmak konusunda hissettiğim zorluk karşısında bazen neredeyse acı hissettim. Bu kavram her ne kadar saygının ve bağlılığın fani ilişkileri içinde insanın en yüce kavramsallaşmasını tam olarak karşılasa da; bu kavram aynı zamanda sıklıkla, tamamiyle soylu olmayan ve bütünüyle uygunsuz herhangi bir kelimeyle de bilinebilecek olan insan ilişkilerinin birçoğuna atıfla ve yaşayan Tanrı’nın kendi kainatın yaratılmışları için karşılaştırılamaz şefkatinin tanımlayıcısı olarak da kullanılmaktadır! Ne kadar talihsizdir ki, Cennet Yaratıcısı’nın kutsal sevgisinin harikulade güzel önemini ve gerçek doğasını insan aklında yer edecek bir biçimde yüce ve tamamiyle sıra dışı başka bir kavramı kullanamıyorum.
\vs p002 5:12 İnsan bir kişisel Tanrı’nın sevgisini gözden kaçırırsa ona Tanrı’nın hükümranlığı sadece iyiliğin hükümranlığı gibi görünmeye başlar. Kutsal doğanın sınırsız bütünlüğünün yanı sıra Tanrı’nın kendi yaratılmışlarıyla ilgili kişisel ilişkilerinin baskın karakteri yine sevgidir.
\usection{6.\bibnobreakspace Tanrı’nın İyiliği}
\vs p002 6:1 Fiziksel âlemde biz kutsal güzelliği görebilir, akli dünyada ebedi gerçeği anlayabiliriz, fakat Tanrı’nın iyiliği sadece bireyin dinsel deneyimlerin ruhani dünyasında bulunabilir. Gerçek özünde din Tanrı’nın iyiliğine olan bir inanç\hyp{}güvenidir. Tanrı mükemmel ve mutlak, felsefi olarak akıl sahibi ve kişisel olabilir, fakat dinsel olarak Tanrı’nın ahlaki olması ve bunun sonucunda iyi olması gerekir. Tanrı’nın iyiliği Tanrı kişiliğinin bir parçası olup, onun bir bütün olarak kendini gerçekleştirmesi sadece Tanrı’nın inanç sahibi evlatlarının bireysel olan dini deneyimlerinde açığa çıkar.
\vs p002 6:2 Din, ruhani doğanın üstün dünyasının insan dünyasının temel gereksinimlerinin farkındalığı ve ona karşı gösterilen bir cevap niteliğindedir. Evrimsel din etik hale gelebilir, fakat sadece açığa çıkarılan din (vahiy edilen din) gerçek ve ruhani olarak ahlaki olabilir. Tanrı’nın geçmiş zamanlardaki kavramsallaşması hakimane bir ahlak anlayışı tarafından baskın bir konumda olan bir İlahiyat’tı. Bu buyurucu ahlak anlayışı İsa tarafından daha ince ve düşünceli ton ile, fani deneyimlerde daha sevecen ve güzel başka bir ilişkide bulunamayacak ebeveyn\hyp{}çocuk ilişkisi içerisinde sıcak aile ahlakının sevgi dolu etkileyici seviyesine getirildi.
\vs p002 6:3 “Tanrı’nın iyiliğinin zenginliği kusurlu insanı pişmanlığa kavuşturur.” “Her iyi bağış ve her kusursuz hediye aydınlanmanın Yaratıcı’sı tarafından gelir.” “Koruyucu Tanrı bağışlayıcı ve merhamet sahibidir. O gerçeklikte ve iyilikte fazlasıyla sabırlı ve cömerttir.” “Tanrı’nın iyiliğin kendisi olduğunu deneyimle ve gör! Kutsanan ona inanan insandır.” “Koruyucu inayetli ve tamamen merhamet sahibidir. O kötülüklerden arınmanın Tanrı’sıdır.” “O kırık kalbi iyileştirir, ve ruhun yaralarını sarar. O insanın tüm\hyp{}kuvvetinin kökeni ve sahibidir.”
\vs p002 6:4 Tanrı’nın bir buyurgan olarak kavramsallaşması her ne kadar yüksek ahlaki seviyeyi ileri bir noktaya taşıdıysa ve insanları yasalara saygılı bir zümre haline getirdiyse de, onları inançlı bireyler olarak zamanın ve ebediyetin içerisinde kendi düzeyi hakkında güven bunalımına sokan üzüntülü bir konumda bıraktı. Daha sonra gelen İbrani peygamberler Tanrı’yı İsrail’in bir yaratanı olarak ilan ettiler, fakat bunun karşısında İsa Tanrı’yı her insanoğlunun bir Yaratıcı’sı olarak açığa çıkardı. Tanrı’nın bütüncül ahlaki kavramsallaşması aşkın bir biçimde İsa’nın yaşamı tarafından aydınlatılmıştır. Ebeveyn sevgisinin doğasında kendini unutma başta gelir. Tanrı bir \bibemph{baba gibi} sevmez, bir \bibemph{baba olarak} çok sever. O her evren kişiliğinin Cennet Yaratıcısı’dır.
\vs p002 6:5 Doğruluk, evrenin ahlaki yasasının Tanrı olduğunu atfeder. Gerçeklik Tanrı’yı bir gerçekleri açığa çıkaran ve bir öğretmen olarak gösterir. Fakat şefkatini gösterirken onu tekrar geri almayı fazlasıyla arzular, ebeveyn ile çocuk arasındaki kenetlenmeyi anlamanın peşine düşer. Doğruluk kutsal bir düşünce olabilir, fakat sevgi bir babada gözlenen özelliktir. İlahiyat’ın doğasında varsayıldığı gibi bir bütünlüğün olmadığından ve bundan dolayı doğrudan İsa’nın kefaret çekmesi savının yorumlanmasına dayandırmaktan hareketle, Tanrı’nın doğruculuğunun cennetsel Yaratıcı’nın içinde kendini unuttuğu sevgisiyle bağdaştırılamayacağını farz etmeye dayanan hatalı varsayımlar Tanrı’nın özgür istençli oluşunun ve birlikteliğinin ikisininde üzerine yapılan felsefi birer saldırıdır.
\vs p002 6:6 Ruhaniyeti dünya üzerindeki çocuklarında ikame eden sevgi dolu cennetsel Yaratıcı ne adalet ve bağışlamadan oluşan bölünmüş bir kişiliktir, ne de Tanrı’nın görüşleri ve bağışlayışı arasında dengeyi sağlamak için bir arabulucuya ihtiyaç duyar. Kutsal doğruluk sıkı sıkıya tertip edilmiş dağıtımsal bir adalet anlayışı tarafından baskın değildir; Tanrı adaletini gösterirken bir yaratıcı bünyesinden bir hâkim olan Tanrı’ya doğru aşkınlaşır.
\vs p002 6:7 Tanrı hiçbir zaman ne kin doludur, ne öç alıcıdır veya ne de nefret dolu bir düşmanlık besler. Şu gerçektir ki, adaletin şartları bağışlama durumunun reddedilmesini gerektiriyorsa onun erdemi sık sık onun gösterdiği sevgiyi kısıtlar. Doğruluğun bir parçası olarak onun sevgisi, kötülük karşısında yaratılmış eş değer nefretin bir uzantısı olarak algılanışının yıkılmasına başlı başına yararlı olmaz. Yaratıcı tutarsız bir kişilik değildir; kutsal birliktelik tamamiyle kusursuzdur. Cennetin Kutsal Üçlemesi’nde Tanrı’nın yardımcılarının ebedi kimliklerine rağmen mutlak bir bütünlük vardır.
\vs p002 6:8 ‘Tanrı günahkârı bile sever, fakat o sadece günahtan \bibemph{nefret eder}’ söylemi felsefi olarak doğru bir söylemdir; fakat Tanrı aşkın bir kişiliktir ve sadece kişiler diğer kişileri sever veya ondan nefret ederler. Kötülük bir kişilik değildir. Tanrı kötülük yapanı sever, çünkü kötülük işleyen bile potansiyel olarak ebedi olabilecek bir kişilik gerçekliğidir. Tanrı kişisel bir tavırla kötülüğe karşı bir duruş almamasının altında günahın ruhani bir gerçekliğe sahip bir kişilik olmamasının sebebi yatar. Bu sebeple, Tanrı sadece adaleti vasıtasıyla onun varlığıyla yüzleşir. Tanrı’nın sevgisi kötülük sahibini kurtarırken Tanrı’nın kanunları onun günahlarını yok eder. Kutsal doğanın böyle bir niteliği, eğer kötülük sahibinin kendisini sonunda kesin bir teslimiyetle günahla özdeşleştirmesi durumunda tamamen değişikliğe uğrar. Böyle bir teslimiyet, bireyden beklenen, onun fani aklının barınan ruhani Düzenleyicisi ile bütüncül bir birleşimi arzusuyla tıpatıp benzerlik gösterir, fakat önceki durum sonraki durumun tamamen zıt yönündedir. Böyle bir kötülükle özdeşleşen fani gittikçe tamamen doğasının özünde ruhundan mahrum kalarak kişisel olarak gerçek dışı olur, ve varlığın nihai yok oluşunu deneyimlemek zorunda kalır. Gerçek dışılık, yaratılmışın doğasının tamamlanmamışlığında bile, genişleyen ruhani evrende ve ilerleyen bir gerçeklikte sonsuza kadar varlığını koruyamaz.
\vs p002 6:9 Kişiliğin doğası karşısında, Tanrı seven bir insan olarak keşfedilir; ruhani dünya önünde O bireysel bir sevgidir; ve dinsel deneyimde O bu iki durumun her ikisidir. Sevgi Tanrı’nın iradesinden doğan istencini açığa çıkarır. Tanrı’nın iyiliği; sevgiye olan evrensel yönelimin kutsal özgür istenci temelinde bulunur, merhamet gösterir, sabrını dışa vurur ve affetmeyi sağlar.
\usection{7.\bibnobreakspace Kutsal Doğruluk ve Güzellik}
\vs p002 7:1 Tüm sınırlı bilginin ve yaratılmışların anlayışı \bibemph{görecelidir}. En yüksek kaynaklardan derlenen bilgi ve akıl bile göreceli olarak kesin, yerel olarak doğru ve kişisel olarak gerçektir.
\vs p002 7:2 Fiziksel gerçekler adilane bir biçimde tek tiptir, fakat doğru olarak atfedilen değerler evrenin barındırdığı felsefesinde yaşayan ve esnek bir değişkendir. Evrimleşen kişilikler kendi iletişimlerinde sadece belirli bir ölçüde mantıklı ve göreceli olarak doğru olabilirler. Onların kişisel deneyimleri ancak sürmeye devam ettikçe onlar herhangi bir şeyden emin olabilirler. Çünkü tamamen bir yerde doğru olarak görünen bir şey yaratılmışın diğer bölümünde ancak göreceli olarak doğru olabilir.
\vs p002 7:3 Kesin olan kutsal gerçek sabit ve evrenseldir, fakat yaşam alanlarından gelen sağanak şeklince gelen birçok birey tarafından anlatıldığı şekliyle bu doğruların altında yatan oluşum biçimleri ruhanidir. Bu hikâyelerin detayları zaman zaman bilginin tamamlanmışlığından kaynaklanan görecelik sebebiyle ve bireysel deneyimin doygunluk düzeyi, uzunluğu ve bu deneyimin ölçeği ölçüsünde değişkenlik gösterebilir. İlk Muhteşem Kaynak ve Merkez’in yasaları ve kuralları, düşünceleri ve davranışları ebedi, sınırsız ve evrensel olarak doğrudur; ve aynı zamanda onların her evrene, sisteme, dünyaya ve yaratılmış akla uygulanması ve onun için düzenlenmesi Yaratan Evlatlar’ın kendi idareleri altında oldukları evrenlerde ve tüm diğer katılımsal gökyüzü kişiliklerinin ve Sınırsız Ruhaniyet’in yerel tasarıları ve işleyişleriyle uyum içinde faaliyet gösterirlerken uyguladıkları yöntemler ve tasarılarıyla iniltilidir.
\vs p002 7:4 Maddiyatçılığın yanlış yapılan bilimi fani insanı evrende dışlanmış hale getirmek için cezalandırır. Böyle kısmi bilgi taşıdığı potansiyel olarak kötüdür; böyle bir bilgi iyi ve kötü olan unsurlardan meydana gelmiştir. Doğru güzeldir, çünkü onun yargısı hem tamamlanmış hem de düzgün bir biçimde simetrik olarak örülmüş görülür. İnsan doğruluğu bulmak için yola çıktığında aslında o kutsal bir biçimde gerçek olanı amaçlar.
\vs p002 7:5 Filozoflar kendileri için en büyük hataya soyutlamanın yanılgısı, gerçeğin sadece bir parçasına odaklanma yöntemi ve bunun sonucunda yalıtılmış bu parçayı bütüncül bir gerçek gibi sunması tarafından yanlış yönlendirilerek düşerler. Akıl sahibi bir filozof evrensel olgular bütününün arkasında ve mevcudiyet\hyp{}öncesinde yaratıcı bir tasarıyı arar. Yaratan düşüncesi her koşulda değişecek bir biçimde yaratıcı eylemin varlığına ihtiyaç duyar ve onu takip eder.
\vs p002 7:6 Akli birey\hyp{}bilinci doğrunun güzelliğini, onun ruhani niteliğini, sadece onun kavramsallaşmasının felsefi tutarlılığıyla değil fakat daha kesin ve emin bir biçimde ezeli Doğruluğun Ruhaniyeti’nin yanılmaz karşılığı tarafından keşfedilebilir. Mutluluk doğruluğun tanınmasının ardından ortaya çıkar, çünkü bu doğrular \bibemph{davranışlarla ortaya konulup}, onlar yaşamın içerisinde deneyimlenebilir. Hayal kırıklığı ve ıstıraplar yanlışa hizmet eder, çünkü bir gerçeklik olmadığından deneyimlerle gerçekleştirilemezler. Kutsal gerçek en iyi bir biçimde onun\bibemph{ ruhani farklılığıyla} bilinir.
\vs p002 7:7 Ebediyeti arayış bütünleşmek ve kutsal birliktelik içindir. Cennet Adası içinde uçsuz bucaksız fiziksel kâinat; Birleştirici Bünye olan Tanrı’nın aklında mantıksal âlem, Ebedi Evlat’ın kişiliğinde ruhani evren bütünleşir. Bunun karşısında, Kâinatın Yaratıcısı’nın ve içimizde ikamet eden Düşünce Düzenleyicileri’nin arasındaki doğrudan ilişkiyle, Yaratıcı olan Tanrı’nın mevcudiyetinde zaman ve mekânın soyutlanmış fanileri bütünleşir. İnsanın Düzenleyicisi Tanrı’nın bir nüvesidir ve bitip tükenmeyen bir biçimde kutsal birlikteliğin peşindedir; İlk Kaynak ve Merkez’in Cennet İlahiyatı’yla ve onun içinde bütünlüğe ulaşır.
\vs p002 7:8 Yüce güzelliğin algısı gerçekliğin bir araya gelmesi ve bunun keşfidir: ebedi doğrulukta kutsal iyiliğin algısının tamda kendisi nihai güzelliktir. İnsan sanatının büyüsü bile onun birlikteliği ve uyumunda oluşur.
\vs p002 7:9 İbrani dininin en büyük hatası Tanrı’nın iyiliğini bilimin bilgiye dayanan doğrularına ve sanatın çekici güzelliğine iniltilendirme hususundaki eksikliğinden kaynaklanan başarısızlığıydı. Medeniyetler geliştikçe ve din Tanrı’nın iyiliğini doğruluğun reddi ve güzelliğin görmezden gelinmesi olarak haddinden fazla bir biçimde aynı mantıksızlıkla vurgulamayı amaçlamaya devam ettikçe, yapay bir biçimde ayrıştırılmış iyiliğin ilişkisini kaybetmiş soyut kavramsallaşmasından belirli tip insanların yollarını ayırmaları biçiminde artan bir yönelim gelişti. Aşırı bir biçimde tekrarlanan ve soyutlanan çağdaş dinin birçok yirminci yüzyıl insanının sadakatini ve bağlılığını tutamayan ahlakı; onun ahlaki emirlerinin yanı sıra bilimin, felsefenin, ve ruhani deneyimin doğrularına ve buna ek olarak fiziksel yaratılmışın güzelliklerine, akıl dolu sanatın büyüsüne ve benzersiz karakter erişiminin yaratacağı ihtişam karşısında düşünce önemini verirse kendisini tedavi edebilecektir.
\vs p002 7:10 Bu çağın dinsel zorluk; kutsal iyiliğin, evrensel güzelliğin ve kozmik doğruluğun pürüzsüz bir biçimde iç içe geçmiş ve genişlemiş çağdaş kavramsallaşması içerisinde, yeni ve çekici gelecek yaşayan felsefeyi yaratmayı arzulayan ruhani donanıma sahip ileri görüşlü ve ileriye\hyp{}bakmayı isteyen kadın ve erkekler içindir. Böyle bir ahlakın yeni ve doğru bakış açısı insan ruhunda en iyi şeyi bulmadaki zorluğu ve insan aklında güzele dair her şeyi beraberinde kendisine çekecektir. Doğruluk, güzellik, ve iyilik kutsal gerçeklikler olup, insan ruhani yaşam ölçeğine yükseldikçe Ebediyet’in bu yüce nitelikleri sevgi olan Tanrı’nın bünyesinde giderek bütünleşir ve düzenlenir.
\vs p002 7:11 Maddi, felsefi veya ruhani tüm doğrular hem güzel hem de iyidir. Maddi sanat veya ruhani eşlenikten oluşan tüm gerçek güzellik hem doğru hem de iyidir. Bireysel ahlak, sosyal hakkaniyet veya kutsal hizmetten hangisi olursa olsun tüm benzersiz iyilikler eşit bir biçimde doğru ve güzeldir. Sağlık, akıl sağlığı ve mutluluk doğruluğun, güzelliğin ve iyiliğin insan deneyimlerinde birleşmiş halinin bütünlüğüdür. Bu türden bir etkili yaşam enerji, düşünce ve ruhani sistemlerin birleşmesiyle ortaya çıkmıştır.
\vs p002 7:12 Doğru bütüncül, güzellik çekici, iyilik düzen sağlayıcıdır. Ve insan deneyimlerinde gerçek olan doğrunun bu değerleri eş güdümlü hale gelince sonuç; sevginin yüksek bir düzeyinin bilgelikle belirlenişi ve sadakatle nitelikli hale gelmesidir. Tüm kâinat eğitiminin esas amacı dünyaların soyutlanmış evladının daha iyi uyumlu hale gelmesine etkide bulunmak ve bunu onun genişleyen deneyimiyle birlikte daha büyük gerçekliklere yaymaktır. Gerçeklik insan seviyesinde sınırlıdır, daha yüksek ve kutsal seviyelerde ise sınırsız ve ebedidir.
\vs p002 7:13 [Uversa üzerinde Zamanın Ataları’nın yönetimi altında hareket eden bir Kutsal Danışman tarafından sunulmuştur.]
