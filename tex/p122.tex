\upaper{122}{İsa’nın Doğumu ve Bebekliği}
\vs p122 0:1 Filistin’in Mikâil’in bahşediliş yeri olarak tercih edilişine, ve özellikle, tam da neden Yusuf ve Meryem’in ailesinin Urantia üzerinde bu Tanrı’nın Evladı’nın ortaya çıkışı için öncül çevre olarak seçilişine götüren birçok nedeni tamamiyle açıklamak, neredeyse hiçbir biçimde mümkün olamayacaktır.
\vs p122 0:2 Melçizedekler tarafından hazırlanmış tecrit altındaki dünyalara dair özel bir rapor üzerinde gerçekleştirilmiş bir incelemeden sonra, Mikâil nihai olarak, son bahşedilişini üzerinde gerçekleştireceği gezegen olarak Urantia’yı seçti. Bu kararı takiben, Cebrail, Urantia’ya kişisel bir ziyarette bulundu; ve, insan topluluklarını inceleyişinin, ve, dünya ve onun insanlarının sahip oldukları ruhsal, ussal, ırksal ve coğrafi nitelikleri üzerinde gerçekleştirdiği çalışmasının bir sonucu olarak, İbraniler’in; bahşedilme ırkı olarak seçilmelerini haklı kılacak görece üstün yanlara sahip olduklarına karar verdi. Bu kararı Mikâil’in kabul etmesi üzerine, Cebrail; kendilerine Musevi aile yaşamı üzerinde bir inceleme başlatılması görevi verilen, evren kişiliklerin daha yüksek düzeyleri arasından seçilmiş haldeki --- On İki Üyeli Aile Heyeti’ni Urantia’ya gönderdi. Bu heyet çalışmalarını sonlandırdığında, Cebrail; Urantia üzerinde mevcut bulunup, heyetin görüşüne göre, Mikâil’in tasarlanmış vücutlaşımı için bahşedilme aileleri olarak eşit bir biçimde elverişli haldeki, varlığın üç olası birlikteliğini aday gösteren raporu teslim aldı.
\vs p122 0:3 Aday gösterilmiş üç çiften, Cebrail; Yusuf ve Meryem’den yana kişisel tercihinde bulunarak, bahşedilme evladının dünya annesi haline gelmek için seçildiğine dair mutlu haberleri ona aktardığı an olarak, daha sonra Meryem’e kişisel görünüşünü gerçekleştirdi.
\usection{1.\bibnobreakspace Yusuf ve Meryem}
\vs p122 1:1 (Yeşu bin Yusuf olarak) İsa’nın insan babası olarak Yusuf, her ne kadar, atalarının kadın kolları tarafından zaman zaman kendi soy ağacına zamanında eklenmiş birçok Musevi\hyp{}dışı ırk kökenlerini taşısa da, Museviler’e ait bir Musevi idi. İsa’nın babasının kökeni; İbrahim’in dönemine, ve bu saygı değer ırk büyüğünden Sümer ve Nod unsurlarına varan kalıtımın öncül kollarına, ve ilk çağ mavi insanların güney kabileleri boyunca ta Andon ve Fonta’ya kadar gitmekteydi. Davut ve Süleyman, Yusuf’un soyunun doğrudan kolu içinde değildi; ne de, Yusuf’un kol kökeni doğrudan Âdem’e kadar uzanmaktaydı. Yusuf’un doğrudan ataları --- inşaatçılar, marangozcular, taşçılar ve demirciler olarak --- ustalardı. Yusuf’un kendisi, bir marangoz ve daha sonra bir inşaatçıydı. Onun ailesi; Urantia üzerinde dinin evrimi ile ilişkili olarak zamanında kendilerini ön plana çıkarmış görülmemiş bireylerin aralıklarla gerçekleşmiş ortaya çıkışları tarafından belirgin hale gelen, olağan insanların içindeki soylu topluluğunun uzun ve saygın bir koluna aitti.
\vs p122 1:2 İsa’nın dünya annesi olarak Meryem, Urantia’nın ırksal tarihi içinde en dikkate değer kadınların çoğunu içine alan benzersiz ataların uzun bir kolunun bir soyuydu. Her ne kadar Meryem, oldukça olağan bir mizaca sahip olan bir biçimde, döneminin ve neslinin ortalama bir kadını olmuş olsa da, Annon, Tamar, Ruth, Bathsheba, Ansie, Cloa, Eve, Enta, ve Ratta gibi çok iyi bilinen kadınları içine alan atalara sahip olduğu bilinmektedir. Bu dönemin hiçbir Musevi kadını, ortak soylarının daha saygın bir koluna veya daha elverişli başlangıçlara uzanan bir kökene sahip değildi. Meryem’in soyu, Yusuf’unki gibi; medeniyetin ilerleyişinde ve dinin ilerleyici evriminde sayısız mükemmel kişilik tarafından aralıklarla destek görmüş bir biçimde, güçlü ancak olağan bireylerin baskınlığıyla nitelenmekteydi. Irksal bakımdan değerlendirildiğinde, Meryem’i bir Musevi kadını olarak görmek neredeyse hiçbir şekilde mümkün değildir. Kültür ve inanış bakımından o bir Musevi’idi; ancak, kalıtımsal edinmişlik bakımından, o, ırksal kalıtımı Yusuf’unkinden daha genel olan bir biçimde, Suriye, Hitit, Finik, Yunan ve Mısırlı ırk kökenlerinin bir bileşimiydi.
\vs p122 1:3 Mikâil’in tasarlanmış bahşedilişinin zaman zarfında Filistin’de yaşayan tüm çiftler arasında, Yusuf ve Meryem; geniş bir alana yayılmış ırksal ilişkilerin en ideal bileşimini ve kişilik edinmişliklerinin üstün bir ortalamasını taşımaktaydı. Dünya üzerinde \bibemph{ortalama} bir insan olarak ortaya çıkmak Mikâil’in tasarımıydı; olağan insanların onu anlaması ve onlar tarafından karşılık görmesi için; bundan dolayı, Cebrail, bahşedilme ebeveynleri haline gelmesi için Yusuf ve Meryem olarak tam da bu iki bireyi seçmişti.
\usection{2.\bibnobreakspace Cebrail’in Elizabet’e Görünüşü}
\vs p122 2:1 İsa’nın Urantia üzerindeki yaşam\hyp{}görevi, gerçekten de, Vaftizci Yahya tarafından başlatılmıştı. Yahya’nın babası olan Zekeriya Musevi din\hyp{}adamlığına ait iken, annesi olan Elizabet, İsa’nın annesi Meryem’in aynı zamanda ait olduğu aynı büyük aile topluluğunun daha varlıklı bir kolunun bir üyesiydi. Zekeriya ve Elizabet, her ne kadar birçok yıldır evli olsalar da, çocuksuzlardı.
\vs p122 2:2 Tıpkı daha sonra mevcudiyetini Meryem’e bildirdiği gibi, Cebrail’in öğlen vakti Elizabet’e kendisini görünür kılışı, Yusuf ve Meryem’in evliliğinden yaklaşık olarak üç ay sonra, M.Ö. 8. yılda Haziran ayının sonunda gerçekleşmişti. Cebrail şunu söylemişti:
\vs p122 2:3 “Her ne kadar kocan Zekeriya Kudüs’de sunağından huzurunda dursa da, ve, bir araya gelmiş insanlar bir kurtarıcın gelişi için dua etse de, Cebrail olarak ben; bu kutsal öğretmenin habercisi olacak bir evladı yakın zamanda taşıyacağını ve oğlunu Yahya koyacağını bildirmek için geldim. O; senin Tanrın olan Koruyucu’ya adanmış bir biçimde büyüyecek, ve, erişkin hale geldiğinde, kalbini neşeyle ısıtacak, çünkü o birçok ruhu Tanrı’ya döndürecek, ve, o aynı zamanda, insanlarının ruh\hyp{}iyileştiricisinin ve tüm insanlığın ruhaniyet\hyp{}özgürleştiricisinin gelişini duyuracak. Senin kadın akraban Meryem söz verilmiş bu çocuğun annesi olacak, ve ben aynı zamanda ona da kendimi görünür kılacağım.”
\vs p122 2:4 Bu görünme Elizabet’i fazlasıyla korkutmuştu. Cebrail’in ayrılışından sonra, o, bu görkemli ziyaretçinin söyledikleri üzerinde uzunca bir süre düşünen bir biçimde, aklında bu deneyim üzerinde etraflıca düşündü; ancak, o, bu açığa çıkarılıştan, daha sonra gerçekleştirdiği ertesi yılın Şubat ayının başında Meryem’i ziyaretine kadar, kocası hariç hiç kimseye bahsetmemişti.
\vs p122 2:5 Beş ay boyunca, buna rağmen, Elizabet, sahip olduğu sırrı kocasından bile saklamıştı. Cebrail’in ziyaretine dair hikâyeyi Elizabet’in ortaya çıkarışından sonra, Zekeriya; Elizabet’in çocuk beklediğini artık sorgulamaz olduğunda Cebrail’in karısını olan ziyaretine yalnızca yarı\hyp{}gönüllü inanır hale gelen biçimde, oldukça kuşkulu olup, haftalar boyunca bütüncül deneyimden şüphe duymuştu. Zekeriya, Elizabet’in olası anneliği ile ilgili oldukça fazla bir biçimde kafası karışmış hale gelmişti; ancak, o, ileri yaşına rağmen, karısının dürüstlüğünden şüphe duymamıştı. Zekeriya’nın, etkileyici bir rüyanın sonucu olarak; Mesih’in gelişi için zemini hazırlayacak birisi niteliğinde, Elizabet’in nihai sonun bir evladının annesi olacağından bütünüyle emin hale gelişi, Yahya’nın doğumundan yaklaşık olarak altı hafta öncesine kadar gerçekleşmemişti.
\vs p122 2:6 Cebrail Meryem’e, M.Ö. 8. yılda Kasım ayının yaklaşık olarak ortasında görünüşünü gerçekleştirirken, Meryem Nasıra evinde çalışır haldeydi. Daha sonra, Meryem; bir anne olacağını kuşkusuz olarak bildikten sonra, Yusuf’u, Kudüs’ün dört mil batısındaki Yehuda Şehrine, tepelerinde Elizabet’i ziyaret etmek amacıyla seyahat etmesine izin vermesi için ikna etti. Cebrail bu anne\hyp{}olacakların her birini, diğerine gerçekleştirmiş olduğu görünümü hakkında bilgilendirmiş konumdaydı. Doğal olarak onlar; bir araya gelmekten, deneyimlerini karşılaştırmaktan ve evlatlarının olası gelecekleri hakkında konuşmaktan çekinmekteydiler. Meryem, üç hafta boyunca uzak kuzeni ile beraber kaldı. Elizabet; âlemin ortalama ve olağan bir bebeği olarak, yardıma muhtaç bir yeni\hyp{}doğan halinde dünyaya çok yakın zamanda sunacağı nihai sonun evladına olan annelik görevine daha bütüncül bir biçimde adanmış olarak evine geri dönmesi için, Meryem’in Cebrail’in görünüşüne dair inancını kuvvetlendirmeye fazlasıyla çaba sarf etti.
\vs p122 2:7 Yahya, M.Ö. 7. yılda Mart ayının 25’inde Yehuda Şehri’nde doğmuştu. Zekeriya ve Elizabet fazlasıyla, Cebrail’in önceden söz vermiş olduğu biçimiyle bir evladın onlara gelişinin gerçekleşmesinden sevinç duymuşlardı; ve, sünnet için çocuğun sunulduğu sekizinci günde, onlar resmi bir biçimde çocuklarını, daha öncesinden yönlendirildiği haliyle, Yahya olarak vaftiz ettiler. Hâlihazırda Zekeriya’nın bir yeğeni; Elizabet’in, bir evladın kendisine doğduğunu ve isminin Yahya olacağını duyuran Meryem’e olan iletisini taşıyan bir biçimde, Nasıra’ya hareket etmek üzere ayrılmıştı.
\vs p122 2:8 Bebekliğinin en başından itibaren Yahya, bir ruhsal önder ve dini öğretmen haline gelecek şekilde büyüyeceği düşüncesiyle, ebeveynleri tarafından yerinde bir biçimde etkilenmişti. Ve, Yahya’nın kalbinin toprağı, bu türden destekleyici tohumların yeşermesi için sürekli olarak karşılık gösterir konumdaydı. Bir çocuk olarak bile, o sıklıkla, babasının hizmet verdiği dönemlerde tapınakta bulunmaktaydı; ve, o devasa bir biçimde, gördüğü her şeyin taşıdığı önemden etkilenmişti.
\usection{3.\bibnobreakspace Cebrail’in Meryem’e Olan Duyuruşu}
\vs p122 3:1 Yusuf’un eve dönüşünden önce, yaklaşık olarak günbatımı sularında bir akşam, Cebrail, alçak bir taş masanın yanında Meryem’e kendini görünür kıldı; ve, Meryem kendisini toparladıktan sonra, Cebrail şunları söyledi: “Ben, benim Üstün’üm ve senin seveceğin ve besleyeceğin kişinin emri üzerine geldim. Sana, Meryem, ben, içinde gebe olduğun varlığın cennetten emredilmiş olduğunu duyurduğumda, neşeli haberleri getirmekteyim; sen onun adını Yeşu koyacak, ve, o, dünya üzerinde ve insanlar arasında cennetin krallığını başlatacak. Yusuf, ve, aynı zamanda kendisine de görünür kıldığım, ve aynı zamanda yakın bir süre içinde, ismi Yahya olacak ve senin evladının büyük güçle ve derin kendinden eminlikle insanlara duyuracağı kurtuluşun iletisi için zemin hazırlayacak, bir evlat dünyaya getirecek olan kadın akraban Elizabet dışında bu konu hakkından kimseye bahsetme. Ve, benim sözümden kuşku duyma, Meryem; zira, bu ev, nihai sonun evladının fani yaşam alanı olarak seçilmiştir. Benim takdisim senin üzerinedir, En Yüksek Unsurlar’ın gücü seni güçlendirecektir, ve tüm yeryüzünün Koruyucusu seni gölgeleyecektir.”
\vs p122 3:2 Meryem; kocasına bu olağandışı gelişmeleri ortaya çıkarma cüreti göstermeden önce, çocuğa sahip olduğunu kesin bir biçimde bildiği ana kadar, birçok hafta boyunca kalbinde bu ziyaret üzerinde gizlice düşündü. Yusuf bunların hepsini duyduğunda, her ne kadar Meryem’e büyük bir güven duysa da, fazlasıyla bocalamış, birçok gece uyuyamamıştı. İlk başta Yusuf, Cebrail’in ziyareti hakkında şüphelere sahipti. Daha sonra, o; Meryem’in bu sesi gerçekten duyduğuna ve onun kutsal ileticinin görüntüsüyle karşılaştığına neredeyse tamamen ikna olduğunda, bu tür şeylerin nasıl gerçekleşebileceğini derin derin düşünerek aklı ikiye ayrılmıştı. İnsan varlıklarının doğumu nasıl olurda kutsal nihai sonun bir avladı olabilirdi? Yusuf, düşüncenin birkaç haftasından sonra bu çelişkili düşünceleri nihai olarak bir araya getirebildikten sonra, hem kendisi hem de Meryem; her ne kadar beklenen kurtarıcının kutsal bir doğaya ait oluşu neredeyse hiçbir biçimde Musevi kavramsallaşması olmasa da, Mesih’in ebeveynleri haline gelmek için seçilmiş oldukları sonucuna vardılar. Bu çok önemli yargıya varmaları üzerine, Meryem, Elizabet ile gerçekleştirilecek bir buluşma için ayrılmaya bir an önce hazırlandı.
\vs p122 3:3 Geri dönüşü üzerine, Meryem; Yoakim ve Anna olarak ebeveynlerini ziyaret etmeye gitti. Her ne kadar, onlar, bu zaman zarfında tabi ki de, Cebrail’in ziyareti hakkında hiçbir şey bilmeseler de, onun iki abisi ve kız kardeşine ek olarak ebeveynleri, her zaman, İsa’nın kutsal görevi hakkında oldukça kuşku duymaktalardı. Ancak, Meryem, kız kardeşi Soleme ile; evladının büyük bir öğretmen olma nihai sonuna sahip olduğu sırrını paylaştı.
\vs p122 3:4 Cebrail’in Meryem’e olan duyuruşu; İsa’nın gebeliğinin ertesi günü gerçeklemiş olup, Meryem’in söz verilmiş çocuğu taşıma ve ona olan hamileliğindeki bütüncül deneyimi ile ilişkili doğa\hyp{}üstü gerçekleşmiş tek olaydır.
\usection{4.\bibnobreakspace Yusuf’un Rüyası}
\vs p122 4:1 Yusuf; oldukça etkileyici bir rüyayı deneyimleyinceye kadar, Meryem’in olağanüstü bir çocuğun annesi hale geleceği düşüncesine ikna olamamıştı. Bu rüyada, muhteşem bir göksel iletici Yusuf’a kendisini görünür kıldı, ve diğer şeylerle birlikte şunları söyledi: “Yusuf, ben; şu an en yüksekte hükmeden O’nun emri ile ortaya çıkmış bulunmakta olup, Meryem’in taşıyacağı ve dünyada büyük bir ışık haline geleceği evlat hakkında seni bilgilendirmeye emredildim. Onda yaşam olacak, ve onun yaşamı insanlığın ışığı olacak. O ilk olarak, kendi insanlarına gelecek; ancak, o, neredeyse hiçbir biçimde onu kabul etmeyecekler; ancak, birçokları onu kabul etmeye başlayınca, o, kendilerinin Tanrı’nın evlatları olduklarını açığa çıkaracak.” Bu deneyimden sonra, Yusuf bir daha hiçbir zaman; Cebrail’in ziyaretine ve bu doğmamış çocuğun dünyaya kutsal bir iletici haline gelecek olması sözüne dair Meryem’in hikâyesinden kuşku duymamıştı.
\vs p122 4:2 Tüm bu ziyaretlerde, Davud’un evi hakkında hiçbir şey söylenmemişti. Hiçbir zaman, İsa’nın bir “Musevilerin kurtarıcısı” haline gelişine ve hatta onun uzun zamandır beklenen Mesih olacağına hiçbir imada bulunulmamıştı. İsa, Museviler’in önceden beklemiş oldukları türden bir Mesih olarak gelmemişti; ancak, o, \bibemph{dünyanın kurtarıcısıydı}. Onun görevi, tüm ırk ve insan topluluklarına karşıydı, herhangi bir topluluğa değil.
\vs p122 4:3 Yusuf, Kral Davud’un soy koluna ait değildi. Meryem, Yusuf’dan daha çok Davut soyunu taşımaktaydı. Yusuf’un, Roma nüfus sayımı için kaydolmak amacıyla Beytüllahim olarak Davud’un Şehri’ne gidişi doğrudur; ancak, bu, Yusuf’un, altı nesil öncesinde Davud’un doğrudan bir soyu olan bir Zadok tarafından evlat edilmiş bir öksüz olarak, bu nesildeki babadan gelen atası nedeniyle gerçekleşmişti; bu nedenle, Yusuf, “Davud’un evi” içinde sayılmıştı.
\vs p122 4:4 Eski Ahit sahip olduğu tarafınızdan adlandırılmakta olan Mesihsel kehanetlerinin çoğu, hayatı dünya üzerinde yaşanıldıktan uzun bir süre sonra İsa’ya atfedilmiş hale getirilmişti. Çağlar boyunca İbrani peygamberleri, bir kurtarıcının gelişini duyurdular; ve, bu sözler, ilerleyen nesiller tarafından, yeni bir Musevi yöneticinin Davud’un tahtına oturacağı ve, Musa’nın ünlü gizemli yöntemleri vasıtasıyla, Musevileri Filistin’de tüm yabancı egemenliğinden arınmış güçlü bir millet olarak kuracağı şeklinde yorumlanmıştı. Tekrar ifade edilecek olursa, İbrani yazıtlarında bulunan birçok mecazi bahis daha sonra, İsa’nın yaşam görevine yanlış bir biçimde eklenmişti. Birçok Eski Ahit sözü, Üstün’ün dünya yaşamının belirli bir dönemine uyar şekilde görünmesi için çarpıtılmıştı. İsa’nın kendisi bir seferinde, Davud’un hanedan soyu ile herhangi bir ilişkisinin bulunduğunu herkese duyuran bir biçimde reddetti. “Bir genç kadın bir erkek taşıyacak” bahsi bile, “bir bakir bir evlat doğuracak” haline getirilerek okundu. Bu durum aynı zamanda, Mikâil’in dünya üzerindeki sürecinin sonrasında, Yusuf ve İsa’nın soyağaçları hakkında yapılan birçok tahmin için gerçeklik taşımaktadır. Bu bağlantıların çoğu, Üstün’ün soyunun çoğunu taşımaktadır; ancak, bütünü itibariyle, onlar, özgün olmayıp, tarihsel bilgi olarak güvenilemeyen niteliktedir. İsa’nın öncül takipçileri, haddinden fazla bir biçimde; eski dönemlerin tüm kehanetsel ifadelerinin, sahip oldukları Koruyucu ve Üstün’ün yaşamında yerine geldiğini göstermenin çekiciliğine düşmüşlerdi.
\usection{5.\bibnobreakspace İsa’nın Dünya Ebeveynleri}
\vs p122 5:1 Yusuf; oldukça sakin, aşırı bir biçimde vicdan sahibi ve insanlarının dini kabulleri ve uygulamalarına her açıdan tabi biriydi. O az konuşur, fakat çok düşünürdü. Musevi insanlarının üzücü durumu, Yusuf’u fazlasıyla üzmüştü. Gençken, sekiz erkek ve kız kardeşi arasında, en neşeli olandı; ancak, evlilik yaşamının öncül yıllarında (İsa’nın çocukluğu boyunca) orta düzeyde ruhsal hayal kırıklığına maruz kalmıştı. Bu ani duygusal değişimlerin dışavurumları, zamansız ölümünden hemen önce ve sahip olduğu ailenin ekonomik durumu bir marangozun seviyesinden varlıklı bir inşaatçı konumuna olan gelişimiyle geliştikten sonra, fazlasıyla iyileşmiş durumdaydı.
\vs p122 5:2 Meryem’in mizacı, kocasının sahip olduğundan tam da tersiydi. O oldukça sık bir biçimde; neşeli, oldukça nadiren üzgün, ve sürekli güler yüzlü bir çehreye sahip olmuştu. Meryem; duygusal hislerini özgürce ve sıklıkla ifade etmeye izin verip, Yusuf’un ansızın gerçekleşen ölümünden sonra bile hiçbir zaman kederli gözlenmemişti. Ve, Meryem, bu derin şaşkınlıktan; hayrete düşmüş bakışları altında çok hızlı bir biçimde gerçekleşmekte olan en büyük evladının olağanüstü sürecinin yarattığı endişeler ve sorgular omuzlarına yüklendiğinde, hiç de bütünüyle kurtulabilmiş değildi. Ancak, tüm bu olağandışı deneyim boyunca, Meryem; tuhaf ve az\hyp{}anlaşılmış ilk\hyp{}doğan çocuğu ve bu evladın hayattaki erkek ve kız kardeşleri arasındaki ilişkisi içinde serinkanlı, cesur ve oldukça bilge olmuştu.
\vs p122 5:3 İsa, sahip olduğu insan doğasının olağandışı inceliğinin ve muhteşem anlayışının çoğunu babasından almıştı; o, büyük bir öğretmen olarak armağanını ve haklı yere gösterilen kızgınlığının devasa yetisini annesinden miras almıştı. Erişkin\hyp{}yaşam çevresine olan duygusal tepkilerinde, İsa, babası gibi, zaman zaman düşünceli ve ibadetkar, zaman zaman ise gözle görülür biçimde üzgün haldeydi; ancak, daha sıklıkla, o, annesinin ümitli ve kararlı yönelimi içinde ileri doğru hareket etmişti. Son kertede, Meryem’in mizacı; İsa büyürken ve erişkin yaşamının çok önemli gelişmelerine adım atarken, kutsal Evlat’ın süreci üzerinde egemen olma eğilimi göstermişti. Belirli açılardan, İsa, ebeveynlerinin kişilik özelliklerinin bir karışımıydı; diğer açılardan ise, o, bir ebeveyninin diğerine tezat oluşturan kişilik özelliklerini sergilemişti.
\vs p122 5:4 Yusuf’dan, İsa; Musevi tören adetlerinindeki ciddi eğitimini ve İbrani yazıtları ile olan olağandışı yakınlaşımını kazanmıştı; Meryem’den, o, dini yaşamın daha geniş bir bakış açısını ve kişisel nitelikli ruhsal özgürlüğün daha özgür bir kavramsallaşmasını elde etmişti.
\vs p122 5:5 Hem Yusuf’un hem de Meryem’in ailesi, zamanlarına göre çok iyi eğitim görmüşlerdi. Yusuf ve Meryem, yaşam içindeki dönemleri ve durumlarına göre ortalamanın oldukça çok üzerinde eğitim görmüşlerdi. Yusuf, bir düşünürdü; Meryem, uyum sağlamada uzman ve doğrudan uygulamada eli yatkın olarak, tasarlayıcıydı. Yusuf, bir siyah\hyp{}gözlü kumraldı; Meryem, kahverengi\hyp{}gözlü neredeyse sarışın bir tipe sahipti.
\vs p122 5:6 Yusuf; yaşamış olsaydı, kuşkusuz olarak, en büyük oğlunun kutsal görevine güçlü bir inanan hale gelirdi. Meryem diğer çocukları, ve, arkadaşları ve akrabalarının bakışından fazlasıyla etkilenen bir biçimde inanma ve kuşku duyma arasında gidip gelmişti; ancak, o her zaman, bu çocuğa gebe kaldıktan hemen sonra Cebrail’in ortaya çıkışının hafızasıyla nihai tutumunu korumuştu.
\vs p122 5:7 Meryem uzman bir dokumacı olup, bu dönemin ev sanatlarının çoğunda ortalamadan çok daha yetenekliydi; o, iyi bir ev idarecisi ve üstün bir ev inşacısıydı. Hem Yusuf hem de Meryem, iyi öğretmenlerdi; ve, onlar, bulundukları dönemin eğitiminde yetkin hale gelmelerine önem verdiler.
\vs p122 5:8 Yusuf genç bir adamken, Meryem’in babası tarafından evine gerçekleştirdiği bir ek yapının işi için tutuldu; ve, bir öğle yemeği sırasında, Meryem Yusuf’a bir bardak su getirdiğinde, İsa’nın ebeveynleri haline gelme nihai sonuna sahip çiftin yakınlaşması gerçek anlamıyla başlamış oldu.
\vs p122 5:9 Yusuf ve Meryem; Yusuf yirmi üç yaşında iken, Nasıra’nın sınırlarındaki Meryem’in evinde, Musevi âdeti uyarınca evlenmişlerdi. Bu evlilik, yaklaşık iki yıl süren olağan bir yakınlaşma sürecini sonlandırdı. Bundan yakın bir zaman sonra onlar, kardeşlerinin ikisinin yardımıyla Yusuf tarafından inşa edilmiş bulunan, Nasıra’daki yeni evlerine taşınmışlardı. Bu ev, şehrin çevreleyen kırsal alanını oldukça büyüleyici bir şekilde üzerinden gören, yakındaki tepe arazinin eteklerinin yanında konumlanmıştı. Özellikle hazırlanmış olarak bu evde, bu genç ve beklenti içindeki ebeveynler; Yahudiye’nin Beytüllahim şehrindeki evlerinden uzakta olduklarında bir evrenin bu çok önemli olayının gerçekleşmekte olacağının çok az farkında olan bir biçimde, söz verilmiş evladı karşılamayı düşünmüşlerdi.
\vs p122 5:10 Yusuf’un ailesinin daha büyük bir kısmı, İsa’nın öğretilerinin inananları haline geldi; ancak, Meryem’in akrabalarının çok azı hiçbir şekilde, İsa bu dünyadan ayrılana kadar ona inanmamıştı. Yusuf, beklenen Mesih’e dair ruhsal kavramsallaşmaya daha çok eğilim göstermekteydi; ancak, özellikle onun babası olmak üzere, Meryem ve onun ailesi, Mesih’in düşüncesini geçici bir kurtarıcı ve siyasi yönetici olarak düşünmekteydiler. Meryem’in ataları, bu dönemin özellikle yakın zamanları içinde Makabi etkileriyle baskın bir biçimde özdeşleşmiş konumdalardı.
\vs p122 5:11 Yusuf çok etkin bir biçimde, Musevi dininin Doğu, veya diğer bir değişle Babil, görüşlerini benimsemekteydi. Meryem güçlü bir biçimde, kanun ve peygamberlerin daha geniş Batı, veya diğer bir değişle Helenistik, yorumuna eğilim göstermekteydi.
\usection{6.\bibnobreakspace Nasıra’daki Ev}
\vs p122 6:1 İsa’nın evi; kasabanın doğu bölümünde bulunan köy pınarından biraz uzaktaki, Nasıra’nın kuzeyde kalan bölümü içinde yüksek tepeden çok uzakta değildi. İsa’nın ailesi, şehrin dış çeperlerinde ikamet etmişti; ve, bu, onun daha sonra, şehrin dışındaki kırsalda gerçekleştirdiği sık yürüyüşleri memnuniyetle deneyimlemesini, ve, doğuya uzanan Tabor Dağı sırası ve yaklaşık olarak aynı yükseklikte bulunan Nain’in tepesi dışında, güney Celile’nin tepelerinin tümü içinde en yükseği olan bu yakın yükseltilerin doruklarına ziyaretlerde bulunmasını tamamiyle kolay hale getirmişti. Onların evi; bu tepenin güney burnunun biraz güney ve doğusuna doğru ve bu tepenin ayağı ile Kana’ya doğru giden yolun ortasında konumlanmıştı. Tepeye tırmanmak dışında, İsa’nın gözde yürüyüşü; Sephoris yolu ile birleşen bir noktaya kadar bir kuzeydoğu doğrultusunda, tepenin eteğine doğru kıvrılarak uzanan dar bir patikayı takip etmekti.
\vs p122 6:2 Yusuf ve Meryem’in evi, düz bir çatı ve hayvanları barındıran bir yan bina ile birlikte tek odalı bir taş yapıydı. Mobilya; alçak bir taş masadan, çömlek ve taş tabaklar ve tencerelerden, bir dokuma tezgâhından, bir mumluktan, birkaç küçük metal aletten ve taş zeminde uyumak için sedirlerden oluşmaktaydı. Arka bahçede, yan hayvan barınağının yakınında, fırını ve tahıl öğütmek için değirmeni çevreleyen baraka bulunmaktaydı. Bu değirmen çeşidini kullanmak için, birinin öğütmeyi diğerinin ise tahılı koymayı sağlamak durumunda olduğu, iki kişi gerekmekteydi. Bir küçük oğlan olarak İsa çoğu kez, annesi öğütücüyü döndürürken bu değirmene tahıl koymuştu.
\vs p122 6:3 Daha sonraki yıllarda, aile büyüklük bakımından genişledikçe, onların hepsi; yemeğin ortak bir tabağından, veya tencereden, beslenerek, yiyeceklerini afiyetle yemek için genişletilmiş taş masanın etrafında çömelirlerdi. Kış boyunca, akşam yemeklerinde masa, zeytinyağı ile doldurulmuş, küçük, düz bir çömlek lambası tarafından aydınlatılırdı. Marta’nın doğumundan sonra, Yusuf; gün boyunca bir marangoz atölyesi ve gece olunca bir yatak odası olarak kullanmış olan, geniş bir odayı bu eve ek olarak inşa etti.
\usection{7.\bibnobreakspace Beytüllahim’e Olan Ziyaret}
\vs p122 7:1 M.Ö. 8. yılda Mart ayında (Yusuf ve Meryem’in evlendikleri ayda), Sezar Augustus; daha iyi vergilendirmeyi gerçekleştirmek için kullanılabilecek bir nüfus sayımının yapılması gerekliliği olarak, Roma İmparatorluğu’nun tüm sakinlerinin sayılmasının emrini verdi. Museviler her zaman, “insanların sayılmasına” dair her girişime karşı ön yargılı bir konumda bulunmaktalardı; ve, bu, Yahudiye’nin Kralı olarak Hirodes’in yaşadığı birçok ciddi zorluk ile ilişkili olan bir biçimde, bir yıllığına Musevi krallığı içinde bahse konu bu nüfus sayımının gerçekleştirilişinin ertelenmesine neden olan gelişmeyi ortaya çıkarmış konumda bulunmaktaydı. Tüm Roma İmparatorluğu boyunca, bu nüfus sayımı; bir yıl sonra, M.Ö. 7.yılda düzenlenmiş olduğu Hirodes’in Filistin krallığı dışında, M.Ö. 8.yılda kaydedilmişti.
\vs p122 7:2 Yusuf’un ailesini kaydetmeye yetkisi olan bir biçimde --- Meryem’in Beytüllahim’e kaydolmak için gidişi zorunlu değildi; ancak, maceraperest ve ısrarkar bir kişi olarak Meryem, Yusuf’a eşlik etmek için ısrar etti. O, Yusuf yokken çocuğu doğarsa diye yalnız bırakılmaktan korkmuştu; ve, tekrar ifade edilmesi gerekirse, Beytüllahim’in Yahudiye Şehri’nden çok da uzak olması nedeniyle, kadın akrabası Elizabet’e yapılacak olası bir hoş ziyareti öngörmüştü.
\vs p122 7:3 Yusuf, Meryem’in kendisine eşlik etmesini neredeyse yasaklamıştı; ancak, bu hiçbir sonuç vermemişti; yolluk yiyeceği üç veya dörtlük için hazırlandığında, o kumanyaları iki katına çıkararak, yolculuk için hazır hale geldi. Ancak, onlar yola tam olarak çıkmadan önce, Yusuf, Meryem’in kendisiyle beraber gelişine razı oldu; ve, onlar neşeli bir biçimde, günün ağarmasıyla Nasıra’dan ayrıldılar.
\vs p122 7:4 Yusuf ve Meryem fakirlerdi; ve, onlar, yük taşıyan tek bir hayvana sahip oldukları için, bir çocukla karnı burnunda olarak Meryem, Yusuf hayvanı yönlendirir bir biçimde yürürken, yolculuk eşyaları ile birlikte hayvan üstünde seyahat etti. Bir evi inşa etmek ve döşemek, Yusuf üzerinde büyük bir yük olmuştu; çünkü o aynı zamanda, babası yakın bir zaman içinde elden ayaktan kesildiği için, ebeveynlerine yardım etmek zorundaydı. Ve, bu, Musevi çifti, Beytüllahim’e yolculukları için M.Ö. 7.yılda Ağustos’un 18’nde sabahın erken saatlerinde mütevazı evlerinden ayrıldılar.
\vs p122 7:5 Seyahatlerinin ilk günü, Ürdün nehrinin kenarında gece için konakladıkları yer olan Gilboğa dağının etekleri etrafında onları taşıdı; ve, bu ilk gün, Yusuf’un bir ruhsal öğretmenin kavramsallaşmasına bağlı kaldığı ve Meryem’in ise İbrani milletinin bir kurtarıcısı olarak bir Musevi Mesihi’nin düşüncesini beslediği bir biçimde, nasıl bir tür evladın onlara doğacak oluşuna dair birçok varsayım içinde geçti.
\vs p122 7:6 Ağustos’un 19’nun berrak ve erken sabah vaktinde, Yusuf ve Meryem, tekrar yola koyuldu. Onlar; Ürdün vadisini üzerinden gören Sartaba dağının eteğinde öğlen yemeklerini yedi, ve, şehrin sınırlarında yol üzerindeki handa durdukları yer olan, gece için Eriha’ya ulaşan bir biçimde seyahatlerine devam ettiler. Akşam yemeğini takiben, ve, Roma yöneticisi Hirodes’in baskıcılığı, nüfus kaydı ve Musevi eğitim ve kültürün merkezleri olarak Kudüs ve İskenderiye’nin karşılaştırmalı etkisi üzerine karşılıklı uzun konuşmalarından sonra, yolcular gece dinlencesi için istirahata çekildi. Ağustos’un 20’nde sabahın erken vakti; öğleden önce Kudüs’e ulaşan, mabedi ziyaret eden, burada kaldıkları yerden yola tekrar koyulan ve ikindi vakti Beytüllahim’e ulaşan bir biçimde, yollarına devam ettiler.
\vs p122 7:7 Han haddinden fazla bir biçimde kalabalıktı ve Yusuf bu nedenle uzak akrabalarında kalacak bir yer bulmaya çalıştı; ancak, Beytüllahim’de her oda ağzına kadar doluydu. Hanın bahçesine geri döndüklerinde, Yusuf’a; öncesinde hayvanlardan arındırılmış ve han sakinlerini ağırlamak için köşe bucak temizlenmiş olan, hanın tam altında kaya kenarından oluşturulmuş kervan atlarının ahırlarının bulunduğu bildirildi. Eşeği bahçede bırakarak, Yusuf; giyecek torbalarını ve yolluklarını sırtlayıp, Meryem ile birlikte taş merdiven basamaklarından konakladıkları yere indiler. Onlar kendilerini, ahır bölmeleri ve yemliklerinin karşısındaki öncesinde bir tahıl kileri olan yerde yerleştirdiler. Çadır perdeler öncesinden geriliydi, ve onlar kendilerini, böyle rahat bölmelere sahip oldukları için şanslı saydılar.
\vs p122 7:8 Yusuf öncesinde, bir vakit dışarı çıkıp kayıt olmayı düşünmüştü, ancak, Meryem yorgundu; o, dikkate değer bir biçimde rahatsız halde olup, onun yanı başında kalmaya devam etmesini rica etmişti; Yusuf, Meryem’in isteğini yerine getirmişti.
\usection{8.\bibnobreakspace İsa’nın Doğumu}
\vs p122 8:1 O gecenin tamamı boyunca Meryem o kadar rahatsızdı ki, ikisi de uyuyamamıştı. Gün doğumuyla birlikte, çocuğun doğum sancıları tamamiyle belirgin haldeydi; ve, M.S. 7.yılda Ağustos’un 21’nde öğlen vakti, kadın akran yolcularının yardımı ve iyi niyetli hizmetleriyle, Meryem, bir erkek çocuğu dünyaya getirdi. Nasıralı İsa; dünyaya doğmuş, Meryem’in bu tür olası bir tesadüf nedeniyle öncesinden beraberinde getirmiş olduğu kıyafetlere sarılmış ve yakındaki bir samanlığın üzerine konulmuştu.
\vs p122 8:2 Bu güne kadar ve bu günden sonra doğmuş olan tüm bebekler gibi, söz verilmiş evlat aynı şekilde dünya gelmişti; ve, Musevi âdetine uyarınca, sekizinci günde sünnet edilmiş olup, kendisine resmi bir biçimde Yeşu (İsa) verilmişti.
\vs p122 8:3 İsa’nın doğumunun ertesi günü Yusuf, nüfuz kaydını gerçekleştirdi. Eriha’da iki gece öncesinde konuşmuş oldukları bir kişi ile buluştuktan sonra, bu kişi tarafından Yusuf; hanın içinde bir odaya sahip olan ve Nasıralı çift ile memnuniyetle yerlerini değiştirecek varlıklı bir arkadaşa götürüldü. O öğleden sonrası onlar, Yusuf’un uzak bir akrabasının evinde kalacak yeri bulana kadar, neredeyse üç hafta boyunca yaşayacakları yer olan hana içine geçtiler.
\vs p122 8:4 İsa’nın doğumunun ikinci gününde, Meryem; evladının dünyaya gelmiş olduğu haberini Elizabet’e ulaştırmış olup, karşılığında, Zekeriya ile her şeyi konuşmak için Yusuf’u Kudüs’e kadar davet eden haberi aldılar. Ertesi hafta Yusuf, Zekeriya ile görüş alış\hyp{}verişinde bulunmak için Kudüs’e gitti. Hem Zekeriya hem de Elizabet; İsa’nın gerçekten de Musevi kurtarıcısı, Mesih olacağına ve evlatları Yahya’nın, nihai sonun sağ kolu olarak İsa’nın yardımcılarının başı haline geleceğine dair samimi yargının egemenliği altında bulunmaktaydılar. Ve, Meryem bu aynı düşünceleri beslediği için, Yusuf’u; İsa’nın, tüm İsrail’in tahtı üzerinde Davud’un varisi haline gelen bir biçimde büyüyebilmesi amacıyla, Davud’un Şehri Beytüllahim’de kalmaları hakkında ikna etmek zor olmamıştı. Bunun uyarınca, onlar; Yusuf’un bu zaman zarfı boyunca bir takım marangoz işinde çalıştığı biçimde, bir yıldan fazla bir süre boyunca Beytüllahim’de kalmaya devam ettiler.
\vs p122 8:5 İsa’nın öğle vakti doğumunda, yöneticileri tarafından bir araya toplanmış olarak Urantia’nın yüksek melekleri, Beytüllahim samanlığına doğru ihtişamın marşlarını söylediler; ancak, övgünün bu ifadeleri insan kulakları tarafından duyulmamaktaydı. Hiçbir çoban veya diğer fani yaratılmış, Zekeriya tarafından Kudüs’den gönderilmiş olan Ur’lu belirli din\hyp{}adamlarının vardığı güne kadar, Beytüllahim bebeğine hürmet ziyaretinde bulunmaya gelmemişti.
\vs p122 8:6 Mezopotamya’dan gelen bu din\hyp{}adamlarına daha öncesinde belirli bir zaman zarfında; ülkelerinin bir tuhaf\hyp{}dini öğretmeni tarafından, bu öğretmenin gördüğü bir rüyasında kendisinin “yaşam ışığının” bir bebek olarak ve Musevi unsurlarının arasında ortaya çıkacak oluşu hakkında bilgilendirildiği, söylenmişti. Kudüs’de sonuç vermeyen birçok hafta süren arayıştan sonra, onlar; Zekeriya onlarla buluşup, İsa’nın aradıkları kişi olduğuna dair görüşünü ortaya çıkardığında, ve, bebeği buldukları ve dünya annesi olan Meryem’e hediyelerini bıraktıkları yer olan Beytüllahim’e doğru onları gönderdiğinde, Ur’a geri dönmek üzerelerdi. Bu bebek, ziyaretlerinin gerçekleştiği zaman neredeyse üç haftalıktı.
\vs p122 8:7 Bu bilge insanlar, Beytüllahim’e onları yönlendiren hiçbir yıldızı görmemişlerdi. Beytüllahim yıldızının alımlı efsanesi şöyle ortaya çıkmıştı: İsa, M.S. 7.yılda Ağustos’un 21’nde öğlen doğmuştu. M.S. 7.yılda Mayıs’ın 29’nda, Pisces takımyıldızında Jüpiter ve Satürn’ün olağanüstü bir birleşimi ortaya çıkmıştı. Ve, aynı yılın Eylül ayının 29’nda ve Aralık ayının 5’nde benzer birleşimler ortaya çıkmış olması dikkate değer bir gökbilimsel gerçekliktir. Bu olağanüstü ancak tamamiyle doğal kökenli olayların temelinde, daha sonraki neslin iyi niyetli fanatikleri; Beytüllahim yıldızının ilgi çekici efsanesini yaratıp, hayranlık duyan Magi’yi, yeni doğmuş bebeği gözlediği ve ona ibadet ettiği yer olan samanlığa doğru göndermiştir. Doğu ve yakın\hyp{}Doğu akılları, masallardan büyük keyif duymaktadırlar; ve, onlar sürekli olarak, dini önderlerinin ve siyasi kahramanlarının yaşamlarına dair bu tür güzel mitleri yaratmaktadırlar. Matbaanın yokluğunda, insan bilgisinin çoğu bir nesilden diğerine ağızla yayıldığında, mitlerin tarihsel anlatılar haline gelmesi ve tarihsel anlatıların nihai olarak kabul edilmiş bilgiler halinde gelmesi oldukça kolaydı.
\usection{9.\bibnobreakspace Mabed’deki Sunuş}
\vs p122 9:1 Musa Museviler’e öncesinden, her ilk doğan oğlanın Koruyucu’ya ait olduğunu öğretmişti; ve, böyle bir evladın, ebeveynlerinin onaylanmış herhangi bir din\hyp{}adamına yapılacak beş şekellik ödemeyle kurtarması karşılığında, yaşayabilmesi olarak, yabancı milletler arasında bu evladın feda edilmesi yerine bir adet bulunmaktaydı. Orada aynı zamanda; bir annenin, belirli bir sürecin geçmesinden sonra, kendisinin (veya kendisi adına uygun feda verişi gerçekleştireceği birisinin) arınmak için mabette sunmasını emreden, bir Musa\hyp{}dönemine\hyp{}ait yönerge bulunmaktaydı. Bu iki törensel âdeti aynı anda gerçekleştirmek zorunluydu. Bunun uyarınca, Yusuf ve Meryem bizzat; İsa’yı din\hyp{}adamlarına sunmak ve onun günahlardan arınışını gerçekleştirmek, ve aynı zamanda, çocuk doğumunun iddia edilen kirlenmişliğinden Meryem’in törensel arınışını güvence altına almak için gerekli feda verişte bulunmak için Kudüs’deki mabede çıktılar.
\vs p122 9:2 Mabet bahçesi etrafında her zaman, bir şarkıcı olarak Şimon ve bir kadın şair olarak Anne gibi iki çok önemli kişilik dolaşmaktaydı. Şimon bir Yahudiyeli’idi, ancak Anna, bir Celileli’idi. Bu çift sürekli olarak birbirlerine eşlik etmekteydi; ve, her ikisi de, daha öncesinden Yusuf ve İsa’nın sırrını onlarla paylaşmış olan din\hyp{}adamı Zekeriya’nın candaşlarıydılar. Hem Şimon hem de Anna, Mesih’in gelişini derinden arzulamaktaydılar; ve, onların Zekeriya’ya olan güveni, İsa’nın Musevi topluluğunun beklenen kurtarıcısı olduğuna inanmalarına yol açtı.
\vs p122 9:3 Zekeriya, Yusuf ve Meryem’in mabette İsa ile beklenen ortaya çıkışlarının gününü bilmekteydi; ve, Zekeriya, havadaki elinin işareti ile ilk\hyp{}doğan çocukların ilerleyişinde hangisinin İsa olduğunu Şimon ve Anna’ya göstermeyi önceden ayarlamıştı.
\vs p122 9:4 Böyle bir etkinlik için, Anna öncesinden; Yusuf’u, Meryem’i ve mabet bahçesinde toplanan herkesi fazlasıyla şaşırtan bir biçimde, Şimon’un şarkısını söylemeye başladığı bir şiir yazmıştı. Ve, bu, ilk\hyp{}doğan erkek evladın günahlardan arınışına dair onların ilahileriydi:
\vs p122 9:5 Koruyucu’ya minnettar kalın, İsrail’in Tanrısı’na,
\vs p122 9:6 O ki bizleri ziyaret etti ve insanları için günahlardan arınışı mümkün kıldı;
\vs p122 9:7 O hepimiz için kurtuluşun bir boynuzunu çıkardı
\vs p122 9:8 Hizmetçisi Davud’un evinde.
\vs p122 9:9 Kutsal peygamberlerinin ağzından konuşurken ---
\vs p122 9:10 Düşmanlarımızdan ve bizlerden nefret edenlerin hepsinin ellerinden kurtuluşu;
\vs p122 9:11 Atalarımızı bağışlamayı ve onun kutsal sözleşmesini hatırlamayı ---
\vs p122 9:12 Onun atamız İbrahim’e verdiği,
\vs p122 9:13 Düşmanlarımızın ellerinden kurtarılmış olarak bizlerin,
\vs p122 9:14 Korku duymadan hizmet etmemizi,
\vs p122 9:15 Her günümüzde onun karşısında, kutsallığa yaraşır ve doğruluk içinde bulunmamızı sağlayacak yemini.
\vs p122 9:16 İşte, sen, söz verilmiş çocuk, En Yüksek Unsur’un peygamberi olarak adlandırılacaksın;
\vs p122 9:17 Sen ki, ona ait krallığını kurmak için Koruyucu’nun huzuruna çıkacaksın;
\vs p122 9:18 Onun insanlarına kurtuluşun bilgisini,
\vs p122 9:19 Günahlarından arınışlarında vermeyi.
\vs p122 9:20 Tanrımız’ın iyi kalpli bağışlayışını gönlünüzce yaşayın, çünkü yukarıdan bahar şimdi bizlere geldi
\vs p122 9:21 Karanlıkta ve ölümün gölgesinde oturanların üzerine doğmak için;
\vs p122 9:22 Barışın yollarına ayaklarımızı yönlendirmek için.
\vs p122 9:23 Ve şimdi tıpkı verdiğin söz gibi sen Ey Koruyucumuz, hizmetçinin barışla ayrılışına izin ver,
\vs p122 9:24 Ve şimdi tıpkı verdiğin söz gibi sen Ey Koruyucumuz, hizmetçinin barışla ayrılışına izin ver,
\vs p122 9:25 Tüm insanlarının gözleri önünde hazırlamış olduğun kurtuluşu;
\vs p122 9:26 Musevi olmayanları bile aydınlatacak bir ışık
\vs p122 9:27 Ve, senin İsrail insanlarının ihtişamı.
\vs p122 9:28 Beytüllahim’e olan geri dönüş yolculuklarında, Yusuf ve Meryem; kafaları karışmış ve endişelenecek derecede etkilenmiş bir biçimde --- sessizlerdi. Meryem, deneyimli kadın şair olan Anna’nın elveda eden el sallaması tarafında fazlasıyla rahatsız olmuş, ve Yusuf, İsa’nın herkese ilan edilen bir biçimde Musevi topluluğunun beklenmekte olan Mesih’i haline getirilme çabasının bu vaktinden önce gerçekleşen çabasını onaylamamaktaydı.
\usection{10.\bibnobreakspace Hirodes’in Eylemleri}
\vs p122 10:1 Ancak, Hirodes’in gözcüleri hareketsiz konumda değillerdi. Gözcüler, Ur’dan gelen din adamlarının ziyaretini ona sunduklarında, Hirodes, bu Keldani unsurlarını huzuruna çağırdı. Hirodes etraflıca bir biçimde, bu yeni “Museviler’in kralı” hakkında bahse konu bilge insanları sorguladı; ancak, onlar, kocasıyla birlikte nüfus sayımına kayıt olmak için Beytüllahim’e inmiş olan bir kadından bebeğin doğmuş olduğunu açıklayarak, Hirodes’i çok az tatmin ettiler. Bu cevapla tatmin olmamış bir halde Hirodes, din\hyp{}adamlarını bir keseyle gönderip, çocuğunda kendisine gelip ibadet edebilmesini için çocuğu bulmalarını emretti; zira, bu din\hyp{}adamları öncesinden, Hirodes’in krallığının ruhsal olduğunu, zamansal olmadığını, duyurmuş halde bulunmaktaydılar. Ancak, bilge insanlar geri dönmediğinde, Hirodes kuşkulanmaya başladı. Bu meseleleri kafasına döndürüp döndürüp tekrar düşünürken, onun muhbirleri; geri dönüp, kendisine İsa’nın günahlardan arınma törenlerinde söylenmiş olan Şimon şarkısının belli kısımlarının bir nüshasını getiren bir biçimde, mabet içindeki yakın zamanda meydana gelen olayların bütüncül sunumunda bulundular. Ancak, onlar, Yusuf ve Meryem’i takip etmede başarısız olmuşlardı; ve, Hirodes, kendisine hangi çiftin bebeği aldığını söyleyemediklerinde, onlara baya kızgındı. O daha sonra, Yusuf ve Meryem’in yerini bulmak için arayıcılar göndermişti. Hirodes’in Nasıra ailesinin peşine düştüğünü bilen bir biçimde, Zekeriya ve Elizabet, Beytüllahim’den uzak durdular. Erkek bebek, Yusuf’un akrabaları tarafından gizlenmekteydi.
\vs p122 10:2 Yusuf, yapacak iş aramaktan korkuyordu; ve, onların küçük çaplı birikimleri hızlıca yok olmaktaydı. Mabetteki arınma törenleri zamanında bile, Yusuf; Musa’nın fakirler arasında annelerin arınması için emrettiği haliyle, iki genç güvercini Meryem için sunuşunu gerektirecek bir biçimde kendisini yeteri kadar yoksul görmekteydi.
\vs p122 10:3 Bir yıldan fazla bir süre aramalarından sonra Hirodes’in hafiyeleri İsa’nın yerini tespit edemediklerinde ve bebeğin hala Beytüllahim’de gizlenmekte olduğuna dair kuşkusu nedeniyle, o; Beytüllahim’deki her evde düzenli bir aramanın yapılmasını ve iki yaşın altındaki her erkek bebeğin öldürülmesini hükmeden bir emir hazırladı. Böylelikle, Hirodes, “Museviler’in kralı” haline gelecek olan bu evladın yok edilmesini kesin bir biçimde sağlayacağını ümit etmişti. Ve, böylece, Yahudiye’nin Beytüllahimi’nde bir günde on altı erkek bebek yok olmuştu. Ancak, kendi birinci elden ailesi içinde bile olmak üzere, oyun ve cinayet, Hirodes’in sarayında yaygın olarak gerçekleşen olaylardı.
\vs p122 10:4 Bu bebeklerin katliamı, İsa bir yaşının biraz daha fazlasını aldığında, M.S. 6.yılda Ekim’in ortalarında gerçekleşmişti. Ancak, Hirodes’in saray görevlileri arasında bile gelmekte olan Mesih’in inananları bulunmaktaydı; ve, Beytüllahim erkek çocuklarını ortadan kaldıran emri öğrenmiş olarak, onlar bir tanesi, karşılığında Yusuf’a bir iletici göndermiş olan Zekeriya ile iletişime geçti; ve, katliamın öncesindeki akşam, Yusuf ve Meryem bebekleriyle birlikte, Mısır’daki İskenderiye için Beytüllahim’den ayrıldı. İlgi çekmemek için, onlar İsa ile birlikte, Mısır’a yalnız seyahat ettiler. Onlar İskenderiye’ye, Zekeriya tarafından sağlanmış maddi kaynakla gittiler; ve, orada Yusuf işini icra ederken, Meryem ve İsa, Yusuf’un ailesinin varlıklı akrabalarıyla birlikte konakladı. Onlar İskenderiye’de, Hirodes’in ölümüne kadar Beytüllahim’e geri dönmeyen bir biçimde tam iki yıl boyunca konakladıklar.
