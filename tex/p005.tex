\upaper{5}{Tanrı’nın Bireyle olan İlişkisi}
\vs p005 0:1 Eğer insanın sınırlı aklı, bir Tanrı olarak Kâinatın Yaratıcısı’nın ebedi ikamesinden sınırsız bir kusursuzlukta bireysel insan yaratılmışlığıyla bütünleşmek için yeryüzüne inmesinin nasıl mükemmel ve harikulade olduğunu kavrayamıyorsa; böyle bir durumda bu tür sınırlı bir akli yapı, yaşayan Tanrı’nın mevcut bir nüvesinin her olağan akılda ve ahlaksal bakımdan bilinç sahibi Urantia fanilerinde ikamet ettiği kutsal birlikteliğin doğrusuna dayandığı gerçeğinin güvencesiyle iç huzura erişmelidir. İkame eden Düşünce Denetleyicileri Cennet Yaratıcısı’nın ebedi İlahiyat’ının bir parçasıdır. İnsanın, Tanrı’yı bulmak ve onunla gönüldaşlık kurmaya çalışmak için bu ruhani\hyp{}gerçeklik mevcudiyetinin ruh tasavvurlarının kendi içindeki deneyimlemelerinden daha uzağa gitmesine gerek yoktur.
\vs p005 0:2 Tanrı kendi ebedi doğasının sınırsızlığını kendisinin altı mutlak düzenleyicilerinin varoluşçu gerçeklikleri boyunca paylaştırmıştır, fakat onun birey öncesi nüvelerinin kurumsallığı sayesinde herhangi bir bölgeyle, faz ile veya herhangi bir çeşit yaratılmışla doğrudan kişisel bir iletişim kurabilir. Buna ek olarak, Ebedi Tanrı kişilik döngüsü boyunca kişisel varlıklarla doğrudan veya ebeveynsel iletişimini sağlamanın ayrıcalığını fazlasıyla saklı tutarken, ebedi Tanrı kâinatın âlemlerinin tümünün kişilik bahşedeni olmasının imtiyazını aynı zamanda tam anlamıyla saklı tutar.
\usection{1.\bibnobreakspace Tanrı’ya olan Erişim}
\vs p005 1:1 Sınırlı yaratılmışların sınırsız Yaratıcı’ya olan erişimden yoksunluğu Yaratıcı’nın onlara karşı beslediği soğukluktan değil, fakat yaratılmış varlıkların sınırlı ve maddi kısıtlılıklarının doğasından gelir. Evren mevcudiyetinin en yüksek kişiliği ve yaratılmış akli yapıların daha düşük düzeydeki birimleri arasındaki ruhani farklılaşmanın ölçeği idrak edilemez derecede büyüktür. Akli yapıların düşük düzeyde bulunan katmanları Yaratıcı’nın kendi mevcudiyetine anında ulaştırılsalardı, onlar onun mevcudiyetine eriştikten sonra aslında nereye geldiklerini içsel bir biçimde kavrayamazlardı. Tıpkı şimdi onların dünyalarında neden bulunduklarından habersiz oldukları gibi böyle bir durumda Kâinatın Yaratıcısı’nın mevcudiyetinden aynı biçimde habersiz olacaklardı. Kâinatın Yaratıcısı’nın Cennet mevcudiyetine güvenli bir biçimde erişim için fani insanın tutarlı bir biçimde ve olanaklılığın sınırları içerisinde istekte bulunmasından çok önce onun önünde izlemesi gereken çok uzun bir yol bulunmaktadır. İnsanın Yedi Üstün Ruhaniyet’in bir tanesini bile görebilmesine olanak yaratacak ruhani bakışı ona sağlayacak bir düzeye ulaşabilmesi için insanın ruhani bakımdan birçok kez değişmesi ve dönüşmesi gerekir.
\vs p005 1:2 Bizim Yaratıcı’mız hiçbir biçimde ne gizli saklıdır, ne de keyfi bir soyutlanmanın içerisindedir. O kutsal bilgeliğin kaynaklarını bitip tükenmeyen bir çabayla kendisini evrensel bölgelerinin çocuklarına açığa çıkarmak için seferber etmiştir. Onu kavrayan, seven ve ona yaklaşmak isteyen her yaratılmış varlığıyla olan birlikteliğini arzulamasına sebep olan kendi sevgisinin ihtişamıyla iniltili sınırsız bir büyüklük ve tarif edilemez bir cömertlik vardır. Sizin sınırsız kişiliğinizden ve maddi mevcudiyetinizden ayrılamaz bir biçimde, bu durum her şeyin merkezinde bulunan Yaratıcı’nın mevcudiyetine fani yükselişin yolculuğunun hedefiyle erişebileceğiniz ve onda ikamet edeceğiniz zaman, mekân ve koşulları belirler.
\vs p005 1:3 Yaratıcı’nın Cennet mevcudiyetine olan erişim sizin ruhani gelişiminizin en yüksek sınırlılık seviyelerine ulaşmanızı bekliyor olsa da; sizin içsel ruhunuzla ve ruhanileşen benliğinizle oldukça içten bir biçimde ilişkili olan Yaratıcı’nın bahşedilmiş ruhaniyetiyle birlikte ezelden beri var olan dolaysız bütünleşmenin olasılığını tanımada siz bu memnuniyeti duyumsamalısınız.
\vs p005 1:4 Zaman ve mekanın fanileri özlerinden gelen yetenekleri ve akli yapılarının bağışlanma dereceleri bakımından fazlasıyla değişkenlik gösterebilir, onlar aynı zamanda toplumsal gelişmişliğe ve ahlaki ilerleyişe eşine az rastlanacak elverişli çevrelerde bulunmanın ayrıcalığına sahip olabilir, veya medeniyetin sanatı içerisinde kültüre ve beklenen ilerlemeye katkıda bulunacak her türlü insani yardımdan mahrum kalabilirler; fakat yükselimin süreci içerisinde ruhani gelişimin olanaklılığı herkes için eşit bir düzeydedir. Evrimsel dünyalar üzerinde ruhani derinliğin yükselen seviyeleri ve kâinatsal anlamları bu bahsi geçen değişkenlik gösteren maddi çevrelerin tüm bu tür toplumsal ve ahlaki farklılaşmalarından fazlasıyla bağımsız bir biçimde eşit ölçüde erişilir.
\vs p005 1:5 Bu rağmen Urantia fanileri kendilerinin akli, toplumsal, ekonomik ve hatta ahlaki olanaklılıklarının ve kendilerine bağışlanan niteliklerinin bakımından değişkenlik gösterse de, unutmayınız ki onların ruhani bağışlanmışlıkları benzersiz ve eş değerdir. Onların Yaratıcı tarafından verilen kutsal mevcudiyetin aynı hediyesine sahiptirler, ve onlar Gizem Görüntüleyicileri’nin eşdeğer ruhani üstünlüğünü eşit bir biçimde kabul etmeyi tercih ederlerken kutsal kökenin ikame eden ruhaniyetiyle içten bireysel bir bütünleşmeyi aramanın ayrıcalığını hepsi eşit bir biçimde tadarlar. Paper 3
\vs p005 1:6 Eğer fani insan samimi bir biçimde ruhaniyet tarafından yönlendirilmeyi amaç edinir, koşulsuz bir biçimde Yaratıcı’nın iradesini gerçekleştirmeye adanırsa; ikame eden ve kutsal olan Düzenleyici tarafından oldukça kesin ve etkili bir biçimde ruhaniyete sahip edildiği için böylelikle Tanrı’yı tanımanın kutsal bilincini ve Tanrı’yı bulmak amacı için varlığını devam ettirmenin göksel güvencesini bireyin deneyimlemelerinde gerçekleştirememenin başarısızlığı her zaman daha fazla onun gibi olmanın ilerleyici deneyimiyle söz konusu olamayacaktır.
\vs p005 1:7 İnsan ruhani olarak her koşulda varlığını sürdüren bir Düşünce Denetleyicisi tarafından donatılmıştır. Eğer böyle bir insan aklı samimi ve ruhani bir biçimde yönlendirilmeyi amaç edinir, eğer böyle bir insan ruhu Tanrı’yı tanımak ve onun gibi olmayı arzularsa ve Tanrı’nın iradesini yerine getirmeyi dürüst bir biçimde isterse; ne kutsallıkla yönlendirilmeyi arzulamış bir ruhu Cennet’in ana kapılarına güvenle yükselmekten alı koyacak olası bir müdahalenin olumlu bir gücü ne de fani yoksunluğun olumsuz etkisi söz konusu olabilir.
\vs p005 1:8 Yaratıcı tüm yaratılmışlarının kendisiyle birlikte kişisel bütünleşme halinde olmalarını arzular. Böyle bir erişimi erişilebilir haline getirecek varlığını devam ettirici bir düzeye ve ruhani doğaya sahip olan varlıklarının tümünü barındırabileceği Cennet üzerinde bir yere sahiptir. Bu sebeple felsefenizi şimdi sonsuza kadar bu yargılar boyunca belirleyin: “Her biriniz ve hepimiz için Tanrı ulaşılabilir, Yaratıcı erişilebilirdir ve bunu sağlayacak olan yol herkese açıktır; kutsal sevginin güçleri ve biçimleri buna ek olarak kutsal yönetimin araçları, Kainatın Yaratıcısı’nın Cennet mevcudiyeti karşısında her bir evrenin tüm muktedir akıl sahiplerinin gelişimini sağlamak için bir emekte bunların tümü bir araya gelmiştir.
\vs p005 1:9 Tanrı’ya erişim süreci boyunca geçen büyük bir zamanın bilgisi Sınırsızlık’ın kişiliğinin mevcudiyetini daha az gerçek haline getirmez. Sizin yükselmeniz yedi aşkın evrenin dairesel döngüsünün bir parçasıdır, ve onun çevresinde sayısız defa salınmanıza rağmen siz ruhaniyet ve içinde bulunduğunuz seviyede başından beri içe doğru bir dönüşü bekleyebilirsiniz. Siz bölgeden bölgeye, dışsal döngülerden içsel merkezin yakınlarına olan terfiiniz bakımından koşullu bir doğaya sahip olabilirsiniz, fakat şundan kuşku duymayınız ki siz kutsallığın ve merkezi mevcudiyetin karşısında durmalı ve onu, görsel olarak söylemek gerekirse, karşı karşıya görmelisiniz. Bu durum sadece mevcut ve belirli olan ruhani seviyelere olan erişimin bir sorunsalıdır; ve bu ruhani düzeyler bir Gizem Görüntüleyicileri tarafından ikame edilmiş bununla birlikte ebedi bir biçimde Düşünce Denetleyiciler’i tarafından bütünleştirilmiş her varlık tarafından erişilebilir.
\vs p005 1:10 Yaratıcı ruhani bir gizleniş içerisinde değildir; fakat onun birçok yaratılmışları, kendilerini irade dâhilinde gerçekleştirdikleri kararların sis perdesi arkasına saklarlar ve yine kabul edilemez yöntemleri tercihi etmeleri ve hoşgörüsü olmayan ruhsuz doğalarının kendisini açığa çıkarmalarına karşı koyamamaları sonucunda kendilerini onun ve Evladı’nın ruhaniyetinin birlikteliğinden bir süreliğine ayırırlar.
\vs p005 1:11 Fani insan Tanrı’ya yakınlaşabilir ve aynı zamanda karar verme gücü kendisinde bulunmaya devam ettikçe kutsal iradeyi istediği kadar dışlayabilir. İnsanın geri döndürülemez biçimde son buluşu Yaratıcı’nın iradesini seçme gücünü kaybedene kadar gerçek bir kesinliğe kavuşmaz. Onun çocuklarının ihtiyaçlarına ve ricalarına karşı Yaratıcı’nın kalbinin herhangi bir biçimde kapanması söz konusu bile değildir. Bu kapanış sadece, ondan doğumu olan çocuklarının onun kutsal iradesi olan onu tanımak ve onun gibi olmak biçimindeki öğretisini uygulamada tüm istencini kesin ve ebediyen kaybettiğinde onların kalplerinin sonsuza kadar Yaratıcı’nın çekici gücüne kapanması şeklinde kendisini gösterir. Yine aynı şekilde, insanın ebedi nihai sonu Düzenleyici’nin birleşiminin evrene yaptığı “bu tür bir yükselen varlığın Yaratıcı’nın iradesini gerçekleştirmek için kesin ve geri dönüşü olmayan bir tercihte bulundu” açıklamasıyla birlikte kesinleşir.
\vs p005 1:12 Muhteşem Yaratıcı fani insanlarla birlikte dolaylı yollardan iletişim halindedir, çünkü kendisinin sınırsız, ebedi ve kavranamaz derecedeki benliğinin bir parçasını onu yaşamak ve onun içinde ikamet etmek için bu varlığına verir. Tanrı bu ebedi serüvene insan ile birlikte yola çıkmıştır. Eğer siz kendi içinizdeki ve çevrenizdeki ruhani güçlerin yönlendirmelerine izin verirseniz, mekânın evrimsel dünyalarından onun yükselen yaratılmışlarının evrensel hedefi olarak sevgi dolu bir Tanrı tarafından oluşturulmuş en yüksek nihai sona ulaşmada başarısız olamazsınız.
\usection{2.\bibnobreakspace Tanrı’nın Mevcudiyeti}
\vs p005 2:1 Sınırsızlığın fiziksel mevcudiyeti maddi evrenin gerçekliğidir. İlahiyat’ın akli mevcudiyetinin anlaşılması bireysel zihinsel deneyimlerinin derinliği ve evrimsel kişilik düzeyi tarafından yakından alakalıdır. Kutsallık’ın ruhani varlığının evrende farklılaşan bir niteliğe sahip olması ihtiyaç bakımından zorunludur. Onun varlığı algılamanın ruhani yetisi ve yaratılmışın kutsallığın iradesini uygulamadaki adanmışlığın derecesi tarafından belirlenir.
\vs p005 2:2 Tanrı onun her ruhaniyetle doğan evladının içinde yaşar. Cennet Evlatları Tanrı’nın varlığına “Yaratıcı’nın sağ kolu olarak” her zaman erişim haline sahiptirler. Bununla birlikte onun yaratılmış kişiliklerinin tümü “Yaratıcı’nın bağrına” olan bağlanmışlığa sahiptirler. Bu durumun kendisi kişilik döngüsünün kaynağını oluşturur; buna göre nerede, ne zaman ve hangi koşulda olursa olsun, onun ikamet ettiği merkezi yerleşkesinde veya Cennet’in yedi kutsal bölgelerinden biri üzerinde bulunan diğer belirlenmiş yerlerde sürekli temas halinde olan veya böyle olmadığı durumlarda Kâinatın Yaratıcısı ile kişisel, öz bilinç dâhilinde bir iletişim ve bütünlük mevcuttur.
\vs p005 2:3 Kutsal mevcudiyet, buna rağmen, doğa üzerinde hiçbir yerde ve hatta Tanrı\hyp{}bilen fanilerinin yaşamlarında oldukça tamamlanmış ve kesinleşmiş haliyle Gizem Gözetleyicileri ve Cennet Düşünce Denetleyicileri’nin ikamesiyle bütünleşmenin çabasında bile keşfedilemez. Kâinatın Yaratıcısı’nın ruhaniyeti sizin kendi aklınız içerisinde ikamet ediyorken Tanrı’yı ufkun ve gökyüzünün çok ötesinde düşlemeye çalışmak ne de büyük bir hatadır!
\vs p005 2:4 Tanrı nüvesinin sizin içinizde olan ikamesi sebebiyle, siz Düzenleyici’nin ruhani yönlendirmesiyle uyumlu hale gelmede ilerlerken sizin başat etken bir parçanız olarak faaliyet göstermeyen fakat sizin üzerinizde veya çevrenizde diğer ruhani tesirlerin dönüştürücü gücünü ve mevcudiyetini daha fazla bir biçimde tamamen algılayacağınızı ümit edebilirsiniz. Sizin bilgisel bakımdan ikamet eden Düzenleyici ile yakın ve içten olan ilişkinizin bilincine sahip olmamanız böyle bir engin deneyimin gerçekleşmemesini bir parça bile olsun desteklemez. Kutsal Düzenleyici ile olan bütünleşmenin kanıtı tamamen, bireysel inananın hayat deneyiminde üretilmiş ruhaniyetin içinden türeyen sonuçların kapsamından ve doğasından oluşur. “Onların sonuçlarından onların aslında ne olduklarını bilmelisin.”
\vs p005 2:5 Cennet Düzenleyicileri gibi bu tür kutsal varlıkların ruhani faaliyetlerinin belirli bir bilincine varmak eksik bir biçimde ruhaniyetini tamamlayamamış fani insanın maddi aklı için oldukça zordur. Düzenleyici yaratılmışlık ve birleşik aklın ruhu artan bir biçimde kendi varlığını hissettirmeye başlayınca, Gizem Görüntüleyicileri’nin varlığını, tanımlayıcı ruhani yönlendirmelerini ve diğer madde üstü faaliyetlerini deneyimlemeye yetkin hale gelecek ruh bilinci yeni bir faza doğru gelişir.
\vs p005 2:6 Düzenleyici birlikteliğin bütüncül deneyimi bir katılımcı ahlaki düzey, ussal yönelim ve ruhani deneyimdir. Böyle bir başarının kendisini gerçekleştirmesi başlıca olarak fakat istisnai bir özellik göstermeden ruh bilincinin bölgeleriyle sınırlıdır, buna rağmen tüm bu tür içsel ruh iletişimcilerinin yaşamlarında ruhaniyetin ürünlerinin kendilerini açığa çıkarması hususunda kanıtlar oldukça zengin ve her zaman her ihtiyaca cevap verecek niteliktedir.
\usection{3.\bibnobreakspace Gerçek İbadet}
\vs p005 3:1 Evrenin bakış açısından Cennet İlahiyatları bir olmasına rağmen onların Urantia’da ikamet eden bu tür varlıklarla olan ruhani ilişkilerinde onlar aynı zamanda üç farklı ve üç ayrı kişiliktirler. Tanrılıklar arasında kişisel çağrılar, bütünleşmeler ve diğer içten ilişkiler konularında bir farklılık vardır. En yüce bağlamda, biz Kâinatın Yaratıcısı’na yalnızca ona olmak üzere ibadet ederiz. Yaratıcı’nın Yaratan Evlatları’nda kendini dışa vurduğu haliyle bizim onlara ibadet edebilmemiz ve bunu gerçekleştiriyor olmamız doğru bir yargıdır, fakat Yaratıcı doğrudan veya dolaylı olarak bizim ibadet ettiğimiz ve hayran olduğumuz kişiliktir.
\vs p005 3:2 Yakarışların tüm türleri Ebedi Evlat ve Evlat’ın ruhani idare alanına aittir. Dualar tüm resmi iletişimler olarak Kâinatın Yaratıcısı’na olan ibadet ve hayranlığın dışında yerel bir evreni ilgilendiren hususlardır; dualar bu bakımdan bir Yaratan Evlat’ın yetki alanına alışılagelmiş bir biçimde girmez. Fakat ibadet kuşkusuz Yaratıcı’nın kişiliğinin döngüsünün bir faaliyeti tarafından Yaratan’ın bünyesine yönlendirilir ve ona ulaştırılır. Biz buna ek olarak, bir Düzenleyici\hyp{}ikame edilmiş yaratılmışın bu tür onursal tescilinin Yaratıcı’nın ruhani mevcudiyeti tarafından zemin hazırlandığına inanmaktayız. Bu tür inanışı destekleyecek olağanüstü bir ölçekte kanıtlar bulunmaktadır, ve Yaratıcı nüvelerinin tüm emirlerinin Kâinatın Yaratıcısı’nın varlığında kabul edilecek bir biçimde onun bireylerinin tüm samimiyetiyle ona hayranlık beslemesini sağlamak için verildiğinin bilgisine sahibim. Kuşkuya hiçbir biçimde yer bırakmayacak bir biçimde Düzenleyiciler aynı zamanda Tanrı’yla bütünleşmenin doğrudan birey öncesi bağlantılarını kullanır, ve onlar buna ek olarak Ebedi Evlat’ın ruhani\hyp{}çekim döngülerinden yararlanmaya yetkindir.
\vs p005 3:3 İbadette bulunmak sadece ibadeti gerçekleştirmek içindir, dua ise öz benliği veya yaratılmışın kendisi için herhangi bir isteğinin unsurunu somutlaştırır; bu bakımdan ibadet ile dua arasında büyük bir fark vardır. Gerçek bir ibadette kesinlikle ne bireysel rica ne de kişisel beklentilerin diğer unsurlarından biri bulunur; biz sade bir değişle Tanrı’ya onu nasıl idrak ettiğimiz uyarınca ibadet ederiz. İbadet kendisinin yerine getirilmesi için hiçbir koşul öne sürmez, ve ibadet onu gerçekleştirenden hiçbir beklentisi yoktur. Biz Yaratıcı’ya karşılığında ondan herhangi bir saygı elde etmek beklentisiyle ibadet etmeyiz. Bunun yerine, biz bu tür sadakati gösterir ve böyle saf bir ibadetle içli dışlı olurken bunu sadece onun sevgi dolu doğası ve hayranlık uyandırıcı özellikleri ve Yaratıcı’nın eşi benzeri olmayan kişiliğinin tanımak için bir doğal ve kendiliğinden olan yansıma biçiminde yaparız.
\vs p005 3:4 Bireysel beklenti unsuru ibadet süreci içerisine zorla dâhil olduğu andan itibaren kendiliğinden olan bireyin kendini adaması ibadetten duaya dönüşür, ve böyle bir durumda bu tür bir değişimin yaşandığı ibadet daha uygun olan bir biçimde Ebedi Evlat’ın veya Yaratan Evlat’ın kişiliği adına yapılmalıdır. Fakat uygulanan dinsel deneyim içinde neden duanın Yaratıcı olan Tanrı’ya gerçek ibadetin bir parçası olarak yapılamayacağı hakkında aleni herhangi bir sebep yoktur.
\vs p005 3:5 Siz günlük yaşamınızın işleyişsel olaylarıyla başa çıktığınızda Üçüncü Kaynak ve Merkez’in içinde kökeni olan ruhaniyet kişiliklerinin gözetiminde olup siz Bütünleştirici Bünye’nin kurumlarıyla birlikte eş güdüm halindesinizdir. Ve böylelikle: Siz Tanrı’ya dua eder, ve Evlat’la bütünleşir; bununla beraber sizin dünyanızda ve evreniniz boyunca faaliyette bulunan Sınırsız Ruhaniyet’in akli yapılarıyla ilişki dâhilinde kendi dünyevi kısa süreli olan ikamenizin detaylarının sorunlarını çözmeye çalışırsınız.
\vs p005 3:6 Yerel Evrenlerin nihai sonları üzerinde hüküm süren Yaratan veya Egemen Evlatlar Kâinatın Yaratıcısı’nın ve Cennetin Ebedi Evladı’nın yerleşkesinde onların vekâletinde görevlerini sürdürürler. Evren Evlatları Yaratıcı adına ibadetin hayranlığını şükranlıkla kabul eder ve onların ilgili yaratılmışları boyunca bu yaratılmışların öz benlikleriyle ilgili ricalarına kulak verirler. Yerel bir evren çocuklarına göre bir Mikâil Evladı tüm işlevsel gerekçeleri ve niyetleri bakımından Tanrı’dır. O Kâinatın Yaratıcısı’nın ve Ebedi Evlat’ın yerel evren kişilikleştirilmiş halidir. Sınırsız Ruhaniyet, Cennet Yaratan Evlatları’nın yaratıcı ve idari yardımcıları olan Evren Ruhaniyetleri boyunca bu âlemlerin çocuklarıyla birlikte kişisel iletişimi sürdürür.
\vs p005 3:7 İçten ibadet, eş güdüm halindeki Düşünce Denetleyicileri’nin kutsal yönlendirmelerine bağlı ve evrimleşen ruhun baskınlığı altında insan kişiliğinin tüm güçlerinin devinimine atıfta bulunur. Maddi kısıtlanmaların aklı, gerçek ibadetin taşıdığı asli önemin yüksek bilincine hiçbir zaman nail olamaz. İnsanın ibadet deneyimini gerçekleştirmesi onun evrimleşen ruhunun gelişimci düzeyi tarafından başlıca olarak belirlenir. Ruhun ruhani gelişimi akli birey bilincinden tamamen bağımsız bir biçimde gerçekleşir.
\vs p005 3:8 İbadet deneyimi, Tanrı’yı bulmaya çalışan fani aklın birleşik yaratılmışlığı ve Tanrı’yı açığa çıkaran ölümsüz Düzenleyici’den oluşan insan ruhunun tarifsiz derecede yoğun arzularının ve açıklanamayacak özlemlerinin kutsal Yaratıcı’yla olan iletişimini sağlamak için süreç dâhilinde ilişki içerisinde bulunan Düzenleyici’nin ulvi çabalarından bir araya gelir. Böylelikle ibadet, maddi aklın onun ruhanileşen öz benliğini onaylaması uğraşının bir faaliyeti olarak, ilişkide bulunduğu ruhaniyetin rehberliği altında Kâinatın Yaratıcısı’nın bir inanç evladı olarak Tanrı’yla iletişim kurmasıdır. Fani akıl ibadet etmeye razı olur; sınırsız ruh derin bir biçimde ibadeti arzular ve onu gerçekleştirir; kutsal Düzenleyici varlığı böyle bir ibadeti evrimleşen ölümsüz ruhun ve fani aklın adına yerine getirir. Gerçek ibadet son kertede dört kâinatsal düzeyde bir deneyim halinde gerçekleşir: aklın bilinci olarak ussallık, ruhun bilinci olarak morontial, ruhaniyetin bilinci olarak ruhi ve tüm bunların kişilikte birleşimi olarak kişisel seviyeleridir.
\usection{4.\bibnobreakspace Din İçerisinde Tanrı’nın Yeri}
\vs p005 4:1 Evrimin dinlerinin ahlak anlayışı insanları Tanrı arayışına korkunun güdüsel gücü vasıtasıyla ileriye doğru \bibemph{sürükler}. Gerçekleri açığa çıkarmanın oluşturduğu dinler insanları bir Tanrı sevgisinin peşine düşmek için onların \bibemph{aklını çeler}, çünkü onlar Tanrı gibi olmanın derin bir arzusunu duyarlar. Fakat din yalnızca “mutlak bir bağlılık” ve “hayatı devam ettirmenin olmazsa olmazı” gibi niteliklerin oluşturduğu bir durağan hissiyat değildir; bunun yerine din insanlığın hizmetine dayandırılmış kutsallığa erişimin yaşayan ve sürekli devinim içerisinde olan halidir.
\vs p005 4:2 Gerçek dinin büyük ve doğrudan hizmeti, insan deneyiminde gerçekleşecek sonsuza kadar sürecek bir barışta ve engin derinlikte bütünlüğü sağlamanın oluşumudur. İlk insanla birlikte, çok tanrılı dinler bile İlahiyat’ın evrimleşen kavramsallaşmasının göreceli bir bütünlüğüdür. Er ya da geç Tanrı nihai olarak değerlerin gerçekliği, anlamların özü ve doğruluğun yaşamı olarak kavranmasının yazgısına sahiptir.
\vs p005 4:3 Tanrı sadece kaderin bir belirleyicisi değildir; \bibemph{o }aynı zamanda insanın ebedi istikametidir. Tüm din\hyp{}dışı insan faaliyetleri evreni, bireyin zarar verici hizmeti doğrultusunda şekillendirme amacındadır; içten bir biçimde inanan dindar birey, kendi benliğini evrenin bütüncül varlığı ile tanımlar ve bunun sonucunda bu bütünleşmiş bünyesinin faaliyetlerini insan ve insan\hyp{}üstü unsurlar olarak ortak kaderi paylaştığı varlıkların oluşturduğu evren ailesinin hizmetine adar.
\vs p005 4:4 Sanat ve felsefenin etki alanı insan özünün dini ve dinsel olmayan faaliyetlerinin arasında bir yerde kendisine yer bulur. Sanat ve felsefe vasıtasıyla, maddi akla sahip olan insan ebedi anlamların kâinatsal değerlerinin ve ruhani gerçekliklerinin tasavvuru içine çekilir.
\vs p005 4:5 Tüm dinler İlahiyat’ın ibadetini ve insanın kurtuluşunun bazı öğretilerinin öğrenilmesini amaçlar. Budist dini ıstıraplardan kurtuluşu sonu gelmeyecek bir barış içinde; Musevi dini zorluklardan kurtuluşu zenginliğin doğruluk üzerine dayanmasında; Yunan dini uyumsuzluktan ve çirkinlikten kurtuluşu güzelliğin kendisini açığa çıkarmasında; Hıristiyanlık günahtan kurtuluşu kutsallıkta; İslamiyet ise kurtuluşu Musevilik ve Hıristiyanlık’ın katı ahlaki öğretilenlerin arınışta müjdeler. İsa’nın dini ise zaman ve ebediyetin içerisinde yaratılmışın tecridinin kötülüklerinden arındırılışı ve bireyin kendisinden \bibemph{kurtuluşudur}.
\vs p005 4:6 İbraniler dinlerini iyilik, Yunanlılar ise güzellik üzerine dayandırdılar; sonuçta bu iki din de bu öğretileriyle doğruyu aramaya çalıştılar. İsa ise bir sevginin Tanrı’sını açığa çıkardı, çünkü derin sevgi hem gerçeğin, hem güzelliğin ve hem de iyiliğin bütününü kapsamı içerisine alır.
\vs p005 4:7 Zerdüştlerde ahlaki ilkelerin, Hindularda metafiziğin, Konfüçyanizmde ise etik değerlerin oluşturduğu bir din anlayışı mevcuttur. İsa ise \bibemph{hizmetin dinini} yaşadı. Bahsi geçen bu üç din de içlerinde İsa’nın dininin özüne olan geçerli yaklaşımları barındırmaları sebebiyle bir değer teşkil ederler. Din insan deneyiminde iyi, güzel ve gerçek olanın hepsinin bütünsel ruhani birleşiminin gerçekliği haline nihayeten gelmesinin yazgısına sahiptir.
\vs p005 4:8 Yunan dini “Kendini tanı” biçiminde genel bir öğretiye sahipti; İbrahimler “Tanrı’nı tanı” öğretisini merkezine aldı; Hıristiyanlar İncil’in öğretilerinden biri olan “Koruyucu Hazreti İsa’nın bilgisi” vaazını verdiler. Bunların karşısında ise İsa “Tanrı’yı bilmenin sizin Tanrı’nın bir evladı olarak bilmeniz” anlamına geleceğinin olumlu haberini bildirdi. Dinin amaçsal farklılaşan bu kavramları, bireyin değişken hayat şartlarda onların davranışlarını belirler, ve bireysel dua alışkanlıklarının doğasının ve ibadetinin derinliğinin habercisi olur. Bu bakımdan herhangi bir dinin ruhani düzeyi onun dualarının doğası tarafından belirlenebilir.
\vs p005 4:9 Yarı\hyp{}insan ve kıskanç Tanrı kavramsallaşması çoklu dinler ile ulvi tek tanrılı dinler arasında kaçınılmaz olan bir geçiş döneminin ürünüdür. Tanrı’ya insana dair niteliklerin atfedilmesi ve onun bu özellikler tarafından tahayyül edilmesinin bir engin biçimi saf olarak evrimleşen dinin en yüksek erişim düzeyidir. Hıristiyanlık bu insanbiçimcilik kavramsallaşmasını insanın nihai hedeflerinden yüceltilmiş Hazreti İsa’nın kutsal ve aşkın kişilik kavramsallaşmasına yüceltmiştir. Ve bu anlamsal yüceltme insanın algılayabileceği en yüksek insanbiçimciliğidir.
\vs p005 4:10 Tanrı’nın Hıristiyan kavramsallaşması birbirinden ayrı üç öğretinin birleştirilmesinin bir denemesidir.
\vs p005 4:11 1.\bibnobreakspace \bibemph{İbrani dininin kavramsallaşması} --- Tanrı ahlaki değerlerin yargılarının haklılığını doğrulayan olarak, doğruluğun Tanrısı.
\vs p005 4:12 2.\bibnobreakspace \bibemph{Yunan dininin kavramsallaşması} --- Tanrı bir bütünleştirici olarak, bilgeliğin bir Tanrısı.
\vs p005 4:13 3.\bibnobreakspace \bibemph{İsa’nın kavramsallaşması} --- Tanrı yaşayan bir arkadaş, bir sevgi dolu Yaratıcı olarak, kutsallığın mevcudiyeti Tanrı.
\vs p005 4:14 Bu bakımdan birçok farklı öğelerden oluşmuş Hıristiyan tanrıbiliminin kendi bünyesinde tutarlılığa ulaşmada büyük zorlukla karşılaşmasının nedeni bariz olmalıdır. Bu zorluk Hıristiyanlığın erken dönem öğretilerinin genel olarak İskenderiyeli Filo, Nasıralı İsa, ve Tarsuslu Paul’dan oluşan üç farklı kişinin bireysel dini deneyimlerine dayanması sebebinin gerçeğiyle daha fazla bir biçimde derinleşmiştir.
\vs p005 4:15 İsa’nın dinsel yaşamının irdelenmesinde ona olumlu olarak bakın. Onun doğruluğu ve günahlardan arınmışlığı hakkında fazla düşünmek yerine, onun sevgi dolu hizmetini önemseyin. İsa cennetsel Yaratıcı’nın İbrani dinindeki kavramsallaşmasında açığa vurulmuş durağan sevgi anlayışından, her bireyin hatta kötülük işleyenin Yaratıcı’sı olan bir Tanrı’nın daha üstün olan \bibemph{etkin} ve yaratılmış\hyp{}sevgi şefkatinin yüksek kavramsallaşmasını açığa çıkarmıştır.
\usection{5.\bibnobreakspace Tanrı’nın Bilinci}
\vs p005 5:1 Ahlak’ın kökeni birey öz bilincinin nedenselliğindedir; bu durum hayvanlar üstü bir durum arz eder, fakat tamamen evrimseldir. İnsan evrimi, Gerçekliğin Ruhaniyeti’nin beslenişine ve Düzenleyiciler’in bahşedilişine öncüllük eden, kendisine ihsan edilmiş tüm niteliklerin ortaya çıkmasıyla karşılaşır. Fakat ahlakın bu erişim düzeyleri insanı fani yaşamının gerçek mücadelelerinden özgürleştirmez. İnsanın fiziksel çevresi varoluşun mücadelesini zorunlu kılar; onun toplumsal çevrilmişlikleri etik düzenlemelerinin varlığını gerektirir; ahlaki durumlar nedenselliğin en yüksek düzeylerinde tercihler yapmayı şart koşar; ve nihayet Tanrı’yı gerçekleştirmeden doğan ruhani deneyim ise insanın onu bulmasını ve samimi bir biçimde onun gibi olmasını arzulamasını ondan bekler.
\vs p005 5:2 Din; ne bilimin gerçeklerinde, ne toplumun ödevlerinde, ne felsefenin varsayımlarında ne de ahlakın ima edilen görevlerinde temellenmiştir. Din hayat şarlarına karşı insan tepkilerinin bağımsız bir düzeyidir ve ahlak sonrası olan insan gelişiminin tüm seviyelerinde hataya yer bırakmayacak bir biçimde dışa vurulur. Din, değerlerin ve evren birlikteliğinin coşkusunun kendisini gerçekleştirmesinin dört düzeyinin tümüne nüfuz edebilir. Bu düzeyler; bireyin kendini korumasının maddi ve fiziksel düzeyi, birlikteliğin toplumsal ve duygusallık düzeyi, nedenselliğin ahlaki ve görevsel düzeyi, ve kutsal ibadet boyunca evren birliktelik bilincinin ruhani düzeyidir.
\vs p005 5:3 Gerçeği arayan bilim adamı kudretin bir Tanrı’sı biçiminde Tanrı’yı İlk Sebep olarak algılar. Duygusal sanatçı estetiğin bir Tanrı’sı biçiminde Tanrı’yı nihai güzellik olarak görür. Nedensel düşünen filozof Tanrı’yı bir evrensel bütünlük hatta bir panteistik İlahiyat olarak önermeye zaman zaman yatkınlaşır. İnancın sofusu varlığı devam ettiren olarak Tanrı’ya, cennetteki Yaratıcı’ya ve sevginin Tanrı’sına inanır.
\vs p005 5:4 Ahlaki davranış her zaman evrimleşen dinin ve hatta açığa çıkarılan dinin bir parçasının öncülüdür, fakat bu davranış hiçbir zaman dinsel bir deneyimin tümünü teşkil etmez. Toplumsal hizmet ahlaki düşünüşün ve dinsel yaşamın bir sonucudur. Ahlak, dinsel deneyimin daha yüksek olan ruhani düzeylerine biyolojik olarak öncülük etmez. Yapay güzelliğe olan hayranlık Tanrı’ya ibadet değildir; ne doğanın yüceltilmesi ne de bütünlüğe duyulan derin saygı Tanrı’nın ibadeti olamaz.
\vs p005 5:5 Evrimsel din; insanı, fiziksel algı düzeyinden Düzenleyiciler’in bahşedilişinin ve Gerçekliğin Ruhaniyeti’nden sonuçlarının dâhil olduğu açığa çıkarılmış dine yükselten bilimin, sanatın ve felsefenin anasıdır. Evrimsel ve biyolojik, açığa çıkarımsal ve dönemsel olan birbirinden çok farklı nitelikte dinlerin olmasına rağmen insan varlığının evrimsel resmi dinle başlar ve dinle biter. Ve böylelikle, din insana olağan ve doğal görülürken aynı zamanda onun için dinler arasında farklılaşmadan dolayı tercihseldir. Dolayısıyla insan kendi rızasına aykırı gelecek bir biçimde dindar olma zorunluluğunda değildir.
\vs p005 5:6 Öz bakımından ruhani olan dinsel deneyim maddi akıl tarafından hiçbir zaman tamamen anlaşılamaz; bu bakımdan tanrı biliminin faaliyeti ve dinin psikolojik olgusallığı ussal değildir. Tanrı’nın insan gerçekleştirmesinin temel öğretisi bu sınırlı olan kavrama yetisinde bir çelişki yaratır. Tanrı’nın her bireyin içinde ve onun bir parçası olması, onun aşkınlığının düşüncesi ile birlikte kâinatın âlemlerinin tümünün kutsal üstünlüğünün ulvi içkinliğinin kavramsallaşmasını uyumlu hale getirmek neredeyse imkânsızdır. İlahiyat’ın bu iki kavramsallaşması, kişiliğin varlığını devam ettirme ümidini doğrulamak ve ussal ibadeti haklı çıkarmak için kişisel bir Tanrı’nın aşkınlığının kavramının içinde inanç algılayışında ve Tanrı’nın bir nüvesinin ikamet eden mevcudiyetinin kendisini gerçekleştirmesinde bütünleşmelidir. Din içerisinde var olan anlayışa dayalı zorluklar ve karmaşalar; onun önermiş olduğu doğruların, ussal kavrayış için fani yetkinliğin tamamiyle ötesinde bulunması gerçeğinden kaynaklanmaktadır.
\vs p005 5:7 Fani insan dinsel deneyimlerden dünya üzerindeki geçici ikamesi sürecinde geçen günlerde bile üç büyük tatminiyet elde eder:
\vs p005 5:8 1.\bibnobreakspace \bibemph{Ussal olarak} daha fazla bütünleşmiş bir insan bilincinin memnuniyetini elde eder.
\vs p005 5:9 2.\bibnobreakspace \bibemph{Felsefi olarak} kendi nihai amaçları içinde ahlaki değerlerinin doğrulanmasının coşkusunu yaşar.
\vs p005 5:10 3.\bibnobreakspace \bibemph{Ruhani olarak} gerçek ibadetin ruhani memnuniyetinde olan kutsal bütünleşmesinin deneyiminde gözle görülür bir biçimde büyür ve gelişir.
\vs p005 5:11 Tanrı bilinci, âlemlerin evrimleşen bir fanisi tarafından deneyimlendiği gibi gerçekliğin ortaya çıkmasının üç farklı düzeyinden ve onların taşıdığı içerik bakımından üç değişken etkenden meydana gelir. Birincil olarak Tanrı \bibemph{düşüncesinin} kavranması --- akli bilinç bulunur. Bunun akabinde Tanrı \bibemph{düşüncesinin} gerçekleştirilmesi --- ruh bilinci onu takip eder. Sonuncusu ise Tanrı’nın\bibemph{ ruhani gerçekliğinin} gerçekleştirilmesi --- ruhaniyet bilinci olarak kendisine yer bulur. Hangi bir biçimde nasıl tamamlanmamış olduğundan bağımsız, kutsal kendini gerçekleştirmenin bu etmenlerinin bütünleşmesi tarafından fani kişilik tüm zamanlarda bir Tanrı’nın \bibemph{kişiliğinin} ortaya çıkışıyla birlikte bilinç düzeylerinin bütününe yayılır. Kesinliğe Erişecek Olanların Birlikleri’ne erişen bu fanilerde tüm bu düzeyler Tanrı’nın \bibemph{yüceliğinin} zaman içerisinde ortaya çıkışına öncülük edecektir ve Cennet Yaratıcısı’nın absonit üstün bilincinin bazı fazları olan Tanrı’nın \bibemph{nihayetinin} gerçekleşmesinde birbirini takip eden biçimlerde sonuçlanmasına sonlanabilirler.
\vs p005 5:12 Tanrı\hyp{}bilincinin deneyimi kuşaktan kuşağa değişmeyen bir biçimde aynı kalır, fakat her ilerleyen çağda Tanrı’nın tanrıbilimsel ve felsefi kavramsallaşmasının bilgisi değişmek \bibemph{zorundadır}. Tanrı’nın bilgisini bilme durumu ve dinsel bilinç bir evren gerçekliğidir, fakat gerçek dinsel deneyim her ne kadar geçerli olursa olsun bu gerçeklik kendisini ussal eleştirilere ve mantıksal felsefi yorumlara açıklılıkta istence sahip olması gerekir. Bu bağlamda insan deneyimlerinin bütünlüğünden ayrık bir biçimde bir şeyin arayışında bulunmaması zorunludur.
\vs p005 5:13 Kişiliğin ebedi varlığını devam ettirmesi, kararları ölümsüz ruhun varlığını devam ettirme olanağını belirleyen fani aklın tercihine tamamen bağımlıdır. Akıl Tanrı’ya inandığı ve ruh Tanrı’yı tanıdığı zaman, bununla birlikte destekleyici Düzenleyici’yle birlikte bu esnada hepsinin Tanrı’yı \bibemph{arzulamasıyla} ruhun varlığını devam ettirmesi kesinleşir. Akli yapının sınırlılığı, eğitimin perdelenmesi, kültürden yoksunluk, toplumsal düzeyin fakirleşmesi, hatta eğitimin, kültürün ve toplumsal faydaların talihsiz eksikliğinden kaynaklanan ahlakın insani ölçütlerindeki düşüklük bile, kutsal ruhun mevcudiyetini böyle bir talihsiz ve insani bir biçimde engellenmiş ama inançlı bireylerde varlığını ortadan kaldıramaz. Gizem Görüntüleyicisi’nin ikamesi ölümsüz ruhun varlığını devam ettirmesinin kuruluşunu oluşturur ve onun olası ilerlemesinin olanaklılığının teminat altına alır.
\vs p005 5:14 Fani ebeveynlerin doğurganlık yetisi onların eğitimsel, kültürel, toplumsal veya mali düzeylerinden bağımsızdır. Doğal koşullar altında ebeveynsel etkenlerin birliği, doğumu başlatmak için fazlasıyla yeterlidir. Tanrı’ya ibadetin yetisine sahip ve neyin doğru neyin yanlış olduğunu anlayan insan aklı, bir kutsal Düzenleyici’yle bütünlük içerisinde; eğer böyle bir ruhaniyet\hyp{}ihsanına sahip birey Tanrı’yı arıyor ve samimi bir biçimde onun gibi olmayı arzuluyor, dürüstçe cennette bulunan Yaratıcı’nın iradesini yerine getirmeyi seçiyorsa, fani insanda varlığını devam ettirme niteliklerine sahip ölümsüz ruhun yeniden üretimini başlatmasının ve onu desteklemesinin tüm koşullarına sahiptir.
\usection{6.\bibnobreakspace Tanrı’nın Kişiliği}
\vs p005 6:1 Kâinatın Yaratıcısı kişiliklerin Tanrı’sıdır. Evren kişiliğinin nüfuz alanı, kişilik düzeyinin maddi ve en düşük fani yaratılmışlığından kutsal düzeyin ve yaratan saygınlığının en yüksek bireylerine kadar, Kâinatın Yaratıcısı’nda kendi odağına ve çembersel merkeze sahiptir. Yaratıcı olan Tanrı her kişiliğin bahşedicisi ve koruyucusudur. Buna ek olarak, Cennet Yaratıcısı benzer bir biçimde kutsal iradeyi tüm kalbiyle gerçekleştirmeyi tercih eden, Tanrı seven ve onun gibi olmayı özlemleyen tüm sınırlı kişiliklerin nihai kaderidir.
\vs p005 6:2 Kişilik âlemlerin çözülemeyen gizemlerinden biridir. Biz kişiliğin düzeyleri ve birçok seviyelerinin düzenlemesine katılan etkenlerin yeterli kavramsallaştırmasını oluşturacak yetiye sahibiz, fakat kişiliğin kendisinin gerçek doğasını tamamiyle kavrayamayız. İnsan kişiliğinin yönsel devinimini bir araya getirilince oluşturan birçok etkeni açık bir biçimde algılayabiliriz, fakat böyle sınırlı bir kişiliğin doğasını ve önemini tamamiyle kavrayamayız.
\vs p005 6:3 Kişilik, en düşük düzeyde bulunan bireyin öz benliğinin bilincinden en yüksek Tanrı bilincine kadar değişen bir akıl ihsanına sahip tüm yaratılmışlarda bir potansiyeldir. Buna rağmen akıl ihsanı tek başına ne kişilik, ne ruhaniyet ne de fiziksel enerjidir. Kişilik; birliktelikte ve eş güdüm halinde bulunan ruhaniyet, akıl ve madde enerjilerinin bu yaşayan sistemlerde Yaratıcı olan Tanrı tarafından ayrıcalıklı bir biçimde bahşedilen kâinatsal gerçeklikteki nitelik ve değerdir. Kişilik bu bakımdan ne de ilerleyici bir başarıdır. Kişilik maddi veya ruhani olabilir, fakat kişiliğin varlığı ya var ya da yok olma durumudur. Cennet Yaratıcısı’nın doğrudan faaliyeti dışında kişilik düzeyine kişilikten başka hiçbir şey ulaşamaz.
\vs p005 6:4 Kişiliğin bahşedilişi, Kâinatın Yaratıcısı’nın göreceli yaratıcı bilincinin ve bu sebeple özgür irade düzenlemesinin özellikleriyle ihsan ettiği yaşayan enerji sistemlerinin kişilikleştirilmesi olan onun ayrıcalıklı bir faaliyetidir. Yaratıcı olan Tanrı’nın kişiliğinden ayrı bir biçimde hiçbir kişilik yoktur, ve Yaratıcı olan Tanrı’nın haricinde hiçbir kişilik var olamaz. İnsanın bireyselliğinin temel özellikleri, insan kişiliğinin mutlak Düzenleyici çekirdeği dâhil olmak üzere, Kâinatın Yaratıcısı’nın ayrıcalıklı kişiliğinin nüfuz alanındaki kâinatsal hizmetinde faaliyet göstere onun bahşedişidir.
\vs p005 6:5 Birey öncesi düzeyin Düzenleyiciler’i birden çok çeşitte bulunan fani yaratılmışlarda ikamet eder; bu sebeple, bu aynı varlıkların morontia yaratılmışları olarak nihai ruhaniyet erişimiyle kişilikleştirilmesi için fani ölümden kurtularak varlığını devam ettirebilir. Bunun için, kişilik ihsanının böyle bir yaratılmış aklı kişisel Yaratıcı’nın birey öncesi bahşedişi olan ebedi Tanrı’nın ruhaniyetinin bir nüvesi tarafından ikame edildiğinde, bu sınırlı kişilik kutsal ve ebedi olanaklılığı elinde bulundurur ve bununla birlikte, Nihayet’e benzer olan bir sona ve hatta Mutlaklık’ın bir kendisini gerçekleştirmesine ulaşmayı amaç edinir.
\vs p005 6:6 Kutsal kişilik için yeti birey öncesi Düzenleyici’nin doğasında bulunur; insan kişiliği için yeti insan varlığının kâinatsal\hyp{}akıl ihsanının olanaklılığıdır. Fakat, fani yaratılmışlığın maddi yaşam yönlendirmesi Kâinatın Yaratıcısı’nın özgürleştirici kutsallığı tarafından dokunuluncaya, böylece bir benlik bilinci, benliğin göreceli bir biçimde kendini belirlemesi ve bireysel yaratıcı kişiliği olarak deneyimin enginliğinin onun üzerinde harekete geçirilmesine kadar fani insanın deneyimsel kişiliği faal ve işlevsel bir gerçeklik olarak gözlenemez. Maddi benlik kelimenin tam anlamıyla ve\bibemph{ koşulsuz olarak bireyseldir}.
\vs p005 6:7 Maddi benlik geçici bir kimlik olarak kişiliğe ve kimliğe sahip olup birey öncesi ruhaniyet Düzenleyicisi de ebedi kimlik olarak kimliğe sahiptir. Bu maddi kişilik ve birey\hyp{}öncesi ruhaniyet, ölümsüz ruhun varlığını devam ettiren kimliğini mevcudiyete dönüştüren yaratıcı özelliklerini birleştirmede oldukça yetkindir.
\vs p005 6:8 İnsanın içsel benliğinin öncül nedensellik üzerine dayanan mutlak bağlılığının engellerinden insanın içsel bünyesini özgürleştirdiği ve böylece ölümsüz ruhunun gelişimini sağladığı için Yaratıcı bu hususta kendisini kenara çekmiştir. Tüm bunların sonucunda; nedenselliğin karşılığının engellerinden özgürleşmesiyle birlikte insan en azından ebedi nihai sona uygun olarak ve ölümsüz ruh olan benliğin ilerlemesinin yargısına varılmasıyla, bu ebedi ve varlığını devam ettiren bünyenin yaratılmışlığını kısıtlamak veya onun yaratılmasında irade gösterme konusu tercih için kendisine bırakılmıştır. Hiçbir diğer varlık, güç, yaratıcı, veya kurum uçsuz bucaksız olan kâinat âlemlerinin tümünde, ebedi kutsallığın kişiliğinin tercihte bulunan fanisi olarak ölümlü özgür iradenin mutlak egemenliğine hiçbir derecede müdahale edemez. Ebedi varlığı sürdürmeye ilişkin olarak Tanrı maddi ve fani iradenin egemenliğini buyurmuştur ve bu hüküm kesin bir mutlaklık arz eder.
\vs p005 6:9 Yaratılmışın kişiliğinin bahşedilişi, ilkel nedenselliğe karşı kölece boyun eğmeden göreceli bir bağımsızlaşmayı sağlar, bununla birlikte tüm bu tür ahlaki varlıkların kişilikleri, evrimsel ve diğer geride kalanlar olarak farklılaşmasından bağımsız, Kâinatın Yaratıcısı’nın kişiliğinin merkezindedir. Ebedi Tanrı’nın bütünleştirici döngüsünü ve sayısız evrensel aile çevresini oluşturan varlıksal kan bağı tarafından, onlar ezelden beri onun Cennet mevcudiyetine doğru çekilirler. Tüm kişiliklerde kutsal kendiliğinden gerçekleşmenin bir kan bağı bulunmaktadır.
\vs p005 6:10 Kâinat âlemlerinin tümünün kişilik döngüsü Kâinatın Yaratıcısı’nın kişiliğinde merkezi bir konumdadır, ve Cennet Yaratıcısı bireysel olarak öz benlik mevcudiyetinin tüm düzeylerinin içinde bütün kişiliklerinin bireysel olarak bilincinde ve kişisel olarak onlarla iletişim halindedir. Ve tüm yaratılmışların bu kişilik bilinci Düşünce Düzenleyicileri’nin amacından bağımsız olarak mevcuttur.
\vs p005 6:11 Cennet Adası’nda çevrimleştirilmiş yer çekiminin bütünü, Bütünleştirici Bünye’de ve Ebedi Evlat’ta ruhaniyetin tümünde döngüleştirilen, ve bunun sonucunda Kâinatın Yaratıcısı’nın bireysel varlığında kişiselliğin tamamının çevrilmesi olarak bu döngü hataya yer bırakmayacak biçimde tüm kişiliklerin ibadetini Özgün ve Ebedi Kişiliğe ulaştırır.
\vs p005 6:12 Düzenleyici’nin ikamet etmediği bu kişiliklerle ilgili olarak; onların tercih\hyp{}özgürlüğünün özelliği aynı zamanda Kâinatın Yaratıcısı tarafından bahşedilmiştir, ve bu bireyler benzer biçimde Kâinatın Yaratıcısı’nın kişilik döngüsü olan kutsal sevginin büyük döngüsünde bütünleşir. Tanrı gerçek kişiliklerin tümünün egemen tercih hakkını sağlamıştır. Hiçbir kişisel yaratılmışlık ebedi serüveni yaşamaya zorlanamaz; ebediyetin kapısı sadece özgür iradenin Tanrı’sının özgür iradeye sahip çocuklarının özgür tercihlerine verilen karşılıkta açıktır.
\vs p005 6:13 Ve bu makale yaşayan Tanrı’nın zamanın çocuklarıyla olan ilişkisini sunma çabalarımı yansıtır. Bununla birlikte her şey söylendiğinde ve geriye yapılacak bir şey kalmadığında, Tanrı’nın sizin kâinatınızın Yaratıcı’sı olduğunu ve onun tüm gezegensel çocuklarının sizin bünyenizde taşındığını tekrar tekrar söylemekten daha yararlı bir şey yapamam.
\vs p005 6:14 [Bu makale Uversa’nın bir Kutsal Danışmanı tarafından Kâinatın Yaratıcısı’nın anlatımının sunuşunu konu alan dizinin beşinci ve son kısmıdır.]
