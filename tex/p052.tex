\upaper{52}{Gezegensel Fani Çağları}
\vs p052 0:1 Bir evrimsel gezegen üzerinde yaşamın alınmasından ışık ve yaşam dönemi içinde onun nihai yeşerişine kadar, dünya faaliyet düzeyi üzerinde insan yaşamının en az yedi çağı ortaya çıkar. Bu birbirlerini takip eden çağlar, gezegensel görevler ve kutsal Evlatlar tarafından belirlenmiştir; ve ortalama bir yerleşik dünya üzerinde bu çağlar şu düzen içerisinde ortaya çıkmaktadır:
\vs p052 0:2 1.\bibnobreakspace Gezegensel Prens öncesi İnsan.
\vs p052 0:3 2.\bibnobreakspace Gezegensel Prens sonrası İnsan.
\vs p052 0:4 3.\bibnobreakspace Âdem sonrası İnsan.
\vs p052 0:5 4.\bibnobreakspace Hakimane Evlat sonrası İnsan.
\vs p052 0:6 5.\bibnobreakspace Bahşedilmiş Evlat sonrası İnsan.
\vs p052 0:7 6.\bibnobreakspace Eğitim Evladı sonrası İnsan.
\vs p052 0:8 7.\bibnobreakspace Işık ve Yaşam’ın Dönemi.
\vs p052 0:9 Mekânın bu dünyaları, fiziksel olarak yaşam için elverişli olur olmaz, Yaşam Taşıyıcıları’nın kayıtlarına giriş yapmaktadır; ve bu zaman zarfında bu Evlatlar, bu türden gezegenlere yaşamın başlatılmasını amacıyla gönderilir. Yaşamın başlatılmasından insanın ortaya çıkışına kadar geçen bütüncül dönem; insan öncesi dönem olarak adlandırılmakta olup, bu anlatımlarda ele alınan dizi halindeki fani çağlardan önce gelmektedir.
\usection{1.\bibnobreakspace İlkel İnsan}
\vs p052 1:1 İnsanın hayvan düzeyinden ortaya çıkış zamanından --- Yaratan’a ibadet etmeyi tercih etmeye yetkin oldukları zaman zarfından --- Gezegensel Prens’in varışına kadar, irade sahibi fani yaratılmışlar \bibemph{ilkel insanlar olarak} adlandırılır. Orada, ilkel insanlara ait altı temel tür veya ırk bulunmaktadır; ve bu öncül insanlar sıralı bir biçimde kırmızı ile başlayarak ışık tayfı renklerinin düzeni içinde ortaya çıkmaktadır. Bu öncül yaşam evrimi içinde harcanan zamanın uzunluğu, Urantia zamanına göre yüz elli bin yıldan başlayarak bir milyon yıldan fazlasına kadar uzanan bir süreç içerisinde farklı dünyalar üzerinde oldukça büyük bir biçimde çeşitlilik gösterir.
\vs p052 1:2 Evrimsel renk ırkları --- kırmızı, turuncu, sarı, yeşil, mavi ve çivit olmak üzere --- ilkel insanın basit bir dil geliştirdiği ve yaratıcı hayalini uygulamaya koyulduğu zaman zarfına yaklaşık bir anda ortaya çıkmaya başlar. Bu zaman zarfında insan, iki bacağı üzerinde yürümeye oldukça alışmış bir halde bulunmaktadır.
\vs p052 1:3 İlkel insanlar, güçlü avcılar ve çetin savaşçılardır. Bu çağın yasası, en güçlünün fiziksel olarak varlığını devam ettirmesidir; bu zamanların yönetim düzeni bütünüyle kabileseldir. Birçok dünya üzerinde öncül ırk mücadeleleri boyunca evrimsel ırkların bazıları, Urantia üzerinde gerçekleşmiş olduğu gibi tamamen yok olmuştur. Varlığını sürdüren bu unsurlar genel olarak, Âdemsel insanları olarak ileriki zamanlarda aktarılan eflatun ırkı ile daha sonradan karışmıştır.
\vs p052 1:4 Bir sonraki medeniyetin ışığı altında ilkel insanın bu çağı; uzun, karanlık ve kanlı bir dönemdir. Vahşi hayat etiği ve ilkel ormanların ahlaki değerleri, açığa çıkarılan din ve daha yüksek ruhsal gelişimin sonraki yazgı dönemlerinin ölçütleri ile uyuşmamaktadır. Olağan ve deneyim dışı dünyalar üzerinde bu çağ, Urantia üzerinde bahse konu çağı tanımlayan uzun süreli ve olağandışı acımasız mücadelelerden oldukça farklıdır. İlk dünya deneyimizden ayrıldığınızda, bu uzun ve acı dolu mücadelelerin evrimsel dünyalar üzerinde neden gerçekleştiğini anlamaya başlayacaksınız; ve Cennet doğrultusu üzerinde ilerlediğinizde siz artan bir biçimde, bu apaçık tuhaf olaylara dair bilgeliği anlayacaksınız. Ancak insanın ortaya çıkışının ilk çağlarına ait inişli çıkışlı rastlantıların tümüne rağmen, ilkel insanın dışavurumları; zaman ve mekânın bir evrimsel dünyasının tarihi olayları içerisinde kahramansal olarak bile nitelendirilebilecek görkemli bir dönemdir.
\vs p052 1:5 Öncül evrimsel insan, renkli bir yaratılmış değildir. Genel olarak bu ilkel faniler, mağara sakinleri veya demiz yamaçlarındaki yerleşimcilerdir. Onlar aynı zamanda, büyük ağaçlar içinde basit barakalar inşa etmektedirler. Usun bir yüksek düzeyini onların elde etmelerinden önce gezegenler zaman zaman, hayvanların daha geniş türleriyle istila edilmektedir. Ancak bu çağın ilk zamanları içerisinde faniler ateş yakmayı ve ateşi idare etmeyi öğrenirler; yaratıcı hayal gücünün artışı ve kullanılan aletlerdeki gelişim ile birlikte evrim halindeki insan yakın zaman içerisinde, daha büyük ve daha ağır hayvanlar üzerinde üstünlük kurmuştur. Bu ilk ırklar aynı zamanda, daha büyük uçan hayvanlardan oldukça geniş bir biçimde yararlanmaktadır. Bu devasa kuşlar, beş yüz milin üstündeki aralıksız bir uçuş boyunca bir yâda iki ortalama insanı taşımaya yetkindir. Bazı gezegenler üzerinde bu kuşlar; âlemin dillerine ait birçok kelimeyi konuşmaya sıklıkla yetkin olarak, usun yüksek bir düzeyini ellerinde bulundurması sebebiyle, verilen büyük bir hizmetin parçalarıdır. Bu kuşlar en zeki, en tabi, ve inanılmaz bir biçimde sevgi doludur. Bu türden taşıma kuşların nesli, uzun bir süredir Urantia üzerinde tükenmiş bir konumda bulunmaktadır; ancak sizlerin ilk ataları, onların hizmetlerinden memnuniyetle yararlanmışlardır.
\vs p052 1:6 Ahlaki irade biçiminde, insanın etiksel karar vermeye dair kazanımı genellikle öncül dilin ortaya çıkmasıyla çakışan bir zamana denk gelmektedir. İnsan düzeyine erişim üzerine, fani iradenin bu ortaya çıkışından sonra bu varlıklar; kutsal Düzenleyiciler’in geçici ikamesine karşılık veren bir hale gelmekte olup, ölüm zamanlarında onların birçoğu takip eden yeniden diriliş ve Ruhaniyet birleşmesi için baş melekler tarafından kurtuluş unsurları olarak seçilip onaylanır. Baş melekler; her zaman Gezegensel Prensler’e eşlik etmekte olup, âlemin yazgı dönemi hükmü prensin varışı ile birlikte eş zamanlı olarak gerçekleşmektedir.
\vs p052 1:7 Düşünce Düzenleyicileri tarafından ikamet edilen fanilerin tümü, potansiyel ibadet unsurlarıdır; onlar, “gerçek ışık tarafından aydınlatılmışlardır;” ve onlar kutsallık ile birlikte karşılıklı ilişkiyi aramak için yetkiliği ellerinde bulundurur. Yine de ilkel insanın öncül veya diğer bir değişle biyolojik dini geniş bir biçimde, bilgisizce duyulan endişeden duyulan saygı ve kabilesel hurafelerden beslenen hayvansal korkunun varlığını devam ettiren bir niteliğidir. Hurafenin Urantia ırkları içinde varlığını devam ettirmesi; ne sizin evrimsel gelişmenizi tamamlayan bir nitelikte, ne de maddi ilerleyiş içinde sizin geride kalan muhteşem kazanımlarınız ile uyumludur. Ancak bu öncül korku dini, bahse konu ilkel yaratılmışların vahşi duygularının bastırılmasında oldukça değerli bir amaca hizmet etmektedir. Bu durum, medeniyetin habercisi olmasına ek olarak Gezegensel Prens ve onun hizmetkârları tarafından açığa çıkarılan dinin tohumlarının daha sonradan ekilmesi için gerekli toprağı sağlamaktadır.
\vs p052 1:8 Yüz bin yıl içinde insanın iki ayağı üzerinde yürümesinden itibaren, Gezegensel Prens genel olarak; Yaşam Taşıyıcıları’nın durum raporu üzerine Sistem Egemeni tarafından görevlendirilen bir biçimde, bu dünyaya varış yapmakta olup, her ne kadar göreceli olarak az sayıdaki bireylerin böyle bir biçimde gelişmiş olmasına rağmen faaliyet göstermeye devam edecektir. İlkel faniler genel olarak, Gezegensel Prens ve onun görünebilen yönetim çalışanlarını karşılamaktadır; gerçekte onlar sıklıkla, kısıtlanmamış bir konumda bulunuyorlarsa bu unsurlara neredeyse ibadet eden bir tutum içerisinde korku ve saygıyla bakmaktadır.
\usection{2.\bibnobreakspace Gezegensel Prens sonrası İnsan}
\vs p052 2:1 Gezegensel Prens’in varışı ile birlikte yeni bir yazgı dönemi başlar. Hükümet dünya üzerinde ortaya çıkar, ve gelişmiş kabile çağı erişilir. Büyük toplumsal gelişmeler, bu düzenin birkaç yüzyıl süreci içerisinde gerçekleştirilmektedir. Olağan koşullar altında faniler, bu çağ boyunca medeniyetin yüksek bir düzeyine erişmektedir. Urantia ırklarının gerçekleştirdiklerinin aksine bu insanlar, çok uzun bir süre barbarlık içinde mücadele etmemektedirler. Ancak yerleşik bir dünya üzerinde yaşam, başkaldırı yüzünden öyle bir biçimde değişikliğe uğramıştır ki; siz, olağan bir gezegen üzerinde bu türden bir düzen hakkında herhangi bir fikre sahip olabilecek bir konumda neredeyse hiçbir biçimde bulunamazsınız.
\vs p052 2:2 Bu yazgı döneminin ortalama uzunluğu, bazılarının daha uzun ve daha kısa olan bir biçimde değişiklik göstermesi ile birlikte, yaklaşık beş yüz bin yıldır. Bu dönem boyunca gezegen; sistemin döngüleri içinde sabit bir şekilde konumlandırılmış olup, yüksek meleksel ve göksel yardımcıların tüm çalışanları bu gezegen idaresi için görevlendirilmektedir. Düşünce Düzenleyicileri artan sayılarda gelmekte, ve yüksek meleksel koruyucular fani yüksek denetimine ait kendi düzenlerini büyütmektedir.
\vs p052 2:3 Gezegensel Prens ilkel bir dünyaya vardığı zaman, korkunun ve bilgisizliğin evrimleşmiş dini varlığını devam ettirmektedir. Prens ve onun yönetim çalışanları, daha yüksek doğru ve evren düzeninin ilk açığa çıkarımlarını gerçekleştirir. Açığa çıkarılan dinin bu öncül sunumları oldukça basit bir nitelikte bulunmaktadır; ve bu sunumlar genel olarak yerel sistemin olayları ile ilgilidir. Din, Gezegensel Prens’in varışı öncesinde bütünüyle evrimsel bir süreçtir. Daha sonrasında ise din, evrimsel büyümeye ek olarak kademeli olarak aktarılan açığa çıkarılışlar vasıtası ile gelişmektedir. Her fani çağı olarak her yazgı dönemi, ruhsal gerçeklik ve dinsel etiğin gelişmiş bir temsilini almaktadır. Bir dünyanın sakinleri içinde algılamanın dinsel yetkinliğine dair evrim, onların ruhsal gelişiminin derecesini ve dinsel açığa çıkarılışının kapsamını belirlemektedir.
\vs p052 2:4 Bu yazgı dönemi, ruhsal bir başlangıca şahitlik etmektedir; ve farklı ırklar ve onların çeşitli kabileleri, dinsel ve felsefi düşüncenin özelleşmiş düzenlerini geliştirme eğilimi göstermektedir. Bu ırksal dinlerin tümü boyunca ortak bir biçimde bahse konu iki temel eğilim bulunmaktadır: bu eğilimler, ilkel insanların öncül korkuları ve Gezegensel Prensler’in daha sonraki dinsel açığa çıkarışları. Birtakım yönlerden Urantia unsurlarının, gezegensel evrimin bu düzeyinden bütünüyle kurtulmuş olduğu gözlenmemektedir. Bu çalışmayı amaçlarken siz, daha açık bir biçimde dünyanızın evrimsel ilerleyiş ve gelişiminin ortalama doğrultusundan ne kadar fazla bir biçimde uzaklaştığını kavrayacaksınız.
\vs p052 2:5 Ancak Gezegensel Prensler “Barışın Prensi” değildir. Irksal mücadeleler ve kabilesel savaşlar bu yazgı dönemi boyunca devam etmekte ancak bu devamlılık azalan sıklıkta ve yoğunlukta gerçekleşmektedir. Bu dönem; ırksal çözülümün büyük çağı olup, yoğun milliyetçiliğin bir sürecinin zirve dönemidir. Renk, kabilesel ve milliyetsel toplulukların temelidir; farklı ırklar sıklıkla ayrı diller geliştirmektedir. Fanilerin her büyüyen topluluğu, ayrışmayı arzulayan bir eğilim göstermektedir. Bu ayrışma, birçok dilin mevcudiyeti tarafından nedenselleşmektedir. Birkaç ırkın birleşmesinden önce onların acımasız savaşları zaman zaman, bütün insanların yok oluşuyla sonuçlanmaktadır; turuncu ve yeşil ırk insanları özel olarak bu türden yok oluşa maruz kalmaktadır.
\vs p052 2:6 Ortalama dünyalar üzerinde, prensin idaresinin daha sonraki süreçleri boyunca milli yaşam, kabilesel düzenlenişin yerini almakta veya diğer bir değişle mevcut olan kabile topluluk oluşumları üzerine eklemlenmektedir. Ancak prensin çağının büyük toplumsal kazanımı, aile yaşamının ortaya çıkışıdır. Bu ana kadar, insan ilişkileri başlıca kabilesel bir nitelikte bulunmaktadır; bu andan itibaren ev anlayışı maddileşmeye başlamaktadır.
\vs p052 2:7 Bu süreç, cinsiyet eşitliğinin gerçekleşmesine dair yazgı dönemidir. Bazı gezegenler üzerinde erkek, kadın üzerinde yönetimsel üstünlüğe sahiptir; diğerleri üzerinde ise bu durumun tersi hüküm sürmektedir. Bu çağ boyunca olağan dünyalar; ev yaşamının nihai amaçlarının daha bütünsel gerçekleştirilmesinin başlangıç süreci olarak, cinsiyetlerin bütüncül eşitliğini oluşturmaktadır. Bu süreç, evin altın çağının doğuşudur. Kabile idaresinin fikri kademeli olarak kendisini milli ve aile yaşamın çifte kavramsallaşmasına bırakmaktadır.
\vs p052 2:8 Bu çağ boyunca tarım ilk ortaya çıkışını gerçekleştirmektedir. Aile fikrinin büyümesi, avcılığın gezgin ve istikrarsız yaşamı ile bağdaşmamaktadır. Kademeli olarak, yerleşik yaşam alışkanlıklarının uygulamaları ve toprağın ekimi istikrara kavuşan bir hale gelmektedir. Hayvanların evcilleştirilmesi ve ev sanatlarının gelişimi süratle ilerlemektedir. Biyolojik evrimin zirve noktasına ulaşılması üzerine medeniyetin yüksek bir düzeyi erişilmiş olur; ancak orada mekanik düzenin zayıf bir gelişimi söz konusudur; icat, bir sonraki çağın temel niteliğidir.
\vs p052 2:9 Bu çağ sona ermeden ırklar saflaştırılmakta ve fiziksel kusursuzluğun ve ussal gücün yüksek bir düzeyine getirilmektedir. Olağan bir dünyanın öncül gelişimi büyük bir ölçüde; alt düzey fanilerinin orantılı engellenişi ile birlikte, fanilerin yüksek türlerinin artışını sağlayan tasarım vasıtasıyla desteklenmektedir. Ve ilkel insanların bu türler arasında böylelikle ayrım yapılmasına dair başarısızlığı, mevcut Urantia ırkları arasında bulunan oldukça fazla sayıdaki birçok yetersiz ve bozulan bireylerin mevcudiyetini açıklamaktadır.
\vs p052 2:10 Prensin çağına ait büyük kazanımlardan biri, akılsal olarak kusurlu ve toplumsal bakımdan uyuşmayan bireylerin çoğalımıyla ilgili bu kısıtlamadır. Âdemler olarak ikinci Evlatlar’ın varış zaman zarfından çok daha önce; Urantia insanlarının henüz ciddi bir biçimde ele almadığı bir konu olarak, birçok dünya ciddi bir biçimde kendilerini ırk saflaştırılmasının görevlerine adamıştır.
\vs p052 2:11 Irk gelişiminin bu sorunu, insan evriminin bu öncül zaman zarfında hedef alınan türden kapsamlı bir girişim değildir. Irk kurtuluşu içinde kabile mücadeleleri ve çetin rekabetin önceki dönemi ölçüsüz ve kusurlu ırkların birçoğunun kökünün kurutmaktadır. Budala bir unsur, ilkel ve savaşan konumda bulunan kabilesel toplum düzeni içerisinde kurtuluşun çok fazla şansına sahip değildir. Evrimsel insan unsurlarının ümitsiz bir konumda bulunan kusurlu ırkları, kısmi biçimde kusursuzlaştırılmış medeniyetlerinizin desteklemesi, koruması ve kollaması onların sahip olduğu yanlış bir eğilimdir.
\vs p052 2:12 Kurtarılamaz nitelikteki olağan dışı ve alt düzeyde bulunan faniler olarak, bozulmuş insan varlıkları üzerinde yararsız duygudaşlığı bahşetmek ne şefkat ne de fedakârlıktır. Evrimsel dünyaların en olağan âlemlerinde bile; evrimleşen insanlığın toplumsal olarak uygun olmayan ve ahlaksal bakımdan bozulmuş ırklarını kollamadan, fedakâr duygu ve bencil olmayan fani hizmetin bu soylu niteliklerinin tümünün bütünsel uygulaması için bireyler ve sayısız toplumsal birliktelikler arasında yeterli farklılıklar bulunmaktadır. Ahlaksal mirasını geri dönüşü olmayacak bir biçimde yitirmemiş ve ruhsal doğum\hyp{}hakkını sonsuza kadar sürecek etkiyle zarar vermemiş olan bahse konu talihsiz ve ihtiyaç halinde bulunan bireyler yararına, hoşgörünün uygulanması ve fedakârlığın faaliyetine dair bol miktarda olanak mevcuttur.
\usection{3.\bibnobreakspace Âdem sonrası İnsan}
\vs p052 3:1 Evrimsel yaşamın kökensel etkisi kendisine ait biyolojik doğrultusunda ilerlediği zaman, insan hayvan gelişiminin en yüksek noktasına ulaştığı zaman; orada evlatlığın ikinci düzeyi gelmekte olup, lütuf ve şükranın ikinci yazgı dönemi başlatılır. Bu durum, evrimsel dünyaların tümü üzerinde gerçeklik taşımaktadır. Evrimsel yaşamın olası en yüksek düzeyi erişildiği zaman, ilkel insan biyolojik ölçek içinde en uzak konuma yükseldiği zaman; bir Maddi Erkek ve Kız Evlat, Sistem Egemeni tarafından atanan bir biçimde her zaman gezegen üzerinde ortaya çıkar.
\vs p052 3:2 Düşünce Düzenleyicileri artan bir biçimde Âdem sonrası insanlar üzerine bahşedilmektedir; ve sürekli bir biçimde çoğalan sayıları içerisinde bu faniler, takip eden Düzenleyici bütünleşmesi için yetkinliği elde eder. Alçalan Evlatlar olarak faaliyet gösterirken Âdemler Düzenleyiciler’e sahip değillerdir; ancak onların gezegensel doğumu --- doğrudan veya melez bir biçimde --- bu zaman zarfı içinde, Gizem Görüntüleyicileri’nin alınması için resmi olarak uygun adaylar haline gelmektedir. Âdem sonrası çağın sonlandırılışının zaman zarfında gezegen, göksel hizmetkârlarının olması gereken bütüncül çalışanlarına sahip bir konumdadır; yalnızca bütünleşim Düzenleyicileri evrensel bir biçimde henüz bahşedilmemiştir.
\vs p052 3:3 Medeniyetin avcılık ve toplayıcılık aşamasından, daha sonra medeniyet için yapılacak şehir ve sanayi eklemlemelerin ortaya çıkmasıyla desteklenecek, tarımsal ve bahçıvansal aşamaya kadar olan sürecin tamamlanması için evrim halindeki insanı etkilemek Âdemsel düzenin temel amacıdır. Biyoloji canlandırıcılara ait bu yazgı döneminin on bin yılı, muhteşem bir dönüşümü yerine getirmek için yeterlidir. Gezegensel Prens ve Maddi Evlatlar’ın ortak bilgeliğine ait bu türden bir idarenin yirmi beş bin yılı genellikle, bir Hakimane Evlat’ın varışı için âlemi olgunlaştırmaktadır.
\vs p052 3:4 Bu çağ genel olarak, uygunsuz olan ırkın saf dışı bırakılmasına şahit olmakta ve ırk niteliklerini daha fazla saflaştırmaktadır; olağan dünyalar üzerinde kusurlu olan yabani eğilimler, âlemin çoğalım unsurlarından neredeyse tamamen saf dışı bırakılmıştır.
\vs p052 3:5 Âdemsel soysal çoğalımlar hiçbir zaman, evrimsel ırkların alt düzeyde bulunan kolları ile birlikte karışmamaktadır. Buna ek olarak evrimsel unsurlar ile birlikte Gezegensel Âdem veya Havva’nın kişisel olarak birliktelik kurması kutsal tasarım değildir. Bu ırk\hyp{}gelişim tasarımı, onların soysal çoğalımlarının görevidir. Ancak Maddi Erkek ve Kız Evladı’nın doğumları, ırk\hyp{}karışımlarına ait hizmetin başlamasından önce nesiller boyunca yürütülmektedir.
\vs p052 3:6 Âdemsel yaşam plazmasının fani ırklar için armağanının sonucu, ussal yetkinliğin doğrudan bir yükselişi ve ruhsal ilerleyişin bir hızlanışıdır. Orada genellikle bazı fiziksel ilerlemelerde bulunmaktadır. Ortalama bir dünya üzerinde Âdem sonrası yazgı dönemi büyük icatlar, enerji düzenlemeleri ve mekanik gelişimlere ait bir çağdır. Bu dönem, doğal kuvvetlerin çok türlü üretimi ve denetiminin ortaya çıktığı çağdır; bu dönem, gezegenin keşfinin ve onun üzerinde kurulan nihai denetimin altın çağıdır. Bir dünyanın maddi ilerleyişinin büyük bir kısmı, tıpkı Urantia’nın mevcut an içerisinde deneyimlediği türden bir çağ olarak, fiziksel bilimlerin gelişiminin başlamasına dair bu zaman zarfı boyunca gerçekleşmektedir. Sizin dünyanız; bütüncül bir yazgı dönem âlemi olup, ortalama gezegensel programın çok gerisindedir.
\vs p052 3:7 Âdemsel yazgı döneminin sonunda olağan bir gezegen üzerinde ırklar; “Tanrı, milletlerin tamamını tek bir kandan yarattı” sözünün ve onun Evladı’nın “o insanların tümünü tek bir renkten oluşturdu” ifadesinin gerçek anlamıyla bildirilebilmesi için, neredeyse tamemen birbirlerine karıştırılmaktadır. Bu türden karışan ırkın rengi, âlemlerin “beyaz” ırkı olarak eflatun tonunun bir zeytin karaltısına bir biçimde benzeyen renkte bulunmaktadır.
\vs p052 3:8 İlkel insan, büyük bir çoğunlukla etçildir; Maddi Erkek ve Kız Evlatlar et yememektedirler; ancak onların doğumları, her ne kadar soylarının bütün toplulukları zaman zaman beden dokusu yemeyen biçimde kalmaya devam etseler de, birkaç nesil içerisinde genellikle hem etçil hem de otçulluğa yönelirler. Âdem sonrası ırklarının bu çifte kökeni, bu türden karışan insan ırklarının otobur ve etobur hayvan topluluklara ait olan anatomik kalıntıları nasıl sergilendikleri açıklamaktadır.
\vs p052 3:9 Irksal karışımın on bin yıllık süreci içerisinde ortaya çıkan ırklar; bazı ırk kollarının et yemeyen atalarının izlerini daha çok taşıyan ve diğerlerinin ise evrimsel etçil nesillerinin ayırt edici niteliklerini ve fiziksel karakterlerini daha çok sergileyen bir biçimde, anatomik karışımın çeşitlilik gösteren derecelerini ortaya sermektedir. Bu dünya ırklarının büyük bir çoğunluğu yakın zaman içerisinde, hayvan ve bitki krallığından gelen yiyeceklerin geniş bir kapsamından oluşan bir biçimde, hem etçil hem de otçul hale gelmektedir.
\vs p052 3:10 Âdem sonrası çağ, uluslararası düzenin yazı dönemidir. Irk karışımının tamamlanılmasına yaklaşılması ile birlikte milliyetçilik zayıflar ve insanların kardeşliği gerçek anlamıyla maddileşmeye başlar. Temsili hükümet, idarenin monarşik veya saltanatsal düzeyinin yerini alır. Eğitim sistemi evrensel hale gelir, ve kademeli olarak ırkların dilleri eflatun insanlarının diline yol verir. Evrensel barış ve eş güdüm, ırkların oldukça yerinde bir biçimde karışmasına ve ortak bir dil konuşmasına kadar nadiren erişilir.
\vs p052 3:11 Âdem sonrası çağın kapanış çağları boyunca sanat, müzik, edebiyat gelişmektedir; bu dünya genelinde gerçekleşen uyanış, bir Hakimane Evlat’ın ortaya çıkması için beklenen işarettir. Bu dönemin taçlandırıcı gelişimi, gerçek felsefe biçiminde ussal gerçekliklere olan evrensel ilgidir. Din; daha az milli, gittikçe artan bir biçimde gezegensel olay haline gelmektedir. Gerçekliğin yeni açığa çıkarılışları, insanların olaylarında faaliyet göstermeye başlamaktadır. Gerçek, takımyıldızlarının idaresi düzeyine kadar açığa çıkarılmaktadır.
\vs p052 3:12 Büyük etiksel gelişme bu dönemi nitelemektedir; insanın kardeşliği, toplumunun hedefidir. Dünya çaplı barış --- ırk çatışması ve milli düşmanlığın sonlanması biçiminde --- Hakimane Evlat olarak evlatlığın üçüncü düzeyinin varışı için geleneksel olgunluğun göstergesidir.
\usection{4.\bibnobreakspace Hakimane Evlat sonrası İnsan}
\vs p052 4:1 Olağan ve sadık gezegenler üzerinde bu çağ, fani ırkların karışımı ve onların biyolojik olarak zindeliğiyle başlamaktadır. Orada ne bir yâda renk sorunu bulunmaktadır; kelimenin tam anlamıyla milletlerin ve ırkların tümü tek bir kana ait bir hale gelmektedir. İnsanın kardeşliği yeşermekte, ve milletler dünya üzerinde barış ve huzur içinde yaşamayı öğrenmektedir. Bu türden dünyalar, büyük ve zirve noktasına erişmekte olan ussal gelişimin son sürecinde bulunmaktadır.
\vs p052 4:2 Evrimsel bir dünya böylelikle hakimane çağ için olgun hale geldiğinde Avonal Evlatları’na ait olan yüksek düzeyin bir unsuru, bir hakimane görev üzerinde ortaya çıkışını gerçekleştirmektedir. Gezegensel Prens ve Maddi Evlatlar, yerel evren kökenine aittir; Hakimane Evlatlar Cennet’ten gelmektedir.
\vs p052 4:3 Cennet Avonalları; yazgı dönemi hâkimleri olarak yargı faaliyetleri üzerinde fani âlemlere geldiğinde, onlar hiçbir zaman bir vücuda büründürülmemektedirler. Ancak onlar hakimane görevleri için geldiklerinde, en azından başlangıçsal olan bu görevde; her ne kadar doğumu deneyimlemeseler de ve buna ek olarak âlemin ölümlü niteliği sonucu hayatlarını yitirmeseler de, her zaman vücuda büründürülürler. Onlar, belirli gezegenler üzerinde idareciler olarak kaldıkları bu durumlarda nesiller boyu yaşayabilir. Görevleri tamamlandığında onlar, gezegensel yaşamlarını bırakıp kutsal evlatlığa ait önceki düzeylerine geri dönerler.
\vs p052 4:4 Her yeni yazgı dönemi, açığa çıkarılmış dinin ufkunu genişletir; Hakimane Evlatlar, yerel evren ve onun alt bağımlı âlemlerinin tümünün olaylarını temsil etmek amacıyla gerçeğin açığa çıkarılışını genişletir.
\vs p052 4:5 Bir Malikâne Evlat’ın başlangıçsal ziyaretinden sonra ırklar, yakın bir zaman içinde kendilerine ait mali bağımsızlıklarını yerine getirirler. Bir unsurun bağımsızlığının idaresi için gereken günlük çalışma, zamanınızın iki buçuk saati tarafından temsil edilmektedir. Bu türden etik ve ussal fanileri özgürleştirmek bütünüyle güvenlidir. Bu türden özgürleştirilmiş insanlar, gezegensel ilerleyişin bireysel gelişimi amacıyla boş zamanlarını nasıl kullanacaklarını oldukça iyi bilmektedir. Bu çağ, daha az zinde ve zayıf kazanıma sahip bireyler arasında doğumun kısıtlanması vasıtasıyla ırksal kolların daha ileri bir derecedeki saflaştırılmasına şahit olmaktadır.
\vs p052 4:6 Irkların siyasi hükümeti ve toplumsal idaresi, özerk hükümetin bu çağın sonunda oldukça iyi bir biçimde oluşturulmasıyla gelişmeye devam eder. Özerk yönetim vasıtasıyla biz, temsili hükümetin en yüksek türünden bahsetmekteyiz. Bu türden dünyalar, toplumsal ve siyasi yükümlülükleri taşımada en yetkin önderler ve idarecileri desteklemekte ve onları onurlandırmaktadır.
\vs p052 4:7 Bu çağ boyunca dünya fanilerinin büyük bir çoğunluğu, Düzenleyici’nin ikamet ettiği unsurlardır. Ancak bu zaman zarfında dahi henüz, kutsal Görüntüleyiciler’in bahşedilişi her zaman evrensel değildir. Bütünleşme nihai sonuna ait olan Düzenleyiciler, gezegensel fanilerin tümü üzerinde henüz bahşedilmemiştir; irade sahibi yaratılmışların Gizem Görüntüleyicileri’ni tercih etmeleri hala gereklilik arz etmektedir.
\vs p052 4:8 Bu yazgı döneminin kapanış çağları boyunca toplum, yaşamın daha basitleşmiş türlerine geri dönmeye başlar. Gelişen bir medeniyetin karmaşık doğası kendi doğrultusunda ilerlemektedir; ve faniler daha doğal ve daha etkin bir biçimde yaşamayı öğrenmektedir. Ve bu eğilim bir sonraki her çağ ile birlikte artış göstermektedir. Bu dönem sanat, müzik ve daha yüksek bir konumdaki öğrenmenin yeşerdiği çağdır. Fen bilimleri hali hazırda gelişimlerinin zirve noktasına erişmiş bir konumda bulunmaktadır. İdeal bir dünya üzerinde bu çağın sonlanışı, dünya genelinde ruhsal aydınlanma olarak büyük bir dinsel uyanışın bütünselliğine şahit olmaktadır. Ve ırklara ait ruhsal doğaların bu geniş kapsamlı uyanışı, bahşedilmiş Evlat’ın varışı ve beşinci ahlaki çağın başlatılması için beklenen işarettir.
\vs p052 4:9 Birçok dünya üzerinde, gezegenin tek bir hakimane görev vasıtasıyla bir bahşedilmiş Evlat için hazır hale getirilmediğine dair yapılan çıkarım gerçeklik kazanmaktadır; bu durumda, her biri bir yazgı döneminden diğerine bahşedilmiş Evlat’ın armağanı için gezegen hazır hale gelene kadar ırkları geliştirecek bir biçimde, ikinci hatta Hakimane Evlatlar’ın bir dizisine bile sahne olacak şekilde unsurlar orada ortaya çıkacaktır. İkinci ve daha sonraki görevleri üzerinde Hakimane Evlatlar, vücuda büründürülebilir veya büründürülmez. Ancak ne kadar Hakimane Evlat ortaya çıkabilirse çıksın, --- ve bahşedilmiş Evlat’tan sonra bu biçimlerde de gelebilen bir biçimde --- her birinin varışı bir yazgı döneminin sonunu ve diğerinin başını belirlemektedir.
\vs p052 4:10 Hakimane Evlatlar’ın bu yazgı dönemleri, Urantia zamanının yirmi beş bin yılından başlayarak elli bin yılına kadar uzanabilen bir süreci kapsamaktadır. Zaman zaman bu türden bir çağ daha kısa ve nadir durumlarda ise daha bile uzundur. Ancak zamanın bütünlüğü içerisinde bahse konu bu Hakimane Evlatları’ndan biri, Cennet bahşedilme Evladı olarak doğacaktır.
\usection{5.\bibnobreakspace Bahşedilme sonrası İnsan}
\vs p052 5:1 Ussal ve ruhsal gelişimin belirli bir ortak ölçüsü yerleşik bir dünya üzerinde erişildiğinde her zaman, bir Cennet bahşedilme Evladı’nın bu gezegene varışı gerçekleşmektedir. Olağan dünyalar üzerinde bu unsur, ırklar ussal gelişme ve etiksel kazanımın en yüksek seviyelerine yükselmedikçe ortaya çıkmamaktadır. Ancak Urantia üzerinde bahşedilmiş Evlat, hatta sizin Yaratan Evladı’nız, Âdemsel yazgı döneminin sonunda ortaya çıkmıştır; ancak bu durum, mekânın dünyaları üzerinde olağan bir işleyiş değildir.
\vs p052 5:2 Dünyalar ruhsallaştırma için olgun bir hale geldiği zaman, bahşedilmiş Evlat bu gezegene olan varışı gerçekleşir. Bu Evlatlar her zaman; --- her yerel evrende bir kere olmak üzere, Nebadon Mikâil’i kendisini fani ırklar üzerinde bahşetmek için Urantia üzerinde ortaya çıktığında gerçekleştiği gibi, birtakım evrimsel dünya üzerinde kendi geçici bahşedilişini hazırladığı durum haricinde --- Hakimane veya Avonal düzeyine aittir. Yaklaşık olarak on milyon âlem içerisinde yalnızca tek bir dünya bu türden bir armağanını memnuniyetle deneyimleyebilir; diğer dünyaların tümü ruhsal bir biçimde, Avonal düzeyine ait bir Cennet Evladı’nın bahşedilişi ile geliştirilmektedir.
\vs p052 5:3 Bahşedilmiş Evlat; oldukça yüksek bir biçimde eğitilmiş kültürün bir dünyasına varmakta olup, ileri düzey öğretileri algılamak ve bahşedilme görevini takdir etmek için ruhsal olarak eğitilmiş ve hazırlanmış bir ırk ile karşılaşır. Bu dönem, ahlaksal kültür ve ruhsal gerçekliğin dünya kapsamlı arayışı tarafından nitelenen bir çağdır. Bu yazgı döneminin fani tutkusu, kâinatsal gerçekliğe erişme ve ruhsal gerçeklik ile bütünleşmedir. Gerçekliğin açığa çıkarılışları, aşkın\hyp{}evreni içine alacak bir şekilde genişletilmiştir. Eğitim ve hükümetin yepyeni sistemleri, önceki zamanların ham düzenlerinin yerini alacak düzeye gelen bir biçimde büyümektedir. Yaşam sevinci yeni bir renk almaktadır; ve yaşamın tepkileri, ses ve tınının cennetsel doruklarına yükseltilmiştir.
\vs p052 5:4 Bahşedilmiş Evlat, bir dünyanın fani ırklarının ruhsal canlandırılışı için yaşamakta ve ölmektedir. O “yeni ve yaşayan bir doğrultuyu” oluşturmaktadır; onun yaşamı, beden içinde insanın özgür olması gerektiğine dair bilgideki bahse konu gerçeklik biçiminde --- hatta Gerçekliğin Ruhaniyeti olarak --- Cennet gerçeğinin bir vücut buluşudur.
\vs p052 5:5 Urantia üzerinde bu “yeni ve yaşayan doğrultunun” oluşturulması, gerçeğe ek olarak kesin değerinde olan bir hakikatti. Urantia’nın Lucifer isyanı içindeki tecridi, fanilerin ölümleri üzerine doğrudan bir biçimde malikâne dünyalarının kıyılarına geçecekleri işleyiş biçimini askıya almıştır. Ancak Urantia üzerinde Hazreti Mikâil’in mevcut bulunduğu zaman zarfından önce bütün ruhlar, yazgı dönemi veya özel bin yıllık yeniden dirilişlere kadar uyumuştur. Musa’nın bile; Caligastia olarak mağlup Gezegensel Prens’in bu türden geçişe karşı çıktığı, özel bir yeniden dirilişin ortaya çıktığı olaya kadar diğer tarafa geçmesine izin verilmemiştir. Ancak Hamsin’in gerçekleştiği zaman zarfından bu yanan Urantia fanileri yeniden, doğrudan bir biçimde morontia âlemlerine ilerleyebilmektedir.
\vs p052 5:6 Bir bahşedilme Evladı’nın yeniden dirilişi üzerine, vücut kazandırılmış yaşamının teslim edilişinden sonra üçüncü gün içerisinde, Kâinatın Yaratıcısı’nın sağ koluna yükselir, bahşedilme görevinin kabulüne dair teminatı alır ve yerel evren yönetim merkezinde Yaratan Evlat’a geri döner. Bunun üzerine bahşedilmiş Avonal ve Yaratan Mikâil, Gerçekliğin Ruhaniyeti olarak kendilerine ait ortak ruhaniyeti bahşedilen dünyaya gönderir. Bahse konu bu durum, “muzaffer Evlat’ın ruhaniyetinin bedenin tümü içine aktarıldığı” zamanda gerçekleşen olaydır. Evren Ana Ruhaniyeti aynı zamanda Gerçekliğin Ruhaniyeti’ne ait bu bahşedilmeye katılmaktadır; ve bu olayla eş zamanlı olarak gerçekleşen bir biçimde, Düşünce Düzenleyicileri’nin bahşedilme buyruğu yürürlüğe girer. Bunun sonrasında bu dünyanın olağan akla sahip irade sahibi yaratılmışları, ruhsal tercihe ait olan bir biçimde ahlaksal sorumluluğun yaşına erişir erişmez Düzenleyicileri almaktadırlar.
\vs p052 5:7 Eğer bu türden bir bahşediliş Avonal, bahşedilme görevi sonrasında bir dünyaya dönmek zorunda olursa; bu unsur vücuda bürünmeyip “yüksek meleksel ev sahipleri ile birlikte ihtişam içinde” gelecektir.
\vs p052 5:8 Bahşedilme sonrası Evlat çağı, on bin ila yüz bin yıl arasında değişen süreçlerde uzayabilir. Bu yazgı dönemlerinin herhangi biri için keyfi bir biçimde ayrılmış zaman zarfı bulunmamaktadır. Bu dönem, büyük etik ve ruhsal ilerleyişin zamanıdır. Bu çağların ruhsal etkisi altında insan karakteri, devasa dönüşümler süreçlerinden geçmekte ve olağanüstü gelişimleri deneyimlemektedir. Burada temel ahlaki ilkeyi işlevsel bir biçimde uygulamaya koymak mümkün hale gelmektedir. İsa’nın öğretileri gerçek anlamda, karakterin yüceleşmesi ve kültürün yükseltilmesine ait yazgı dönemleri ile birlikte bahşedilme öncesi Evlatlar’ın öncül hazırlığına sahip bir fani için uygulanabilir hale gelmektedir.
\vs p052 5:9 Bu dönem boyunca hastalık ve ihmalin sorunları neredeyse tamamen çözümlenmiştir. Yozlaşma, seçici çoğalım vasıtasıyla büyük oranda saf dışı bırakılmıştır. Âdemsel ırk kollarının yüksek düzeyde bulunan dirençli yetkinlikleri vasıtasıyla ve önceki çağların tıbbi bilimlerinin gerçekleştirmiş olduğu keşiflerin dünya çaplı uygulanışı yardımıyla hastalık üzerinde neredeyse bütüncül bir hâkimiyet kurulmuştur. Bu dönem boyunca ortalama yaşam uzunluğu, Urantia zamanının yüz yılının çok daha üzerine tırmanmaktadır.
\vs p052 5:10 Bu çağ boyunca hükümetsel yüksek denetimin kademeli olarak azalışı mevcut bulunmaktadır. Gerçek özerk idare faaliyet göstermeye başlamaktadır; bu dönemde gittikçe azalan bir biçimde kısıtlayıcı yasalar gerekmektedir. Milliyetsel savunmanın askeri birimleri yol olmaktadır; uluslararası uyumun çağı gerçek anlamıyla gelmektedir. Orada, büyük bir çoğunlukla karasal dağıtım vasıtasıyla belirlenen milletler bulunmaktadır; ancak orada yalnızca tek bir ırk, tek bir dil ve tek bir din bulunmaktadır. Fani olayları neredeyse, ancak tamamiyle gerçekleşen bir şekilde bulunmamakta olup, ülküsel bir düzendir. Bu dönem, harika ve muhteşem bir çağdır!
\usection{6.\bibnobreakspace Urantia’nın Bahşedilme sonrası Çağı}
\vs p052 6:1 Bahşedilme, Barışın Prensi’dir. Kendisi, “Dünya üzerinde barış, insanlar arasında iyi niyet” biçimindeki bildirimiyle beraber gelmektedir. Olağan dünyalar üzerinde bu süreç, dünya kapsamlı barışın bir yazgı dönemidir. Ancak bu türden yararlı etkiler, Hazreti Mikâil olan bahşedilmiş Evlat’ınızın gelişine eşlik etmemiştir. Urantia, olağan düzende ilerlememektedir. Dünyanız, gezegensel ilerleyiş içinde doğrultu dışında seyir etmektedir. Sizin Hâkim’iniz; dünya üzerinde bulunduğu zaman, kendisinin dünyaya varışının Urantia üzerinde olağan barış düzenini getirmeyeceği yönünde takipçilerini uyarmıştır. Kendisi açık bir biçimde, orada “savaşlar ve savaş söylentilerinin” bulunacağını ve milletlerin diğerlerine karşı geleceğini ifade etmiştir. Diğer bir zaman zarfında ise kendisi, “Dünyaya barış getirmek için geldiğimi düşünmeyin” demiştir.
\vs p052 6:2 Olağan evrimsel dünyalar üzerinde insana ait dünya çapındaki kardeşliğin gerçekleştirilmesi kolay bir kazanım değildir. Urantia gibi karışmış ve düzeni bozulmuş bir gezegenin bu türden bir kazanımı çok daha uzun bir zaman almakta ve çok daha büyük çabayı gerektirmektedir. Yardım görmemiş toplumsal evrim, ruhsal olarak tecrit edilmiş bir âlem üzerinde bu türden mutlu sonuçları neredeyse hiçbir biçimde elde edemez. Dinsel açığa çıkarılış, Urantia üzerinde kardeşliğin gerçekleştirilmesi için hayati derece öneme sahiptir. İsa, ruhsal kardeşliğin doğrudan erişimi için yol gösterirken; toplumsal kardeşliğin dünyanız üzerinde gerçekleştirilmesi, fazlasıyla şu kişisel dönüşümlere ve gezegensel düzenlemelerin kazanımlarına bağlıdır:
\vs p052 6:3 1.\bibnobreakspace \bibemph{Toplumsal kardeşlik}. Uluslar ve ırklar arası toplumsal ilişkilerin ve kardeşsel birlikteliklerin seyahat, ticaret ve rekabetçi oyunlar vasıtasıyla çoğalımı. Ortak bir dilin gelişmesi ve çok dilliliğin çoğalımı. Öğrenciler, öğretmenler, sanayiciler ve dinsel filozoflar arasında ırksal ve milli paylaşım.
\vs p052 6:4 2.\bibnobreakspace \bibemph{Ussal karşı\hyp{}çiftleşme}. Kardeşlik, sonu gelmeyen bencilliğin budalalığını tanımada başarısız olan derecede sakinlerinin oldukça ilkel olduğu bir dünya üzerinde imkânsızdır. Orada, milli ve ırksal edebiyatın bir değişimi gerçekleşmek zorundadır. Her ırk, ırkların tümünün düşüncesi ile aşina olmak durumundadır; her millet, milletlerin tümünün duygularını bilmek zorundadır. Cehalet şüpheyi beslemektedir, ve şüphe duygudaşlık ve derin sevginin temel tutumuyla uyuşmamaktadır.
\vs p052 6:5 3.\bibnobreakspace \bibemph{Etiksel uyanış}. Yalnızca etiksel bilinç, insan hoşgörüsüzlüğünün ahlak dışılığına ve kardeş kavgasının günahkârlığına dair gerçeği açığa çıkarabilir. Yalnızca bir ahlaksal vicdan, milli kıskançlık ve ırksal hasetin kötülüklerini kınayabilir. Yalnızca fani varlıklar, temel ahlaki ilke uyarınca yaşamak için hayati derecede önemli olan ruhsal kavrayışı sürekli arayacaklardır.
\vs p052 6:6 4.\bibnobreakspace \bibemph{Ahlaksal bilgelik}. Duygusal olgunluk bireysel denetim için temel niteliğindedir. Yalnızca duygusal olgunluk, savaşa dair barbarca verilen kararın medenileşen hükmün uluslararası işleyiş biçimleriyle olan değişimini teminat altına alacaktır. Bilge devlet adamları zaman zaman, kendilerine ait milli veya ırksal toplulukların çıkarlarını desteklemeyi amaçlarken bile insanlığın refahı için çalışacaktır. Bencil toplumsal bilgelik, gezegensel topluluk kurtuluşunu teminat altına alan bahse konu bu dayanıklı niteliklere zarar veren bir biçimde, nihai olarak kendini yok eden bir niteliktedir.
\vs p052 6:7 5.\bibnobreakspace \bibemph{Ruhsal kavrayış}. İnsanın kardeşliği, nihai olarak, Tanrı’nın yaratıcılığının tanınmasına dayanmaktadır. Urantia üzerinde insanın kardeşliğinin tanınması için en kolay yol, mevcut\hyp{}zaman insanlığının ruhsal dönüşümlerini yerine getirmektir. Toplumsal evrimin doğal gidişatını hızlandırmak için tek işleyiş biçimi; her faninin diğer her bir faniyi anlaması ve onun için derin seviyi beslemesi için ruh yetkinliğini geliştiren bir biçimde, yukarıdan ruhsal baskıyı uygulama ve böylece ahlaksal kavrayışı arttırmaktır. Karşılıklı anlayış ve kardeşsel sevgi, insanın kardeşliğinin dünya çapındaki tamamlanışı içinde aşkın medeniyet değerleri ve kudretli etkenlerdir.
\vs p052 6:8 Eğer siz; sahip olduğunuz mevcut geri kalmış ve kafa karışıklığını yaşayan dünyanızdan bahşedilme çağı içinde herhangi bir olağan gezegene aktarılacak olsaydınız; geleneklerinizin cennetine gönderilmiş olduğunuzu düşünecek olurdunuz. Siz bu konumda, insan yerleşkesinin fani bir âleminin olağan evrimsel çalışmalarını gözlemlemekte olduğunuza neredeyse hiçbir biçimde inanmazdınız. Bu dünyalar; kendilerine ait olan âlemin ruhsal döngüleri içinde bulunmakta olup, onların tümü evren yayınlarının faydalarını ve aşkın\hyp{}evrenin yansıtıcı hizmetlerini memnuniyetle deneyimlemektedir.
\usection{7.\bibnobreakspace Eğitmen Evlat sonrası İnsan}
\vs p052 7:1 Olağan evrimsel dünya üzerine gelecek diğer düzeyin Evlatları, Cennet Kutsal Üçlemesi’nin kutsal Evlatları olan Kutsal Üçleme Eğitmen Evlatları’dır. Yine bu hususta da Urantia’yı, İsa’nın dönmek için söz vermiş olduğu kardeş âlemlerine kıyasla düzen dışına çıkmış bir halde bulmaktayız. İsa’nın kesin bir biçimde yerine getireceği bu söz hakkında hiç kimse, Urantia üzerinde onun ikinci gelişinin Hakimane veya Eğitmen Evlatlar’ın ortaya çıkışlarından önce mi veya sonra mı gerçekleşeceğini bilmemektedir.
\vs p052 7:2 Eğitmen Evlatlar, ruhsallaştırılan dünyalara topluluklar halinde gelmektedir. Bir gezegensel Eğitmen Evlat; yetmiş birinci derece Evlat, yirmi ikinci derece evlat ve Daynallar’ın yüce düzeyine ait en yüksek ve en deneyimli üç unsurundan oluşan bir biçimde yardım görür ve desteklenir. Bu birlik, evrimsel çağlardan ışık ve yaşamın dönemine uzanan geçişi yerine getirmeye yetecek kadar --- gezegensel zamana göre bin yıldan daha az olmayacak şekilde ve sıklıkla bu süreden çok daha fazla zamanı alan bir biçimde --- dünya üzerinde bir süreliğine kadar kalmaya devam edecektir. Bu görev, yerleşik bir dünya üzerinde hizmet vermiş kutsal kişiliklerin tümüne ait önceki çabalara yapılan bir Kutsal Üçleme katkısıdır.
\vs p052 7:3 Gerçeğin açığa çıkarılışı bu dönem içerisinde, merkezi evren ve Cennet’e kadar genişletilmiş bir konumda bulunmaktadır. Irklar, yüksek bir biçimde ruhsal hale gelmektedir. Büyük bir insan kitlesi evrimleşmiş olup, büyük bir insan çağı yaklaşmaktadır. Gezegenlerin eğitimsel, mali ve idari sistemleri köklü dönüşümlere uğramaktadır. Yeni değerler ve ilişkiler oluşturulmaktadır. Cennetin krallığı yeryüzü üzerinde ortaya çıkmakta, ve Tanrı’nın ihtişamı dünya içinde tüm gücüyle parıldamaktadır.
\vs p052 7:4 Bu süreç, yaşayanlar arasından birçok faninin aktarıldıkları yazgı dönemidir. Kutsal Eğitmen Evlatları’nın dönemi gelişme gösterirken, zamanın fanilerinin ruhsal bağlılığı gittikçe artan bir biçimde evrensel hale gelir. Doğal ölüm, Düzenleyiciler’in artan bir biçimde beden içindeki iyeleri ile birlikte bütünleşmesiyle daha az sıklıkla rastlanan bir hale gelir. Gezegen nihai olarak, fani yükselişin ikinci düzeyde değişikliğe uğratılmış unsurları olarak sınıflandırılır.
\vs p052 7:5 Bu dönem boyunca yaşam güzel ve yararlıdır. Uzun evrimsel mücadelenin bozulan ve toplum dışı sonuçları neredeyse tamamen ortadan kaldırılmıştır. Yaşam uzunluğu, Urantia zamanına göre beş yüz yıllık bir süreye yaklaşmaktadır; ırksal artışın doğum oranı ussal bir biçimde düzenlenmektedir. Toplumun bütünüyle yeni bir düzeyi gelmiş bulunmaktadır. Orada faniler arasında hali hazırda büyük farklılıklar bulunmaktadır; ancak toplumun düzeyi daha yakın bir biçimde toplumsal kardeşlik ve ruhsal eşitliğin nihai amaçlarına yaklaşmaktadır. Temsili hükümet ortadan kalkmakta, ve dünya bireysel öz denetimin idaresi altında yürütülmektedir. Hükümetin görevi başlıca toplumsal idare ve mali eş güdümün ortak görevlerine yönlendirilmiştir. Altın çağ süratle gelmekte olup, uzun ve yoğun gezegensel nitelikteki evrimsel mücadelenin geçici amacı görünür bir hüviyet kazanmıştır. Çağların ödülü yakında verilecek olup, Tanrı’nın bilgeliği kısa bir süre içinde açığa çıkarılacaktır.
\vs p052 7:6 Bu çağ boyunca bir dünyanın fiziksel idaresi, her erişkin birey için yaklaşık bir saat sürmektedir; bu süre, Urantia’nın bir saatine denktir. Gezegen, evren olayları ile yakın ilişki içindedir; ve gezegen insanları, günlük gazetelerinizin en son baskılarında şimdilerde sergilemiş olduğunuz aynı hevesli merakla en son yayınları gözden geçirmektedir. Bu ırklar, sizin dünyanız için bilinmeyen bin merakı yoğun bir biçimde duymaktadır.
\vs p052 7:7 Artan bir biçimde Yüce Varlık’a olan gerçek gezegensel bağlılık artmaktadır. Nesilden nesile, adaleti uygulayan ve bağışlamayı deneyimleyen unsurların konumuna ırkların daha fazlası yükselmektedir. Yavaş ancak kesin bir biçimde dünya, Tanrı’nın evlatlarının neşeli hizmeti için kazanılmış bir konuma gelmektedir. Fiziksel zorluklar ve maddi sorunlar büyük bir ölçüde çözülmüştür; gezegen, ileri yaşam ve daha fazla istikrara kavuşturulmuş mevcudiyet için olgunlaşmaktadır.
\vs p052 7:8 Zaman zaman yazgı dönemleri boyunca Eğitmen Evlatlar, bu barışçı dünyalara gelmeye devam ederler. Onlar; ilgili gezegen ile ilgili evrimsel tasarımın pürüzsüz bir biçimde çalıştığını gözlemlemeden bir dünyadan ayrılmamaktadırlar. Bu türden Evlatlar’ın bazıları, Eğitmen Evlatları’nın ayrılış zamanında faaliyet gösterirken ve bu yargısal eylemler çağdan çağa zaman ve mekânın fani düzeni boyunca devam ederken, Adaletin bir Hakimane Evladı genellikle takip eden görevleri içinde Eğitmen Evlatlar’a eşlik etmektedir.
\vs p052 7:9 Kutsal Eğitmen Evlatları’nın tekrarlayan her görevi bu türden tanrısal bir dünyayı bilgeliğin, ruhsallığın ve kâinatsal aydınlığın sürekli artan zirve noktalarına yükseltir. Ancak bu türden bir âlemin soylu yerlileri, sınırlı ve fanidir. Hiçbir şey kusursuz değildir; yine de orada, kusursuz bir dünyanın düzensel işleyişi ve ona ait insan sakinlerinin yaşamları bakımından kusursuzluğa yakın düzeyde evrimsel bir nitelik bulunmaktadır.
\vs p052 7:10 Kutsal Üçleme Eğitmen Evlatları, aynı dünyaya birçok kez geri dönebilir. Ancak en sonunda görevlerinden birinin sonlanmasıyla ilişkili olarak Gezegensel Prens, Gezegensel Egemen’in düzeyine yükseltilir; ve Sistem Egemeni, bu türden bir dünyanın ışık ve yaşamın dönemine girdiğini duyurmak için ortaya çıkar.
\vs p052 7:11 Eğitmen Evlatları’nın dönem sonu görevinin tamamlanışı (en azından olağan bir dünyanın tarihsel bir sırası teşkil edecek biçimde), Yahya’nın “Ben yeni bir cennete ek olarak yeni bir dünya gördüm, ve yeni Kudüs prens için bezenen bir prenses gibi hazırlanmış olarak cennetten kökenini alan biçimde Tanrı’dan inmektedir” biçimindeki ifadesiyle gerçekleşmiştir.
\vs p052 7:12 Bu âlem; eski kâhinin şu sözleri kaleme aldığında geleceği gördüğü gibi “Tanrı: ‘çünkü, benim yapacağım yeni cennetler ve yeni dünya benim önümde durmaya devam etmeli ve böylece siz ve sizin çocuklarınız varlıklarını devam ettirmelidir; yeni bir aydan diğerine ve bir Şabat’tan diğer bir Şabat’a bedenlerin tümünün benim önümde ibadet etmek için gelmeleri gerçekleşmelidir’ demiştir,” gelişmiş bir gezegensel aşama olarak aynı dünyanın yenileştirilmiş halidir.
\vs p052 7:13 “Yeni bir nesil, sadık bir ruhbanlık, kutsal bir millet, yüceltilmiş bir insanlar topluluğu” olarak tanımlanan “ve siz, karanlıktan muhteşem aydınlığa sizi çağıran Onun övgülerini anlatacaksınız” ifadesiyle tasvir edildiğiniz bu türden bir çağa ait fanilersiniz.
\vs p052 7:14 Bir bireysel gezegenin özel doğa tarihi ne olursa olsun, bir âlemin bütünüyle sadık, kötülükler ile lekelenmiş veya günah ile lanetlenmiş oluşundan bağımsız olarak --- ataları kim olursa olsun --- nihai olarak Tanrı’nın lütfu ve meleklerin hizmeti, Kutsal Üçleme Eğitmen Evlatları’nın varışı zamanında her şeyi yeniden başlatan süreci sağlayacaktır; ve son görevlerini takiben onların ayrılışı, ışık ve yaşamın bu muhteşem dönemini başlatacaktır.
\vs p052 7:15 Satania’nın dünyalarının tümü, bir kişinin “Yine de biz, Onun sözü uyarınca, doğruluğun içinde ikamet ettiği yeni bir cennet ve yeni bir dünyayı aramaktanız. Bu nedenle siz sevilenler, bu şeyleri arayış halinde göründüğünüz için Onun tarafından barış içinde lekeden ve kusurdan uzak bir biçimde bulunabilmeniz için dirayetli olun” biçimindeki ümide katılabilir.
\vs p052 7:16 İlk veya birkaç takip eden idarenin sonunda Eğitim Evlat birliklerinin ayrılışı, zamandan ebediyetin girişine olan geçişin eşiği biçiminde ışık ve yaşamın çağının doğuşunu başlatmaktadır. Işık ve yaşamın bu döneminin gezegensel gerçekleşmesi; cenneti doğrudan nihai son ve kurtuluşa erişen fanilerin nihai ikamet yerleşkesi olarak temsil eden dinsel inançlar içinde bütünleşen gelecek yaşam kavramlarından daha fazlasını düşünmemiş olan, Urantia fanilerinin en arzu duydukları beklentilerinin eşiti olmaktan çok daha fazlasıdır.
\vs p052 7:17 [Bu anlatım, Cebrail’in yönetim çalışanlarına geçici olarak bağlanmış bir Kudretli İletici tarafından sağlanmıştır.]
