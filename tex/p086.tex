\upaper{86}{Dinin Öncül Evrimi}
\vs p086 0:1 Bir önceki, ilkel tapınma dürtüsünden gelen dinin evrimi açığa çıkarışa bağlı değildir. Evrensel ruhaniyet bahşedilişine ait altıncı ve yedinci akıl\hyp{}ruhaniyetlerinin doğrudan etkisi altında insan aklının olağan faaliyeti, bu türden gelişmeyi sağlamak için tamamiyle yeterlidir.
\vs p086 0:2 Doğa güçlerine karşı insanın en öncül, din\hyp{}öncesi korkusu; insan bilinci içinde doğanın kişiselleşmesi, ruhanileşmesi ve nihai olarak ilahlaşmasıyla birlikte kademeli olarak dini hale gelmişti. İlkel bir türün dini böylelikle, bu türden akılların bir kere doğa\hyp{}üstü kavramları düşünmesinden sonra evrimleşen hayvan akıllarının psikolojik hareketsizliğinden doğan doğal bir biyolojik sonuçtu.
\usection{1.\bibnobreakspace Şans: İyi Talih and Kötü Talih}
\vs p086 1:1 Doğal ibadet dürtüsü dışında öncül evrimsel din --- olağan oluşumlar halindeki talih olarak adlandırdığınız --- insanların şans deneyimlerinden kökenini almaktaydı. İlkel insan bir yiyecek avcısıydı. Avlanma sonuçları her gün sürekli olarak değişiklik göstermek zorundadır; ve bu durum, \bibemph{iyi talih} ve \bibemph{kötü talih} olarak insanın yorumladığı deneyimlere belirli bir biçimde kaynaklık etmektedir. Şansızlık, tekin olmayan ve tehditkâr mevcudiyetin dar eşiğinde sürekli bir biçimde yaşayan erkek ve kadınların hayatlarında büyük bir etkendi.
\vs p086 1:2 İlkel insanın kısıtlı ussal ufku şans üzerine o kadar odaklanmaktadır ki, talih onun yaşamında devamlı bir etken haline gelmektedir. İlkel Urantia unsurları yaşama mücadelesi için çabalamışlardı, belirli bir ortak yaşam koşulları için değil. Bilinmeyen ve daha önce görülmeyen felaketten duyulan devamlı korku, her hazlarını etkin bir biçimde gölgede bırakan bir çaresizlik bulutu olarak bu ilkel insanlar üzerinde asılı kaldı. Hurafelere inanan ilkel insanlar sürekli iyi giden bir talihten korkmuşlardı; onlar bu türden iyi şansı yaklaşmakta olan belirli bir felaket habercisi olarak gördüler.
\vs p086 1:3 Kötü talihe dair duyulan bu sürekli etkin korku felç ediciydi. Hiçbir şey yapmadan bir şey elde eden bir biçimde, birisi tesadüfen iyi bir talihle karşılaşırken neden, bir şeyler yapıp hiçbir şey elde edemeyen bir biçimde, çok çalışıp kötü talihin sonuçları elde edilmeliydi ki? Düşünmeyen insanlar --- umursamazlıktan gelen bir biçimde --- iyi talihi unuturlar, fakat onlar acı verici bir şekilde kötü şanslarını hatırlarlar.
\vs p086 1:4 Öncül insan belirsizlik ve devamlı --- kötü talih biçiminde --- şans korkusu içinde yaşadı. Yaşam heyecan verici bir şan oyunuydu; mevcudiyet bir kumardı. Kısmi bir biçimde medenileşmiş insan toplulukların hala şansa inanıp, kumara olan hali hazırda devamlılığını sürdüren eğilimleri göstermeleri şaşılacak bir durum değildir. İlkel insan iki güçlü beklenti arasında gidip gelmiştir: bunlar, hiçbir şey yapmadan bir şey elde etmeye dair tutku ve bir şey yapıp hiçbir şey elde edememeye dair korkudur. Ve bu mevcudiyet kumarı, öncül ilkel insan aklının ana beklentisi ve en yüksek büyüleyicisiydi.
\vs p086 1:5 Daha sonraki sürü sahipleri, şans ve talihin aynı görüşlerine sahip olmuşlarsa da; daha da sonraki tarımla uğraşanlar insanlar artan bir biçimde, neredeyse hiçbir biçimde denetleyemediği birçok şey tarafından ekinlerin doğrudan bir biçimde etkilendiğinin bilincine vardılar. Çiftçi kendisini; kıtlığa, sele, doluya, haşeratlara, bitki hastalıklarına ek olarak sıcak ve soğuğun kurbanı olan bir konumda buldu. Ve bu doğal etkilerin tümü bireyin refahını etkilerken, onlar iyi talih ve kötü talih olarak görülmüştü.
\vs p086 1:6 Şans ve talihe dair bu görüş, ilkçağ insan topluluklarının tümünün felsefesine güçlü bir biçimde hâkim oldu. Yakın dönemlerde bile Süleyman’ın Bilgeliğinde şu söylenmiştir: “Döndüm ve baktım ki ne ırk tez canlılara, ne savaş güçlülere, ne ekmek bilgelere, ne zenginlik anlayanlara ve ne de lütuf mahirlere aittir; ancak kader ve şans hepsine karşı gelmektedir. Çünkü insan kaderini bilmemektedir; tıpkı balıkların kötü niyetli bir ağa takılmaları ve kuşların bir tuzakla yakalanmaları gibi, insan evlatları kötü bir anda başlarına aniden inen tuzakla avlanmaktadırlar.”
\usection{2.\bibnobreakspace Şansın Kişileştirilmesi}
\vs p086 2:1 Endişe, ilkel çağ aklının doğal bir durumuydu. Erkek ve kadınlar aşırı derecede endişeye kapıldıklarında, yalın bir değişle, çok uzak soylarının doğal düzeylerine geri dönmektedirler; ve endişe gerçek anlamda acı verici hale geldiği zaman, etkinliği kısıtlayıp evrimsel değişikliklerin ve biyolojik uyumların temelini atmaktadır. Acı ve ızdırap, ilerleyici evrim için hayati derecede önemlidir.
\vs p086 2:2 Yaşam mücadelesi o kadar acı vericidir ki, belirli geri kalmış kabileler hala bile her yeni gün doğumunda ulumakta ve inlemektedir. İlkel insan kendisine sürekli Kim bana işkence ediyor? sorusunu sormuştu. Dertleri için maddi bir kaynağı bulamayan bir biçimde, ruhani bir açıklamada karar kılmıştı. Ve böylece din; gizemli olana karşı duyulan korkudan, görülmemiş şeyin gerçekleşeceğine dair hissedilen dehşetten, ve bilinmeyene dair beslenen derin endişeden doğmuştu. Doğa korkusu böylelikle, ilk olarak şans daha sonra ise gizem nedeniyle yaşam mücadelesinde bir etken haline gelmişti.
\vs p086 2:3 İlkel insan mantıklıydı, fakat ussal ilişki için çok az düşünceye sahipti; ilkel akıl, tamamiyle basit bir biçimde, eğitilmemişti. Eğer bir olay diğerini takip ettiyse, ilkel insan onu sebep sonuç ilişkisi içinde değerlendirmişti. Medenileşmiş insanın hurafe olarak değerlendirdiği şey yalnızca, ilkel insanda mevcut bulunmuş düz bir cahillikti. İnsan türü her zaman, amaçlar ve sonuçlar arasında herhangi bir ilişkisinin zorunlu bir biçimde bulunmasının gerekmediğini anlamada yavaş kalmıştır. İnsan varlıkları daha yeni yeni, mevcudiyetin tepkilerinin eylemler ile onların sonuçları arasında rol aldığını anlamaya başlamaktadır. İlkel insan maddi olmayan ve soyut olan her şeyi kişiselleştirmeyi amaçlamaktadır, ve böylece doğa ve şans --- ruhaniyetler olarak --- hayaletler ve daha sonra tanrılar biçiminde kişiselleşen hale gelmektedir.
\vs p086 2:4 İnsan doğal bir biçimde; en yakın veya en uzak çıkarını bilen bir biçimde kendisi için neyin en iyi olduğuna inanma eğilimi gösterir; birey çıkarı geniş ölçüde mantığı gölgelemektedir. İlkel insanların ve medenileşmiş insanların akılları arasındaki fark, nitelik yerine derece biçiminde doğasal olanın aksine daha çok içerikseldir.
\vs p086 2:5 Ancak kavranılması zor olan şeyleri doğaüstü sebeplere atfetmeye devam etmek, ussal sıkı çalışmanın tüm türlerinden tembel ve kolay bir biçimde kaçınmadan başka bir şey değildir. Talih yalnızca, insan mevcudiyetinin herhangi bir dönemindeki açıklanamaz olanın üstünü örtmek için yaratılmış bir kavramdır. Şans, insanın sebepleri belirlemek için haddinden fazla umursamaz ve üşengeç oluşunu simgeleyen bir kelimedir. İnsanlar; ırkların bireysel girişimcilik ve maceradan yoksun oldukları bir durum olarak sadece merak ve hayal gücünden mahrum oldukları anda, doğal bir oluşumu bir kaza veya kötü talih biçiminde değerlendirmektedirler. Yaşam olgularının keşfi er yâda geç; sahip olduklarını, içinde tüm etkilerin belirli nedenleri takip ettiği bir evren kanun ve düzeniyle değiştirerek şansa, talihe veya kazalar olarak adlandırdıkları oluşumlara olan inancını yok edecektir. Böylelikle mevcudiyet korkusunun yerini yaşam sevinci alacaktır.
\vs p086 2:6 İlkel insan, bir şey tarafından sahip olunan bir biçimde tüm doğaya canlı gözüyle bakmıştı. Medenileşmiş insan hala, önüne çıkan ve kendisine çarpan bu cansız nesneleri tekmelemekte ve ona lanet okumaktadır. İlkel insan herhangi bir şeyi hiçbir zaman kaza eseri olarak görmemişti; her şey her zaman bir amaç uğruna gerçekleşmişti. İlkel insan için, ruhani dünya biçimindeki talihin faaliyeti olarak kaderin nüfuz alanı tıpkı ilkel toplum gibi düzenlenmemiş ve gelişi güzel bir durumdaydı. Talihe, ruhani dünyanın değişken ve nasıl davranacağı belli olmayan tepkisi gözüyle bakılmaktaydı; daha sonra talih, tanrıların mizahı olarak görülmüştü.
\vs p086 2:7 Ancak tüm dinler canlısallıktan gelişme göstermemişti. Doğaüstülüğün diğer kavramları canlısallığın çağdaşlarıydı; ve bu inanışlar aynı zamanda tapınmaya sebebiyet vermişti. Doğallık bir din değildir --- dinin bir doğumudur.
\usection{3.\bibnobreakspace Ölüm --- Açıklanamaz}
\vs p086 3:1 Ölüm evrimleşen insan için, şans ve gizemin en kafa karıştırıcı birleşimi olarak en yüksek düzeyde sarsıntıya sebep olan oluşumdu. Yaşamın kutsallığı değil ölümün sarsıntısı korkuyu harekete geçirmiş ve böylece dini etkin bir biçimde yeşertmişti. İlkel insan toplulukları arasında ölüm şiddet nedeniyle o kadar sıklıkla görülen bir haldeydi ki şiddetsiz gerçekleşen ölüm artan bir biçimde gizemli hale gelmişti. Yaşamın doğal ve beklenen bir sonu olarak ölüm ilkel insanların bilinçleri için açık değildi; ve onun kaçınılmazlığının anlaşılması çağlar üstüne çağların geçmesini gerektirmiştir.
\vs p086 3:2 Öncül insan yaşamı bir gerçeklik olarak kabul etmişken, ölümü bir tür felaket olarak görmüştü. Tüm ırklar, ölüme olan öncül tutumunun kalıntısal tarihi anlatımları olarak, ölmeyen efsanevi insanlara sahiptirler. İnsan aklında hali hazırda bir biçimde, insan yaşamında açıklanamaz olan her şeyin geldiği bir alan olarak belirsiz ve düzenlenmemiş bir ruhani dünya bulunmuştu; ve ölüm anlaşılamamış olgulara dair bu uzun listeye eklenmişti.
\vs p086 3:3 İlk olarak, tüm insan hastalıkları ve doğal ölümün ruhaniyet etkisi sebebiyle gerçekleştiğine inanılmıştı. Mevcut zaman içerisinde bile bazı medenileşmiş ırklar, hastalığın “belirli bir düşman” tarafından üretildiğine ve iyileşmenin dini ayinlere bağlı olduğuna inanmaktadır. Din bilimlerinin daha sonraki ve daha karmaşık sistemleri hala, bütün bunların ilk günaha ve insanın çöküşüne dair savlara yol açtığı bir biçimde, ölümü ruhani dünyanın faaliyetine atfetmektedir.
\vs p086 3:4 Yaşamın bu gizemli ani değişikliklerinin kaynağı olarak belirsiz bir biçimde göz önünde canlandırdığı madde\hyp{}üstü dünyadan yardımı aramaya sevk eden şey hastalık ve ölüm karşısında insanın zayıflığının tanınmasıyla birlikte doğa karşısındaki güçsüzlüğün farkındalığıdır.
\usection{4.\bibnobreakspace Ölüm\hyp{}Kurtuluş Kavramı}
\vs p086 4:1 Fani kişiliğin madde\hyp{}üstü bir fazına dair kavramsallaşma, günlük yaşam olaylarının bilinçdışı ve tamamiyle kaza eseri gerçekleşen birleşiminden doğmuştur. Kabilesinin bir kaç üyesi tarafından topluluklarının hayatını yitirmiş bir önderini sürekli olarak rüyada görülmesi, eski önderin gerçekten bir şekilde geri geldiğine dair ikna edici kanıtı oluşturan görünüme sahipti. Bunların hepsi, bu tür rüyalardan kan ter içinde, titreyerek ve çığlık atarak uyanan ilk çağ insanları için oldukça gerçekti.
\vs p086 4:2 Rüyalara dayanan gelecek bir mevcudiyete olan inanç, görülen şeyler vasıtasıyla hiç yaşanmamış şeyleri her zaman hayal etme eğilimini açıklamaktadır. Ve yakın bir süre içerisinde bu yeni rüya\hyp{}hayalet\hyp{}gelecek yaşam kavramı, birey korunumunun biyolojik içgüdüsüyle ilişkili olan ölüm korkusuna etkin bir biçimde karşı koymaya başlamıştır.
\vs p086 4:3 Öncül insan, başlıca soğuk iklimlerde olmak üzere, soluk verildiğinde bir bulut gibi ortaya çıkan kendi nefesinden de fazlasıyla endişeye kapılmıştı. Ve\bibemph{ yaşam nefesi}, canlı olan ile ve ölümü olanı belirleyen bir olgu biçiminde görülmüştü. O, nefesin bedeni terk edebileceğini bilmekteydi; ve uykudayken tuhaf şeylerin her türünü gerçekleştirdiği rüyaları, bir insan varlığı bütünlüğünde madde dışı bir takım şeylerin bulunduğuna kendisini ikna etmişti. Hayalet olarak insan ruhuna ait en ilkel düşünce, nefes\hyp{}rüya düşünce\hyp{}sisteminden elde edilmişti.
\vs p086 4:4 Nihai olarak ilkel insan kendisini --- beden ve nefes halinde --- bir çifte bütünlük içerisinde düşündü. Beden olmadan nefes bir hayalet olarak bir ruhaniyete eşitti. Her ne kadar oldukça kesin bir insan kökenine sahip olsalar da, hayaletler, veya ruhaniyetler, insan\hyp{}üstü olarak görülmüştü. Bedene sahip olmayan ruhaniyetlerin varlığına dair bu inanç; olağandışı, olağanüstü, sıra dışı, ve açıklanamaz olan şeylerin ortaya çıkışını anlamlı kılan bir görüntüye sahip oldu.
\vs p086 4:5 Ölümden sonra bireyin varlığını sürdürüşüne dair ilkel sav doğrudan bir biçimde ölümsüzlüğe olan bir inanca denk düşmemektedir. Yirmiden fazlasını sayamayan varlıkların, sınırsızlık ve ebediyet hakkında düşünceye sahip olmaları neredeyse hiçbir şekilde mümkün değildi; onlar bunun yerine tekrar eden yaşam dönemlerini düşündüler.
\vs p086 4:6 Turuncu ırk özellikle, bir ruhun ölümden sonra diğer bir bedene geçişine ve yeniden dünyaya gelme düşüncesine yönelmişti. Yeniden doğum düşüncesi, doğumların atalara olan kalıtımsal ve karakter benzeyişlerinin gözlenişinden kaynağını almıştı. Dede ve ninelere ek olarak diğer ataların isimlerini çocuklara verme âdeti, yeniden doğuma olan inanç sebebiyle gerçekleşmişti. Daha sonraki dönemlerin belirli ırkları, insanın üç ila yedi kez öldüğüne inanmıştı. Bu inanç (malikâne dünyalara dair Âdem’in öğretilerinin kalıntısı olarak) ve açığa çıkarılmış dinin geride kalanları, yirminci yüzyılın gelişmemiş topluluklarının, tezat bir biçimde, abes savları arasında bulunabilir.
\vs p086 4:7 Öncül insan, cehennemin veya gelecekteki cezalandırışın hiçbir düşüncesini yürütmemişti. İlkel insanlar, bütünüyle kötü talih dışında, gelecek yaşamı tıpkı bu günkü yaşam gibi gördüler. Daha sonra --- cennet ve cehennem olarak --- iyi hayalet ve kötü hayaletler için ayrı bir nihai son fikri yürütüldü. Ancak birçok ilkel ırk bu dünyayı terk ettikten sonra bir diğerine giriş yaptığına inandığı için, yaşlı ve eli ayağı tutmaz hale geliş düşüncesinden hoşnut duymadı. Yaşlılar, haddinden fazla zayıf düşmeden önce öldürülmeyi fazlasıyla tercih etmişlerdi.
\vs p086 4:8 Neredeyse her topluluk, hayalet ruhunun nihai sonu ile ilgili farklı bir düşünceye sahip olmuştu. Yunanlılar, zayıf insanların zayıf ruhlara sahip olduklarına inanmışlardı; böylelikle onlar, bu türden cansız ruhların kabulü için uygun bir mekân olarak Hades’i yarattılar; bu kudretli olmayan türlerinin aynı zamanda daha kısa gölgeleri oldukları varsayılmıştı. Öncül And toplulukları hayaletlerinin, atalarının sahip oldukları ana vatanlarına döndüklerini düşündüler. Çin ve Mısırlı topluluklar bir zamanlar, ruh ve bedenin beraber kalmaya devam ettiğine inandılar. Mısırlılar arasında bu düşünce, dikkatli mezar inşasına ve beden korunumunda gösterilen çabalara yol açmıştı. Çağdaş topluluklar bile, ölünün çürümesini engellemeye çabalamaktadır. Museviler, bireyin bir hayali nüshasının Sheol’a gittiğini düşündüler; bu kısım, yaşam ülkesine geri dönemezdi. Onlar, ruhun evrimleşme savında bu önemli gelişmeyi sağladılar.
\usection{5.\bibnobreakspace Hayalet\hyp{}Ruh Kavramı}
\vs p086 5:1 İnsanın maddi olmayan kısmı çeşitli biçimde; hayalet, ruhaniyet, gölge, hayal, hortlak ve en son olarak \bibemph{ruh} şeklinde kavramsallaştırılmıştır. Ruh, öncül insanın hayali nitelikteki çifte bütünlüğüydü; o, dokunmaya karşılık vermemesi dışında her bakımdan tamamiyle faninin kendisiydi. Rüyaya olan inanç doğrudan bir biçimde, insanlara ek olan bir biçimde canlı ve cansız olan her şeyin ruha sahip olduğu fikrine yol açtı. Bu kavram, doğa\hyp{}ruhaniyet inanışların devamlılığına uzun süreler sebebiyet teşkil etti; Eskimo toplulukları hala doğada olan her şeyin bir ruhaniyete sahip olduğunu düşünmektedir.
\vs p086 5:2 Hayalet ruhu duyulabilir ve görülebilirdi, ancak dokunulamazdı. Kademeli bir biçimde ırkın rüya yaşamı bu evrimleşen ruhaniyetin etkinliklerini öyle bir biçimde geliştirmiş ve genişletmişti ki, ölüm nihai olarak “bireyin kendisini hayalete teslim etmesi” biçiminde görülmüştü. Hayvanların çok az bir derece üstünde bulunanlar dışında tüm ilkel kabileler, ruhun belirli bir kavramsallaşmasını geliştirmiştir. Medeniyet ilerledikçe, ruhun hurafesel kavramı yok edilmiştir; ve insan bütünüyle, Tanrı’yı bilen fani insan ve Düşünce Düzenleyicisi olarak ikamet eden onun kutsal ruhaniyetinin ortak yaratımı olarak ruha dair yeni bir düşüncesi için açığa çıkarılışa ve kişisel dini deneyime bağımlıdır.
\vs p086 5:3 Her fani, ikamet eden bir ruhaniyete ait kavramları evrimsel doğaya ait bir ruhtan ayırmada genellikle başarısız olmaktadır. İlkel insanın kafası, hayalet ruhaniyetinin bedene mi ait olduğu yoksa bedene sahip dış bir birim mi olduğu konusunda fazlasıyla karışıktı. Şaşkınlığın mevcudiyetindeki mantıksal nitelikli düşünce yoksunluğu; ilkel insanların ruhlara, hayaletlere ve ruhaniyetlere bakışındaki çok büyük tutarsızlıkları açıklamaktadır.
\vs p086 5:4 Ruhun beden ile olan ilişkisi, kokunun çiçek ile ilişkilisi gibi düşünülmüştü. İlkçağ insanları, ruhun bedeni şu gibi çeşitli türlerde terk edebileceğine inanmışlardı.
\vs p086 5:5 1.\bibnobreakspace Olağan ve geçici bayılma.
\vs p086 5:6 2.\bibnobreakspace Doğal rüya görme olarak uyuma.
\vs p086 5:7 3.\bibnobreakspace Hastalık ve kazalarla ilişkili baygınlık ve bilinçsiz
\vs p086 5:8 4.\bibnobreakspace Kalıcı göç olarak ölüm.
\vs p086 5:9 İlkel insan hapşırmayı, ruhun bedenden kaçışının başarısız bir girişimi olarak gördü. Uyanık ve tetikte olarak beden, ruhun kaçma girişimini engellemeye yetkindi. Daha sonra hapşırmaya her zaman, “Tanrı seni korusun!” gibi bir takım dini ifadeler ile eşlik edilmişti.
\vs p086 5:10 Evrimin başında uyku; hayalet ruhunun bedenden ayrı olduğunun ispatı biçiminde değerlendirilmiş, ve uyuyan kişinin isminin anılması veya onun çağrılmasıyla geri getirilebileceğine inanılmıştı. Bilinç\hyp{}dışılığın diğer türlerinde ruhun, --- yaklaşan ölüm halinde \hyp{}\hyp{}\hyp{} belki de bir daha gelmemecesine kaçamaya çalıştığı biçiminde çok uzaklarda olduğu düşünülmüştü. Rüyalar, geçici bir süreliğine bedenden ayrı bir konumda iken uyku boyunca ruhun deneyimleri olarak görülmüştür. İlkel insan rüyalarının, uyanık deneyimlerinin herhangi bir kısmı kadar gerçek olduğuna inanmaktadır. İlkçağ insanları, ruhun bedene dönüşü için belirli bir zamanın geçebileceği varsayımıyla, uyuyan bireyleri kademeli olarak uyandıran bir uygulama geliştirdiler.
\vs p086 5:11 Çağların en başından bu yana insanlar, gece vaktinde beliren ölmüş ruhların hayaletlerinin dehşetine kapılmışlardır; Museviler bu duruma istisna değillerdi. Onlar, Musa’nın bu düşünceye karşı uyarılarına rağmen, Tanrı’nın kendileriyle rüyalarında konuştuğuna gerçekten inandılar. Ve Musa haklıydı; çünkü olağan rüyalar, maddi varlıklar ile iletişime geçmeyi amaçladıklarında ruhani dünyanın kişilikleri tarafından uygulanan yöntemler değildir.
\vs p086 5:12 İlkçağ insanları, ruhun hayvanlara veya cansız nesnelere bile girebileceğine inandılar. Bu düşünce, hayvanların tanımlanmasına ait kurt adam düşünceleriyle sonuçlanmıştı. Bir insan gündüz vakti yasalara uyan bir vatandaş olabilirdi; ancak uykuya daldığı zaman onun ruhu, gece talanları içinde sinsice fırsat kollayan bir biçimde bir kurt veya başka bir hayvana dönüşürdü.
\vs p086 5:13 İlkel insan, ruhun nefes ile ilişkili olduğunu ve onun niteliklerinin nefes yoluyla aktarılabileceğini veya devredilebileceğini düşündü. Cesur önder, cesaret aktaran bir biçimde yeni doğan çocuğa üflerdi. Öncül Hıristiyanlar arasında Kutsal Ruhaniyet’in bahşedilme ayini, adaylara üflenilmesiyle eşlik edilmişti. Zebur yazarı şu ifadede bulunmuştur: “Koruyucu’nun sözüyle cennetler, ağzından çıkan nefes ile onların tüm sakinleri yaratıldı.” En büyük erkek evladın ölmekte olan babasının son nefesini yakalaması uzunca bir süredir adetti.
\vs p086 5:14 Daha sonra, nefes ile birlikte eşit derecede korkulan ve derin bir biçimde saygı duyulan gölge gelmişti. Bir bireyin sudaki yansıması zaman zaman, çifte bütünlüğün kanıtı olarak da görülmüştü; ve aynalara hurafesel dehşet ile bakılmıştı. Şimdi bile birçok medenileşmiş bireyler, ölüm esnasında aynayı duvara doğru çevirmektedir. Bazı geri kalmış kabileler hala; resim yapmanın, çizimlerin, kalıpların veya fotoğrafların bedenden ruhun bir kısmını veya hepsini götürdüğüne inanmaktadır.
\vs p086 5:15 Ruhun genel olarak nefes ile özdeş olduğu düşünülmüştü; ancak ruh aynı zamanda çeşitli insan toplulukları tarafından kafada, saçta, kalpte, karaciğerde, kanda ve yağda konumlandırılmıştı. “Habil’in yerdeki kanının haykırışı” kan içindeki hayaletin mevcudiyetine beslenen bir dönemdeki inancın dışavurumudur. Sami toplulukları, ruhun beden yağı içinde konumlandığını düşündüler; ve birçokları arasında hayvan yağı yemek bir tabuydu. Kafa avlamak, kafa derisini yüzmek gibi, bir düşmanın ruhaniyetini elde etme yöntemiydi. Yakın zamanlarda gözlerin ruhun pencereleri olduklarını düşünülmektedir.
\vs p086 5:16 Üç veya dört ruha dair sava inanan kişiler; bir ruhu kaybetmenin huzursuzluk, ikisinin hastalık ve üçünün ölüm anlamına geldiğine inandılar. Bir ruh nefes içinde, biri kafa içinde, biri saç içinde ve bir diğeri ise kalp içinde yaşamıştı. Hasta insanlara, kendilerinden ayrılmış ruhları yeniden yakalama ümidiyle açık havada yürüyüş yapmaları tavsiye edilmekteydi. Sağlıkçıların en iyilerinin, “yeniden doğum” olarak hastalıklı bir kişinin hasta ruhunu yenisi ile değiştirdikleri düşünülmekteydi.
\vs p086 5:17 Bodanan topluluğunun çocukları, nefes ve gölge olarak iki ruha dair bir inancı geliştirdi. Öncül Nod ırkları insanı, ruh ve bedenden olarak iki bireyden meydana gelen bir biçimde değerlendirdi. İnsan mevcudiyetinin felsefesi daha sonra Yunan görüşünde temsil edilmişti. Yunanlılar’ın kendileri üç ruha inanmışlardı; bunlar, midedeki bitkisel yaşam, kalpteki hayvansal yaşam ve kafadaki ussal yaşamdı. Eskimo toplulukları insanın üç kısma sahip olduğuna inandı; beden, ruh ve isim.
\usection{6.\bibnobreakspace Hayalet\hyp{}Ruhaniyet Çevresi}
\vs p086 6:1 İnsan doğal bir çevre mirasını devraldı, toplumsal bir çevre kazandı ve bir hayalet çevresini hayal etti. Devlet, insanın kendi doğal çevresine; ev, toplumsal çevresine; ve kilise, hayali hayalet çevresine olan tepkisidir.
\vs p086 6:2 İnsanlık tarihinin daha başında, hayalet ve ruhaniyetlerin ait oldukları hayali dünyanın gerçeklikleri herkes tarafından inanılan bir hale gelmişti; ve bu yeni hayal edilen ruhaniyet dünyası, ilkel toplumda bir güç konumuna gelmişti. Tüm insanlığının düşünsel ve ahlaki yaşamı, insan düşünüşü ve faaliyetinde bu yeni etkenin ortaya çıkışıyla birlikte sürekli olarak değişikliğe uğramıştı.
\vs p086 6:3 Hayal ve cehaletin ana savı üzerine, fani korkusu ilkel toplulukların daha sonraki batıl inançları ve dinlerinin tümünü inşa etmiştir. Bu inanış, açığa çıkarılış dönemlerine kadar insanın tek diniydi; ve bugün dünya ırklarının çoğu evrimin sadece bu ilkel dinine sahiptir.
\vs p086 6:4 Evrim ilerledikçe iyi talih iyi ruhaniyetler ile ve kötü talih kötü ruhaniyetler ile ilişkilendirilen hale gelmişti. Değişen bir çevreye olan zorunlu uyumun yarattığı huzursuzluk, ruhaniyet tanrılarının memnuniyetsizliği olarak kötü talih biçiminde değerlendirilmişti. İlkel insan, içkin ibadet dürtüsüyle ve şansa dair kavram yanılgısıyla dini yavaşça geliştirmiştir. Medenileşmiş insan, bu şans olaylarının üstesinden gelmek için sigorta türlerini sağlamaktadır; çağdaş bilim, hayali ruhaniyetler ve tuhaf tanrıların yerine matematiksel hesaplamalar ile birlikte sigorta uzmanlığını koymaktadır.
\vs p086 6:5 Her geçen nesil atalarının budalaca hurafelerine gülümserken, gelecek aydınlamış kuşakta daha fazla gülümsemeye sebebiyet verecek bir biçimde düşünceye ve ibadete dair mevcut yanlış inanışlarını beslemeye devam etmektedir.
\vs p086 6:6 Ancak en sonunda ilkel insanın aklı, içkin biyolojik dürtülerinin tümünü aşan düşüncelerle dolmuştu; en sonunda insan, maddi uyarımlara verilen tepkilerden daha fazlası olan bir takım şeylere dayanan bir yaşam sanatını geliştirmeye başlamaktaydı. İlkel bir felsefi yaşam siyasasının ilk adamları ortaya çıkmaktaydı. Yaşamın doğa\hyp{}üstü bir ortak ölçütü ortaya çıkmaya başlamaktaydı; çünkü eğer ruhaniyet hayaleti sinirli bir biçimde kötü talihe, memnuniyet içinde iyi şansa sebebiyet vermekteyse bunun sonrasında insan davranışı buna göre düzenlenmek zorundaydı.
\vs p086 6:7 Bu kavramların ortaya çıkmasıyla birlikte; mezarlar, kurbanlar ve din adamları uğrunda insan çabalarının uzun süreli ziyanı olan bir biçimde evrimsel dini korkuya kölesel esaret halinde bulunan, hiçbir zaman memnun olmayan ruhaniyetleri tatmin etmek için uzun ve ziyankâr çaba başladı. Bu ödenmesi gerek çok kötü ve korkunç bir bedeldi; ancak o tüm maliyete değerdi; çünkü insan bu süreç içerisinde, göreceli doğru ve yanlışın doğal bir bilincini elde etti; insanın etik kuralları doğmuştu!
\usection{7.\bibnobreakspace İlkel Dinin İşlevi}
\vs p086 7:1 İlkel insan, kendisini teminat altına alma ihtiyacı hissetti; ve o bu nedenle, kötü talihe karşı büyü sigorta poliçesi için korkudan, hurafeden, dehşetten ve din adamı hediyelerinden oluşan külfetli primlerini istekli bir biçimde ödemişti. İlkel din yalın bir değişle, ormanların tehlikelerine karşı sigorta primlerinin ödenmesiydi; ilkel insan, üretim kazaları ve yaşamın çağdaş türlerinin acil durumlarına karşı maddi primlerini ödemektedir.
\vs p086 7:2 Çağdaş toplum sigorta faaliyetini, din adamları ve dinin nüfuz alanından taşıyıp ekonomik ilişkilerin alanına yerleştirmektedir. Din artan bir biçimde kendisini, mezarın ötesindeki yaşamın teminat altına alınışıyla ilgili kılmaktadır. Çağdaş insanlar, en azından düşünenler, talihi denetim altına almak için ziyankâr primleri artık vermemektedirler. Din yavaş bir biçimde, kötü talihe karşı bir sigorta düzeni olarak eski faaliyetine kıyasla daha yüksek felsefi düzeylere yükselmektedir.
\vs p086 7:3 Ancak dinin bu ilkçağ düşünceleri insanın kaderci ve ümitsiz bir biçimde karamsar hale gelişini engellemiştir; onlar, en azından kaderi etkileyecek bir şeyleri yapabileceklerine inandılar. Hayalet korkusunun dini, insanın nihai sonunu denetleyen madde\hyp{}üstü bir dünya biçiminde, \bibemph{davranışlarını düzenlemelerinin} zorunda olduklarını anlamalarını sağlamıştı.
\vs p086 7:4 Çağdaş medeni ırklar, talih ve ortak eşitsizliklerin mevcudiyetinin bir açıklaması olarak hayalet korkusundan yeni yeni kurtulmaktadır. İnsanlık, kötü talihin hayalet\hyp{}ruhaniyet açıklamasının yarattığı esaretten kurtuluşu elde etmektedir. Ancak insanlar yaşamdaki ani değişikliklerin bir ruhaniyet sebebine dayandığına dair hatalı savdan kurtulurken, tüm insan eşitsizliklerini yanlış siyasi uyum, toplumsal adaletsizlik ve üretimsel rekabete dayandırmaya kendilerini iten neredeyse eşit derecedeki dayanaksız bir öğretiyi kabul etmek için şaşırtıcı bir isteklilik göstermektedirler. Ancak yeni yasama, artan toplumsal fedakârlık ve daha fazla üretimsel nitelikteki yeniden düzenlemeler, özünde iyi olsalar da, doğumun gerçekleri ve yaşamın kazalarına deva olmayacaktır. Sadece gerçeklerin kavranılışı ve doğa yasalarını bilge bir biçimde kişisel yarara kullanma, insanı istedikleri şeyi elde etmesine ve istemediklerinden kaçınmasına yetkin kılacaktır. Bilimsel eyleme yol açan bilimsel bilgi, kaza eseri gerçekleşen hastalıklar olarak tanımladığınız şeyler için tek çaredir.
\vs p086 7:5 Üretim, savaş, kölelik ve toplumsal hükümet; doğal çevresi içinde insanın toplum evrimine verdiği tepkiden doğmuştu; din benzer bir biçimde, hayali hayalet dünyasının düşsel çevresine verdiği tepki olarak doğmuştu. Din, bireyin kendisini idare edişinin evrimsel bir gelişimiydi; ve din, her ne kadar başta kavramsal olarak hatalı ve tamamen mantık dışı olsa da, görevini yerine getirmiştir.
\vs p086 7:6 İlkel din; gerçek olmayan korkunun güçlü ve dehşet verici kuvveti ile birlikte, Düşünce Düzenleyicisi olarak doğa\hyp{}üstü kökenin gerçek bir ruhsal kuvvetinin bahşedilişi için, insan aklının toprağını hazırlamıştı. Ve kutsal Düzenleyiciler bahse konu zamandan bu yanan Tanrı\hyp{}korkusunu Tanrı\hyp{}sevgisine dönüştürmek için çaba sarf etmektedirler. Evrim yavaş olabilir, ancak hataya yer bırakmayan bir biçimde etkindir.
\vs p086 7:7 [Nebadon’un bir Akşam Yıldızı tarafından sunulmuştur.]
