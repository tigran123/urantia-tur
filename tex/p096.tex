\upaper{96}{Yahveh --- Museviler’in Tanrısı}
\vs p096 0:1 İlahiyat’in düşünülüşünde insan ilk önce tanrıları hesaba kadar; daha sonra tüm yabancı tanrıları kendi kabilesel ilahiyatına bağlı kılar; ve nihai olarak son ve yüce değerdeki tek Tanrı dışında her tanrıyı dışlar. Museviler tanrıların tümü, İsrail’in Koruyucu Tanrısı’nın daha yüce kavramsallaşmasında birleştirmişti. Hint toplulukları benzer bir biçimde, Rig\hyp{}Veda’da tasvir edilen “tanrıların tek ruhsallığı” altında çeşitli ilahiyatları bir araya toplamışlardı; bunun karşısında Mezopotamyalı topluluklar tanrılarını, daha odaklaşmış Bel\hyp{}Marduk kavramsallaşmasına indirgemişlerdi. Tek\hyp{}tanrı inancına dair bu düşünceler, Filistin’deki Salem’de Maçiventa Melçizedeği’nin ortaya çıkışından çok daha uzun bir süre geçmeden tüm dünyada olgunlaşmıştı. Ancak İlahiyat’ın Melçizedek kavramsallaşması; ekleme, bağlı kılma ve dışarıda bırakmadan oluşan evrimsel felsefeninkine benzerlik göstermemekteydi; o ayrıcalıklı bir biçimde \bibemph{yaratıcı güce} dayanmış olup, çok yakın bir zaman içinde Mezopotamya, Hindistan ve Mısır’ın en yüksek ilahiyat kavramsallaşmalarını etkilemişti.
\vs p096 0:2 Salem dini, Ken ve birkaç diğer Kenani kabile tarafından derin saygı duyulan bir tarihsel anlatıydı. Ve Melçizedek’in bedene bürünüşünün amaçlarından bir tanesiydi şuydu: Tek Tanrı’ya ait bir din, bu tek Tanrı’nın bir Evladı’nın dünyasal bahşedilişi için zemin hazırlayacak bir biçimde desteklenmelidir. Mikâil, içinde ortaya çıkabileceği Kâinatın Yaratıcısı’na inanan bir topluluk mevcut olana kadar Urantia’ya neredeyse hiçbir şekilde gelemezdi.
\vs p096 0:3 Salem dini, Filistin’deki Ken toplulukları arasında öğretileri olarak varlığını sürdürdü; ve daha sonra Museviler tarafından alınan bu din, ilk olarak Mısırlı ahlaki öğretiler tarafından etkilendi; bu süreci daha sonra Babil’in din\hyp{}kuramsal düşüncesi izledi; ve son olarak kötülük ve iyiliğe dair İranlı kavramsallaşmalar geldi. Gerçekte Musevi dini, İbrahim ve Maçiventa Melçizedeği arasındaki sözleşmeye dayanmaktadır; evrimsel olarak o, birçok benzersiz nitelikli durumsal koşulların gelişimidir; ancak kültürel olarak o, bütün Levant’ın dininden, ahlakından ve felsefesinden yararlanmıştır. Musevi dini vasıtasıyla Mısır, Mezopotamya ve İran’ın ahlaksal ve dini düşüncesinin çoğu Batı topluluklarına aktarılmıştı.
\usection{1.\bibnobreakspace Sami Toplulukları İçindeki İlahiyat}
\vs p096 1:1 Öncül Sami toplulukları her şeyin bir ruhaniyet tarafından ikamet edildiğini düşünmüştü. Orada, hayvan ve bitki dünyalarının ruhaniyetleri bulunmaktaydı; doğurganlığın koruyucusu olarak senelik ruhaniyetler; ateş, hava ve suyun ruhaniyetleri; korkulması ve ibadet edilmesi gereken ruhaniyetlerin dikkate değer bir birliği mevcuttu. Ve bir Kâinatsal Yaratan’a dair Melçizedek öğretisi hiçbir zaman, bu alt ruhaniyetler veya diğer bir değişle doğa tanrılarına olan inancı bütünüyle yok edemedi.
\vs p096 1:2 Museviler’in çoktanrıcılıktan, tek bir tanrıya inanıp çoktanrıcılığı reddetmeyen aşamaya ve oradan da saf tektanrıcılığa doğru ilerleyişi; birbirine doğrudan bağlı ve devamlı bir kavramsal gelişme değildi. Onlar, İlahiyat kavramsallaşmalarının evrimde birçok gerilemeyi deneyimledi; bunun karşısında, zaman zaman birçok terim Tanrı’ya ait kavramlarına uygulandı; ve kafa karışıklığını engellemek için bu çeşitli İlahiyat isimleri, Musevi din kuramının evrimiyle ilişkili olarak tanımlanacaktır:
\vs p096 1:3 1.\bibnobreakspace \bibemph{Yahveh}; Sina yanardağı olan Horeb Dağı ile birlikte ilahiyatın bu kavramsallaşmasıyla ilişkilendirilmiş, güney Filistinli kabilelerin tanrısıydı. Yahveh yalnızca, üzerinde ilgiyi toplamış ve Sami kabileleri ve topluluklarının ibadeti için duyurulmuş doğa tanrılarının yüz binlercesinden yalnızca biriydi.
\vs p096 1:4 2.\bibnobreakspace \bibemph{El Elyon}. Melçizedek’in Salem’deki kısa süreli ikametinden sonra çağlar boyunca onun İlahiyat savı çeşitli uyarlamalar içinde varlığını sürdürdü; ancak o genel olarak, gökyüzünün En Yüksek Tanrısı olarak El Elyon ismiyle çağrıldı. İbrahim’in doğrudan soylarını içine alan bir biçimde birçok Sami topluluğu, çeşitli dönemlerde hem Yahveh’e hem de El Elyon’a ibadet etti.
\vs p096 1:5 3.\bibemph{ El Shaddai}. El Shaddai’nin ne anlama gelmiş olduğunu açıklamak zordur. Tanrı’ya dair bu düşünce; İkhnaton’un Aton savı tarafından değişikliğe uğramış ve buna ilaveten El Elyon’un kavramsallaşmasında vücut bulan Melçizedek öğretileri tarafından etkilenmiş olan Amenemope’nin Bilgelik Kitabı’na ait öğretilerden elde edilmiş bir derlemeydi. Ancak El Shaddai’nin kavramsallaşması Musevi aklına nüfuz ederken, bütüncül bir biçimde çöldeki Yahveh inanışları tarafından renklenmişti.
\vs p096 1:6 Bu döneme ait dinin baskın düşüncelerinden bir tanesi, maddi refahın El Shaddai’ye edilen hizmetin bir ödülü olduğunu sunan öğreti olarak, kutsal Yazgı’ya dair Mısırlı kavramsallaşmaydı.
\vs p096 1:7 4.\bibnobreakspace \bibemph{El}. İsimlendirmenin bu kafa karışıklığı ve kavramsallaşmanın bu muğlâklığı arasında birçok dindar inanan içten bir biçimde, kutsallığın bu evrimleşen düşüncelerinin tümüne ibadet etmeye çalıştı; ve orada, bu birleşik İlahiyatı El olarak tanımlandırma uygulaması doğdu. Ve bu terim, Bedevi doğa tanrılarının diğerlerini de içine aldı.
\vs p096 1:8 5.\bibnobreakspace \bibemph{Elohim}. Kiş ve Ur’da uzunca bir süre boyunca, Âdem ve Melçizedek’in dönemlerine ait tarihsel anlatımlara dayanmış üç katmanlı bir Tanrı kavramsallaşmasını öğreten Sümerli\hyp{}Keldani toplulukları varlıklarını sürdürdü. Bu sav, bu Kutsal Üçleme’ye Elohim veya tekil olarak Eloah ismi altında ibadet edildiği yer olan Mısır’a taşındı. Mısır’ın felsefe camialarına ek olarak Musevi kökenden gelen daha sonraki İskenderiye öğretmenleri, çoğul Tanrılar’ın bu birliğini öğretti; ve Musa’nın danışmanlarından çoğu büyük göç zamanında bu kutsal Üçleme’ye inandı. Ancak üç\hyp{}katmanlı Elohim kavramsallaşması hiçbir zaman, Babil unsurlarının toplumsal baskısı altına girmeden Musevi din\hyp{}kuramının gerçek bir parçası haline gelmedi.
\vs p096 1:9 6.\bibemph{ Çeşitli diğer isimler}. Sami toplulukları, İlahiyatları’nın isimlerini anmayı sevmiyorlardı; ve onlar bu nedenle, zaman zaman şunlar gibi çeşitli unvanlara başvurmuşlardı: Tanrı’nın Ruhaniyeti, Koruyucu, Koruyucu’nun Meleği, Her Şeye Gücü Yeten, Kutsal Olan, En Yüksek Unsur, Adonai, Zamanın Ataları, İsrailin Koruyucu Tanrı’sı, Gökyüzü ve Yeryüzü’nün Yaratanı, Kurios, Yah, Meleklerin Koruyucusu, Gökyüzündeki Yaratıcı.
\vs p096 1:10 \bibemph{Yehova}, uzun Musevi deneyimi içinde nihai olarak evirilmiş Yahveh’in tamamlanmış kavramsallaşmasını tanımlamak için yakın dönemlerde kullanılmış olan bir terimdir. Ancak Yehova, İsa’dan sonra bin beş yüzlü yıllara kadar kullanılmamıştı.
\vs p096 1:11 Yaklaşık M.Ö. 2000’li yıllara kadar Sina Dağı bir yanardağ niteliğinde, en sonu bu bölgedeki İsrail topluluklarının kısa süreli ikametinde gerçekleşen bir biçimde ara sıra meydana gelen patlamalar olarak düzensiz aralıklar halinde faaldi. Bu yanardağın püskürmeleriyle ilişkili sağır edici patlamalarla birlikte ateş ve dumanın tümü; çevre bölgelerdeki Bedevileri etkilemiş, hayretler içinde bırakmış olup, Yahveh’den fazlasıyla korku duymalarına neden olmuştu. Horeb Dağı’nın bu ruhaniyeti daha sonra Musevi Sami topluluklarının tanrısı haline geldi; ve onlar nihai olarak, bu tanrının tüm diğer tanrıların en üstünü olduğuna inandılar.
\vs p096 1:12 Kenani toplulukları uzun bir süre boyunca Yahveh’e derin saygı beslemiş bir konumdalardı; ve her ne kadar Ken topluluklarının büyük bir kısmı Salem dininin üstün\hyp{}tanrısı olan El Elyon’a az veya çok inanmış olsa da, Kenani topluluklarının büyük büyük bir kısmı eski kabile ilahiyatlarına yapılan ibadete gevşek bir biçimde bağlılardı. Onlar kâinatsal\hyp{}ilahiyat kafasında değillerdi; ve böylece bu kabileler, Yahveh’i ve Bedevi sürü sahiplerine ait Sina yanardağının ruhaniyetinin kavramsallaşmasını simgeleyen gümüş ve altın buzağıları içine alan kabile ilahiyatlarına ibadet etmeyi sürdürmüşlerdi.
\vs p096 1:13 Suriyeliler, tanrılarına ibadet ederken aynı zamanda Museviler’in Yahveh’i ne de inanmışlardı; çünkü onların tanrı\hyp{}elçisi Suriye kralına şunları söylemişti: “Onların tanrıları tepelerin tanrılarıdır; bu nedenle onlar geçmişte bizden daha güçlü olmuştu; ancak onlarla düzlükte savaşmamıza izin ver, ve kesin bir biçimde biz burada onlardan daha güçlü olacağız.”
\vs p096 1:14 İnsan kültür bakımından gelişme gösterince, daha küçük tanrılar, yüce bir ilahiyata bağlanmışlardı; büyük Jüpiter varlığını, yalnızca bir haykırış olarak devam ettirmektedir. Tek\hyp{}tanrı inananları bağımlı tanrılarını; ruhaniyetler, ecinniler, mireler, Nereidler, periler, cüce cinler, cüceler, ölüm cadıları ve kem göz olarak muhafaza etmektedirler. Museviler tek tanrıya inanan ancak diğer tanrıların varlığını kabul eden inanıştan geçip, uzunca bir süre Yahveh’den başka tanrıların mevcudiyetine inandı; ancak onlar artan bir biçimde, bu yabancı ilahiyatları Yahveh’e bağlı görmüşlerdi. Onlar, Amor topluluklarının tanrısı olan Çemoş’un mevcudiyetini kabul etmişlerdi, fakat bu tanrının Yahveh’e bağlı olduğunu öne sürmüşlerdi.
\vs p096 1:15 Yahveh düşüncesi, Tanrı’ya dair tüm fani savların en geniş çaplı gelişiminden geçmişti. Onun ilerleyici evrimi yalnızca; tıpkı Yahveh kavramsallaşmasının nihai olarak Kâinatın Yaratıcısı’nın düşüncesine yol açtığı gibi sonunda Kâinatsal Mutlak’ın kavramsallaşmasına yol açan bir biçimde, Asya’da Buda kavramsallaşmasının başkalaşımı ile karşılaştırabilir. Ancak özünde tarihsel bir gerçek olarak; her ne kadar Museviler İlahiyat’a dair görüşlerini Horeb Dağı’nın kabile tanrısından daha sonraki dönemlerin sevgi dolu ve bağışlayıcı Yaratan’a doğru değiştirmiş olsalar da, bu tanrının ismini değiştirmemiş olmamaları görülmelidir; onlar, İlahiyat’ın bu evrimleşen kavramsallaşmasını Yahveh olarak adlandırmaya en başından beri devam etmişlerdi.
\usection{2.\bibnobreakspace Sami Toplulukları}
\vs p096 2:1 Doğu’nun Samileri; doğu bölgelerinin verimli hilalini ele geçirmiş oldukça iyi örgütlenmiş ve oldukça iyi yönetilmiş atlılar olup, Babil unsurları ile birlikte bütünleşmişlerdi. Ur yakınındaki Keldani toplulukları, doğu Samileri içinde en gelişmiş olanlarıydı. Fenikeliler, Akdeniz sahilini de içine alan bir biçimde Filistin’in batı bölgesini elinde bulundurmuş karma Sami topluluklarının üstün ve oldukça iyi örgütlenmiş bir birimiydi. Irksal olarak Sami toplulukları, dokuz dünya ırkının neredeyse hepsinden kalıtım kökenleri taşıyarak Urantia topluluklarının en karışmış olanları arasındaydı.
\vs p096 2:2 Sürekli tekrarlanan bir biçimde Arap Samileri, “süt ve bal akan” yer olan Vaat\hyp{}edilmiş Topraklar’ın kuzeyine girebilmek için savaştılar; ancak onlar aynı sık içerisinde, daha iyi örgütlenmiş ve daha yüksek bir biçimde medenileşmiş kuzey Samileri ve Hititliler tarafından kovulmuşlardı. Daha sonra, olağandışı derecede sert geçen bir kuraklık boyunca, bu başıboş Bedeviler büyük sayılar halinde; sadece, Nil Vadisi’nin yaygın görülen, ezilmiş işçilerinin günlük zor işlerinde köleliğin acı deneyimlerini yaşarken kendilerini bulan bir biçimde, Mısır’ın imar işlerinde görev yapan sözleşmeli işçiler olarak Mısır’a giriş yapmışlardı.
\vs p096 2:3 Yalnızca Maçiventa Melçizedeği ve İbrahim’in dönemi sonrasında Sami topluluklarının belirli kabileleri, belirli dini inanışları nedeniyle; önce İsrail’in çocukları, daha sonra İbraniler, Museviler ve “seçilmiş insanlar” olarak adlandırılmışlardı. İbrahim, İbraniler’in tümünün ırksal babası değildi; o, Mısır’da esir tutulmuş Bedevi Samileri’nin tümünün atası bile değildi. Onun doğumunun, Mısır’dan gelerek daha sonraki Musevi topluluğunun çekirdeğini kurduğu doğrudur; ancak İsrail kavimlerinin bir parçası haline gelen erkek ve kadınların çok büyük çoğunluğu daha öncesinde hiçbir zaman Mısır’da ikamet etmemişti. Onlar sadece, İbrahim’in çocukları olarak Musa’nın önderlerini takip etmeyi seçmiş akran göçebelerdi; ve Mısır’dan gelen Sami birliktelikleri kuzey Arabistan boyunca ilerlemişlerdi.
\vs p096 2:4 Melçizedek’in En Yüksek Unsur olan El Elyon’a ve inanışla gelen kutsal lütfun sözleşmesine dair öğretisi, İbrani milletini yakın bir zaman içinde kuracak olan Sami topluluklarının Mısır köleliği zamanında neredeyse geniş bir ölçüde unutulmuş haldeydi. Ancak esaretin bu dönemi boyunca bu Arap göçebeleri, ırksal ilahiyatları olarak Yahveh’e olan, hala varlığını sürdüren geleneksel inanışlarını sürdürdü.
\vs p096 2:5 Yahveh’e, yüzden fazla ayrı Arap kabilesine tarafından ibadet edilmişti; ve Mısır’ın daha fazla eğitim görmüş toplumsal sınıfları arasında varlığını sürdürmüş Melçizedek’in El Elyon kavramsallaşmasının izi dışında, buna İbrani ve Mısır ırk kollarının karışımı da ek olarak, esir İbrani kölelerinin en alt düzeyde bulunanların dini büyü ve fedadan oluşan eski Yahveh ayininin değişikliğe uğramış bir türüydü.
\usection{3.\bibnobreakspace Benzersiz Musa}
\vs p096 3:1 Bir Yüce Yaratan’a dair İbrani kavramsallaşmaları ve ideallerinin evriminin başlangıcı, Sami topluluklarının Mısır’dan büyük önder, öğretmen ve düzenleyici olan Musa’nın altındaki göçüne dayanır. Musa’nın annesi, Mısır’ın kraliyet ailesinden gelmektedir; onun babası, hükümet ve Bedevi esirleri arasında bir Sami irtibat görevlisiydi. Musa böylelikle üstün ırk kökenlerinden gelen niteliklere sahipti; onun soyu, herhangi bir ırksal topluluk içinde sınıflandırmanın imkânsız olduğu ölçüde oldukça karışmış bir haldeydi. O bu kadar karışık bir koldan gelmeseydi; Mısır’dan Arap Çölü’ne önderleri altında kaçarak göç etmiş olan bu Bedevi Samileri ile nihai olarak ilişkili hale gelmiş, çeşitlilik gösteren topluluğu idare etmede kendisini yetkin hale getiren olağanüstü derecedeki çok yönlülüğü ve uyumluluğu hiçbir zaman sergileyemezdi.
\vs p096 3:2 Nil krallığının kültürünün baştan çıkarışlarına rağmen Musa şansını babasının insanlarından yana denemeyi tercih etti. Bu büyük düzenleyici babasının insanlarının nihai bağımsızlığı için tasarılarını oluşturduğunda, Bedevi esirleri neredeyse, ismi söylenmeye değer bir dine sahip değillerdi; onlar adeta Tanrı’nın gerçek bir kavramsallaşmasından ve dünyadaki umuttan yoksunlardı.
\vs p096 3:3 Hiçbir önder bu döneme kadar; insan varlıklarının daha ümitsiz, daha hüzünlü, daha mutsuz ve daha cahil bir topluluğunu kökten değiştirme ve geliştirme sorumluluğunu üstlenmemişti. Ancak bu köleler, kalıtımsal ırk kollarında gelişimin gizli olasılıklarını taşımışlardı; ve orada, isyan gününe ve özgürlük grevine hazırlık için Musa tarafından yönlendirilen etkin düzenleyicilerin bir birliğini oluşturacak kadar yeterli bir sayıda eğitilmiş önder bulunmaktaydı. Bu üstün insanlar öncesinde, insanlarının yerli denetleyicileri olarak görevlendirilmişlerdi; onlar, Mısır yöneticileri üzerindeki Musa’nın etkisi nedeniyle belli bir eğitim almış haldelerdi.
\vs p096 3:4 Musa, akran Sami topluluklarının özgürlüğü için diplomatik bir biçimde müzakere etmeye çabalamıştı. O ve kardeşi, Nil vadisinden Arap Çölü’ne barışçıl bir biçimde ayrılmalarının iznini aracılığıyla almış oldukları, Mısır kralı ile resmi bir anlaşma yaptılar. Onlar, Mısır’daki uzun hizmetlerinin bir simgesi olarak mütevazı ölçüde para ve eşya alacaklardı. İbraniler, kendi çıkarları bakımından; Firavunlar ile dostane ilişkileri sürdürmek ve Mısır’a karşı hiçbir ittifak içine girmemek için bir anlaşma içine girdiler. Ancak kral daha sonra; hafiyelerinin Bedevi köleleri içindeki sadakatsizliği keşfetmiş olduğu bahanesini kullanarak bu anlaşmayı bozmayı kendisinde hak gördü. O, bu insanların çöle daha sonra Mısır’ı karşı göçebeleri örgütlemek amacıyla gitmeyi amaçladığını duyurdu.
\vs p096 3:5 Ancak Musa, ümitsizliğe kapılmadı; o kendi zamanını bekledi, ve bir yıldan daha az bir süre içinde, Mısırlı askeri güçler güneyden gelen güçlü bir Libya hücumu ile kuzeyden gelen bir Yunan donanma işgalinin saldırılarına eş zamanlı bir biçimde karşı koymada tamamiyle meşgul iken, bu gözü pek düzenleyici yurttaşlarını muhteşem bir gece kaçışında Mısır’dan dışarı yönlendirdi. Özgürlük için bu atılım, özenle tasarlanmış ve hünerle uygulanmıştı. Ve onlar, daha fazla ganimete sebebiyet veren bir biçimde hepsinin kaçaklarının önüne düştüğü Mısırlılar’ın küçük bir topluluğuna ek olarak Firavun tarafından öfkeye takip edilmelerine rağmen başarılı olmuşlardı; bu maddi kayıplar, atalarının çöl evine doğru ilerlemekte olan kaçış halindeki kölelere yaklaşan bir birimin yağmasıyla artış göstermişti.
\usection{4.\bibnobreakspace Yahveh’in Duyurusu}
\vs p096 4:1 Musa öğretisinin evrimi ve yükselişi; dünyanın nereyse bir yarısını etkilemiş olup, yirminci yüzyılda hala bunu gerçekleştirmektedir. Her ne kadar Musa daha gelişmiş Mısır dini felsefesini kavramış olsa da, Bedevi köleleri bu türden öğretiler hakkında çok az şey bilmekteydi; ancak onlar hiçbir zaman, atalarının öncesinden Yahveh olarak adlandırmış oldukları Horeb Dağı’nın tanrısını unutmamışlardı.
\vs p096 4:2 Musa; kraliyet kanından gelen bir kadın ve esir bir ırktan gelen bir erkek arasındaki olağandışı birlikteliği açıklayan dini inanışlarının ortaklığı olarak, hem babasından hem de annesinden Maçiventa Melçizedeği’nin öğretilerini önceden duymuş bir konumdaydı. Musa’nın kayınpederi El Elyon’a ibadet eden bir Ken inananıydı; ancak özgürleştiricinin ebeveynleri El Shaddai’ye inanmaktalardı. Musa böylelikle bir El Shaddaist olarak eğitildi; kayınpederinin etkisiyle bir El Elyonist haline geldi; ve Mısır’dan kaçıştan sonra Sina Dağı yakınında İbrani konaklayışı zamanında o, eski kabile tanrıları olan Yahveh’in genişlemiş bir kavramsallaşması olarak insanlarına bilgece duyurmaya karar verdiği, (tüm eski inanışlarından elde ettiği biçimde) İlahiyat’ın yeni ve büyümüş bir kavramsallaşmasını tasarlamış haldeydi.
\vs p096 4:3 Musa, El Elyon düşüncesini bu Bedevi unsurlara öğretme çabasını önceden vermiş bir haldeydi; ancak Mısır’dan ayrılmadan önce o, bu insanların bu savı hiçbir zaman bütünüyle kavrayamayacağından emin olmuştu. Bu nedenle o bilinçli bir biçimde, takipçilerinin tek bir tanrısı olarak onların kabilelerinin çöl tanrısını uyarlama tavizinde bulunmaya karar vermişti. Musa özellikle, diğer toplulukların ve milletlerin başka tanrılara sahip olamayacaklarının öğretiminde bulunmamıştı; ancak o kararlı bir biçimde Yahveh’in, başta İbraniler olarak her şeyin üstünde ve ötesinde olduğunu öne sürmüştü. Ancak o her zaman; Bedevi kabilelerinin altın buzağısıyla her dönemde simgeleştirilmiş olan ilkçağ Yahveh teriminin görünüşü altında, bu cahil kölelere İlahiyat’a dair kendisinin yeni ve daha yüksek düşüncesini sunmaya çalışmanın alışılmadık zorluğundan muzdarip olmaktaydı.
\vs p096 4:4 Yahveh’in kaçmakta olan İbraniler’in tanrısı olduğu gerçeği; Sina kutsal dağı önünde onların niçin çok uzun bir süre beklediklerini, ve neden orada, Horeb tanrısı olan Yahveh adına Musa’nın duyurmuş olduğu on emri aldıklarını açıklamaktadır. Sina önündeki bu uzun süreli ikamet boyunca yeni evirilmekte olan İbrani ibadetinin dini tören düzenleri daha ileri derecede kusursuzlaştırılmıştı.
\vs p096 4:5 Eteklerindeki ibadetsel ikametlerinin üçüncü haftası boyunca Horeb’in şiddetli patlayışı olmasaydı, Musa’nın bir ölçüde gelişmiş törensel ibadetin kurulmasında ve çeyrek asır boyunca takipçilerini bir arada tutmada herhangi bir biçimde başarılı olabileceği mümkün görünmemektedir. “Yahveh’in dağı ateşler tarafından yutuldu, ve duman bir bacanınki gibi yükseldi, ve bütün dağ fazlasıyla sallandı.” Bu ansızın gerçekleşen felaket göz önünde bulundurulduğunda, Musa’nın kardeşlerine Tanrıları’nı “kudretli, korkunç, yok edici bir ateş, korkutucu ve her şeye gücü yeten” olarak sunan öğretiyi aşılaması şaşırtıcı değildir.
\vs p096 4:6 Musa Yahveh’in, İbranileri kendi seçilmiş insanları olarak belirlemiş İsrail’in Koruyucusu Tanrısı olduğunu duyurdu; o yeni bir millet inşa etmekteydi, ve o bilge bir biçimde, bir “kıskanç Tanrı” olarak Yahveh’in zor bir görev verici olduğunu takipçilerine söyleyerek dini öğretilerini milleştirmişti. Ancak yine de o; Yahveh’in “tüm bedenlere ait ruhaniyetin Tanrı’sı” olduğunu öğrettiğinde ve “Ebedi Tanrı sizin sığınağınız, ve onun etekleri sonsuz kollarınızdır” dediğinde, onların kutsallık kavramını genişletmeye çalışmıştı. Musa, Yahveh’in sözleşmeye sadık bir Tanrı olduğunu öğretti; ve Musa şu ifadede bulunmuştu: o “sizi ne yalnız bırakacak, size ne zarar verecek ne de atalarınızın sözleşmesini unutacaktır, çünkü Koruyucu sizi derinden sevmektedir, ve atalarına içtiği andı unutmayacaktır.”
\vs p096 4:7 Musa; Yahveh’i “yaptığı her şeyde adil ve doğru olan, hiçbir kötülüğü barındırmayan gerçekliğin Tanrısı” olarak sunduğunda, onu yüce bir İlahiyat’ın soyluluğuna yükseltmede kahramanca bir çaba harcamıştı. Ancak yine de, bu yüceltilmiş öğretiye rağmen, takipçilerinin sınırlı anlayışı; kızgınlığın, öfkenin ve şiddetin uyarımlarına bağlı oluşuna ek olarak, hatta, intikamcı ve insanın davranışından kolayca etkilenen bir biçimde insan suretinde Tanrı’dan bahsetmeyi gerekli kılmıştı.
\vs p096 4:8 Musa’nın öğretileri altında bu kabile doğa tanrısı Yahveh; el değmemiş arazilerden, tüm toplulukların Tanrısı şeklinde yakın bir zamanda düşünüleceği yer olan sürgüne kadar bile kendilerini takip etmiş, İsrail’in Koruyucu Tanrısı haline gelmişti. Museviler’i Babil’de köleleştiren daha sonraki esaret nihai olarak, tüm milletlerin Tanrısı’nın tek\hyp{}tanrı niteliğini alacak şekilde Yahveh kavramsallaşmasının evirilmesinin önünü açmıştı.
\vs p096 4:9 İbraniler’in dini tarihinin en benzersiz ve şaşırtıcı özelliği; İlahiyat’ın kavramsallaşmasının, Horeb Dağı’nın ilkel tanrısından başlayarak birbirini takip eden ruhsal önderlerinin öğretileri vasıtasıyla sevgi dolu ve bağışlayıcı Yaratan Yaratıcı’nın muhteşem kavramsallaşmasını duyurmuş İşaya’nın bu İlahiyat öğretileri içinde temsil edilen gelişimin en yüksek düzeyine kadar ki devamlı evrimle ilgilidir.
\usection{5.\bibnobreakspace Musa’nın Öğretileri}
\vs p096 5:1 Musa; bir askeri önder, toplumsal düzenleyici ve dini önderin olağanüstü bir birleşimiydi. O, Maçiventa ve İsa dönemleri arasındaki en önemli birey ve dünya öğretmeniydi. Musa İsrail’de, kaydı tutulmamış birçok köklü değişikliği getirmeyi denedi. Bir insanın yaşamı boyunca o, İbraniler olarak adlandırılan topluluğun çok dilli kalabalığını kölelikten ve medeni olmayan bir biçimde başıboş dolanmadan kurtarmıştı; bunun yanı sıra o, bir milletin gelecekteki doğumu ve bir ırkın devamlılığı için temel atmıştı.
\vs p096 5:2 İbraniler’in büyük göç zamanında hiçbir yazılı dile sahip olmayışları nedeniyle Musa’nın gerçekleştirdiği büyük şeylerin çok azı kayıtlarda bulunmaktadır. Musa’nın dönemine ve onun yaptıklarına dair tutulan kayıtlar, büyük önderin ölümünden bin yıldan daha fazla bir süre boyunca hala varlığını sürdürmüş tarihsel anlatılardan elde edilmişti.
\vs p096 5:3 Mısırlılar’ın ve çevredeki Levant kabilelerin dini üzerinde ve ötesinde Musa’nın gerçekleştirdiği ilerlemelerden birçoğunun temeli, Melçizedek dönemine dair Ken tarihsel anlatımlarıydı. İbrahim ve onun çağdaşlarına olan Maçiventa öğretisi olmasaydı, İbraniler ümitsiz karanlık içinde Mısır’dan hiçbir zaman çıkamayacaktı. Musa ve onun kayınpederi Jethro, Melçizedek dönemine ait geriye kalan tarihsel anlatımları bir araya topladı; ve Mısırlılar’ın öğrenimine sunulan bu öğretiler Musa’ya, İsrail topluluklarının gelişmiş dini ve ayininin yaratımında rehberlik etmişti. Musa bir düzenleyiciydi; o, Mısır ve Filistin’in din ve adetlerinde bulunanların en iyisini seçti; ve bu uygulamaları Melçizedek öğretilerine ait tarihsel anlatımlarla birleştirerek ibadetin İbrani tören düzenini örgütledi.
\vs p096 5:4 Musa, Yazgı’ya inanan biriydi; onun düşünceleri öncesinden, Nil ve doğanın diğer güçlerinin üzerinde doğaüstü bir denetimin varlığına dair Mısır savlarıyla bütünüyle etkilenmişti. Musa Tanrı’ya dair muhteşem bir kavrayışa sahipti; ancak Tanrı’ya itaat etmeleri halinde olacaklar hususunda Museviler’e şunları öğrettiğinde tamamen içtendi: “O sizi derinden sevecek, kutsayacak ve sizi çoğaltacak. O, rahimlerinizin meyvesini ve --- mısır, şarap, yağ ve sürüleriniz olarak --- topraklarınızın meyvesini çoğaltacaktır. Siz tüm insanların üstünde gelişecek olup, sizin Tanrı’nız olan Koruyucu tüm hastalıklarınızdan sizi kurtarıp, Mısır’ın kötü hastalıklarından hiçbirini üzerinize salmayacaktır.” Musa şunu bile söylemişti: “Sizin Tanrı’nız olan Koruyucu’yu hatırlayın, çünkü refahı elde etmek için size güç veren O’dur.” “Birçok millete ödünç vermelisiniz, ama kimseden ödünç almamalısınız. Birçok millet üstünde hâkimiyet kurmalısınız, ama onlar sizin üzerinizde hâkimiyet kurmamalıdır.”
\vs p096 5:5 Ancak Musa’nın bu büyük aklının; cahil ve okuma yazma bilmeyen İbraniler’in kavrayışına, En Yüksek unsur olan El Elyon’a dair yüce kavramsallaşmasını uyumlu hale getirmeye çalışmasını izlemek gerçekten de acınası durumdu. Bu bir araya getirdiği önderlere o şunu haykırmıştı: “Sizin Tanrınız olan Koruyucu tek bir Tanrı’dır; ondan başka hiç kimse yoktur”; bunun yanında karışık kalabalığa şunu duyurmuştu: “Tüm diğer tanrılar içinde kim sizin Tanrı’nız gibidir?” Musa şu ifadeleri duyurarak putlaşmış şeylere ve putlara tapınmaya karşı cesur ve kısmi bir biçimde başarı olan karşı duruşta bulunmuştu: “Ateşin ortasında Horeb’de sizin Tanrınız’ın sizinle konuştuğu güne benzer hiçbir şeyi görmediniz.” O aynı zamanda, her türde resim yapmayı yasaklamıştı.
\vs p096 5:6 Musa; Tanrı’nın adaleti korkusuyla insanlarının duydukları dehşetle gelen saygısını tercih eden bir biçimde şu ifadelerde bulunurlar, Yahveh’in merhametini duyurmaktan endişe etmişti: “Sizin Tanrı’nız olan Koruyucu Tanrılar’ın Tanrısı, Koruyucuların Koruyucusu, büyük bir Tanrı, insanlara acımayan kudretli ve korkunç bir Tanrı’dır.” Ve yine Musa, “sizin Tanrı’nız ona itaat etmediğinizde öldürmekte, ona itaat ettiğinizde iyileştirmekte ve yaşam vermektedir” duyurusunda bulunduğunda sorunlu kavimler üzerinde denetim sağlamayı amaçlamıştı. Ancak Musa bu kabilelere, yalnızca “tüm emirlerine uyulduğunda ve yönergeleri takip edildiğinde” onların Tanrı’nın seçilmiş insanları haline geleceklerini öğretmişti.
\vs p096 5:7 Tanrı’nın merhametinin çok azı, bu öncül dönemlerde İbraniler’e öğretilmişti. Onlar Tanrı’yı “Her Şeye Gücü Yeten” olarak öğrenmişti; “Koruyucu, düşmanlarını parçalara ayıran, güç bakımından muazzam nitelikli savaşların Tanrı’sı olarak bir savaş adamıdır.” “Siz Tanrı’nız olan Koruyucu, sizleri kurtarmak için savaş çadırlarının ortasında dolaşmaktadır.” İsrailoğulları Tanrı’larını; kendilerini derini seven ama aynı zamanda “Firavun’un kalbini mühürleyen” ve “düşmanlarını lanetleyen” biri olarak düşündüler.
\vs p096 5:8 Musa İsrail çocuklarına kâinatsal ve yardımsever bir İlahiyat’ın kısa süreli bakışlarını sunarken, bütünü itibariyle onların gündelik Yahveh kavramsallaşması, çevredeki toplulukların kabile tanrılarından yalnızca biraz daha gelişmiş olan bir Tanrı’ya ait kavramsallaşmaydı. Onların Tanrı kavramsallaşması ilkel, gelişmemiş ve insansıydı; Musa öldüğünde bu Bedevi kabileleri hızlı bir biçimde, sahip oldukları eski Horeb ve çöl tanrılarına ait yarı\hyp{}barbar düşüncelerine geri dönmüşlerdi. Musa’nın zaman zaman önderlerine sunduğu Tanrı’nın bu gelişmiş ve daha yüce görüşü yakın bir süre içinde ortadan kaybolmuştu; bunun karşısında bahse konu insanların çok büyük bir kısmı, Yahveh’e dair Filistinli sürü sahiplerinin simgesi olan putlaşmış altın buzağılarına yapılan ibadete geri dönmüşlerdi.
\vs p096 5:9 Musa İbraniler’in yönetimini Yeşu’ya devrettiğinde o hali hazırda; İbrahim, Nahor, Lut ve diğer ilgili kabilelerin kökenleri ortak soylarının binlercesini bir araya getirmiş, ve onları, kırsal kahramanlardan meydana gelmiş kendi kendilerini idame eden ve kısmi olarak kendilerini tek başına düzene sokabilen bir millet haline getirerek bütüncül denetimi sağlamıştı.
\usection{6.\bibnobreakspace Musa’nın Ölümünden Sonra Tanrı Kavramsallaşması}
\vs p096 6:1 Musa’nın ölümü üzerine Yahveh’e dair onun yüce kavramsallaşması hızlı bir biçimde kötüleşti. Yeşu ve İsrail’in diğer önderleri tamamiyle bilge, yardımsever ve her şeye gücü yeten Tanrı’ya dair Musasal tarihsel anlatımlara sığınmaya devem etti; ancak halk hızlı bir biçimde Yahveh’e dair daha eski çöl düşüncesine geri döndü. Ve İlahiyat’ın kavramsallaşmasının bu geriye doğru gerçekleşen kayışı, sözde Hâkimler olarak adlandırılan çeşitli kabile şeyhlerinin birbirlerini takip eden yönetimi altında artarak devam etti.
\vs p096 6:2 Musa’nın olağanüstü kişiliğinin büyüsü, Tanrı’nın sürekli artan bir biçimde genişleyen kavramsallaşması için ilhamı takiplerinin kalplerinde canlı tutmuştu; ancak onlar bir kez Filistin’in verimli arazilerine ulaştığında, hızlı bir biçimde göçebesel sürü sahipleri konumundan yerleşik ve bir ölçüde yatışmış çiftçilere evirilmişti. Ve yaşam uygulamalarının bu evrimi ve dini bakış açısındaki bu değişiklik, Yahveh olarak Tanrıları’nın doğasına dair kavramsallaşmalarının niteliğinde belli bir değişikliği zorunlu kıldı. Sina’nın katı, kaba, zorlayıcı ve oldukça sinirli çöl tanrısından daha sonra yeni yeni ortaya çıkan bir sevgi, adalet ve merhamet Tanrısı’nın kavramsallaşmasına olan başkalaşımın başlangıcı boyunca İbraniler Musa’nın yüce öğretilerini neredeyse tamamen kaybetmişlerdi. Onlar, tektanrıcılığa dair kavramsallaşmanın bütününü kaybetmeye yaklaşmışlardı; onlar, her şeyin Yaratıcısı’na ait bir bahşedilme Evladı’nın bedene bürünme dönemine kadar tek Tanrı’ya dair Melçizedek öğretisini muhafaza edebilecek topluluk olarak, Urantia’nın ruhsal evriminde hayati bir halka olarak görev yapacak topluluk haline gelme imkânını neredeyse kaybettiler.
\vs p096 6:3 Umutsuz bir biçimde Yeşu, şu duyuruya yol açan bir biçimde, kabile üyelerinin akıllarında yüce bir Yahveh’in kavramsallaşmasına bağlı kalmayı amaçlamıştı: “Geçmişte Musa ile birlikte olduğum gibi, sizlerle birlikte olacağım; Ne sizi hayal kırıklığına uğratacağım, ne de sizi yalnız bırakacağım.” Yeşu, eski ve özgün dinlerine inanmaya gönüllü olan ancak inanç ve doğruluğun dininde ilerlemeye gönülsüz olan, inkâr içindeki insanlarına katı bir müjdeyi yoğurmayı gerekli gördü. Yeşu’nun öğretisinin zorunluluğu şu oldu: “Yahveh kutsal bir Tanrı’dır; o kıskanç bir Tanrı’dır; o ne karşı geldiğiniz şeyleri ne de günahlarınızı affedecektir.” Bu çağın en yüksek kavramsallaşması Yahveh’i, bir “gücün, yargının ve adaletin Tanrı’sı” olarak resmetti.
\vs p096 6:4 Ancak bu karanlık çağda bile zaman zaman, yalnız bir öğretmen çıkıp Musasal şu kutsallık kavramsallaşmasını duyuran bir biçimde ortaya çıkmaktaydı: “Siz ahlaksızlığın çocukları, sizler Koruyucu’ya hizmet edemezsiniz, çünkü o kutsal bir Tanrı’dır.” “Fani insan Tanrı’dan daha adil olabilir mi? bir insan Yaratanı’ndan daha saf olabilir mi?” “Aramayla Tanrı’yı bulabilir misiniz? Her Şeye Gücü Yetenin kusursuz olduğunu bulabilir misiniz? Bakın, Tanrı muhteşemdir, ve biz onu bilmemekteyiz. Her Şeye Gücü Yeten’e dokunuyoruz, ama biz onu bulamıyoruz.”
\usection{7.\bibnobreakspace Eyüp’ün Kitabı ve Mezmurları}
\vs p096 7:1 Şeyhlerin ve din\hyp{}adamlarının önderliği altında İbraniler, Filistin’de çok sıkı olmayan bir biçimde yerleşmişlerdi. Ancak yakın bir zaman içinde onlar, acınası cahillikteki çöl inanışlarına geri dönüp, daha az gelişmiş Kenani dinsel uygulamalarıyla kirlenmiş hale gelmişlerdi. Onlar puta inanır ve sınırlandırılmamış cinsel ilişkilere düşer konuma geldiler; ve onların İlahiyat’a dair düşüncesi, hayatta kalmış belirli Salem toplulukları tarafından sürdürülen, ve bazı Memurlar’da ve Eyüp’ün Kitabı’nda kaydedilmiş olan, Mısırlı ve Mezopotamyalı Tanrı kavramsallaşmalarının çok altında kalmışlardı.
\vs p096 7:2 Mezmurlar, yirmi veya yirminden daha fazla yazarın bir çalışmasıdır; onların çoğu Mısırlı ve Mezopotamyalı öğretmenler tarafından yazılmıştı. Bu dönemler boyunca Levant doğa tanrılarına ibadet ederken, En Yüksek Unsur olarak El Elyon’un yüceliğine inanmış epeyce sayıda kişi bulunmaktaydı.
\vs p096 7:3 Dini yazıtların hiçbir derlemesi Mezmurların Kitabı kadar büyük bir bağlığa ve Tanrı’ya dair ilham veren düşüncelere yer vermemektedir. Ve, ibadetsel edebiyatın bu muhteşem derlemesi incelendiğinde, başka hiçbir bir derlemenin tekil olarak bu kadar uzun bir sürelik zamanı kapsamadığı akılda tutularak yüceltme ve hayranlığın her ayrı ilahisinin kaynağı ve tarihsel sırasına önem vermek oldukça yararlı olacaktır. Bu Mezmurlar Kitabı; Levant boyunca Salem dinine inananlar tarafından düşünülmüş Tanrı’nın çeşitlilik gösteren kavramsallaşmalarının kaydı olup, Amenemope’den İşaya’ya kadar bütün dönemi içine almaktadır. Mezmurlar’da Tanrı; bir kabile ilahiyatına ait ilkel düşünceden, içinde Yahveh’in sevgi dolu bir idareci ve bağışlayıcı Yaratıcı olarak resmedildiği daha sonraki İbraniler’in oldukça genişlemiş idealine kadar kavramsallaşmanın tüm fazları içinde tasvir edilmektedir.
\vs p096 7:4 Ve böyle değerlendirildiğinde Mezmurlar’ın bu topluluğu; insan tarafından yirminci yüzyıla kadar bir araya getirilmiş bağlılıksal eğilimlerinin en değerli ve en yardımcı derlemesini oluşturmaktadır. İlahilerin bu derlemesinin ibadetsel ruhaniyeti, dünyanın tüm diğer kutsal kitaplarınınkini aşmaktadır.
\vs p096 7:5 Eyüp’ün kitabındaki İlahiyat’ın çok renkli resmi, üç yüz yıldan daha fazla bir süreye yayılmış bir biçimde Mezopotamya’lı dini liderlerin yirmiden fazlasının bir ürünüydü. Ve Mezopotamya inanışlarının bu derlemesindeki kutsallığın yüce kavramsallaşmasını okuduğunuzda, gerçek bir Tanrı düşüncesinin Filistin’in karanlık dönemleri boyunca Kelda’nın Ur mahallesinde en iyi şekilde muhafaza edilmiş olduğunu anlayacaksınız.
\vs p096 7:6 Filistin’de Tanrı’nın her şeye nüfuz edişine dair bilgelik sıklıkla kavranılmıştı; ancak nadiren onun derin sevgisi ve merhameti anlaşılmıştı. Bu dönemlerin Yahveh’i “düşmanlarının ruhlarını ele geçirmek için kötü ruhaniyetler göndermekteydi”; o kendi, itaatkâr çocuklarının refah düzeyini yükseltirken, tüm diğerlerini lanetleyip, onları acımasız kararlarla cezalandırmaktaydı. “O kurnazın emellerini boşa çıkarır; düzenbazın hilesini bozar.”
\vs p096 7:7 Sadece Ur’da, Tanrı’nın merhametini duyurmak için şunu söyleyen bir ses ortaya çıktı: “O Tanrı’ya dua etmeli, lütfu ondan bulmalı ve yüzünü neşeli görmelidir; çünkü Tanrı insana kutsal doğruluğu verecektir.” Böylelikle Ur’dan, inanç vasıtasıyla gelen kutsal lütuf olarak kurtuluş duyurulmuştur: “O, tövbekâr karşısında merhamet sahibi olup ‘Onun kuyudan aşağı düşmesine engel ol, çünkü ben bir kefaret buldum’ der. Eğer herhangi biri ‘Ben günah işledim, doğru olana karşı geldim, bana hiçbir yararı olmadı’ derse, Tanrı onun ruhunun kuyudan aşağı düşmesine engel olacak, ve o kişi ışığı görecektir.” Melçizedek döneminden beri Levant; Ur’un tanrı\hyp{}elçisi ve, Mezopotamya’da bir zamanlar Melçizedek'in küçük halkı olan topluluğun kalıntısı olan Salem inananlarının din\hyp{}adamı Elihu’nun bu olağanüstü öğretisi şeklinde, insan kurtuluşunun bu türden kesin ve neşeli iletisini duymamıştı.
\vs p096 7:8 Ve böylece Mezopotamya’da Salem din\hyp{}yayıcılarının geride kalanları; Yahveh kavramsallaşması evriminin doruk noktası olan her şeyin Kâinatsal ve Yaratan Yaratıcısı’na dair en yüksek düşünceye erişene kadar, kavram kavram, inşa ederlerken hiçbir zaman durmamış İsrail öğretmenlerinin bu uzun kolunun ilki ortaya çıkana dek, İbrani topluluklarının düzensizlik dönemi boyunca gerçekliğin ışığını muhafaza etmişlerdi.
\vs p096 7:9 [Nebadon’un bir Melçizedek unsuru tarafından sunulmuştur.]
