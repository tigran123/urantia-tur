\upaper{63}{İlk İnsan Ailesi}
\vs p063 0:1 Urantia; ikizler olarak ilk iki insan varlığı on bir yaşında iken, ve onlar mevcut insan varlıklarının ikinci neslinin ilk doğan üyelerinin ebeveynleri olmadan önce yerleşik bir dünya olarak kaydedilmiştir. Ve Salvington’dan gelen baş melek iletisi, resmi gezegensel tanınmanın bu durumu üzerine, şu cümleler ile sonlanmıştır:
\vs p063 0:2 “İnsan\hyp{}aklı Satania’nın 606’ncı dünyası üzerinde ortaya çıkmıştır, ve yeni ırkın bu ebeveynleri \bibemph{Andon} ve \bibemph{Fonta} olarak adlandırılmalıdır. Ve bütün baş melekler, bu yaratılmışların Kâinatın Yaratıcısı’nın ikamet eden ruhaniyet hediyesini ile hızlı bir biçimde kazanabilmesini umut etsinler.”
\vs p063 0:3 Andon; “insanın kusursuzluk açlığını sergileyen ilk Yaratıcısal yaratılmış” anlamına gelen Nebadon ismidir. Fonta’nın anlamı ise “insanın kusursuzluk açlığını sergileyen ilk Evlatsal yaratılmıştır.” Andon ve Fonta, Düşünce Düzenleyicileri ile bütünleşme zamanına kadar kendilerine bu isimlerinin bahşedildiğini hiçbir şekilde bilmemekteydiler. Urantia üzerindeki fani ikametleri boyunca onlar birbirlerini, Sonta\hyp{}an ve Sonta\hyp{}en olarak çağırdılar; Sonta\hyp{}an “annenin sevgilisi” ve Sonta\hyp{}en ise “babanın sevgilisi” anlamına gelmekteydi. Onlar birbirlerine bu isimleri vermiş olup, bunların anlamları karşılıklı saygı ve sevgiyi göstermesi bakımından önemlidir.
\usection{1.\bibnobreakspace Andon ve Fonta}
\vs p063 1:1 Birçok açıdan Andon ve Fonta, dünya üstünde bahse konu zaman zarfına kadar yaşamış olan insan varlıklarının en dikkate değer çiftiydi. Bu muhteşem çift, insan varlıklarının hepsinin mevcut ebeveynleri olarak; birincil soylarının çoğundan her bakımdan üstün olup, birincil ve ikincil akrabalarının tümünden köklü bir biçimde farklıydılar.
\vs p063 1:2 Her ne kadar bu ilk insan çiftinin ebeveynleri; taş atmayı ve kavgada sopaları kullanmayı ilk öğrenen topluluk olarak, daha ussal üyeler arasında olsalar da, kabilelerinin ortalama niteliklerinden görünüşte çok az farka sahip unsurlardı. Onlar aynı zamanda taş, çakmaktaşı ve kemiğin sivri uçlarından faydalanmışlardır.
\vs p063 1:3 Ebeveynleri ile yaşarken bile, bir sopanın ucuna çakmaktaşının sivri bir parçasını hayvan tendonlarını kullanıp bağlamışlardır; ve bir düzineden daha az olmayan durumda bu türden bir silahı, kendi yaşamını ve keşif gezintilerinde kendisini hiç yalnız bırakmayan onunla eşit derecede bulunan maceraperest ve araştırmacı kız kardeşininkini kurtararak yararlı bir biçimde kullanmıştır.
\vs p063 1:4 Andon ve Fonta’nın Primat kabilelerinden ayrılma kararı, maymun kabilelerinin alt düzey akla sahip kuzenleri ile çiftleşmekten vazgeçen daha sonraki soylarının birçoğunu niteleyen temel ussun çok üzerinde bir akıl niteliğini simgelemektedir. Ancak onların hayvanlardan biraz daha fazlası olduklarına dair belirsiz hisleri, kişiliğe sahip olmalarından kaynaklanmış olup ve Düşünce Düzenleyicisi’nin ikamet eden mevcudiyeti tarafından çoğalmıştır.
\usection{2.\bibnobreakspace İkizlerin Kaçışı}
\vs p063 2:1 Andon ve Fonta kuzeye doğru kaçmaya karar verince, babalarını ve öncül ailelerini özellikle gücendirme endişesi biçiminde, kısa bir süreliğine korkularını yendiler. Onlar; akrabaları tarafından pusuya düşürülmeyi öngörmüş olup, böylelikle ölümlerini kıskanç kabile üyelerinin ellerinde tatma olasılığının farkına varmışlardır. Genç bireyler olarak ikizler, vakitlerinin büyük bir kısmını beraber geçirmişlerdir; bu nedenden dolayı onlar, Primat kabilesinin hayvan kuzenleri arasında fazlasıyla gözde bireyler hiçbir zaman olmamışlardır. Buna ek olarak onlar, ayrı ve çok üstün bir ağaç evi inşa etmek kendilerine duyulan saygınlığı arttırmamıştır.
\vs p063 2:2 Ve ağaç tepeleri arasında bu yeni ev içerisinde bir gece şiddetli bir kasırgayla uyandıktan sonra, birbirlerini korkuyla ve sevgiyle sarmalarken, kabile yaşamından ve ağaç tepesi evinden ayrılmaları gerektiği konusunda nihai ve bütünsel olarak karar verdiler.
\vs p063 2:3 Onlar, kuzey doğrultusunda bulunan kendilerine ait ilkel bir ağaç tepesi barınağını yarım gün içerisinde çoktan hazırlamışlardı. Bu barınak, ağaç ormanlarından ayrı kalacakları ilk gün için kendilerine ait gizli ve güvenli gizlenme yeriydi. Her ne kadar ikizler; Primatlar’a ait gece vakti yerde bulunmanın ölümcü korkusunu taşısalar da, kuzeye doğru olan yolculuklarında havanın kararmasından önce kısa bir süreliğine ilerlediler. Dolunay zamanında bile bu türden bir gece gezintisine girişmek onların olağan dışı cesaretini gerektirirken, onlar yokluklarının belli olmayacağının ve kabile üyelerine ek olarak akrabaları tarafından takip edilmeyeceklerinin doğru bir biçiminde çıkarımında bulundular. Ve onlar gece yarınsından kısa bir süre sonra daha öncesinden hazırlamış oldukları buluşma yerlerine güvenli bir biçimde ulaştılar.
\vs p063 2:4 Kuzeye doğru gerçekleştirdikleri seyahatlerinde onlar, açıkta bulunan bir çakmaktaşı parçası keşfettiler; ve çeşitli birçok amaç için elverişli bir şekle sahip birçok taş parçası bularak bunları gelecekte kullanmak amacıyla biriktirdiler. Belirli amaçlar için hazır hale gelmesi amacıyla bu çakmaktaşlarını yontma girişiminde Andon; onların kıvılcım özelliğini keşfetmiş olup, ateş yakma fikrini algıladı. Ancak bu zaman zarfında iklim hâlihazırda elverişli olduğu ve çok az bir ateş ihtiyacı bulunduğu için bu düşünce onun aklında çok derin bir yer teşkil etmedi.
\vs p063 2:5 Ancak sonbahar güneşi gökyüzünde alçalmaktaydı, ve onlar kuzeye doğru ilerlerken geceler gittikçe soğuk bir hale gelmekteydi. Hâlihazırda onlar hayvan derilerini ısınmak için kullanmak zorunda kalmışlardı. Ulaşacakları yerleşkeden bir ay öncesinde Andon eşine, çakmaktaşından ateş yakabileceği düşüncesini ifade etti. Onlar, çakmaktaşı kıvılcımını bir ateş yaratmak için kullanma üzerinde iki ay uğraş verdiler; ancak onlar yalnızca başarısızlıktan başka bir şey elde edemediler. Her gün bu çift, çakmaktaşlarını çakmakta ve odunlarını ateşe vermeye çalışmaktaydı. En sonunda bir gece güneş batımı zamanında ateş yakma tekniğinin sırrına Fonta, terk edilmiş bir kuş yuvasını elde etmek için yakın bir ağaca çıkışı esnasında erişti. Yuva kuru ve oldukça yanıcıydı, bunun sonucunda kıvılcımın buraya düştüğü andan itibaren bütüncül bir ateş parıltısı önünde belirdi. Onlar başarılarına o kadar şaşırmışlar ve bir o kadar afallamışlardı ki, neredeyse ateşi kaybetmişlerdi; ancak onlar ateşi, elverişli bir sıvıyı ekleyerek kurtarıp, bunun sonrasında tüm insanlığın ebeveynleri olarak gerçekleştirdikleri çıraların ilk arayışına giriştiler.
\vs p063 2:6 Bu durum, kısa fakat önemli yaşamlarında en neşe dolu anlardan bir tanesiydi. Bütün gece boyunca onlar; iklime meydan okumalarını ve böylelikle güney bölgesindeki hayvan akrabalarından sonsuza kadar bağımsız olmalarını kendileri için mümkün kılan bir keşfi gerçekleştirdiklerinin farkına henüz belirgin olmayan bir biçimde vararak, ateşlerinin yanışını oturup izlediler. Üç günlük dinlenişlerinden ve ateş eğlencelerinden sonra onlar, yolculuklarına devam ettiler.
\vs p063 2:7 Andon’un Primat ataları, yıldırım sonucunda yanan ateşi tazelemeyi sıklıkla gerçekleştirmişlerdi; ancak bu zaman zarfından önce dünya üzerindeki yaratılmışların hiçbiri, irade dâhilinde ateş yakmanın yöntemine sahip değildi. Ancak bu gelişmeler, ikizlerin kuş yuvalarına ek olarak kuru yosunların ve diğer maddelerin ateşi yakacağını öğrenmesinden uzun bir zaman önce gerçekleşmiştir.
\usection{3.\bibnobreakspace Andon’un Ailesi}
\vs p063 3:1 İlk çocuklarının doğuş anı ile ikizlerin evden ayrıldıkları gece arasında neredeyse iki yıl geçmiştir. Onlar bu çocuğun ismini Sontad koymuşlardır; ve Sontad, doğum anında koruyucu örtüler içinde sarınan Urantia üzerinde doğmuş ilk yaratılmıştı. İnsan ırkı ilerleyiş sürecine çoktan başlamıştı; ve bu yeni evrim ile birlikte, daha saf bir biçimde hayvan olan türlerin aksine, ussal düzeye ait aklın ilerleyici gelişimini niteleyen artan bir şekilde zayıf doğan bebeklere gösterilen yerinde bir ilgi açığa çıktı.
\vs p063 3:2 Andon ve Fonta toplam on dokuz çocuğa sahipti; ve onlar, neredeyse elli torun ve altı büyük torundan oluşan bir aile birlikteliğini memnuniyetle deneyimlediler. Aile, birbirine bitişik dört kaya sığınağı veya diğer bir değişle yarı\hyp{}mağara için konumlanmıştı; bunların üçü birbirine, Andon’un çocukları tarafından geliştirilen çakmaktaşı aletleri ile yumuşak kireçtaşının kazılmasıyla oluşturulmuş koridorlar ile bağlıydı.
\vs p063 3:3 Bu öncül Andonsal unsurları, oldukça belirgin bir kavim ruhaniyetini sergilediler; onlar, topluluklar içinde avlanıp, ev yerleşkesinden hiçbir zaman çok uzağa gitmediler. Onlar; kendilerinin tecrit edilmiş ve benzersiz yaşayan varlıklar olduklarının, bu nedenle ayrılmalarından kaçınmaları gerektiğinin farkına varmış bir görünüme sahip oldular. İçten akrabalığın bu duygusu kuşkusuz bir biçimde, emir\hyp{}yardımcı ruhaniyetlerinin gelişmiş akıl hizmetinin sayesinde gerçekleşmiştir.
\vs p063 3:4 Andon and Fonta, sürekli bir biçimde kavimlerini büyütmek ve geliştirmek için çabaladıklar. Onlar; üzerlerinde sallanan bir kayanın bir deprem sonucunda düşüp ölümlerine sebep oldukları zaman zarfına kadar, kırk iki yıl yaşadılar. Onların beş çocuğu ve on bir torunu onlar ile birlikte hayatlarını kaybetti; ve neredeyse bu kadar soy üyeleri ciddi bir biçimde yaralandılar.
\vs p063 3:5 Ebeveynlerinin ölümü üzerine Sontad, ciddi bir biçimde yaralanan ayağına rağmen derhal kavminin önderliğini üstlendi; ve o, en büyük kız kardeşi olan karısı ona yetkin bir biçimde yardımda bulundu. Onların ilk görevi; ölü ebeveynlerini, erkek ve kız kardeşlerine ek olarak çocuklarını etkin bir biçimde defnetmek için kayaları onların etrafında çitlemek oldu. Ölümden sonra kurtuluşlarına dair onların düşünceleri, hayali ve çeşitli düş dünyasından büyük ölçüde kaynağını olan bir biçimde oldukça belirsiz ve kesinlikten uzaktı.
\vs p063 3:6 Andon ve Fonta’nın bu ailesi; kavimin dağılmasına sebep olan yiyecek sıkıntısı ve toplumsal anlaşmazlıklar ile birleşen gelişmelerin baş gösterdiği zaman zarfı olan, yirminci neslin başlamasına kadar birbirine sıkıca bağlı bir biçimde yaşamıştır.
\usection{4.\bibnobreakspace Andonsal Kavimler}
\vs p063 4:1 Andonsal unsurlar olarak ilkel insan, siyah gözlere ek olarak sarı ve kırmızının bir karışımına benzer bir biçimde esmer ten rengine sahiplerdi. Melanin, insan varlıklarının derilerinde bulunan bir renk verici özdür. Bu öz, kaynak Andonsal deri renklendiricisidir. Olağan dış görünüş ve ten rengi bakımından bu öncül Andonsal unsurlar, yaşayan insan varlıklarının herhangi birine kıyasla daha yakın bir biçimde çağdaş Eskimolar’a benzemekteydi. Onlar, soğuğa karşı bir korunum olarak hayvan derilerinden yararlanan ilk yaratılmışlardı; onlar, çağdaş insanlara kıyasla bedenlerinde biraz daha fazla kıla sahipti.
\vs p063 4:2 Bu öncül insanların hayvan atalarına ait kabile yaşamı, sayısız toplumsal davranışın başlangıcını simgelemiştir; ve bu varlıkların genişleyen duyguları ve çoğalan beyin güçleri ile orada, toplumsal örgütlenme içinde doğrudan bir gelişme ve yeni bir kavim iş bölümü gerçekleşmiştir. Onlar aşırı bir biçimde taklitsel varlıklardı; ancak oyun içgüdüsü yalnızca az bir biçimde gelişmişti; ve mizah anlayışından neredeyse bütünüyle yoksundular. İlkel insan ara sıra gülümsemekteydi; fakat o, derinden gelen bir kahkahaya hiçbir zaman sahip olmamıştı. Mizah, daha sonraki Âdem ırkının bir mirasıydı. Bu öncül insan varlıkları, daha sonraki evrimleşen fanilerin birçoğu gibi ne acıya oldukça duyarlı ne de olumsuz durumlara karşı oldukça tepki gösteren niteliğe sahipti. Çocuk doğumu, Fonta ve onun birincil nesli için ne acı veren ne de ıstırap çektirici bir deneyimdi.
\vs p063 4:3 Onlar muhteşem bir kavimdi. Erkekler, çiftleri ve doğumlarının güvenliği için kahramanca mücadele ederlerdi; kadınlar ise çocuklarına sevgi dolu bir biçimde bağlanmıştı. Ancak onların sevgi dolu doğaları bütünüyle birincil kavimleriyle sınırlıydı. Onlar ailelerine oldukça sadıklardı; çocuklarını korumak için bir an bile düşünmeden ölürlerdi; ancak onlar, torunları için dünyayı daha iyi bir hale getirmeye çalışmanın düşüncesini kavramaya yetkin değillerdi. Her ne kadar dinin doğuşu için hayati derece önemli olan duyguların hepsi hâlihazırda bu Urantia yerlileri içinde mevcut bulunmuşsa da, toplumsal fedakârlık henüz insan kalbinde doğmamıştı.
\vs p063 4:4 Bu öncül insanlar, akranları için etkileyici derecede bir sevgi beslemişlerdir; ve kesin bir içimde onlar, her ne kadar ilkel bir düzeyde bulunsa da, arkadaşlık düşüncesine sahip olmuşlardır. Aşağı düzeyde bulunan kabileler ile yaptıkları tekrar eden savaşlarında bu ilkel insanların bir tanesini; cesurca bir eliyle savaşırken diğer eliyle yaralanmış bir savaşçı yoldaşını korumak ve onun hayatını kurtarmak için çabalarken görmek, daha sonraki zamanların sıklıkla gözlenen bir olguydu. Sonraki evrimsel gelişmenin en soylu ve en yüksek insan niteliklerinden çoğu, etkileyici bir biçimde bu ilkel insanlarda geleceğin habercisi olarak sergilenmiştir.
\vs p063 4:5 Kökensel Andonsal kavim, yirmi yedinci nesle kadar önderliğin istikrarlı bir idaresine sahip olmuştur; bu süreçten sonra, Sontad’ın birincil doğumları arasında hiçbir erkek dünyaya gelmediği zaman, kavimin iki olası idarecisi üstünlük için kavgaya girişmişlerdir.
\vs p063 4:6 Andonsal kavimlerinin geniş ölçekli parçalanışından önce oldukça gelişmiş bir dil, iletişim için öncül çabaları sonucunda evrimleşmiştir. Bu dil büyümeye devam etmiştir; ve, bu etkin, oldukça hareketli ve maceraperest insanlar tarafından geliştirilen bir çevre için gerçekleştirilen icatlar ve uyarlamalar nedeniyle neredeyse günlük eklentiler bu dile kazandırılmaktaydı. Ve bu dil, renkli ırkların daha sonraki ortaya çıkışlarına kadar öncül insan ailesinin ana dili olarak Urantia’nın sözü haline gelmişti.
\vs p063 4:7 Zaman geçtikçe Andonsal kavimlerin nüfusu arttı, ve genişleyen ailelerin birbirleriyle olan ilişkileri gerginlikleri ve anlaşmazlıkları beraberinde getirdi. Sadece iki husus bu insanların akıllarında yer etmekteydi: yiyecek bulmak için avlanmak ve komşu kabileler tarafından uğradıkları, gerçeklik taşıyıp taşımamasından bağımsız olarak, adaletsizlik veya haksızlık karşısında öçlerini almaktı.
\vs p063 4:8 Aile anlaşmazlıkları arttı, kabile savaşları patlak verdi, ve ciddi kayıplar daha yetkin ve daha gelişmiş olan toplulukların en iyi nitelikleri arasında gözlenmekteydi. Bu kayıplardan bazıları onarılamaz nitelikte olanlardı; yetkinlik ve usun en değerli özelliklerinden bazıları sonsuza kadar dünya üzerinden yok oldu. Bu öncül ırk ve onun ilkel medeniyeti, kavimlerin bitmek tükenmek bilmeyen savaşları nedeniyle yok olmakla karşı karşıya kaldı.
\vs p063 4:9 Bu türden ilkel varlıkları barış içerisinde beraber yaşamayı arzulamaya ikna etmek imkânsızdır. İnsan kavgacı hayvanların soyundan gelmektedir; ve yakın bir biçimde bir araya geldiklerinde kültürsüz insanlar, birbirlerini kızdırıp gücendirmektedirler. Yaşam Taşıyıcıları, evrimsel yaratılmışlar arasında bu eğilimi bilmekteydi; ve bunun uyarınca onlar, en az üç ve daha sıklıkla gerçekleşen bir biçimde altı farklı ve ayrışmış ırka gelişen insan varlıkların nihai bölünmelerine dair karara varmaktadırlar.
\usection{5.\bibnobreakspace Andonsal Unsurlar'ın Dağılımı}
\vs p063 5:1 Öncül Andon ırkları, Asya’nın çok derinlerine girmediler; ve onlar ilk başta Afrika kıtasına gitmediler. Bahse konu zamanların coğrafyası, onların kuzeye doğru ilerleyişini sağlamıştır; ve bu insanlar, üçüncü buzul döneminin yavaşça ilerleyen buzulu tarafından engelleninceye kadar kuzeye doğru ilerlemeye devam ettiler.
\vs p063 5:2 Bu geniş buz tabakası Fransa ve Britanya Adaları’na ulaşmadan önce Andon ve Fonta’nın soyları; Avrupa üzerinden kuzeye doğru ilerleyip, Kuzey Denizi’nin bahse konu zamanlarda sıcak olan sularına açılan büyük nehirler boyunca sayıca binden daha fazla ayrı yerleşke kurdular.
\vs p063 5:3 Bu Andonsal kabileler, Fransa’nın öncül nehir sakinleriydi; onlar, binlerce yıl boyunca Somme nehri boyunca yaşamışlardı. Somme, bahse konu zaman zarfında bugünkü gibi denize dökülmekte olan bir biçimde, buzullar tarafından seviyesi değişmeyen bir nehirdir. Ve bu durum, Andonsal soylarına dair neden bu kadar çok kanıtın bu nehir vadisi uzantısı boyunca bulunmakta olduğunu açıklamaktadır.
\vs p063 5:4 Her ne kadar ağaç tepelerine hala kendilerini tehlike anında atsalar da, Urantia’nın bu yerlileri ağaç sakinleri değillerdi. Onlar düzenli bir biçimde, barındıkları yerleşkeye yaklaşmakta olan varlıkları görmeleri için iyi bir açı sağlayan ve onları hava olaylarından koruyan nehir boylarındaki asılı yamaçlara veya dağ eteği mağaralarına yerleşmişlerdir. Böylelikle onlar, dumandan çok rahatsız olmadan ateşlerinin sağladığı rahatlığı memnuniyetle deneyimlemişlerdir. Her ne kadar ilerleyen zamanlarda geç buz tabakaları daha güneye doğru ilerleyip sonraki soylarını mağaralara doğru itmişse de, onlar gerçek anlamıyla mağara sakinleri de değillerdi. Onlar, bir ormanın eşiğinde ve bir ırmağın kenarında konaklamayı tercih etmişlerdir.
\vs p063 5:5 Onlar çok öncesinden, özellikle seçilmiş korunaklı yerleşkelerini saklamakta dikkate değer bir zekâya sahip oldular; ve onlar, gece vakti içlerine kıvrılıp yatacakları kubbe biçimli kaya barakaları olarak uyku odaları inşa etmede büyük bir beceri gösterdiler. Bu türden bir barakaya giriş, bu girişin karşısına konumlandırılmış yuvarlanan bir kaya tarafından kapanmaktaydı; bu büyük kaya, tavan kayaları nihai olarak yerleştirilmeden önce sadece bu amaç için içeriye konulmuştu.
\vs p063 5:6 Andonsal unsurlar; korkusuz ve başarılı avcılar olup, ağaçların yaban dutları ve belirli meyveleri dışında ayrıcalıklı bir biçimde etten besinlerini sağlamaktaydılar. Andon taş baltasını icat ederken, onun soyundan gelen unsurlar ise öncül bir biçimde çubuk ve zıpkın fırlatımını etkin bir biçimde kullandılar. En sonunda alet yaratım aklı, el becerisi ile birlikte faaliyet göstermekteydi; ve bu öncül insanlar, çakmaktaşı aletlerine biçim vermede oldukça hünerli hale geldiler. Onlar uzak ve geniş alanlara, tıpkı çağdaş insanların dünyanın sonuna kadar altın, platin ve elmas peşinde seyahatlere çıkmaları gibi, çakmaktaşı aramak için yolculuklarda bulundular.
\vs p063 5:7 Ve birçok açıdan bu Andon kabileleri, her ne kadar ateş yakmanın çeşitli yöntemlerini sürekli bir biçimde yeniden keşfettilerse de, gerileyen soylarının yarın milyon yıldır erişemedikleri ussun bir seviyesini sergilediler.
\usection{6.\bibnobreakspace Onagar --- İlk Doğruluk Öğretmeni}
\vs p063 6:1 Andonsal kopuşlar genişlerken, Onagar’ın ortaya çıkışına kadar neredeyse on bin yıl boyunca kavimlerin kültürel ve ruhsal düzeyi gerilemiştir; bu kabilelerin önderliğini üstlenen Onagar, “insanlara ve hayvanlara Nefes Verici’ye” olan ibadete onların hepsini yönlendirerek ilk kez onlara barışı getirmiştir.
\vs p063 6:2 Andon’un felsefesi, büyük bir kafa karışıklığı içerisindeydi; o, şans eseri ateşi buluşundan kaynaklanan büyük rahatlık nedeniyle kıl payı farkla bir ateş tapıcısı olmaktan kurtulmuştu. Nedensellik, buna rağmen, kendi keşfinden güneşi daha üstün ve ısıya ek olarak ışığın daha merak uyandırıcı kaynak biçiminde tanımasına onu sevk etmiştir; ancak bu kavrayış derinsel değildi, ve bu nedenle o bir güneş tapıcısı olmayı başaramamıştır.
\vs p063 6:3 Andonsal unsurlar öncül olarak, --- gök gürültüsü, yıldırım, yağmur, kar, dolu ve buz gibi --- hava olaylarından duyulan bir korkuyu geliştirmişlerdir. Ancak açlık, bu öncül zamanların sürekli ortaya çıkan dürtüsüydü; ve onlar büyük oranda hayvanlardan beslendikleri için, hayvanlara yapılan ibadetin bir türünü nihai olarak geliştirdiler. Andon için büyük hayvanlar, yaratıcı kudret ve gücü elinde bulundurmanın simgeleriydi. Zaman zaman, bu büyük hayvanları ibadetin nesneleri olarak tanımlamak adet haline gelmişti. Bir özel hayvanın revaçta olduğu zamanlarda, onun dış hatları ilkel bir biçimde mağara duvarlarına resmedilirdi; ve daha sonra bu ilerleyen gelişme olarak sanatta gerçekleşirken, bu türden bir hayvan tanrısı çeşitli süslemeler üzerine kazınmıştır.
\vs p063 6:4 Çok öncül Andonsal topluluklar, kabilenin taptığı hayvan etini yemeden sakınmanın alışkanlığını getirdiler. Yakın bir süre zarfında, kabile gençlerinin akıllarını daha yerinde bir biçimde etkilemek için, bu tapılan hayvanların bir tanesine ait beden etrafında düzenlenen bir saygı töreni geliştirdiler; ve daha sonra, bu ilkel sergiler, kabile soylarının daha derin kurbanlık törenlerine varıncaya kadar gelişme gösterdi. Ve bu gelişme, kurbanların ibadetin bir parçası olarak ortaya çıkışını oluşturmaktaydı. Bu düşünce; İbrani geleneği içinde Musa tarafından ayrıntılı bir biçimde detaylandırılmış olup, “kan dökülmesi” kavramı tarafından günahın temizlenmesi savı şeklinde ilke olarak Aziz Paul tarafından korunmuştur.
\vs p063 6:5 Bu ilkel insan varlıklarının yaşamları içinde bu önemli şeyin yiyecek olduğu, onların büyük öğretmeni olan Onagar tarafından bu basit insanlara öğretilen dua ile sergilenmiştir. Ve bu dua şuydu:
\vs p063 6:6 “Sen Yaşam Nefesi, bu gün bize günlük yiyeceğimizi ver, buzun lanetinden bizleri koru, orman düşmanlarımızdan bizleri kurtar, ve bağışlama ile bizleri Büyük Ahiret’e kabul et.”
\vs p063 6:7 Onagar; Mezopotamya’nın güney kısmından kuzeye doğru olan seyahat rotasının batı dönüşünde bekleme yeri biçimindeki Oban olarak adlandırılan bir yerleşkede, bugünün Hazar Gölü bölgesinde tarihi Akdeniz’in kuzey kıyı yakasında yönetim merkezini idare etmiştir. Oban’dan, tek İlahiyat’a dair yeni savlarını ve Büyük Ahiret olarak adlandırdığı kavramsallaşmasını yaymak için uzak yerleşkelere eğitmenler göndermiştir. Onagar’ın bu elçileri, dünyanın ilk din elçileriydi; onlar aynı zamanda, yiyeceğin hazırlanışında ateşi en önce düzenli olarak kullanmaya başlayan ilk insan varlıklarıydı. Onlar eti, çubukların ucunda ve aynı zamanda sıcak kayalarda pişirdiler; daha sonra onlar, büyük parçalar halinde eti ateşte közlediler; ancak onların soyları neredeyse bütünüyle eti çiğ olarak yemeye geri döndü.
\vs p063 6:8 (M.S. 1934 yılına göre) Onagar 983.323 yıl önce doğmuş, ve altmış dokuz yaşına kadar yaşamıştır. Gezegensel Prens\hyp{}öncesi zamanının bu üstün aklı ve ruhani önderinin kazanımlarına dair kayıtlar, bu ilkel insanların gerçek bir toplum birlikteliğine olan örgütlenişine dair heyecan verici bir öyküdür. O, birçok bin boyunca takip eden nesiller tarafından erişilmeyecek etkin bir kabilesel hükümeti kurmuştur. Gezegensel Prens’in varışına kadar bu türden yüksek bir medeniyet dünya üzerinde bir daha gerçekleşmemiştir. Bu basit insanlar, gerçek fakat ilkel bir dine sahip oldular; ancak bu din, kötüleşen soyları tarafından ilerleyen süreçte kaybedilmiştir.
\vs p063 6:9 Her ne kadar Andon ve Fonta Düşünce Düzenleyicileri’ni birçok soyları gibi almış olsalar da, Onagar’ın zamanına kadar Düzenleyiciler ve koruyucu yüksek melekler geniş sayılarda Urantia’ya gelmemiştir. Bu zaman zarfı gerçekten ilkel insanın altın çağıydı.
\usection{7.\bibnobreakspace Andon ve Fonta’nın Kurtuluşu}
\vs p063 7:1 İnsan ırkının muhteşem kurucuları olan Andon ve Fonta, Gezegensel Prens’in varışı üzerine Urantia’nın yazgı döneminin kayıt altına alınması sürecinde tanınmışlardır; ve bu zaman aralığında onlar, Jerusem üzerinde vatandaşlık düzeyi ile malikâne dünyasının düzeni içerisinde yeniden dirilmişlerdir. Her ne kadar onların Urantia’ya olan dönüşlerine hiçbir zaman izin verilmemiş olsa da, kurdukları ırkın tarihinden haberdar haldedirler. Onlar, Caligastia ihanetinden derin bir biçimde üzüntüyü Âdemsel başarısızlık nedeniyle kederlenen bir biçimde hissettiler; ancak kendi dünyalarını Mikâil’in nihai bahşedilişinin ana mekânı olarak tercih ettiğinin duyurusunu aldıklarında fazlasıyla mutlu oldular.
\vs p063 7:2 Jerusem üzerinde Andon ve Fonta, Sontad’ı içine alan bir biçimde birkaç çocukları ile birlikte, Düşünce Düzenleyicileri ile bütünleştiler; ancak onların birincil soylarının bile büyük bir çoğunluğu yalnızca Ruhaniyet ile bütünleşmeyi elde etmişlerdir.
\vs p063 7:3 Jerusem’e varışlarından kısa bir süre sonra Andon ve Fonta, Sistem Egemeni’nden; Urantia’dan cennetsel âlemlere gelen kutsal yolcuları karşılayan morontia kişilikleri olarak hizmet etmek için ilk malikâne dünyasına geri dönüş izni almışlardır. Ve onlar, bu göreve süresiz olarak atanmışlardır. Onlar, Urantia’ya bu açığa çıkarışlar ile ilişkili olarak selam göndermeyi amaçladılar; ancak bu rica, bilge bir biçimde reddedilmiştir.
\vs p063 7:4 Ve böylece bu anlatım; tüm insanlığın benzersiz ebeveynlerine ait evrimin, yaşam mücadelelerin, ölümün ve ebedi kurtuluşunun hikâyesi olarak Urantia’nın tüm tarihi içinde en kahramansal ve en heyecan verici bölümün bir öyküsüdür.
\vs p063 7:5 [Urantia üzerinde ikamet eden bir Yaşam Taşıyıcısı tarafından sunulmuştur.]
