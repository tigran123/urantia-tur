\upaper{87}{Hayalet İnanışları}
\vs p087 0:1 Hayalet inanışları, kötü şansın tehditleri karşısında bir denge unsuru olarak gelişti; onun ilkel dini yükümlülükleri, kötü talih endişesi ve ölümden duyulan olağandışı korkunun ürünleriydi. Bu öncül dinlerin hiçbiri, İlahiyat’ın tanınmasıyla veya insan\hyp{}üstü olana karşı duyulan derin saygıyla iniltili değildi; onların ayinleri hayaletlerden kaçınma, onları kaçırma veya baskı altına alma biçiminde çoğunlukla olumsuz nitelikteydi. Hayalet inanışı niteliksel olarak, felakete karşı gerçekleştiren sigortadan daha fazla veya daha az değildi; onun, daha üstün ve gelecekteki geri dönüşler için gerçekleştirilen yatırım ile hiçbir ilişkisi bulunmamaktaydı.
\vs p087 0:2 İnsan, hayalet inanışıyla verilen uzun ve çetin bir mücadeleye sahip olmuştur. İnsan tarihi içinde hiçbir şey, insanın hayalet\hyp{}ruhaniyet korkusuna olan küçük düşürücü köleliğinin bu resminden daha fazla acıma duygusu verecek bir biçimde tasarlanmamıştır. Bu korkunun doğumuyla birlikte insanlık, dini evrimini yükseltmeye başlamıştı. İnsanın hayal gücü bireyin sahillerinden ayrılmış olup, gerçek bir Tanrı olarak doğru bir İlahiyat kavramına varıncaya kadar güvenli bir limanı tekrar bulamayacaktır.
\usection{1.\bibnobreakspace Hayalet Korkusu}
\vs p087 1:1 Ölümden, bedensel bünyeden diğer hayaletin özgür kalması anlamına geldiği için korku duyulmaktaydı. İlkçağ insanları, yeni bir hayalet ile uğraşma sıkıntısından kaçınma olarak ölümü engellemek için elinden gelen her şeyi yerine getirmişlerdi. Onlar, ölüm\hyp{}yerleşkesine yapacakları yolculuğu başlatan bir biçimde ölüm mahallini terk etmeleri için hayaletleri teşvik etmeden her zaman endişe duymuşlardı. Hayaletten en fazla; ölüm zamanında ortaya çıkışı ile, cennet görünümü veren belirsiz ve ilkel bir kavram biçimindeki hayalet anavatanı için daha sonraki hareketi arasında gerçekleştiği varsayılan geçiş süreci boyunca korku duyulmuştu.
\vs p087 1:2 Her ne kadar ilkel insan; doğa\hyp{}üstü güçlere hayaletlere atfetmişse de, onları neredeyse hiçbir biçimde doğa\hyp{}üstü usa sahip olan bir biçimde düşünmemişti. Birçok hile ve aldatmaca, hayaletleri oyuna getirmede ve kandırmada kullanılmıştı; medeni insan, dindarlığın dışsal bir gösteriminin her şeyden haberi olan bir İlahiyat’ı bile bir biçimde kandıracağı ümidine fazlasıyla umut bağlamaktadır.
\vs p087 1:3 İlkel insanlar, genellikle ölümün bir habercisi olduğunu gözlemlemeleri nedeniyle hastalıktan korku duymuştu. Eğer kabile tıbbı hastalığa yakalanmış bir kişiyi iyileştirmede başarısız olduysa hasta insan genellikle, daha küçük bir barakaya taşınarak veya açık havada ölüme terk edilerek aile barakasından taşınırdı. Ölümün gerçekleşmiş olduğu bir ev genellikle yıkılırdı; eğer bu yapılmazsa, her zaman ondan kaçınılırdı; ve bu korku, öncül insanların devasa konutlar inşa etmelerini engellemişti. O aynı zamanda, kalıcı köylerin ve şehirlerin kurulmasını zorlaştırmıştı.
\vs p087 1:4 İlkel insanlar, kavimin bir üyesi öldüğü zaman bütün gece oturup konuşmuşlardı; onlar, bir cesedin yakınında uykuya dalarlarsa aynı şekilde öleceklerinden korku duymuşlardı. Cesetten bulaşan hastalık, ölümden korku duyulmasına dayanak teşkil etti; ve tüm topluluklar, bir dönem içerisinde kesin olarak, ölüye dokunmasından sonra bir kişiyi temizlemek için tasarlanan detaylı arınma ayinlerini uyguladılar. İlkçağ toplulukları, bir cesede ışığın verilme zorunluluğuna inandılar; ölü bir bedenin karanlıkta kalmasına hiçbir zaman izin verilmemekteydi. Yirminci yüzyılda mumlar ölüm odalarında hala yanmaktadır; ve insanlar hala ölüler ile birlikte oturmaktadır. Medeni insan olarak adlandırdığınız bireyler hala, yaşam felsefelerinden ölü bedenlerden duyulan korkuyu tamamen atabilmiş değillerdir.
\vs p087 1:5 Ancak tüm bu korkuya rağmen insanlar hala, hayaletleri aldatmanın yolunu aramışlardı. Eğer ölüm barakası yıkılmamışsa ceset duvardaki bir deliğe doğru taşınırdı, hiçbir zaman kapı tarafında doğru değil. Bu önlemler, onun oyalanmasını önleyen ve geri dönüşüne engel olan bir biçimde hayaleti şaşırtmak için alınmıştı. Cenaze katılımları bir törenden, hayalet takip etmesin diye farklı bir yoldan da geri dönerlerdi. Bireyin geldikleri yolları hesaba katmaları ve diğer bir sürü taktikler, hayaletin mezardan geri dönmemesini teminat altına almak için uygulanmıştı. Farklı cinsler hayaleti aldatmak için sıklıkla kıyafet değiştirmişlerdi. Yas kıyafetleri, yaşanların kimliğini gizlemek için tasarlanmıştı; daha sonra ise, ölülere saygı göstermek ve böylece hayaletleri memnun etmek amacıyla düzenlenmişlerdi.
\usection{2.\bibnobreakspace Hayaletlerin Sakinleştirilmesi}
\vs p087 2:1 Din içinde hayaletlerin sakinleştirilmesine dair olumsuz nitelikli işleyiş uzun bir süre, ruhaniyetlere dayatım ve ricaya dair olumlu nitelikli işleyişten çok önce gelmekteydi. İnsan ibadetinin ilk eylemleri, savunma olgularıydı derin saygı değil. Çağdaş insan, yangına karşı sigorta yaptırmayı bilgelik olarak saymaktadır; benzer bir biçimde ilkel insan, hayalet kötü şansına karşı sigorta sağlamayı bilgeliğin iyi yanı olarak düşünmüştü. Bu korumayı sağlama çabası, hayalet inanışının yöntemleri ve ayinlerini oluşturmuştu.
\vs p087 2:2 Bir zamanlar bir hayaletin büyük arzusunun, rahatsız edilmeden ölüm\hyp{}yerleşkesine ilerleyebilmesi için hızlıca bir biçimde “uzanması” olduğu düşünüldü. Hayaleti yatırma ayininde yaşayanların eylemlerinde gerçekleşecek herhangi bir fazlalık ve eksiklik hatası, hayalet\hyp{}yerleşkesine olan ilerleyişi geciktirmek için yeterliydi. Bunun hayaletin canını sıktığına inanıldı; ve sinirlendirilmiş bir hayaletin felaket, talihsizlik, ve mutsuzluğun bir kaynağı olduğu varsayıldı.
\vs p087 2:3 Cenaze hizmeti, hayalet ruhunun gelecekteki evine gidişi için ikna edilmesinde insanın verdiği çabadan doğmuştu; ve cenaze konuşması kökensel olarak, yeni hayalete buraya nasıl gideceğinin öğretilmesi için tasarlanmıştı. Hayaletin yolculuğu için yiyecek ve giyecek sağlamak adetti, bu eşyalar mezarın içine veya onun yanına konurdu. İlkel insan --- mezarın çevresinden uzaklaştırmak amacı biçiminde --- “hayaleti yatırmak” için üç günden bir yıla kadar sürenin geçmesi gerekliliğine inanmıştı. Eskimo toplulukları hala, ruhun beden ile birlikte üç gün kaldığına inanmaktadır.
\vs p087 2:4 Hayaletin eve geri dönüş arzusu hissetmemesi nedeniyle, bir ölümden sonra sessizlik veya yas gözlenmekteydi. Yaralar biçimde, bireyin kendisine işkence etmesi yas tutmanın ortak bir türüydü. Birçok ileri öğretici bunu engellemeye çalıştı, ancak başarısız oldu. Oruç ve bireyin nefsine halim oluşunun diğer türleri, ölüm\hyp{}yerleşkesine yapacakları mevcut seyahatlerinden önce sinsice bekledikleri geçiş dönemi boyunca yaşamdaki sıkıntılardan zevk alan hayaletler için memnuniyet verici olarak düşünülmüştü.
\vs p087 2:5 Yas eylemsizliğinin uzun ve sık tekrarlanan süreçleri, medeniyetin gelişimi için büyük engellerden bir tanesiydi. Her yılda haftalar ve hatta aylar, bu üretken olmayan ve yararsız yas eylemi içinde kelimenin tam anlamıyla heba edilmekteydi. Cenaze durumları için uzman yas tutucuların tutulduğu gerçeği yasın bir tören olduğunu göstermektedir, kederin bir kanıtını değil. Çağdaş insanlar ölü için saygı ve yakınlarını kaybetme hissi nedeniyle yas tutabilir, ancak ilkel çağ insanları bunu \bibemph{korku} nedeniyle gerçekleştirmişti.
\vs p087 2:6 Ölülerin isimleri hiçbir zaman anılmamıştı. Gerçekte onlar sıklıkla dil içinde yasaklanmışlardı. Bu isimler tabu haline gelmişti; ve böylelikle diller sürekli bir biçimde fakirleşmişti. Bu durum nihai olarak, “birinin hiçbir zaman bahsetmediği isim veya gün” olarak simgesel konuşmanın ve betimsel ifadenin bir birleşimi yarattı.
\vs p087 2:7 İlkçağ insanları, yaşam boyunca arzu edilebilecek her şeyi sundukları bir hayaletten kurtulmaktan çok endişe duymaktalardı. Hayaletler kadınlar ve hizmetçiler istemişlerdi; bir varlıklı ilkel birey, ölümünde en az bir köle kadın eşin diri diri yakılmasını bekledi. Daha sonra, kocasının mezarı üzerinde bir dul eşin intihar etmesi adet haline gelmişti. Bir çocuk öldüğü zaman anne, teyze veya nine; erişkin bir hayaletin çocuk hayalete eşlik edebilmesi ve ona bakabilmesi için sıklıkla boğulurdu. Ve yaşamlarından bu şekilde vazgeçen bu bireyler bunu genellikle oldukça istekli bir biçimde yerine getirdiler; gerçekten de, âdete karşı gelen bir biçimde yaşamlarını sürdürselerdi, hayalet gazabının bu korkusu ilkel insanların memnuniyetle deneyimledikleri yaşamın birkaç keyfini onlardan almış olacaktı.
\vs p087 2:8 Ölen bir öndere eşlik etmesi için çok sayıda kişiyi göndermek adetti; köleler, hayalet\hyp{}yerleşkesinde onlara hizmet edebilmeleri için sahipleri öldüğünde öldürülmektelerdi. Borneo toplulukları hala, bir rehber dost sağlamaktadır; bir köle, ölen sahibiyle birlikte hayalet yolculuğunu sağlaması için kurban edilmektedir. Öldürülen kişilerin hayaletlerinin, katillerine köleleri olarak sahip olmalarından büyük memnuniyet duyduklarına inanılmaktaydı; bu inanç, insanları kafa avcılığını gerçekleştirmeye teşvik etti.
\vs p087 2:9 Hayaletlerin yiyecek kokusundan keyif aldığı varsayılmaktaydı; cenaze yemeklerinde yiyeceklerin sunulması bir zamanlar herkes tarafından uygulanmaktaydı. Yemeğe başlamadan önce yemek duasının ilkel yöntemi, sihirli bir nakarat mırıldanırken ruhaniyetleri yatıştırma amacıyla ateşe bir parça yiyeceğin atılmasıydı.
\vs p087 2:10 Ölülerin, yaşamda kendilerinin olan alet ve silahların hayaletlerini kullandıkları varsayılmaktaydı. Bir eşyayı kırmak, onu “öldürmek” ve böylece onun hayaletini hayalet\hyp{}yerleşkesindeki hizmeti için salıvermek anlamına gelmekteydi. Özel mülkiyet kurbanları aynı zamanda yakma veya gömme işlemi ile gerçekleştirilmişti. İlkçağ cenaze atıkları devasa bir düzeydeydi. Daha sonraki ırklar kâğıt kalıplar yapıp, bu ölüm kurbanlarında gerçek nesneler ve kişiliklerle bu çizimleri değiştirdi. Soydaşlıktan gelen miras özel mülkiyetin yakılmasının ve gömülmesinin yerini aldığında, bu durum medeniyet içinde büyük bir gelişmeydi. İrokua Yerlileri, cenaze artıklarında birçok köklü yeniden düzenlemeyi gerçekleştirdi. Ve özel mülkiyetin bu korunumu, kuzey kırmızı insanlarının en güçlüleri haline gelmeye muktedir kıldı. Çağdaş insanın hayaletlerden korku duyması beklenmemektedir; ancak adetler güçlü olup, dünyasal servetin büyük bir kısmı hala cenaze ayinlerinde ve ölüm törenlerinde harcanmaktadır.
\usection{3.\bibnobreakspace Atalara Tapınma}
\vs p087 3:1 Gelişen hayalet inanışı atalara olan ibadeti kaçınılmaz kıldı, çünkü o, ortak hayaletler ve evrimleşen tanrılar biçimindeki daha yüksek ruhaniyetler arasındaki birleştirici halka haline gelmişti. Öncül tanrılar yalın bir değişle, yüceltilmiş ölen insanlardı.
\vs p087 3:2 Atalara yapılan ibadet kökensel olarak bir ibadetten çok bir korkunun ürünüydü; ancak bu türden inanışlar kesin bir biçimde, hayalet korkusu ve ibadetinin daha fazla yayılmasına katkı sağlamıştı. Öncül ata\hyp{}hayalet inanışlarının takipçileri, kötü niyetli bir hayaletin bedenlerine bu gibi zamanlarda girebilir diye esnemekten bile korkmuşlardı.
\vs p087 3:3 Çocukların evlatlık edinme âdeti, ruhun huzuru ve ilerleyişi için ölümden sonra birisinin bağışta bulunmasının kesinleştirme amacıydı. İlkel insan; akranlarının hayaletlerinden duydukları korkuyla yaşamış ve boş zamanını ölümden sonra kendi hayaletinin güvenli idaresini tasarlamak için harcamıştır.
\vs p087 3:4 Birçok kabile en azından yılda bir kere tüm ruhlar için verilen bir yemeği düzenlemişlerdi. Romalılar hayaletler için verilen on iki tane yemeğe ve her yıl onlara eşlik edilen törenlere sahipti. Yılın yarısı, bu ilkçağ inanışları ile iniltili bir takım törenlere ayrılmıştı. Bir Romalı imparator, bir yıldaki ziyafet günlerini 135’e düşürerek bu uygulamalarda köklü bir değişiklikte bulunmayı denemiştir.
\vs p087 3:5 Hayalet inancı devamlı evrim içerisindeydi. Hayaletlerin tamamlanmamış bir düzeyden mevcudiyetin daha yüksek bir fazına doğru geçtikleri tahayyül edilirken, inanış benzer bir biçimde ruhaniyetlere olan ibadetlere ve hatta tanrılara kadar gelişme gösterdi. Ancak daha ileri ruhaniyetlere duyulan çeşitlilik gösteren inanışlardan bağımsız olarak tüm kabileler ve ırklar bir dönem hayaletlere inanmışlardır.
\usection{4.\bibnobreakspace İyi ve Kötü Ruhaniyet Hayaletleri}
\vs p087 4:1 Hayalet korkusu, dünya dininin hepsinin ortak kökeniydi; ve çağlar boyunca birçok kabile, hayaletlerin bir sınıfına beslenen inanca bağlı kaldı. Onlar; hayaletler memnuniyet duyduğunda iyi talihe, sinirlendiğinde kötü talihe sahip olduklarını düşünmüştü.
\vs p087 4:2 Hayalet inanışı gelişlerken, herhangi bir bireysel insan ile kesin bir biçimde tanımlanamayan ruhaniyetler biçiminde ruhaniyetlerin daha yüksek türlerinin tanınışı gerçekleşmişti. Onlar, hayalet\hyp{}yerleşkesinin düzeyinin ötesine geçip ruhaniyet\hyp{}yerleşkesinin daha yüksek âlemlerine ilerlemiş olan üstünleşmiş veya diğer bir değişle yüceltilmiş hayaletlerdi.
\vs p087 4:3 Ruhaniyet hayaletlerinin iki türüne dair düşünce, dünyanın tümü boyunca yavaş ve kesin bir biçimde gelişme göstermişti. Bu yeni çifte ruhaniyetsellik, kabileden kabileye yayılmak zorunda değildi; o tüm dünyada bağımsız bir biçimde aniden türedi. Genişleyen evrimsel aklın etkilenişinde bir düşüncenin gücü, onun gerçekliğinde veya makul oluşunda değil, bunun yerine onun \bibemph{keskinliğine} ek olarak hazır ve basit uygulanışında yatmaktadır.
\vs p087 4:4 Daha da sonra insanın hayal gücü, hem iyi hem de kötü doğa\hyp{}üstü düzenlenişe dair kavramsallaşmayı tahayyül etti; bazı hayaletler hiçbir zaman iyi ruhaniyetler düzeyine erişmemişti. Hayalet korkusunun öncül tek\hyp{}ruhaniyetselliği kademeli bir biçimde, dünya olaylarının görünmez denetimine dair yeni bir kavramsallaşma biçiminde çifte bir ruhaniyetselliğe doğru evirilmekteydi. En azından iyi ve kötü talih, kendilerine ait düzenleyicilere sahip bir biçimde hayal edilmişti. Ve bu iki sınıf içinde kötü talihi getiren topluluğun daha etkin ve kalabalık olduğuna inanılmıştı.
\vs p087 4:5 İyi ve kötü ruhaniyetlere dair sav nihai olarak olgunlaştığında, tüm dini inanışların en yaygın ve en kalıcı olanı haline gelmişti. Bu ikilik büyük bir dinsel\hyp{}felsefi gelişimi temsil etmişti, çünkü, davranışlarıyla bir ölçüde tutarlı olan fani\hyp{}ötesi varlıklara aynı zamanda inandırarak iyi ve kötü talihin ikisinin de anlaşılmasında insanı yetkin kılmıştı. Ruhaniyetler iyi ve kötü olarak sınıflandırılabilirdi; ancak onlar, dinlerin en ilkel olanlarının tek\hyp{}ruhaniyetliğe ait öncül hayaletler hakkında yürüttüğü fikirler gibi ne yapacağı tamamiyle bilinmez nitelikte düşünülmemekteydi. İnsan en sonunda, davranışı ile tutarlı olan fani\hyp{}ötesi kuvvetleri düşünmeye yetkin hale gelmişti; ve bu durum, din evriminin bütüncül tarihi ve insan felsefesinin gelişimi içinde gerçekliğin en önemli keşiflerinden biriydi.
\vs p087 4:6 Evrimsel din, buna rağmen, çifte ruhaniyetselliğin korkunç bir bedelini ödemişti. İnsanın öncül felsefesi; bir iyi ve diğeri olan kötü biçiminde, sadece ruhaniyetlerin iki türünü doğru varsayarak ruhaniyet tutarlılığını geçici talihin ani değişiklikleriyle bütünleştirmeye yetkin hale gelmişti. Ve bu düşünce; değişmez fani\hyp{}ötesi kuvvetlere dair bir kavramsallaşma ile birlikte şansın değişkenlerinin birleştirilmesinde insanı muktedir kılarken, kâinatsal birlikteliğin düşünülmesini bu dönemden beri dindarlar için zor kılmıştır. Evrimsel dinin tanrılarına genellikle karanlığın kuvvetleri tarafından karşı gelinmiştir.
\vs p087 4:7 Tüm bunların altında yatan trajedi; bu düşünceler ilkel insanın aklında kök salarken, dünyanın tümü üzerinde gerçek anlamda hiçbir kötü veya uyumsuz ruhaniyetin bulunmadığı gerçeğiydi. Bu türden bir talihsiz durum; Caligastia isyanının sonrasında kadar gelişmeyip, sadece Hamsin Yortusu’na kadar varlığını sürdürmüştü. Temel kâinatsal nitelikler olarak iyilik ve kötülüğe dair kavramsallaşma, yirminci yüzyılda bile, insan felsefesinde oldukça canlıdır; dünya dinlerinin çoğu hala, ortaya çıkan hayalet inançlarının uzun zaman önce sonlanmış dönemlerinin bu kültürel doğum izini taşımaktadır.
\usection{5.\bibnobreakspace Gelişen Hayalet İnanışı}
\vs p087 5:1 İlkel insan ruhaniyet ve hayaletleri, neredeyse sınırsız haklara sahip ama hiçbir sorumluğu taşımayan bir biçimde gördü; ruhaniyetlerin insanları çok katmanlı sorumluluklara sahip, ancak hiçbir hakkı taşımayan bir biçimde gördüklerini düşündü. İnsanlığın genel düşüncesi, insan olaylarına karışmamasının bedeli olarak hayaletlerin devamlı bir hizmet vergisi koyduğuydu; ve talihsizlik, olabilecek en küçük düzeyde hayalet etkinliklerine atfedilmişti. Öncül insanlar; tanrılara gösterilmesi gereken hürmette bir eksiklikte bulunmaktan o kadar korkmuşlardı ki, bütün bilinen ruhaniyetleri kurban ettikten sonra, sadece tümüyle güvende olmak için “bilinmeyen tanrılara” diğer bir yönelişte bulunmuşlardı.
\vs p087 5:2 Ve mevcut an içerisinde basit hayalet inanışının yerini; insanın ilkel hayal gücü içerisinde evrimleştiği biçimde daha yüksek ruhaniyetlere yapılan hizmet ve ibadet halindeki, daha gelişmiş ve görece katmanlaşmış ruhaniyet\hyp{}hayalet inancının uygulamaları almıştır. Dini merasim düzeni, ruhsal evrim ve ilerleyişe uyum sağlamak zorundadır. Genişleyen inanç yalnızca, ruhaniyet çevresine olan bireysel uyum biçimindeki doğa\hyp{}üstü varlıklara duyulan inanış ile ilgili olarak bireyin kendisini idare edilişinin sanatıydı. Üretimsel ve askeri düzenler, doğal ve toplumsal çevrelere olan uyum düzenlemeleriydi. Ve evlilik iki cinsliliğin taleplerini karşılamaktan doğarken, dini düzen daha yüksek ruhaniyet kuvvetlerine ve ruhsal varlıklara olan inanç karşılığında evrim göstermişti. Din, şansın gizemine dair hayal gücünün ürünlerine olan uyumu temsil etmektedir. Ruhaniyet korkusu ve onun sonrasındaki ibadet, refah poliçeleri olarak talihsizliğe karşı sigorta biçiminde uygulanmıştı.
\vs p087 5:3 İlkel insan iyi ruhaniyetleri, insan varlıklardan çok az bir şey talep eden bir biçimde kendiişlerine koyulan unsurlar olarak hayal etmektedir. Neşesinin yerinde tutulması gereken kötü hayaletler ve ruhaniyetlerdi. Bu nedenle ilkel topluluklar, kötü niyetli hayaletlerine iyi huylu ruhaniyetlerine nazaran daha fazla ilgi göstermişlerdi.
\vs p087 5:4 İnsan refahının, kötü ruhaniyetlerin kıskançlığını özellikle tetiklediği varsayılmıştı; ve onların intikamı bir insan vasıtasıyla ve \bibemph{kem göz} yöntemiyle karşılık vermekti. Ruhaniyetten kaçınmayla ilişkili olan inancın bu fazı, kem gözün gizlice çevirdiği dolaplar ile fazlasıyla ilgiliydi. Ondan korkmak neredeyse dünyanın tamamına yayılan bir hale gelmişti. Çekici kadınlar, kem gözden korunmaları için kapatılırdı; bunun sonrasında güzel olarak düşünülmek isteyen birçok kadın bu uygulamayı gerçekleştirmişti. Kötü ruhaniyetlere duyulan bu korku nedeniyle çocuklar karanlık çöktükten sonra nadiren dışarı salınırdı; ve öncül dualar her zaman “bizi kem gözden sakın” dileğini içinde barındırmıştı.
\vs p087 5:5 Kuran kem göze ve büyü sözlerine ayrılmış koca bir bölüme sahip olup, Museviler onlara tamamen inanmıştı. Erkeğin cinsel organına dair bütüncül inanç, kem göze karşı bir savunma olarak gelişmişti. Doğum organları, güçsüz kılmaya yetkin, büyülü güçlere sahip olduğuna inanılıp tapınılan tek nesneydi. Kem göz, doğum lekeleri olarak çocukların anne karnında işaretlenilmelerine dair ilk hurafelerin ortaya çıkmasına kaynaklık etmişti; ve bu inanış bir zamanlar neredeyse herkes tarafından takip edilmekteydi.
\vs p087 5:6 Kıskançlık kökleşmiş bir insan niteliğidir; bu nedenle ilkel insan onu öncül tanrılarına atfetmişti. Ve insan bir zamanlar hayaletleri aldatmayı denediği için, yakın bir zamanda ruhaniyetleri kandırmaya başlamıştı. O, “eğer ruhaniyetler bizim güzelliğimizden ve refahımızdan kıskançlık duyarsa, kendimizi çirkinleştirip başarımızdan üstün körü bir biçimde bahsederiz” demişti. Öncül alçak gönüllülük, bu nedenle, benliğin alçaltılması değil, bunun yerine kıskaçlık duyan ruhaniyetleri bir atlatma ve kandırma girişimiydi.
\vs p087 5:7 Ruhaniyetlerin insan refahından kıskançlık duymasını engellemek için uygulanan yöntem, bazı talihli veya çok sevilen nesne ve insan üzerinde ağız dolusu olumsuz ifade kullanmaktı. Küçük düşürücü övgü âdeti, bu şekilde bir insanın kendisi veya ailesinin kökenini belli etmektedir; ve bu nihai olarak medenileşmiş tevazuya, dizginlenmeye ve kibarlığa doğru gelişmişti. Aynı amaç doğrultusunda çirkin görünmek moda haline gelmişti. Güzellik, ruhaniyetlerin kıskançlığına sebep olmaktaydı; o, günahkâr insan gururunun habercisiydi. İlkel insan çirkin bir isim arzulamıştı. İnanışın bu özelliği, sanatın gelişimi için büyük bir engeldi; ve uzun bir süre boyunca dünyayı renksiz ve çirkin halde tuttu.
\vs p087 5:8 Ruhaniyet inanışı altında yaşam en iyi haliyle, ruhaniyet denetiminin sonucu olarak bir kumardı. Bir kişinin geleceği, ruhaniyeti etkilemek için kullanılabilmesi dışında çaba, üretim veya kabiliyetinin sonucu değildi. Ruhaniyetlerin sakinleştirilme törenleri, yaşamı bıktırıcı ve neredeyse dayanılmaz kılan bir biçimde ağır bir yük haline getirmişti. Çağdan çağa ve nesilden nesile, her ırk bu hayalet\hyp{}ötesi savı geliştirmeyi amaçladı; ancak hiçbir nesil şimdiye kadar onu bütünüyle reddetme cesareti gösteremedi.
\vs p087 5:9 Ruhaniyetlerin istek ve arzuları alamet, keramet ve işaret araçlarıyla irdelenmişti. Ve bu ruhaniyet iletileri kehanet, müneccimlik, sıkıntılar ve yıldızbilimi tarafından yorumlanmıştı. Bu inancın bütünlüğü, bahse konu örtülü rüşvet vasıtasıyla ruhaniyetlerin sakinleştirilmesi, tatmin edilmesi ve satın alınması için tasarlanmıştı.
\vs p087 5:10 Ve böylece şu bileşenlerden meydana gelen yeni ve genişlemiş bir dünya felsefesi gelişmişti:
\vs p087 5:11 1.\bibnobreakspace \bibemph{Görev} --- ruhaniyetleri istenen bir biçimde etki altına almak için, en azından tarafsız tutmak amacıyla, yapılması zorunlu olan şeyler.
\vs p087 5:12 2.\bibnobreakspace \bibemph{Doğruluk} --- bir kişinin çıkarları uyarınca ruhaniyetlerin gönlünü sürekli bir biçimde kazanmak için tasarlanmış doğru davranışlar ve ayinler.
\vs p087 5:13 3.\bibnobreakspace \bibemph{Gerçek} --- ruhaniyetlerin doğru bir biçimde anlaşılması ve onlara karşı doğru bir tutumun beslenmesi, ve böylelikle yaşam ve ölüm için aynı tutumun takınılması.
\vs p087 5:14 İlkçağ toplulukları sadece meraktan doğan bir biçimde geleceği bilmeyi arzulamadılar; onlar kötü talihi savuşturmak istediler. Kehanet yalın bir değişle, bir sorundan kaçınma girişimiydi. Bu dönemler boyunca rüyalar kâhinsel olarak değerlendirilirken, olağandışı her şey bir gelecek alameti biçiminde görülmüştü. Ve mevcut anda bile medenileşmiş ırklar, eski dönemlerin gelişmekteki hayalet inanışına ait işaretler, simgeler ve diğer hurafesel kalıntılara beslenen inançtan olumsuz bir biçimde etkilenmektedir. Yavaş bir biçimde, fazlasıyla yavaşça, insan bu yöntemleri bırakırken, bunun sonucunda oldukça kademeli ve acı verici bir halde yaşamın evrimsel ölçeğinde yükselmişti.
\usection{6.\bibnobreakspace Zorlama ve Kötü Ruhları Kovma}
\vs p087 6:1 İnsanlar sadece ruhlara inanmış olup, dini tören daha az örgütlenmiş bir biçimde daha bireyseldi; ancak daha yüksek ruhaniyetlerin tanınması, kendileriyle ilişkileri düzenleyen “daha ruhsal yöntemlerin” kullanılmasını gerektirmişti. Ruhaniyetlerin sakinleştirilme yönteminin geliştirilmesi ve detaylandırılmasına dair bu girişim doğrudan bir biçimde, ruhaniyetlere karşı savunma türlerinin yaratımına yol açtı. İnsan, dünyasal yaşamda faaliyet gösteren denetlenemez kuvvetler karşısında kendini gerçekten de çaresiz hissetmişti; ve onun acizlik düşüncesi, insan ve kâinat mücadelesinin tek taraflı savaşında üstünlükleri dengelemek için bir yöntem biçiminde, bir takım telafi edici düzenleme bulma girişimine yönlendirmişti.
\vs p087 6:2 İnanışın öncül dönemlerinde insanın hayalet faaliyetini etkileme çabaları, kötü talihi rüşvetle satın alma çabaları biçiminde onları yatıştırmakla sınırlıydı. Hayalet inanışının evrimi iyiye ek olarak kötü ruhaniyetlerin varlığına dair kavramsallaşmaya doğru ilerlerken, bu törenler, iyi talihi kazanma biçiminde daha olumlu bir nitelikteki çabalara dönüştü. İnsanın dini artık tamamiyle olumsuz tutumu barındıran nitelikte değildi; buna ek olarak insan, iyi talihi elde etmek için çabalamayı bırakmamıştı; o kısa bir süre içinde, ruhaniyet iş birliğini denetim altına almayla sonuçlanacak düzenleri oluşturmaya başlamıştı. Artık dindar birey, kendi icat ettiği ruhaniyet hayallerinin bitmek tükenmeyen talepleri karşısında savunmasız bir konumda durmamaktadır; ilkel insan, ruhaniyeti eylemini zorla elde edebileceği ve ruhaniyet yardımını mecbur kılabileceği silahları icat etmeye başlamıştı.
\vs p087 6:3 İnsanın savunmadaki ilk çabaları hayaletlere yöneltilmişti. Çağlar ilerledikçe yaşayanlar, ölülere karşı koymada yöntemler geliştirmeye başladılar. Hayaletlerin korkutulması ve uzaklaştırılmaları için birçok yöntem geliştirilmişti; bunlar arasında şunlar sıralanabilir:
\vs p087 6:4 1.\bibnobreakspace Başın kesilmesi ve mezarda bedene bağlanması.
\vs p087 6:5 2.\bibnobreakspace Ölü evinin taşlanması.
\vs p087 6:6 3.\bibnobreakspace Cesedin hadım edilmesi veya bacaklarının kırılması.
\vs p087 6:7 4.\bibnobreakspace Çağdaş mezar taşının bir kökeni olarak taşların altına gömülmesi.
\vs p087 6:8 5.\bibnobreakspace Hayaletin sorun yaratmasını engellemek için daha sonraki bir icat olarak ölünün yakılması.
\vs p087 6:9 6.\bibnobreakspace Bedenin denize bırakılması.
\vs p087 6:10 7.\bibnobreakspace Vahşi hayvanlar tarafından yenilmesi için bedenin terk edilişi.
\vs p087 6:11 Hayaletin ses tarafından rahatsız edildiği ve korkutulduğu varsayılırdı; bağırma, çanlar ve davullar onları yaşayanlardan uzaklaştırmaktaydı; ve bu ilkçağ yöntemleri hala, ölüler için onların “başında beklenilme” aşamasında yaygındır. Pis kokan karışımlar, istenmeyen ruhaniyetleri kovmak için kullanılmıştı. Ruhaniyetlerin oldukça çirkin görüntüleri, kendilerine baktıklarında aceleyle kaçmaları için oluşturulmuştu. Köpeklerin hayaletlerin yaklaşımlarını tespit edebildiklerine, onların uluyarak sahiplerini uyardıklarına, horozların hayaletler yakında olduğu zaman öttüklerine inanılmaktaydı. Bir horozun bir rüzgârgülü olarak kullanılışı, bu hurafenin devamıdır.
\vs p087 6:12 Hayaletlere karşı su en iyi korunma aracı olarak görülmüştü. Kutsal su, din adamlarının ayaklarını yıkadıkları su olarak, tüm diğer türlerden üstündü. Hem ateş hem de suyun, hayaletler için geçilmez engeller oluşturduklarına inanılmıştı. Romalılar, ölü etrafında üç defa su taşımışlardı; yirminci yüzyılda bedene kutsal su serpilmekte olup, mezarlıkta el yıkama hala bir Musevi âdetidir. Vaftiz, daha sonraki su ayininin bir özelliğiydi; ilkel banyo, dini bir törendi. Sadece yakın dönemlerde banyo bir sağlık uygulaması haline gelmişti.
\vs p087 6:13 Ancak insan hayaletleri denetim altına almaya son vermemişti; dinsel ayinler ve diğer uygulamalar vasıtasıyla, ruhaniyet faaliyetini denetlemeye girişecekti. Kötü ruhun çıkarılması, diğer bir ruhaniyeti denetlemesi ve kovması için bir ruhaniyetin kullanılmasıydı; ve bu yöntemler aynı zamanda, hayaletleri ve ruhaniyetleri korkutmak için de kullanılmıştı. İyi ve kötü kuvvetlerin çifte\hyp{}ruhaniyetselliği insana, bir bünyeyi diğeri için kullanma girişimde bolca imkân sundu; çünkü eğer güçlü bir insan zayıf olanı alt edebilirse, bunun sonucunda güçlü bir ruhaniyet alt düzey bir hayalet üzerinde kesinlikle hâkimiyet kurabilirdi. İlkel lanetleme, küçük ruhaniyetleri sindirmek için tasarlanan bir denetleyici uygulamaydı. Daha sonra bu gelenek, düşmanlara lanet okumaya kadar genişlemişti.
\vs p087 6:14 Daha eski ilkçağ adetleri uygulamalarına dönerek ruhaniyetlerin ve yarı\hyp{}tanrıların istenen eylemi gerçekleştirmeye mecbur bırakılabileceklerine uzun bir süre inanılmıştı. Çağdaş insan, aynı işleyişi gerçekleştirmekten ötürü suçludur. Siz birbirinize günlük dil olarak ortak bir biçimde hitap etmektesiniz, ancak dua etmeye başladığınızda ciddi üslup olarak adlandırdığınız diğer neslin eski hitabet türüne geri dönmektesiniz.
\vs p087 6:15 Bu sav aynı zamanda, tapınak fuhuşu gibi cinsel bir nitelikte olan birçok dini\hyp{}ayine yapılan geri dönüşleri açıklamaktadır. İlkel adetlere olan bu geri dönüşler, birçok afetlere karşı kesin koruyucular olarak değerlendirilmişti. Ve bu basit akıllı insan toplulukları ile birlikte bu tür uygulamaların tümü, çağdaş insanın hafifmeşreplik olarak adlandıracağı davranışlardan tamamiyle uzaktı.
\vs p087 6:16 Daha sonra ayinsel yeminlerin uygulaması gelmiş olup, onlar yakın zaman içerisinde dinsel sözler ve kutsal antlar tarafından takip edilmişti. Bu yeminlerin çoğu, bireyin kendisine işkence edişi ve kendisini yaralayışı tarafından eşlik edilmekteydi; daha sonra onlar oruç ve dua ile gerçekleştirilmişti. Bireyin nefsine hâkim oluşu ileri dönemlerde kesin bir denetleyici olarak görülmüştü; bu özellikle cinsel arzunun baskılanmasında gerçeklik taşımaktaydı. Ve böylece ilkel insan öncül bir biçimde; bu türden sıkıntı ve yoksunluklarına olumlu bir biçimde tepki göstermesi için isteksiz ruhaniyetleri zorlamaya yetkin ayinleri olarak bireysel işkence ve bireysel nefis denetiminin etkinliğine duyulan bir inanç biçiminde, dinsel uygulamalarında kararlı bir kısıtlılığı geliştirmişti.
\vs p087 6:17 Her ne kadar İlahiyat ile bir pazarlık yapma eğilimi sergilese de, çağdaş insan artık açık bir biçimde ruhaniyetleri denetim altına almaya girişmemektedir. Ve hala o lanet okumakta, tahtaya vurmakta, parmaklarını bağlamakta ve, bir zamanlar büyülü bir kural olan belirli bir basmakalıp nakaratla birlikte boğazını temizleyip tükürmektedir.
\usection{7.\bibnobreakspace Belirli Nesnelere ve Bünyelere Beslenen Ortak İnancın Doğası}
\vs p087 7:1 Toplumsal düzenin inanç türü, ahlaki hissiyatların ve dini bağlılıkların korunumu ve uyarımı için bir simge düzeni sağlaması nedeniyle varlığını devam ettirmişti. Nesne ve bünyelere olan ortak inanış “eski ailelere” dair tarihsel anlatımlarından türemiş olup, kurumsallaşmış bir yapı olarak varlığını devam ettirdi; her aile bu türden bir inanışa sahiptir. İlham verici her nihai amaç --- kültürel farklılığın kurtuluşunu teminat altına alacak ve gerçekleşmesini kolaylaştıracak belirli yöntemlerini arayan bir biçimde --- devamlılığını sürdüren bir takım simgeciliğe sıkı sıkı sarılmaktadır; ve inanış bu gayesini, duyguları desteklemesi ve onları tatmin edişiyle gerçekleştirmektedir.
\vs p087 7:2 Medeniyetin doğumundan bu yana, toplumsal kültür veya dini gelişim içerisinde her cazibeli hareket; simgesel bir merasim düzeni şeklinde bir ayini geliştirdi. Bu ayin daha bilinç\hyp{}dışı büyüme hali kazandığında, daha güçlü bir biçimde onun takipçilerini yakaladı. Nesne ve bünyelere beslenen ortak inanç hissiyatı korudu ve duyguları tatmin etti; ancak o her zaman, toplumun yeniden yapılanmasında ve ruhsal ilerleyişinde en büyük engel olmuştur.
\vs p087 7:3 Her ne kadar bu inanış her zaman toplumsal ilerlemeyi yavaşlatmışsa da, ahlaki ölçütlere ve ruhsal ideallere inanç besleyen birçok çağdaş bireyin --- karşılıklı desteklenen hiçbir inanca sahip olmayışı biçiminde --- \bibemph{ait olunan} herhangi bir şeyin bulunmaması şeklinde yeterli hiçbir simgeselliği barındırmaması çok üzücü bir durumdur. Ancak dini bir inanış birden yoktan var edilemez; o büyümek zorundadır. Eğer yönetim gücü tarafından ayinleri keyfi bir biçimde ortaklaştırılmazsa hiçbir iki topluluk birbirine özdeş hale gelemeyecektir.
\vs p087 7:4 Öncül Hıristiyan inanışı, o döneme kadar düşünülmüş veya oluşturulmuş ayinlerin en etkili, çekici ve dayanıklı olanıydı; ancak onun değerinin büyük bir kısmı, temelinde yatan kökensel inanışların oldukça fazlasının bir bilim çağı içinde tahrip edilmesiyle yok edilmiştir. Hıristiyan inanışı, birçok temel düşüncesini kaybetmesiyle cansız hale gelmiştir.
\vs p087 7:5 Geçmişte, gerçeklik; simgeselliğin genişleyebildiği bir biçimde inanışın esnek olduğu zamanlarda hızlıca gelişmiş ve özgürce genişlemiştir. Zengin gerçeklik ve uyumlaştırılabilen bir inanış, toplumsal ilerleyişin hızlılığına zemin hazırlamıştır. Anlamsız bir inanış, felsefenin yerini almaya ve nedenselliği köleleştirmeye çalıştığında dini yozlaştırmaktadır.
\vs p087 7:6 Eksiklikleri ve kısıtlıklarına rağmen gerçekliğin her yeni açığa çıkarılışı, yeni bir inanışın doğumuna neden olmuştur; ve İsa’nın dininin yeniden ifadesi, yeni ve uygun bir simgeselliği geliştirmek zorundadır. Çağdaş insan; yeni ve genişleyen düşünceleri, idealleri ve bağlılıkları için belirli bir yetkin simgeselliği bulmak zorundadır. Bu gelişmiş simge, ruhsal deneyim olarak dini yaşamdan doğmak zorundadır. Ve daha yüksek bir medeniyetin bu daha yüksek simgeselliği; Tanrı’nın Yaratıcılığı kavramsallaşmasına dayanmak zorunda olup, insanın kardeşliğinin kudretli idealini gelecekte açığa çıkarışını içinde barındırmalıdır.
\vs p087 7:7 Eski inanışlar haddinden fazla bir biçimde birey\hyp{}merkezciydi; yeni olanlar uygulamalı sevgiden doğmalıdır. Yeni inanış, eskisi gibi, hissiyatı teşvik etmeli, duyguları tatmin etmeli ve sadakati desteklemelidir; ancak bunlardan daha fazlasını yapmak zorundadır: O; ruhsal ilerleyişi kolaylaştırmak, kâinatsal anlamları derinleştirmek, ahlaki değerleri arttırmak, toplumsal gelişimi desteklemek ve kişisel dini yaşamın daha yüksek bir türünü harekete geçirmek zorundadır. Yeni inanış --- toplumsal ve ruhsal olarak --- hem geçici hem de ebedi olan yaşamın yüce amaçlarını sağlamak zorundadır.
\vs p087 7:8 Hiçbir inanış; \bibemph{ev kurumunun} biyolojik, toplumsal ve dini önemine dayanmadığı müddetçe, toplumsal medeniyetin gelişimine ve bireysel düzeydeki ruhsal erişime katkı sağlayamaz. Varlığını sürdüren bir inanış, sonu gelmez değişimin mevcudiyetinde kalıcı olanı simgelemek zorundadır; sürekli değişen toplumsal başkalaşımın akıntılarını birleştiren şeyi yüceltmek durumundadır. Doğru anlamları tanımak, güzel ilişkileri övmek ve gerçek soyluluğun iyi değerlerini yüceltmek zorundadır.
\vs p087 7:9 Ancak yeni ve tatminkâr bir simgeselliği bulmadaki büyük zorluk; çağdaş insanların bir topluluk olarak bilimsel tutuma bağlı kalırken, hurafelerden uzak dururken ve bilgisizlikten nefret ederken, bireyler olarak hepsinin gizemi derinden arzulamaları ve bilinmeyene derin bir biçimde saygı duymalarıdır. Hiçbir inanç; belirli bir üstün gizemi barındırmadıkça ve birtakım kıymetli erişilemezi saklamadıkça varlığını sürdüremez. Yeniden ifadeyle; yeni simgesellik sadece toplum için önemli olmamalıdır, birey için de anlamlı olmalıdır. Yararlı herhangi bir simgeselliğin türleri, bireyin öz girişimleriyle yerine getirebileceği ve aynı zamanda akranları ile birlikte memnuniyetle deneyimleyebileceği şeyler olmalıdır. Eğer yeni bir inanış durağan yerine sadece devinimsel olursa, hem geçici hem de ruhsal olarak insanlığın ilerleyişi için değerli olan bazı şeyleri gerçekten de katabilir.
\vs p087 7:10 Ancak --- ayinler, ortak söylemler veya hedefler olarak --- bir inanış, haddinden fazla katmanlaş bir halde olursa faaliyet gösteremeyecektir. Ve orada, bağlılığın karşılığı olarak sadakatin talebi bulunmalıdır. Her etkin din kesin bir biçimde değerli bir simgeselliği geliştirmektedir. Ve onun takipçileri bu türden bir ayinin; toplumsal, ahlaki ve ruhsal ilerleyişin tümünü yalnızca engelleyecek ve onu yavaşlatacak sınırlayıcı, çirkinleştirici ve baskıcı niteliklere sahip olan kalıplaşmış merasim düzenlerine dönüşen bir biçimde katılaşmasını engellemek için elinden gelen her şeyi yapacaktır. Ahlaki gelişimi gerileten ve ruhsal ilerleyişi teşvik etmede başarısız olan hiçbir inanış varlığını sürdüremez. İnanış, --- gerçek din olarak --- bireysel nitelikteki ruhsal deneyimin yaşayan ve faal bedeninin etrafında geliştiği iskeletsel yapıdır.
\vs p087 7:11 [Nebadon’un bir Berrak Akşam Yıldızı tarafından sunulmuştur.]
