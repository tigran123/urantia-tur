\upaper{60}{Öncül Kara\hyp{}Yaşam Dönemi Sürecinde Urantia}
\vs p060 0:1 Ayricalikli deniz yaşamının dönemi sona ermiştir. Karanın yükselişi, kabuğun soğumasına ek olarak okyanusların su sıcaklığının düşmesi, deniz çekilmesi ve bunun sonucunda onun derinleşmesi ve kuzey enlemler içinde kara seviyesindeki büyük bir artış hep birlikte, ekvator bölgesinden oldukça uzak bir konumda yerleşmiş bölgelerin tümünde dünya ikliminin ciddi ölçüde değişmesine sebebiyet vermiştir.
\vs p060 0:2 Bir önceki çağın son dönemleri, gerçek anlamıyla kurbağaların devriydi; ancak kara omurgalıların bu ataları, büyük ölçüde azalan miktarlarda hayatta kalan bir düzeyde artık hâkimiyet halinde bulunmamaktadır. Oldukça az tür, biyolojik sıkıntının bir önceki dönemine ait çetin sınavlardan geçip hayatta kalmıştır. Spor taşıyan bitkiler bile bu aşamada, neredeyse nesilleri tükenmiş bir konumda bulunmaktaydı.
\usection{1.\bibnobreakspace Öncül Sürüngen Çağı}
\vs p060 1:1 Bu sürecin aşınma birikintileri büyük ölçüde yığıntılar, şistler ve kumtaşlarıydı. Amerika ve Avrupa üzerindeki tortullaşmalar içindeki kalsiyum sülfat ve kırmızı tabakaları, bu kıtların ikliminin kurak olduğuna işaret etmektedir Bu kurak bölgeler, çevreleyen yükseltiler üzerindeki şiddetli ve dönemsel yağmurlar sebebiyle meydana gelen büyük toprak kaymalarına maruz kalmıştı.
\vs p060 1:2 Bu tabakalar içinde çok az fosil bulunmaktadır; ancak kara sürüngenlerinin sayısız kumtaş izi bu fosillerde gözlenebilir. Birçok bölge içinde bu sürece ait bin fitlik kırmızı kumtaşı hiçbir fosil taşımamaktadır. Kara hayvanlarının yaşamı yalnızca Afrika’nın belirli kısımlarında devam eden bir niteliğe sahipti.
\vs p060 1:3 Bu birikintiler, Büyük Okyanus sahili üzerinde 18.000 fite bile ulaşabilen ölçüde, 3.000 ila 10.000 fit arasında değişen kalınlığa sahiptir. Lav daha sonra bu tabakaların birçoğu arasına girmiştir. Hudson Nehri’nin Palisades bölgesi, bu Triyasik tabakalar arasına volkanik karataşının girmesiyle oluşmuştur. Volkanik faaliyet dünyanın farklı bölgelerinde geniş bir ölçüde gerçekleşmekteydi.
\vs p060 1:4 Özellikle Almanya ve Rusya olmak üzere Avrupa üzerinde, bu dönemin birikintileri bulunabilir. İngiltere içinde Yeni Kırmızı Kumtaşı bu döneme aittir. Kireçtaşı; bir deniz baskınının sonucu olarak güney Alpler’de tortullaşmış olup, mevcut an içerisinde bu bölgelerin kendine özgü dolomit kireçtaşı duvarları, zirveleri ve sütunlarında görülebilir. Bu tabaka, Afrika ve Avustralya’nın tümü üzerinde bulunabilir. Carrara mermeri bu türden değişikliğe uğramış kireç taşından kökenini almaktadır. Bu sürece ait hiçbir şey, su altında kalan ve bu nedenle yalnızca önceki ve sonraki çağların arasındaki aralıksız bir su ve deniz birikimini yansıtan Güney Amerika’nın güney bölgelerinde bulunamayacaktır.
\vs p060 1:5 \bibemph{150.000.000} yıl önce, dünya tarihinin öncül kara\hyp{}yaşam çağı başlamıştır. Yaşam, genel olarak, yeterli ölçüde kendisini idame ettirmedi; ancak yaşam, deniz\hyp{}yaşam döneminin çetin ve amansız sonundan daha iyi bir konumda bulunmaktaydı.
\vs p060 1:6 Bu çağ açıldığında Kuzey Amerika’nın doğu ve merkezi kısımları, Güney Amerika’nın kuzey kısmı, Avrupa’nın büyük bir bölümü ve Asya’nın tamamı deniz seviyesinin çok üstündeydi. Kuzey Amerika ilk kez coğrafi açıdan bağımsız bir konuma gelmiştir; ancak yakın bir zaman içinde kıtayı Asya’ya bağlayan Bering Boğazı kara köprüsü tekrar ortaya çıkmaya başlamıştır.
\vs p060 1:7 Atlas ve Büyük Okyanus’a paralel geniş dağ boğazları Kuzey Amerika içinde oluşmuştur. Bir tarafının nihai olarak iki mil yerin altına doğru çöktüğü büyük doğu Connecticut fayı ortaya çıkmıştır. Bu Kuzey Amerika boğazlarının çoğu daha sonra, aynı zamanda birçoğunun dağ bölgelerinin tatlı ve tuzlu su göllerinin havzaları da olduğu, aşınma birikintileri ile dolmuştur. Bu süreci takiben bu doldurulmuş kaya çöküntüleri, toprak altında oluşan lav akıntıları tarafından büyük ölçüde yükselmiştir. Birçok bölgenin taşlaşmış ormanları bu çağa aittir.
\vs p060 1:8 Kıtasal batışlar boyunca genel olarak su üstünde bulunan Büyük Okyanus sahili, Kaliforniya’nın güney kısmı haricinde ve şu anki Büyük Okyanus sınırları içinde barınan geniş bir ada dışında sular altında kalmıştır. Bu tarihi Kaliforniya denizi deniz yaşamı bakımından zengin olup, orta batı bölgesinin eski deniz havzayı ile birleşmek için doğuya doğru genişlemiştir.
\vs p060 1:9 \bibemph{140.000.000} yıl önce \bibemph{ansızın} ve bir önceki dönem boyunca Afrika’da gelişen iki sürüngen atasının kökeni ile sürüngenler, uçsuz bucaksız tür içinde ortaya çıkmıştır. Onlar hızlı bir biçimde gelişmiş, yakın zaman içinde büyük sürüngenler şeklinde timsahlara ve nihai olarak deniz yılanları ve uçan sürüngenlere evirilmişlerdir. Onların geçiş ataları hızlı bir biçimde ortadan kaybolmuştur.
\vs p060 1:10 Bu hızla evirilen sürüngensi dinozorlar yakın bir zaman içerisinde bu çağın hâkimleri haline gelmiştir. Onlar; yumurta ile çoğalan canlılar olup, daha sonra kırk tona kadar varacak olan beden kütlelerini düzenleyen bir paunttan daha az beyne sahip olarak küçük beyinleri ile diğer canlılardan ayırt edilen bir niteliğe sahip olmuşlardır. Ancak öncül sürüngenler daha küçük, etçil ve kanguru gibi arka ayakları üzerinde yürüyen canlılardı; ve onların fosil izlerinin birçoğu, büyük kuşların bazı türleri ile karıştırılmaktadır. Bu sürecin sonrasında otobur dinozorlar evirilmiştir. Onlar; dört ayağının üstünde yürümüş olup, bu topluluğun bir ayağı koruyucu zırh geliştirmiştir.
\vs p060 1:11 Birkaç milyon yıl sonra ilk memeliler ortaya çıkmıştır. Onlar karınbağı\hyp{}olmayan bir niteliğe sahip canlılar olup, çok geçmeden başarısız unsurlar oldukları açığa çıkmıştır; onların hiçbiri hayatta kalamamıştır. Bu oluşum, memeli türlerini geliştirmek için deneyimsel bir çabaydı; ancak Urantia üzerinde bu çaba başarı ile sonuçlanmadı.
\vs p060 1:12 Bu sürecin deniz yaşamı yetersiz bir seviyede bulunmaktaydı, ancak sığ sularda tekrar geniş sahil şeritleri yaratan yeni deniz baskını ile birlikte hızlı bir biçimde gelişmişti. Avrupa ve Asya etrafında daha sığ suların var olması nedeniyle en zengin fosil yatakları bu kıtaların etrafında bulunmaktadır. Mevcut an içerisinde eğer siz bu çağın yaşamını inceleyecek olursanız; Himalayalar, Sibirya, Akdeniz bölgelerine ek olarak Hindistan ve güney Büyük Okyanus havzasına dikkatlice bakınız. Deniz yaşamının temel bir özelliği, dünyanın tümü üzerinde bulunabilen fosil kalıntılarına sahip ilgi çekici ammonit kabuklularının mevcudiyetiydi.
\vs p060 1:13 \bibemph{130.000.000} yıl önce denizler çok az bir değişikliğe uğramıştır. Sibirya ve Kuzey Amerika, Bering Boğazı kara köprüsü ile bağlantılı bir halde bulunmaktaydı. Zengin ve benzersiz deniz yaşamı, binden fazla ammonit canlısının kafadanbacaklıların daha yüksek türlerinden geliştiği yer olan Kaliforniya Büyük Okyanus sahili üzerinde ortaya çıkmıştır. Bu dönemin yaşam değişiklikleri, her ne kadar geçici bir nitelikte bulunmuş ve kademeli bir biçimde gerçekleşmiş olsa da, gerçek anlamıyla devrimseldi.
\vs p060 1:14 Bu devre; yirmi beş yılı aşkın bir süreci kaplamış olup, \bibemph{Triyasik} devri olarak bilinmektedir.
\usection{2.\bibnobreakspace İleriki Sürüngen Çağı}
\vs p060 2:1 \bibemph{120.000.000} yıl önce, sürüngen çağının yeni bir fazı başlamıştır. Bu dönemin büyük gelişimi, dinozorların evrimi ve onların sayılarının azalması olmuştur. Kara\hyp{}hayvan yaşamı; sahip olduğu türlerin büyüklüğü bakımından en büyük gelişim düzeyine ulaşmış olup, bu çağın sonuyla beraber dünya yüzeyinden neredeyse tamamen yok olmuştur. Dinozorlar; iki fitten az türlerindeki bir uzunluktan başlayarak, yaşayan herhangi bir canlı ile kütle bazında o zamandan bu yana karşılaştırılamayacak düzeyde yirmi beş fit yüksekliğindeki dev etçil olmayan dinozorlara kadar uzanan, değişken büyüklüklerin tümü içerisinde evrimleşmiştir.
\vs p060 2:2 Dinozorların en büyüğü, Kuzey Amerika’nın batı kesimlerinde ortaya çıkmıştır. Bu korkunç büyüklükteki sürüngenler Kayalık Dağ bölgeleri boyunca Kuzey Amerika’nın Atlas Okyanus sahilinin tümüyle beraber batı Avrupa, Güney Afrika ve Hindistan üzerinde toprağa gömülmüştür; bu kalıntılara Avustralya’da rastlanmamaktadır.
\vs p060 2:3 Bu devasa canlılar, gittikçe büyürken daha az hareketli ve güçlü hale geldiler; ancak beslenmeleri için olağanüstü miktarlardaki yiyeceğe ihtiyaç duydukları ve kara onların bu ihtiyaçları karşısında büyük kıtlığa düştüğü için, ---bu durumla başa çıkabilecek akla sahip olmadıklarından dolayı --- gerçek anlamıyla açlıktan ölmüş ve böylece nesilleri tükenmiştir.
\vs p060 2:4 Uzun bir süreden beri su seviyesinin üstünde bulunan Kuzey Amerika’nın doğu kesimlerinin çoğu bu zaman zarfında, sahilinin bugünkü konumundan birkaç yüz metre daha ileri doğru genişlemesine sebebiyet verecek ölçüde Atlas Okyanusu seviyesine inmiş ve onun suları altında kalmıştır. Bu kıtanın batı kesimi hala su seviyesinin üstünde bulunmaktaydı; ancak bu bölgeler bile daha sonra, Dakota Kara Tepeler bölgesine doğru doğu yönlü genişleyen kuzey denizi ve Büyük Okyanus’un ikisi tarafından da istilaya uğramıştır.
\vs p060 2:5 Bu dönem; Colorado, Montana ve Wyoming’in Morrison yatakları olarak adlandırılan yerleşkesine ait zengin tatlı su fosillerini tarafından gösterildiği gibi, birçok iç göller tarafından simgelenen bir tatlı su çağıdır. Tatlı ve tuzlu suyun birleştiği bu birikimlerin kalınlığı 2.000 ila 5.000 fit arasında değişiklik göstermektedir; ancak çok az kireçtaşı bu tabakalar arasında mevcut bulunmaktadır.
\vs p060 2:6 Kuzey Amerika’yı bütünüyle kaplayan bir biçimde genişleyen kutup denizi yakın bir zaman zarfında ortaya çıkacak olan And Dağları dışında, benzer bir biçimde Güney Amerika’nın tümünü sular altında bırakmıştır. Çin ve Rusya’nın büyük bir kısmı su baskınına uğramıştır; ancak bu en büyük hacimli su istilası Avrupa içinde gerçekleşmiştir. Bu kara batışı boyunca; eski böceklerin en narin kanatlarının resmedildiği örneğindeki fosillerin tıpkı dün gibi tabakalarda korunduğu biçimiyle, güney Almanya’nın ilgi çekici taşbaskı kayası tortullaşmıştır.
\vs p060 2:7 Bu çağın bitki örtüsü, bir önceki döneme oldukça benzer bir halde bulunmaktaydı. Eğrelti otları var olmaya devam ederken, kozalak ve çam ağaçları günümüzdeki çeşitliliklerine gittikçe daha yakın bir hale geldiler. Bazı kömür oluşumları, kuzey Akdeniz sahilleri boyunca gerçekleşmeye devam etmekteydi.
\vs p060 2:8 Denizlerin geri dönüşleri havayı geliştirmişti. Mercanlar, iklimin hala ılıman ve dengeli olduğunu doğrular bir biçimde, Avrupa sularına yayılmıştır; ancak onlar, yavaş bir şekilde soğuyan kutup denizlerinde bir daha ortaya çıkmadılar. Bu süreçlerin deniz yaşamı, özellikle Avrupa sularında, büyük bir ölçüde iyileşme ve gelişme göstermiştir. Mercanlar ve denizlaleleri, bu zamana kadar gözlenen nüfuslarından daha fazla miktarlarda geçici olarak ortaya çıkmaya başlamıştır; ancak ammonitler, her ne kadar bir türleri sekiz fit çapına erişmiş olsa da, üç ila dört fit arasında değişen ortalama genişliklere sahip bir biçimde okyanusların omurgasız yaşamının hâkimi olmuşlardır. Süngerler her yerde bulunmaktaydı, buna ek olarak mürekkep balıkları ve istiridyeler evirilmeye devam etmiştir.
\vs p060 2:9 \bibemph{110.000.000} yıl önce deniz yaşamının içkin olanakları kendisini açığa çıkarmaya devam etmekteydi. Denizkestanesi, bu çağın olağanüstü başkalaşımlarından biridir. Yengeçler, ıstakozlar ve deniz kabukluların bugünkü türleri olgunlaşmıştır. Bir mersin balık türünün ilk kez ortaya çıkışı biçiminde balık ailesinde gözle görülür değişiklikler meydana gelmiştir; ancak kara sürüngenlerinden türemiş olan yırtıcı deniz yılanları denizlerin tümünü hala istila etmekteydi, ve onlar balık ailesinin tümünün ortadan kalma tehlikesini yaratmıştı.
\vs p060 2:10 Bu süreç başlıca dinozorların çağı olmaya devam etti. Onlar karayı o kadar kıtlık haline düşürmüşlerdi ki; iki tür deniz istilasının bir önceki dönemi boyunca besin için suya uyum sağlamıştır. Bazı yeni türler gelişirken, topluluğun bazı unsurları istikrarlı bir konumda kalmaya, diğerleri ise daha önceden geldikleri düzeye gerilemektedirler. Ve bu son durum, karayı terk eden sürüngenlerin bu iki türünün deneyimlediği değişimdir.
\vs p060 2:11 Zaman geçtikçe deniz yılanları; oldukça hantal hale gelecekleri seviyeye kadar büyümüşlerdir, ve devasa bedenlerini korumak için yeterli akla sahip olmamalarından dolayı nihai olarak ortadan yok olmuşlardır. Her ne kadar bu devasa ichthyosaurslar, büyük bir çoğunluğunun genişlik bakımından otuz beş fitin üstünde olduğu bir büyüklükte, zaman zaman elli fit uzunluğuna kadar büyüse de, beyinleri iki onsdan daha hafif gelmekteydi. Deniz timsah aileleri benzer bir biçimde, sürüngenlerin kara türünden olan bir başa doğru eviriliş gelişimiydi; ancak deniz yılanlarının aksine bu hayvanlar yumurtalarını bırakmak için her zaman karaya dönmüşlerdir.
\vs p060 2:12 Yaşamlarını sürdürebilmek için dinozorların iki türünün nafile bir çaba içerisinde suya doğru göçünden yakın bir zaman sonra, dinozorların diğer iki türü; kara üzerindeki çetin yaşam rekabeti nedeniyle havaya yönelmiştir. Ancak bu uçan pterosaurlar, sonraki çağların gerçek kuş ataları değillerdi. Onlar, içi boş kemiğe sahip olan sekerek ilerleyen dinozorlardan evirilmişti; ve onların kanatları, yirmi ila yirmi beş fitlik bir kanat açıklığı ile yarasa oluşumuna benzemekteydi. Bu eski uçan sürüngenler; on fit uzunluğuna kadar büyümüş olup, bugünkü yılanların sahip oldukları gibi ayrılabilen çenelere sahiptiler. Bir süreliğine bu uçan sürüngenler başarılı bir görünüm sergilediler; ancak onlar, hava canlıları olarak varlıklarını devam ettirmeye yetkin hala getirecek süreci yakalayamadılar. Onlar, kuş soyunun varlığını devam ettirmeyen kolunu temsil etmektedirler.
\vs p060 2:13 Bu süreç içerisinde kaplumbağaların sayıları, ilk olarak Kuzey Amerika’da ortaya çıkan bir biçimde, artmaya devam etmiştir. Onların ataları Asya’dan kuzey kara köprüsü vasıtasıyla gelmiştir.
\vs p060 2:14 Yüz milyon yıl önce sürüngen çağı sona gelmekteydi. Devasa kütlelerinin tümü bakımından dinozorlar, bu türden devasa bedenleri beslemek için yeterli besini sağlayacak usu taşımayan bir biçimde, neredeyse tamamen akılsız varlıklardı. Ve böylece bu hantal sürüngenler sürekli artan sayılarda ortadan yol olmuşlardır. Bu nedenle evrim, fiziksel kütleyi değil beynin gelişimini takip edecek; ve beyinlerin gelişimi, hayvan evrimi ve gezegensel ilerleyişin her bir sonraki çağını simgeleyecektir.
\vs p060 2:15 Bu süreç; sürüngenlerin doruk noktasını ve onların düşüşünün başlangıcını içine alan bir biçimde yaklaşık olarak yirmi beş milyon yılı kapsamış olup, \bibemph{Jura} devri olarak bilinmektedir.
\usection{3.\bibnobreakspace Tebeşir Devir Aşaması\\Çiçekli\hyp{}Bitkiler Dönemi\\Kuşların Çağı}
\vs p060 3:1 Büyük Tebeşir Devri ismini, denizler içerisindeki verimli tebeşir\hyp{}üreten deniz deliklilerin hakimiyetinden alırlar. Bu süreç; Urantia’yı uzun süren sürüngen hakimiyetinin sonuna yaklaştırmakta olup, kara üzerindeki çiçekli bitkilerin ve kuşların ortaya çıkışına şahitlik yapar. Bu süreçler aynı zamanda, devasa kabuk tahribatları ve bununla eş zamanlı olarak gelişen geniş çaplı lav akıntılarına ek olarak büyük volkanik faaliyetlerin eşlik ettiği, kıtaların batı ve güney doğrultuda ayrılışlarının sonlanış dönemleridir.
\vs p060 3:2 Bir önceki yeryüzü sürecinin sonuna doğru kıta karasının büyük bir kısmı; her ne kadar orada hiç dağ zirvesi bulunmuş olmasa da, su seviyesinin üstünde bulunmaktaydı. Ancak kıtasal kara ayrılışları devam ederken, Büyük Okyanus’un derin tabanı üzerindeki ilk büyük engel ile karşılaştı. Yeryüzü kuvvetlerinin bu mücadelesi, Alaska’dan Meksika boyunca Horn Burnu’na kadar genişleşen çok geniş kuzey ve güney bütüncül dağ sıralarının oluşumunu tetikledi.
\vs p060 3:3 Bu süreç böylelikle, yeryüzü tarihinin \bibemph{bugünkü dağ\hyp{}oluşum aşaması} haline gelmiştir. Bu zaman zarfından önce, yalnızca büyük genişlikteki yüksek kara sırtları olarak birkaç dağ zirvesi bulunmaktaydı. Bu aşamada ise Büyük Okyanus sahili yükselmeye başlamıştır; ancak bu oluşum bugünkü kıyı şeridinin yedi yüz mil batısında konumlanmıştı. Sierra Dağları, bu çağın lav akıntılarının sonucu olarak altın taşıyan kuvars tabakaları biçiminde, oluşumlarını ilk kez gerçekleştirmekteydi. Kuzey Amerika’nın doğu kesiminde, Atlas Okyanusu’nun deniz basıncı aynı zamanda kara yükselmesini tetiklemekteydi.
\vs p060 3:4 \bibemph{100.000.000} yıl önce, Kuzey Amerika kıtası ve Avrupa’nın bir kısmı su seviyesinin çok üstünde bulunmaktaydı. Amerika kıtalarının bükülüşü; Güney Amerika’daki And Dağlarının başkalaşımı ve Kuzey Amerika’nın batı vadilerinin kademeli olarak yükselişi ile sonuçlanan bir biçimde, gelişimlerine devam etmekteydi. Meksika’nın büyük bir kısmı su altında kalmıştı; ve Atlas Okyanusu’nun güneyi, nihai olarak bugünkü kıyı şeridine erişerek Güney Amerika’nın doğu sahilini yağmaladı. Atlas ve Hint Okyanusları, bugünkü konumlarına benzer bir şekilde bulunmaktaydı.
\vs p060 3:5 \bibemph{95.000.000} yıl önce Amerika ve Avrupa kara kütleleri tekrar batmaya başlamıştı. Güney denizler; Kuzey Amerika’nın istilasına başlayıp, kıtanın ikinci en büyük kara batışını meydana getirerek Kuzey Buz Denizi ile kuzeye doğru kademeli olarak genişlemiştir. Bu deniz nihai olarak çekildiğinde, suların terk ettiği kıta yaklaşık olarak bugünkü halindeydi. Ancak bu büyük batış başlamadan önce, doğu Appalachian dağları neredeyse bütünüyle su seviyesine kadar alçalmıştı. Bugün seramik üretiminde kullanılmakta olan saf kilin birçok renge sahip olan tabakası; kalınlığı yaklaşık olarak 2.000 fiti bulan düzeyde, bu çağ boyunca Atlas Okyanusu’nun sahil bölgeleri üzerinde tortullaşmıştır.
\vs p060 3:6 Büyük volkanik faaliyetler Alpler’in güneyinde ve bugünkü Kaliforniya kıyı sıradağları hattında meydana gelmiştir. En büyük kabuksal bozulmalar milyonlarca yıl boyunca Meksika’da gerçekleşmiştir. Büyük değişikler aynı zamanda Avrupa, Rusya, Japonya ve Güney Amerika’nın güneyinde meydana gelmiştir. İklim artan bir biçimde çeşitlilik göstermeye başlamıştır.
\vs p060 3:7 \bibemph{90.000.000} yıl önce kapalı tohumlu bitkiler bu öncül Tebeşir Devri denizlerinden ortaya çıkmış olup, yakın bir zaman zarfı içinde kıtaları kaplamıştır. Bu kara bitkileri \bibemph{ansızın} incir ağaçları, manolyalar ve lale ağaçları ile birlikte ortaya çıkmıştır. Bu gelişmelerden yakın bir zaman sonra incir ağaçları, ekmek ağaçları ve hurma ağaçları Avrupa’yı ve Kuzey Amerika’nın batı düzlüklerini kapladı. Bu aşamada hiçbir yeni kara hayvanı ortaya çıkmamıştır.
\vs p060 3:8 \bibemph{85.000.000} yıl önce Bering Boğazı kara köprüsü, kuzey denizlerinin soğuyan suları tarafından kapandı. Bu vakte kadar Atlas\hyp{}Körfez sularına ve Büyük Okyanus’a ait deniz yaşamı, bugün tek\hyp{}tip hale gelen, suyun bu iki bedenindeki ısı farklılıkları nedeniyle büyük farklılıklar göstermekteydi.
\vs p060 3:9 Tebeşir ve yeşil kum kireçtoprağının birikintileri bu sürece ismini vermiştir. Bu süreçlerin tortullaşmaları; tebeşir, şist, kumtaşı, küçük miktarlarda kireç taşı ve önemsenmeyecek ölçüdeki kömür veya linyitten meydana gelmiş bir biçimde rengârenk bir oluşumda bulunmakta olup, birçok bölge içerisinde onlar petrol taşımaktadır. Bu tabakaların kalınlığı, 200 fit ila Kuzey Amerika’nın batı kesimleri ve Avrupa çevresinde olduğu gibi 10.000 fite kadar değişkenlik göstermektedir. Kayalık Dağları’nın batı sınırları boyunca bu birikintiler, yukarı doğru bükülmüş tepelerde gözlenebilir.
\vs p060 3:10 Dünyanın tümü üzerinde bu tabakalara tebeşir nüfuz etmiştir; ve boşluklu yarı\hyp{}kaya oluşumlarının bu tabakaları suyu yukarı doğru bakan deliklerinden almakta, ve dünyanın mevcut haldeki kurak bölgelerinin büyük bir kısmının su ihtiyacını karşılamak için aşağıya doğru taşımaktadır.
\vs p060 3:11 \bibemph{80.000.000} yıl önce dünya kabuğunda büyük karışıklıklar meydana gelmiştir. Kıtasal ayrılışın batı ayağı sabit bir konuma gelmektedir; buna ek olarak iç kıta kütlesinin hantal devinimine ait devasa enerji Kuzey ve Güney Amerika’nın Büyük Okyanus kıyı şeridini yukarıya doğru bükmüş, Asya’nın Büyük Okyanus sahilleri boyunca köklü sonuçları getirecek etkileri başlatmıştır. Bugünkü dağ sıraları ile sonuçlanan bu Büyük Okyanus’a özgü kara yükselişi, yirmi beş milden daha fazladır. Bu yükselişin doğumuna katılan kabuksal kabartılar, Urantia üzerinde yaşamın oluşmasından bu yana meydana gelen en büyük yüzey bozulmalarıydı. Yüzeyin üstünde ve altında gerçekleşen lav akıntıları, yoğun ve her tarafa yayılmış niteliğe sahip gelişmelerdi.
\vs p060 3:12 \bibemph{75.000.000} yıl öncesi, karasal ayrılışın sonunu simgelemektedir. Alaska’dan Horn Burnu’na kadar uzun Büyük Okyanus sahil dağ sıraları oluşumlarını tamamlamıştı, ancak buralarda henüz çok az dağ zirvesi bulunmaktaydı.
\vs p060 3:13 Durmuş kıtasal ayrılışın ters istikametteki itişi Kuzey Amerika’nın batı düzlüklerinin yükselişinin devam etmesine sebebiyet verirken, Atlas Okyanusu sahilinde bulunan alçalmış Appalachian Dağları’nın batısı neredeyse hiçbir eğime sahip olmadan yükselmekteydi.
\vs p060 3:14 \bibemph{70.000.000} yıl önce, Kayalık Dağ bölgesinin olası en yüksek gelişimi ile iniltili kabuksal bozulmalar ortaya çıkmıştır. Kayanın büyük bir kısmı, British Columbia içinde yüzeyin on beş mili ölçeğinde kıvrılma göstermişti; burada Kambriyen kayaları, Tebeşir Devri tabakaları üzerinde eğimli bir biçimde kıvrılmaya uğramıştı. Kanada sınırında olan Kayalık Dağları’nın batı yamacı üzerinde, fazlasıyla dikkate değer başka kıvrılma bulunmaktaydı; burada, bahse konu zaman zarfında gerçekleşen Tebeşir Devir birikintileri üzerine yaşam\hyp{}öncesi kaya tabakaların üstten kıvrılış eklemlenişi gözlenebilir.
\vs p060 3:15 Bu dönem, dünyanın tümü üzerinde sayısız derecedeki küçük bağımsız volkanik hunilerin yükselmesine sebebiyet veren bir volkanik etkinlik çağıydı. Deniz altı volkanları, suyun altında kalmış Himalaya bölgesi üzerinde patlamıştı. Sibirya’ya ek olarak Asya’nın geride kalan büyük bir kısmı hala suyun altında bulunmaktaydı.
\vs p060 3:16 \bibemph{65.000.000} yıl önce tüm zamanların en büyük lav akışlarından biri gerçekleşmiştir. Bu ve bundan önceki lav akışlarına ait birikim tabakaları; Amerika kıtaları, Kuzey ve Güney Afrika, Avustralya ve Avrupa’nın belirli kesinlerinin tümü üzerinde bulunabilir.
\vs p060 3:17 Kara hayvanları küçük değişikliklere uğramıştır; ancak daha büyük ölçekte kıtanın su yüzeyinde belirmesi nedeniyle, özellikle Kuzey Amerika içinde, onlar hızlı bir biçimde çoğalmışlardır. Avrupa’nın büyük bir kısmı su altında bir konumda bulunuşuyla, Kuzey Amerika bu zamanların kara\hyp{}hayvan evriminin büyük oluşum mekânıydı.
\vs p060 3:18 İklim hala sıcak ve tek\hyp{}tipti. Kutup bölgeleri, Kuzey Amerika’nın merkez ve güney kesimlerindeki bugünkü iklime çok benzer hava değişimlerini deneyimlemekteydi.
\vs p060 3:19 Büyük bitki\hyp{}yaşam evrimi gerçekleşmekteydi. Kara bitkileri arasında kapalı tohumlu bitkiler baskın bir konumda bulunmaktaydı; buna ek olarak kayın, huş, meşe, ceviz, çınar, akçaağaç ve çağdaş hurma ağaçlarını içine alan bir biçimde bugünkü ağaçlar ilk kez ortaya çıkmıştı. Meyveler, çayırlar ve hububatlar boldu; ve insan ataları için --- insanın ortaya çıkışı bakımından ikincil evrimsel öneme sahip olan --- hayvan dünyası ne anlama geliyorsa, bu tohum taşıyan çayırlar ve ağaçlar bitki dünyası için o anlama gelmekteydi. \bibemph{Ansızın} ve öncül herhangi bir kökene sahip olmaksızın, çiçekli bitkilerin büyük ailesi başkalaşmıştı. Ve bu yeni bitki örtüsü, yakın bir zaman içerisinde dünyanın tümüne yayılmıştır.
\vs p060 3:20 \bibemph{60.000.000} yıl önce, her ne kadar kara sürüngenlerinin sayıları azalışta olsa da; etçil dinozorların sekerek ilerleyen küçük kanguru çeşitlerinin daha çevik ve etkin türleri tarafından bu aşamada türün temsil edildiği biçimde, dinozorlar karanın hâkimleri olmayı sürdürdüler. Ancak geçmişte belirli bir zaman zarfında, kara bitkilerinin çayır ailesinin ortaya çıkışı nedeniyle sayıları hızlı yükselen hem etçil hem otçul dinozorlarının yeni türleri boy göstermişti. Bu çayır yiyen dinozorların bir türü, iki boynuza ve bir burunsu omuz uzantısına sahip gerçek bir dört ayaklıydı. Yirmi fit uzunluğundaki kaplumbağanın kara türü, bugünkü timsahlar ve günümüz türlerindeki gerçek yılanlara ek olarak ortaya çıkmıştır. Büyük değişiklikler aynı zamanda, balıklar ve deniz yaşamının diğer türleri arasında da gerçekleşmekteydi.
\vs p060 3:21 Önceki çağların yürüyen ve yüzen kuş\hyp{}öncesi türleri, bu dönemdeki uçan dinozorlara ek olarak, havada bir başarı sağlayamamıştır. Orada, yakın zamanda nesilleri tükenen kısa ömürlü türler mevcut bulunmuştur. Onlar da, bedenlerine kıyasla çok küçük beyne sahip oldukları için dinozorların yıkıcı kaderini paylaşmışlardır. Atmosferde yönlerini belirleyecek hayvanları yaratmanın bu ikinci teşebbüsü de, bu ve bir önceki çağ boyunca memelileri açığa çıkarmanın beyhude çabası gibi, başarısızlığa uğramıştır.
\vs p060 3:22 \bibemph{55.000.000} yıl önce evrimsel ilerleyiş, kuş yaşamının tümünün atası olan küçük güvercin\hyp{}vari bir varlık biçiminde \bibemph{gerçek kuşların} ilkinin \bibemph{ansızın} ortaya çıkışı tarafından simgelenmiştir. Bu varlık, dünya üzerinde ortaya çıkmış uçabilen canlıların üçüncü türüdür; ve bu canlı, ne bu zamanın uçan dinozorlardan ne de dişe sahip olan karakuşlarının ilkel türlerden gelen bir biçimde, doğrudan olarak sürüngen topluluğundan türemiştir. Ve bu dönem, sürüngenlerin kapanmakta olan çağına ek olarak \bibemph{kuşların çağı} isminde de bilinmektedir.
\usection{4.\bibnobreakspace Tebeşir Dönemi’nin Sonu}
\vs p060 4:1 Büyük Tebeşir Devri sona yaklaşmakta ve onun sonlanışı, kıtaların deneyimlediği büyük deniz baskınlarının sonunu simgelemektedir. Bu durum özellikle, tamı tamına yirmi dört büyük su baskını açığa çıkmış olan, Kuzey Amerika’da gerçeklik taşımaktadır. Ve her ne kadar orada bu zaman zarfından sonra küçük çaplı batışlar gerçekleşmiş olsa da, bunların hiçbiri bu ve bundan önceki geniş ve uzun süreli deniz baskınları ile karşılaştırılabilecek düzeyde bulunmamaktadır. Kara ve deniz hâkimiyetinin bu dönüşümlü süreçleri, milyonlarca yıl süren çevrimler içinde meydana gelmiştir. Orada, okyanus seviyesi ve kıta kara düzeyinin yükselip alçalması ile ilişkili çağlar boyu süren bir ahenk var olmuştur. Ve bu ahenksel kabuk hareketleri, bu zaman zarfından dünya tarihi boyunca azalan sıklıkta ve genişlikte gerçekleşmeye devam edecektir.
\vs p060 4:2 Bu süreç aynı zamanda, kıta ayrılışlarının sonuna ve Urantia’nın bugünkü dağlarının oluşumuna şahit olmuştur. Ancak kıta kütlelerinin basıncı ve onların çağlar boyunca süren ayrılışlarına ait engellenmiş devinim, dağ oluşumundaki ayrıcalıklı etkilerden değildir. Bir dağ sırasının yerleşkesinin belirlenmesinde ana ve temel etken, geçmiş çağların kara aşınması ve deniz baskınlarına ait göreceli daha hafif birikintiler ile dolmuş olan hali hazırdaki mevcut düzlük veya diğer bir değişle boğazdır. Karanın bu daha hafif bölgeleri zaman zaman 15.000 ila 20.000 fit kalınlığında değişiklik göstermektedir; bu nedenle kabuk herhangi bir sebepten dolayı basınca maruz kaldığında bu hafif bölgeler, dünyanın kabuğu veya kabuğun altında faal olan çekişme ve çatışma halindeki kuvvetler ve basınçlar için telafisel düzenlemeyi sağlamak amacıyla bükülme biçiminde yukarı doğru kıvrılır ve yükselir. Zaman zaman karanın yukarı yönlü itişleri bükülme olmadan gerçekleşmektedir. Ancak Kayalık Dağları’nın yükselişi ile ilişkili olarak büyük bükülme ve eğilim, yer altı ve yer üstündeki çeşitli tabakaların devasa kabuk itişleri eşliğinde meydana gelmiştir.
\vs p060 4:3 Dünyanın en eski dağları, tarihi doğu\hyp{}batı dağ topluluklarının üyeleri olarak Asya, Grönland ve kuzey Avrupa’da konumlanmıştır. Orta yaşa sahip olan dağlar, yaklaşık olarak aynı anda doğmuş biçimde, Büyük Okyanus çevresinde ve ikinci Avrupa doğu\hyp{}batı dağ toplulukları içindedir. Bu devasa yükseliş, Avrupa’dan Batı Hint kara yükseltilerine kadar uzanan bir biçimde yaklaşık olarak on bin mil uzunluğundadır. En genç dağlar; her ne kadar daha yüksek karaların bazılarının adalar olarak kaldığı biçimde bir sonraki süreçte sadece deniz tarafından kaplanabilecek şekilde çağlar boyunca kara yükseltilerinin gerçekleştiği, Kayalık Dağ topluluklarıdır. Orta\hyp{}yaş dağlarının oluşumunu takiben gerçek bir dağ yükseltisi, doğanın etkenlerinin zanaatı ile birlikte bugünkü Kayalık Dağları’na doğru takip eden süreçlerde dönüşme nihai sonu kazandırılmış bir biçimde yükselmiştir.
\vs p060 4:4 Mevcut Kuzey Amerika Kayalık Dağ bölgesi, kayanın özgün yükseltisi değildir; bu yükselti uzun süreçler boyunca aşınma tarafından alçalmış ve sonra tekrar yükselmiştir. Dağların mevcut ön sırası, tekrar yükselmiş olan özgün sıranın geride kalanlarıdır. Pikes Zirvesi ve Longs Zirvesi, dağ yaşamlarının iki veya daha fazla neslini kapsayan bir biçimde, bu dağ etkinliğinin olağanüstü örnekleridir. Bu iki zirve başlarını, önceki su baskınlarının bazıları süresince suyun üstünde tutmuştur.
\vs p060 4:5 Yeryüzü koşullarına ek olarak biyolojik bakımdan bu dönem, kara üzerinde ve su altında kayda değer gelişmelerin yaşandığı etkin bir çağdır. Denizkestanelerinin sayısı artarken, mercanlar ve denizlalelerinin sayısı azalmıştır. Bir önceki çağ boyunca hâkim etkiye sahip olan ammonitlerin sayıları, aynı zamanda hızlı bir biçimde azalma göstermiştir. Kara üzerinde eğrelti otu ormanlarının yerlerini büyük ölçüde, devasa kızılağaçlara ek olarak çam ve çağdaş ağaçlar almıştır. Bu sürecin sonunda karınbağı memelisi henüz evrimleşmemişken biyoloji aşama, gelecek memeli türlerinin öncül atalarının takip eden çağda ortaya çıkışı için bütünüyle hazır hale gelmiştir.
\vs p060 4:6 Kara yaşamının öncül ortaya çıkışından insan türleri ve onun ortak soylarına ait doğrudan ataların daha yakın zamanlarına uzanan bir biçimde, dünya evriminin bu üzün dönemi böylelikle sona ermektedir. \bibemph{Tebeşir devri} olan bu süreç; elli milyon yılı kaplamakta olup, \bibemph{Mesozoyik} devri olarak bilinen ve yüz milyon yıllık bir döneme yayılan biçimde, kara yaşamının memeliler\hyp{}öncesi devrinin sonu getirmektedir.
\vs p060 4:7 [Satania’ya atanmış ve mevcut an içerisinde Urantia üzerinde faaliyet göstermekte olan bir Nebadon Yaşam Taşıyıcısı tarafından sunulmuştur.]
