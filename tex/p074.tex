\upaper{74}{Âdem ve Havva}
\vs p074 0:1 Âdem ve Havva Urantia’ya, 37.848 yıl önce, M.Ö. 1934 yılında geldi. Onların geldikleri dönem, tomurcukların açtığı sürecin zirve noktası olan mevsim ortaları içindeydi. Öğle vakti sularında ve önceden haber verilmeyen bir biçimde, Urantia’ya biyolojik canlandırıcıları ulaştırma görevi verilen Jerusem sorumluları tarafından eşlik edilen iki yüksek meleksel taşıyıcı; Kâinatın Yaratıcısı mabedinin civarında bir yere, dönüş halindeki gezegenin yüzeyine yavaşça indi. Âdem ve Havva bedenlerinin yeniden ete kemiğe büründürülme çalışmalarının tümü, bu yeni yaratılmış tapınağın kapladığı alan içinde gerçekleştirilmiştir. Ve varışlarından, dünyanın yeni idarecileri şeklindeki temsilleri için çifte\hyp{}insan\hyp{}bünyesi içinde yeniden yaratılmalarına kadar on gün geçmiştir. Onlar, bilinçlerine eş zamanlı bir biçimde tekrar kavuşmuşlardır. Maddi Erkek ve Kız Evlatlar her zaman beraber hizmet etmektedirler. Her zaman ve her yerde birbirlerinden herhangi bir şekilde ayrılmamaları hizmetlerinin temel niteliğidir. Onlar, çiftler halinde görev yapmak için tasarlanmışlardır; onlar nadiren yalnız başlarına faaliyet göstermektedirler.
\usection{1.\bibnobreakspace Jerusem’de Âdem ve Havva}
\vs p074 1:1 Urantia’nın Gezegensel Âdem ve Havva’sı, ortak bir biçimde 14.311’incisi oldukları, Jerusem’in kıdemli Maddi Evlatlar birliği üyeleridir. Onlar; üçüncü fiziksel düzeye ait olup, sekiz fitten biraz daha uzunlardı.
\vs p074 1:2 Urantia’ya gelmek için seçildiğinde Âdem, Jerusem’in fiziksel deney laboratuarlarında eşi ile birlikte çalışmaktaydı. On beş bin yıldan daha fazla bir süre zarfı boyunca onlar, yaşam türlerinin dönüştürülmesinde uygulanan deneyimsel enerji biriminin yöneticiliğini yapmaktaydılar. Bu görevlerinden çok uzun bir zaman önce onlar, Jerusem’e yeni gelenler için vatandaşlık okullarında öğretmenlik yapmaktaydılar. Ve tüm bunların hepsi, Urantia üzerindeki bir sonraki faaliyetleri ile ilgili olarak akılda tutulmalıdır.
\vs p074 1:3 Urantia üzerinde Âdemsel serüven görevi için gönüllerin aranmasına dair duyuru yapıldığında, kıdemli Maddi Erkek ve Kız Evlatlar birliğinin her üyesi gönüllü olmak için başvurmuştur. Lanaforge’nin ve Edentia’nın En Yüksek Unsurları’nın onayı ile Melçizedek müfettişleri nihai olarak, Urantia’nın biyolojik canlandırıcıları olarak faaliyet göstermek için gelecek Âdem ve Havva’yı seçmişlerdi.
\vs p074 1:4 Âdem ve Havva, Lucifer isyanı boyunca Mikâil’e sadık kalmayı sürdürmüşlerdi; yine de bu çift, teftiş ve eğitim için Sistem Egemeni ve onun yönetim sorumlularının tümünün huzuruna çağırılmıştır. Urantia olaylarının ayrıntıları bütüncül bir biçimde kendilerine sunulmuştur; onlar, ihtilafla parçalanmış bu türden bir dünya üzerinde yönetim sorumluluklarını kabul etmede izlenecek tasarımlar konusunda etraflı bir biçimde eğitilmişlerdi. Onlar, Edentia’nın En Yüksek Unsurları’na ve Salvington’un Mikâili’ne olan bağlılık üstüne ortak bir biçimde yemin ettirilmişlerdi. Ve Âdem ve Havva’ya yerinde bir biçimde; bu çiftin görevlendirildikleri dünyada yönetimde bulunan bünyenin idari yetkilerini kendilerine devretmeyi uygun bulacakları vakte kadar, kendilerini Melçizedekler’in Urantia Birliği’ne taabi olarak görmeleri gerektiği hususunda tavsiyede bulunulmuştur.
\vs p074 1:5 Bu Jerusem çifti gerilerinde; ilerleme sürecindeki tuzaklardan kurtulmuş ve Urantia için ebeveynlerinin ayrıldıkları dönemde evren yönetiminin sadık koruyucuları olarak tümünün görevlendirilmiş bir konumda bulunduğu --- elli erkek ve kız evlattan meydana gelen --- muhteşem yaratılmışlar biçimindeki yüz çocuğunu, Satania’nın başkentinde veya başka yerleşkelerde bırakmışlardı. Ve onların tümü, bahşedilişin son kabul törenleri kapsamındaki veda gösterilerine katılmak için Maddi Evlatlar’ın güzel mabedinde hazır bulunmuşlardı. Bu evlatlar, düzeylerine ait yeniden\hyp{}maddileştirme ana birimine kadar ebeveynlerine eşlik etmişlerdir; ve onlar, yüksek meleksel taşıma için hazırlanma aşaması öncesinde kişilik bilinç kaybı içinde uykuya dalarlarken ebeveynlerine son kez elvedada ve iyi yolculuklarda bulunan unsurlar olmuşlardır. Bu çocuklar; ebeveynlerinin Satania sisteminin 606’ncı gezegeninin dikkate değer idarecileri, gerçekte ise tek yöneticileri, haline yakın bir zamanda gelecek olmalarını kutlayarak aile yerleşkesinde bir süre beraberce vakit geçirmişlerdir.
\vs p074 1:6 Ve böylelikle Âdem ve Havva Jerusem’i, vatandaşlarının takdirleri ve iyi dilekleri arasında terk etmiştir. Onlar yeni sorumluluklarına, Urantia üzerinde karşılaşılabilecek her görev ve tehlike için yerinde bir biçimde donatılmış ve bütüncül bir şekilde eğitilmiş halde gönderilmişlerdir.
\usection{2.\bibnobreakspace Âdem ve Havva’nın Varışı}
\vs p074 2:1 Âdem ve Havva Jerusem üzerinde uykuya dalmıştır; ve Urantia üzerinde kendilerini karşılamak için toplanmış çok büyük bir kalabalığın karşısında gözlerini açtıklarında, hakkında çok şey duymuş oldukları Van ve onun sadık yardımcısı Amadon ile yüz yüze gelmişlerdir. Caligastia ayrılığının bu iki kahramanı, yeni bahçe evlerinde bu çifti karşılayan ilk unsurlar olmuşlardır.
\vs p074 2:2 Cennet Bahçesi’nin dili, Amadon tarafından konuşulduğu biçimiyle bir Andonsal lehçeydi. Van ve Amadon, yirmi dört harften oluşan yeni bir alfabeyi yaratarak bu dili dikkate değer bir ölçüde geliştirmişlerdi; ve onlar, Cennet Bahçesi kültürün tüm dünyaya yayılacağı bir biçimde bu lisanın Urantia dili haline gelmesini ümit ettiler. Âdem ve Havva; dünyasının yüceltilmiş yöneticisinin kendisini kendi diliyle çağırdığını Andon’un bu evladının duyması için, Jerusem’den ayrılmadan önce bu insan dili üzerinde bütünüyle ustalık kazanmışlardır.
\vs p074 2:3 Ve ulakların büyük bir telaşla her bir taraftan gelen haberci güvercinlerinin toplandığı buluşma yerine giderek “Salın tüm kuşları; bırakın onlar söz verilmiş Evlat’ın geldiği haberini herkese yaysınlar” şeklinde bağırdıkları gün, tüm Cennet Bahçesi boyunca büyük bir coşku ve neşe hâkimdi. Yüzlerce inanan, inançlı bir biçimde daha öncesinden, sadece bu türden bir durum için bu evde yetiştirilen güvercinlerin sayısını her yıl aynı düzeyde tutmuştu.
\vs p074 2:4 Âdem’in varış haberi dışa yayılınca, çevre kabile mensuplarının binlercesi Van ve Amadon’un öğretilerini kabul etti; bunun yanı sıra aylar boyunca kutsal yolcular, Âdem ve Havva’yı karşılamak ve görünmez Yaratıcıları’na hürmet göstermek için Cennet Bahçesi’ne akın etmeyi sürdürdü.
\vs p074 2:5 Uyanışlarından yakın bir zaman sonra Âdem ve Havva’ya, mabedin büyük kuzey tümseğinde gerçekleşecek resmikabulleri için eşlik edildi. Bu doğal tepe çok öncesinden büyütülmüş ve dünyanın yeni yöneticilerinin görevlendirilişi için hazır hale getirilmişti. Burada öğle vaktinde Urantia karşılama heyeti, Satania sisteminin bu Erkek ve Kız Evladı’nı karşıladı. Amadon, altı Sangik ırkın her birinin temsilcisinden oluşan on iyi üyeden meydana gelmiş bir biçimde bu heyetin başkanıydı. Bu heyet; yarı\hyp{}ölümlü unsurların geçici başkanı, Nodit ırkının sözcüsü ve sadık bir kız evlat olan Annan, Cennet Bahçesi’nin mimarı ve yapım ustasının oğlu olan ve ölen babasının tasarımlarını hayata geçiren Nuh’a ek olarak ikamet halindeki iki Yaşam Taşıyıcısı’ndan meydana gelmekteydi.
\vs p074 2:6 Karşılamadan hemen sonraki faaliyet, Urantia üzerindeki kabul heyetinin başkanı olan kıdemli bir Melçizedek tarafından gezegensel sorumluluk görevinin Âdem ve Havva’ya verilmesiydi. Bu Maddi Erkek ve Kız Evlat; Norlatiadek’in En Yüksek Unsurları’na ek olarak Nebadon Mikâili’ne bağlılık yemini etmiş olup, Melçizedek alıcılarının izniyle yüz elli bin yıldan daha fazla bir süre boyunca ellerinde bulundurdukları unvansal yetkiyi böylelikle devreden Van tarafından onların dünya yöneticileri oldukları ilan edilmişti.
\vs p074 2:7 Ve Âdem ve Havva’ya, dünya idareciliğine olan resmi girişlerinin gerçekleştiği bu etkinlikle, hâkimiyet kaftanı verilmiştir. Dalamatia sanatlarının hepsi bu dünyada kaybolmuş bir halde değildi; selamlama hala Cennet Bahçesi döneminde kullanılmaktaydı.
\vs p074 2:8 Bu gelişme sonrasında baş meleklerin duyurusu işitildi; ve Cebrail’in yayındaki seslenişi, Urantia’nın ikincisi gerçekleşecek olan yargı yoklama çağrısını ve Satania’nın 606’ncı âlemi üzerindeki şükran ve bağışlamanın ikinci yazgı dönemine ait olan uyku halindeki kurtuluş unsurlarının uyanmasını emretti. Prens’in yazgı dönemi sonra ermiştir; üçüncü gezegensel çağ olan Âdem’in dönemi, yalın ihtişamın gösterileri ortasında açılmaktadır; ve her ne kadar dünya çapında mevcut olan kafa karışıklığı gezegen üzerinde yönetimde bulunmuş seleflerinin işbirliğini sağlama eksikliğinden kaynaklanmışsa da, Urantia’nın yeni yöneticileri görünüşte elverişli koşullarda hükümranlığına başlamıştı.
\usection{3.\bibnobreakspace Âdem ve Havva’nın Gezegeni Tanıması}
\vs p074 3:1 Ve resmikabullerinden sonraki aşamada Âdem ve Havva acı verici bir biçimde, gezegensel tecrit halinde bulunduklarının farkına vardılar. Onların aşina oldukları yayın araçları sessizdi; ve gezegenler arası iletişimin hiçbir hattı ortada yoktu. Onların Jerusem akranlar bu gibi dünyalar üzerinde ilk deneyimleri boyunca, kendilerini karşılamak için hazır durumda bekleyen ve kendileriyle işbirliğinde bulunmaya yetkin deneyimli yönetim görevlileriyle birlikte oldukça istikrarlı bir Gezegensel Prens ile pürüzsüz bir biçimde faaliyet göstermişlerdi. Ancak Urantia üzerindeki isyan her şeyi değiştirmişti. Burada Gezegensel Prens hali hazırda tamamen mevcut bir haldeydi; ve her ne kadar kötülük işleme gücünün büyük bir kısmı elinden alınmışsa da, Âdem ve Havva’nın görevini zorlaştırmaya ve onu bir ölçüde tehlikeye atmaya yetkindi. Bir sonraki gün için tasarımlarını tartışarak dolunayın parıltısı altında Cennet Bahçesi’nde gece vakti yürüyen iki kişi, Jerusem’in ciddi ve üzüntüyle gerçeklerin yeni farkına varan Erkek ve Kız Evladı’ydı.
\vs p074 3:2 Caligastia ihaneti sonrasında kafa karışıklığına uğramış gezegen olan tecrit altındaki Urantia üzerinde Âdem ve Havva’nın ilk günü böylelikle sona ermiş oldu; ve onlar, dünya üzerindeki ilk geceleri olan --- ve oldukça yalnız olan --- o gecenin ilerleyen saatlerine kadar yürüyüp konuştular.
\vs p074 3:3 Âdem’in dünya üzerindeki ikinci günü, gezegensel alıcılar ve danışma heyeti ile birlikte toplantı halinde geçti. Melçizedeklerden, ve onların birlikteliklerinden, Âdem ve Havva; Caligatia isyanının ayrıntılarına ek olarak bu başkaldırının dünyanın ilerleyişi üzerindeki etkisi hakkında daha fazla bilgiyi elde etti. Ve dünya olaylarının yanlış idaresine dair bu anlatım bütünüyle cesaret kırıcı bir hikâyeydi. Onlar, toplumsal evrim sürecinin hızlandırılması hususunda Caligastia idaresinin bütüncül çöküşüne dair her bilgiyi elde ettiler. Onlar aynı zamanda, ilerleyişin kutsal tasarımından bağımsız olarak gezegensel gelişimi elde etmeye çalışma akılsızlığının bütünüyle farkına vardılar. Ve böylece --- onların Urantia üzerindeki ikinci günleri olan --- üzücü ancak aydınlatıcı bir gün sonra ermiş oldu.
\vs p074 3:4 Üçüncü gün, Cennet Bahçesi için yapılacak bir teftişe ayrılmıştı. Fandorlar olan büyük taşıyıcı kuşlardan Âdem ve Havva, yeryüzü üzerindeki en güzel köşe olan bu yerleşkenin üzerinden havada taşınırken Cennet Bahçesi’nin geniş alanlarına doğru aşağı bakmaktalardı. Bu inceleme günü, Cennet Bahçesi kültürüne ait güzellik ve ihtişamın bu yerleşke ürününü yaratmak için emek vermiş herkesin onuruna devasa bir ziyafet ile sona erdi. Ve tekrar, üçüncü günlerinin gecesinde Evlat ve onun eşi; Cennet Bahçesi içerisinde yürüyüp, sorunlarının ne kadar büyük olduğu hakkında konuştular.
\vs p074 3:5 Dördüncü günde Âdem ve Havva, Cennet Bahçesi birlikteliğine seslenmişti. Açılış töreninden itibaren onlar; dünyanın iyileştirilmesine dair tasarımları ile ilgili insanlara seslenip, günah ve isyanın bir sonucu olarak oldukça alt seviyelere düşmüş olan Urantia’nın toplumsal kültürünü eski haline getirmeye dair izleyecekleri yöntemleri sıraladılar. Bu muhteşem bir gündü, ve dünya olaylarının yeni idaresinde sorumluluk almak için seçilmiş olan erkek ve kadınların oluşturduğu heyet için verilen ziyafetle sona erdi. Şunu dikkatinizden kaçırmayın! Erkeklere ek olarak kadınlar da bu topluluğun içerisindeydi; ve Dalamatia döneminden bu yana yeryüzü üzerinde bu türden bir şey ilk defa gerçekleşmekteydi. Dünya olayları onurlarını ve sorumluluklarını bir erkek ile paylaşan bir kadın olarak Havva’ya bakmak, hayretler içerisinde bırakan bir değişimdi. Ve böylece dünya üzerindeki dördüncü gün sona erdi.
\vs p074 3:6 Beşinci gün, Melçizedek alıcılarının Urantia’dan ayrılmaları gereken vakte kadar faaliyet gösterecek olan yönetim biçimindeki geçici hükümetin örgütlenişiyle geçti.
\vs p074 3:7 Altıncı gün, insan ve hayvanların sayısız türü için yapılan bir incelemeye ayrıldı. Cennet Bahçesi’nin doğusu yönünde uzanan duvarlar boyunca Âdem ve Havva’ya bütün gün eşlik edildi; onlar gezegenin hayvan yaşamını gözlemleyip, yaşayan canlıların bu kadar fazla çeşidinin bulunduğu bir dünyanın karışıklığına düzen getirmek için nelerin yapılması gerektiğine dair daha iyi bir anlayışa varmışlardı.
\vs p074 3:8 Kendisine gösterilen binlerce hayvanın doğası ve işlevi hakkında nasıl bütüncül bir kavrayışa sahip olduğunu gözlemlemek, bu gezi esnasında Âdem’e eşlik edenlerde büyük bir şaşkınlığa sebebiyet vermişti. Bir hayvanı gördüğü anda, onun doğası ve davranışı hakkında bilgi verebilirdi. Âdem; maddi yaratılmışların tümü için bir bakışta onların kökenine, doğasına ve faaliyetine dair tanımlayıcı bilgiler verebilirdi. Bu inceleme gezisinde ona rehberlik eden bireyler, dünyanın yeni idarecisinin tüm Satania içerisindeki en bilgili anatomi uzmanlarından biri olduğunu bilmemektelerdi; buna ek olarak Havva da Âdem ile eşit derecede bilgi yetkinliğine sahipti. Âdem, insan gözleri için görülmeyecek derecede küçük olan yaşayan canlı yapılarını tarif ederek beraberinde bulunduğu kişileri hayrete düşürdü.
\vs p074 3:9 Dünya üzerindeki ikametlerinin altıncı günü sonlandığında Âdem ve Havva, “Cennet Bahçesi’nin doğusundaki” yeni evlerinde ilk kez konakladılar. Urantia serüveninin ilk altı günü oldukça yoğun bir biçimde geçmişti; ve onlar, tüm etkinliklerden bağımsız tamamiyle özgür bir gün geçirmeyi büyük bir sevinçle arzulamaktalardı.
\vs p074 3:10 Ancak şartlar aksini gerektirmişti. Âdem’in oldukça bilge bir şekilde ve fazlasıyla etraflı bir biçimde Urantia’nın hayvan yaşamından bahsettiği bir gün önceki deneyimi, ustaca sunduğu açılış konuşması ve büyüleyici edası ile birlikte, Cennet Bahçesi sakinlerinin kalplerini öyle bir şekilde kazanmıştı ki ve onların akıllarına öyle bir üstünlük kurmuştu ki; onlar, Jerusem’in bu yeni gelen Erkek ve Kız Evladı’nı idarecileri olarak kabul etmeye yalnızca tüm içtenlikleriyle meyilli hale gelmeyip, büyük bir kısmı tanrıları olarak önlerinde diz çöküp onlara ibadet etmeye başlamaya hazır hale gelmişti.
\usection{4.\bibnobreakspace İlk Başkaldırı}
\vs p074 4:1 Altıncı günü takip eden gece Âdem ve Havva uykudayken, Cennet Bahçesi’nin merkezi kısmında Yaratıcı’nın mabedi yakınında garip şeyler meydana gelmekteydi. Orada, pürüzsüz ayın parıltıları altında, heyecanlı ve coşkulu yüzlerce erkek ve kadın saatler boyunca önderlerinin sakince sundukları istekleri dinlemişlerdi. Onlar iyi niyete sahiplerdi, fakat yeni yöneticilerinin birliktelikçi ve demokratik tutumlarının yalınlığını bir türlü anlamamışlardı. Yeni gün doğumu başlamadan çok önce dünya olaylarından sorumlu yeni ve geçici vekiller, Âdem ve eşinin katışıksız bir biçimde haddinden fazla mütevazı ve alçakgönüllü olduğu yargısına neredeyse oybirliği ile ulaştılar. Onlar; Âdem ve Havva’nın gerçek tanrılar oldukları veya hürmetkâr ibadete layık olacak bir şekilde bu türden tanrı mertebesine yakın oldukları varsayımıyla, Kutsallık’ın yeryüzüne beden içinde indiğine karar verdiler.
\vs p074 4:2 Âdem ve Havva’nın bu ilk altı gününe dair muhteşem olaylar bütünüyle, dünyanın en gelişmiş insanlarının bile sahip olduğu hazırlıksız beyinler için çok fazlaydı; onların başları dönmekteydi; herkesin hürmetkâr ibadetlerini sergileyen bir biçimde onların önlerinde diz çökebilmesi ve alçakgönüllü biatlerini dışa vuran bir biçimde secdelerine kapanabilmesi için, bu kutsal çiftin Yaratıcı’nın mabedine ayın en tepede olduğu vakit getirilmesi fikrine kapılmışlardı. Ve Cennet sakinleri, bu düşüncelerin samimiydiler.
\vs p074 4:3 Van bu fikre karşı geldi. Amadon, kendisine Âdem ve Havva’yı gece vakti koruma onuru verilmiş olduğu için bu süreçte orada bulunmaktaydı. Ancak Van’ın itirazı bir kenara itildi. Ona, kendisinin de haddinden fazla mütevazı ve alçakgönüllü olduğu söylendi; eğer tanrıdan farksız olmasaydı, dünya üzerinde nasıl bu kadar uzun süre yaşayabileceği ve Âdem’in varışı gibi bir etkinliğe nasıl önayak olabileceği kendisine soruldu. Ve Cennet Bahçesi sakinleri kendisini kucaklayıp hayranlık gösterisi için hükümranlık tepesine taşıyacakken; Van bir şekilde kalabalıktan kurtulup, yarı\hyp{}ölümlü unsurlar ile iletişim yetkinliğine sahip olarak, bu unsurların başını çabucak Âdem’e gönderdi.
\vs p074 4:4 Âdem ve Havva bu iyi niyetli ancak yanlış yönlendirilmiş fanilerin isteklerine dair ürkütücü haberi aldığı an, dünya üzerinde onların yedinci gününün gün doğumuna yakındı; ve bunun sonrasında, her ne kadar taşıyıcı kuşlar hızlı bir biçimde onları mabede taşımak için kanatlanmaya yetkin olsalar da, bu tür şeyleri yapabilme kabiliyeti olan yarı\hyp{}ölümlü unsurlar Âdem ve Havva’yı Yaratıcı’nın mabedine ulaştırdı. Yedinci günün erken sabah saatlerinde ve henüz yeni düzenlenmiş kabul törenlerinin gerçekleştiği tepeden Âdem; kutsal evlatlığın düzeyleri hakkında konuşma düzenleyip, sadece Yaratıcı’ya ve onun belirlediği unsurlara ibadet edilebileceğini bu dünya akıllarına izah etti. Âdem, kendisine sunulabilecek her türlü onuru ve saygıyı kabul edebileceğinin ancak hiçbir şekilde ibadeti kabullenemeyeceğinin altını kesin bir biçimde çizdi.
\vs p074 4:5 Öğle vaktinden hemen önce, dünya yöneticilerinin göreve gelmesine dair Jerusem kabulünü taşıyan yüksek melek habercisinin varışı suları çok dikkate değer bir andı; Âdem ve Havva kalabalıktan ayrılıp, Yaratıcı’nın mabedine dönerek şunları söyledi: “İşte şimdi Yaratıcı’nın görünmez mevcudiyetinin bu maddi simgesine gidin ve hepimizi yaratan ve yaşamın içerisinde tutan onun önünde ibadet içinde eğilin. Ve bu hareketiniz, bir daha tekrar Tanrı’dan başka kimseye ibadet etme cazibesine kapılmayacağınızın samimi vaadi olsun.” Onların tümü Âdem’in emrettiği gibi yaptılar. Toplanan insanlar mabedin etrafında secdeye kapanırken, Maddi Erkek ve Kız Evlat eğik başlarıyla kutsal tepede tek başlarına dikildiler.
\vs p074 4:6 Ve bu olay, Şabat\hyp{}günü geleneğinin kökenini oluşturmuştur. Cennet Bahçesi içerisinde yedinci gün her zaman, mabette gerçekleştirilen öğlen buluşmasına adanmıştır; uzun bir süre boyunca bu günü bireysel gelişime adamak bir gelenek halindeydi. Öğleden öncesi fiziksel gelişime, öğle vakti ruhsal ibadete, öğleden sonraları zihinsel gelişime ayrılırken, akşamları ise toplumsal eğlenceye adanmıştı. Bu adetsel etkinlikler Cennet Bahçesi’nde hiçbir zaman bir kanun olarak uygulanmamaktaydı; ancak, Âdemsel idare doğru işleyişinden ayrıldığı vakte kadar bir gelenek olarak kabul görmekteydi.
\usection{5.\bibnobreakspace Âdem’in İdaresi}
\vs p074 5:1 Âdem’in varışından sonra neredeyse yedi yıl boyunca Melçizedek alıcıları görevlerini sürdürmeye devam etti; ancak dünya olaylarının idaresini Âdem’e teslim edip Jerusem’e dönmelerinin vakti nihayeten gelmişti.
\vs p074 5:2 Alıcıların uğurlanması bir tam gün sürdü; ve akşam boyunca kişisel olarak Melçizedekler, Âdem ve Havva’ya veda nasihatlerini verip ve en iyi dileklerini sundular. Âdem bir kaç defa, danışmanlarının dünya üzerinde kendisi ile birlikte ikamet etmesini talep etmişti; ancak bu talepler her zaman reddedilmişti. Maddi Evlatlar’ın, dünya olaylarının işleyişi ile ilgili bütüncül sorumluluğu zorunlu olarak almalarının vakti gelmişti. Ve böylece, gece vakti, Satania’nın yüksek melek taşıyıcıları beraberinde on dört unsur ile birlikte Jerusem için gezegenden ayrıldı; Van ve Amadon’un aktarımı, on iki Melçizedek unsurunun ayrılışı ile birlikte eş zamanlı olarak gerçekleşmişti.
\vs p074 5:3 Urantia üzerinde her şey bir süre oldukça iyi bir biçimde ilerledi; ve Cennet Bahçesi medeniyetinin kademeli olarak genişlemesini sağlamak için Âdem’in nihai olarak bazı tasarımlarda bulunmaya yetkin olduğu ortaya çıktı. Melçizedekler’in tavsiyesini izleyerek o, dış dünya ile ticaret ilişkilerini geliştirme düşüncesiyle birlikte sanat ve imalatı desteklemeye başladı. Cennet Bahçesi dağıldığında; çalışır halde yüzden fazla ilkel imalat atölyesi var olup, yakın kabileler ile birlikte hâlihazırda geniş ticaret ilişkileri kurulmuştu.
\vs p074 5:4 Daha öncesinde Âdem ve Havva çağlar boyunca, evrimsel medeniyetin ilerlemesi için özelleşmiş katkılarını sunmaya bir dünyayı hazır hale getirerek onu geliştirme yöntemi üzerinde eğitilmişlerdi; ancak bu aşamada onlar yabani, barbar ve yarı\hyp{}medeni insan varlıklarından oluşan bir dünya üzerinde adalet ve düzeni sağlamak gibi ivedilikle çözülmesi gereken sorunlar ile karşı karşıya kalmışlardı. Cennet Bahçesi’nde toplanmış haldeki dünya nüfusunun en üst tabakasında bulunan unsurlardan başka, sadece tek tük birkaç topluluk Âdemsel kültürü özümsemeye tamamen hazır bir haldeydi.
\vs p074 5:5 Âdem, bir dünya hükümetini kurmak için kahraman vari ve kararlı bir çaba sergiledi; ancak her hamlesinde inatçı bir direniş ile karşılaştı. Âdem; Cennet Bahçesi bütününde bir toplumsal denetim düzenini her zaman faal halde uygulamış olup, Cennet Bahçesi birlikteliğine bu toplulukların tümünü eklemlemişti. Ancak sorunlar, ciddi ölçekteki sorunlar; kendisinin Cennet Bahçesi dışına çıkıp, bu düşünceleri çevre kabilelere uygulamaya çalışmasıyla başladı. Âdem’in yardımcıları Cennet Bahçesi’nin dışında çalışmaya başladığı an, Caligastia ve Daligastia’nın doğrudan ve oldukça iyi tasarlanmış direnişiyle karşılaştılar. Tahtını kaybetmiş Prens bir dünya yöneticisi olarak görevden alınmıştı; ancak gezegenden alınmamıştı. O hala dünya üzerinde, ve insan toplumunun iyileştirilmesi için Âdem’in sahip olduğu tasarımların tümüne karşı koymada en azından bir ölçüde yetkin konumda bulunmaktaydı. Âdem, Caligastia’ya karşı ırkları uyarmaya çalıştı; ancak baş düşmanının faniler için görünmez oluşu yüzünden işi çok zordu.
\vs p074 5:6 Cennet Bahçesi unsurları arasında bile, ölçüsüz kişisel özgürlüğe dair Caligastia öğretisine meyil eden kafa karışıklığı içerisindeki akıllar mevcuttu; ve onlar Âdem’e sürekli sorunlar çıkarmaktaydılar; onlar her zaman, düzenli ilerleme ve köklü gelişim için en iyi düşünülmüş tasarımları baltalamaktaydılar. Âdem nihai olarak, bütüncül uygulamalarından vazgeçip acil toplumsallaşmayı tercih etmek zorunda kalmıştır; o, Cennet Bahçesi sakinlerini başlarında bir önder olacak şekilde yüzerli topluluklara ayıran ve bu toplulukların on tanesinin başına bir baş sorumlu atayan, Van’ın toplumsal düzen yöntemine geri dönmüştü.
\vs p074 5:7 Âdem ve Havva, monarşik bir yapı içerisinde temsili hükümeti sağlamak için gelmişti; ancak onlar, koca yeryüzünün hiçbir tarafında bu isme layık bir yönetimi bulamadılar. Bir süreliğine Âdem, temsili hükümeti kurmak için tüm çabalarını bir kenara bıraktı; ve Cennet Bahçesi düzeninin çöküşünden önce o, güçlü bireylerin kendisi adına yönetimde bulunduğu çevrede konumlanmış neredeyse yüz ticaret ve toplumsal merkezi kurmada başarılı oldu. Bu merkezlerin büyük bir kısmı Van ve Amadon’dan tarafından daha öncesinde örgütlenmişti.
\vs p074 5:8 Bir kabileden diğerine elçi göndermek Âdem’in döneminden kökenini almaktadır. Bu faaliyet, hükümetin evriminde büyük bir ileri adımdı.
\usection{6.\bibnobreakspace Âdem ve Havva’nın Ev Yaşamı}
\vs p074 6:1 Âdemsel aile yerleşkesi, yaklaşık olarak on üç kilometrekarelik bir alanı kaplamaktaydı. Bu ev alanının hemen çevresinde üç yüzden binden fazla saf doğumun bakılması için arazi açılmasına dair emir verilmişti. Ancak bu tasarlanan binaların sadece yüzde biri bu süreç içerisinde tamamlanmıştı. Âdemsel aile bu öncül yönergelerden daha fazla büyümeden önce, Cennet Bahçesi’ne dair bütüncül tasarım sekteye uğramış ve Bahçe terk edilmişti.
\vs p074 6:2 Ademoğlu, Urantia’nın sahip olduğu eflatun ırkının ilk doğan insanıydı; bu doğumu, onun kız kardeşine ek olarak Âdem ve Havva’nın ikinci oğlu olan Havvaoğlu takip etti. Havva, Melçizedekler gezegeni terk ettiklerinde, üç oğlan ve iki kıza sahip olarak, beş çocuk annesiydi. Bu çocukları takip eden doğum ikiz bebeklerdi. Havva; doğru gidişattan ayrınılmadan önce, otuz iki kız ve otuz bir erkek çocuğu sahip olarak, altmış üç çocuğu dünyaya getirmişti. Âdem ve Havva Cennet Bahçesi’ni terk ettiklerinde aileleri, saf soyun üyeleri olan 1.647 evladı kapsayan dört nesilden meydana gelmişti. Onlar Cennet Bahçesi’ni terk ettikten sonra, dünyanın fani ırk kollarından olan katılımsal ebeveynliklerinden türemiş iki doğumun yanı sıra, kırk iki çocuğa sahip olmuşlardır. Ve bu sayı, Âdem’in Nodit ve diğer evrimsel ırklara olan katılımını kapsamamaktadır.
\vs p074 6:3 Âdemsel çocuklar, bir yaşında anne sütüyle olan beslenmeleri kesildiğinde hayvanların sütünü kullanmamışlardır. Havva, çok çeşitli yemişlerin sütüne ve birçok meyvenin öz suyuna erişmekteydi; ve bu yiyeceklerin sahip olduğu kimyayı ve enerjiyi oldukça bütüncül bir biçimde bilerek, çocuklarının dişleri çıkıncaya kadar onların beslenmesi için bu besinleri yararlı biçimlerde bir araya getirdi.
\vs p074 6:4 Yiyecekleri pişirme her ne kadar Cennet Bahçesi’nin Âdemsel kesimin hemen çevresinde evrensel olarak kullanılmaktaysa da, Âdem’in evinde hiçbir şey pişmiyordu. Onlar --- meyveler, yemişler ve tahıllar olarak --- yiyeceklerini, olgunlaştıklarında yemeye hazır olarak görmekteydiler. Onlar, öğleden biraz sonra olmak üzere, günde bir kez yemek yemekteydiler. Âdem ve Havva aynı zamanda “ışık ve enerjiyi”, yaşam ağacının hizmetiyle ilişkili belirli mekân kaynaklarından doğrudan bir biçimde özümsemekteydiler.
\vs p074 6:5 Âdem ve Havva’nın bedenleri bir ışık parıltısı yaymaktaydı; ancak onlar her zaman, birliktelik halinde oldukları unsurların geleneklerine uyan elbiseleri giymişlerdi. Her ne kadar gündüz vakti çok az kıyafet kiyseler de, akşamları gece örtülerini üzerlerine giymekteydiler. Dindar ve kutsal varsayılan insanların başlarını çevreleyen geleneksel halelerin kökeni Âdem ve Havva’nın dönemine dayanmaktadır. Âdem ve Havva’nın bedenlerinden sızan ışık çok büyük ölçekte kıyafetleri tarafından engellendiği için, sadece başlarından yansıyan parıltı fark edilebilmekteydi. Ademoğlu’nun soyları her zaman böylelikle, ruhsal gelişim bakımından olağanüstü olduğuna inanılan insan türlerini resmetmişlerdir.
\vs p074 6:6 Âdem ve Havva, yaklaşık elli millik bir uzaklıktan birbirleri ve doğrudan çocukları ile iletişim kurabilirlerdi. Bu düşünce alış\hyp{}verişi, beyin yapılarına yakın bir yerde konumlanan hassas gaz odaları aracılığı ile gerçekleştirilmekteydi. Bu işleyiş vasıtasıyla onlar, düşünce titreşimlerini gönderebilir ve onları alabilirlerdi. Ancak bu güç, aklın kendisini uyumsuzluğa ve kötülüğün kargaşasını teslim etmesiyle anında askıya alınmıştı.
\vs p074 6:7 Âdemsel çocuklar, on altı yaşına gelene kadar kendi okullarında eğitim gördüler; genç olarak yaşlılar tarafından eğitilmekteydi. Küçük çocuklar her otuz dakikada bir içinde bulundukları etkinlikleri değiştirirken, yaşlılar bunu her saatte bir gerçekleştirmekteydi. Ve Âdem ve Havva’nın bu çocuklarını, tamamen sadece eğlence amacıyla keyif ve mutluluk verici etkinlik olarak, oyun oynarken görmek Urantia üzerinde kesinlikle yeni bir şeydi. Bugünkü ırkların oyun ve mizahı büyük ölçüde Âdemsel ırk kollarından kaynağını almıştır. Âdemsel unsurların tümü, keskin bir mizah anlayışına ek olarak büyük bir müzik beğenisine sahiplerdi.
\vs p074 6:8 Nişanlanmanın ortalama yaşı on sekizdi; ve bu gençler bahse konu süreci takiben, evlilik sorumluluklarının üstlenilmesine hazırlık amacıyla iki yıllık eğitim dönemine girmektelerdi. Yirmi yaşında onlar evlenmeye uygun hale gelmektelerdi; ve evliliklerinden sonra onlar, hayatlarını adadıkları mesleklere veya bu meslekler için özel bir biçimde hazırlanmaya başlamaktaydılar.
\vs p074 6:9 Daha sonraki bazı milletlerin, tanrılardan türediği varsayılan, kraliyet ailelerinin sahip olduğu ağabey ile kız kardeşi evlendirme uygulaması Âdemsel doğumların geleneklerine dayanmaktadır --- çiftleşme, zorunlu olarak, birini diğerine gerekli kılmaktadır. Cennet Bahçesi’nin ilk ve ikinci neslinin evlilik törenleri her zaman Âdem ve Havva tarafından yerine getirildi.
\usection{7.\bibnobreakspace Cennet Bahçesi’nde Yaşam}
\vs p074 7:1 Âdem’in çocukları, dört yıllık batı okullarında olan öğrenimleri dışında, “Cennet Bahçesi’nin doğusunda” yaşayıp burada çalışmışlardı. Onlar, Jerusem okullarının yöntemleri uyarınca on altı yaşına kadar ussal olarak eğitilmişlerdi. On altı yaşından yirmi yaşına kadar onlar, Cennet Bahçesi’nin diğer ucunda eğitilmişlerdi; bu dönemde onlar aynı zamanda, alt sınıflara öğretmenler olarak hizmet vermekteydiler.
\vs p074 7:2 Cennet Bahçesi’nin sahip olduğu batı okul düzeninin bütüncül amacı, \bibemph{toplumsallaşmaydı}. Öğleden önceki mola dönemleri uygulamalı bahçecilik ve tarıma ayrılmışken; öğleden sonraki bu dönemler rekabete dayalı oyunlara adanmıştı. Akşamları, toplumsal etkileşim ve kişisel arkadaşlığın geliştirilmesinde değerlendirilmekteydi. Din ve cinsel eğitim, ebeveynlerin görevi olarak evin oluşturduğu özel yaşam kapsamında görülmekteydi.
\vs p074 7:3 Bu okullardaki eğitim şu hususlardaki öğrenimi kapsamaktaydı:
\vs p074 7:4 1.\bibnobreakspace Beden sağlığı ve bakımı.
\vs p074 7:5 2.\bibnobreakspace Toplumsal etkileşimin ortak ölçütü olarak altın kural.
\vs p074 7:6 3.\bibnobreakspace Bireysel hakların topluluk hakları ve toplum ödevleri ile olan ilişkisi.
\vs p074 7:7 4.\bibnobreakspace Çeşitli dünya ırklarının tarihi ve kültürü.
\vs p074 7:8 5.\bibnobreakspace Dünya ticaretini ilerletme ve geliştirme yöntemleri.
\vs p074 7:9 6.\bibnobreakspace Birbirine tezat teşkil eden görevler ve duyguların eş\hyp{}güdümü.
\vs p074 7:10 7.\bibnobreakspace Fiziksel kavga yerine oyun, mizah ve rekabete dayalı emsallerinin geliştirilmesi.
\vs p074 7:11 Okullar, gerçekte Cennet Bahçesi’nin her etkinliği, ziyaretçilere her zaman açıktı. Silah taşımayan gözlemciler, Cennet Bahçesi’ne yapacakları kısa ziyaretleri için herhangi bir kısıtlama olmadan kabul edilmekteydiler. Cennet Bahçesi’nde ikamet edebilmek için bir Urantia’lı unsurun “evlatlık edilmesi” gerekmekteydi. Bu bireyin; Âdemsel bahşedilmenin amaç ve gayesine dair eğitimleri alması, bu göreve bağlı kalmada niyetini ifade etmesi, ve bunun sonrasında ise Âdem’in toplumsal yönetimine ve Kâinatın Yaratıcısı’nın ruhsal egemenliğine olan sadakatini bildirmesi gerekmekteydi.
\vs p074 7:12 Cennet Bahçesi’nin yasaları; Dalamatia’nın eski hükümlerine dayanmakta olup, yedi başlık altında sunulmuştu:
\vs p074 7:13 1.\bibnobreakspace Sağlık ve temizlik kanunları.
\vs p074 7:14 2.\bibnobreakspace Cennet Bahçesi’nin toplumsal yönergeleri.
\vs p074 7:15 3.\bibnobreakspace Alış\hyp{}veriş ve ticaret yönetmelikleri.
\vs p074 7:16 4.\bibnobreakspace Adil oyun ve rekabet kanunları.
\vs p074 7:17 5.\bibnobreakspace Ev yaşamı kanunları.
\vs p074 7:18 6.\bibnobreakspace Altın kuralın toplumsal yasaları.
\vs p074 7:19 7.\bibnobreakspace Yüce ahlaki yönetimin yedi emri.
\vs p074 7:20 Cennet Yaşamı’nın ahlak kanunu, Dalamatia’nın yedi emrinden biraz daha farklıydı. Ancak Âdem unsurları, bu emirlere birçok yeni başlık ekleyip öğrettiler; bu konuda bir örnek, öldürmeye karşı kesin yasaklayıcı emir alanında verilebilir: ikamet eden Düşünce Düzenleyicisi’nin varlığı, insan yaşamının yok edilmemesinde ilave bir neden olarak sunulmuştu. Onlar, “her kim bir insanın kanını akıtırsa, onun da kanı akıtılmalıdır; çünkü insan Tanrı’nın görüntüsünde yaratılmıştır” öğretisini aktarmışlardır.
\vs p074 7:21 Cennet Bahçesi’nin ibadet saati öğlendi; güneş batımı, ailenin ibadet saatiydi. Âdem, etkin bir duanın “ruhun arzusu” halinde olması gereken bir biçimde bütünüyle bireysel kalma zorunluluğunu öğreterek, belirli kalıplara oturtulmuş duaların edilmesinden insanları vazgeçirmek için elinden gelini yaptı; ancak Cennet Bahçesi sakinleri, Dalamatia döneminden beri süre gelen duaları ve bilindik yöntemleri kullanmaya devam etti. Âdem aynı zamanda, dinsel törenler için kan akıtılarak kurbanlık verilmesi yerine topraktan elde edilen meyvenin paylaşılma uygulamasını getirmek için çabaladı; ancak, Cennet Bahçesi’nin karmaşasından önce bu alanda çok az ilerleme kaydetti.
\vs p074 7:22 Âdem, ırklara cinslerin eşit olduğunu öğretmeye çalıştı. Havva’nın eşinin yanı başında onunla beraber çalışması, Cennet Bahçesi içindeki her sakin üzerinde derin bir etki bıraktı. Âdem detaylı bir biçimde, yeni bir canlıyı dünyaya getirmek için bütünleşen yaşam etkenlerine erkek ile eşit ölçüde kadının da katkıda bulunduğunu öğretti. Bu vakte kadar insanlar, doğumların tümünün “babanın kasıklarından” dünyaya geldiğini varsaymışlardı. Onlar anneye yalnızca, doğmamışların dünyaya gelmesi ve yeni doğmuşların büyütülmesi için araç gözüyle bakmışlardı.
\vs p074 7:23 Âdem çağdaşlarına kavrayabilecekleri her şeyi öğretmişti; ancak göreceli olarak bakıldığında bu öğretiler çok da fazla değildi. Yine de, dünyanın daha fazla us sahibi ırkları, eflatun ırkının üstün evlatlarına karışarak onlarla evlenmelerine izin verilecekleri vakti iple çekmektelerdi. Ve ırkların gelişen bir biçimde canlandırılmasına dair bu tasarım uygulanabilseydi Urantia ne de farklı bir dünya olurdu! Böyle sonuçlanmamış olsa bile, evrimsel toplulukların kazara kurtarmış oldukları bu aktarılmış ırkın küçük miktardaki kanından büyük kazançlar elde edilmiştir.
\vs p074 7:24 Ve Âdem, kısa süreli ikamet dünyasının refahı ve canlandırılması için bu şekilde çalışmıştır. Ancak bu karma ve melez toplulukları daha iyi bir yöne çekecek şekilde yönlendirmek zor bir görevdi.
\usection{8.\bibnobreakspace Yaratım’a Dair Efsane}
\vs p074 8:1 Urantia’nın altı günde yaratıldığına dair hikâye, Âdem ve Havva’nın Cennet Bahçesi için yaptıkları ilk incelemelerin yalnızca altı gün sürmüş olmasına dair tarihe dayanmaktaydı. Bu durum, ilk olarak Dalamatia unsurları tarafından getirilen bir biçimde, haftanın bu gününün neredeyse kutsal bir biçimde ayrılmasına sebebiyet vermişti. Âdem’in Cennet Bahçesi’ni teftişte ve başlangıç tasarımlarını hazırlamakla harcadığı altı gün önceden hesaplanmamıştı; bu süreç günden güne işleyerek bu bütünlüğe ulaştı. Yedinci günü ibadet için tercih etme eylemi, burada anlatılmış bilgiler ile ilişkili bir biçimde bütünüyle tesadüfîydi.
\vs p074 8:2 Dünyanın altı günde yaratıldığına dair efsane sonradan ortaya çıkmış bir düşünceydi; gerçekte bu düşünce otuz bin yıldan daha fazla bir süre sonrasında türemiştir. Güneş ve ayın birden ortaya çıkışı biçiminde anlatılan hikâyenin bir kısmı; güneş ve ayı uzun bir süre boyunca gölgeleyen ufak maddelerden meydana gelmiş yoğun bir uzay bulutunun bir seferliğine ansızın ortaya çıkışına dair tarihi gerçeklerden kaynağını almış olabilir.
\vs p074 8:3 Havva’nın Âdem’in kaburgasından yaratılmış olduğuna dair anlatım, Âdem’in varışına ek olarak dört yüz elli bin yıldan fazla bir süre önce Gezegensel Prens’in bedensel görevlilerin gelişiyle ilişkili yaşam özlerinin değişimde gerçekleşen göksel ameliyata dair iç içe geçmiş bir kafa karışıklığıdır.
\vs p074 8:4 Urantia’ya ulaştıklarında onlar için yaratılmış fiziksel bedenlere Âdem ve Havva’nın sahip olduğu tarihi gerçeğinden dünya insanlarının büyük bir çoğunluğu etkilenmiştir. İnsan’ın topraktan yaratıldığına dair inanç dünyanın Doğu Yakası’nda neredeyse ortak bir kabuldür; bu inanış geleneğin izleri, Filipin Adaları’ndan dünya çevresi boyunca Afrika’ya kadar takip edilebilir. Ve birçok insan topluluğu; evrim biçimindeki ilerleyici yaratıma dair öncül inanışlar dâhilinde, özel yaratımın bir çeşidi vasıtasıyla insanın toprak kökeninden gelişi ilgili bu hikâyeyi kabul etti.
\vs p074 8:5 Dalamatia ve Cennet Bahçesi’nin etkileri dışında insanlık, insan ırkının kademeli olarak yükselişine dair inanca meyletmiştir. Evrime dair bilgi çağdaş bir keşif değildir; eski dönemlerin insanları, insan ilerleyişinin yavaş ve evrimsel kimliğini anladılar. Antik Yunan bireyleri, Mezopotamya’ya olan uzaklıklarına rağmen bu türden gelişime dair kesin düşüncelere sahiptiler. Her ne kadar dünyanın çeşitli ırkları evrime dair düşüncelerinde üzücü bir biçimde kafa karışıklığına düşmüşseler de, ilkel kabilelerin birçoğu, buna rağmen, çeşitli hayvanlardan türemiş olduklarına inanıp bunu öğrettiler. İlkel insan toplulukları, geldiklerini varsaydıkları hayvan soyunu simgeleyen “totemlerini” seçme uygulamasında bulundular. Belirli Kuzey Amerika Kızılderili kabileler, kunduzlardan ve bozkır kurtlarından geldiklerine inandılar. Belirli Afrika kabileleri sırtlanların, bir Malay kabilesi lemurların, ve bir Yeni Gine topluluğu ise papağanların soyları olduklarını üyelerine öğrettiler.
\vs p074 8:6 Âdem unsurlarının sahip olduğu medeniyetin kalıntıları ile doğrudan ilişki halinde oldukları için Babilliler, insanın yaratımına dair hikâyeyi genişletip onu süsledi; onlar insanın doğrudan bir biçimde tanrılardan geldiğini öğretti. Onlar, topraktan yaratılma savı ile bile bağdaşmayan ırklarının aristokratik bir kökene sahip olduğuna dair inancı beslediler.
\vs p074 8:7 Eski Ahit’in yaratıma dair anlatımı, Musa’nın varlığından çok sonraki bir dönemde başlamıştır; ancak Musa hiçbir zaman, İbraniler’e bu türden çarpıtılmış bir hikâye öğretmemiştir. Ancak o, İsrail’in Koruyucu Tanrı’sı olarak adlandırdığı Kâinatın Yaratıcısı olan Yaratan’a ibadet edilmesi için var olan isteğini güçlendirmesini umarak, İsrailoğullarının yaratımına dair basit ve yoğun bir anlatımı sunmuştur.
\vs p074 8:8 Öncül öğretilerinde Musa oldukça bilge bir biçimde, Âdem’in zamanına giderek kaynaklık göstermeye teşebbüs etmedi; ve Musa İbraniler’in yüce öğretmeni olduğu için, Âdem’e dair anlattığı hikâyeler içkin bir biçimde sadece yaratımla ilgili olanlar haline gelmişti. Âdem\hyp{}öncesi medeniyeti tanıyan daha önceki tarihi gerçekliklerin varlığını; Âdem’in döneminden önceki insan olaylarına dair her türlü bilgiyi silme gayesi içindeki daha sonraki düzelticilerin, karısını elde ettiği yer olan “Nod yerleşkesine” yaptığı Kabil’in göçüne dair dedikodusal atfı ortadan kaldırmayı gözden kaçırmış oldukları açıkça ortaya konulmaktadır.
\vs p074 8:9 Filistin'e ulaştıktan sonra uzunca bir süre İbraniler, ortak olarak kullanılan yazılı bir dile sahip değillerdi. Onlar bir alfabeyi kullanmayı, daha gelişmiş bir uygarlık olan Girit’ten gelen siyasi mülteciler halindeki komşu Filistinliler’den öğrenmişti. İbraniler yaklaşık olarak M.Ö. 900 yılı yakınlarına kadar çok az şey yazmışlar, ve bu türden geç bir tarihe kadar da hiçbir yazılı dile sahip olmamışlardı; onlar kulaktan kulağa aktarılan halde yaratıma dair birçok farklı hikâyeye sahiplerdi, ancak Babiller’in esaretinden sonra bu hikâyeler arasından dönüşüme uğramış bir Mezopotamya türünü kabul etmeye daha meyilli hale geldiler.
\vs p074 8:10 Musevi tarih geleneği Musa üzerinde yoğunlaştı; ve Musa İbrahim’in neslini Âdem’e kadar sürmeye çabaladığı için, Museviler Âdem’in tüm insanlığın ilk bireyi olduğunu varsaydılar. Yehova Yaratan’dı; ve Âdem ilk insan olarak varsayıldığı için, dünyayı Âdem’in hemen öncesinde yapmış olmalıydı. Ve bunun sonrasında Âdem’in altı günlük tarihi gerçeği bu anlatımın içine kaynaştı; ve sonuç olarak, Musa’nın dünya üzerindeki ikametinden yaklaşık olarak bin yıl sonra dünyanın altı günde yaratıldığına dair anlatım geleneği yazıya geçti ve bunun hemen sonrasında bu bilgi Musa’ya atfedildi.
\vs p074 8:11 Musevi din adamları Kudüs’e döndüklerinde, her şeyin başlangıcına dair anlatımlarını kaleme almayı çoktan tamamlamışlardı. Daha sonra onlar, bu anlatımın Musa tarafından yazılmış olan yakın bir zamanda keşfedilmiş yaratım hikâyesi olduğunu öne sürdüler. Ancak M.Ö. 500 yılı yakınlarında yaşayan İbraniler, bu yazılanları kutsal açığa çıkarışlar olarak değerlendirmediler; onlar, daha sonraki insanların mitolojik anlatımlara yaklaşımlarına benzer bir biçimde, onları değerlendirdiler.
\vs p074 8:12 Musa’nın öğretileri olduğu varsayılan bu sahte belge Mısır’ın Yunan kralı Batlamyus’a tanıtıldı; bu kral, İskenderiye’deki yeni kütüphanesi için belgeyi Yunancaya yetmiş âlimden oluşan bir komisyon tarafından çevirttirdi. Ve böylece bu anlatım, İbrani ve Hıristiyan dinlerinin “kutsal yazıtlarına” ait daha sonraki dönemlerde toplanmış kaynakların ileride bir parçası haline gelen bu yazılar içinde yerini almıştır. Ve bahse konu din inanış düzenleri ile ilişkilendirilerek bu türden kavramlar uzun bir süre boyunca birçok Batılı insan topluluğunu derin bir biçimde etkilemiştir.
\vs p074 8:13 Hıristiyan öğretmenleri, insan ırkının emirle yoktan var edildiğine dayanan inancı koruyup yaşattılar; ve tüm bunların hepsi doğrudan bir biçimde, bir zamanlar mevcut bulunmuş olası en yüksek mutluluğun altın çağı savına ek olarak toplumun bu yüksek mertebeden inişinin sebebi olan insanın veya üstün insanın çöküşüne dair kuramın yaratılmasına sebebiyet vermişti. Yaşama ve insanın evrendeki konumuna yönelik bu türden bakış açıları en iyi ihtimalle yıldırıcı bir etkiyi beraberinde getirmekteydi; çünkü onlar böylece, bir zamanlar görevde bulunmuş belirli evren yöneticilerin hatalarının cezasını ödetmek için insan ırkına öfke kusan intikam içindeki bir İlahiyat’ın varlığını ima eden bir biçimde, ilerlemeden çok gerilemeye dair bir inancı anlamlı görerek ona yöneldiler.
\vs p074 8:14 “Altın çağ” bir mittir; ancak Cennet Bahçesi tamamiyle gerçek olup, Bahçe medeniyeti gerçekte ortadan kaldırılmıştı. Âdem ve Havva; Havva’nın sabırsızlığı ve Âdem’in hatalı kararları nedeniyle, hızlı bir biçimde kendilerine felaket getirerek ve Urantia’nın tümün gelişimsel ilerleyişine zarar verici gerilemeye sebebiyet vererek emredilen gidişattan ayrılmaya teşebbüs ettikleri vakte kadar yüz on yedi yıl Cennet Bahçesi’nde görevlerine devam etmişlerdi.
\vs p074 8:15 [Bu anlatım, “Bahçe’nin {yüksek} meleksel sesi” olan Solonia tarafından gerçekleştirilmiştir.]
