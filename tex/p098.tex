\upaper{98}{Batıdaki Melçizedek Öğretileri}
\vs p098 0:1 Melçizedek öğretileri Avrupa’ya, birçok rota üzerinden girmişti; ancak başlıca olarak onlar, Mısır istikametinden gelip, bütünüyle Helenleşmesini ve daha sonra Hıristiyanlaşmasını takiben Batı felsefesi bünyesine nüfuz etmişti. Batı dünyasının idealleri özü bakımından Sokratik’di; ve Batı’nın daha sonraki dini felsefesi, hepsinin Hıristiyan din\hyp{}kurumu içinde bir araya geldiği şekliyle, evrimleşen Batı felsefesi ve dini ile olan iletişim vasıtasıyla değiştirilmiş ve ödün verilmiş olan İsa’nın felsefesi haline gelmişti.
\vs p098 0:2 Uzunca bir süre boyunca Avrupa’da Salem din\hyp{}yayıcıları; dönemsel olarak ortaya çıkan, inanışlar ve ayinsel topluluklarının birçoğunu kademeli olarak içine katan bir biçimde etkinliklerini sürdürmüşlerdi. En saf bütünlüğü içinde Salem öğretilerini muhafaza edenler arasında Kinikler’den bahsedilmelidir. Daha sonra yeni yeni oluşum içindeki Hıristiyan dinine katılan bir biçimde Tanrı’ya olan inanç ve güvenin bu duyurucuları hala, İsa’dan sonraki ilk yüzyılda Roma Avrupası’nda faal haldeydi.
\vs p098 0:3 Salem inanç savının büyük bir kısmı Avrupa’da, Batıdaki askeri mücadelelerinin birçoğunda savaşmış Musevi kökenli paralı askerler tarafından yayılmıştı. İlkçağ dönemlerinde Museviler, din\hyp{}kuramsal tuhaflıkları kadar askeri yiğitlikleri ile ünlülerdi.
\vs p098 0:4 Yunan felsefesi, Musevi din\hyp{}kuramı ve Hıristiyan etik kurallarının temel inanç savları özünde, öncül Melçizedek öğretilerinin sonuçlarıydı.
\usection{1.\bibnobreakspace Yunanlılar Arasındaki Salem Dini}
\vs p098 1:1 Salem din\hyp{}yayıcıları; ibadetin ayrıcalıklı topluluklarının örgütlenmesini yasaklayan ve her öğretmenden, sadece yiyecek, giyecek ve barınak dışında hiçbir zaman dini hizmet için ücret almamasını isteyen bir biçimde hiçbir zaman bir din\hyp{}adamı olarak faaliyet göstermemesi sözünü şart koşan Maçiventa tarafından getirilen bir taahhüt olarak, din hizmetine olan kabul antlarını katı bir biçimde yorumlamış olmasalardı, Yunanlılar arasında büyük bir dini yapı inşa edilmiş olabilirdi. Melçizedek öğretmenleri Helen\hyp{}öncesinin Yunanistanı’na girdiğinde, hala Âdemoğlu’na dair tarihsel anlatımlara ek olarak And unsurlarının dönemlerini teşvik eden bir topluluk bulmuştu; ancak bu öğretiler, artan sayıda Yunan sahillerine getirilmekte olan alt düzeydeki köle topluluklarının görüşleri ve inanışlarıyla fazlasıyla bozulmuş bir halde bulunmaktaydı. Bu bozulma, alt düzeydeki sınıfların mahkûm edilen suçluların idamından bile tören yaptığı biçimde, kanlı ayinler ile birlikte her şeyde ruhun bulunduğunda dair olgunlaşmamış bir inanca doğru bir geri dönüş yaratmıştı.
\vs p098 1:2 Salem öğretmenlerinin öncül etkisi, güney Avrupa ve Doğu’dan gelen Ari olarak adlandırılmaktaki işgal ile neredeyse tamamen yok edilmişti. Bu Helen istilacıları kendileri ile birlikte, Ari akranlarının daha öncesinden Hindistan’a taşımış olduklarınınkine benzer insansı Tanrı kavramsallaşmaları getirmişlerdi. Dışarıdan gelen bu aktarım, Yunan tanrı ve tanrıça ailesinin evrimini başlatmıştı. Bu yeni din bir ölçüde, gelmekte olan Helen kökenli barbarların inanışlarına dayanmıştı; ancak o aynı zamanda, Yunanistan’ın eski sakinlerine ait mitleri de içinde barındırmaktaydı.
\vs p098 1:3 Helen Yunanlıları Akdeniz dünyasını büyük ölçüde anne inanışının egemenliğinde buldu; ve onlar bu Akdeniz topluluklarına, diğer ilahiyatların mevcudiyetini dışlamayan ancak tek tanrıya inanan Sami toplukları arasındaki Yahveh gibi, bağımlı tanrılardan oluşan bütüncül bir Yunan tanrı birliğinin başı haline çoktan gelmiş olan Gökyüzü\hyp{}Zeus’u olarak insan tanrılarını dayatmışlardı. Ve Yunanlılar nihai olarak, Kader’in üstün\hyp{}denetimine dair düşüncelerine bağlı kalmasalardı, Zeus’un kavramsallaşması içinde gerçek bir tektanrı dinini kazanacaklardı. Nihai değere sahip bir tanrı, kendi başına, kaderin karar vericisi ve nihai sonun yaratıcısı olmak durumundadır.
\vs p098 1:4 Dini evrim içindeki bu etkenlerin bir sonucu olarak orada yakın zaman içerisinde; kutsaldan çok insan olan ve us sahibi Yunanlılar’ın hiçbir zaman çok fazla ciddiye almadığı, Olimpos Dağı’nın tasasız tanrılara olan yaygın inanış gelişti. Onlar, kendi yaratımları olan bu kutsallıkları ne fazlasıyla sevip, ne de ondan fazlasıyla korkmuşlardı. Onlar Zeus ve onun yarı insan, yarı tanrı ailesi için yurtsever ve ırksal bir duygu beslemektelerdi; ancak neredeyse hiçbir şekilde onlara büyük saygı beslemiş veya ibadet etmişlerdi.
\vs p098 1:5 Helen unsurları; öncül Salem öğretmenlerinin dinadamlığı\hyp{}kurumunun\hyp{}yaptıkları\hyp{}şeylere\hyp{}karşıt inanç savını o derece özümsemişlerdi ki, herhangi bir öneme sahip hiçbir dinadamlığı kurumu Yunanistan’da doğmamıştı. Tanrılar için resimler yapmak bile, bir ibadet durumundan çok bir sanatsal faaliyetti.
\vs p098 1:6 Olimposlu tanrılar, insanın örnek insansı\hyp{}nitelikler\hyp{}atfedişini resmetmektedir. Ancak Yunan mitolojisi etiksel nitelikten çok estetikseldi. Yunan dini, bir ilahiyat topluluğu tarafından bir evrenin yönetildiğini tasvir edişi bakımından yararlıydı. Ancak, Yunan ahlaki değerleri, etik kuralları ve felsefesi yakın zaman içinde tanrı kavramsallaşmasının çok ötesine doğru gelişme gösterdi; ve, ussal ve ruhsal büyüme arasındaki bu orantısızlık, Hindistan’a gerçekleştiği kadar Yunanistan için yıkıcı oldu.
\usection{2.\bibnobreakspace Yunan Felsefi Düşüncesi}
\vs p098 2:1 Çok ciddiye alınmayan ve yapay olan bir din; özellikle, adetlerini teşvik edecek ve takipçilerinin kalplerini korku ve huşu ile dolduracak herhangi bir dinadamlığı kurumuna sahip olmadığında, varlığını sürdüremez. Olimposlu din, kurtuluşun sözünü vermedi; buna ek olarak, inananlarının ruhsal susursuzluğunu gidermedi; bu nedenle yok olmaya mecburdu. Kurumsal başlangıcının bin yılı içerisinde neredeyse tamamen ortadan kaybolmuştu; ve Yunanlılar, Olimposlu tanrıların daha iyi akıllara olan temelini yitirişiyle, milli bir dinden mahrum kalmıştı.
\vs p098 2:2 Bahse konu olaylar; İsa’dan önceki altıncı yüzyıl boyunca, Doğu ve Levant, ruhsal bilincin bir yeniden canlanışına ek olarak tektanrı inancının tanınmasına olan yeni bir uyanışını deneyimlediğindeki durumdu. Batı bu yeni gelişimi paylaşmamıştı; ne Avrupa ne de kuzey Afrika geniş bir biçimde, bu dini yeniden doğuşa katılmamıştı. Yunanlılar, buna rağmen, muhteşem bir ussal ilerlemeye girişmişti. Onlar korkunun üstesinden gelmeye hali hazırda başlamış olup, artık dini bunun bir panzehiri olarak arzulamamaktaydılar; ancak onlar gerçek dini ruh açlık, ruhsal kaygı ve ahlaki çaresizlik için ilaç olduğunu algılamamışlardı. Onlar ruhun tesellisini --- felsefe ve metafizik olarak --- derin düşünmede aramışlardı. Onlar, bireyin korunuşundan --- kurtuluşundan, bireyin kendisini gerçekleştirmesi ve kendisini anlaması üzerinde düşünmeye yönelmişlerdi.
\vs p098 2:3 Titiz düşünceyle Yunanlılar, kurtuluşa olan inancın bir muadili olarak hizmet verecek güvencenin bilincine erişmeye çabaladılar; ancak onlar bunda tamamen başarısız oldular. Helen topluluklarının daha yüksek sınıfları içinde sadece daha us sahibi olanlar bu yeni öğretiyi kavrayabildi; eski nesillerin kölelerine ait soylardan gelen alt düzeydeki bireyler, dinin yerini alan bu düşünceyi algılamak için hiçbir yetiye sahip değildi.
\vs p098 2:4 Filozoflar; her ne kadar neredeyse hepsi, “evrenin Usu,” “Tanrı düşüncesi,” ve “Büyük Köken’e” dair Salem din savına olan bir inanışın temeline bir kenarından bağlı olsa da, ibadetin her türünü küçümsemişlerdi. Kutsal ve sınırlı\hyp{}olanın\hyp{}ötesini tanımaları düzeyinde, Yunan filozofları içten bir biçimde tektanrıcıydı; onlar, Olimposlu tanrılar ve tanrıçaların bütüncül gökadasını çok küçük bir tanımada bulunmuştu.
\vs p098 2:5 Beşinci ve altıncı yüzyılın Yunan şairleri, dikkate değer bir biçimde Pindar, Yunan dininin köklü değişimine girişti. Onlar bu dinin ideallerini yükseltti; ancak onlar, dindarlardan çok sanatçılardı. Yüce değerleri teşvik ve muhafaza etmek için bir yöntem geliştirmede başarısız oldular.
\vs p098 2:6 Ksenofanes, tek Tanrı’yı öğretti; ancak onun ilahiyat kavramsallaşması, kişisel bir Yaratıcı olmayacak kadar haddinden fazla bir biçimde Tanrı’nın her şeyde barındığına inanmaktaydı. Anaksagoras, bir İlk Akıl olarak bir İlk Sebep’i tanıması dışında, her şeyin yalnızca sebep sonuç ilişkilerinin sonucunda gerçekleştiğine inanan biriydi. Sokrates ve onun varisleri olan Plato ve Aristo, erdemin bilgi olduğunu öğretti; iyiliğin, ruhun sağlığı olduğuna; adaletsizlikten zarar görmenin suçlu olmaktan daha iyi olduğu, kötülüğe kötülükle cevap vermenin yanlış olduğu, ve tanrıların bilge ve iyi olduğuna. Onların en önemli gördükleri erdemleri şunlardı: bilgelik, cesaret, ölçülülük ve adaletti.
\vs p098 2:7 Dini felsefenin Helen ve İbrani toplulukları içindeki evrimi, kültürel ilerlemeyi belirleyen bir kuruluş olarak din\hyp{}kurumunun faaliyetinin tezat bir görünümünü sergilemektedir. Filistin’de insan düşüncesi o kadar dinadamı temelli denetlenmekte ve yazıtlar ile yönetilmekteydi ki, felsefe ve estetik değerler bütünüyle din ve ahlakın tarafından yutulmuştu. Yunanistan’da, dinadamları ve “kutsal yazıtların” neredeyse bütüncül yoksunluğu, derin düşüncenin şaşırtıcı bir gelişmesiyle sonuçlanan bir biçimde insan aklını özgür ve bağımsız kıldı. Ancak kişisel bir deneyim olarak din, kâinatın doğası ve gerçekliğine doğru yapılan ussal araştırmaların hızıyla başa çıkmada başarısız oldu.
\vs p098 2:8 Yunanistan’da inanmak düşünmeye tabi kılınmıştı; Filistin’de düşünmek, inanmaya bağlanmıştı. Hıristiyanlığın iyi olduğu tarafların büyük bir kısmı, hem İbrani ahlakından hem de Yunan düşüncesinden fazlasıyla beslenmiş oluşundan kaynaklanmaktadır.
\vs p098 2:9 Filistin’de dini dogma o kadar katılaşmıştı ki, daha ileri büyümeyi riske atmıştı; Yunanistan’da insan düşüncesi o kadar soyut bir hal almıştı ki, Tanrı’nın kavramsallaşması kendisini, Brahman filozoflarının birey\hyp{}dışı Sınırsızlığı’ndan çok da farklı olmayan, her şeyin temelinde Tanrı’nın yattığı düşüncesinin sisli bir buharına dönüşmüştü.
\vs p098 2:10 Ancak bu dönemlerin ortalama insanı; Yunan felsefesinin sahip olduğu bireyin kendisini gerçekleştirişine ek olarak soyut bir İlahiyat’ı ne kavrayabildi, ne de ona ilgi besledi; onlar bunun yerine, dualarını duyabilecek kişisel bir Tanrı ile birlikte kurtuluş sözlerini arzulamışlardı. Onlar; filozoflarını sürgüne göndermişler, Salem inanışının, ki her iki inanış savı bu dönemde oldukça iç içe gelmiştir, geride kalanlarını idam etmişler, ve, bu dönemlerde Akdeniz adalarına haddinden fazla yayılan gizem inanışlarının budalalıklarına doğru korkunç sefahat eğlencelerine dalmasının zemini hazırlamıştı. Eleusinian gizemleri, bereketliliğe olan bir Yunan türü ibadeti biçiminde Olimposlu tanrı birliği içinden doğmuştu; Dionysos doğa ibadeti yeşermişti; bu inanışların en iyisi, ahlaki salıkları ve kurtuluş sözleri birçoklarının büyük ilgisini çeken Orfik kardeşlikti.
\vs p098 2:11 Yunanistan’ın tamamı, bu duygusal ve ateşli törenler biçimindeki, kurtuluşa ermenin bu yeni yöntemlerine katılmıştı. Bu döneme kadar hiçbir millet, bu kadar kısa bir süre içinde sanatsal felsefenin bu gibi doruklarına erişmemişti; bu döneme kadar hiçbir topluluk, neredeyse tamamen İlahiyat olmadan ve insan kurtuluşunun sözünden bütünüyle mahrum olarak, etik kurallarının bu türden gelişmiş bir düzenini yaratamamıştı; bu döneme kadar hiçbir millet, gizem inanışlarının ateşli girdabına kendilerini serbest bıraktıklarında bu Yunan topluluklarının deneyimlediği gibi, ussal durgunluğun, ahlaksal bozukluğun ve ruhsal fakirliğin bu türden derinlerine oldukça çabuk, derinlemesine ve şiddetle batmamıştı.
\vs p098 2:12 Dinler uzunca bir süre boyunca, felsefi destek olmadan varlıklarını sürdürmüşlerdir; ancak bu gibi çok az felsefi düzen, düşünülenin aksine, din ile belli bir ilişkilendirilimi olmadan uzunca bir süre varlığını sürdürebilmiştir. Düşünme karşısında eylem ne ise, din karşısında felsefe odur. Ancak olası en yüksek insan bütünlüğü; içinde felsefenin, dinin ve bilimin bilgelik, inanç ve deneyimin bir araya gelmiş eylemi tarafından anlamlı bir bütünlüğe doğru kaynaştığı durumdur.
\usection{3.\bibnobreakspace Roma’daki Melçizedek Öğretileri}
\vs p098 3:1 Aile tanrılarına olan ibadetin daha öncül dini türlerinden, savaş tanrısı olan Mars’a duyulan derin kabile hürmetine doğru evirilmiş bir biçimde, Latin topluluklarının sahip oldukları daha sonraki din; Yunan ve Brahman ussal sistemlerinden veya diğer birkaç topluluğun daha ruhsal dinlerinden daha çok siyasi bir gelenek niteliğindeydi.
\vs p098 3:2 İsa’dan önceki altıncı yüzyıl boyunca Melçizedek müjdesinin büyük tektanrılı yeniden doğuşu süreci içerisinde, Salem din yayıcılarının çok az İtalya’ya girmişti; ve bunu gerçekleştirenler, hepsinin daha sonra Roma devlet dinine doğru örgütlendiği, tanrıları ve tapınaklarından oluşan yeni bir gökadasıyla birlikte hızlıca yayılan Etrüsk dinadamlığının etkisinin üstesinden gelmede başarısız olmuşlardı. Latin kabilelerinin bu dini, Yunanlılar sahip oldukları gibi ciddiyetten uzak ve satın alınabilir değildi; buna ek olarak onlar, İbranilerinkiler gibi sert ve baskıcı değildi; onun büyük bir kısmı tekil adetlerden, yeminlerden ve tabulardan meydana gelmişti.
\vs p098 3:3 Roma dini fazlasıyla, Yunanistan’dan alınan geniş kültürel aktarımlardan etkilenmişti. Nihai olarak Olimpos tanrılarının büyük bir kısmı, Latin tanrı birliğine aşılanmış ve eklemlenmişti. Yunan toplulukları uzunca bir süre boyunca aile ocağının ateşine ibadet etmişlerdi --- Hestia, ocağın bakir tanrıçasıydı; Vesa, ev kurumunun Roma tanrıçasıydı. Zeus Jüpiter oldu; Afrodit ise Venüs; ve bu dönüşüm birçok Olimpos ilahiyatını takip etti.
\vs p098 3:4 Roma gençlerinin dine olan kabulü, devlet hizmetine olan, ciddiyetle gerçekleştirilen atanımları olayıydı. Vatandaşlık antları ve kabulleri, gerçekte dini törenlerdi. Latin topluluklar; tapınakları, sunakları ve mabetleri muhafaza ettiler; ve bir kriz durumunda kâhinlere danışırlardı. Onlar, kahramanların kemiklerini saklamışlardı, ve bunu daha sonra Hıristiyan azizlerininki için yaptılar.
\vs p098 3:5 Dini görünümde olan vatanseverliğin bu resmi ve duygusal olmayan türü çökmeye mahkûmdu, Yunanlılar’ın oldukça ussal ve sanatsal ibadeti bile, gizem inanışlarının hararetli ve oldukça fazla duygusal ibadeti karşısında çökmüştü. Bu zarar verici inanışların en büyüğü; Roma’da şu anki Aziz Peter kilisesinin bulunduğu yerleşkenin tam da üzerinde yönetim merkezine sahip olmuş olan, Anne Tanrı mezhebinin gizem diniydi.
\vs p098 3:6 Ortaya çıkış halindeki Roma devleti siyasetle fetihlerini gerçekleştirdi; ancak o, Mısır’ın, Yunanistan’ın ve Levant’ın inanışları, ayinleri, gizemleri ve tanrı kavramsallaşmalarıyla fethedildi. Bu dışarından aktarılan inanışlar; tamamiyle siyasal ve şehirsel nedenlerle, gizemleri ortadan kaldırma ve eski siyasi dini tekrar canlandırmanın kahramanca ve bir açıdan başarılı bir çabasını vermiş Augustus’un zamanına kadar Roma devleti boyunca yeşermeye devam etti.
\vs p098 3:7 Devlet dininin dinadamlarından bir tanesi Augustus’a; Salem öğretmenlerinin, doğa\hyp{}üstü varlıkların tümü üzerinde yönetimde bulunan nihai bir İlahiyat olarak tek Tanrı’nın öğretisini yaymadaki öncül çabalarından bahsetti; ve bu düşünce imparator üzerinde o kadar güçlü bir etkide bulundu ki, imparator birçok tapınak inşa edip, hepsini güzel resimlerle donatıp, devletin dinadamlık kurumunu yeniden düzenleyip, devlet dinini yeniden belirleyip, kendisini herkesin temsili en yüksek din adamı olarak atayıp, bunları yaparken de kendisini yüce tanrı olarak duyurmada tereddüt etmemişti.
\vs p098 3:8 Augustus ibadetinin bu yeni dini yeşermiş olup, Museviler’in evi olan Filistin dışında, yaşamı boyunca imparatorluğun tümü üzerinde gözlenmişti. Ve, insan tanrılarının bu dönemi; resmi Roma inanışı, her birinin mucizevî doğumlara ve diğer insan\hyp{}üstü niteliklere sahip olduklarını ifade ettiği, kırktan fazla kendi kendilerini yüceltmiş insan ilahiyatlarının bir listesine sahip olana kadar devam etti.
\vs p098 3:9 Salem inananlarının sayıları azalan topluluğunun son direnişi; vahşi ve duygusuz dini ayinlerini bırakıp, Yunan felsefesiyle etkileşimi sonucunda dönüşüme uğramış ve bozulmuş olan, Melçizedek’in müjdesini içinde barındıran ibadetin bir türüne dönmelerini Roma topluluklarından şiddetle talep etmiş içten bir vaiz topluluğu olan Kinikler tarafından gerçekleştirilmişti. Ancak insanların büyük bir bölümü Kinikler’i reddetmişlerdi; onlar, yalnızca kişisel kurtuluş umudunu sunmayan ancak aynı zamanda ciddiyetten uzaklaşma, heyecan ve eğlence arzusunu tatmin eden gizemlerin ayinlerine dalmayı tercih etmişlerdi.
\usection{4.\bibnobreakspace Gizem İnanışları}
\vs p098 4:1 Greko\hyp{}Romen dünyasındaki insanların büyük bir çoğunluğu; ilkçağ ailelerini ve devlet dinlerini kaybetmelerine ek olarak Yunan felsefesinin anlamını kavramaya yetkin olmayan veya bunu yapmada gönülsüz bir konumda bulunan bir biçimde, ilgilerini Mısır ve Levant’dan muhteşem ve duygusal gizem inanışlarına çevirmişlerdi. Halk --- bugünün dini tesellisine ek olarak ölümden sonraki yaşamın devam edişine dair beslenen ümidin güvencesi olarak --- kurtuluşun sözlerini arzulamıştı.
\vs p098 4:2 En yaygın hale gelmiş üç gizem inanışı şunlardı:
\vs p098 4:3 1.\bibnobreakspace Kibele ve onun oğlu Attis’e dayanan Frig inanışı.
\vs p098 4:4 2.\bibnobreakspace Osiris ve onun annesi İsis’e dayanan Mısır inanışı.
\vs p098 4:5 3.\bibnobreakspace Günahkâr insanlığın kurtarıcısı ve bağışlayıcısı olarak Mitra ibadetine dayanan İran inanışı.
\vs p098 4:6 Frig ve Mısır gizemleri, kutsal evladın (sırasıyla Attis ve Osiris olmak üzere) ölümü önceden deneyimlemiş olduğunu ve kutsal güç tarafından yeniden diriltildiğini öğretmişti; ve buna ek olarak, olması gerektiği gibi gizeme kabul edilen ve tanrının ölümü ve yeniden dirilişinin yıl dönümünü derin saygıyla kutlayan herkesin böylelikle onun kutsal doğası ve ölümsüzlüğünün parçaları haline geleceğini öğretmişti.
\vs p098 4:7 Frig törenleri göz kamaştırıcı, ancak bozucu etkiye sahipti; onların kanlı şenlikleri, bu Levant gizemlerinin ne kadar düşkün ve ilkel olduğunu göstermektedir. En kutsal gün, Attis’in kendi sebep olduğu ölümü anan bir biçimde “kanın günü” olarak Kara Cuma’ydı. Attis’in fedası ve ölümünün kutlanışından üç gün sonra şenlik, onun yeniden doğumu onurundan duyulan neşeye dönmekteydi.
\vs p098 4:8 İsis ve Osiris ibadetinin ayinleri, Frig inanışınınkilere kıyasla daha gelişmiş ve etkileyiciydi. Mısır ayini; tüm yaşayan bitkilerin bahar dönüşümünün takip ettiği bitki büyümesinin senelik duraksayışının gözlenmesinden elde edilen kavramsallaşma olarak, ölen ve yeniden dirilen bir tanrı şeklinde eskinin Nil tanrısı efsanesi temelinde inşa edilmişti. Bu gizem inanışlarının yerine getirilmesinden doğan taşkınlık durumu ve, kutsallığın gerçekleştirilmesi sonucunda “coşkuya” götürdükleri varsayılan, törenlerinin yaydığı safahat zaman zaman olası yüksek derecede isyankâr nitelikteydi.
\usection{5.\bibnobreakspace Mitra İnanışı}
\vs p098 5:1 Frig ve Mısır gizemleri en sonunda, Mitra ibadeti olarak tüm gizem inanışlarının en büyüğüne zemin hazırladı. Mitrasal inanış; insan doğasının geniş bir kapsamına hitap etmiş olup, kademeli bir biçimde iki öncülünün de yerini aldı. Mitracılık Roma İmparatorluğu’na, bu dinin moda olduğu yer olan Levant’dan toplanmış Roma birliklerinin örgütlü yayım faaliyetleriyle yayılmıştı; zira onlar bu inanışı gittikleri her yere taşımışlardı. Ve bu yeni dini ayin, öncül gizem inanışları üstüne büyük bir gelişimdi.
\vs p098 5:2 Mitra inanışı İran’da ortaya çıkmış olup, uzunca bir süre boyunca, Zerdüşt takipçilerinin askeri karşıtlığına rağmen, anavatanında varlığını sürdürmüştü. Ancak Mitracılık Roma’ya ulaştığında, Zerdüşt’ün öğretilerinin çoğunu özümlemesiyle fazlasıyla gelişmiş hale geldi. Başlıca Mitrasal inanış aracılığıyla Zerdüşt dini, daha sonranın ortaya çıkış halindeki Hristiyanlık üzerinde bir etkiye sahip oldu.
\vs p098 5:3 Mitrasal inanış; cesur eylemlere girişen ve suyu bir kayadan oklarıyla hedeflere püskürten, bir büyük kayadan kaynağını alan bir biçimde askeri bir tanrıyı tasvir etmişti. Orada; özellikle inşa edilmiş bir kayıtla bir kişinin kaçtığı bir sele ek olarak Mitra’nın göklere yükselişinden önce güneş\hyp{}tanrısı ile kutladığı son akşam yemeği bulunmaktaydı. Bu güneş\hyp{}tanrısı, veya ismiyle Sol İnvictus, Zerdüştlüğün Ahura\hyp{}Mazda ilahiyatının bir bozuluşuydu. Mitra, karanlığın tanrısı ile verdiği mücadele de güneş\hyp{}tanrısının hayatta kalan galip savaşçısı olarak düşünülmekteydi. Ve, onun efsanevi kutsal boğayı öldürüşünün tanınışında, Mitras; gökteki tanrılar arasında, insan ırkı için onun adına af dileyen konuma yükseltilmiş bir biçimde ölümsüz kılınmıştı.
\vs p098 5:4 Bu inanışın takipçileri; mağara ve diğer gizli yerlerde ibadet edip, ilahiler söyleyip, büyü mırıldanıp, kurbanlık hayvanların etini yiyip, kan içmişlerdi. Günde üç kez onlar ibadet etmişlerdi; buna ilaveten, güneş\hyp{}tanrısının gününde özel haftasal törenlerde ve Aralık’ın yirmi beşinci günü olarak Mitra’nın yıllık şenlik gününde en detaylı adetlerle ibadetlerini yerine getirmişlerdi. Ayinlerde sunulan şeyleri yemek; yargı gününe kadar mutluluk içinde beklenilecek yer olan Mitra’nın bağrına ölümden sonra derhal geçilerek ebedi yaşamı getirdiğine inanılmaktaydı. Yargı gününde Mitra’nın gök anahtarları, inançlı olanların kabulü için Cennet’in kapılarını açacaktı; bundan hemen sonra, Mitra’nın dünyaya dönüşü üzerine kutsanmamış tüm yaşayan ve ölüler yok edilecekti. Bir insan öldüğünde, Mitra’nın önüne yargısı için çıktığı öğretilmekteydi; ve dünyanın sonunda Mitra’nın, son yargıyla yüzleşmeleri için ölülerin tümünü mezardan çağıracak olduğunu. Kötüler ateşle yok edilecek, ve doğru olanlar Mitra ile sonsuza kadar hüküm sürecek.
\vs p098 5:5 İlk başta o sadece erkekler için olan bir dindi; ve orada, inananların sırasıyla kabul edildikleri yedi farklı düzey bulunmaktaydı. Daha sonra, inananların eşleri ve kızları, Mitra mabetlerine katılmış olan Büyük Anne tapınaklarına kabul edilmişlerdi. Kadınların inanışı; Mitrasal inanış ile Attis’in annesi Kibele’ye dayanan Frig inanışının törenlerinin bir karışımıydı.
\usection{6.\bibnobreakspace Mitracılık ve Hristiyanlık}
\vs p098 6:1 Gizem inanışlarının ve Hıristiyanlık’ın gelişinden önce kişisel din, Kuzey Afrika ve Avrupa’nın medeni yerleşkelerinde bağımsız bir kurum olarak neredeyse hiçbir biçimde gelişme göstermemişti; o daha çok bir ailesel, şehir\hyp{}devletsel, siyasi ve imparatorluksak bir olaydı. Helen Yunanlıları hiçbir zaman, merkezileşmiş ibadet düzenini geliştirmediler; ayin düzeni yereldi; onlar hiçbir dinadamlığı kurumuna ve “kutsal kitaba” sahip değillerdi. Romalılar’a oldukça benzeyen bir biçimde, onların dini kurumları; daha yüksek ahlaki ve ruhsal değerlerin korunumu için güçlü bir yönlendirici kurumdan yoksundu. Her ne kadar, dinin kurumsallaşmasının genellikle ruhsal derinlikten çalarak gerçekleştiği doğru olsa da; herhangi bir dinin, az veya çok olmak üzere bir düzeyde kurumsal örgütlenmenin yardımı olmadan bu kadar gelişemeyecek olması da bir gerçektir.
\vs p098 6:2 Batı dini böylelikle Kuşkucular, Kinikler, Epikürcüler ve Stoacılar’a kadar kan kaybetmişti; ancak bu durum özellikle, Mitracılık ve Pavlus’un yeni Hristiyanlık dini arasında gerçekleşen rekabet dönemlerine kadar gerçeklik taşımaktaydı.
\vs p098 6:3 İsa’dan sonraki üçüncü yüzyıl boyunca Mitrasal ve Hıristiyan din\hyp{}kurumları, görünüşlerinde ve ayinlerinin niteliğinde birbirine oldukça benzerlerdi. İbadetin bu türden mekânlarının büyük bir çoğunluğu, yer altında bulunmaktaydı; ve iki inanışın mabetleri de, arka\hyp{}planları, günahla lanetlenmiş bir insan ırkına hâlihazırda kurtuluşu getirmiş olan kurtarıcının ızdıraplarını çeşitli biçimlerde tasvir etmiş sunakları barındırmaktaydı.
\vs p098 6:4 Mitra’ya ibadet edenlerin uygulaması her zaman, mabede girerken kutsal suya parmaklarını batırmak olmuştu. Ve, bazı bölgelerde aynı anda iki dine de ait olan kişiler bulunduğu için; bu âdeti, Roma’nın yakınlarındaki Hıristiyan kiliselerinin büyük bir kısmına tanıştırdılar. Her iki din de kutsamayı uygulamış olup, ekmek ve şaraptan oluşan ayinde bunları tüketti. Mitracılık ve Hristiyanlık arasındaki bir büyük fark, Mitra ve İsa’nın kişilikleri dışında, biri askeri zihniyeti teşvik ederken diğerinin ise aşırı derecede barışçıl olmasıydı. Mitracılık’ın diğer dinlere olan hoşgörüsü (daha sonraki Hıristiyanlık dışında olmak üzere) nihai çöküşüne neden oldu. Ancak ikisi arasındaki mücadele de belirleyici etken, kadınların Hıristiyan inancının bütüncül birlikteliğine olan kabulüydü.
\vs p098 6:5 En sonunda bu dönemin Hıristiyan inancı Batı’da baskın hale geldi. Yunan felsefesi, etik değerlerin kavramsallaşmalarını sağlamıştı; Mitracılık, ibadetin yerine getirilişinin dini adetsel düzenini; ve Hristiyanlık olarak bu dönemlerde tanımlanan inanış, ahlaki ve toplumsal değerlerin korunum yöntemini sunmuştu.
\usection{7.\bibnobreakspace Hıristiyan Dini}
\vs p098 7:1 Bir Yaratan Evlat, kızgın bir Tanrı ile uzlaşmak için fani beden suretinde vücuda bürünüp, kendisini Urantia’nın insanlığına bahşetmemişti; aslında bütün bunlar, Yaratıcı’nın derin sevgisine ve Tanrı ile olan evlatlığının kendisini gerçekleştirişine tüm insanlığı kazandırmak için meydana gelmişti. Sonuç olarak, kefaret öğretisinin büyük savunucusu bile bu gerçekliğin bir kısmını yerine getirmişti; zira o, “Tanrı’nın, dünyayı kendisiyle uzlaştıran bir biçimde İsa’nın içinde olduğunu” duyurmuştu.
\vs p098 7:2 Hıristiyan dininin kökeni ve yayılımına değinmek bu makalenin özel kapsamı değildir. Hıristiyan dininin; Urantia’da seçilmiş kişi Mesih olarak bilinen, Nebadon Mikâil Evladı’nın insan bedenine bürünmüş bireyi olan Nasıralı İsa’nın kişiliği üzerine inşa edildiğini söylemek yeterlidir. Hristiyanlık, Levant ve Batı boyunca bu Celile sakininin takipçileri tarafından yayılmıştı; ve onların din yayma azmi, Seth ve Salem unsurları olarak meşhur seleflerine ilaveten Budist öğretmenleri olan içten Asyalı çağdaşlarına eş değer düzeydeydi.
\vs p098 7:3 Hıristiyan dini, bir Urantialı inanış düzeni olarak; şu öğretilerin, etkilerin, dini düşüncelerin, inanışların ve kişisel nitelikli bireysel tutumların birleşiminden doğmuştu:
\vs p098 7:4 1.\bibnobreakspace Son dört bin yıl içinde doğmuş tüm Batı ve Doğu dinleri içinde temel bir etken olan, Melçizedek öğretileri.
\vs p098 7:5 2.\bibnobreakspace İbrani ahlak, etik ve din\hyp{}kuram düzenine ek olarak hem Yazgı hem de yüce Yahveh’e olan inanış.
\vs p098 7:6 3.\bibnobreakspace Hali hazırda hem Yehud hem de Mitra inanışına izini bırakmış olan, kâinatsal iyilik ve kötülük arasındaki mücadeleye dair Zerdüştçü kavramsallaşma. Mitracılık ve Hristiyanlık arasındaki mücadeleler sonucunda gerçekleşen uzamış iletişim süreciyle İranlı tanrı\hyp{}elçisinin din savları; İsa’nın öğretilerinin Helen ve Latinleşmiş türlerine ait dogmaların, eğilimlerin ve kâinat düşünüşünün din\hyp{}kuramsal ve felsefi içerik ve yapısını belirlemede güçlü bir etken haline geldi.
\vs p098 7:7 4.\bibnobreakspace Özellikle Mitracılık olmak üzere ama aynı zamanda Frig inanışındaki Büyük Anne ibadeti olarak gizem inanışları. Urantia’da İsa’nın doğumuna dair efsaneler bile; dünya üzerindeki başlangıcına, beklenilen bu olay hakkında melekler tarafından önceden bilgi sahibi kılınmış yalnızca az sayıdaki hediye taşıyan çobanın şehit olabildiği varsayılan İranlı kurtarıcı\hyp{}kahraman Mitra’nın mucizevî doğumunun Romalı anlatımından etkilenmişti.
\vs p098 7:8 5.\bibnobreakspace Tanrı’nın Evladı yüceltilmiş Mesih olarak Nasıralı İsa’nın gerçekliği biçiminde Yeşu bin Yusuf’un insan yaşamının tarihi gerçekliği.
\vs p098 7:9 6.\bibnobreakspace Tarsuslu Pavlus’un kişisel görüşü. Ve, Mitracılık’ın, Pavlus’un ergenliği boyunca Tarsus’un baskın dini olduğunun altı çizilmelidir. Pavlus’un; dinini değiştirdiği kişilere yazdığı iyi niyetli mektupların bir gün, çok daha sonrasının Hıristiyanları tarafından “Tanrı’nın sözü” olarak görülebileceği aklına bile gelmezdi. Bu türden iyi niyetli öğretmenler, daha sonraki varisler tarafından yazılarının kullanılmasından sorumlu tutulamazlar.
\vs p098 7:10 7.\bibnobreakspace İskenderiye ve Antakya’dan Yunanistan boyunca Siraküza ve Roma’ya kadar Helen topluluklarının felsefi düşüncesi. Yunanlılar’ın felsefesi; mevcut tüm diğer dini sisteme kıyasla Pavlus’un Hıristiyan uyarlamasıyla daha uyumlu olup, Hıristiyanlık’ın Batı’daki başarısında önemli bir etken olmuştu. Pavlus’un din\hyp{}kuramı ile birlikte Yunan felsefesi, hala Avrupa’nın etik değerlerinin temelini oluşturmaktadır.
\vs p098 7:11 İsa’nın özgün öğretileri Batı’ya girdiğinde, onlar Batılaştı; ve onlar Batılaşınca, ırkların tümüne ve insanların her türüne olan, içinde barındırdığı evrensel hitabını kaybetmeye başladı. Hristiyanlık, bugün, beyaz ırkların toplumsal, ekonomik ve siyasi adetlerine oldukça iyi uyum sağlamış bir din haline gelmiştir. O; içten bir biçimde öğretilerini takip etmeyi arzulayan bireylere İsa hakkında güzel bir dini hala cesur bir biçimde resmediyor olsa da, uzunca bir süredir İsa’nın dini oluşuna son vermiştir. Hristiyanlık İsa’yı, Tanrı tarafından Mesihsel seçilmiş biri, Kurtarıcı olarak yüceltmişti; ancak o Hâkim’in şu kişisel müjdesini unutmuştu: Tanrı’nın Yaratıcılığı ve her insanın evrensel kardeşliği.
\vs p098 7:12 Ve bu anlatım, Urantia üzerinde Maçiventa Melçizedeği öğretilerinin uzun hikâyesiydi. Nebadon’un bu acil durum Evladı’nın Urantia’ya bahşedilişinden beri, yaklaşık olarak dört bin yıl geçmiştir; ve bu süreç içerisinde “En Yüksek Tanrı, El Elyon’un dinadamının” öğretileri ırkların ve insan topluluklarının tümüne nüfuz etmiştir. Ve Maçiventa, olağandışı bahşedilişinin amacına ulaşmada başarılı olmuştu; Mikâil Urantia’da ortaya çıkmaya hazır hale geldiğinde, Tanrı kavramsallaşması, uzayın dönen gezegenlerinde ilgi çekici nitelikteki geçici hayatlarını yaşarlarken Kâinatın Yaratıcısı’nın bu çok çeşitli çocuklarının canlı ruhsal deneyimlerinde hala yeniden alevlenen aynı Tanrı kavramsallaşması olarak, erkek ve kadınların kalplerinde mevcuttu.
\vs p098 7:13 [Nebadon’un bir Melçizedek unsuru tarafından sunulmuştur.]
