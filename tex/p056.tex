\upaper{56}{Evrensel Bütünlük}
\vs p056 0:1 Tanri, bütünlüğün kendisidir. İlahiyat, evrensel bir biçimde eş güdümsel hale getirilmiştir. Kâinat âlemlerinin tümü; tek bir sınırsız akıl tarafından mutlak bir biçimde düzenlenen, birliktelik haline getirilmiş çok geniş bir işleyiş biçimidir. Evrensel yaratımın fiziksel, ussal ve ruhsal nüfuz alanları kutsal bir biçimde birbirleriyle ilişkilendirilmiştir. Kusursuz ve kusursuz olmayan nitelikler gerçek anlamıyla birbirleriyle iniltilidir; ve bu nedenle sınırlı evrimsel yaratılmış, Kâinatın Yaratıcısı’nın “Kusursuz olun, hatta benim olduğum kadar kusursuz olun” biçimindeki emrine itaat eden bir biçimde Cennet’e yükselebilir.
\vs p056 0:2 Yaratımın çeşitli düzeylerinin tümü, Üstün Evren’e ait Mimarların tasarımlarında ve idaresinde bütünlenmiş bir nitelikte bulunmaktadır. Zaman\hyp{}mekân fanilerinin kısıtlı akılları, görünebilen uyumsuzluğu dışa vuran ve etkin eş güdümün var olmayışına işaret eden birçok sorun ve durumu açığa çıkarabilir; ancak evren olgular bütününün daha geniş ufuklarını gözlemlemeye yetkin olan, ve yaratıcı çeşitliliği simgeleyen temel bütünlüğü saptamaya ek olarak çeşitliliğin bu biçimde işleyişinin tümünü içine alan kutsal biricik bütünlüğün keşfedilme sanatında daha deneyimli olan bizim gibi unsurlar, evrensel yaratıcı enerjinin bu çok katmanlı dışavurumlarının tümü içinde sergilenen kutsal ve temel amacı daha iyi algılamaktadır.
\usection{1.\bibnobreakspace Fiziksel Eş Güdüm}
\vs p056 1:1 Fiziksel veya maddi yaratım sınırsız değildir; ancak bu yaratım, kusursuz bir biçimde eş güdümsel hale getirilmektedir. Orada kuvvet, enerji ve güç mevcut bulunmaktadır; ancak onların tümü köken bakımından birdir. Yedi aşkın\hyp{}evren görünüşte ikircikli bir yapıya sahiptir; merkezi evren üç katmanlı bir niteliğe sahiptir; ancak Cennet tek bir oluşumdan meydana gelmiştir. Ve Cennet; geçmiş, şimdiki zaman ve gelecek biçiminde, maddi evrenlerin tümünün mevcut kaynağıdır. Ancak bu kâinatsal kapsamdaki türevsel yaratım bir \bibemph{ebediyet} etkinliğidir; geçmiş, şimdiki zaman veya gelecek biçiminde hiçbir \bibemph{zaman zarfında} ne mekân veya ne maddi kâinat, Işık Adası çekirdeğinden mevcudiyet kazanmamaktadır. Kâinatsal kaynak olarak Cennet, mekân ve zaman öncesi faaliyet göstermektedir; bu nedenle, onun türevsel yaratımları, mekân içinde nihai muhafızları ve zaman içinde açığa çıkarıcılarına ek olarak onların düzenleyicileri olan Koşulsuz Mutlak vasıtasıyla ortaya çıkmamış olsaydı, onlar zaman ve mekân tarafından belirlenen bir görüneme sahip olacaktı.
\vs p056 1:2 İlahi Mutlak maddi gerçekliğin tümünün seçkin bir üst denetimini gerçekleştirirken, Koşulsuz Mutlak fiziksel evreni korumaktadır; ve bu iki Mutlak işlevsel bir biçimde Evrensel Mutlak tarafından bütünleşmiştir. Maddi evrenin bu birbirini tamamlayan ilişkisi; maddi, morontia, absonit veya ruhsal bir biçimdeki kişiliklerin tümü tarafından, alt Cennet’i merkezine alan çekime karşı hilesiz maddi gerçekliğin tümünün gösterdiği karşılığın gözlenmesi vasıtasıyla, en iyi bir biçimde anlaşılabilir.
\vs p056 1:3 Çekim bütünlüğü, evrensel ve değişmez nitelikte bulunmaktadır; saf\hyp{}enerji karşılığı benzer bir biçimde evrensel ve kaçınılmazdır. Saf enerji (temel kuvvet) ve saf ruhaniyet, çekim karşısında bütünüyle önceden tepki gösteren bir nitelikte bulunmaktadır. Mutlaklıklar içinde içkin olan bu başat kuvvetler, kişisel olarak Kâinatın Yaratıcısı tarafından düzenlenmektedir; böylelikle saf enerji ve saf ruhaniyete ait Kâinatın Yaratıcısı’nın kişisel mevcudiyetine ek olarak onun aşkın maddi yerleşkesi içinde çekim merkezinin tümü, bu biçimde düzenlenmektedir
\vs p056 1:4 Saf ruhaniyet, temel enerji sistemlerinin hepsinin kutsal ve yönlendirici yüksek denetiminin potansiyeli iken; saf enerji, birbiriyle ilgili, ruhaniyet olmayan işlevsel gerçekliklerin atasıdır. Ve, mekân boyunca dışa vurulmasına ek olarak zamanın devinimleri içinde gözlendiği biçimiyle oldukça çeşitli bir nitelikte bulunan bu gerçekliklerin ikisi de, Cennet Yaratıcısı’nın kişiliği içinde odaklanır. Onun içinde bu gerçeklikler --- bütünleşmek durumunda olan bir biçimde --- tekdir, çünkü Tanrı bir tekdir. Yaratıcı’nın kişiliği mutlak bir biçimde bütünleşmiştir.
\vs p056 1:5 Yaratıcı olan Tanrı’nın sınırsız doğası içinde olası bir biçimde, fiziksel ve ruhsal biçimde gerçekliğin ikircikli bir yapısı kesinlikle mevcut bulunamaz; ancak Cennet Yaratıcısı’nın kişisel değerlerine ait sınırsız düzeyleri ve mutlak gerçeklikleri dışarıda tuttuğumuz an bizler, bu iki gerçekliğin mevcudiyetini deneyimler ve onların bütünüyle Cennet Yaratıcısı’nın kişisel mevcudiyetine karşılık gösterdiğini ayırt ederiz; onun içinde her şey bir bütün haline gelmektedir.
\vs p056 1:6 Cennet Yaratıcısı’na ait sınırsız kişiliğin koşulsuz kavramsallaşmasından uzaklaştığınız an siz, AKLI; BEN biçimindeki İlk Kaynak ve Merkez olarak, kökensel nitelikteki tekil Yaratan kişiliğinin bu çifte evren dışavurumlarının sürekli bir biçimde genişleyen farklılaşmasının bütünleşmesine ait kaçınılmaz işleyiş biçimi şeklinde düşünmek durumundasınız.
\usection{2.\bibnobreakspace Ussal Bütünlük}
\vs p056 2:1 Düşünce\hyp{}Yaratıcısı; ruhaniyet dışavurumunu Söz\hyp{}Evladı içerisinde gerçekleştirmekte olup, gerçeklik genişlemesine uçsuz bucaksız maddi evrenler içinde Cennet boyunca ulaşır. Ebedi Evlat’ın ruhsal dışavurumu, Sınırsız Ruhaniyet’in faaliyetleri vasıtasıyla yaratımının maddi düzeyleri ile ilişkilidir; aklın ruhaniyet\hyp{}karşılık hizmeti vasıtasıyla ve aklın fiziksel\hyp{}yönlendirici faaliyetleri içinde, İlahiyat’ın ruhsal gerçeklikleri ve İlahiyat’ın maddi dışavurumları birbirileriyle ilişkili hale getirilmiştir.
\vs p056 2:2 Akıl, Sınırsız Ruhaniyet’in işlevsel kazanımıdır; bundan dolayı bu kazanım, potansiyel bakımından sınırsız ve bahşedilme bakımından evrenseldir. Kâinatın Yaratıcısı’nın temel düşüncesi, Cennet Adası ve onun İlahiyat eşitine ek olarak ruhsal ve Ebedi Evlat biçiminde çifte dışavurum içerisinde ebedileştirir. Bu türden ebediyet gerçekliğinin ikircikliği; kaçınılmaz nitelikteki, Sınırsız Ruhaniyet biçiminde Tanrı’yı akıl içinde mümkün kılar. Akıl, ruhsal ve maddi gerçeklikler arasındaki iletişimin hayati kanalıdır. Maddi evrimsel yaratılmış, yalnızca aklın hizmeti vasıtasıyla ikamet eden ruhaniyeti algılayabilir ve onu kavrayabilir.
\vs p056 2:3 Bu sınırsız ve evrensel akıl, kâinatsal akıl olarak zaman ve mekân evrenleri içinde hizmet vermektedir; ve emir\hyp{}yardımcı ruhaniyetlerinin ilkel hizmetinden bir evrenin baş yöneticisinin muhteşem aklına uzanan bir kapsamda bulunmasına karşın, bu kâinatsal akıl bile, zaman ve mekânın Yüce Aklı ile sonuçsal olarak eş güdümsel hale getirilmiş ve Sınırsız Ruhaniyet’in her şeyi kapsayan aklıyla kusursuz bir biçimde ilişkilendirilmiş Yedi Üstün Ruhaniyet’in yüksek denetimi içinde gerektiği bir şekilde bütünleştirilmiştir.
\usection{3.\bibnobreakspace Ruhsal Bütünleşme}
\vs p056 3:1 Evrensel akıl çekimi, Sınırsız Ruhaniyet’in Cennet kişisel mevcudiyeti içinde merkezi bir konuma getirilmişken; benzer bir biçimde evren ruhaniyet çekimi, Ebedi Evlat’ın Cennet kişisel mevcudiyeti içinde odaklandırılmıştır. Kâinatın Yaratıcısı tekdir; ancak zaman\hyp{}mekân bakımından o, saf enerji ve saf ruhaniyetin ikircikli olgular bütünlüğü içerisinde açığa çıkarılmıştır.
\vs p056 3:2 Cennet ruhaniyet gerçeklikleri benzer bir biçimde tekdir; ancak zaman\hyp{}mekân durumları ve ilişkilerinin tümü içinde bu tek ruhaniyet, Ebedi Evlat’ın ruhaniyet kişilikleri ve onun doğumlarına ek olarak Sınırsız ruhaniyet ve onun birliktelik içerisindeki yaratımlarının ruhaniyet kişilikleri ve etkinliklerine ait ikircikli olgular bütünü içinde açığa çıkarılmıştır; ve orada bunlara ek olarak, --- saf\hyp{}ruhaniyet nüvesel bölünmeleri olarak --- birey\hyp{}öncesi nitelikte bulanan Yaratıcı’nın bahşettiği Düşünce Düzenleyicileri ve diğer ruhaniyet birimleri şeklinde, hali hazırda üçüncü bir olgu bulunmaktadır.
\vs p056 3:3 Ruhsal olgular bütünüyle karşılaştığınız veya ruhaniyet varlıklarıyla ilişkiye geçtiğiniz evren etkinliklerinin seviyesinden bağımsız olarak sizler; bu olgu ve unsurların hepsinin ruhaniyet olan Tanrı’dan, Ruhaniyet Evladı’nın ve Sınırsız Akıl Ruhaniyeti’nin hizmeti vasıtasıyla kökensel olarak ortaya çıktığının bilgisine sahip olabilirsiniz. Ve bu uçsuz bucaksız ruhaniyet, yerel evrenlerin yönetim merkezlerinden yönlendirilen bir biçimde zamanın evrimsel dünyaları içinde bir olgu olarak faaliyet göstermektedir. Yaratan Evlatlar’ın bu başkentlerinden, maddi akılların daha alt düzeyde bulunan ve evirilen seviyeleri için emir\hyp{}yardımcı akıl\hyp{}ruhaniyetlerinin hizmeti ile birlikte Kutsal Ruhaniyet ve Gerçekliğin Ruhaniyeti gelmektedir.
\vs p056 3:4 Akıl, Yüce Varlık ile ilişki içerisinde ve kâinatsal akıl olarak Mutlak Akıl’a tabi olan bir biçimde daha bütünleşmiş bir halde iken; evrimsel dünyalara olan ruhaniyet hizmeti, yerel evrenlerin yönetim merkezleri içinde ikamet eden kişilikler içinde ve zaman\hyp{}mekân ruhaniyet dışavurumlarının nihai bütünleşmesinin gerçekleştiği Ebedi Evlat’ın Cennet çekim döngüsü ile neredeyse kusursuz bir biçimde sonuçsal olarak ilişkilendirmiş baş yönetici konumunda bulunan Kutsal Hizmetkârlar'ın bireyleri içinde daha doğrudan bir şekilde bütünleştirilmiştir.
\vs p056 3:5 Kusursuzlaştırılmış yaratılmış mevcudiyeti; Cennet Kutsal Üçlemesi’nin bireylerinden biri tarafından gerçekleştirilen Kutsal Üçleme\hyp{}öncesi ruhaniyet bahşedilmesinin bir nüvesinin bilinçli akıl ile birleşimi vasıtasıyla erişilir, idare edilir ve ebedi hale getirilir. Fani akıl, Ebedi Evlat ve Sınırsız Ruhaniyet’e ait Erkek ve Kız Evlatlar’ın yaratımıdır; ve kökeni Yaratıcı’dan gelen Düşünce Düzenleyicisi ile bu akıl bütünleştiğinde, evrimsel âlemlerin üç katmanlı ruhaniyet kazanımının bir parçası haline gelir. Ancak bu üç ruhaniyet dışavurumu; her ne kadar Ebedi Evlat ve Sınırsız Ruhaniyet’in Kâinatın Yaratıcısı haline Evrensel BEN’in gelmesinden önce bile ebediyeten onun mevcudiyeti içinde oldukça bütüncül bir nitelikte bulunduğu gibi, kesinlik unsurları içinde kusursuz bir biçimde bütünleşmiş hale gelir.
\vs p056 3:6 Ruhaniyet her zaman ve nihai olarak, dışavurumu bakımından üç katmanlı ve kesin gerçekleşmesi bakımından Kutsal Üçleme ile bütünleşmiş bir hale gelmek zorundadır. Ruhaniyet, üç katmanlı bir dışavurum boyunca tek bir kaynaktan ortaya çıkmaktadır. Ve kesinlik bakımından Ruhaniyet; --- kutsallık ile bir bütün haline gelme durumu olarak --- ebediyet içinde Tanrı’yı bulmanın deneyimlendiği kutsal bütünleşme içinde, ve Yaratıcı’nın evrensel düşüncesine ait ebedi sözün sınırsız dışavurumunun sahip olduğu kâinat aklın hizmeti aracılığı içinde, bütüncül gerçekleşmesine erişmek zorunda olup bunu hali hazırda gerçekleştirir.
\usection{4.\bibnobreakspace Kişilik Bütünleşmesi}
\vs p056 4:1 Kâinatın Yaratıcısı, kutsal bir biçimde bütünleşmiş bir kişiliktir; böylelikle, Yaratıcı’nın emrine itaat eden bir biçimde maddi faniler içinde ikamet etmek için Cennet’ten gönderilmiş Düşünce Düzenleyicileri’nin geldikleri istikamete geri dönme devinimi tarafından Cennet’e taşınan Kâinatın Yaratıcısı’nın yükseliş evlatlarının tümü, Havona’ya ulaşmalarından önce buna benzer bir biçimde tamamiyle bütünleştirilmiş kişilikler haline geleceklerdir.
\vs p056 4:2 Kişilik içkin olarak, bileşen gerçekliklerinin tümünü bir araya getirmeyi amaçlar. Kâinatın Yaratıcısı olarak İlk Kaynak ve Merkez’in sınırsız kişiliği, Sınırsızlığın yedi bileşen Mutlaklık’ın tümünü bir araya getirir; Kâinatın Yaratıcısı’nın ayrıcalıklı ve doğrudan bir bahşedilmesi olarak fani aklın kişiliği benzer bir biçimde, fani yaratılmışın bileşen etkenlerini bir araya getirmenin potansiyelini elinde bulundurur. Yaratılmış kişiliğin tümünün bu türden bütünleştirme yaratıcılığı, onun yüksek ve ayrıcalıklı kökenine ait bir doğum izidir; ve bu yaratıcılık, yaratılmışın kişiliğinin Cennet üzerinde bulunan kişiliğin tümünün Yaratıcısı ile doğrudan ve devam eden ilişkisini sağlayan araçlar tarafından, kişilik döngüsü vasıtasıyla bu aynı köken ile olan koparılmaz nitelikteki iletişime dair ilave kanıttır.
\vs p056 4:3 Her ne kadar, Yedi Katmanlılık’ın nüfuz alanlarından başlayarak yücelik ve nihayet boyunca Mutlak olan Tanrı’ya kadar uzanan bir kapsamda Tanrı mevcut bir durumda bulunsa da; Cennet üzerinde ve Yaratıcı olan Tanrı’nın kişiliği içinde merkezi bir şekilde konumlanan biçimde kişilik döngüsü, ussal mevcudiyetin tüm düzeylerine ek olarak kusursuz, kusursuzlaştırılmış ve kusursuzlaşmakta olan evrenlerin âlemlerinin tümü içinde, yaratılmış kişiliklerinin tümünü içine alan bir kapsamda onlarla ilgili kutsal kişiliğin bu çeşitli dışavurumlarının tümüne ait tamamlanmış ve kusursuz bütünleşmeyi sağlamaktadır.
\vs p056 4:4 Tanrı, evrenler için ve onlar içerisinde bizim tasvirlerimizin tümü şeklinde mevcut bulunurken; sizler ve Tanrı’yı tanıyan tüm diğer yaratılmışlar için kendisi, sizin ve onların Yaratıcısı olarak tekdir. Kişilik bakımından Tanrı çoğul olamaz. Tanrı, yaratılmışlarının her biri için Yaratıcı’dır; ve herhangi bir çocuğun birden fazla babaya sahip olması gerçek anlamıyla imkânsızdır.
\vs p056 4:5 Felsefi, kâinatsal ve dışavurumların çeşitli seviyeleri ve yerleşkeleri bakımından sizler; çoğul İlahiyatlar’ın faaliyetlerini algılayabilir ve hatta gerçekte çoğul Kutsal Üçleme unsurlarının mevcudiyetini düşünmelisiniz; ancak merkezi evren boyunca her ibadet eden kişiliğe ait kişisel iletişimin ibadetsel deneyimi içinde Tanrı tekdir; buna ek olarak bütünleşmiş ve kişisel İlahiyat bizlerin Cennet ebeveyni, Yaratıcı olan Tanrı’sı, bahşedicisi, koruyucusu, ve yerleşik dünyalar üzerinde bulunan fani insandan başlayarak merkezi Işık Adası üzerindeki Ebedi Evlat’a uzanan bir kapsamda kişiliklerin tümünün Yaratıcısı’dır.
\usection{5.\bibnobreakspace İlahiyat Bütünlüğü}
\vs p056 5:1 Cennet İlahiyatı’nın bölünmezliği olarak biricikliği, varoluşsal ve mutlaktır. Kâinatın Yaratıcısı, Ebedi Evlat ve Sınırsız Ruhaniyet biçiminde, İlahiyat’ın üç ebedi kişileşmesi mevcut bulunmaktadır; ancak Cennet Kutsal Üçlemesi içinde onlar, \bibemph{gerçekte} bölünmez ve bir bütün olarak tek bir İlahiyat’dır.
\vs p056 5:2 Varoluşsal gerçekliğin kökensel Cennet\hyp{}Havona düzeyinden iki alt bağımlı mutlak seviyesi farklılaşmış olup, orada Yaratıcı, Evlat ve Ruhaniyet sayısız kişisel birliktelikler ve alt bağımlı unsurların yaratımına katılmıştır. Ve her ne kadar bu konu ile ilgili olarak, nihayetin aşkın seviyeleri üzerinde absonit ilahiyat bütünleşmesini ele almaya girişmek uygun olmasa da; yaratımın çeşitli birimleri ve ussal varlıkların farklı düzeyleri için kutsallığın işlevsel bir biçimde içinde açığa vurulduğu çeşitli İlahiyat kişileşmelerinin bütünleyici faaliyetine ait bir takım niteliklere göz atmak mümkündür.
\vs p056 5:3 Aşkın\hyp{}evrenler içinde kutsallığın mevcut faaliyeti, --- yerel evren Yaratan Evlatlar ve Ruhaniyetleri, aşkın\hyp{}evren Zamanın Ataları, ve Cennet’in Yedi Üstün Ruhaniyeti olarak --- Yüce Yaratanlar’ın işleyişsel faaliyetleri içinde etkin bir biçimde dışa vurulmuştur. Bu varlıklar, Kâinatın Yaratıcısı’na doğru iç doğrultuda hareket ederek Yedi Katmanlı Tanrı’nın ilk üç düzeyini meydana getirirler; ve Yedi Katmanlı Tanrı’nın bu bütüncül nüfuz alanı, evrim halindeki Yüce Varlık’ın deneyimsel ilahiyatının ilk seviyesinde eş güdüm faaliyeti içerisinde bulunmaktadır.
\vs p056 5:4 Cennet üzerinde ve merkezi evren içinde İlahiyat bütünlüğü, mevcudiyetin bir gerçeğidir. Zaman ve mekânın evrim halindeki evrenleri boyunca İlahiyat bütünlüğü bir kazanımdır.
\usection{6.\bibnobreakspace Evrimsel İlahiyat’ın Bütünleşmesi}
\vs p056 6:1 İlahiyat’ın üç ebedi bireyi; bölünmez İlahiyat olarak Cennet kutsal üçlemesi içerisinde faaliyet gösterdiği zaman, onlar kusursuz birlikteliği elde ederler; benzer bir biçimde onlar birliktelik halinde veya ikişerli topluluklar halinde yarattıkları zaman, kendilerine ait Cennet köken soyları kutsallığın niteliksel bütünlüğünü yansıtmaktadır. Buna ek olarak, zaman\hyp{}mekân nüfuz alanlarına ait Yüce Yaratanlar ve İdareciler tarafından dışa vurulan bu kutsallığın amacı; evrenin kişisel olmayan enerji birlikteliğinin mevcudiyeti içerisinde deneyimsel İlahiyat’ın deneyimsel kişilik gerçeklikleri ile ancak yerinde bir biçimde bütünleşmeyle çözülecek olan bir gerçeklik gerilimini meydana getiren, deneyimsel yüceliğin egemenliğine ait bütünleştiren güç potansiyelini mevcut kılmaktadır.
\vs p056 6:2 Yüce Varlık’ın kişilik gerçeklikleri; Cennet İlahiyatları’ndan kökenini almakta olup, dışsal Havona döngüsünün öncül dünyası üzerinde asli evrenin Yaratan kutsallıklarından gelen Her Şeye Gücü Yeten Yücelik’in güç ayrıcalıkları ile bütünleşir. Yüce olan Tanrı bir birey olarak Havona içinde yedi aşkın\hyp{}evrenin yaratılmasından önce mevcut bir konumda bulunmaktaydı; ancak kendisi bu süreç içerisinde yalnızca ruhsal düzeylerde faaliyet göstermiştir. Evrim halindeki evrenler içinde çeşitli kutsallık birleşimi vasıtasıyla Yücelik’in Her Şeye Gücü Yeten kudretinin evrimi; Sınırsız Ruhaniyet’in sınırsız aklı içinde barınan potansiyeli Yüce Varlık’ın etkin işlevsel aklına eş zamanlı bir biçimde çeviren Yüce Akıl’ın araçları vasıtasıyla Havona içinde Yücelik’in ruhsal bireyi ile eş güdümsel halde görev yapan İlahiyat’ın yeni bir güç varlığını mevcut kılar.
\vs p056 6:3 Yedi aşkın\hyp{}evrene ait evrimsel dünyaların maddi akıl kazandırılmış yaratılmışları İlahiyat’ı yalnızca, Yüce Varlık’ın bu güç\hyp{}kişilik birleşimi içinde evrimleşmesi faaliyetinde kavrayabilir. Mevcudiyetinin hiçbir seviyesi içerisinde Tanrı, bu türden bir düzey içerisinde yaşayan varlıkların kavramsal yetisini aşamaz. Fani insan gerçeğin tanınması, güzelliğin takdiri, iyiliğin ibadeti vasıtasıyla Tanrı’nın derin bir sevgi olduğuna dair kavrayışını geliştirir; ve bunun sonrasında o, ilahiyat seviyeleri içinde yükselerek Yücelik’in kavrayışına doğru ilerler. İlahiyat, güç bakımından bütünleşmiş bir biçimde bu şekilde kavranıldığında, bunun sonrasında yaratılmış anlayışı ve erişimi için ruhaniyet bakımından kişileştirilebilir.
\vs p056 6:4 Yüceliş varlıkları, aşkın\hyp{}evrenlerin başkentleri üzerinde Her Şeye Gücü Yeten’in kudretine ve Havona’nın dış döngüleri üzerinde Yücelik’in kişiliğine dair kavrayışa erişirken; gerçekte onlar, Cennet İlahiyatları’nı kesin bir biçimde bulacakları gibi Yüce Varlık’ı bulamayacaklardır. Altıncı\hyp{}düzey ruhaniyetleri olarak kesinlik unsurları bile, Yüce Varlık’ı bulamamışlardır; buna ek olarak onların, yedinci\hyp{}düzey ruhaniyet konumuna erişene ve Yücelik gerçek bir biçimde gelecek dış evrenlerin etkinlikleri içinde işlevsel bir konuma gelene kadar onu bulamayacakları muhtemeldir.
\vs p056 6:5 Ancak yükseliş unsurları; Kâinatın Yaratıcısı’nı Yedi Katmanlı Tanrı’nın yedinci düzeyi olarak bulduklarında, evren yaratılmışları ile kişisel ilişkilerin \bibemph{tüm} ilahiyat düzeylerine ait İlk Birey’in kişiliğine erişmiş olur.
\usection{7.\bibnobreakspace Evrensel Nitelikteki Evrimsel Sonuçlar}
\vs p056 7:1 Zaman\hyp{}mekân evrenleri içinde evrimin istikrarlı ilerleyişine, tüm us yaratılmışları için İlahiyat’ın sürekli bir biçimde genişleyen açığa çıkarışları eşlik eder. Bir sistem, takımyıldızı, evren, aşkın\hyp{}evren veya asli evren içerisinde evrimsel ilerleyişin doruk noktasına olan erişim, yaratımın bu ilerleyen birimleri için ve onlar içerisinde ilahiyat faaliyetinin ilgili genişlemelerini simgeler. Ve kutsallığın kendisini gerçekleşmesine ait bu türden her yerel gelişime, yaratımın tüm diğer birimleri için genişlemiş ilahiyat dışavurumlarının oldukça kesin bir biçimde tanımlı olan belirli sonuçları eşlik eder. Cennet’ten dışarı doğru uzanan bir doğrultuda gerçekleştirilmiş ve erişilmiş evrimin her yeni nüfuz alanı, kâinat âlemlerinin tümü için deneyimsel İlahiyat’ın yeni ve genişlemiş bir açığa çıkarılışını oluşturmaktadır.
\vs p056 7:2 Yerel bir evrenin birleşen parçaları, ışık ve yaşam içinde ilerleyen bir biçimde istikrara kavuşturulurken; Yedi Katmanlı Tanrı, artan bir şekilde dışa vurulan bir hale getirilir. Zaman\hyp{}mekân evrimi bir gezegen üzerinde denetim içinde, --- Yaratan Evlat\hyp{}Yaratıcı Ruhaniyet birlikteliği olarak --- Yedi Katmanlı Tanrı’nın ilk dışa vurulumu ile başlar. Işık içinde bir sistemin istikrara kavuşturulması ile birlikte bu Evlat\hyp{}Ruhaniyet birlikteliği, faaliyetin bütünlüğüne erişir; ve bütün bir takımyıldızı böylece istikrara kavuşturulduğunda, Yedi Katmanlı Tanrı’nın ikinci fazı bu türden bir âlem boyunca daha etkin hale gelmektedir. Bir yerel evrenin tamamlanmış idari evrimine, aşkın\hyp{}evren Üstün Ruhaniyetleri’nin yeni ve daha doğrudan hizmetleri eşlik eder; ve bu aşamada orada aynı zamanda, altıncı Havona döngüsünün dünyaları boyunca geçiş halinde iken yükseliş unsurunun Yüce Varlık’ı kavrayışıyla sonuçlanan Yüce olan Tanrı’nın sürekli olarak genişleyen açığa çıkarılışı ve kendini gerçekleştirişi başlamaktadır.
\vs p056 7:3 Kâinatın Yaratıcısı, Ebedi Evlat ve Sınırsız Ruhaniyet ussal yaratılmışlar için deneyimsel ilahiyat dışavurumlarıdır; bu nedenle onlar, yaratımın tümünün akli ve ruhaniyet yaratılmışları ile olan kişilik ilişkilerinde benzer bir biçimde genişlememektedirler.
\vs p056 7:4 Yükseliş fanilerinin; kişisel varlıklar olarak bu ilahiyatları deneyimsel olarak ayırt etme ve onlar ile iletişime geçme düzeyine erişmek amacıyla yeterli bir biçimde ruhsal ve gerekli bir şekilde eğitilmiş hale gelmelerinden çok daha önce, İlahiyat’ın ilerleyen seviyelerine ait kişisel olmayan mevcudiyeti deneyimleyebilmelerinin olasılık dâhilinde bulunduğu belirtilmelidir.
\vs p056 7:5 Kutsallık dışavurumları tarafından mekânın her yeni yerleşimine ek olarak yaratımın bir birimi içindeki her yeni evrimsel erişime; yaratımın tümünün bu aşamadaki mevcut olan ve daha öncesinden gelen düzenlenmiş birimleri içerisinde İlahiyat’ın işlevsel açığa çıkarılışlarının eş zamanlı gerçekleşen genişlemeleri eşlik eder. Evrenlerin ve onların birleşen birimlerinin idari görevine ait bu yeni faaliyet girişimi, burada özetlenen işleyiş biçimi uyarınca her zaman tam olarak uygulanmayan bir görünüme sahip olabilir; çünkü bu girişim, yeni idari yüksek denetimin takip eden ve ilerleyen dönemleri için koşulları hazırlamak amacıyla idarecilerin gelişmiş topluluklarını gönderme uygulamasıdır. Nihai olan Tanrı bile, ışık ve yaşam altında istikrara kavuşturulmuş bir yerel evrenin daha sonraki aşamaları boyunca kendisine ait evrenlerin aşkın yüksek denetiminin işaretini önceden vermektedir.
\vs p056 7:6 Zaman ve mekânın yaratımları ilerleyen bir biçimde evrimsel düzeyde istikrara kavuşturulurken, orada; Yedi Katmanlı Tanrı’nın ilk üç dışavurumunun karşılıksal çekilişi ile eş zamanlı gerçekleşen, Yüce olan Tanrı’nın yeni ve daha bütünsel faaliyetinin gözlenmekte olduğu bir gerçektir. Eğer Yüce olan Tanrı’nın zaman ve mekânın bu yaratılmışlarının doğrudan denetimini üstlendiği biliniyorsa, asli evren ışık ve yaşam içinde istikrara kavuşturulmuş hale geldiğinde, kaldı ki eğer bu durum gerçekleşirse, bunun sonrasında Yedi Katmanlı Tanrı’nın Yaratan\hyp{}Yaratıcı dışavurumlarının gelecekteki faaliyeti ne olacaktır? Zaman\hyp{}mekân evrenlerinin bu düzenleyicileri ve önderleri, dışsal uzay içinde benzer etkinlikler için serbest mi bırakılacaklardır? Bizler bu soruların cevaplarını kesin olarak bilmemekteyiz; ancak biz, bu ve bu konu ile ilgili hususlar hakkında fikir yürütmekteyiz.
\vs p056 7:7 Deneyimsel İlahiyat’ın sınırları Koşulsuz Mutlak’ın nüfuz alanlarına doğru dışarı doğrultuda genişlerken, Yedi Katmanlı Tanrı’nın geleceğin bu yaratılmışlarının daha önceki evrimsel çağları boyunca faaliyet göstereceğini öngörmekteyiz. Biz, Zamanın Ataları ve aşkın\hyp{}evren Üstün Ruhaniyetleri’nin gelecek düzeyinin ne olacağının öngörüsüne dair ortak bir fikre sahip değiliz. Buna ek olarak biz, Yüce Varlık’ın burada yedi aşkın\hyp{}evrenlerde olduğu gibi faaliyet gösterip göstermeyeceğine dair bir bilgiye sahip değiliz. Ancak hepimiz; Yaratan Evlatlar olarak Mikâiller’in bu dış evrenler içinde faaliyet gösterme nihai sonuna sahip olduklarını tahmin etmekteyiz. İçimizden bazıları, yardımcı Yaratan Evlatlar ve Kutsal Hizmetkârlar arasında daha yakın türden bir birlikteliğe gelecek çağların şahit olacağının görüşünü öne sürmektedir; bu türden yaratan birliğinin nihai bir doğaya ait yardımcı\hyp{}yaratan kimliğinin yeni birtakım dışavurumu ile sonuçlanabileceği bile mümkündür. Ancak biz, açığa çıkarılmamış geleceğin bu olasılıkları ile ilgili kesin hiçbir bilgiye gerçek anlamıyla sahip değiliz.
\vs p056 7:8 Buna rağmen bizler; zaman ve mekânın evrenleri içinde Yedi Katmanlı Tanrı’nın Kâinatın Yaratıcısı’na doğru ilerleyici bir yaklaşım sağladığının, ve bu evrimsel yaklaşımın deneyimsel bir biçimde Yüce olan Tanrı içinde bütünleştiğinin bilgisine sahibiz. Bizler, bu türden bir tasarımın dışsal evrenler içinde de varlığını devam ettirmek zorunda olduğunu öngörebiliriz; öte yandan, bu evrenler içinde herhangi bir zaman zarfında ikamet edebilecek varlıkların yeni düzeyleri, nihai seviyeler üzerinde ve absonit işleyiş biçimleri tarafından İlahiyat’a yaklaşmaya yetkin olabilirler. Kısacası bizler, dışsal uzayın gelecek evrenleri içinde işlerlik kazanabilecek ilahiyat yaklaşımının işleyiş biçimine dair en ufak bir kavrama sahip değiliz.
\vs p056 7:9 Yine de bizler, kusursuzlaştırılmış aşkın\hyp{}evrenlerin; bahse konu dışsal uzay yaratımları içinde ikamet edebilecek olan bu varlıkların Cennet\hyp{}yükseliş süreçlerinin bir şekilde birer parçaları olacaklarını öngörmekteyiz. Gelecek çağ içerisinde dışsal uzay unsurlarının; Yedi Üstün Ruhaniyet’in işbirliği ile veya onların iş birliği olmadan Yüce olan Tanrı tarafından idare edilen bir biçimde, yedi aşkın\hyp{}evren vasıtasıyla Havona’ya yaklaşmalarını gelecek bir çağ içerisinde gözlemlememiz oldukça olasıdır.
\usection{8.\bibnobreakspace Yüce Bütünleştirici}
\vs p056 8:1 Yüce varlık, fani insan deneyimi içinde üç katmanlı bir faaliyete sahiptir. Bunlardan ilki, Yedi Katmanlı Tanrı olarak zaman\hyp{}mekân kutsallığının bütünleştiriciliğidir. İkincisi, sınırlı yaratılmışların gerçek anlamda kavrayabileceği en üst düzey İlahiyat olmasıdır. Üçüncüsü ise onun; absonit akıl, ebedi ruhaniyet ve Cennet kişiliğinin eşliğinde mevcut olan aşkın deneyime insanın ulaşabileceği tek doğrultu niteliğinde bulunmasıdır.
\vs p056 8:2 Yerel evrenler içinde doğmuş, aşkın\hyp{}evrenlerde yetişmiş ve merkezi evrende eğitilmiş olan yükseliş unsurları, kişisel deneyimleri içerisinde; Yücelik içinde bir bütün haline gelen Yedi Katmanlı Tanrı’nın zaman\hyp{}mekân kutsallığının kavranışına dair potansiyelin tümü ile bütünleşir. Kesinlik unsurları, kökensel âlemlerinin aksine, aşkın\hyp{}evrenler içerisinde sıralı bir biçimde faaliyet gösterir; böylelikle onlar, olası yaratılmış deneyimin yedi katmanlı çeşitliliğine dair bütünlük sağlanana kadar deneyimlerinin üstüne deneyim katmış olurlar. İkamet eden Düzenleyiciler’in hizmeti vasıtasıyla kesinlik unsurlarının Kâinatın Yaratıcısı’nı \bibemph{bulmaları} mümkün kılınmıştır; ancak deneyimin bahse konu işleyiş biçimleri vasıtasıyla bu türden kesinlik unsurları Yüce Varlık’ı gerçek anlamıyla \bibemph{bilmeye} erişir, ve dışsal uzayın gelecek evrenleri içinde ve bu evrenler için Yüce İlahiyat’ın hizmetleri ve \bibemph{açığa çıkarılışlarını} gerçekleştirmenin nihai sonuna sahip olurlar.
\vs p056 8:3 Yaratıcı olan Tanrı ve onun Cennet Evlatları’nın bizler için yaptıkları karşısında hepimizin bu yapılanların aynısını, ortaya çıkmakta olan Yüce Varlık’ta ve onun için ruhaniyet içinde gerçekleştirmenin olanağına sahip olduğumuzu unutmayınız. Evren içinde derin sevgi, sevinç, ve hizmetin deneyimi karşılıklıdır. Yaratıcı olan Tanrı, kendisinin bahşettiği şeyleri evlatlarının tekrar kendilerine geri getirmesine ihtiyacı yoktur; ancak onlar, bahşedilenler karşılığında tüm kazanımlarını akranlarına ve evrim halindeki Yüce Varlık’a aktarmaktadırlar (veya bunu gerçekleştirebilirler).
\vs p056 8:4 Yaratımsal olgular bütününün tümü, atasal yaratan\hyp{}ruhaniyet etkinliklerinin yansımasıdır. İsa şu sözleri dile getirmiştir ve bu ifadeler tamamıyla doğruluk taşımaktadır: “Evlat yalnızca, gördüğü Yaratıcı’nın faaliyetlerinin aynısını gerçekleştirmektedir.” Zaman içinde siz faniler, Yücelik’in akranlarınıza olan açığa çıkarılışlarına başlayabilirsiniz; ve siz, Cennet doğrultusunda yükselirken bu açığa çıkarılışın kapsamını artan bir biçimde arttırabilirsiniz. Ebediyet içerisinde sizlerin, yüce düzeylerde --- hatta nihai düzeyde bile --- yedinci\hyp{}düzey kesinlik unsurları olarak evrimsel yaratılmışların bu Tanrı’sının artan bir biçimde açığa çıkarılışını gerçekleştirmenize izin verilebilir.
\usection{9.\bibnobreakspace Evrensel Nitelikteki Mutlak Bütünlük}
\vs p056 9:1 Koşulsuz Mutlak ve İlahi Mutlak, Evrensel Mutlak içerisinde bütünleşmiştir. Bu Mutlaklıklar; Nihayet içinde eş güdümsel hale getirilir, Yücelik içinde belirlenir, ve Yedi Katmanlı Tanrı içerisinde zaman\hyp{}mekân bakımından değişikliğe uğratılır. Alt\hyp{}sınırsızlık seviyeleri üzerinde \bibemph{üç} mutlaklık mevcut bulunmaktır; ancak sınırsızlık içerisinde onlar \bibemph{bir} \bibemph{tek} olarak gözlemlenir. Cennet üzerinde İlahiyat’ın üç kişilikleşmesi mevcut bulunmaktadır, ancak Kutsal Üçleme içerisinde onlar \bibemph{tekdir}.
\vs p056 9:2 Üstün evrenin temel felsefi önermesi şudur: Mutlak (sınırsızlık içinde bir bütün olarak üç Mutlak) Kutsal Üçleme’den önce mevcut bulunmakta mıydı? Ve Mutlak, Kutsal Üçleme’nin atası mıdır? Veya Kutsal Üçleme mi Mutlak’ın atasıdır?
\vs p056 9:3 Koşulsuz Mutlak, Kutsal Üçleme’den bağımsız olan bir kuvvet mevcudiyeti midir? İlahi Mutlak’ın kişiliği, Kutsal Üçleme’nin sınırsız faaliyetine mi işaret etmektedir? Ve Evrensel Mutlak, Kutsal Üçlemeler’in bir Kutsal Üçlemesi olarak, Kutsal Üçleme’nin nihai faaliyeti midir?
\vs p056 9:4 İlk bakışta, --- Kutsal Üçleme’nin bile atası biçiminde --- her şeyin kökeni olarak Mutlak’ın bir kavramsal düşüncesi, tutarlığın sağlanışını ve felsefi bütünlüğün geçici bir tatminini sağlıyor görünebilir; ancak bu türden her çıkarım, Cennet Kutsal Üçlemesi’nin ebediyeti tarafından geçersiz kılınmaktadır. Bizler; doğaları ve mevcudiyetleri bakımından Kâinatın Yaratıcısı ve onun Kutsal Üçleme birlikteliklerinin ebedi olduklarının bilgisini almış olup ve biz bu gerçekliğe inanmaktayız. Bu gerçekliğin sonrasında orada yalnızca tek bir tutarlı felsefi çıkarım bulunmaktadır ve o ise; Mutlak’ın tüm evren usları için, evrenler arası veya evrenler dışı temel ve başat mekân durumlarının tümü kapsamında Kutsal Üçleme’nin (Kutsal Üçlemeler’in) kişilik\hyp{}dışı ve eş güdümsel tepkisi olduğudur. Asli evrenin tüm kişilik unsurları için Cennet Kutsal Üçlemesi sonsuza kadar; kişilik kavrayışı ve yaratımın kendisini gerçekleştirmesine ait bütün işlevsel amaçlar için, kesinlik, ebediyet, yücelik ve ebediyet içinde mutlak nitelikte yerini almaktadır.
\vs p056 9:5 Yaratılmış akılları bu sorunu irdelerken, Evrensel BEN’in nihai düşüncesini Kutsal Üçleme ve Mutlak’ın başat nedeni ve koşulsuz kaynağı olarak değerlendirmeye yönelmektedirler. Bu nedenle biz Mutlak’ın kişisel bir kavrayışını sunmayı arzuladığımızda, Cennet Yaratıcısı’na dair düşüncelerimize ve nihai çıkarımlarımıza tekrar başvurmaktayız. Bu aksi durumda kişilik\hyp{}dışı konumda bulunan Mutlak’ın kavrayışını önermeyi veya ona dair bilinci arttırmayı arzuladığımızda, Kâinatın Yaratıcısı’nın mutlak kişiliğin deneyimsel Yaratıcısı olduğu gerçeğine geri dönmekteyiz; Ebedi Evlat Mutlak Birey’dir, ancak bu Evlat Mutlak’ın kişilikleşmesi olarak deneyimsel anlamda bu nitelikte bulunmamaktadır. Ve bunun sonrasında bizler; --- Kutsal Üçlemeler’in Kutsal Üçlemesi olarak --- yüceliğin, nihayetin ve sınırsızlığın bütünleşmiş ve eş güdümsel hale getirilmiş İlahiyat birlikteliklerine ait kişilik\hyp{}dışı etkinliklerinin dışa vurulan mevcudiyetine dair evren ve evren dışı olgular bütününü meydana getiren bir biçimde Evrensel Mutlak’ı görürken, İlahi Mutlak’ın deneyimsel kişilikleşmesi içinde sonuçlanan bir kavram niteliğinde deneyimsel Kutsal Üçlemeler’i tasavvur etmeye devam ederiz.
\vs p056 9:6 Yaratıcı olan Tanrı, sınırlı olandan sınırsız seviyeye kadar tüm düzeyler üzerinde kavranabilir; ve her ne kadar Cennet’ten başlayarak evrimsel dünyalara kadar onun yaratılmışları kendisini daha önceden kavramış olsalar da, yalnızca Ebedi Evlat ve Sınırsız Ruhaniyet kendisini sınırsız olarak bilmektedir.
\vs p056 9:7 Ruhsal kişilik yalnızca Cennet üzerinde mutlaktır; ve Mutlak’ın kavramı sadece sınırsızlık içinde koşulsuzdur. İlahiyat mevcudiyeti, yalnızca Cennet üzerinde mutlaktır. Buna ek olarak Tanrı’nın açığa çıkarılışı; onun gücü Koşulsuz Mutlak’ın mekân güç etkisi içinde deneyimsel bir biçimde sınırsız hale gelene, onun kişilik dışavurumu deneyimsel olarak İlahi Mutlak’ın dışa vurulmuş mevcudiyeti içinde sınırsız bir konumda olana, ve sınırsızlığın bu iki potansiyeli Evrensel Mutlak içinde gerçek\hyp{}oluşumuyla\hyp{}bütünleşene kadar, Tanrı’nın açığa çıkarılışı her zaman kısmi, göreceli ve ilerleyen bir nitelikte bulunmak zorundadır.
\vs p056 9:8 Ancak alt\hyp{}sınırsızlık düzeylerin ötesinde üç Mutlaklık; \bibemph{bir bütün} olup, böylelikle, sınırsızlığın bilincini mevcudiyetin herhangi bir başka düzeyinin bireysel olarak kendiliğinden gerçekleştirmesinden bağımsız bir biçimde sınırsız nitelikte İlahiyat’ın\hyp{}kendini\hyp{}gerçekleştirmesidir.
\vs p056 9:9 Her ne kadar diğer ebediyet, --- bir ebedi sınırsızlık olarak --- bir sınırsız ebediyet içinde içkin olan deneyimsel potansiyellerin bireysel gerçekleştirilişinin deneyimlenmesini gerektirebilse de; ebediyet içindeki deneyimsel düzey, sınırsızlığın deneyimsel birey bilinci anlamına gelmektedir.
\vs p056 9:10 Ve Yaratıcı olan Tanrı, kâinat âlemlerinin tümü boyunca ussal yaratılmışlar ve ruhaniyet varlıklarının hepsi için İlahiyat ve gerçekliğin bütün dışavurumlarına ait kişisel kaynaktır. Kişilikler olarak, mevcut an içerisinde veya ebedi geleceğin ilerleyen evren deneyimleri içinde; Yedi Katmanlı Tanrı’ya erişimi gerçekleştirmiş olmanızdan, Yüce olan Tanrı’yı kavramış olmanızdan, Nihai olan Tanrı’yı bulmuş olmanızdan veya Mutlak olan Tanrı’nın kavramını algılamaya girişmiş olmanızdan bağımsız olarak, --- evren kişiliklerinin tümünün Cennet Yaratıcısı niteliğinde --- ebedi Tanrı’yı yeniden keşfetmiş bir biçimde, yeni deneyimsel düzeyler üzerinde sahip olduğunuz her serüvenin tamamlanmasında deneyimleyeceğiniz ebedi memnuniyeti keşfedeceksiniz.
\vs p056 9:11 Kâinatın Yaratıcısı; --- koşulsuz Gerçeklik niteliğinde --- mutlak değerler ve anlamların nihayet\hyp{}sonrası bütünlüğü içinde yüce ve hatta nihai bir biçimde deneyimlenmesi gereken, evren bütünlüğünün açıklamasıdır.
\vs p056 9:12 Üstün Kuvvet Düzenleyicileri; enerjilerinin Kâinatın Yaratıcısı’nın Cennet çekimine karşılık veren bir hale gelmesi için, mekân üzerinde hareket eder ve onları harekete geçirir; ve bunun sonrasında, olası tüm kutsallık nitelikleri içerisinde Yaratıcı gibi olmak için Cennet Yaratıcısı’nın ruhaniyetini kendileri üzerine alan ve bunu takiben Yaratıcı’ya yükselen ussal yaratılmışların içinde evrimleştiği yerleşik evrenlere doğru çekime\hyp{}karşılık veren bu kuvvetleri düzenleyen Yaratan Evlat’lar buraya gelmektedir.
\vs p056 9:13 Mekân boyunca Cennet yaratıcı kuvvetlerinin sürekli bir biçimde devam eden ve genişleyen ilerleyişi; Kâinatın Yaratıcısı’na ait çekim kapsamının sürekli bir biçimde genişleyen nüfuz alanına ek olarak, Tanrı’yı derin bir biçimde sevebilen, onun tarafından derin sevgiyle sevilebilen, ve böylelikle Tanrı’yı bilen hale gelerek Cennet’e erişmek ve Tanrı’yı bulmayı tercih edebilmek olarak onun gibi olmayı seçebilen ussal yaratılmışların çeşitli türlerine ait sonu olmayan çoğalımlarına işaret eden bir görüntüye sahiptir.
\vs p056 9:14 Kâinat âlemlerinin tümü hep birlikte bütünleşmiştir. Tanrı, güç ve kişilik bakımından tekdir. Orada, enerjinin tüm seviyeleri ve kişiliğin tüm fazlarının eş güdümü bulunmaktadır. Kavramsal ve gerçeklik bakımından felsefi ve deneyimsel olarak şeylerin ve varlıkların tümü, Cennet Yaratıcısı’nda merkezi olarak konumlanmaktadır. Tanrı her şey ve her şeyim içindedir; hiçbir şey veya varlık onun dışında var olamaz.
\usection{10.\bibnobreakspace Gerçeklik, Güzellik, İyilik}
\vs p056 10:1 Işık ve yaşam içinde istikrara kavuşturulmuş dünyalar birinci aşamadan yedinci çağa doğru ilerlerken, onlar; Yaratan Evlat’ın hayranlığından başlayarak onun Cennet Yaratıcısı’na olan ibadete uzanan kapsam dâhilinde, Yedi Katmanlı Tanrı’nın gerçekliğinin farkındalığını birbirini takip eden bir biçimde kavramaktadırlar. Bu türden bir dünyanın devam eden yedinci aşaması boyunca sürekli bir biçimde gelişen faniler, Yüce olan Tanrı’nın bilgisi içinde yetişirler; bunun karşısında onlar, Nihai olan Tanrı’nın gelecekteki hizmetine dair gerçekliği bulanık bir kesinlikte algılarlar.
\vs p056 10:2 Bu muhteşem çağ boyunca sürekli bir biçimde gelişen fanilerin başlıca gayesi, --- gerçeklik, güzellik ve iyilik olarak --- İlahiyat’ın kavranabilen etkenlerinin daha iyi ve daha bütüncül farkındalığına ulaşma arayışıdır. Bu arayış, insanın Tanrı’yı akıl, madde ve ruhaniyet içinde kavrama çabasını yansıtmaktadır. Ve fani bu arayışını sürdürürken, kendisini artan bir biçimde felsefe, kâinat bilimi ve kutsallığın deneyimsel çalışmasıyla içli dışlı hale gelmiş bir biçimde bulur.
\vs p056 10:3 Siz felsefeyi bir şekilde algılamaktasınız; kutsallığı ibadet, toplumsal hizmet ve kişisel nitelikteki ruhsal deneyim içinde kavramaktasınız; ancak kâinat bilimi olarak güzelliğin arayışını hepiniz, çok sık bir biçimde insanın olgunlaşmamış sanatsal çabalarının çalışmalarıyla kısıtlamaktasınız. Sanat olarak güzellik büyük ölçüde zıtlıkların bütünleşmesi durumudur. Çeşitlilik, güzelliğin kavramı için hayati derecede önemlidir. Güzel sanatların doruk noktası olarak yüce güzellik, Yaratan ve yaratılmışın kâinatsal uç unsurlarına ait enginliğin bütünleşmesine dair olaylar dizisidir. Yaratılmışın Yaratan gibi kusursuz hale gelmesi olarak, insan’ın Tanrı’yı bulması ve Tanrı’nın insanı bulması; kâinatsal sanatın doruk noktasına olan erişim niteliğinde yüce bir şekilde güzel olanın ulvi kazanımıdır.
\vs p056 10:4 Bu nedenle tanrı tanımamazlık olarak maddiyat, güzelliğin sınırlı nitelikteki en yüksek tezadı biçiminde çirkinliğin doruk noktasına çıkarılmasıdır. En yüksek güzellik, mevcudiyet\hyp{}öncesi uyumlu gerçeklikten doğan çeşitliliğin bütünlüğüne ait toplu görünümden meydana gelmektedir.
\vs p056 10:5 Düşünüşe ait kâinatsal düzeylerin erişimi şu nitelikleri kapsamaktadır:
\vs p056 10:6 1.\bibnobreakspace \bibemph{Merak}. Uyum için açlık ve güzelliğe duyulan susuzluk. Uyumlu kâinat ilişkilerin yeni düzeylerini keşfetmek için kararlı girişimler.
\vs p056 10:7 2.\bibnobreakspace \bibemph{Estetiksel takdir}. Güzelliğe karşı duyulan derin sevgiye ek olarak, gerçekliğin tüm düzeyleri üzerinde yaratıcı dışavurumların hepsinin sanatsal niteliğine dair sürekli gelişen beğeni.
\vs p056 10:8 3.\bibnobreakspace \bibemph{Etiksel hassasiyet}. Gerçekliğin farkındalığı boyunca güzelliğin takdiri, tüm varlıklar ile olan İlahiyat ilişkileri içinde kutsal iyiliğin tanınması üzerinde etkide bulunan koşullara dair ebedi durumun hissiyatına götürmektedir; ve böylece kâinat bilimi bile --- Tanrı bilinci olarak --- kutsal gerçeklik değerlerinin arayışıyla sonuçlanmaktadır.
\vs p056 10:9 Işık ve yaşam içinde istikrara kavuşturulan dünyalar gerçeklik, güzellik ve iyilik ile oldukça bütüncül bir biçimde ilgilidir; çünkü bu nitelik değerleri, zaman ve mekânın âlemleri ile İlahiyat’ın açığa çıkarılışını bütünleştirmektedir. Ebedi gerçekliğin anlamları, fani insanın ussal ve ruhsal doğaları için birleşik bir etkide bulunmaktadır. Evrensel güzellik, kâinatsal yaratımın uyumlu ilişkileri ve ahenkleri ile bütünleşmektedir; bu durum daha ayırt edilen bir biçimde ussal çekim niteliğinde olup, ve maddi evrenin bütünleşmiş ve uyumlu hale gelmiş kavrayışına götürmektedir. Kutsal iyilik, içinde algılanacak ve insan kavrayışının ruhsal düzeyinin bahse konu eşiğine yükseltilecek sınırlı aklın sınırsız değerlerinin açığa çıkarılışını yansıtmaktadır.
\vs p056 10:10 Gerçeklik, dinin ussal oluşumunu sunan bir biçimde bilim ve felsefenin temelidir. Güzellik, insan deneyimin tümüne ait sanata, müziğe ve anlamlı ahenklere zemin hazırlamakta ve onların tümünü desteklemektedir. İyilik, --- deneyimsel kusursuzluk\hyp{}açlığı niteliğinde --- etik değerlerin, ahlakın ve dinin algılanışıyla bütünleşir.
\vs p056 10:11 Güzelliğin mevcudiyeti, ilerleyici evrimin Yüce Aklın hâkimiyetine işaret etmesi gibi aynı kesinlikte takdir edici yaratıcı aklın varlığını gösterir. Güzellik, mevcudiyet\hyp{}öncesi ve ebedi bir bütünlükten kaynaklanan her şeyin olgular bütünlüğünün gerçekliğine ait uçsuz bucaksız çeşitliğinin uyumlu zaman\hyp{}mekân birleşimine dair ussal tanımadır.
\vs p056 10:12 İyilik, kutsal kusursuzluğun çeşitli düzeylerine ait göreceli değerlerin akli tanınmasıdır. İyiliğin tanınması, iyilik ve kötülüğü ayırt etme yetkinliğine sahip bir kişisel akıl biçiminde, ahlaki düzeyin bir aklı anlamına gelmektedir. Ancak büyüklük biçiminde iyiliğe sahip olma, gerçek kutsallık erişiminin bir ölçüsüdür.
\vs p056 10:13 \bibemph{Gerçek ilişkilerin} tanınması, gerçek ve yanlışı ayırt etmeye yetkin bir akıl anlamına gelmektedir. Urantia’nın insan akıllarını meydana getiren Gerçekliğin bahşedilmiş Ruhaniyeti, --- Tanrı doğrultusunda ebedi yükseliş içinde eş güdüm haline getirilirken her şeyin ve her varlığın yaşayan ruhaniyet ilişkileri olarak --- hataya yer bırakmayan bir biçimde gerçekliğe karşılık vermektedir.
\vs p056 10:14 Her elektrona, her düşünceye veya her ruhaniyete ait her bir dürtü, evrenin bütünü içinde faaliyet halinde olan bir birimdir. Yalnızca günah tecrit altına alınmış olup, kötülük akli ve ruhsal düzeyler üzerinde çekime karşı gelmektedir. Evren bir bütündür; hiçbir şey veya varlık tecrit içinde var olamaz veya yaşayamaz. Bireyin kendisini gerçekleştirmesi eğer toplum karşıtı bir nitelikte bulunuyorsa potansiyel olarak kötülüğü beraberinde getirir. Kâinatsal toplumsallaşma, bireysel bütünleşmenin en yüksek türüdür. İsa şunu ifade etmiştir: “Sizlerin en büyük olanının herkesin hizmetkârı olmasına izin verin.”
\vs p056 10:15 İnsanın aklın, maddenin ve ruhaniyetin evrenine olan ussal yaklaşımı biçiminde --- gerçeklik, güzellik, iyilik bile kutsal ve yüce olan bir\bibemph{ nihai amacın} bütünleşmiş tek bir kavramı haline gelmek zorundadır. Fani kişilik madde, akıl ve ruhaniyet ile insan deneyimini bütünleştirirken; aynı şekilde bu kutsal ve yüce nihai amaç, Yücelik içinde güç\hyp{}ile\hyp{}bütünleşen bunun sonrasında ise baba sevgisinin bir Tanrı’sı olarak kişilikleşen hale gelmektedir.
\vs p056 10:16 Herhangi bir bütünlük için onu bir araya getiren bileşenlerin ilişkisine dair kavrayışın tümü, bütün ile bu bileşenlerin bütüncül ilişkisine dair bir anlayış derinliğini gerektirmektedir; ve evren içinde bu durum, Yaratıcı Bütünlük ile yaratılmış bileşenlerin ilişkisi anlamına gelmektedir. İlahiyat böylelikle, evrensel ve ebedi erişimin amacı niteliğinde hatta sınırsız bile olarak aşkın hale gelmektedir.
\vs p056 10:17 Evrensel güzellik, maddi yaratım içerisinde Cennet Adası’nın yansımasının tanınmasıdır; bunun karşısında ebedi gerçeklik ise, kendilerini fani ırklara bahşetmekle kalmayan aynı zamanda sahip oldukları Gerçekliğin Ruhaniyeti’ni insan toplulukların tümüne bile aktaran Cennet Evlatları’nın özel hizmetidir. Kutsal iyilik, Sınırsız Ruhaniyet’in çok katmanlı kişiliklerine ait sevgi dolu hizmet içinde daha bütüncül bir biçimde gösterilmiştir. Ancak bu üç niteliğin toplamı olarak derin sevgi, kendisine ait ruhaniyet Yaratıcısı olarak insanın Tanrı’nın algısıdır.
\vs p056 10:18 Fiziksel madde, mutlak İlahiyatlar’ın Cennet enerji\hyp{}ışıltısına ait zaman\hyp{}mekân gölgesidir. Gerçek anlamlar, --- yüce kavramların zaman\hyp{}mekân kavrayışı olarak --- İlahiyat’ın ebedi sözünün fani\hyp{}us sonuçlarıdır. Kutsallığın iyilik değerleri, evrimsel âlemlerin sınırlı zaman\hyp{}mekân yaratılmışları için Evrensel, Ebedi ve Sınırsızlık’ın ruhaniyet kişiliklerine ait bağışlayıcı hizmetlerdir.
\vs p056 10:19 Kutsallığın bu anlamlı kişilik değerleri; kutsal derin sevgi olarak, her kişisel yaratılmış ile Yaratıcı’nın ilişkisi içinde bir araya gelmektedir. Onlar, Evlat ve onun Evlatları içinde kutsal bağışlama olarak eş güdümsel hale gelmiştir. Onlar niteliklerini, zamanın evlatları için sevgi dolu bağışlamanın temsili biçiminde kutsal hizmet olarak Ruhaniyet ve kendi ruhaniyet evlatları vasıtasıyla dışa vururlar. Bu üç kutsallık başlıca olarak güç\hyp{}kişilik birleşimi niteliğinde Yüce Varlık tarafından dışa vurulur. Onlar, yedi yükseliş seviyesi üzerinde kutsal anlamlar ve değerlerin yedi farklı birlikteliği içinde Yedi Katmanlı Tanrı tarafından çeşitli biçimlerde gösterilmiştir.
\vs p056 10:20 Sınırlı insan için gerçeklik, güzellik ve iyilik kutsallık gerçekliğinin bütüncül açığa çıkarılışı ile bütünleşmektedir. İlahiyat’ın bu derin sevgi kavrayışı ruhsal dışavurumunu Tanrı’yı tanıyan fanilerin yaşamında bulur. Orada; ussal barış, toplumsal ilerleyiş, ahlaki tatmin, ruhsal neşe ve kâinatsal bilgelik biçiminde kutsallığın meyveleri yetişmektedir. Işık ve yaşamın yedinci aşamasında bulunan bir dünya üzerindeki gelişmiş faniler, evren içinde derin sevginin en büyük şey olduğunu öğrendiler --- ve onlar Tanrı’nın derin sevgi olduğunu bilmektedirler.
\vs p056 10:21 Derin sevgi başkalarına iyilik yapma arzusudur.
\vs p056 10:22 [Nebadon Açığa Çıkarım Birliği’nin talebi üzerine, ve Urantia’nın vekil Gezegensel Prens’i olan belirli bir Melçizedek ile iş birliği içerisinde, Urantia üzerinde ziyarette bulunun bir Kudretli İletici tarafından sunulmuştur.]
\separatorline
\vs p056 10:23 Evrensel Bütünlük üzerine bu makale; on iki unsurdan oluşan ve Mantutia Melçizedek’in yönlendirmesi altında hareket eden Nebadon kişiliklerinin bir heyeti tarafından bir topluluk olarak desteklenen biçimde, çeşitli yazarlar tarafından oluşturulmuş bir sunumlar dizisinin yirmi beşinci anlatımıdır. Biz; Urantia zamanının 1934 yılında üst unsurlarımız tarafından resmi olarak onaylanan bir işleyiş biçimi vasıtasıyla, bu anlatımları kâğıda döküp İngilizce dilinde yazıya geçirdik.
