\upaper{27}{Birinci Derece Birincil Hizmetkâr Ruhaniyetleri’nin Hizmeti}
\vs p027 0:1 Birinci Derece Birincil Hizmetkâr Ruhaniyetleri, Cennet’in ebedi Ada’sı üzerinde İlahiyatlar’ın tanrısal hizmetlileridir. Onların, doğruluğun ve ışığın doğrultusundan bir zaman bile ayrıldıkları bilinmemektedir. Kabul edilen unsurların yoklama çağrısı tamamlanmıştır; bahse konu bu muhteşem ev sahiplerin bir tanesi bile ebediyetin bu listesinden çıkmamıştır. Bahse konu bu yüksek birincil hizmetkâr ruhaniyetleri, kusursuzluk bakımından yüce olan bir şekilde kusursuz varlıklardır; fakat onlar ne absonit ne de mutlaktırlar. Kusursuzluğun özünün varlıkları olarak bahse konu Sınırsız Ruhaniyet’in bu evlatları, kendilerine ait olan çok katmanlı görevlerinin tüm fazlarında irade dâhilinde ve dönüşümlü olarak görevde bulunurlar. Her ne kadar onlar, merkezi evrenin çeşitli bin yıllık toplanmalarına ve topluluksal birleşmelerine katılsalar da; onlar kapsamlı bir biçimde Cennet dışında faaliyet göstermezler. Onlar aynı zamanda İlahiyatlar’ın özel ileticileri olarak hareket edip, büyük topluluklar biçiminde Teknik Danışmanlar haline gelebilmek için yükselişte bulunurlar.
\vs p027 0:2 Birinci derece birincil hizmetkâr ruhaniyetleri aynı zamanda, isyan sebebiyle tecrit edilmiş dünyalar üzerinde yüksek meleksel ev sahiplerinin emri altında konumlandırılır. Bir Cennet Evladı; bu türden bir dünya üzerine bahşedilir, burada kendi görevini tamamlar, Kâinatın Yaratıcısı’na doğru yükselişte bulunur, bunun sonrasında ise onun tarafından kabul edilerek bahse konu tecrit edilmiş dünyanın resmi olarak tanınmış kurtarıcısı olarak geri döner. Bu durum üzerine birincil derece birincil hizmetkâr ruhaniyetlerinin bir tanesi her zaman; yeniden düzeni tekrar sağlanan âlem içinde görevde bulunmak üzere, hizmetkâr ruhaniyetlerinin görevini yerine getirmek amacıyla görevin baş idarecileri tarafından atanır. Birincil hizmetkâr ruhaniyetleri bu özel hizmet içinde, dönemsel olarak dönüşümlü bir biçimde değiştirilmektedir. Hazreti Mikâil’in bahşedildiği zamanlardan beri, Urantia üzerinde mevcut olan “yüksek meleklerin baş idarecisi”, bu görevi yerine getirmek amacıyla bu düzeyinin ikinci unsurudur.
\vs p027 0:3 Ebediyetten beri birinci derece birincil hizmetkâr ruhaniyetleri Işığın Adası üzerinde hizmet vermekte olup; mekânın dünyaları için liderliğin görevlerine katılmıştır. Fakat zamanın Havona kutsal yolcularının Cennet üzerinde varışlarından beri onlar, mevcut haliyle tanımlandıkları biçimde hizmette bulunmaktadır. Bahse konu bu yüksek melekler mevcut an itibariyle başlıca olarak hizmetin şu yedi düzeyinde hizmette bulunmaktadır:
\vs p027 0:4 1.\bibnobreakspace İbadetin Yönlendiricileri.
\vs p027 0:5 2.\bibnobreakspace Felsefenin Üstatları.
\vs p027 0:6 3.\bibnobreakspace Bilginin Sorumluları.
\vs p027 0:7 4.\bibnobreakspace Davranışın Yöneticileri.
\vs p027 0:8 5.\bibnobreakspace Etiğin Yorumcuları.
\vs p027 0:9 6.\bibnobreakspace Görevin Baş İdarecileri.
\vs p027 0:10 7.\bibnobreakspace İstirahatın Başlatıcıları.
\vs p027 0:11 Yükseliş halinde bulunan kutsal yolcuların Cennet yerleşimine mevcut olarak erişmelerinden önce onlar, bahse konu birincil hizmetkâr ruhaniyetlerinin doğrudan etkisinin altına girmemektedirler; bu sürecin sonrasında onlar, topluluklarının sıralandırılmasının tersinden başlayarak bahse konu meleklerin yönlendirmesi altında bir hazırlanma deneyiminden geçerler. Bu süreç; Cennet sürecinize istirahat başlatıcılarınızın gözetimi altında giriş yapmanız, ve bunu takip eden aşamalardaki düzeylerin katıldığı dönemlerin sonrasında ise ibadetin yönlendiricilerin eğitim süreci ile tamamlamanızdan oluşmaktadır. Bunun sonrasında ise siz, kesinliğe erişecek olan bir unsurun sonsuz sürecine başlayamaya hazır hale gelirsiniz.
\usection{1.\bibnobreakspace İstirahatın Başlatıcıları}
\vs p027 1:1 İstirahatın başlatıcıları, merkezi Ada’dan Havona’nın en içte bulunan döngüsüne hareket eden Cennet müfettişleridir; burada onlar, birincil hizmetkâr ruhaniyetlerinin ikinci düzeyinin istirahat tamamlayıcıları biçimindeki görev arkadaşları ile işbirliğinde bulunur. Cennet’in hoşnutiyeti için hayati öneme sahip olan bir nitelik, kutsal dinlenme biçimindeki istirahattır; buna ek olarak istirahatın bahse konu bu başlatıcıları, ebediyetin tanıtımı için zamanın kutsal yolcularını hazır hale getiren nihai eğitmenlerdir. Onlar, görevlerine nihai erişim döngüsü üzerinde başlayıp; mekânın bir yaratılmışını ebediyetin âlemine mezun eden hafif uyku biçimindeki son geçiş uykusundan zamanın yolcusu uyandığı an görevlerine devam eder.
\vs p027 1:2 İstirahat yedi katmanlı bir doğaya aittir. Alt seviyede bulunan yaşam düzeyleri içinde uykunun ve eğlencenin istirahatı, yüksek varlıkların düzeyinde keşfin istirahatı ve ruhaniyet kişiliğinin en yüksek türü içinde ise ibadetin istirahatı bulunmaktadır. Aynı zamanda orada, ruhsal veya fiziksel enerji ile birlikte varlıkların yeniden enerji dolumu biçimindeki enerji alımının olağan istirahatı bulunmaktadır. Buna ek olarak bir âlemden diğerine olan geçiş zamanı olarak yüksek melekler uykusuna yatırılma anındaki bilinçsiz bir hafif uyku biçimindeki geçiş uykusu bulunmaktadır. Bunların tümünden bütünüyle farklı bir biçimde; bir aşamadan diğerine, bir yaşamdan diğer yaşama ve bir mevcudiyet düzeyinden bir diğerine olan biçimdeki geçiş istirahatı şeklinde başkalaşımın derin uykusu bulunmaktadır. Bu uyku, herhangi bir düzeyin çeşitli \bibemph{aşamaları} boyunca gerçekleşen evrime tezat olan bir biçimde mevcut evren \bibemph{düzeyinden} olan geçişe katılmaktadır.
\vs p027 1:3 Fakat bahse konu son başkalaşım uykusu, yükseliş sürecinin takip eden düzey erişimlerine taşıyan önceki geçiş hafif uykularından biraz daha farklıdır; bunun sonucunda zaman ve mekânın yaratılmışları, Cennet’in zamansız ve mekânın yerleşim yerleri içinde yerleşik düzeye erişmek için zamansal ve mekânsal olan ölçümlerin en içte bulunan sınırlarında kat ederler. Tıpkı yüksek meleklerin ve onların birliktelik içinde bulunduğu varlıkların, ölümden fani yaratılmışın varlığını devam ettirmesi için hayati derecede temel öneme sahip olması gibi; istirahatın başlatıcıları ve tamamlayıcıları, bahse konu bu aşkınlaşan başkalaşım için aynı öneme sahiptir.
\vs p027 1:4 Nihai olan Havona döngüsü üzerindeki istirahatınıza giriş yaptığınız zaman siz, ebedi olarak Cennet üzerinde yeniden dirilirsiniz. Buna ek olarak siz orada ruhsal olarak yeniden kişilikleştirildiğiniz zaman; Havona’nın en içte bulunan döngüsü üzerinde nihai uykuyu yaratan bahse konu bu birinci derece birincil hizmetkâr ruhaniyetleri olarak ebedi kıyılarda sizi karşılayan istirahatın başlatıcılarını eş zamanlı olarak tanıyacaksınız; ve siz, Kâinatın Yaratıcısı’nın ellerine koruduğunuz kimliğinizi bir kez daha emanet etmek için hazır hale geldiğinizde, inancın bu son muhteşem güçlenişini hatırlayacaksınız.
\vs p027 1:5 Zamanın en son istirahatı hoşnut bir biçimde yaşandığında; en son geçiş uykusu deneyimlendiğinde; ebedi yerleşkenin kıyılarında sonsuza kadar sürecek olan yaşam için uyanırsınız. “Bununla birlikte burada artık hiçbir uyku bulunmamaktadır. Tanrı’nın ve onun Evladı’nın mevcudiyeti sizden önce gelmekte olup, siz ebedi olarak onun hizmetinde bulunmaktasınız. Onun yüzünü görmüş olacaksınız, ve onun ismi ruhaniyetinizdir. Orada gece olmamakta, ve güneşin ışığına ihtiyaç duyulmamaktadır; çünkü Muhteşem Kaynak ve Merkez onlara ışığı sağlamaktadır. Onlar orada sonsuza kadar yaşayacaklardır. Bununla birlikte Tanrı gözlerden tüm yaşları silecektir; artık ne ölüm bulunacak, ne bir keder ne bir hüzün, ne artık bir acı olacaktır, çünkü orada her şey artık geçmiş olacaktır.”
\usection{2.\bibnobreakspace Görevin Baş İdarecileri}
\vs p027 2:1 Bu topluluk zaman zaman, “özgün doğum\hyp{}biçimi meleği” olan birincil hizmetkâr ruhaniyetlerinin baş idarecisi tarafından; birinci, ikinci ve üçüncü derecede bulunan bu meleklerin üç düzeyinin tümünün işleyişsel düzeni üzerine hâkimiyet sağlaması için tasarlanmıştır. Bir beden olarak birincil hizmetkâr ruhaniyetleri; bahse konu ruhaniyet kişiliklerinin tümü üzerinde hâkimiyete en başından beri sahip olan Cennet’in ilk meleği biçimindeki onların ortak baş idarecisinin faaliyetleri dışında, bütünüyle özerk olup öz denetime sahiptir.
\vs p027 2:2 Görevin melekleri, Cennet’in yüceltilmiş fani sakinlerinin Kesinliğe Erişecek Olanların Birlikleri’ne kabul edilmelerinden önce onların mevcudiyetiyle fazlasıyla ilgilidir. Eğitim ve öğretim, Cennet’e ulaşanların ayrıcalıklı görevleri değildir; hizmet aynı zamanda kesinliğe erişimden önceki Cennet’in eğitimsel deneyimlerinde temel bir rol oynar. Buna ek olarak yükseliş fanileri serbest zamanların belirli dönemlerine sahip olduğunda, görevin birincil hizmetkâr ruhaniyetleri baş idarecilerinin yedek birlikleriyle birlikte onların kardeşçe bir bütünleşmeyi yeğlemeyi tercih etmelerini gözlemlemiş bulunmaktayım.
\vs p027 2:3 Siz yükselişte bulunanlar faniler Cennet’e eriştiğinizde, engin ve kutsal olan varlıkların bir ev sahipliğine ek olarak yüceltilmiş yoldaş fanilerin benzer bir topluluğuyla irtibattan daha büyük olan bir ilişki içerisine gireceksiniz. Siz aynı zamanda; Cennet Vatandaşları’nın sayıca üç bini aşan farklı düzeyleriyle, Aşkınlar’ın çeşitli topluluklarıyla, Urantia üzerinde henüz açığa çıkarılmamış geçici ve kalıcı biçimdeki Cennet sakinlerinin sayısız diğer türleriyle kardeşsel bir biçimde bütünleşmelisiniz. Cennet’in bahse konu bu kudretli uslarıyla irtibatı sağladıktan sonra, aklın meleksel türleriyle ziyarette bulunmak aynı zamanda oldukça rahatlatıcı olacaktır; onlar, zamanın fanilerinin yüksek meleklerinin kiminle bu türden uzun bir irtibata ve bir canlandırıcı birlikteliğe sahip olduğunu hatırlatır.
\usection{3.\bibnobreakspace Etiğin Yorumlayıcıları}
\vs p027 3:1 Hayatın ölçeğinde daha yukarı yükseldiğinizde, kâinat etiğine daha fazla ilgi göstermeniz gerekmektedir. Etiksel farkındalık, tüm bireylerin veya herhangi birinin mevcudiyetinde içkin olan herhangi bir bireyin haklarının temel bir biçimde tanınmasıdır. Fakat ruhsal etik, kişisel ve topluluk ilişkilerinin fani ve hatta morontia kavramsallaşmasını oldukça aşmaktadır.
\vs p027 3:2 Etik; Cennet’in ihtişamlarına zamanın kutsal yolcularının uzun yükselişinde gerektiği gibi öğretilip, onlar tarafından yetkin bir biçimde öğrenilir. Bahse konu bu içsel olan yükseliş süreci, mekânın özgünlük dünyalarından itibaren gerçekleşmeye başlayınca; yükseliş halinde bulunan unsurlar, kâinat birlikteliklerinin en başından beri genişleyen döngüsü için bir topluluktan diğerine eklenmesine devam eder. Görev arkadaşlarının her yeni topluluğu, tanınmak ve uyumlu hale gelmek için fanilerin Cennet erişimine yükseldiği ana kadar etiğin artan düzeyleriyle buluşmaya devam eder; onlar, etiksel olan yorumlamalar ile ilgili yardımcı ve arkadaşsal tavsiyeyi sağlamak için birilerine gerçek anlamda ihtiyaç duyar. Onların, etik üzerine eğitim almalarına gerek yoktur; fakat onlar, oldukça yeni olanla irtibat kurmanın olağanüstü görevi ile yüz yüze getirilmiş olarak, fazlasıyla emek verilerek öğrenilen niteliklerin olması gerektiği gibi \bibemph{yorumlanışına} ihtiyaç duymaktadır.
\vs p027 3:3 Etiğin yorumlayıcıları; yerleşik düzeyin erişiminden Kesinliğe Ulaşacak Olanların Fani Birlikleri’ne yapılacak resmi girişe kadar uzanan önemli süreç boyunca, görkemli varlıkların sayısız toplulukları için onların düzenlenmesi amacıyla onlara yardımda bulunmak bakımından Cennet’e varan unsurların paha biçilmez desteğine aittirler. Yükseliş kutsal yolcuları, Cennet Vatandaşları’nın sayısız türlerinin birçoğuyla Havona’nın yedi döngüsü üzerinde çoktan tanışmış bir halde olacaktır. Yüceltilmiş faniler aynı zamanda; bahse konu varlıkların eğitimlerinin büyük bir çoğunluğunu aldıkları yer olan, iç Havona döngüsü üzerindeki bütünleştirici birliklerin yaratılmış kökene sahip kutsal bir biçimde üçleştirilmiş evlatları ile birlikte içten bir ilişkiyi yaşamaktan memnuniyet duyarlar. Buna ek olarak yükseliş kutsal yolcuları diğer döngüler üzerinde, geleceğin açığa çıkarılmamış olan görevleri için hazırlanma amacıyla topluluk eğitimlerine ulaşmaya çalışan Cennet\hyp{}Havona sistemlerinin sayısız derecedeki açığa çıkarılmamış olan sakiniyle buluşur.
\vs p027 3:4 Tüm bu göksel dostluklar devamlı bir biçimde karşılıklıdır. Yükseliş halinde bulunan siz faniler sadece, bahse konu ardışık kâinat dostlarından ve bu türden olan gittikçe çoğalan kutsal birlikteliklerin sayısız düzeninden yararlanmazsınız; siz aynı zamanda, bahse konu kardeşsel varlıkların her birini zaman ve mekânın evrimsel dünyalarından bir yükseliş fanisiyle önceden kurmuş olduğu birlikteliği için sonsuza kadar daha iyi ve farklı yapan kendi kişiliğiniz ve deneyiminizden bazı şeyleri onlara aktarırsınız.
\usection{4.\bibnobreakspace Davranışın Yöneticileri}
\vs p027 4:1 Ne anlamsız adetler ne de yapay olan sınıflandırmanın emirleri halinde bulunan fakat bunun yerine içkin şekildeki görgü kuralları biçimindeki Cennet ilişkilerinin etiği içinde çoktan eğitim görmeleri sebebiyle yükselen faniler; Işığın ve Yaşamın merkezi Ada’sı üzerinde kısa süreli olarak ikamet edecek olan yüksek varlıkların kusursuz davranışının uygulamalarında Cennet toplumunun yeni üyelerini eğiten, davranışın birincil hizmetkâr ruhaniyetleri yöneticilerinin tavsiyesini almayı yararlı olarak görürler.
\vs p027 4:2 Ahenk, merkezi evrenin temel taşıdır; ve algı tarafından kavranabilen düzey Cennet üzerinde hüküm sürmektedir. Olması gereken davranış, felsefe vasıtasıyla elde edilen biçimdeki bilgi kanalıyla gelişme için hayati öneme sahiptir. Kutsallık’a yaklaşmada kutsal bir işleyiş düzeni bulunmaktadır; buna ek olarak bu işleyiş düzeninin elde edilmesi mutlaka Cennet üzerine kutsal yolcuların varışını beklemek zorundadır. Bahse konu durumunun bu ruhaniyeti Havona’nın döngülerine aktarılmıştır, fakat zamanın kutsal yolcularının eğitiminin nihai dokunuşları sadece Işığın Adası’na onların mevcut bir biçimde erişmelerinden sonra uygulanabilir.
\vs p027 4:3 Tüm cennet davranışı, her bakımdan doğal ve özgür olması itibariyle bütünüyle kendiliğinden gerçekleşmektedir. Fakat ebedi Ada üzerinde bir takım şeyleri gerçekleştirmenin olması gereken ve kusursuz bir yolu her zaman bulunmakta olup; davranışın yöneticileri, onları eğitmeye ek olarak adımlarını kusursuz bir refaha atmaları için yönlendirmek ve aksi halde kaçınılmaz olan karmaşadan ve belirsizlikten kutsal yolcuların kaçınmasını aynı zamanda etkin hale getirmek amacıyla “geçitler arasında bulunan yabancıların” yanındadır. Yalnızca bu türden bir düzenleme vasıtasıyla sınırsız olan karmaşadan kaçınılabilir; ve karmaşa Cennet üzerinde hiçbir zaman ortaya çıkmamaktadır.
\vs p027 4:4 Davranışın bu yöneticileri gerçekten, yüceltilmiş öğretmenler ve rehberler olarak hizmet eder. Başlıca olarak onlar, neredeyse sonsuz olan kapsamda bulunan yeni durumlar ve aşina olunmayan uygulamalar ile ilgili yeni fani sakinlerini eğitmek ile ilgilidir. Her ne kadar bahse konu bu uzun süreli hazırlanmanın tümü onun için gerçekleştirilse ve uzun seyahat buraya doğru olsa da, Cennet hala dışa vurulmayacak biçimde garip ve yerleşik düzeye nihai olarak erişenler için beklenmeyen bir biçimde yenidir.
\usection{5.\bibnobreakspace Bilginin Sorumluları}
\vs p027 5:1 Bilginin birincil hizmetkâr ruhaniyet sorumluları, yüksek “yaşayan kutsal mektuplar” olarak bilinir ve onlar Cennet üzerinde ikamet eden herkes tarafından okunur. Onlar, gerçek bilginin yaşayan kitapları olarak gerçeğin kutsal kayıtlarıdır. Siz, “hayatın kitabı” içindeki kayıtları duymuş bir halde bulunmaktasınız. Bilginin sorumluları, kutsal yaşamın ve yüce teminatın ebedi levhaları üzerine yazılmış olan kusursuzluğun kayıtları biçimindeki bu türden yaşayan kitaplardır. Onlar gerçekte, kendiliğinden ortaya çıkan kütüphaneler biçimindeki yaşayan unsurlardır. Evrenlerin gerçekleri, mevcut bir halde bu meleklerde kaydedildiği biçimiyle bahse konu bu birinci derece birincil hizmetkâr ruhaniyetleri içinde içkin bir haldedir; buna ek olarak bu durum için aynı zamanda, zamanın usu ve ebediyetin gerçeğinin bahse konu bu kusursuz ve tamamlanmış kaynaklarının akıllarında doğru olmayan bir şeyin kendisine yer bulması imkânsızdır.
\vs p027 5:2 Bahse konu bu sorumlular, ebedi Ada’nın sakinleri için eğitimin resmi olmayan derslerini yönlendirirler; fakat onların başlıca işlevi, kaynak gösterme ve doğrulama görevleridir. Cennet üzerinde kısa süreliğine ikamet eden herhangi bir unsur irade dâhilinde, bilmek istediği belirli bir bilgi veya gerçeğin yaşayan kaynağını kendi yanında bulabilir. Ada’nın kuzey uçlarında; aranılan bilginin toplu bir biçimde elde bulundurulmasının yöneticisini atayacak ve bunun sonucunda bilmek istediğiniz tam da bu muazzam varlıkların \bibemph{kendisinin} derhal oluşmasını sağlayacak, bilginin yaşayan bulucuları mevcut bir durumdadır. Burada artık siz ussal bilgiyi, bütünüyle düşünceye daldığınız kâğıtlarda aramak zorunda bulunmamaktasınız; bunun yerine şimdi siz yaşayan ussal bilgi ile yüz yüze bütünleşebilirsiniz. Siz bu nedenle yüce bilgiyi, onların nihai sorumluları olan yaşayan varlıklardan elde edersiniz.
\vs p027 5:3 Sizin doğrulamayı arzuladığınız bilginin tam olarak kendisi olan birincil hizmetkâr ruhaniyetini konumlandırdığınızda, tüm âlemlerin \bibemph{bütün} bilinen bilgilerini sizin için ulaşılabilir bir halde bulacaksınız; çünkü bilginin bu sorumluları, yerel ve aşkın\hyp{}evrenlerin yüksek melekleri ve ikincil hizmetkâr ruhaniyetlerinden Havona’nın üçüncül hizmetkâr ruhaniyetlerinin başlıca kaydedicilerine uzanan kaydedici meleklerin geniş ağının nihai ve yaşayan özetleridir. Buna ek olarak bilginin bu yaşayan birikimi, kâinatsal tarihin bir araya gelmiş olan özeti biçiminde bulunan Cennet’in resmi kayıtlarından farklıdır.
\vs p027 5:4 Gerçeğin bilgeliği kaynağını merkezi evrenin kutsallığından almaktadır, fakat deneyimsel bilgi biçimindeki bilgi başlangıcına büyük bir ölçekte zaman ve mekânın nüfuz alanlarında sahiptir. Bu nedenden dolayı, yüksek meleklerin ve birincil hizmetkâr ruhaniyetlerinin uçsuz bucaksız aşkın\hyp{}evren düzenlemelerinin yürütülmesi için ortaya çıkan gereksinim Göksel Kaydediciler tarafından sağlanır.
\vs p027 5:5 Kâinat bilgisini içkin bir biçimde elinde bulunduran bahse konu birinci derece birincil hizmetkâr ruhaniyetleri aynı zamanda, bu bilginin düzenlenmesi ve sınıflandırılmasından da sorumludur. Kâinat âlemlerinin tümünün yaşayan kaynak kütüphanesi olarak kendilerini oluşturmalarında onlar bilgiyi, her biri bir milyon alt bölüme ayrılmış olan yedi muhteşem düzeyde sınıflandırmıştır. Cennet’in sakinlerinin bu geniş bilgi kaynağına danışabileceği bahse konu imkân, yalnızca bilginin sorumlularının gönüllü ve ussal çabaları sayesinde gerçekleşmiştir. Koruyucular aynı zamanda; Havona döngülerinin herhangi biri üzerindeki tüm varlıklar için yaşayan hazinelerini karşılık beklemeden onlara sunan, ve Zamanın Ataları’nın mahkemeleri tarafından her ne kadar dolaylı olsa da geniş çaplı bir biçimde kullanılan unsurlar halinde merkezi evrenin engin öğretmenleridir. Fakat merkezi evrenin ve aşkın\hyp{}evrenin kullanımına açık olan bu yaşayan kütüphane yerel yaratımlar için erişilebilir bir niteliğe sahip değildir. Yalnızca dolaylama ve yansıtma biçiminde Cennet bilgisinin yararları yerel evrenlerde teminat altına alınabilir.
\usection{6.\bibnobreakspace Felsefenin Üstatları}
\vs p027 6:1 Nitelik bakımından ibadetin yüce memnuniyetinin hemen yanında felsefenin hoşnutiyeti bulunmaktadır. Şimdiye kadar bu kadar yükseğe veya ileri düzeye erişmiş olmasaydınız, bir çözüm teşebbüsü içinde felsefenin uygulanmasını gerektirecek sayıca bini bulan gizem kalmaya devam etmeyecekti.
\vs p027 6:2 Cennet’in felsefe üstatları; kâinat sorunlarını çözmeye girişmenin memnuniyet verici arayışı içinde, özgün ve yükseliş halinde bulunan unsurlar biçimindeki sakinlerinin akıllarını yönlendirmekten büyük keyif duyar. Bahse konu felsefenin birincil hizmetkâr ruhaniyet üstatları, bilinmeyen üzerinde üstünleşmek amacıyla bilginin doğruluğunu ve deneyimin gerçekliğini kullanan bilgeliğin varlıkları biçiminde, “cennetin ussal varlıklarıdır.” Onlarla birlikte bilgi doğruluğa erişip, deneyim bilgeliğe yükselir. Cennet üzerinde mekânın yükseliş kişilikleri varlığın yüksek ufuklarını deneyimler: Onlar bilgiye sahiptir; onlar gerçeği bilir; onlar doğruluğu düşünme biçiminde felsefe yapabilirler; hatta onlar, Nihayet’in kavramlarını kapsama arayışına girişebilir ve Mutlaklıklar’ın işleyiş biçimlerini kavramak için teşebbüste bulunabilirler.
\vs p027 6:3 Geniş Cennet nüfuz alanlarının güney sınırlarında felsefenin üstatları, bilgeliğin işlevsel yetmiş bölümü içinde ayrıntılı dersleri yönlendirirler. Burada onlar, Sınırsızlık’ın amaçları ve tasarıları üzerine söylevde bulunarak, deneyimleri eş güdüm haline getirmeyi amaçlarlar. Buna ek olarak onlar, kendilerinin bilgeliğine ulaşan herkesin bilgisini oluştururlar. Onlar, birçok kâinat sorununa karşı bir hayli özelleşmiş bakış açısını geliştirirler, fakat onların nihai yargıları her zaman görüş birliği bakımından bütünsel bir birlikteliktedir.
\vs p027 6:4 Bu Cennet filozofları; Havona’nın yüksek kavrama biçimi ve iletişimsel bilginin belirli Cennet şekillerini içine alan olası her türlü eğitim yöntemiyle öğretimde bulunur. Bilgi aktarımının ve düşünce taşınıcının bu yüksek biçimlerinin tümü, en yüksek derecede gelişmiş insan aklının kavrayış yetisinin bile tamamiyle ötesindedir. Cennet üzerinde bir saatin eğitimi, Urantia’nın kelime\hyp{}hafıza yöntemlerinin on bin yılına denk gelir. Siz, bu türden olan iletişim biçimlerini kavrayamazsınız; ve yalın bir ifadeyle fani deneyim içinde, bahse konu bu yöntemlerin karşılaştırılabileceği veya benzerlik gösterdiği hiçbir şey bulunmamaktadır.
\vs p027 6:5 Felsefenin üstatları, kâinat âlemlerinin tümünün yorumlanmasını mekânın dünyalarından yükselen bahse konu olan varlıklara aktarmada yüce olan nitelikte bir haz alır. Buna ek olarak felsefe her ne kadar yargılarında bilginin gerçekleri ve deneyimin doğruları olarak hiçbir zaman sabitleştirilemese de yine de siz; ebediyetin çözüme kavuşmamış sorunları ve Mutlaklıklar’ın uygulamaları üzerine bahse konu bu birinci derece birincil hizmetkâr ruhaniyetlerinin söylevlerini dinlediğinizde, bu yetkinleşmemiş sorularla ilgili belirli ve etkisi sürekli devam eden bir memnuniyeti hissedeceksiniz.
\vs p027 6:6 Cennet’in bahse konu bu ussal arayışları yayınlanmamaktadır; kusursuzluğun felsefesi sadece kişisel olarak mevcut olanlar için erişilebilir bir durumdadır. Çevreleyen yaratılmışlar, bu öğretileri sadece bu deneyimden geçenler ve mekânın âlemleri için bu bilgiyi peşi sıra ileri aşamalar için taşıyanlar vasıtasıyla öğrenir.
\usection{7.\bibnobreakspace İbadetin Yönlendiricileri}
\vs p027 7:1 İbadet, yaratılmış akli varlıkların tümünün en yüksek ayrıcalığı ve ilk sorumluluğudur. İbadet, Yaratanlar’ın kendilerine ait olan yaratılmışları ile içten ve kişisel olan ilişkilerinin doğruluğu ve gerçekliğinin tanınmasına ek olarak onun tasdik edilmesinin bilinçli ve memnuniyet verici eylemidir. İbadetin niteliği, yaratılmışın algısının derinliği tarafından belirlenir; buna ek olarak Tanrılar’ın sınırsız karakterinin bilgisi genişlemeye devam ederken ibadet eylemi, yaratılmış varlıklar tarafından bilinen en yüksek deneyimsel haz ve en seçkin memnuniyetin ihtişamına nihai olarak erişene kadar, her şeyi artan bir biçimde kapsamaya devam edecektir.
\vs p027 7:2 Cennet Adası ibadetin belirli mekânlarını içinde barındırırken, o neredeyse kutsal hizmetin bir geniş ibadethanesidir. İbadet; Tanrı’nın mevcudiyetine erişmek için onun hakkında yeteri kadar bilgi edinmiş varlıkların kendiliğinden duyumsadıkları coşku biçimindeki, onun mutluluk dolu kıyılarına tırmanan herkesin ilk ve baskın olan tutkusudur. Havona’ya olan içsel yolculuk boyunca bir döngüden diğerine ibadet, Cennet üzerinde onun dışavurumunun yönlendirilmesi ve bunun haricinde düzenlenmesi için gereksinim duyulana kadar artan bir tutkudur.
\vs p027 7:3 Cennet üzerinde memnuniyetle yaşanan yüce hayranlık ve ruhsal takdirin dönemsel, anlık, topluluksal ve diğer özel dışavurumları; birinci derece birincil hizmetkâr ruhaniyetlerinin özel bir birliğinin öncüllüğü altında uygulanır. İbadetin bahse konu yönlendiricilerinin öncülüğü altında bu türden onur verici hürmet, yüce memnuniyetin yaratılmış varlık amacına erişmesine ek olarak ulvi olan öz\hyp{}dışavurumun ve kişisel hoşnutiyetin kusursuzluğunun doruklarına erişir. Tüm öncül birincil hizmetkâr ruhaniyetleri, ibadetin yönlendiricileri olmaya can atar; ve şayet bahse konu görevin baş idarecileri dönemsel olarak bu toplulukları ayırmasa, tüm yükseliş varlıkları ibadet tutumunda kalmaya devam edecektir. Fakat ibadette bütüncül memnuniyete erişene kadar hiçbir yükseliş varlığı hiçbir zaman, ebedi hizmetin görevlerine katılmakta yükümlü değildir.
\vs p027 7:4 Yükseliş yaratılmışlarının öz\hyp{}dışavurumun bahse konu bu memnuniyetini elde etmelerini etkin hale getirebilecek, ve aynı zamanda Cennet düzeninin temel olan etkinliklerine önem vermelerini sağlayabilecek ibadetin nasıl olması gerektiğini öğretmek, ibadetin yönlendiricilerinin görevidir. İbadetin işleyiş biçimindeki gelişim olmadan, Cennet’e erişen olağan bir fani için ussal minnettarlığın ve yükseliş şükranının duygularını bütüncül ve tatminkâr bir biçimde dışa vurmak yüzlerce yılın geçmesini gerektirecektir. İbadetin yönlendiricileri; mekânın rahminden ve zamanın doğum sancılarından ortaya çıkmış olan bu muhteşem evlatların daha kısa bir süre içinde ibadetin bütüncül tatminini elde etmesini etkin hale getirmek amacıyla, şimdiye kadar dışavurumun bilinmeyen ve yeni alanlarını ortaya çıkarırlar.
\vs p027 7:5 Minnettarlığın dışa vurulması ve taşınmasına ait olan kabiliyetleri derinleştirmeye ve enginleştirmeye muktedir olan bütün kâinatın tüm varlıkları ve sanatları, Cennet İlahiyatları’nın ibadeti içinde onların en yüksek yetkinliği için uygulanır. \bibemph{İbadet, Cennet mevcudiyetinin en yüksek sevincidir}; o, Cennet’in canlandırıcı etkinliğidir. Dünya üzerinde yorgun olan akıllarınız için oyunsal etkinlik nasıl bir etkide bulunuyorsa; ibadet Cennet üzerindeki kusursuzlaştırılmış ruhlarınız için aynı etkide bulunacaktır. Cennet üzerindeki ibadetin türü tamamen fani kavrayışının ötesindedir; fakat onun ruhaniyetini burada Urantia üzerinde bile takdir etmeye başlayabilirsiniz, çünkü Tanrı’nın ruhaniyetleri şu an bile sizin içinizde ikamet etmekte, sizin üzerinizde durmakta ve gerçek ibadet için size esin kaynağı olmaktadır.
\vs p027 7:6 Cennet üzerinde ibadet için ayrılmış zaman ve mekân bulunmaktadır; fakat bunlar, ebedi Ada için deneyimsel yükselişin muazzam varlıklarının artan akli yapılarına ve genişleyen kutsallık tanıyışına ait olan ruhsal duyguların başından beri çoğalan doygunluğunu karşılamakta yetkin değildir. Grandfanda’nın zamanından beri birincil kutsal ruhaniyetleri hiçbir şekilde, Cennet üzerindeki ibadetin ruhaniyetini bütünüyle karşılamakta yetkin bir konumda bulunamamıştır. Orada her zaman, ibadet için hazırlanış tarafından ölçülen ibadetselliğin bir fazlası bulunmaktadır. Ve bu nedenden dolayı içkin kusursuzluğun kişilikleri hiçbir zaman bütüncül bir biçimde, zaman ve mekânın daha alt düzeyde bulunan dünyalarının ruhsal karanlığının derinliklerinden Cennet ihtişamına doğru yukarı bir doğrultuda bulunan yollarında yavaş bir şekilde emek vererek seyir halindeki varlıkların ruhsal duygularının devasa tepkilerini bütüncül olarak takdir edememiştir. Zamanın bu türden olan melekleri ve fanileri Cennet Güçleri’nin mevcudiyetine eriştiği zaman orada, Cennet’in melekleri için şaşırtıcı ve Cennet İlahiyatları’nın kutsal memnuniyetinin yüce sevincinin üretimi olan bir temsil biçimindeki çağların bir araya gelmiş olan duygularının dışavurumu ortaya çıkar.
\vs p027 7:7 Zaman zaman Cennet’in tümü, ruhsal ve ibadetsel dışavurumun baskın bir gel\hyp{}giti içinde kaybolmuş bir hale gelir. Sıklıkla ibadetin yönlendiricileri; ihtişamın kusursuz vatandaşları ve zamanın yükseliş yaratılmışları biçimindeki Cennet’in sakinlerinin içten ibadeti tarafından bütünüyle ve kesin bir biçimde memnun hale gelen Tanrılar’ın kutsal kalbini işaret eden, İlahiyat yerleşkesi ışığının üç katmanlı olan dalgalanmasının ortaya çıkışına kadar bu türden olan olguyu düzenleyemez. Bahse konu bu durum ne de başarı dolu bir işleyiş düzenidir! Yaratan Yaratıcı’nın sınırsız olan sevgisini bütünüyle tatmin etmesi gereken yaratılmışın akli sevgisi, Tanrılar’ın ebedi tasarımının ve amacının nasıl da bir kendini gerçekleştirmesidir.
\vs p027 7:8 İbadetin bütünlüğünün yüce memnuniyetine erişimden sonra siz, Kesinliğe Erişecek Olanların Birliği’ne katılım için yetkin hale geleceksiniz. Yükseliş süreci neredeyse tamamlanmış bir halde bulunmaktadır, ve yedinci sevinç ve bayram töreni kutlanması için hazırlanmaktadır. İlk sevinç ve bayram töreni, varlığı sürdürmenin amacı kesinleştiği andaki Düşünce Denetleyicisi ile yapılan fani anlaşmayı simgeler. İkinci sevinç ve bayram töreni, morontia hayatının uyanışının; üçüncüsü, Düşünce Düzenleyicisi ile bütünleşmenin; dördüncüsü, Havona içindeki uyanışın; beşincisi Kâinatın Yaratıcısı’nı bulmanın ve altıncısı ise zamanın ilk geçiş uykusundan Cennet uyanışı durumunun kutlanmasıdır. Yedinci sevinç ve bayram töreni, kesinliğe erişecek fani birliklerine olan katılımı ve ebedi hizmetin başlamasını simgeler. Kesinliğe erişecek olan bir unsur tarafından ruhani gerçekleşmenin yedinci düzeyine olan erişim muhtemel bir biçimde, ebediyetin sevinç ve bayram törenlerinin ilkinin kutlanmasına işaret edecektir.
\vs p027 7:9 Ve tüm hizmetkâr ruhaniyetlerinin en yüksek düzeyi olan Cennet birincil hizmetkâr ruhaniyetlerinin hikâyesi; kökeninizin dünyasından başlayarak, Kutsal Üçleme yeminini etmenizle ve Kesinliğe Erişecek Olanların Fani Birlikleri için toplanmanızla birlikte ibadetin yönlendiricileri tarafından nihai bir biçimde uğurlanmanıza kadar, kâinatsal bir sınıf olan bu varlıkların en başından beri sizin sürecinize olan katılımı böylelikle bu biçimde sonlanır.
\vs p027 7:10 Cennet Kutsal Üçlemesi’nin sonu gelmez bu hizmeti başlamak üzeridir; ve böylelikle kesinliğe erişecek olan unsur, Nihai olan Tanrı’nın sınayışıyla karşı karşıyadır.
\vs p027 7:11 [Uversa’dan olan bir Bilgeliğin Kesinleştiricisi tarafından sunulmuştur.]
