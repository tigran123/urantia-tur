\upaper{117}{Yüce olan Tanrı}
\vs p117 0:1 Mevcudiyetimiz’e sahip olduğumuz her ne evren konumu içinde olursa olsun Tanrı’nın iradesini yerine getirmemiz ölçüsünde, Yüce’nin her\hyp{}şeye\hyp{}gücü\hyp{}yeten potansiyeli o kadar bir adım daha mevcut hale gelmektedir. Tanrı’nın iradesi; Üç Mutlaklık içinde potansiyel hale geldiği, Ebedi Evlat içinde kişilikleştiği, Sınırsız Ruhaniyeti içinde evren eylemi için bir araya geldiği ve Cennet’in sonsuza kadar süren işleyiş biçimleri içinde ebedileştiği halde, İlk Kaynak ve Merkez’in amacıdır. Ve, Yüce olan Tanrı, Tanrı’nın bütüncül iradesinin en yüksek sınırlı dışavurumu haline gelmektedir.
\vs p117 0:2 Eğer tüm asli evren unsurları Tanrı’nın iradesini bütüncül bir biçimde gerçekleştirmeye göreceli olarak erişmiş olsalardı, bunun sonucunda, zaman\hyp{}mekân yaratılmışları ışık ve yaşam altında istikrara kavuşmuş hale gelir, ve bunun sonrasında, Yücelik’in ilahiyat potansiyeli olan Her\hyp{}Şeye\hyp{}Gücü\hyp{}Yeten, Yüce olan Tanrı’nın kutsal kişiliğinin ortaya çıkışında gerçeksel hale gelirdi.
\vs p117 0:3 Bir evrimleşen akıl kâinatsal aklın döngülerine uyumlu hale geldiğinde, bir evrimleşen evren merkezi evrenin işleyiş biçimine uygun olarak istikrara kavuştuğunda, bir ilerleyen ruhaniyet Üstün Ruhaniyetler’in bütünleşmiş hizmetiyle iletişim kurduğunda, bir yükseliş fani kişiliği nihai bir biçimde ikamet eden Düzenleyici’nin kutsal yönlendirişine uyumlu hale geldiğinde, bunun sonrasında Yücelik’in kutsallığı, kâinatsal gerçekleşime doğru bir adım daha ilerler.
\vs p117 0:4 Asli evrenin kısımları ve bireyleri Yüce’nin bütüncül evriminin bir yansıması olarak evrilirken, Yüce karşılığında, asli evren evriminin bütününün bileşimsel nitelikteki sonuçsal toplamıdır. Fani bakış açısından, her ikisi de evrimsel ve deneyimsel karşılık unsurlarıdır.
\usection{1.\bibnobreakspace Yüce Varlık’ın Doğası}
\vs p117 1:1 Yüce; fiziksel uyumun güzelliği, ussal anlamın gerçekliği ve ruhsal değerin iyiliğidir. O, gerçek başarının yarattığı hoşluk ve sonsuza kadar süren kazanımın yarattığı neşedir. O; asli evrenin ruh\hyp{}ötesi, sınırlı kâinatın bilinci, sınırlı gerçekliğin tamamlanışı ve Yaratan\hyp{}yaratıcı deneyimin kişilikleşimidir. Tüm gelecek ebediyet boyunca Yüce olan Tanrı, İlahiyat’ın kutsal üçleme ilişkileri içinde özgür iradesel deneyimin gerçekliğini sesi olacaktır.
\vs p117 1:2 Yüce Yaratanlar’ın kişilerinde Tanrılar Cennet’den; içinde, Yaratıcı’nın arayışında Cennet\hyp{}erişim yetkinliğine yükselebilecek yaratılmışları yaratmak ve onları evriltmek için, zaman ve mekânın nüfuz alanlarına inmiş halde bulunmaktadırlar. Alçalan konumdaki Tanrı\hyp{}açığa çıkaran Yaratanlar’a ek olarak yükselen konumdaki Tanrı\hyp{}arayan yaratılmışların bu evren ilerleyişi; ebedi ve kâinatsal kardeşliğin keşfi olan, içinde hem alçalış hem de yükseliş unsurlarının karşılıklı anlayışa eriştiği Yüce’nin İlahi evrimini açığa çıkaran niteliktedir. Yüce Varlık böylelikle, kusursuz\hyp{}Yaratan nedenselliğinin deneyimi ile kusursuzlaşmakta\hyp{}olan yaratılmış karşılığının sınırlı bileşimi haline gelir.
\vs p117 1:3 Asli evren, bütüncül bütünleşmenin olasılığını taşımakta, ve sürekli olarak ona ulaşmaya arzulamaktadır; ve, bu, bahse konu kâinatsal mevcudiyetin kâinatsal bütünlük olan Cennet Kutsal Üçlemesi’nin yaratıcı eylemleri ve güç emirlerinin bir sonucu olması gerçekliğinden doğmaktadır. Bu aynı kutsal\hyp{}üçlemesel birlik; evrenler Kutsal Üçleme özdeşleşiminin en yüksek seviyesine ulaşırken, sahip olduğu gerçekliğin artan bir biçimde belirgin hale geldiği Yüce içindeki sınırlı kâinatta dışa vurulmaktadır.
\vs p117 1:4 Yaratan’ın iradesi ve yaratılmışın iradesi, niteliksel olarak farklıdır; ancak, onlar aynı zamanda, deneyimsel olarak ortak kökenden gelmektedir, zira yaratılmış ve Yaratan evren kusursuzluğuna erişimde iş birliğinde bulunabilir. İnsan; Tanrı ile irtibat halinde görevde bulunabilir, böylece, ebedi bir kesinlik unsurunu beraberce yaratabilir. Tanrı; aracılığıyla yaratılmış deneyimin yüceliğine eriştiği, sahip olduğu Evlatlar’ın vücutlaştırılımlarında insanlık olarak bile görevde bulunabilir.
\vs p117 1:5 Yüce Varlık içinde, sahip olduğu irade tek bir kutsal kişiliğin dışavurumu olan tek bir İlahiyat içinde bütünleşmiş haldedir. Ve, bahse konu bu süreçte Nebadon’un Üstün Evladı’nın sahip olduğu irade artık kutsallık ve insanlığın iradesinin bir bütünlüğünden daha fazla bir şey iken, Yüce’nin bu iradesi, hem yaratılmış hem de Yaratan’ın iradesinden daha fazla olan bir şeydir. Cennet kusursuzluğu ve zaman\hyp{}mekân deneyiminin bu birliği, gerçekliğin ilahiyat düzeyleri üzerinde yeni bir anlam değerini açığa çıkarmaktadır.
\vs p117 1:6 Yüce’nin evrimleşen kutsal doğası, asli evren içindeki tüm yaratılmışların ve tüm Yaratanlar’ın sahip olduğu benzersiz deneyimin aslına uygun bir temsili haline gelmektedir. Yüce’de, yaratıcılık ve yaratılmışlık bir bütündür; onlar, kusursuzluğunkine ek olarak ve tamamlanmamışlığın zincirlerinden özgürleşmenin arayışı içinde ebedi yolu amaç edinirken, tüm sınırlı yaratılmışın başına gelen çok katmanlı sorunların çözümünün beraberinde getirdiği iniş ve çıkışlardan doğmuş deneyimle, sonsuza kadar bütünleşmiş halde bulunmaktadır.
\vs p117 1:7 Gerçeklik, güzellik ve iyilik; Ruhaniyet’in hizmetinde, Cennet’in ihtişamında, Evlat’ın bağışlayışında ve Yüce’nin deneyimi içerisinde ortak ilişki halindedir. Yüce olan Tanrı gerçekliğin, güzelliğin ve iyiliğin\bibemph{ tam da kendisidir}; zira kutsallığın bu kavramsallaşmaları, ideal nitelikteki deneyimin sınırlı düzeydeki en yüksek değerlerini yansıtmaktadır. Kutsallığın bu üçlü\hyp{}birlik niteliklerine ait ebedi kaynaklar, sınırlılık\hyp{}ötesi düzeylerdedir; ancak, bir yaratılmış ancak, gerçeklik\hyp{}ötesi, güzellik\hyp{}ötesi ve iyilik\hyp{}ötesi gibi bu tür kaynaklar üzerinde düşüncelere sahip olabilir.
\vs p117 1:8 Bir yaratan olarak Mikâil, dünyasal çocukları için Yaratan Yaratacısı’nın kutsal sevgisini dışa vurmuştu. Ve, bu kutsal şefkati keşfederek ve onu alarak insanlar, beden içindeki kardeşlerine bu derin sevgiyi açığa çıkarmayı amaç edinebilirler. Bu tür yaratılmış şefkati, Yüce’nin sevgisinin gerçek yansımasıdır.
\vs p117 1:9 Yüce, eş bir biçimde kapsayıcıdır. İlk Kaynak ve Merkez üç büyük Mutlak içinde potansiyel; Cennet, Evlat ve Ruhaniyet içinde mevcuttur; ancak, Yüce, yaratılmış çabasına ve Yaratan amacına eşit bir biçimde karşılık veren nitelikte, kişisel yüceliğin ve her\hyp{}şeye\hyp{}gücü\hyp{}yeten gücün bir varlığı olarak, hem mevcut hem de potansiyeldir; o, evren üzerinde kendi başına hareket edip, onun bütünlüğüne kendi başına karşılık göstermektedir; ve, o, tek seferde ve aynı anda yüce yaratan ve yüce yaratılmıştır. Yüceliğin İlahiyatı böylelikle, sınırlılığın bütünlüğünün bütüncül toplamının dışavurumudur.
\usection{2.\bibnobreakspace Evrimsel Büyümenin Kaynağı}
\vs p117 2:1 Yüce, zaman\hyp{}içindeki\hyp{}Tanrı’dır; onunki, zaman içindeki yaratılmış büyümesinin sırrıdır; o aynı zamanda, tamamlanmamış şimdi üzerindeki üstünlük ve kusursuzlaşmakta olan geleceğin tamamlanışıdır. Ve, tüm sınırlı büyümenin nihai meyveleri: kişiliğin bütünleştirici ve yaratıcı mevcudiyeti sayesinde, ruhaniyet tarafından ve akıl aracılığıyla gerçekleştirilen denetlenmiş güçtür. Tüm bu büyümenin bütünleşen sonucu, Yüce Varlık’dır.
\vs p117 2:2 Fanı insan için, mevcudiyet büyümeye denktir. Ve, benzer bir biçimde, bu durum gerçekten de daha büyük olan kâinatsal açıdan bile, doğru görümdedir; zira, ruhaniyet\hyp{}tarafından\hyp{}yönlendirilmiş\hyp{}mevcudiyet kesin bir biçimde, düzeyin derinleşimi olarak --- deneyimsel büyüme ile sonuçlanan görünüme sahiptir. Bizler uzunca bir süredir, buna rağmen; mevcut evren çağındaki yaratılmış deneyimini tanımlayan mevcut büyümenin Yüce’nin bir işlevi olduğu görüşünü beslemiş bulunmaktayız. Bizler eşit bir biçimde; bu türden büyümenin Yüce’nin büyüme çağına özel bir durum olduğu, ve bu büyümenin Yüce’nin büyümesi ile sonlanacağı görüşüne sahibiz.
\vs p117 2:3 Kutsal\hyp{}üçlemesel\hyp{}hale\hyp{}getirilmiş yaratılmış evlatların düzeyine bir bakın: Onlar, mevcut evren çağında doğmakta ve yaşamaktadır; onlar, akıl ve ruhaniyet donanımlarına ek olarak kişiliklere sahiptirler. Bu unsurlar, deneyimlere ve onlara ait hafızaya sahiptirler; ancak, onlar, yükseliş unsurları olarak \bibemph{büyümemektedirler}. Her ne kadar mevcut evren çağı \bibemph{içinde} bulunuyor olsalar da kutsal\hyp{}üçlemesel\hyp{}hale\hyp{}getirilmiş bu yaratılmış evlatların gerçekte; Yüce’nin büyümesinin tamamlanışını takip edecek çağ olarak --- bir sonraki çağa \bibemph{ait oldukları} bizlerin inanışı ve anlayışıdır. Bu nedenle onlar, tamamlanmamışlılıktaki ve onun beraberinde getirdiği büyümedeki Yüce’nin mevcut düzeyi bakımından o Yüce \bibemph{içinde} değildir. Böylelikle onlar; bir sonraki evren çağı için yedekte bekletilen bir biçimde, mevcut evren çağının deneyimsel büyümesin katılmayan niteliktedirler.
\vs p117 2:4 Kutsal Üçleme tarafından bütünleştirilmiş olan Kudretli İleticiler olarak benim ait olduğum düzey, mevcut evren çağının büyümesine katılmayan niteliktedir. Bir açıdan, bizler, gerçekten de Kutsal Üçleme’nin Hareketsiz Evlatları gibi, önceki evren çağıyla aynı düzeyde bulunmaktayız. Tek bir şey kesindir: Bizlerin düzeyi Kutsal Üçleme bütünleşimi tarafından sabitleştirilmiş olup, deneyim büyüme ile sonuçlanmamaktadır.
\vs p117 2:5 Bu durum; kesinlik unsurları veya Yüce’nin büyümesi sürecinin katılımcıları olan evrimsel ve deneyimsel düzeylerin hiçbiri için gerçeklik taşımamaktadır. Cennet erişimini ve kesinlik unsur düzeyini amaç edinebilecek olan mevcut anda Urantia üzerinde yaşayan siz faniler; bu türden bir nihai sonun yalnızca, siz Yüce içinde ve ona ait olduğunuz için, ve böylelikle Yüce’nin büyüme döngüsünün katılımcıları olduğunuz için, gerçekleştirilebilir olduğunu anlamalısınız.
\vs p117 2:6 Bir zaman içerisinde Yüce’nin büyümesinin sonunun yaşanacağı vakit gelecektir; onun düzeyi, (enerji\hyp{}ruhaniyet bakımından) tamamlanmaya erişecektir. Yüce’nin evriminin bu sonlanışı, aynı zamanda, Yüceliğin bir parçası olarak yaratılmış evriminin sonuna şahit olacaktır. Dış uzayın evrenlerini ne tür büyümenin tanımlayacağını bilmemekteyiz. Ancak bizler onun; yedi aşkın\hyp{}evrenin evrimine ait mevcut evren çağı içinde bulunan her şeyden çok farklı bir şeyin olacağından oldukça eminiz. Kuşkusuz bir biçimde; Yüce’nin büyümesinin bu mahrumiyeti için dış uzay unsurlarını telafi etmek, asli evrenin evrimsel vatandaşlarının faaliyeti olacaktır.
\vs p117 2:7 Mevcut evren çağının tamamlanışı üzerine mevcut olan bir biçimde Yüce Varlık, asli evrende deneyimsel bir egemen olarak faaliyet gösterecektir. Dış uzay unsurları --- bir sonraki evren çağının vatandaşları olarak; Her\hyp{}Şeye\hyp{}Gücü\hyp{}Yeten Yüce’nin egemenliğine olan varsayımsal nitelikteki evrimsel erişimin bir yetkinliği olarak, böylece mevcut evren çağının güç\hyp{}kişilik bileşimine olan yaratılmış katılımını dışarıda bırakan bir biçimde, bir aşkın\hyp{}evren\hyp{}sonrası büyüme potansiyeline sahiptir.
\vs p117 2:8 Böylelikle, Yüce’nin tamamlanmamışlığı; mevcut evrenlerin yaratılmış\hyp{}yaratımının evrimsel büyümesini mümkün kıldığı için bir erdem olarak değerlendirilebilir. Yokluğun erdemi bulunmaktadır; zira o deneyimsel olarak doldurulabilir.
\vs p117 2:9 Sınırlı felsefe içinde en ilgi çekici sorulardan bir tanesi şudur: Yüce Varlık asli evrenin evrimine karşılık olarak mı gerçekleşmekte, yoksa, bu sınırlı kâinat Yüce’nin kademeli gerçekleşimine karşılık olarak mı evirilmektedir? Ya da, gelişimleri için onlar karşılıklı olarak birbirlerine bağlılar mıdır? Her birinin diğerinin büyümesini başlattığı biçimde evrimsel nitelikteki karşılık\hyp{}eşleri midir? Bunların arasında biz şundan eminiz: yüksek düzeyde ve alt düzeyde olsun, yaratılmışlar ve evrenler, Yüce içerisinde evirilmektedirler; ve, onlar evrilirlerken, bu evren çağının bütüncül sınırlı etkinliğinin bütünleşmiş toplamı ortaya çıkmaktadır. Ve, bu; tüm kişilikler için, Yüce olan Tanrı’nın her\hyp{}şeye\hyp{}gücü\hyp{}yeten gücünün evriminin görünümü olarak, Yüce Varlık’ın görünümüdür.
\usection{3.\bibnobreakspace Yüce’nin Evren Yaratılmışları için Önemi}
\vs p117 3:1 Yüce Varlık, Yüce olan Tanrı ve Her\hyp{}Şeye\hyp{}Gücü\hyp{}Yeten olarak çeşitli biçimlerde adlandırılmış olan kâinatsal gerçeklik; tüm sınırlı gerçekliklerin ortaya çıkan fazlarının çok katmanlı ve kâinatsal bileşimidir. Ebedi enerji, kutsal ruhaniyet ve kâinatsal aklın uçsuz bucaksız çeşitlenişi; en yüksek düzeydeki sınırlı tamamlanışın ilahiyat düzeyleri üzerinde birey gerçekleşimi olarak, tüm sınırlı büyümenin bütünlüksel toplamı niteliğindeki Yüce’ye ait evrimle olan sınırlı sonuçlanışına erişir.
\vs p117 3:2 Yüce, mekânın gök adasal panoramasına somut olarak dönüşen üçlü ilahiyat birliklerinin yaratıcı sonsuzluğunun boyunca aktığı kutsal kanaldır; bu akımın karşısında, zamanın şu muhteşem kişilik draması ortaya çıkar: aklın aracılığı vasıtasıyla, enerji\hyp{}maddesi üzerindeki ruhaniyet egemenliği.
\vs p117 3:3  İsa “Ben yaşayan yolum” dedi; ve, o, benlik bilincinin maddi düzeyinden Tanrı\hyp{}bilincinin ruhsal düzeyine olan yaşayan yoldur. Ve, her nasıl o bireyden Tanrı’ya olan yükselişin yaşayan yolu ise, Yüce aynı şekilde, sınırlı bilinçten bilincin aşkınlığına olan, hatta absonit düzeyin kavrayışına kadar bile, yaşayan yoldur.
\vs p117 3:4 Sizlerin Yaratan Evladınız, gerçekte; İnsanın Evladı olarak Yeşu bin Yusuf’un gerçek insanlığından, sınırsız Tanrı’nın Evladı olarak Nebadonlu Mikâil’in Cennet kutsallığına kadar ilerleyişin bu evren yolunun bütüncül kat edilişini kişisel biçimde deneyimlediği için, insanlıktan kutsallığa olan bu türden bir yaşayan kanal olabilir. Benzer bir biçimde, Yüce Varlık, sınırlı sınırlılıkların aşkınlığına olan evren yaklaşımı olarak faaliyet gösterebilir; zira, o, tüm yaratılmış evriminin, ilerleyişinin ve ruhsallaşımının mevcut bedensel hali ve kişisel nitelikli simgesel bütünlüğüdür. Cennet’den gelen alçalış kişiliklerinin sahip olduğu asli evren deneyimleri bile; zamanın kutsal yolcularının sahip oldukları yükseliş deneyimlerinden oluşan onun bütünlüğünü tamamlar nitelikteki kendi deneyiminin bir parçasıdır.
\vs p117 3:5 Fani insan, mecazi anlamının ötesinde bir biçimde, Tanrı’nın imgesinde yaratılmıştır. Fiziksel bir bakış açıdan bu ifade neredeyse hiçbir şekilde doğru değildir; ancak, belirli evren potansiyellikleri baz alındığında bu mevcut bir gerçekliktir. İnan ırkı içerisinde evrimsel erişimin aynı dramının benzeri, kâinat âlemlerinin tümü içerisinde, çok daha büyük bir ölçekte, gerçekleştiği gibi adım adım yaşanmaktadır. Bir özgür iradesel kişilik olarak insan, Yüce’nin sınırlı potansiyelliklerinin mevcudiyeti içinde kişilik\hyp{}dışı bir birim niteliğinde bir Düzenleyici ile irtibat halinde yaratıcı hale gelmektedir; ve, sonuç, bir ölümsüz ruhun filizlenişidir. Evrenler içinde zaman ve mekânın Yaratan Kişilikleri; Cennet Kutsal Üçlemesi’nin kişilik\hyp{}dışı ruhaniyeti ile faaliyet göstermekte olup, böylelikle, İlahiyat gerçekliğinin yeni bir güç potansiyelinin yaratıcısı haline gelmektedir.
\vs p117 3:6 Bir yaratılmış olarak fani insan tamı tamına, ilahiyat olan Yüce Varlık’a benzememektedir; ancak, insanın evrimi, bazı açılardan, Yüce’nin büyümesine benzemektedir. İnsan bilinçli bir biçimde, kendi kararlarının kuvveti, gücü ve kararlılığıyla maddi olandan ruhsal olana doğru büyür; o aynı zamanda, sahip olduğu Düşünce Düzenleyicisi ruhsal olandan aşağı yöndeki morontia ruh düzeylerine olan erişim için yeni yöntemler geliştirdikçe büyür; ve, bir kez daha ruh, kendi içinde ve kendi içerisinden büyümeye başlayan bir biçimde, mevcudiyetini kazanır.
\vs p117 3:7 Bu bir şekilde, Yüce Varlık’ın genişlemesine benzemektedir. Onun egemenliği, Yüce Yaratan Kişilikleri’nin eylemleri ve kazanımlarından başat bir biçimde temelini alarak büyümektedir; bu, asli evrenin yöneticisi olarak kendi gücünün ihtişamının evrimidir. Onun ilahi doğası benzer bir biçimde, Cennet Kutsal Üçlemesi’nin mevcudiyet\hyp{}öncesi birliğine bağlıdır. Ancak, orada hâlihazırda, Yüce olan Tanrı’nın evriminin başka bir yönü bulunmaktadır: O yalnızca, evirilmiş\hyp{}Yaratan ve kökenini\hyp{}Kutsal Üçleme’den\hyp{}almış unsur değildir; o, aynı zamanda, kendisi\hyp{}tarafından\hyp{}evirilmiş ve kökenini\hyp{}kendisinden\hyp{}almış unsurdur. Yüce olan Tanrı’nın kendisi, kendi ilahiyat gerçekleşiminin yaratıcı katılımcısı olarak, bir özgür iradesel unsurdur. İnsan düzeyindeki morontiasal ruh benzer bir biçimde, kendisinin ölümsüzleşiminde ortak yaratıcı eş olarak, bir özgür iradesel unsurdur.
\vs p117 3:8 Yaratıcı, Cennet enerjileri üzerinde yapılan değişiklikte ve Yüce’ye olan bu karşılıkları mevcut kılmada Bütünleyici Bünye ile birlikte iş birliğinde bulunmaktadır. Yaratıcı; bir zaman zarfında Yüce’nin egemenliği ile sonuçlanacak eylemlere sahip olan Yaratan kişiliklerin üretiminde, Ebedi Evlat ile iş birliğinde bulunmaktadır. Yaratıcı; Yüce’nin tamamlanmış evriminin bu egemenliği üstlenmek için kendisini yetkin hale getirdiği ana kadar, asli evrenin yöneticileri olarak faaliyet göstermesi için Kutsal Üçleme kişiliklerinin yaratımında hem Yaratan hem de Ruhaniyet ile iş birliğinde bulunmaktadır. Yaratıcı, Yücelik’in evriminin gelişimi için bu ve diğer birçok biçimde kendisine ait İlahiyat ve İlahiyat\hyp{}dışı yardımcı unsurları ile eş güdümde bulunmaktadır; ancak, o aynı zamanda, bu hususlarda tek başına faaliyet göstermektedir. Ve, onun yalnız gerçekleştirdiği faaliyet muhtemelen en iyi, Düşünce Düzenleyicileri ve onların ilişkili birimlerinin hizmetinde açığa çıkarılmaktadır.
\vs p117 3:9 İlahiyat; Cennet Kutsal Üçlemesi’nde varoluşsal, Yüce’de deneyimsel, ve fanilerde, Düzenleyici bütünleşimi içinde yaratılmış\hyp{}tarafından\hyp{}gerçekleştirilen nitelikte, bütünlüktür. Fani insan içindeki Düşünce Düzenleyicileri’nin mevcudiyeti, evrenin temel birliğini açığa çıkarmaktadır; zira, kâinat kişiliğinin olası en alt düzeydeki türü olarak insan, kendi bünyesi içerisinde en yüksek ve ebedi gerçekliğin mevcut bir nüvesini, hatta tüm kişiliklerin kökensel Yaratıcısı’nı, taşımaktadır.
\vs p117 3:10 Yüce Varlık; Cennet Kutsal Üçlemesi ile olan irtibatı sayesinde ve bu Kutsal Üçleme’nin yaratan ve idareci çocuklarının kutsallık başarılarının sonucunda büyümektedir. İnsanın ölümsüz ruhu, kendisine ait ebedi nihai sonu; Cennet Kutsal Üçlemesi’nin kutsal mevcudiyeti ile ilişkilem vasıtasıyla ve insan aklının kişilik kararları uyarınca evirmektedir. Yüce olan Tanrı için Kutsal Üçleme ne ise, evrimleşen insan için Düzenleyici odur.
\vs p117 3:11 Mevcut evren çağı boyunca Yüce Varlık’ın; eylemin sınırlı olasılıklarının, zaman ve mekânın yaratıcı birimleri tarafından tümüyle denendiği durumlar haricinde, bir yaratan olarak doğrudan bir şekilde faaliyet göstermeye yetkin olmadığı görünmektedir. Bu kadar uzun evren tarihi boyunca bu tür bir istisna sadece bir kez gerçekleşmiştir; evren yansıması hususunda sınırlı eylemin olasılıkları tümüyle denendiğinde, bunun sonucunda Yüce, tüm öncül yaratan eylemlerinin yaratıcı sonuçlayıcısı olarak faaliyette bulunmuştu. Ve, bizler, onun; öncül yaratıcılık, yaratıcı etkinliğin yerinde bir döngüsünü tamamladığında, gelecek çağlarda bir sonuçlayıcı olarak tekrar faaliyet göstereceğine inanmaktayız.
\vs p117 3:12 Yüce Varlık insanı yaratmamıştı; ancak, insan, kelimenin tam anlamıyla, bahse konu tam da bu hayatının kökenini aldığı biçimde, Yüce’nin potansiyelliğinden yaratılmıştı. Ne de Yüce, insanı evrimleştirmektedir; buna rağmen, Yüce’nin kendisi, evrim tam da özüdür. Sınırlı olanın bakış açısından bizler gerçekte; Yüce’nin enginliğinde yaşamakta, hareket etmekte ve varlığımıza sahip olmaktayız.
\vs p117 3:13 Yüce görünüşte, kökensel nedenselliği başlatamamaktadır; ancak, göründüğü kadarıyla o, tüm evren büyümesinin hızlandırıcısı olup, tüm deneyimsel\hyp{}evrimsel varlıkların nihai sonu ile ilgili bütünlük sonuçlamasını sağlama nihai sonuna sahiptir. Yaratıcı, sınırlı bir kâinatın kavramsallığının kökenidir; Yaratan Evlat, Yaratıcı Ruhaniyetler’in rızası ve eş güdümüyle zaman ve mekân içerisinde bu düşünceyi gerçeksel hale getirmektedir; Yüce, bütüncül sınırlılığı sonuçlandırmakta ve absonit düzeyin nihai sonu ile birlikte onun ilişkisini kurmaktadır.
\usection{4.\bibnobreakspace Sınırlı Tanrı}
\vs p117 4:1 Varlığın düzeyinin ve kutsallığının kusursuzlaştırılımı için yaratılmış yaratımının sonu gelmez mücadelelerini gördüğümüzde, bu sonu gelmez çabaların, kutsal öz gerçekleşimi için Yüce’nin sonu gelmez çabasının simgesi olduğuna inanmamaktan başka bir şeyi yapamıyoruz. Yüce olan Tanrı, sınırlı İlahiyat’dır; ve, o, tam anlamıyla, sınırlı olanın sorunlarıyla baş etmek zorundadır. Bizlerin mekânın evrimsel gelişmeleri içindeki zamanın iniş\hyp{}çıkışları ile olan mücadeleleri; benliğin gerçekliğininkine ek olarak, sahip olduğu evrimleşen doğasının olasılığın en uç sınırlarına genişlettiği eylemin nüfuz alanı içerisinde egemenliğin tamamlanışına dair çabalarının yansımasıdır.
\vs p117 4:2 Asli evren boyunca Yüce, dışavurum için çabalamaktadır. Onun kutsal evrimi orantısal biçimde, mevcudiyet içindeki her kişiliğin bilge\hyp{}eylemine bağlıdır. Bir insan varlığı ebedi kurtuluşu seçtiğinde, o nihai sonu ortak bir biçimde yaratmaktadır; ve, bu yükseliş fanisinin yaşamı içinde sınırlı Tanrı, kişilik öz gerçekleşiminin artış göstermiş bir oranınkine ek olarak deneyimsel egemenliğin bir genişlemesini bulmaktadır. Ancak, eğer bir yaratılmış ebedi süreci reddederse, bu yaratılmışın tercihine bağlı olmuş olan Yüce’nin bu kısmı, yerine getirilecek veya ortaklaşa sahip olunacak nitelikteki deneyim tarafından telafi edilmesi zorunlu bir mahrumiyet biçiminde, kaçınılmaz gecikmeyi deneyimlemektedir; kurtuluş\hyp{}dışı unsurun kişiliği ise, Yüce’nin İlahiyatı’nın bir parçası haline gelerek, yaratımın ruh\hyp{}ötesi bünyesine katılır.
\vs p117 4:3 Tanrı, o kadar derinden seven biçimde, o kadar güven duymaktadır ki, koruması ve öz gerçekleşimini sağlaması için kutsal doğasının bir parçasını insan varlıklarının bile ellerine vermektedir. Düzenleyici mevcudiyeti olarak Yaratıcı’nın doğası, fani varlığın tercihinden bağımsız olarak yok edilemez niteliktedir. Evrimleşen benlik olarak Yüce’nin evladı, her ne kadar yanlış yönlendirilmiş türden bir benliğin potansiyel olarak bütünleşen kişiliği Yüceliğin İlahiyatı’nın bir etkeni olarak varlığını sürdürecek olsa da, yok edilebilir.
\vs p117 4:4 İnsan kişiliği gerçekten anlamda, yaratılmışlığın bireyselliğini yok edebilir; ve, her ne kadar bu türden kâinatsal bir intiharın yaşamında her şey değerli olmuş olsa da, \bibemph{bu nitelikler bir bireysel yaratılmış olarak varlığını sürdürmemektedir}. Yüce tekrar olarak, evrenlerin yaratılmışları içerisinde dışavurumunu bulacaktır; ancak hiçbir biçimde o özel birey olarak değil; yükseliş\hyp{}dışı bir unsurun benzersiz kişiliği Yüce’ye, bir su damlasının denize geri döndüğü gibi geri dönmektedir.
\vs p117 4:5 Sınırlı olanın kişisel birimlerinin herhangi bir tekil eylemi, Yüce Bütünlük’ün nihai dışavurumu karşısında göreceli olarak önemsizdir; ancak, bütün yine de, çok katmanlı birimlerin bütüncül eylemlerine bağlıdır. Bireysel faninin kişiliği, Yücelik’in bütünü karşısında önemsizdir; ancak, her insan varlığının kişiliği, sınırlı olan içinde yeri doldurulamaz bir anlam\hyp{}değeri temsil etmektedir; bir kez dışa vurulmuş halde kişilik, bu yaşayan kişiliğin devem eden mevcudiyeti dışında özdeş dışavurumunu bir kez daha bulamamaktadır.
\vs p117 4:6 Ve, böylece, bizler birey dışavurumumuzu arzularken, Yüce bizler içinde, ve bizlerle birlikte, ilahiyat dışavurumunu arzulamaktadır. Biz Yaratıcı’yı bulurken, Yüce tekrar, her şeyin Cennet Yaratanı’nı bulmuş olur. Birey gerçekleşiminin sorunlarının üstesinden geldiğimizde, deneyimin Tanrısı, zaman ve mekânın evrenleri içinde her\hyp{}şeye\hyp{}gücü\hyp{}yeten yüceliği elde etmektedir.
\vs p117 4:7 İnsanlık, kâinat içinde çabasız yükselmemektedir; ne de Yüce, amaçsız ve ussal eylem olmadan evrimleşmemektedir. Yaratılmışlar, sadece eylemsizlikle kusursuzluğa erişmemektedir; ne de Yücelik’in ruhaniyeti, sınırlı yaratımın sonu gelmez yardım hizmeti olmadan Her\hyp{}Şeye\hyp{}Gücü\hyp{}Yeten Yüce’nin gücünü gerçek hale getirebilmektedir.
\vs p117 4:8 İnsanın Yüce ile olan zamansal ilişkisi; \bibemph{göreve} olan kâinatsal hassasiyet, ve onun kabulü olarak, kâinatsal ahlakın temelidir. Bu, göreceli doğru ve yanlışın zamansal anlayışının ötesine geçen bir ahlaktır; o, doğrudan bir biçimde, öz bilince sahip yaratılmışın deneyimsel İlahiyat’a karşı olan deneyimsel sorumluluğuna dayanan bir ahlaktır. Fani insan ve tüm diğer sınırlı yaratılmışlar; Yüce içinde mevcut olan enerji, akıl ve ruhaniyetin yaşayan potansiyelinden yaratılmışlardır. Bir kesinlik unsurunun ölümsüz ve kutsal karakterinin yaratımı için Düzenleyici\hyp{}fani yükseliş unsuru Yüce’den faydalanmaktadır. Yüce’nin tam da bahse konu bu gerçekliğinden Düzenleyici, insan iradesinin rızası ile birlikte, Tanrı’nın bir yükseliş evladının ebedi doğasına ait işleyiş biçimlerini dokumaktadır.
\vs p117 4:9 Bir insan kişiliğinin ruhsallaşımı ve ebedileşimi sürecinde Düzenleyici ilerleyişinin evrimi, doğrudan bir biçimde, Yüce’nin sahip olduğu egemenliğin deneyimlediği bir genişlemenin sonucudur. İnsan evriminde bu türden kazanımlar aynı zamanda, Yüce’nin evrimsel gerçekleşimi içindeki kazanımlardır. Yaratılmışların Yüce olmadan evirilemeyecek nitelikte bulunmuş olacakları gerçekken, aynı zamanda muhtemel bir biçimde, tüm yaratılmışların tamamlanmış evriminden bağımsız olarak Yüce’nin evrimine bütüncül bir biçimde erişilemeyeceği de gerçektir. Bu hususta, öz bilince sahip kişiliklerin şu büyük kâinatsal sorumluluğu yatmaktadır: Yüce İlahiyat kesin bir biçimde, fani iradesinin tercihine bağlıdır. Ve, yaratılmış evriminin ve Yüce’nin ortak ilerleyişi, evren yansımasının incelenemeyecek işleyiş biçimleri üzerinden Zamanın Ataları’na aslına uygun bir biçimde ve bütünüyle işaret edilir.
\vs p117 4:10 Fani insana verilen büyük zorluk şudur: Kâinatın deneyimlenebilen değer anlamlarını evrimleşen benliğinize eklemleyen bir biçimde kişilikleştirmeye karar verecek misiniz? Veya kurtuluşu reddederek; Yücelik’in bu sırlarının, sınırlı Tanrı’nın evrimine olan bir yaratılmış katkısına \bibemph{kendine has} biçimle girişecek olan başka bir zamandaki bir diğer yaratılmışın eylemini bekleyen bir şekilde, hareketsiz beklemesine izin mi vereceksiniz? Ancak, bu, onun Yüce’ye olan katılımı olacaktır, sizin değil.
\vs p117 4:11 Bu evren çağının büyük mücadelesi, henüz dışa vurulmamış olan her şeyin gerçekleşimini arzulama biçiminde --- potansiyel ve mevcut olan arasındadır. Eğer fani insan, Cennet serüveni sürecine girişirse, o, ebediyetin nehri içindeki ırmaklar gibi hareket eden bir biçimde, zamanın hareketlerini takip etmektedir; eğer fani insan ebedi süreci reddederse, sınırlı evrenler içindeki olayların akışına karşı hareket etmektedir. Kendiliğinden bir işleyiş biçimiyle gerçekleşen yaratım, alt edilemez bir biçimde, Cennet Yaratıcısı’nın amacının gerçekleşimi doğrultusunda ilerlemektedir; ancak, özgür iradesel yaratım, ebediyetin serüvenine olan kişisel katılım rolünü kabul etme veya reddetme tercihine sahiptir. Fani insan, insan mevcudiyetinin yüce değerleri yok edemez; ancak, o, kendi kişisel deneyimi içinde bu değerlerin evrimini oldukça kesin bir biçimde engelleyebilir. İnsan benliğinin Cennet yükselişine katılmayı bu şekilde reddedişinin ölçüsünde, Yüce’nin asli evren içindeki kutsallık dışavurumuna elde etmesi aynı ölçüde gecikir.
\vs p117 4:12 Fani insanı kollamak için, yalnızca Cennet Yaratıcısı’nın Düzenleyici mevcudiyeti değil, aynı zamanda, Yüce’nin geleceğine ait çok küçük bir payın nihai sonu üzerindeki denetim de verilmiştir. Zira, insan insanlık nihai sonuna eriştikçe, Yüce ilahiyat düzeyleri üzerinde nihai sonu elde etmektedir.
\vs p117 4:13 Ve, böylece, bir zamanlar her birimizi beklemiş olan karar, her birinizi beklemektedir: Fani aklın kararlarına bu kadar bağımlı olan Zamanın Tanrısı’nı başarısız mı kılacaksınız? Evrenlerin Yüce kişililiğini hayvansal gerilemeye ait olan tembellikle başarısız mı kılacaksınız? Her bir yaratıma bu kadar bağımlı olan her yaratılmışın büyük kardeşini başarısız mı kılacaksınız? Sizler; Cennet Yaratıcısı’nın keşfine ek olarak Yüceliğin Tanrı’nın arayışına, ve onun evrimine, olan kutsal katılım biçiminde --- evren sürecinin büyüleyici ufku gözünüzün önünde serilmişken, gerçekleşimi\hyp{}yaşanmamış âleme geçmenize izin mi vereceksiniz?
\vs p117 4:14 Tanrı’nın hediyeleri --- kendisinin gerçeklik bahşedilişi olarak --- ondan gerçekleşmiş ayrılıklar değillerdir; o, yaratımı kendisinden tecrit etmemektedir; ancak o, Cennet’i çevreleyen yaratımlar içinde gerilimler yaratmıştır. Tanrı ilk olarak insanı derinden sevmekte, ve ona --- ebedi gerçeklik olarak --- ölümsüzlüğün potansiyelini bahşetmektedir. Ve, insan Tanrı’yı derinden severken, benzer bir biçimde mevcudiyet içinde ebedi hale gelmektedir. Ve burada gizem söz konusudur; insan derin sevgi vasıtasıyla Tanrı’ya daha yakınlaştığında, bu insanın --- mevcudiyeti olarak --- daha büyük gerçekliği ortaya çıkmaktadır. İnsan Tanrı’dan daha fazla geri çekilirse, daha yakın bir biçimde --- mevcudiyetin sonlanışı olarak --- gerçek\hyp{}dışılığa yaklaşmaktadır. İnsan Yaratıcı’nın iradesini gerçekleştirmek için kendi iradesini kutsal biçimde adadığında, insan Tanrı’ya\bibemph{ sahip olduğu} her şeyi verdiğinde, bunun sonucunda Tanrı, insanı olduğundan daha fazlası yapmaktadır.
\usection{5.\bibnobreakspace Yaratımın Ruh\hyp{}Ötesi Bütünlüğü}
\vs p117 5:1 Büyük Yüce, asli evrenin kâinatsal ruh\hyp{}ötesi bütünlüğüdür. Onun içinde, kâinatın nitelikleri ve nicelikleri kendilerine ait ilahiyat yansımasını bulur; onun ilahi doğası, evrimleşen evrenler boyunca tüm yaratılmış\hyp{}Yaratan doğasının bütüncül enginliğinin mozaiksel bileşimidir. Ve, Yüce aynı zamanda, evrimleşen bir evren amacını içine alır konumdaki bir yaratıcı iradesini bünyesinde barındıran gerçekleşim halindeki bir İlahiyat’dır.
\vs p117 5:2 Sınırlı olana ait ussal, potansiyel olarak kişisel benlikler; Üçüncü Kaynak ve Merkez’den doğmakta olup, Yüce içinde sınırlı nitelikteki zaman\hyp{}mekân İlahiyat bileşimine erişmektedir. Yaratılmış Yaratan’ın iradesine kendisini verdiği zaman, kişiliğini üstün olanın altında yol olacak şekilde vermemekte veya teslim etmemektedir; sınırlı Tanrı’nın gerçekleşiminin bireysel kişilik katılımcıları, bu şekilde faaliyet göstererek özgür iradesel benliklerini yitirmemektedirler. Bunun yerine, bu tür kişilikler ilerleyen bir biçimde, bu büyük İlahiyat serüvenine katılarak derinleşmektedirler; kutsallık ile olan bu türden birliktelik vasıtasıyla insan, yüceliğin tam da eşiğine, sahip olduğu evrimsel benliği yüceltmekte, zenginleştirmekte, ruhsallaştırmakta ve bütünleştirmektedir.
\vs p117 5:3 Maddi aklın ve Düzenleyici’nin ortak yaratımı olarak insanın evrimleşen ölümsüz ruhu; Cennet’e bu şekilde yükselir, ve onu takiben, Kesinliğin Birlikleri’ne alındıklarında\bibemph{, kesinlik unsur aşkınlaşımı} olarak bilinen deneyimin bir yöntemi vasıtasıyla Ebedi Evlat’ın ruhaniyet\hyp{}çekim döngüsüyle yeni bir biçimde bütünleşir hale gelir. Bu türden kesinlik unsurları böylelikle, Yüce olan Tanrı’nın kişilikleri tarafından deneyimsel tanınma için kabul edilir adaylar haline gelir. Ve, Kesinliğin Birlikleri’nin açığa çıkarılmamış gelecek görevlendirmeleri içinde bu fani usları ruhaniyet mevcudiyetin yedinci aşamasına eriştiğinde, bu türden çifte akıllar üç katmanlı birlik haline gelir. İnsan ve mutlak olarak bu iki uyumlaşmış akıl, bu aşamada gerçekleşimini tamamlamış Yüce Varlık’ın deneyimsel aklı ile bütünlük içinde yüceltilmiş hale gelir.
\vs p117 5:4 Kâinatın Yaratıcısı geçmişte İsa’nın yaşamında benzer bir biçimde açığa çıkarılmışken, Ebedi gelecek içerisinde Yüce olan Tanrı; yükseliş insanına ait olan ölümsüz ruh biçimindeki ruhsallaştırılmış akılda --- yaratıcı olarak dışa vurulmuş ve ruhsal olarak sergilenmiş biçimde --- gerçekleşimini tamamlayacaktır.
\vs p117 5:5 İnsan Yüce ile bütünleşmemekte, ve kendi kişisel kimliğini onun başat kimliği altında yitirmemektedir; ancak, insanların tümünün deneyiminin sonucu olan evren etkileri böylece, Yüce’nin kutsal deneyimleyiminin bir parçasını oluşturmaktadır. “Eylem bizlerin, sonuçlar Tanrı’nındır.”
\vs p117 5:6 İlerleyen kişilik, evrenlerin yükselen aşamaları boyunca ilerlerken gerçekleşmiş gerçekliğin bir izini bırakır. İster onlar akıl olsun, ister ruhaniyet veya enerji olsun, zaman ve mekânın büyüyen yaratılmışları, nüfuz alanları boyunca kişiliğin ilerleyişi tarafından değişikliğe uğramaktadır. İnsan etkilerde bulunduğu zaman, Yüce tepkide bulunmaktadır; ve, bu etkileşim, ilerlemenin gerçekliğini oluşturmaktadır.
\vs p117 5:7 Enerji, akıl ve ruhaniyetin büyük döngüleri, hiçbir zaman, yükseliş kişiliğinin kalıcı iyelikleri değildir; bu hizmetler sonsuza kadar, Yücelik’in bir parçası olarak kalmaya devam eder. Fani deneyim içinde insan usu; emir\hyp{}yardımcı akıl\hyp{}ruhaniyetlerinin ritmik atışları içinde ikamet etmekte olup, bu hizmet içindeki döngüleştirme tarafından yaratılmış alan içinde kararlarını gerçekleştirir. Fani ölüm üzerine insan benliği, sonsuza kadar, emir\hyp{}yardımcı döngüden ayrılır. Bu emir\hyp{}yardımcılar hiçbir zaman, deneyimi bir kişilikten diğerine aktaran görünüşte bulunmamaktadırlar; onlar, Yedi Katmanlı Tanrı aracılığıyla Yüce olan Tanrı’ya karar\hyp{}eyleminin kişilik\hyp{}dışı etkinlerini aktarabilmekte olup, bunu hali hazırda yerine getirmektedirler. (En azından bu, ibadet ve bilgeliğin emir\hyp{}yardımcıları için gerçeklik taşımaktadır.)
\vs p117 5:8 Ve, ruhaniyet döngüleri ile olan durum şöyledir: İnsan bunları, evrenler boyunca olan yükselişi içinde kullanır; ancak, o hiçbir zaman, kendi ebedi kişiliğin bir parçası olarak onlara sahip değildir. Ancak, ister Gerçekliğin Ruhaniyeti, ister Kutsal Ruhaniyet veya aşkın\hyp{}evren ruhaniyet mevcudiyetlikleri olsun, ruhsal hizmetin bu döngüleri, yükseliş kişiliği içinde ortaya çıkan değerleri algılar ve onlara karşılık verir niteliktedir; ve, bu değerler, Yedi Katmanlı aracılığıyla Yüce’ye aslına uygun bir biçimde aktarılır.
\vs p117 5:9 Her ne kadar Kutsal Ruhaniyet ve Gerçekliğin Ruhaniyeti olarak bu türden ruhsal etkiler yerel evren hizmetleri olsa da, onların yönlendirişi bütünüyle; belirli bir yerel yaratımın coğrafi sınırlarıyla sınırlanmış nitelikte değildir. Yükseliş halindeki fani kendi kökensel yerel evreninin sınırları ötesine geçerken; kendisine hiç durmadan öğretimde bulunan, ve, yükselişin her buhranında hatasız bir biçimde Cennet kurtuluş yolcusuna sürekli olarak “Bu yoldan” diyerek yol gösteren bir biçimde, morontial dünyaların felsefi labirentleri boyunca kendisine rehberlik eden Gerçekliğin Ruhaniyeti’nin hizmetinden bütünüyle mahrum değildir. Yerel evrenin nüfuz alanlarından ayrıldığınızda, ortaya çıkan Yüce Varlık’ın ruhaniyetinin hizmeti aracılığıyla ve aşkın\hyp{}evren yansımasının emirleri aracılığıyla sizler hala, Tanrı’nın Cennet bahşedilme Evlatları’nın huzur verici yaratıcı ruhaniyeti tarafından Cennet yükselişiniz içinde yönlendirileceksiniz.
\vs p117 5:10 Kâinatsal hizmetin bu çok katmanlı döngüleri nasıl olur da, Yüce içinde anlamlar ve değerlere ek olarak evrimsel deneyimin gerçeklerini kaydetmektedir? Bizler kesin bir biçimde emin değiliz; ancak, bizler bu kaydedişin, zaman ve mekânın bu döngülerinin doğrudan bahşedicileri olan Cennet kökenine ait Yüce Yaratanlar’ın kişileri vasıtasıyla gerçekleştiğine inanmaktayız. Usun fiziksel düzeyine olan hizmetleri içinde yedi emir\hyp{}yardımcı akıl\hyp{}ruhaniyetinin akıl\hyp{}deneyim derlemeleri, Kutsal Hizmetkâr’ın yerel evren deneyiminin bir parçasıdır; ve, bu Yaratıcı Ruhaniyet aracılığıyla onlar muhtemel bir biçimde, Yücelik’in aklı içinde kayıt olanağını bulmaktadır. Benzer bir biçimde Gerçekliğin Ruhaniyeti ve Kutsal Ruhaniyet ile olan fani deneyimleri muhtemelen Yücelik’in kişisi içinde, benzer işleyiş biçimleri aracılığıyla kaydedilir.
\vs p117 5:11 İnsan ve Düzenleyici’nin deneyimi bile, Yüce olan Tanrı’nın kutsallığı içinde yankısını bulmak zorundadır; zira, Düzenleyiciler deneyimlerken, onlar Yüce gibi olup, fani insanın evrimleşen ruhu Yüce içindeki bu deneyim için mevcudiyet\hyp{}öncesi olasılıktan yaratılmıştır.
\vs p117 5:12 Bu şekilde, yaratımın tümünün çok katmanlı deneyimleri, Yüce’nin evriminin bir parçası haline gelir. Yaratılmışlar yalnızca, Yaratıcı’ya yükselirlerken, sınırlı olanın niteliklerini ve niceliklerini kullanırlar; bu türden kullanımın kişilik\hyp{}dışı sonuçları sonsuza kadar, Yüce kişisi olan yaşayan kâinatın bir parçası olarak kalmaya devam eder.
\vs p117 5:13 İnsanın bir kişilik iyeliği olarak kendisiyle birlikte götürdüğü şeyler, kendi Cennet yükselişi içinde asli evrenin akıl ve ruhaniyet döngülerini kullanımının karakter sonuçlarıdır. İnsan karar verdiğinde, ve bu kararını eylemle sonuçlandırdığında, o deneyimlemekte, ve, bu deneyimin anlamları ve değerleri, sınırlı olandan nihai olana uzanan biçimde düzeylerin tümünde kendi ebedi karakterinin bir parçası olmaktadır. Kâinatsal olarak ahlaki ve kutsal olarak ruhsal karakter; içten ibadet tarafından aydınlatılmış, us sahibi derin sevgi tarafından yüceltilmiş ve kardeşsel hizmetle tamamlanmış kişisel kararlardan oluşan yaratılmışın sermaye birikimini temsil etmektedir.
\vs p117 5:14 Evrimleşen Yüce, nihai olarak sınırlı yaratılmışları; kâinatın âlemleri ile olan sınırlı deneyimden fazlasına hiçbir zaman erişemeyecekleri yetkinsizlikleri için telafi edecektir. Yaratılmışlar, Cennet Yaratıcısı’na erişebilir; Ancak, sınırlı nitelikte bulunan onların evrimsel akılları, sınırsız ve mutlak Yaratıcıyı gerçek anlamda anlamaya yetkin değillerdir. Ancak, tüm yaratılmış deneyimleyişi Yüce içinde kaydolduğu, ve onun bir parçası olduğu, için; yaratılmışların tümü sınırlı mevcudiyetin nihai seviyesine eriştiğinde, ve, bütüncül evren gelişimi onların Tanrı’ya olan erişimini mümkün kıldıktan sonra, mevcut bir kutsallık mevcudiyeti olarak Yüce, bunun sonrasında, bu tür iletişim gerçekliğinde içkin olan bir şekilde, bütüncül deneyim ile iletişim halinde bulunur. Zamana ait sınırlı olan, kendisi içinde ebediyetin tohumlarını taşımaktadır; ve, bizlere, evrimin bütünlüğü, kâinatsal büyüme için yetkinliğin tamamiyle kullanımına şahit olduğunda, bütün sınırlılığın, Nihai olan Yaratıcı’nın arayışı içinde ebedi sürecin absonit fazlarına olan serüvenine başlayacağı öğretilmiştir.
\usection{6.\bibnobreakspace Yüce’nin Arayışı}
\vs p117 6:1 Bizler Yüce’yi evrenler içinde aramaktayız, ancak onu bulamamaktayız. “O bizler içinde, ve, hareketli ve hareketsiz her nesneden ve varlıktan yoksun olarak bütünlüğüne sahiptir. Gizemi içinde tanınamaz, uzak olsa da, o yine de yakındır.” Her\hyp{}Şeye\hyp{}Gücü\hyp{}Yeten Yüce “henüz biçim verilmemiş olanın biçimi, yaratılmamış olanın şablonudur.” Yüce sizlerin evren evidir; ve, siz onu bulduğunuzda, o, eve geri dönmek gibi olacaktır. O, sizin deneyimsel ebeveyninizdir; ve, tıpkı insan varlıklarının deneyimde olduğu gibi, o aynı şekilde, kutsal ebeveynliğin deneyiminde büyümektedir. O sizi, hem yaratılmış\hyp{}gibi hem de yaratan\hyp{}gibi olduğu için bilmektedir.
\vs p117 6:2 Eğer siz gerçekten Tanrı’yı bulma arzusu duyarsanız, Yüce’nin bilincinin akıllarınız içinde doğumuna engel olamazsınız. Tanrı sizin kutsal Yaratıcınız iken, benzer biçimde Yüce sizlerin; kendisinde evren yaratılmışları olarak yaşamlarınız boyunca beslendiğiniz, kutsal Anneniz’dir. “Yüce ne de evrenseldir --- o her taraftadır. Yaratımın sonsuz şeyleri, yaşam için onun mevcudiyetine bağlı olup, hiç kimse reddedilmemektedir.”
\vs p117 6:3 Mikâil Nebadon için ne ise, sınırlı kâinat için Yüce odur; onun İlahiyatı, Yaratıcı’nın derin sevgisinin aracılılığıyla tüm yaratıma doğru dışa doğru aktığı muazzam kanaldır, ve, o, sınırlı yaratılmışların, derin sevgi olan Yaratıcı’ya dair arayışları içinde içe doğru ilerlediği muazzam kanaldır. Düşünce Düzenleyicileri bile onunla ilişkilidir; kökensel doğası ve kutsallığı bakımından onlar Yaratıcı gibidir; ancak, mekânın evrenleri içinde zamanın etkileşimlerini deneyimlediklerinde onlar Yüce gibi olurlar.
\vs p117 6:4 Yaratılmışın Yaratan’ın iradesini gerçekleştirmeyi tercih ediş eylemi; bir kâinat değeri olup, muhtemelen, Yüce Varlık’ın sürekli genişleyen eyleminin faaliyeti olarak, eş güdümün açığa çıkarılmamış ancak sürekli karşılaşılan bir kuvveti tarafından doğrudan bir biçimde karşılık verilir nitelikte bir evren anlamına sahiptir.
\vs p117 6:5 Evrimleşen bir faninin morontia ruhu gerçekten de; Kâinatın Yaratıcısı’nın Düzenleyici eyleminin evladı ve Kâinatsal Anne olarak Yüce Varlık’ın kâinatsal tepkisinin çocuğudur. Anne etkisi, büyüyen ruhun yerel evren çocukluğu boyunca insan kişiliği üzerinde baskın konumdadır. İlahiyat etkisi, Düzenleyici bütünleşiminden sonra ve aşkın\hyp{}evren süreci boyunca daha eşit hale gelmektedir; ancak, zamanın yaratılmışları ebediyetin merkezi evrenine doğru hareketine başladıklarında, Yaratıcı doğası, Kâinatın Yaratıcısı’nın tanınması ve Kesinliğin Birlikleri’ne olan katılım üzerine sınırlı dışavurumun doruk noktasına erişen bir biçimde, artan bir biçimde dışa vurulan hale gelir.
\vs p117 6:6 Kesinlik unsur düzeyine erişimin deneyiminde ve onun vasıtasıyla, yükselen benliğin deneyimsel anne nitelikleri, çok büyük bir biçimde; Ebedi Evlat’ın ruhaniyet mevcudiyeti ve Sınırsız Ruhaniyet’in akıl mevcudiyetinin iletişimi ve katılımı tarafından etkilenir. Bunun sonrasında, asli evren içinde kesinlik unsur etkinliği boyunca; deneyimsel anlamların yeni bir gerçekleşimi ve yükseliş sürecinin bütünlüğüne ait deneyimsel değerlerin yeni bir bileşimi olarak, Yüce’nin gizli anne potansiyelinin bir uyanışı ortaya çıkmaktadır. Benliğin bu gerçekleşiminin; Yücelik’in anne kalıtımı, Yaratıcı’nın Düzenleyici kalıtımı ile sınırlı nitelikteki eş zamanlılığa erişene kadar, altıncı\hyp{}düzey kesinlik unsurlarının evren süreçlerinde devam etmeyi sürdürecek oluşu gözlenmektedir. Asli evren faaliyetinin bu ilgi çekici dönemi, yükseliş nitelikli ve kusursuzlaştırılmış faninin devam eden ergenlik sürecini temsil etmektedir.
\vs p117 6:7 Mevcudiyetin altıncı aşamasının tamamlanışı ve ruhaniyet düzeyinin yedinci ve son düzeyine giriliş üzerine, muhtemelen; zenginleşen deneyimin, olgunlaşan bilgeliğin ve kutsallık gerçekleşimin gelişen çağları ortaya çıkacaktır. Kesinlik unsurunun doğası içinde bu, muhtemelen; sınırlı olasılıklarının sınırları içinde, yükseliş insan\hyp{}doğasının kutsal Düzenleyici\hyp{}doğası ile olan eş\hyp{}güdümünün tamamlanışı olarak, ruhaniyet öz gerçekleşimi için akıl mücadelesini tamamlanmış erişimine denk düşecektir. Bu türden muazzam bir evren benliği böylece; Cennet Yaratıcısı’nın ebedi kesinlik unsur evladına ek olarak, evrenlerin hem Babası hem de Annesi’ni, ve, onlara ilaveten, yaratılmış, yaratılmakta olan veya evrimleşen nesnelerin ve varlıkların sınırlı idaresi ile ilgili herhangi eylem veya sorumluluk içinde bulunan kişilikleri temsil etmek için yetkin olan bir evren benliği niteliğindeki, Anne Yüce’nin ebedi evren evladı haline gelir.
\vs p117 6:8 Tüm ruh\hyp{}evrilir nitelikteki insanlar, kelimenin tam anlamıyla, Baba olan Tanrı ile Yüce Varlık olarak Anne olan Tanrı’nın evrimsel evlatlarıdır. Ancak, bu türden bir ana kadar; fani insan kutsal mirasının ruh\hyp{}bilincine varırken, İlahiyat kan bağının bu teminatı inançla gerçekleştirilmelidir. İnsan yaşam deneyimi; içinde, Yüce Varlık’ın evren bahşedilmişlikleri ve Kâinatın Yaratıcısı’nın evren mevcudiyetinin (ki onların hiçbiri kişilikler değillerdir), zamanın morontia ruhununkine ek olarak evren nihai sonuna ve ebedi hizmete ait insan\hyp{}kutsal kesinlik unsur karakterini evrimleştirdiği, kâinatsal kozadır.
\vs p117 6:9 İnsanlar çoğu zaman, Tanrı’nın insan mevcudiyeti içindeki en büyük deneyim olduğunu unutmaktadırlar. Diğer deneyimler sahip oldukları doğaları ve içerikleri bakımından sınırlıdırlar; ancak, Tanrı’nın deneyimi, yaratılmışın kavrama yetisi dışındakiler haricinde hiçbir sınıra sahip değildir, ve, bahse konu tam da bu deneyim, kendi içinde yeti genişleten niteliktedir. Onlar Tanrı’yı bulduklarında, onlar her şeyi bulmuş olurlar. Tanrı’yı arama; bahşedilecek olan yeni ve daha büyük derin sevginin muazzam keşiflerinin beraberinde gerçekleştiği, derin sevginin sınırsız bahşedilişidir.
\vs p117 6:10 Gerçek sevginin tümü Tanrı’dan gelmektedir; ve, insan kutsal şefkati, kendisi bizzat bu derin sevgiyi akranlarına bahşederken almaktadır. Derin sevgi devinimseldir. O, hiçbir zaman teslim alınamaz; o canlı, özgür, heyecan verici ve her zaman hareket eder niteliktedir. İnsan hiçbir zaman; Yaratıcı’nın derin sevgisini alıp, kalbi içinde hapsedemez. Yaratıcı’nın sevgisi ancak, insan kişiliğinin karşılıksal olarak bu derin sevgiyi akranlarına bahşettiği deneyimle fani insanlar için gerçek hale gelir. Derin sevginin büyük döngüsü; Yaratıcıdan gelmekte, evlatları aracılığıyla onların kardeşlerine, ve böylece Yüce’ye ulaşmaktadır. Yaratıcı’nın derin sevgisi, ikamet eden Düzenleyici’nin hizmeti vasıtasıyla insan kişiliği içinde ortaya çıkmaktadır. Bu türden bir Tanrı\hyp{}bilen evlat, bu sevgiyi, kendi evren kardeşleri için açığa çıkarmaktadır; ve, bu birliktelikçi kardeşsel şefkat, Yüce’nin derin sevgisinin özüdür.
\vs p117 6:11 Deneyim dışında Yüce’ye hiçbir yaklaşım bulunmamaktadır; ve, yaratımın mevcut çağlarında, yaratılmışın Yücelik’e olan yalnızca üç yolu bulunmaktadır:
\vs p117 6:12 1.\bibnobreakspace Cennet Vatandaşları; Cennet\hyp{}Havona gerçeklik farklılaşmasının gözlemlenişi aracılığıyla ve Üstün Ruhaniyetler’den Yaratan Evlatlar’a uzanan biçimde Yüce Yaratan Kişilikleri’nin çok katmanlı etkinliklerinin arama sonucu gerçekleşen keşfi vasıtasıyla Yücelik için yetkinliği elde ettikleri yer olan Havona boyunca, ebedi Ada’dan alçalmaktadırlar.
\vs p117 6:13 2.\bibnobreakspace Yüce Yaratanlar’ın evrimsel evrenlerinden gelen zaman\hyp{}mekân yükseliş unsurları; Cennet Kutsal Üçlemesi’nin birliğinin artan takdiri için bir hazırlık niteliğinde olarak, Havona’nın kat edilişinde Yüce’ye olan yakınlaşmayı gerçekleştirirler.
\vs p117 6:14 3.\bibnobreakspace Havona yerlileri, Cennet’den gelmekte olan alçalış kutsal yolcuları ve yedi aşkın\hyp{}evrenden gelen yükseliş yolcuları ile olan iletişimler vasıtasıyla Yüce’nin bir kavrayışını elde eder. Havona yerlileri içkin bir biçimde, ebedi Ada’nın vatandaşları ve evrimsel evrenlerin vatandaşlarının kökeni itibariyle farklı olan bakış açılarını uyumlaştırma konumunda bulunmaktadır.
\vs p117 6:15 Evrimsel yaratılmışlar için, Kâinatın Yaratıcısı’na olan yedi büyük yaklaşım bulunmaktadır; ve, bu Cennet yükselişlerinden her biri, Yedi Üstün Ruhaniyet’den her birinin kutsallığından geçmektedir; ve, bu türden her bir yaklaşım, Üstün Ruhaniyet’in doğasının aşkın\hyp{}evren yansıması içinde yaratılmışın hizmet vermiş oluşu üzerine deneyim algısının bir genişlemesi ile mümkün kılınmaktadır. Bu yedi deneyimin toplam bütünlüğü, Yüce olan Tanrı’nın gerçekliğine ve mevcudiyetine dair bir yaratılmışın bilincinin mevcut\hyp{}an\hyp{}içerisindeki\hyp{}bilenen sınırlarını oluşturmaktadır.
\vs p117 6:16 Sınırlı Tanrı’yı bulmada kendisine engel olan yalnızca insanın sınırlılıkları değildir; o aynı zamanda, evrenin tamamlanmamış olan konumudur; tüm yaratılmışların --- geçmiş, şimdiki zaman ve gelecekteki biçiminde --- tamamlanmamışlığı bile, Yüce’yi ulaşılamayan konumda kılmaktadır. Yaratıcı olan Tanrı, Tanrı\hyp{}gibi\hyp{}olmanın kutsal aşamasına erişmiş olan herhangi bir yaratılmış tarafından bulunabilir; ancak, Yüce olan Tanrı hiçbir zaman, kusursuzluğun kâinatsal erişimi aracılığıyla, \bibemph{tüm} yaratılmışların eş zamanlı olarak kendisini bulacağı zaman olan çok uzak bir ana kadar \bibemph{bir tek} yaratılmış tarafından bile kişisel olarak keşfedilemeyecektir.
\vs p117 6:17 Yaratıcı’yı, Evladı ve Ruhaniyet’i bulabilme yetisine sahipken ve bunu hâlihazırda gelecekte yerine getirecekken, bu evren çağı içerisinde kişisel olarak Yüce’yi bulamayacak olmanız gerçekliğine rağmen, yine de, Cennet yükselişi ve onu takip eden evren süreci kademeli olarak bilinciniz içerisinde, deneyimin tümünün Tanrısı’na ait evren mevcudiyetinin ve kâinatsal eylemin tanınışını yaratacaktır.
\vs p117 6:18 İnsanın Yüce’ye olan gelecek bir zaman zarfındaki erişimi, Cennet İlahiyatı’nın ruhaniyeti ile olan bütünleşmesi üzerine gerçekleşmektedir. Urantia unsurlarında bu, Kâinatın Yaratıcısı’nın Düzenleyici mevcudiyetidir; ve, her ne kadar Gizem Görüntüleyicisi Yaratıcı’dan gelmekte olup Yaratıcı gibi ise de, bizler, bu gibi kutsal bir armağanın bile, sınırlı bir yaratılmış için sınırsız Tanrı’nın doğasını açığa çıkarma gibi imkânsız göreve erişebileceğinden kuşku duymaktayız. Bizler, Düzenleyici’nin geleceğin yedinci aşama kesinlik unsurları için açığa çıkaracağı şeyin Yüce olan Tanrı’nın kutsallığı ve doğası olacağını düşünmekteyiz. Ve, bu açığa çıkarılış sınırlı bir yaratılmış için, Sınırsız’ın mutlak bir varlık için açığa çıkarılışı gibi olacaktır.
\vs p117 6:19 Yüce, sınırsız değildir; ancak, o muhtemelen, sınırlı bir yaratılmışın gerçek anlamıyla kavrayabileceği her şey olan sonsuzluğun tümünü bünyesinde barındırmaktadır. Yüce’den daha fazlasını anlamak için, sınırlıdan daha fazlası olmak gerekir.
\vs p117 6:20 Tüm deneyimsel yaratımlar, nihai sonsal gerçekleşimleri içinde birbirlerine bağımlılardır. Yalnızca varoluşsal gerçeklik, bağımsız ve kendi başına mevcuttur. Havona ve yedi aşkın\hyp{}evren, sınırlı erişimin en yüksek düzeyine ulaşmak için birbirlerine ihtiyaç duymaktadır; benzer biçimde onlar ileride herhangi bir zaman zarfı içinde, sınırlı olanın aşkınlaşımı için dış uzaydaki gelecek evrenlere bağımlı olacaklardır.
\vs p117 6:21 Bir insan yükseliş unsuru, Yaratıcı’yı bulabilir; Tanrı varoluşsal olup, bu nedenle, evrenin bütününün deneyim düzeyinden bağımsız olarak gerçektir. Ancak, hiçbir tekil yükseliş unsuru; yükseliş unsurlarının tümü, keşfe katılmaları için kendilerini eş zamanlı olarak yetkin kılacak en yüksek evren olgunluğuna ulaşıncaya kadar, Yüce’yi hiçbir zaman bulamayacaktır.
\vs p117 6:22 Yaratıcı, kişilileri ayıran nitelikte değildir; o, kendisine ait yükseliş evlatlarının her birini kâinatsal bireyler olarak değerlendirmektedir. Yüce benzer bir biçimde, kişileri ayıran nitelikte değildir; o, kendisine ait deneyimsel çocukları, tekil bir kâinatsal bütünlük olarak değerlendirmektedir.
\vs p117 6:23 İnsan, Yaratıcı’yı kalbinde keşfedebilir; ancak, o, tüm diğer insanların kalplerinde Yüce’yi aramak zorunda olacaktır; ve, tüm yaratılmışlar kusursuz bir biçimde, Yüce’nin sevgisini açığa çıkardıklarında, bunun sonucunda o, tüm yaratılmışlar için bir evren mevcudiyeti haline gelecektir. Ve, bu yalnızca, evrenlerin ışık ve yaşam altında kusursuz hale geleceklerinin başka bir şekildeki ifadesidir.
\vs p117 6:24 Tüm kişilikler tarafından evrenler boyunca kusursuzlaştırılmış dengeye erişeme ek olarak kusursuzlaştırılmış öz gelişime olan erişim; Yüce’nin erişimine denk düşmekte olup, tamamlanmamış mevcudiyetin sınırlarından tüm sınırlı gerçekliğin özgürleşimine şahit olacaktır. Tüm sınırlı potansiyelliklerin bu türden bir bütüncül yerine getirilişi; Yüce’nin tamamlanmış erişimini açığa çıkaracak olup, başka şekilde, Yüce Varlık’ın kendisi olarak tamamlanmış evrimsel gerçekleşimi biçiminde tanımlanabilir.
\vs p117 6:25 İnsanlar Yüce’yi aniden ve görülmemiş bir biçimde, bir depremin fayları kayalara ayırması gibi bulmamaktadır; ancak, onlar Yüce’yi yavaşça ve sabırla, bir nehrin sessizce altındaki toprağı ayırması gibi bulmaktadır.
\vs p117 6:26 Sizler Yaratıcı’yı bulduğunuzda, evrenlerdeki ruhsal yükselişin muazzam nedenini bulacaksınız; Yüce’yi bulduğunuzda, Cennet ilerleyişi içindeki sürecinizin muhteşem sonucunu keşfedeceksiniz.
\vs p117 6:27 Ancak, hiçbir Tanrı\hyp{}bilen fani hiçbir zaman, kâinat boyunca gerçekleştirdiği serüven içinde yalnız kalamaz; zira o, ilerlediği yoldaki attığı her adımında Yaratıcı’nın yanı başında yürüdüğünü, ve bunun karşısında, tam da bu yolun kendisinin, Yüce’nin mevcudiyetinin doğrultusunun kat edilişi olduğunu bilmektedir.
\usection{7.\bibnobreakspace Yüce’nin Geleceği}
\vs p117 7:1 Tüm sınırlı potansiyelliklerin tamamlanmış gerçekleşiminin, tüm evrimsel deneyimin gerçekleşiminin tamamlanışına denk düşmektedir. Bu Yüce’nin, her\hyp{}şeye\hyp{}gücü\hyp{}yeten bir İlahiyat mevcudiyeti niteliğinde evrenler içindeki nihai olarak ortaya çıkışı anlamına gelmektedir. Bizler, Yüce’nin, gelişimin bu aşamasında; Ebedi Evlat’ın olduğu gibi farklı bir biçimde kişilikleşeceğine, Cennet Adası’nın olduğu gibi somut bir biçimde güç kazandırılacağına ve Bütünleştirici Bünye’nin olduğu gibi tamamiyle bir bütün haline getirileceğine, ve, bunların hepsinin mevcut evren çağının sonuçlanışında Yüce’nin sınırlı olasılıklarına ait sınırlılıkların içinde gerçekleşeceğine inanmaktayız.
\vs p117 7:2 Her ne kadar bu, Yüce’nin geleceğine dair tamamiyle yerinde bir kavramsallaşma olsa da, bu kavramsallaşma içerisindeki içkin belirli sorunlara dikkat çekmek isteriz:
\vs p117 7:3 1.\bibnobreakspace Yüce’nin Koşulsuz Yüksek Denetimcileri, onun tamamlanmış evriminden önceki hiçbir aşamada ilahlaştırılamazlar; ama yine de, bu aynı yüksek denetimciler mevcut an içerisinde bile sınırlı biçimde, ışık ve yaşam altında istikrara kavuşturulmuş evrenler ile ilgili yüceliğin egemenliğini uygulamaktadırlar.
\vs p117 7:4 2.\bibnobreakspace Yüce, evren düzeyinin bütüncül mevcudiyetine erişmeden neredeyse hiçbir biçimde Nihai Kutsal Üçleme içinde faaliyette bulunamaz; ama yine de, Nihai Kutsal Üçleme bu aşama da bile sınırlı bir gerçekliktir, ve, sizler, Nihainin Sınırlı Vekilleri’nin mevcudiyetleri hakkında bilgilendirilmiş konumda bulunmaktasınız.
\vs p117 7:5 3.\bibnobreakspace Yüce, evren yaratılmışları için tamamiyle gerçek değildir; ancak orada, Cennet üzerindeki Kâinatın Yaratıcısı’ndan yerel evrenlerin Yaratan Evlaları’na ve Yaratıcı Ruhaniyetleri’ne kadar uzanan biçimde, Yedi Katmanlı İlahiyat için onun oldukça gerçek olduğunun çıkarımında bulunmaya sebep birçok neden mevcuttur.
\vs p117 7:6 Belki de, zamanın aşkın zaman ile bütünleştiği yer olan sınırlı olanın üst sınırları üzerinde, zamansal gelişimin net olmayan ve bileşik nitelikteki bir çeşit gerçekliğinin mevcut oluşu söz konusudur. Belki de, Yüce’nin; bu zaman\hyp{}ötesi düzeyler üzerinde kendi evren mevcudiyetini öngörmeye yetkin olabildiği, bunun sonrasında, sınırlı bir düzeye kadar, bu geleceksel öngörüyü Tasarlanmış Tamamlanmamışlığın Enginliği olarak yaratılmış düzeylerle beraber düşünerek gelecek evrimi tahmin edebildiği söz konusudur. Bu türden olgular, sınırlı olan; insanın tüm ebediyet boyunca sahip olacağı gelecek evren erişimlerine dair dikkate değer tahminleri olan Düşünce Düzenleyicileri’nin ikamet ettiği insan varlıklarının deneyimlerinde olduğu gibi, ne zaman sınırlı\hyp{}ötesiyle ilişki kurarsa gözlemlenebilir.
\vs p117 7:7 Fani yükseliş unsurları Cennet’in kesinlik unsur birliklerine kabul edildiklerinde, Cennet Kutsal Üçlemesi için bir ant içerler; ve, bağlılığın bu yemini ederlerken onlar bunun aracılığıyla, tüm sınırlı yaratılmışlar tarafından kavranıldığı biçimiyle Kutsal Üçleme’nin \bibemph{tam da kendisi} olan Yüce olan Tanrı’ya gerçekleştirilen ebedi sadakatin vaadinde bulunurlar. Bunun sonrasında, eşlik eden kesinlik unsurları evrimleşen evrenler boyunca faaliyette bulunurlarken onlar yalnızca, yerel evrenlerin ışık ve yaşam altında istikrara kavuşturulacağı çok önemli dönemlere gelene kadar, Cennet kökenden gelen unsurların emirlerine tabidirler. Ve, bu kusursuzlaştırılmış yaratımların yeni hükümetsel örgütleri Yüce’nin ortaya çıkan egemenliğinin yansıması olmaya başlarken, bizler, eşlik edenlerin dışında kalan kesinlik unsurların bu türden yeni hükümetlerin yönetim yetkisini tanıdıklarını gözlemlemekteyiz. Yüce olan Tanrı’nın evrimsel Kesinlik Birlikleri’nin bütünleştiricisi olarak evrimleştiği görünmektedir; ancak, bu yedi birliğin nihai sonunun, Nihai Kutsal Üçleme’nin bir üyesi olarak Yüce tarafından yönetileceği kuvvetle muhtemeldir.
\vs p117 7:8 Yüce Varlık, evren dışavurumu için üç sınırlı\hyp{}ötesi olasılığı bünyesinde barındırmaktadır:
\vs p117 7:9 1.\bibnobreakspace İlk deneyimsel Kutsal Üçleme ile absonit işbirliği.
\vs p117 7:10 2.\bibnobreakspace İkinci deneyimsel Kutsal Üçleme ile ortak\hyp{}mutlak ilişkisi.
\vs p117 7:11 3.\bibnobreakspace Kutsal Üçlemelerin Kutsal Üçlemesi’ne olan katılım; ancak, bizler, bunun gerçekte ne olduğuna dair yeterli bir kavramsallaşmaya sahip değiliz.
\vs p117 7:12 Bu, Yüce’nin geleceğine dair genel olarak kabul edilmiş varsayımlarından bir tanesidir; ancak, orada aynı zamanda, ışık ve yaşam düzeyine olan erişimi sonrasında mevcut asli evreniyle sahip olduğu ilişki ile ilgili birçok tahmin ortaya atılmaktadır.
\vs p117 7:13 Aşkın\hyp{}evrenlerin mevcut amacı; nitelikleri olduğu ve potansiyelliklerinde mevcut olduğu biçimiyle, Havona’nın gibi kusursuz hale gelmektir. Bu kusursuzluk; fiziksel ve ruhsal erişimle, hatta idari, hükümetsel ve kardeşsel gelişime kadar uzanan bir biçimde, ilgilidir. Gelecek çağlar içerisinde uyumsuzluğa, kötü uyuma ve yanlış uyuma dair olasılıkların hepsinin denenmiş olacağına inanılmaktadır. Enerji döngüleri, kusursuz denge içerisinde ve akla bütüncül taabiyette olacak iken; kişiliğin mevcudiyeti içerisinde ruhaniyet, akıl üzerindeki baskınlığı elde edecektir.
\vs p117 7:14 Bu çok uzak gelecekte Yüce’nin ruhaniyet bireyinin ve Her\hyp{}Şeye\hyp{}Gücü\hyp{}Yeten’in erişilmiş gücünün; eş güdümsel gelişime erişmiş olacağınınkine ek olarak, Yüce Akıl içinde ve onun tarafından bütünleşmiş bir bütünlükte her ikisinin de, tüm yaratılmış enerjilere karşılık veren, tüm ruhsal unsurlar içinde eş güdümsel hale gelmiş ve tüm evren kişilikleri tarafından deneyimlenmiş nitelikteki tüm yaratılmış usları tarafından gözlenebilen bir mevcudiyet olarak --- evrenler içindeki tamamlanmış bir mevcudiyet halinde Yüce Varlık olarak gerçekleşeceği düşünülmektedir.
\vs p117 7:15 Bu kavramsallaşma, asli evren içindeki Yüce’nin mevcut egemenliği anlamına gelmektedir. Mevcut Kutsal Üçleme yöneticilerin kendisine ait vekiller olarak görevlerine devam edeceği bütünüyle olasıdır; ancak, bizler, yedi aşkın\hyp{}evren arasındaki mevcut ayrımların kademeli olarak ortadan kalkacağına ve asli evrenin bütününün kusursuzlaştırılmış bir bütünlük olarak faaliyet göstereceğine inanmaktayız.
\vs p117 7:16 Yüce’nin bunun sonrasında, zaman yaratılmışlarının idaresini yöneteceği konum olan Orvonton’un yönetim merkezi halindeki Uversa’da kişisel olarak ikamet edebileceği mümkündür; ancak, bu yalnızca, bir varsayımsal düşüncedir. Buna rağmen mutlak olarak, Yüce Varlık’ın kişiliği; her ne kadar kendi İlahiyi mevcudiyetinin her yerde eş zamanlı olarak mevcut oluşu kâinat âlemlerin tümünü kaplamaya muhtemel bir biçimde devam edecek olsa da, belirli özel bir konumdan kesin bir biçimde ulaşılacaktır. Bu çağın aşkın\hyp{}evren vatandaşlarının Yüce ile olan ilişkisinin ne olacağını bilmemekteyiz; ancak, o, Havona yerlileri ve Cennet Kutsal Üçlemesi arasındaki mevcut ilişkiye benzer bir şey olabilir.
\vs p117 7:17 Bu gelecek günlerin kusursuzlaştırılmış asli evreni, mevcut andakinden çok daha farklı olacaktır. Mekânın gökadalarının örgütlenişi içindeki heyecan verici serüvenlere ve zamanın kimliksiz dünyaları üzerinde yaşam tohumlarının atılışına ek olarak, potansiyellerden doğan güzellik, anlamlardan doğan gerçeklik ve değerlerden doğan iyilik biçiminde, kargaşadan doğan uyumun evrimleşimi yok olacaktır. Zaman evrenleri, sınırlı nihai sonun tamamlanışını elde edecektir! Ve, muhtemelen bir mekân için, evrimsel kusursuzluğun çağlar süren mücadelesinden dinleme biçiminde rahatlama dönemi ortaya çıkacaktır. Ancak, çok daha uzun bir süreliğine değil! Kesin bir biçimde, mutlak olarak, ortaya çıkmaktaki Nihai olan Mutlak’ın İlahiyatı’nın gizemli bilmecesi; tıpkı, mücadele vermiş evrimsel atalarının bir zamanlar yüce olan Tanrı’nın arayışında zorlandıkları gibi, istikrara kavuşturulmuş evrenlerin bu kusursuzlaştırılmış vatandaşlarını zorlayacaktır. Kâinatsal nihai sonun perdesi, yaratılmış deneyiminin nihayeti içinde yeni ve daha yüksek seviyeler üzerinde Kâinatın Yaratıcısı’nın erişimi için çekici absonit arayışının aşkın ihtişamını açığa çıkarmak için tekrar aralanacaktır.
\vs p117 7:18 Bu anlatım, Urantia üzerinde geçici olarak ikamet eden bir Kudretli İletici tarafından sağlanmıştır.]
