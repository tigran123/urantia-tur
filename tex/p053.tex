\upaper{53}{Lucifer İsyanı}
\vs p053 0:1 Lucifer, Nebadon’un muhteşem bir birinci derece Lanonandek Evladı’ydı. Kendisi, bu topluluğun yüksek bir danışmanı olarak; birçok sistem içinde hizmeti deneyimlemiş olup, bilgelik, sağduyu ve etkinlik nitelikleri ile seçkin bir konumda diğer unsurlardan ayrışmıştır. Lucifer; kendi düzeyinin 37. unsuru olup, Melçizedekler tarafından görevlendirildiğinde türünün yedi yüz bin unsurundan daha fazlası arasından en yetkin ve muhteşem yüz unsurundan biri olarak görevlendirilmiştir. Bu türden muazzam bir başlangıçtan, kötülük ve hatalar zinciri boyunca; günah ile bütünleşmiş, ve --- kâinatsal ilişkilere olan gaflet olarak evren birlikteliğinin reddedilmesi ve kardeşsel görevlerin hiçe sayılması biçiminde --- birey arzusuna yenik düşmüş ve kendisini sahte kişisel bağımsızlığın gerçeği yansıtmayan bilgeliğine teslim etmiş Nebadon içindeki üç Sistem Egemeni’nden biri olarak mevcut zaman içinde sınıflandırılmaktadır.
\vs p053 0:2 Hazreti Mikâil’in nüfuzu alanı olarak Nebadon evreni içinde, yerleşik dünyaların on bin sistemi mevcut bulunmaktadır. Lanonandek Evlatları’nın tarihinin bütünü içinde, sistemlerin bu binlercesi ve evren yönetim merkezleri boyunca görevlerinin tümü içerisinde, şimdiye kadar yalnızca üç Sistem Egemeni’nin; Yaratan Evlat’ın hükümetinin alaşağı edilmesi tutumu içinde bulunduğu gözlenmiştir.
\usection{1.\bibnobreakspace İsyan’ın Baş Yöneticileri}
\vs p053 1:1 Lucifer, bir yükseliş varlığı değildi; kendisi, yerel evrenin yaratılmış bir Evladı’ydı, ve onun hakkında şu ifadeler sarf edilmiştir: “Sen, yaratıldığın günden doğruluğa karşı gelinişin sende bulunmasına kadar her bakımdan kusursuz bir varlık halindeydin.” Birçok defa kendisi, Edentia’nın En Yüksek Unsurları ile birlikte danışma halindeydi. Ve Luficer; Jerusem’in idari tepesi olan “Tanrı’nın kutsal dağı üzerinde” bütüncül yönetime sahipti, çünkü o, 607 yerleşik dünyanın muhteşem bir sistemine ait baş yöneticiydi.
\vs p053 1:2 Lucifer, muhteşem bir kişilik olarak muazzam bir varlıktı; kendisi, evren yönetiminin doğrudan birimi içinde takımyıldızlarının En Yüksek Yaratıcıları’nın yanında konumlanmıştı. Lucifer’in yönetimi kötüye kullanışına rağmen alt bağımlı düzey usları; Urantia üzerinde Mikâil’in bahşedilişinin öncesinde, ona saygı göstermeyi sonlandırmadan ve onun yönetimini reddetmeden kendilerini sakınmışlardır. Mikâil’in baş meleği bile, Musalar’ın yeniden dirilişi zamanında, “kendisini suçlayıcı hiçbir ifadede bulunmamış ve sadece ‘Adalet senin hakkında gereken yargıyı verecektir’” demiştir. Bu türden hususlar bakımından yargı, aşkın evrenin idarecileri olan Zamanın Ataları’na aittir.
\vs p053 1:3 Lucifer mevcut an içerisinde, Satania’nın yönetimden düşürülmüş ve görevden alınmış Egemeni’dir. Bireyin hiçbir değeri umursamaz bencil düşüncesi, göksel dünyanın yüceltilmiş kişilikleri için bile en zarar verici olanıdır. Lucifer hakkında “Senin kalbin güzelliğin nedeniyle yüceltilmiştir; sen, bilgeliğini muhteşemliğin sebebiyle kötüye kullandın” ifadesi kullanılmıştır. Sizin eski peygamberiniz, kendisinin üzüntü verici durumunu şu sözleri dile getirdiği zaman görmüştü: “Sabahın evladı sen Lucifer, nasıl olurda cennetten düştün! Dünyaları karıştırmaya kalkışan sen nasıl da alaşağı edildin!”
\vs p053 1:4 Gezegeniniz üzerinde gayesini savunması için kendisinin ilk emir\hyp{}eri olan Satan’ı görevlendirmesi sebebiyle Urantia üzerinde Lucifer hakkında çok az şey duyulmuştur. Satan, Lanonandekler’in aynı birinci derece topluluğunun bir üyesiydi; ancak o, hiçbir zaman Sistem Egemeni olarak faaliyet göstermemiştir; kendisi ilk defa bütünüyle, yalnızca Lucifer isyanına katılmıştır. “Şeytan” biçiminde adlandırılan unsur, Urantia’nın görevden alınmış Gezegensel Prensi ve Lanonandekler’in ikinci derece düzeyine ait bir Evlat olarak, Caligastia’nın kendisinden başkası değildir. Mikâil’in Urantia üzerinde beden içinde bulunduğu zaman zarfında; Lucifer, Satan ve Caligastia, Mikâil’in bahşedilme görevinin başarısızlığını gerçekleştirmek için birlik olmuşlardır.
\vs p053 1:5 Abaddon, Caligastia’ya ait yönetim sorumlularının başıydı. Kendisi; başkaldırı için üstünü takip etmiş olup, bahse konu bu zaman zarfından itibaren Urantia isyankârlarının baş yöneticileri olarak hareket etmişti. Beelzebub, ihanet halindeki Caligastia’nın kuvvetleri ile bir olmuş sadakatsiz yarı\hyp{}ölümlü yaratılmışların önderiydi.
\vs p053 1:6 Ejderha nihai olarak, bu kötü kişiliklerin simgesel temsili haline gelmişti. Mikâil’in zaferi üzerine, “Cebrail, Salvington’dan aşağıya inmiş ve ejderhayı (isyan önderlerinin tümünü) bir çağ boyunca zincirlemiştir.” Jerusem yüksek melek isyankârları hakkında şu ifadeler kayıt altına alınmıştır: “Ve kendilerine verilen ilk yerleşkeyi muhafaza etmeyip kendilerine ait yaşam alanlarını terk edenleri, Cebrail adaletin büyük günü için karanlığın sağlam zincirleri içinde bekletmiştir.”
\usection{2.\bibnobreakspace İsyan’ın Nedenleri}
\vs p053 2:1 Lucifer ve onun ilk yardımcısı Satan; kalplerinde Kâinatın Yaratıcısı ve onun sonrasında gelen kendisine ait vekil Evladı Mikâil’e karşı koydukları zaman zarfında, beş yüz bin yıldan fazla Jerusem üzerinde yönetim sahibi olmuş bulunmaktaydılar.
\vs p053 2:2 Satania’nın sistemi içinde isyana sebep olabilecek veya onu haklı çıkarabilecek hiçbir belirli veya özel koşul bulunmamaktaydı. Bu türden bir fikrin kökenini ve biçimini Lucifer’in aklından aldığı ve bu türden bir isyanın fitilini konumlandığı mekândan bağımsız olarak onun her halükarda ateşleyeceği bizim inancımızdır. Lucifer tasarımlarını ilk olarak Satan’a duyurmuştur, ancak kendisine ait yetkin ve muhteşem birlikteliğinin aklını çelmek aylarını almıştır. Ancak, isyan kuramlarına bir kez yöneldikten sonra o, “kendiliğinden doğruyu addetmenin ve özgürlüğün” gözü pek ve azimli bir savunucusu haline gelmiştir.
\vs p053 2:3 Lucifer’e hiç kimse isyanı hiçbir zaman önermemiştir. Mikâil’in iradesi ve Kâinatın Yaratıcısı’nın tasarımları karşısında kendiliğinden doğruyu addetme fikrinin kökeni, Mikâil tarafından da aktarıldığı biçimde, Lucifer’in aklından çıkmıştır. Onun Yaratan Evlat ile ilişkileri bu zaman zarfı öncesinde içten ve her zaman dostane olmuştur. Aklının baştan çıkması öncesinde hiçbir zaman Lucifer açık bir biçimde evren idaresi ile ilgili memnuniyetsizliği ifade etmemiştir. Sessizliğine rağmen, ortak zamanın yüz yılından fazla bir süre boyunca Zamanın Birlikteliği, Lucifer’in aklında her şeyin huzur içerisinde bulunmadığını Uversa’ya bildirmekteydi. Bu bilgi aynı zamanda Yaratan Evlat ve Norlatiadek’in Takımyıldız Yaratıcıları’na iletilmiştir.
\vs p053 2:4 Bu süreç boyunca Lucifer; evren idaresinin bütüncül tasarımı hakkında artan bir biçimde eleştirel hale gelmiş olup, içten olmayan nitelikteki bütüncül sadakati her zaman Yüce İdareciler’e göstermiştir. Onun ilk dışa vurulan sadakatsizliği, Lucifer’in Özgürlük Bildirgesi’nin açık bir biçimde ilan edilmesinden yalnızca birkaç gün önce, Jerusem’e Cebrail’in bir ziyaret etkinliğinde ortaya çıkmıştır. Cebrail; ortaya çıkması beklenen başkaldırının kesinliği karşısında oldukça derinden bir biçimde etkinlenmiştir ki, doğrudan bir isyan durumunda uygulanacak olan önlemler ile ilgili Takımyıldız Yaratıcıları’na danışmak için doğrudan Edentia’ya gitmiştir.
\vs p053 2:5 Lucifer isyanıyla nihai olarak sonuçlanan kesin nedeni veya nedenleri ortaya koymak oldukça zordur. Biz yalnızca tek bir şeyden eminiz, ve bu ise: Bu olayların ilk başlangıçsal nedenleri her ne ise, onların kökenleri Lucifer’in aklından kaynaklanmıştır. Orada; bireyin nihai olarak kendisini aldatışı noktasına kadar gelişen bir bireysel gururun varlığı muhtemel bir biçimde mevcut olmuştur ki, bunun sonucunda sahip olduğu isyan fikrinin evren için olmasa bile gerçekte sistemin yararına olduğuna dair kendisini bir zaman süreci için gerçekten ikna etmiştir. Tasarımları hayal kırıklığına uğradığı zaman aralığında, kökensel ve fesatlık yaratan gururu o kadar oldukça aşarı bir noktaya ulaşmıştır ki kuşkusuz bu gurur onun durmasına izin vermemiştir. Bu deneyim içerisinde bir noktadan sonra kendisi içtenliğini yitirmiş, ve ortaya çıkan kötülük kasti ve irade dâhilinde yapılan bir günaha evirilmiştir. Bunun böyle gerçekleştiği, bu muhteşem yöneticinin daha sonraki tutumu tarafından doğrulanmıştır. Kendisine, pişmanlığın olanağı uzun bir süre önerilmiştir; ancak alt bağımlı unsurlarının şimdiye kadar yalnızca birkaçı sunulan bağışlamayı kabul etmiştir. Edentia’ya ait Zamanın İnançlıları unsuru, Takımyıldız Yaratıcıları’nın talebi üzerine, bahse konu bu pervazsız isyankârların kurtarılması için Mikâil’in tasarımını onlara bizzat sunmuştur; ancak Yaratan Evlat’ın bağışlaması her zaman, giderek artan hor görme ve küçümser nefret ile reddedilmiştir.
\usection{3.\bibnobreakspace Lucifer Bildirisi}
\vs p053 3:1 Lucifer ve Saten’in kalplerinde huzursuzluğun ilk kökenleri her ne idiyse, nihai isyan Lucifer’in Özgürlük Bildirgesi ile şekillenmiştir. İsyankârların başkaldırı nedeni, şu üç başlık altında ifade edilmiştir:
\vs p053 3:2 1.\bibemph{ Kâinatın Yaratıcısı’nın gerçekliği}. Lucifer; Kâinatın Yaratıcısı’nı gerçekte mevcut olmamakla suçlayıp, fiziksel çekim ve mekân\hyp{}enerjisinin evren içinde içkin bir halde mevcut bulunduğunu, ve Yaratıcı’nın Cennet Evlatları tarafından Yaratıcı adına onların evrenlerin idaresini sürdürmelerine olanak sağlaması için yaratılmış bir efsane olduğunu iddia etmiştir. O, kişiliğin Kâinatın Yaratıcısı’nın bir armağanı olduğunu reddetmiştir. O; Cennet üzerinde kavranabilen bir konumda bulunan Yaratıcı’nın mevcut kişiliğinin oldukça doğrudan bir fikrine bilincin Cennet Evlatları tarafından hiçbir zaman bütünleştirilmemesi nedeniyle, kesinlik unsurlarının Cennet Evlatları ile birlikte tüm yaratım üzerinde aldatmacalarını gerçekleştirmek için işbirliğinde bulunduğunu bile öne sürmüştür. O, derin saygıyı cehalet ile sömürmüştür. Suçlama kökten bir biçimde genelleyici, korkunç ve tüm değerleri haince reddeden bir nitelikte bulunmaktaydı. Kesinlik unsurları üzerindeki bu örtülü saldırı, yükseliş vatandaşlarının daha sonra Jerusem üzerinde isyankârların tümünün savlarına karşı koyma bakımından dik durmalarını ve kararlı duruşlarını sergilemeye devam etmelerini kuşkusuz etkilemiştir.
\vs p053 3:3 2.\bibnobreakspace \bibemph{Yaratan Evlat Mikâil’in evren hükümeti}. Lucifer, yerel sistemlerin özerk olması gerektiğini öne sürmüştür. O, Yaratan Evlat olarak Mikâil’in varsayımsal bir Cennet Yaratıcısı adına Nebadon’un egemenliğini üstlenmesini ve bu görünmeyen Yaratıcı ile olan birlikteliğin tüm kişilikler tarafından tanınmasını koşul olarak öne sürmesine karşı gelmiştir. O, ibadetin bütüncül tasarımının Cennet Evlatları’nın düzeysel konumunun yüceltilmesi amacıyla kurgulanmış zekice bir düzen olduğunu belirtmiştir. O; Mikâil’i kendisinin Yaratan\hyp{}babası olarak tanımaya istekliydi, ancak Mikâil’i kendisinin Tanrı’sı ve yasal yöneticisi biçiminde kabul etmek istemiyordu.
\vs p053 3:4 O daha sert bir biçimde; Zamanın Ataları’nın, kendi deyimiyle “yabancı diktaların”, yerel sistemler ve evrenlerin olaylarına müdahale etme yetkisine daha sert bir biçimde saldırıda bulunmuştur. Bu yöneticileri o, zalim ve var olan hakları gasp eden idareciler olarak kötülemiştir. O; insanlar ve meleklerin kendi düşüncelerini sadece doğrudan ifade etme cesaretine sahip olmaları ve kesin bir biçimde haklarını savunmaları koşulunda, bu yöneticilerin hiçbirinin bütüncül ev idaresinin işleyişine en ufak bir ölçüde bile karışmalarının söz konusu olmayacağı savıyla takipçilerini bu idarecilere karşı koymaları için yüreklendirmiştir.
\vs p053 3:5 O; özgün varlıklar kendilerinin sahip özgürlüğü sadece kararlı bir biçimde ifade etmeleri durumunda, Zamanın Ataları’na ait infaz görevlilerinin yerel sistemler içindeki faaliyetlerinden alıkonulabileceğini öne sürmüştür. O; ölümsüzlüğün sistem kişilikleri içinde içkin olduğunu, yeniden dirilişin doğal ve kendiliğinden gerçekleştiğini ve varlıkların tümünün Zamanın Ataları’nın keyfi ve haksız faaliyetleri olmadan ebedi bir biçimde yaşayacaklarını savunmuştur.
\vs p053 3:6 3.\bibnobreakspace \bibemph{Yükseliş fani eğitiminin evren tasarımı üzerine yapılan saldırı}. Lucifer; etik olmamakla ve sağlıksız bir nitelikte bulunmakla suçladığı, evren idaresinin ilkeleri bakımından yükseliş fanilerinin oldukça bütüncül bir biçimde eğitilmesine dair düzen üzerinde oldukça gereğinden fazla zaman ve enerjinin sarf edildiğini savunmuştur. O; herhangi bir bilinmeyen gelecek için mekânın fanilerinin hazırlanışı amacıyla çağlar süren işleyişsel düzene karşı çıkmış, kesinlik birliklerinin su katılmamış hayal ürünü niteliğinde bulunan herhangi bir nihai sona hazırlanmalarını kanıt olarak göstererek Jerusem üzerinde onların mevcudiyetini örnek göstermiştir. O; kesinlik unsurlarının gereğinden fazla denetim ve haddinden fazla uzun eğitimle baştan çıkarıldıklarını, ve bu aşamada yükseliş fanileri için efsanesel bir ebedi nihai sonun safsataları için tüm yaratımın köleleştirilme düzeni ile onların işbirliği içinde bulunması nedeniyle fani akranlarına gerçekte ihanette bulunduklarını bildirmiştir. O, yükseliş unsurlarının bireysel özerkliklerinin özgürlüğünü memnuniyetle deneyimlemelerini savunmuştur. O, Tanrı’nın Cennet Evlatları tarafından desteklenen ve Sınırsız Ruhaniyet tarafından teşvik edilen fani yükselişin bütüncül tasarımına karşı çıkmış ve onu kınamıştır.
\vs p053 3:7 Ve Karanlık ve ölümün bu toplu katılımını Lucifer bu türden bir Özgürlük Bildirgesi ile harekete geçirmiştir.
\usection{4.\bibnobreakspace İsyan’ın Ortaya Çıkışı}
\vs p053 4:1 Lucifer bildirisi; Urantia zamanına göre yaklaşık iki yüz bin yıl önce, bu zaman sürecinin son gününde, Jerusem’in toplanmış ev sahiplerinin mevcudiyetinde, cam denizinin üstünde Satania’nın yıllık oturumunda sunulmuştur. Satan; ibadetin --- fiziksel, ussal ve ruhsal olarak --- evrensel kuvvetler uyarınca gerçekleştirilebileceğini, ancak bu bağlığın yalnızca “insanlar ve meleklerin arkadaşı” ve “özgürlüğün Tanrı’sı” olan mevcut ve hâlihazırdaki yönetici Lucifer’e sunulabileceğini duyurmuştur.
\vs p053 4:2 Kendiliğinden doğruyu addetme, Lucifer isyanının savaş söylemiydi. Onun başlıca savlarından biri; eğer özerk yönetim Melçizedek ve diğer topluluklar için doğru ve adil ise, usun diğer düzeyleri için bu düzenin eşit bir biçimde iyi olduğu yönündeydi. O, “aklın eşitliği” ve “usun kardeşliğinin” savunulmasında gözü pek ve ısrarcıydı. O, tüm hükümetin yerel gezegenler ile ve onların yerel sistemlere olan gönüllü yönetim birliği ile sınırlı olması gerektiğini savunmuştur. Bu bağlamda tüm diğer yüksek denetimi kendisi reddetmişti. O, Gezegensel Prensler’e dünyaları yüce yöneticiler olarak idare etmeleri gerektiğinin sözünü vermişti. O, takımyıldız yönetim merkezleri üzerinde yasama etkinliklerinin konumlanmasını ve evren başkenti üzerinde yargısal olayların işleyişini kötülemişti. O; hükümetin üç faaliyetinin tümünün de sistem başkentleri üzerinde toplanması gerektiğini savunmuş, ve bununla kalmayıp kendisine ait yasama meclisini oluşturmuş, buna ek olarak Satan’ın karar yetkisi altında kendisine ait yüksek mahkemeleri düzenlemiştir. Ve o, düzenden ayrılan dünyaların bütün bu faaliyetlerin aynısını gerçekleştirmelerinin emrini vermiştir.
\vs p053 4:3 Lucifer’in bütüncül idari kurulu; tek bir bünyede toplanmış, ve “özgürleştirilmiş dünyalar ve sistemlerin” yeni baş yöneticisine ait idarenin yönetim sorumluları olarak yemin ettirilmişlerdir.
\vs p053 4:4 Nebadon içinde bahse konu zaman zarfından önce iki öncül isyan gerçekleşmişken, onlar farklı takımyıldızlarında ortaya çıkmıştır. Lucifer bu başkaldırıların, usların büyük bir çoğunluğunun önderlerini takip etmede başarısız olmaları nedeniyle amacına ulaşamadığını savunmuştur. O, “çoğunluğun idaresini” ve “aklın hatasız olduğunu” iddia etmiştir. Evren yöneticileri tarafından kendisine verilen özgürlük, onun alçak iddialarının birçoğunun dışavurumuna göründüğü kadarıyla zemin hazırlamıştır. O, üstlerinin hepsine karşı gelmişti; yine de onun üstleri göründüğü kadarıyla, onun faaliyetlerini ciddiye almamıştı. Kendisine izin verilmeden veya kendisine karşı konulmadan, baştan çıkarıcı tasarımını uygulamak için kendisine açık kapı bırakılmıştı.
\vs p053 4:5 Adaletin bağışlayıcı günlerinin tümünü Lucifer, Cennet Evlatları’nın hükümetinin isyanı durdurmadaki yetkinsizliğine dair kanıt olarak göstermiştir. O; açık bir biçimde Mikâil, Immanuel ve Zamanın Ataları’nı reddetmiş ve kibirli bir biçimde onlara karşı gelmiş olup, ve bunun sonrasında kendisine dair hiçbir eylemde bulunulmamasını evren ve aşkın evren hükümetlerinin yetkinsizliğine dair kendisini haklı gösteren kanıt olarak göstermiştir.
\vs p053 4:6 Cebrail; bütün bu sadakatsiz gelişmeler boyunca kişisel olarak mevcut bir konumda bulunmuş olup, ve yalnızca, gerektiği zaman Mikâil için ifadede bulunacağını buna ek olarak tüm varlıkların tercihlerinde özgür ve rahat bırakılacağını bildirmiştir; ve o, “Yaratıcı’nın Evlatlar’ın hükümeti için sadakat ve bağlılığın gönüllü, içten ve doğruluk taşımayan kuramlardan yalıtılmış bir nitelikte bulunması gerektiğini arzuladığının” söyleminde bulunmuştur.
\vs p053 4:7 Cebrail isyan söylemlerinin ilerleyiş hakkına karşı gelmek için veya onu engellemek için herhangi bir çabada bulunmadan önce, Lucifer’in isyan hükümetini bütünüyle kurmasına ve etraflıca bir biçimde onu düzenlemesine izin verilmiştir. Ancak Takımyıldız Yaratıcıları doğrudan bir biçimde, Satania’nın sistemiyle bu sadakatsiz kişiliklerinin faaliyetini sınırlandırmıştır. Yine de bu gecikme dönemi, Satania’nın tümünün sadık varlıkları için büyük bir sınama ve deneme zamanı olmuştur. Her şey birkaç yıl boyunca oldukça düzensiz bir yapıda seyretmiş olup, malikâne dünyaları üzerinde büyük bir kafa karışıklığı mevcut bulunmuştur.
\usection{5.\bibnobreakspace Çatışma’nın Doğası}
\vs p053 5:1 Satania isyanının çıkması üzerine Mikâil, kendisinin Cennet kardeşi olan Immanuel’in tavsiyesini almıştır. Bu anlık toplantıyı takiben Mikâil; geçmişte gerçekleşen benzer başkaldırılara olan tutumunu niteleyen müdahil olmamanın bir tavrı biçiminde, aynı siyasayı uygulayacağını bildirmiştir.
\vs p053 5:2 Bu isyan ve ondan önceki iki başkaldırı zamanında, Nebadon evreni içinde hiçbir mutlak ve kişisel egemen yönetimi bulunmamaktaydı. Mikâil, Kâinatın Yaratıcısı’nın vekili olarak kutsal hak tarafından yönetimini gerçekleşmiştir; ancak bu hak, henüz kendisinin kişisel hakkı niteliğinde bulunmamaktaydı. O, kendisinin bahşedilme sürecini tamamlamış bir konumda bulunmamaktaydı; ona, “cennet ve dünya üzerinde gücün tümü” henüz verilmemişti.
\vs p053 5:3 İsyanın çıkışından, Nebadon’un egemen yöneticisi olarak idari mevkie getirildiği güne kadar Mikâil, Lucifer’in isyan kuvvetlerine hiçbir zaman müdahalede bulunmamıştır; onların, Urantia zamanına göre neredeyse iki yüz bin yıllığına bağımsız bir istikamette seyretmelerine izin verilmiştir. Hazreti Mikâil mevcut an içerisinde, sadakatsizliğin bu türden isyanları ile vakit kaybetmeden hatta acil bir biçimde mücadele etmek için fazlasıyla güce ve yönetim yetkisine sahiptir; ancak biz, eğer bu türden bir başkaldırı tekrar gerçekleştiğinde şu an sahip olduğu bu egemen yönetiminin kendisinin farklı bir biçimde hareket etmesine yol açacağından kuşku duymaktayız.
\vs p053 5:4 Mikâil, Lucifer isyanının mevcut savaş durumu karşısında müdahil olmamayı tercih ettiği için; Cebrail, Edentia üzerinde kendi kişisel yönetim sorumlularını çağırıp, En Yüksek Unsurlar ile danışma içinde Satania’nın sadık ev sahiplerinin yönetimini üstlenmeyi tercih etmiştir. Cebrail Jerusem’e ilerlerken ve kendisini --- Kâinatın Yaratıcısı’nın kişiliğini Lucifer ve Satan’ın kuşkulu bir biçimde sorguladıkları --- aynı Yaratıcı’ya adanan âlem üzerinde sadık kişiliklerin toplanan ev sahiplerinin mevcudiyetinde iktidarını kurarken, Mikâil Salvington’da kalmaya devam etmiştir; Cebrail, beyaz bir alt zemin üzerinde iç içe geçmiş üç gök mavisi daire biçiminde tüm yaratımın Kutsal Üçleme hükümetinin maddi simgesi olan Mikâil’in armasını sergilemiştir.
\vs p053 5:5 Lucifer’in arması, ortasında parlak içi dolu bir siyah dairenin bulunduğu beyaz alt zemin üzerine bir kırmızı daire simgesinden oluşmaktaydı.
\vs p053 5:6 “Cennet’te bir savaş vardı; Mikâil’in kumandanı ve onun melekleri ejderhaya (Lucifer, Satan ve doğru düzenden ayrılan prenslere) karşı savaştı; ve ejderha ve onun isyankâr melekleri mücadele ettiler ancak bu savaştan galip çıkamadılar.” Bahse konu “cennet içindeki savaş”, Urantia üzerinde herhangi bir çatışmadan anlaşılacağı gibi fiziksel bir savaş değildi. Mücadelenin ilk zamanlarında Lucifer, gezegensel amfi\hyp{}tiyatroyu sürekli bir biçimde nutukları için kullandı. Cebrail, elverişli olarak en yakın yerleşkeden oluşturulan kendisine ait yönetim merkezinden isyankâr kurmacalarına sonu gelmez bir biçimde karşı koyma faaliyetine girişti. Âlem üzerinde mevcut olan kuşkulu çeşitli kişiliklerin onaylayıcı tutumları nihai bir karara varana kadar, bu tartışanın tarafları arasında birinden diğerine seyir etti.
\vs p053 5:7 Ancak cennet içindeki bu savaş oldukça korkunç ve fazlasıyla gerçekti. Olgunlaşmamış dünyalar arasında gerçekleşen fiziksel savaşın oldukça temel niteliği olan vahşiliklerin hiçbirini göstermese de bu çatışma, olgunlaşmamış dünyalardaki örneklerine kıyasla çok daha fazla ölümcüldü; maddi yaşam maddi çatışma tarafından tehlike altında bulunmaktadır, ancak cennet içinde gerçekleşen bu savaş ebedi yaşamın koşulları ve kapsamı içinde ve dâhilinde gerçekleşmiştir.
\usection{6.\bibnobreakspace Bir Sadık Yüksek Melek Kumandanı}
\vs p053 6:1 Düşmanlığın ortaya çıkışı ve yeni sistem yöneticisine ilaveten onun yönetim çalışanlarının varışı arasındaki geçiş dönemi boyunca sayısız kişilik tarafından dışa vurulan bağlılık ve sadakatin soylu ve ilham verici eylemleri bulunmaktaydı. Ancak bağlılığın bu cesur kahramanlıklarının en muazzamı, Satania’nın yönetim merkez yüksek meleklerinin başındaki ikinci derece sorumlu unsur olan Manotia’nın cesur faaliyetidir.
\vs p053 6:2 Jerusem üzerinde isyanın başlangıcında yüksek melek ev sahiplerinin başı Lucifer’in amacına katılmıştır. Bu durum kuşkusuz olarak, sistem idari yüksek melekleri olarak dördüncü düzeyin bu türden geniş sayılar halinde doğru düzenden ayrıldıklarını açıklamaktadır. Yüksek melekler önderi, Lucifer’in muhteşem kişiliğinden başka bir şeyi görmeyerek ruhsal anlamda kör olmuştu; onun etkileyici davranış tarzları, göksel varlıkların daha alt düzey unsurlarını büyülemiştir. Onlar yalın bir değişle, bu türden göz kamaştırıcı kişiliğin yanlış yola sapabilmesinin mümkün olduğunu kavrayamadılar.
\vs p053 6:3 Lucifer isyanının başlangıcı ile ilgili olan deneyimleri tanımlanması bakımından Manotia’nın şu ifadeleri kullanması üzerinden uzun zaman geçmemiştir: “Ancak beni derinden neşelendiren anım, ikinci düzey yüksek melek kumandanı olarak Mikâil’in aşağılanması etkinliğine katılmayı reddettiğim zaman zarfında gerçekleşen Lucifer isyanı ile ilgili olan heyecan verici serüvendir; ve güçlü isyankârlar benin yok edilmemi düzenledikleri irtibat birliklerinin aracılığıyla gerçekleştirmeyi amaçladılar. Jerusem üzerinde devasa bir başkaldırı bulunmaktaydı, ancak tek bir sadık yüksek melek bu başkaldırıdan zarar görmedi.
\vs p053 6:4 “Benim doğrudan üst sorumlumun doğru düzenden ayrılması üzerine, sistemin karışıklığa uğramış yüksek melek olaylarının sorumlu yöneticisi olarak Jerusem’in meleksel ev sahiplerinin yönetimini üstlenmek benim sorumluluğuma düşmüştü. Ben; Melçizedekler tarafından ahlaki bir biçimde korunmuş, Maddi Evlatlar’ın büyük bir çoğunluğu tarafından yetkinlikle desteklenmiş, kendi düzenimin devasa bir topluluğu tarafından ise yalnız bırakılmış, ancak Jerusem üzerinde yükseliş fanileri tarafından muhteşem bir biçimde yardım görmüştüm.
\vs p053 6:5 “Lucifer’in bölünme isteğiyle gelişen olaylar sonucunda kendiliğinden Takımyıldız döngülerinden uzaklaştırılınca bizler, Rantulia’nın yakınında bulunan sistemden Edentia için yardım çağrıları yapan us birliklerimizin sadakatine bağımlı bir konumda bulunmaktaydık; ve biz, düzenin hükümranlığının, sadakatin usunun ve gerçekliğin ruhaniyetinin isyan, kendiliğinden doğruyu addetme ve sözde kişisel özgürlüğün üzerinde içkin bir biçimde zafere sahip olduğunu gözlemledik; biz, Lucifer’in değerli varisi olan yeni Sistem Egemeni’nin varışına kadar görevlerimizi gerçekleştirmeye yetkin bir konumda bulunmaktaydık. Ve ben bunun sonrasında doğrudan bir biçimde; Lucifer’in “yanlış yönetilen ve kötü idare edilen Satania’nın dünyalarının bağımsızlığa âşık, özgür düşünen ve geleceği gören uslarına” olan çağrısında, onun tarafından düzenlenen meşhur Özgürlük Bildirgesi içinde önerilen “özgürleştirilmiş ve bağımsızlaştırılmış kişiliklerin” amaçlanan yeni sistemi için kendi âlemini onun bir üyesi olarak ilan eden ihanet halindeki Caligastia’nın dünyası üzerinde, sadık yüksek melek düzeyleri üzerinde yönetim yetkisini alarak, Urantia’nın Melçizedek alıcılığının birliğine görevlendirildim.”
\vs p053 6:6 Bu melek Urantia üzerinde, yüksek meleklerin baş yönetici yardımcısı konumunda faaliyet göstererek hâlihazırda hizmet vermeye devam etmektedir.
\usection{7.\bibnobreakspace İsyan’ın Tarihi}
\vs p053 7:1 Lucifer isyanı sistem genelinde bir başkaldırıydı. Bölünme yanlısı otuz yedi Gezegensel Prens dünya idarelerini büyük ölçüde baş isyankârın tarafına yöneltmişlerdir. Yalnızca Panoptia üzerinde Gezegensel Prens, kendi insanlarını beraberinde taşımada başarısız olmuştur. Bu dünya üzerinde, Melçizedekler’in rehberliğinde, insanlar, Mikâil’i desteklemek için bir araya gelmiştir. Bu fani âlem içindeki Ellanora ismindeki genç bir kadın; insan ırklarının önderliğini kazanmış olup, bu çatışmanın böldüğü dünya üzerinde tek bir ruh bile Lucifer’in arması altında toplanmamıştır. Ve bahse konu zaman zarfından bu yana bu sadık Panoptia unsurları; Yaratıcı’nın âlemi ve onu çevreleyen yedi tecrit dünyası üzerinde sorumlular ve inşacılar olarak yedinci Jerusem geçiş dünyası üzerinde hizmet etmiştir. Panoptia unsurları; sadece bu dünyaların gerçek sorumluları olarak faaliyet göstermediler, onlar aynı zamanda bu âlemlerin gelecekteki bilinmeyen amaç için süslenmesi yönünde Mikâil’in emirlerini yerine getirdiler. Onlar bu görevi, Edentia doğrultusunda beklerken gerçekleştirdiler.
\vs p053 7:2 Bu süreç boyunca Caligastia, Urantia üzerinde Lucifer’in amacını desteklemekteydi. Melçizedekler; yetkin bir biçimde doğru düzenden ayrılmış Gezegensel Prens’e karşı gelmiştir, ancak sınırsız özgürlüğün içeriği boş savları ve kendiliğinden doğruyu addetmenin aldatıcı etkileri, genç ve gelişmemiş bir dünyanın ilkel insanlarını kandırmak için her imkâna sahipti.
\vs p053 7:3 Bölünme söylemlerinin tümü; bireysel çabalar tarafından yürütülmek zorundaydı, çünkü yayın hizmeti ve gezegenler arası iletişimin diğer kanalları sistem döngü yüksek denetimcilerinin faaliyeti tarafından askıya alınmıştı. İsyanın mevcut bir biçimde ortaya çıkışı üzerine Satania’nın bütün sistemi, takımyıldız ve evren döngüleri içinde tecrit altına alınmıştı. Bu zaman zarfı boyunca gelen ve giden iletilerin tümü, yüksek melek sorumluları ve Yalnız İleticiler tarafından yönlendirilmekteydi. Esaret altına alınmış dünyalara olan döngüler aynı zamanda, Lucifer’in alçak düzeninin ilerlemesi için bu ortamı kullanamaması amacıyla iletişimden koparılmıştı. Ve bu döngüler, baş isyankârlar Satania’nın sınırları içinde yaşamaya devam ettikçe eski haline döndürülmeyeceklerdir.
\vs p053 7:4 Ortaya çıkan bu olay, bir Lanonandek isyanıydı. Her ne kadar isyankâr gezegenler üzerinde konumlanan Yaşam Taşıyıcıları’nın birkaçı, bir biçimde sadakatsiz prenslerin isyanı tarafından etkilenmiş olsa da; yerel evren evlatlığının daha yüksek düzeyleri Lucifer bölünmesine katılmamıştır. Kutsal Bir Biçimde Üçleştirilmiş Evlatlar’ın hiçbiri doğru yoldan ayrılmamıştır. Melçizedekler, baş melekler ve Berrak Akşam Yıldızları’nın tümü; Mikâil’e sadık kalmış olup, Cebrail ile birlikte cesur bir biçimde Yaratıcı’nın idaresi ve Evlat’ın yönetimini savunmuşlardır.
\vs p053 7:5 Cennet kökenine ait hiçbir varlık, sadakatsizliğin bir parçası olmamıştır. Yalnız İleticiler ile birlikte onlar, Ruhaniyet’in dünyası üzerinde yönetim merkezleri üzerinden çalışmalarını sürdürmüş, Edentia’ya ait Zamanın İnançlıları’nın önderliği altına kalmaya devam etmişlerdir. Arabulucu unsurların hiçbiri doğru düzenden ayrılmamışlardır; buna ek olarak Göksel Kaydedicilerin bir tane unsuru bile doğru yoldan uzaklaşmamıştır. Ancak doğru yoldan ayrılanların büyük bir kısmı, Morontia Dostları ve Malikâne Dünya Eğitmenleri’nden çıkmıştır.
\vs p053 7:6 Yüksek meleklere ait yüce düzey içinde hiçbir melek kaybedilmemiştir; ancak bu düzeyin üstünde bulunan daha kıdemli unsurların önemli bir topluluğu kandırılmış ve tuzağa düşürülmüştür. Benzer bir biçimde meleklerin üçüncü veya diğer bir değişle yüksek denetim düzeyine ait birkaç unsur aldatılmıştır. Ancak korkunç ayrışma, genel olarak sistem başkentlerinin sorumluluklarına atanan yüksek melekler biçiminde, idari melekler olarak dördüncü toplulukta gerçekleşmiştir. Monotia neredeyse onların üçte ikisini kurtarmıştır; ancak onların üçte birinden biraz daha fazlası, isyan mevzilendirmesi doğrultusunda liderlerini takip etmiştir. Jerusem çocuksu meleklerin tümünün üçte biri, üstlerinde bulunan sadakatsiz yüksek melekleri ile birlikte yitirilen idari meleklere katılmıştır.
\vs p053 7:7 Maddi Evlatlar için görevlendirilen gezegensel melek yardımcıları içinde onların yaklaşık olarak üçte biri kandırılmış, ve neredeyse geçiş hizmetkârlarının onda biri tuzağa düşürülmüştür. Yahya bu durumu simgesel olarak, büyük kırmızı ejderha hakkında ifadede bulunurken görmüştür: “Ve onun kuyruğu cennetin yıldızlarının üçüncü bir kısmını kendisine çekti ve onlardan aşağı karanlık yaydı.”
\vs p053 7:8 En büyük kayıp, meleksel düzen unsurları içinde gerçekleşmiştir; ancak usun daha alt düzeylerinin büyük bir kısmı sadakatsizliğin bir parçası olmuştur. Satania içinde kaybedilen 681.227 tane Maddi Evlat içinde onların yüzde doksan beşi tek başına Lucifer isyanında yitirilmiştir. Yarı\hyp{}ölümlü yaratılmışların büyük miktarda unsuru, Lucifer amacına katılan Gezegensel Prensler’in ait olduğu bu bireysel gezegenler üzerinde yitirilmiştir.
\vs p053 7:9 Birçok açıdan bu isyan, Nebadon içinde gerçekleşen olayların tümü içinde en geniş ve en fazla yıkıma sebep olan gelişmeydi. Diğer iki başkaldırıya kıyasla bu isyana daha fazla kişilik katılmıştır. Ve Lucifer’e ek olarak Satan’ın temsilcilerinin; kesinlik kültürel gezegen üzerindeki çocuk\hyp{}eğitim okullarını onların amacının dışında bırakmaması, dahası bunun yerine evrimsel dünyalardan bağışlama sonrası kurtarılan bu gelişmekte olan akılları baştan çıkarmayı amaçlamaları, onların sonsuza kadar sürecek olan onursuzluklarıdır.
\vs p053 7:10 Yükseliş fanileri savunmasız bir durumda bulunmaktaydı, ancak onlar daha alt düzey ruhaniyetlere kıyasla isyanın için boş savları karşısında daha azimli bir biçimde durdular. Düzenleyicileri ile nihai bütünleşmeye erişmemiş olan, daha alt düzey malikâne dünyalar üzerinde bulunan birçok unsur bu tuzağa düşerken; Jerusem üzerinde ikamet eden Satania yükseliş vatandaşlığına ait tek bir üyenin bile Lucifer isyanına katılmaması yüksek düzeninin bilgeliğine ait yüceliği göstermiştir.
\vs p053 7:11 Saat saat, gün gün Nebadon’un bütününe ait yayın merkezleri, Melçizedek liderliği altında başarılı bir biçimde bölünme ve günahın armaları etrafında oldukça hızlı bir biçimde toplanan bütün kurnaz günah kuvvetlerinin bütünleşen ve uzun bir sürece yayılan çabalarına başarılı bir şekilde karşı koyan yükseliş fanilerinin şaşmaz sadakatini sürekli bir biçimde aktaran raporların sunulmasıyla birlikte neşelenen ve Satania isyanının haberlerini dikkatle inceleyen göksel usun her tahayyül edilebilen sınıfına ait endişeli gözlemciler tarafından yoğun bir biçimde takip edilmiştir.
\vs p053 7:12 “Cennet içindeki savaşın” başlamasından Lucifer’in varisinin görev mevkisine yerleştirilmesine kadar sistem zamanına göre iki yılın üstünde bir zaman zarfı geçmiştir. Fakat en sonunda yeni Egemen, kendi yönetim çalışanları ile birlikte can denizinin üstüne inerek gelmiştir. Ben, Cebrail tarafından Edentia üzerinde görevlendirilen yedek unsurlar arasında bulunmaktaydım; ve Lanaforge’nin Norlatiadek’in Takımyıldız Yaratıcısı’na olan ilk iletisini çok iyi bir biçimde hatırlamaktayım. Bu ileti şöyleydi: “Tek bir Jerusem vatandaşı bile kaybedilmemiştir. Her yükseliş fanisi, ölümcül denemeden kurtuluşa erişmiş ve çok önemli sınavdan zaferle ayrılmış olup, onların hepsi muzaffer olmuştur.” Ve Salvington, Uversa ve Cennet’e doğru, fani yükselişin kurtuluş deneyiminin isyan karşısında en büyük emniyet ve günah karşısında ise en kesin koruyucu olduğuna dair bu güvence iletisi gönderilmiştir. Bu soylu nitelikteki inançlı fanilerin Jerusem birliğinin nüfusu yalnızca 187.432.811’di.
\vs p053 7:13 Her ne kadar baş isyankârların Jerusem etrafına, morontia âlemlerine ve hatta bireysel yerleşik dünyalara özgürce gitmelerine izin verilmiş olsa da; Lanaforge’nin varışı ile birlikte baş isyankârlar mevkilerinden uzaklaştırıldılar ve tüm idari güçlerinden mahrum bırakıldılar. Onlar, insanlar ve meleklerin akıllarını karıştırma ve onları yanlış yönlendirmek için aldatıcı ve baştan çıkarıcı çabalarına devam ettiler. Ancak Jerusem’in idari tepesi üzerindeki onların faaliyetiyle ilgili olarak, “onların yönetim mevkisi artık bulunmamaktadır.”
\vs p053 7:14 Her ne kadar Lucifer, Satania içindeki idari yönetiminin tümünden mahrum bırakılmış olsa da; bunun sonrasında bu ahlaksız isyankârı gözaltına alacak veya onu yok edecek ne herhangi bir yerel evren gücü ne de yüksek mahkeme ortaya çıkmıştır; bu zaman zarfında Mikâil egemen bir yönetici değildi. Zamanın Ataları, Takımyıldız Yaratıcıları’nın sistem hükümetini ele geçirişini desteklemiştir; ancak onlar Lucifer, Satan ve onların birliktelik unsurlarının mevcut durumları veya gelecekte alacakları yaptırımlar ile ilgili hâlihazırda görüşülmeyi bekleyen birçok başvuru hakkında herhangi bir karara hiçbir zaman varmamıştır.
\vs p053 7:15 Böylelikle, ana söylemleri ve kendiliğinden doğru addetme fikirleri için daha ileri düzeyde etkiyi yakalamak amacıyla baş isyankârların sistemin bütünü içinde serbest bir biçimde hareket etmelerine izin verilmiştir. Ancak Urantia zamanına göre yaklaşık olarak iki yüz bin yıldır onlar, diğer bir dünyayı kandırmayı başaramamışlardır. Otuz yedi âlemin ele geçirilmesinden bu yana, isyan gününden beri nüfusu artan genç dünyalar dâhil bile olmak üzere, hiçbir Satania dünyası yitirilmemiştir.
\usection{8.\bibnobreakspace Urantia üzerinde İnsan’ın Evladı}
\vs p053 8:1 Lucifer ve Satan özgür bir biçimde Satania sistemi içinde, Urantia üzerindeki Mikâil’in bahşedilme görevinin tamamlanmasına kadar hareket etmişlerdir. Onlar en son sizin dünyanız üzerinde, İnsan’ın Evladı üzerinde güçlerini birleştirerek yaptıkları saldırı zamanında birlikte bulunmaktaydılar.
\vs p053 8:2 Daha önceden “Tanrı’nın Evlatları” olarak Gezegensel Prensler dönemsel bir biçimde bir araya getirildiklerinde, doğru düzenden ayrılan Gezegensel Prensler’e ait tecrit edilmiş dünyaların tümünü temsil ettiğini bildirerek bu toplanışa “Satan da gelmiştir.” Ancak o, Mikâil’in nihai bahşedilişi dolayısıyla Jerusem üzerinde bu türden bir özgürlüğe sahip olmamaktaydı. Bahşedilme bedeni içinde Mikâil’i baştan çıkarma çalışmalarını takiben, günahın tecrit edilmiş dünyalarının dışı olmak üzere Satania’nın bütünü boyunca Lucifer ve Satan’a olan anlayışın hepsi ortadan kalkmıştır.
\vs p053 8:3 Mikâil’in bahşedilişi, doğru düzenden ayrılmış Gezegensel Prensler’in gezegenleri dışında Satania’nın tümü içinde Lucifer isyanını sonlandırmıştır. Ve bu durum İsa’nın kişisel deneyimini, beden içindeki ölümünden önce takipçilerine bir gün şu ifadeleri aktardığında simgelemekteydi: “Ve ben Satan’ın cennetten yıldırım gibi düştüğüne kesin bir biçimde şahit oldum.” Satan, Lucifer ile birlikte Urantia’ya geride kalan en son hayati mücadele için gelmişti.
\vs p053 8:4 İnsan’ın Evladı; başarılı olacağından emindi, ve dünyanız üzerindeki zaferinin yalnızca Satania içinde değil aynı zamanda günahın girdiği diğer iki sistem içinde de çağlar boyunca süren düşmanlarının düzeyini sonsuza kadar belirleyeceğini bilmekteydi. Hâkim’iniz Lucifer’in savlarına karşılık olarak sakin ve kutsal güvence ile “Benim tarafıma geç, Satan” cevabını verdikten sonra, faniler için kurtuluş ve melekler için emniyet gerçekleşmişti. Bu anlatılan olay temel olarak Lucifer isyanının gerçek sonudur. Uversa yüksek mahkemelerinin Cebrail’in isyankârların yok edilmesine dair başvurusu ile ilgili yönetimsel karara daha varmamış olması gerçektir; ancak bu türden bir hüküm kuşkusuz olarak zamanın bütünlüğü içinde gerçekleşecektir, çünkü bu davanın görüşülmesinde ilk aşamaya hâlihazırda varılmıştır.
\vs p053 8:5 Caligastia ölüm zamanına yakın bir zaman zarfına kadar İnsan’ın Evladı tarafından Urantia’nın resmi Prensi konumunda tanınmıştır. İsa şunları ifade etmiştir: “Mevcut an bu dünyanın yargısıdır; şimdi bu dünyanın prensi yönetimden alınacaktır.” Ve bunun sonrasında Caligastia’nın yaşam görevinin tamamlanmasına daha yakın bir zamanda ise İsa şu gerçeği duyurmuştur: “Bu dünyanın prensi yargılanmıştır.” Bir zaman diliminde “Urantia’nın Tanrı’sı” olarak adlandırılan Prens ile bahse konu bu görevden alınan ve gözden düşürülen Prens aynı kişiliktir.
\vs p053 8:6 Mikâil’in Urantia’dan ayrılmasından önce son eylemi, Caligastia ve Daligastia’ya bağışlamayı teklif etmesi olmuştur; ancak onlar bu içten öneriyi, aşağılayıcı bir biçimde reddetmişlerdir. Sizin doğru düzenden ayrılmış olan Gezegensel Prens’iniz Caligastia, alçak tasarımlarını yerine getirmek için Urantia üzerinde hala özgür bir konumda bulunmaktadır; ancak o, insanların akıllarına girmek için kesin bir biçimde hiçbir güce sahip değildir; buna ek olarak o, insanlar gerçek anlamıyla onun ahlaksız mevcudiyeti ile lanetlenmeyi arzulamadıkça ruhlarını baştan çıkarmak veya bu ruhları doğru düzenden ayırmak için kendi tarafına çekememektedir.
\vs p053 8:7 Mikâil’in bahşedilmesinden önce karanlığın bu yöneticileri; Urantia üzerinde yönetimlerini sağlamayı arzulayıp, küçük çaplı ve alt bağımlı düzeyde bulunan göksel kişilikleri sürekli bir biçimde engellediler. Ancak Hamsin zamanından beri ihanet içindeki Caligastia ve onun eşit derecedeki bayağı yardımcısı Daligastia; Cennet Düşünce Düzenleyicisi’nin kutsal ihtişamı ve beden üzerine bahşedilen Mikâil’in ruhaniyeti olan koruyucu Gerçekliğin Ruhaniyeti’nin önünde köle konumunda bulunmaktadır.
\vs p053 8:8 Fakat bu durumda bile doğru düzenden ayrılmış hiçbir ruhaniyet, Tanrı’nın çocuklarına ait akıllara tecavüz etmede ve onların ruhlarını taciz etmede herhangi bir güce sahip değildir. Ne Satan ne de Caligastia herhangi bir biçimde Tanrı’nın inanç evlatlarına ne dokunabilmiş ne de onlara yaklaşabilmiştir; inanç, günah ve haksızlık karşısında etkin bir zırhtır. “Tanrı’dan doğan kişi kendisini korur ve ahlaksız olan kişi ona dokunamaz” ifadesi gerçektir.
\vs p053 8:9 Genel olarak zayıf ve uçarı faniler; kötüler ve uğursuzların etkisi altında olduğunda, onlar yalnızca kendilerinin doğal meyilleri tarafından yönlendirilen bir biçimde içkin ve bayağı eğilimlerinin baskınlığı altında bulunmaktadır. Şeytanlığa, aslında kendisine ait olmayan kötülüğün büyük bir etkisi kurgusal olarak verilmiştir. Caligastia, İsa’nın haçından bu yana göreceli olarak güçsüz bir konumda bulunmaktadır.
\usection{9.\bibnobreakspace İsyan’ın Mevcut Düzeyi}
\vs p053 9:1 Lucifer isyanının ilk zamanlarında kurtuluş Mikâil tarafından tüm isyankârlara önerilmiştir. İçten pişmanlıklarını kanıt olarak gösterecek her isyankâra, bütüncül evren egemenliğine olan erişimi üzerine, bağışlamayı ve evren hizmetinin bir türü içinde yeniden görevlendirilmeyi önermiştir. Hiçbir isyan baş sorumlusu, bu bağışlayıcı öneriyi kabul etmemiştir. Ancak Maddi Erkek ve Kız Evlatların yüzlercesine ek olarak meleklerin ve göksel unsurların alt düzeylerinin binlercesi; Panoptia unsurları tarafından duyurulan bağışlamayı kabul etmiş olup, kendilerine bin dokuz yüz yıl önce İsa’nın yeniden dirilişi zamanında eski hallerine dönme hizmeti verilmiştir. Bu varlıklar bu zamandan beri; Uversa mahkemelerinin Cebrail \bibemph{karşısında} Lucifer’in davası hususunda bir karar varacağı zaman zarfına kadar, işleyişsel bir biçimde tutulmak zorunda oldukları Yaratıcı’nın Jerusem dünyasına aktarılmışlardır. Ancak yok edilme kararı yürürlüğüne girdiğinde bu pişman olan ve kurtarılmış kişiliklerin yok olma hükmünden muaf tutulacaklarına dair hiçbir kuşku bulunmaktadır. Bu gözaltında bulunan ruhlar mevcut zaman zarfında, Yaratıcı’nın dünyasının bakımı görevinde Panoptia unsurları ile birlikte görev yapmaktadır.
\vs p053 9:2 Mikâil’i bahşedilme görevini tamamlamasına ek olarak nihai ve kesin bir biçimde Nebadon’un koşulsuz idarecisi konumunda gücünü oluşturmasından vazgeçirmeyi amaçladığı zaman zarfından bu yana baş aldatıcı, Urantia üzerine hiçbir zaman uğramamıştır. Mikâil’in Nebadon evreninin istikrara kavuşmuş baş yöneticisi haline gelmesi üzerine Lucifer; Zamanın Uversa Ataları’nın görevlileri tarafından gözaltına alınmış, ve bu zamandan itibaren Yaratıcı’ya ait Jerusem’in geçiş âlemlerinin topluluklarının on birinci uydusunda bir mahkûm olarak ikamet etmektedir. Ve burada diğer dünyalar ve sistemlerin yöneticileri, Satania’nın inançsız Egemeni’nin sonunu dikkatle gözlemektedir. Paul Mikâil’in bahşedilmesini takiben bu isyankâr yöneticilerin durumunu, Caligastia’nın baş yöneticileri hakkında “cennetsel mekânlarda bulunan ahlaksızlığın ruhani ev sahipleri” ifadelerini kullanırken bilmekteydi.
\vs p053 9:3 Mikâil, Nebadon’un yüce egemenliğini üstlenirken; Urantia zamanınıza göre neredeyse iki yüz bin yıl önce Uversa yüce mahkemesinin kayıtlarına girmiş olan, Cebrail \bibemph{karşısında} Lucifer’in davasında aşkın evren yüksek mahkemelerinin bekleyen hükümleri ile ilgili olarak, Lucifer isyanıyla ilişkili tüm kişilikleri göz hapsine almak için yönetim yetkisini Zamanın Ataları’ndan resmi olarak talep etmiştir. Sistem başkent topluluğu ile ilişkili olarak Zamanın Ataları, Mikâil’in resmi talebini kabul etti ancak bir şartla: Diğer bir Tanrı Evladı’nın doğru yoldan çıkmış bu türden dünyalara varışına veya Uversa’nın mahkemelerinin Cebrail \bibemph{karşısında} Lucifer’in davası hakkında yargıya başlama zamanına kadar doğru düzenden ayrılmış olan dünyalar üzerinde ihanet içindeki prenslere Satan’ın dönemsel olarak ziyarette bulunmasına izin verilecekti.
\vs p053 9:4 Satan Urantia’ya gelebilirdi çünkü dünyanız --- ne Gezegensel Prens ne de Maddi Evlat bakımından --- ikamet eden Evlat düzeyinde hiçbir unsura sahip değildi. Maçiventa Melçizedeği, bu zamandan beri Urantia’nın vekil Gezegensel Prens’i olarak duyurulmuştur; ve Cebrail \bibemph{karşısında} Lucifer’in davasının açılması, tecrit edilen dünyalar üzerinde geçici gezegensel düzenlerin başlamasının işareti oluşturdu. Satan’ın; isyan başlarının yok edilmesi için Cebrail’in talebinin ilk duruşmalarının gerçekleştiği zaman olarak, bu açığa çıkarışların sunumuna kadar geçen süre içerisinde, Caligastia ve doğru düzenden ayrılmış diğer prensleri dönemsel olarak ziyaret ettiği doğrudur. Satan Jerusem mahkûmiyet dünyaları üzerinde mevcut an içerisinde koşulsuz bir biçimde gözaltında bulunmaktadır.
\vs p053 9:5 Mikâil’in nihai bahşedilişinden beri, Satania’nın tümü içerisinde hiçbir unsur, göz hapsine alınan isyankârlara hizmet vermek için mahkûmiyet dünyalarına gitmeyi arzulamamıştır. Bin dokuz yüz yıldır bu durum değişmemiştir.
\vs p053 9:6 Biz, Zamanın Ataları’nın baş isyankârların nihai sonlandırılışını gerçekleştirinceye kadar mevcut Satania kısıtlamalarının giderilmesini öngörmemekteyiz. Sistem döngüleri, Lucifer yaşamaya devam ettiği müddetçe yeniden bağlantılı bir konuma getirilmeyecektir. Bu zaman zarfında Lucifer bütünüyle eylemsiz bir konumda bulunmaktadır.
\vs p053 9:7 İsyan Jerusem üzerinde sonlanmıştır. Kutsal Evlatlar’ın doğru düzenden ayrılan dünyalarına varışıyla birlikte isyan derhal sonlanmıştır. Biz, isyankârların tümünün kendilerine sunulan bağışlamayı hiçbir zaman kabul etmeyeceğine inanmaktayız. Biz, bu göz hapsinde bulunan bu isyankârların yok edilmesini gerçekleştirecek Uversa’nın kararının yönetimsel yayın tarafından duyurulacağını öngörmekteyiz. Bunun sonrasında siz onları arayacak fakat onları bulamayacaksınız. “Ve dünyalar arasında siz isyankârları bilen herkes bulunduğunuz durum karşısında hayretler içinde kalacak; siz bir dehşetin tam da kendisi olmuştunuz, ancak siz artık bu dehşeti artık hiçbir zaman yayamayacaksınız.” Ve böylece alçak hainlerin tümü “sanki hiç yaşamamış hale gelecektir.” Tüm bunların hepsi Uversa hükmünü beklemektedir.
\vs p053 9:8 Ancak çağlar boyunca Satania içindeki ruhsal karanlığın yedi mahkûmiyet dünyası; “düzenin dışına çıkanların kaderinin zor olduğuna dair”, “her günah içinde bireyin kendisini yok edişinin tohumunun saklı olduğuna dair”, “günahın bedelinin ölüm” olduğuna dair büyük gerçeği derin ve etkili bir biçimde simgeleyerek Nebadon’un tümü için ciddi bir uyarıyı oluşturmuştur.
\vs p053 9:9 [Urantia’nın alıcılığına bir kereliğine atanmış bulunan Manovandet Melçizedeği tarafından sunulmuştur.]
