\upaper{70}{İnsan Hükümeti’nin Evrimi}
\vs p070 0:1 İnsan, yaşamını idame etme sorununu kısmen çözdüğünde eş zamanlı olarak insan ilişkilerini düzenleme gerekliliği ile karşılaştı. Üretimin gelişmesi hukuku, düzeni ve toplumsal düzenlemeyi gerektirdi; özel mülkiyet, hükümetin oluşmasını gerekli kıldı.
\vs p070 0:2 Evrimsel bir dünya üzerinde karşıtlıkların mevcudiyeti doğal bir durumdur; barış ancak, toplumsal nitelikli düzenleyici sistemin bir türü tarafından güvence altına alınabilir; birliktelik, bir takım denetimci yönetim varlığı anlamına gelmektedir. Hükümet; kabilelerin, kavimlerin, ailelerin ve bireylerin sahip oldukları karşıtlıkların eş\hyp{}güdümsel hale gelişini gücünü kullanarak yerine getirmektedir.
\vs p070 0:3 Hükümet, bilinç dâhilinde hareket etmeyen bir gelişimdir; bu kurum, deneme ve yanılmayla evrim göstermektedir. Hükümet, hayatın idamesinde önemli bir değere sahiptir; bu nedenle geleneksel hale gelmektedir. Düzenin olmadığı yönetimler, çekilen sıkıntıları geçmiş zamanda daha da arttırmıştır; bu nedenle, her ne kadar kusursuz olmasa da kanun ve düzen olarak hükümet idaresi, geçmişte ortaya çıkmış veya şimdiki zaman içerisinde oluşumunu gerçekleştirmektedir. Varoluş için verilen mücadelenin baskıcı talepleri, insan ırkını kelimenin tam anlamıyla medeniyetin ilerleyici doğrultusuna sürüklemiştir.
\usection{1.\bibnobreakspace Savaşın Kökeni}
\vs p070 1:1 Savaş, evrim halindeki insanın doğal durumu ve geçmişten getirdiği mirasıdır; barış, medeniyetin gelişimini ölçen toplumsal mihenk taşıdır. İlerleyen ırkların kısmi toplumlaşmasından önce insanlık; haddinden fazla bireysel, aşırıcı derecede kuşkucu ve inanılmaz ölçekte kavgacıydı. Şiddet doğanın kanunu, düşmanlık ise doğa çocuklarının istemsiz gerçekleşen tepkisi iken; savaş yalnızca, bu etkinliklerin topluca yapıldığı bir olgular bütününden ibarettir. Ve her nerede ve her ne zaman gerçekleşirse gerçekleşsin medeniyet inşası toplumsal ilerleyişin zorlukları ile gerildiği anda, orada her zaman; insanların karşılıklı birliktelikleri içerisinde mevcut olan gerilimlerin şiddet vasıtasıyla uyumlu hale getirilmesine dair bu öncül yöntemlere doğrudan ve yıkıcı bir başvuruş açığa çıkmaktadır.
\vs p070 1:2 Savaş, yanlış anlaşılmalar ve gerginliklere karşı hayvansal bir tepkidir; barış, bu tür sorunlar ve zorlukların tümünün medeni bir biçimde çözülmesiyle açığa çıkar. Daha sonra kötüleşmiş Âdem ve Nod unsurları ile birlikte Sangik ırklarının tümü düşmansı nitelikler göstermekteydi. Andonsal unsurlara öncül bir biçimde altın kural düşüncesi öğretilmişti; ve bugün bile onların Eskimo soyları büyük ölçüde bu kural uyarınca yaşamaktadırlar; adet onlar arasında güçlü bir konumda olup, şiddetli düşmanlıklardan oldukça uzak bir biçimde yaşamaktadırlar.
\vs p070 1:3 Andon çocuklarına, anlaşmazlıkları sonlandırmak için her birinin bir sopa ile bir ağacı onlara kızarak dövmeleri gerektiğini öğretmişti; sopası ilk kırılan bahse konu anlaşmazlığın galibi sayılmaktaydı. Daha sonraki Andonsal unsurlar, tartışmanın taraflarının birbirleri ile şakalaştığı ve alay ettiği kamuya açık bir gösteriyi düzenlemekteydi; kazananı ise seyirciler alkışları ile birlemekteydi.
\vs p070 1:4 Ancak savaş benzeri olgular bütünü; toplumun barış dönemlerini gerçek bir biçimde deneyimlediği ve savaş benzeri uygulamalara izin verdiği ileri aşamaya olan yeterli bir biçimde evrimleştiği vakte kadar ortaya çıkmamıştır. Savaşın kavramsal içeriği, belirli bir düzey toplumsal örgütlenişin var olduğu anlamına gelmektedir.
\vs p070 1:5 Toplumsal gruplaşmaların ortaya çıkması ile birlikte bireysel gerilimler topluluk aidiyeti altında erimeye başladı; ve bu durum, kabileler\hyp{}arası barışın bozulması pahasına kabile içi huzuru sağladı. Barış böylelikle ilk önce, yabancılar olarak topluluk dışında bulunan üyelerden hiçbir zaman hoşlanmayan ve onlardan nefret eden bireyler tarafından topluluk içinde veya diğer bir değişle kabile kapsamında memnuniyetle deneyimlenmişti. Öncül insan, yabancı kanı akıtmayı bir erdem olarak addetmişti.
\vs p070 1:6 Ancak barışın bu türü bile ilk başta sağlanamamıştı. Öncül kabile önderleri anlaşmazlıkları gidermeye çalıştıklarında, en azından yılda bir kere kabilesel taş savaşlarının yapılmasına izin vermeyi gerekli görmektelerdi. Bahse konu kavim iki topluluğa ayrılıp, bütün gün boyunca savaşırdı. Ve bu türden etkinlik eğlence dışında herhangi bir sebepten dolayı gerçekleştirilmemekteydi; onlar savaşmaktan gerçek anlamıyla büyük bir keyif aldılar.
\vs p070 1:7 Savaş, bir hayvandan evirilmiş olarak insanın mevcut yapısı nedeniyle varlığını sürdürmektedir; hayvanların tümü kavgacıdır. Savaşın ilk sebepleri arasında şunlar bulunmaktaydı:
\vs p070 1:8 1.\bibnobreakspace \bibemph{Açlık}, yiyecek yağmalarını hedef alan saldırılara yol açmıştır. Herkese yetecek toprağın bulunmayışı her zaman savaşı beraberinde getirmiş, ve bu tür mücadeleler boyunca öncül barışçıl kabileler neredeyse tamamen yok edilmiştir.
\vs p070 1:9 2.\bibemph{ Kadınların az sayıdaki nüfusu} --- ev yaşamında duyulan emek gücü kıtlığını giderme girişimi. Kadınları çalmak her zaman savaşlara neden olmuştur.
\vs p070 1:10 3.\bibnobreakspace \bibemph{Gösteriş} --- kabile gücünü gösterme arzusu. Üstün topluluklar alt topluluklara kendi yaşam biçimlerini dayatmak için savaşa başvurmaktaydılar.
\vs p070 1:11 4.\bibnobreakspace \bibemph{Köleler} --- emek gücü gerektiren işlerde çalıştırılacak bireylere duyulan ihtiyaç.
\vs p070 1:12 5.\bibnobreakspace \bibemph{İntikam}, komşu bir kabilenin bir kabile üyelerinin ölümüne neden olduğuna inanıldığı durumlarda savaşın temel nedeniydi. Ölümden duyulan yas, alınan bir başın eve getirilmesine kadar devam etmekteydi. İntikam için verilen savaşlar, göreceli modern çağlara kadar varlığını korumaya devam etmiştir.
\vs p070 1:13 6.\bibnobreakspace \bibemph{Boş zaman eğlencesi} olarak savaş, bahse konu öncül zamanların genç insanları tarafından dinlence etkinliği olarak görülmüştü. Bir savaşın çıkması için herhangi bir iyi ve yeterli neden yoksa ve barış baskıcı bir hale gelince, komşu kabileler; yapmacık bir savaştan keyif alan bir biçimde, yarı\hyp{}dostane savaş mücadelesi içinde karşı kabile topraklarına yapılan saldırılara bir tatil etkinliği olarak katılma alışkanlıkları vardı.
\vs p070 1:14 7.\bibnobreakspace \bibemph{Din} --- inanılan dine yeni üyeler kazandırma arzusu. İlkel dinlerin hepsi savaşa izin vermiştir. Sadece yakın zamanlarda din, savaşa kötü gözle bakmaya başlamıştır. Öncül din adamlığı kurumu ne yazık ki genel olarak askeri güç ile birleşmişti. Çağlar arasındaki en büyük barış hareketlerinden biri, devlet ile kiliseyi birbirinden ayırma girişimi olmuştur.
\vs p070 1:15 Bu eski zaman kabileleri her zaman, önderleri veya sağlıkçıların emriyle tanrılarına ibadet edilmesi arzusuyla savaş vermişlerdir. İbraniler “savaşların Tanrı’sı” gibi bir güce inanmışlardır; ve onların Şuayb şehri insanlarına yaptıkları saldırılarına dair anlatı, ilkel kabile savaşlarının acımasız zulmünün örneksel bir hikâyesidir; ilk önce tüm erişkin erkeklerin soykırımına ek olarak erkek çocukların ve bakire olmayan kadınların hepsinin daha sonra öldürülmesi ile beraber bu türden bir saldırı, iki yüz bin yıl öncesinin bir kabile önderinin sahip olduğu ahlak kurallarını taçlandırmak için yapılırdı. Ve tüm bu uygulamaların hepsi, “İsrail’in Koruyucu Tanrı’sı adına” yapılmaktaydı.
\vs p070 1:16 Bu anlatım; ırkların sahip olduğu sorunları doğal bir biçimde çözümleyişi biçiminde, insanın kendi kaderini dünya üzerinde çizmesine dair toplumun evriminin bir hikâyesidir. Her ne kadar insan kendi sorumluğunu tanrıları üzerine yükleme eğilimi gösterse de, bu türden vahşetlere İlahiyat kaynaklık etmemektedir.
\vs p070 1:17 Askeri bağışlama insan varlıklarına ulaşmada yavaş kalmıştır. İbraniler’i yöneten bir kadın olarak Deborah zamanında bile bahse konu zulümler devam etmiştir. Onun Musevi olmayanlar karşısında zaferle çıkan üst rütbeli komutanı bile “sakinlerin hepsinin kılıçtan geçirilmesine, geriye kimse kalmamasına” neden olmuştur.
\vs p070 1:18 Irk tarihinde en ilk zamanlardan başlamak üzere zehirli silahlar kullanılmıştı. Bireyleri kötürüm haline getirmenin tüm biçimleri uygulanmıştı. Talut, kız kardeşi Mikal’in başlık parası için Davut’tan yüz Filistin sünnet derisi istemekten çekinmemiştir.
\vs p070 1:19 İlkel savaşlar, bir bütün olarak kabileler arasında meydana gelmekteydi; ancak daha sonraki zamanlarda farklı kabilelerden olan iki kişi bir anlaşmazlık yaşadığında onların kabilelerinin savaşması yerine bahse konu bu iki birey bir düello gerçekleştirmekteydi. Aynı zamanda, Davud ve Calüt’ün durumunda olduğu gibi her iki taraftan seçilen bir temsilci arasında gerçekleşen mücadelenin soncuna göre iki ordunun galibiyeti veya mağlubiyeti tamamiyle üstlenmesi bir gelenek haline gelmişti.
\vs p070 1:20 Savaşlara getirilen ilk sınırlama esir alma uygulamalarında gerçekleşmiştir. Bunun sonrasında kadınlar düşmanlıklardan muaf tutulmuş ve daha sonra savaşa müdahil olmayanların tanınması gerçekleşmiştir. Askeri rütbe düzenleri ve hazır ordular yakın bir zamanda savaşın artan karmaşık yapısı ile uyumlu hale gelen bir biçimde gelişme göstermiştir. Bu türden kahramanların kadınlar ile birliktelik kurmaları öncül olarak yasaklamıştır; ve kadınlar, her ne kadar askerleri her zaman beslemiş ve onların savaşa katılmalarını istemiş olsalar da, uzun bir süre önce savaşmayı sonlandırmışlardır.
\vs p070 1:21 Savaş ilan etme uygulaması büyük bir gelişimi temsil etmişti. Bu türden savaş amaçlarını ilan etme, bir adalet duygusunun yerleşmiş olduğunu işaret etmiştir; ve bu yeni uygulamanın yerini, “medeni” savaş kanunlarının kademeli gelişimi takip etmiştir. İlk dönemlerde dini yerleşkeler etrafında savaşmamak bir adet haline gelmiştir; ve daha sonra, belirli dini günlerde şiddetli mücadeleler vermemek gelenek halini almıştır. Bu gelişmeleri mülteci hakkının geniş çaplı bir tanınması izlemiştir; siyasi kaçaklar korunma altına alınmıştır.
\vs p070 1:22 Böylelikle savaş kademeli olarak, ilkel insan avından daha sonraki dönemlerin “medeni” uluslarının sahip olduğu bir ölçüde daha adilane düzene evirilmiştir. Ancak arkadaşlığın toplumsal tutumunun düşmanlığın yerini alması çok yavaş bir biçimde gerçekleşmiştir.
\usection{2.\bibnobreakspace Savaşın Toplumsal Değeri}
\vs p070 2:1 Geçmiş çağlarda çetin bir savaş, on bin yıl içerisinde doğal yollardan gerçekleşmeyecek düzeydeki toplumsal değişimleri oluşturur ve yeni düşüncelerin uygulanmasını gerçekleştirirdi. Bu belirli savaş kazanımları karşısında ödenen korkunç bedel, toplumun geçici olarak yabansı düzeyine geri dönmesiydi; medeniyet aklı savaşın bu olumsuz sonuçlarını kabul etmemeliydi. Savaş, bedeli oldukça yüksek ve fazlasıyla tehlikeli olan tesiri güçlü bir ilaçtır; belirli toplumsal kargaşalar sıklıkla tedavi edici sonuçlar doğursa da, zaman zaman toplumu yok eden bir biçimde tedavi için bekleyen hastayı öldürmektedir.
\vs p070 2:2 Ülke savunmasının sürükle açığa çıkan ihtiyacı, birçok yeni ve gelişmiş toplumsal düzenlemeleri yaratmaktadır. Mevcut zaman içerisinde toplumun memnuniyetle deneyimleyerek yararlandığı birçok buluş ilk başta tamamiyle askeri kullanımlar için yaratılmıştı; buna ek olarak onların yaratılması hatta, askeri tatbikatın ilk türlerinden biri olarak dans etmek için savaşılmasından bile kaynağını almıştır.
\vs p070 2:3 Savaş geçmiş medeniyetler için toplumsal bir değere sahiptir, çünkü onlar:
\vs p070 2:4 1.\bibnobreakspace Zorunlu olarak uygulanan eş\hyp{}güdüm biçiminde disiplin sağlamıştır.
\vs p070 2:5 2.\bibnobreakspace Metanet ve cesareti ödüllendirmiştir.
\vs p070 2:6 3.\bibnobreakspace Milliyetçiliği desteklemiş ve onu sağlamlaştırmıştır.
\vs p070 2:7 4.\bibnobreakspace Zayıf ve elverişsiz toplulukları yok etmiştir.
\vs p070 2:8 5.\bibnobreakspace İlkel topluma ait eşitlik yanılsamasını ortadan kaldırmış ve seçici bir biçimde toplumu katmanlaştırmıştır.
\vs p070 2:9 Savaş, belirli bir evrimsel ve seçici değere sahip olmuştur; ancak tıpkı kölelik gibi savaşında, medeniyet yavaşça ilerledikçe belirli bir zaman zarfı içerisinde terk edilmesi gerekmektedir. Eski dönemlerin savaşları seyahat ve kültürel etkileşimi sağlamıştır; bu yararlar mevcut zaman zarfında, ulaşım ve iletişimin çağdaş yöntemleri ile daha iyi bir biçimde yerine getirilmektedir. Eski dönemlerin savaşları milletleri güçlendirmiştir; ancak çağdaş mücadeleler medeni hale gelen kültürü sekteye uğratmaktadır. İlkel savaşlar, alt düzeyde bulunan toplulukların ortadan kaldırılmasıyla sonuçlanmıştır; modern çatışmanın en doğrudan sonucu, en iyi insan ırk kollarının seçici yıkımıdır. Öncül savaşlar, örgütsel düzen ve verimi sağlamıştır; ancak bu amaçlar şimdi çağdaş üretimin temel gayeleri haline gelmiştir. Geçmiş çağlar boyunca savaş, medeniyeti ileri doğru iten toplumsal bir enzimdi; savaşın bahse konu bu sonucu mevcut zaman içinde geleceğe dönük amaçlar ve icatlar ile daha iyi bir biçimde elde edilmektedir. Eski dönemlerin savaşları, şiddetli mücadeleleri içine alan bir Tanrı kavramsallaşmasını desteklemiştir; ancak çağdaş insana Tanrı’nın derin bir sevgi olduğu anlatılmıştır. Savaş geçmişte birçok değerli amacı sağlamış, medeniyet inşasında hayati derecede öneme sahip bir iskele vazifesi görmüştür; ancak savaş, onun patlak vermesi ile birlikte açığa çıkan hazin kayıpları herhangi bir biçimde orantılı olarak telafi edebilecek toplumsal kazanımları paylaştıran bir üretimden yoksun bir biçimde, kültürel bakımdan çöküntü haline gelmektedir.
\vs p070 2:10 Bir zamanlar doktorlar, kanın dökülmesinin birçok hastalığa karşı deva olduğuna inandılar; ancak bu inanışın hüküm sürdüğü zamandan beri onlar, sağlık bozukluklarının birçoğu için daha iyi tedavileri keşfetmişlerdir. Ve bu nedenle, savaşın yarattığı uluslararası ölçekteki kan dökümü yerini milletlerin hastalıklarının tedavisi için daha iyi yöntemlerin keşfine bırakmak zorundadır.
\vs p070 2:11 Urantia milletleri, milliyetçi asker zihniyeti ve ulusal sanayinin devasa mücadelesine hali hazırda girmiş bulunmaktadır; ve bu çatışma birçok açıdan, sürüyü güden ve avcı olan bireyler ile çiftçi olanlar arasında çağlar boyu sürmüş mücadeleye benzemektedir. Ancak eğer üretim asker zihniyeti karşısında zaferle ayrılacaksa, onun yakasını bırakmayan tehlikelerden uzak durmak zorundadır. Urantia üzerinde filizlenmekte olan sanayinin tehlikeleri şunlardır:
\vs p070 2:12 1.\bibnobreakspace Ruhsal körlük biçiminde maddiyata olan güçlü kayış.
\vs p070 2:13 2.\bibnobreakspace Değerlerin yitirilmesi biçiminde servet temelli güce olan ibadet.
\vs p070 2:14 3.\bibnobreakspace Kültürel hamlık biçiminde şatafatın bayağı davranışları.
\vs p070 2:15 4.\bibnobreakspace Hizmet etmeye karşı duyarsızlaşma biçiminde tembelliğin artan tehlikeleri.
\vs p070 2:16 5.\bibnobreakspace Biyolojik bozulma biçimindeki istenmeyen ırksal zayıflıkların büyüme göstermesi.
\vs p070 2:17 6.\bibnobreakspace Kişilik durağanlığı olarak tek\hyp{}tipleştirilen sanayi köleliği tehlikesi. Emek soylulaştıran bir etkinliktir, ancak angarya düzeyindeki işler bireyi hissizleştirmektedir.
\vs p070 2:18 Askeri zihniyet temelli yönetim --- yabansı nitelikte --- zorba ve zalimdir. Bu zihniyet, galipler arasında toplumsal örgütlenmeyi sağlamaktadır; ancak aynı zamanda bahse konu yönetim, mağlupları toplum düzeninden ayrıştırmaktadır. Sanayi üretimi daha medeni bir düzeyde bulup, girişkenliği destekleyen ve bireyselliği teşvik eden bir biçimde yerine getirilmelidir. Toplum olası her biçimde özgünlüğü desteklemelidir.
\vs p070 2:19 Savaşı yüceleştirme hatasına düşmeyin; bunun yerine savaşın toplum için ne yaptığını iyice algılamaya çalışınız ki medeniyeti ilerletmek için onun yerine konacak uygulamaların ne sağlaması gerektiğini daha doğru bir biçimde tahayyül edebilesiniz. Ve eğer savaşın yerine konulacak bu türden uygulamalar sağlanmaz ise savaşın uzun yıllar varlığını korumaya devam edeceğinden emin olabilirsiniz.
\vs p070 2:20 İnsan; maddi refahı için barışın en iyisi olduğuna tamamen ve sürekli bir biçimde ikna olmadıkça ve insan varlıklarının öz\hyp{}korunum tepkilerine ait en başından beri birikmekte olan duyguları ve enerjileri özgürleştirmek için tasarlanmış ortak bir hareketi dönemsel olarak dışa vurmaya dair içkin eğilimi tatmin etmek amacıyla savaşın barışçıl muadillerini toplum bilge bir biçimde sağlamadıkça, yaşamın olağan bir akışı olarak barışı hiçbir zaman kabul etmeyecektir.
\vs p070 2:21 Ancak mevcut an içerisinde bile savaş, kibirli bireylerden oluşan bir ırkın kendisini --- bir baş idareci olarak --- tek elde toplanmış yönetime bırakmasını zorlamış bir deneyim okulu olarak onurlandırılmalıdır. Eski türden savaşlar önderlik için doğuştan büyük insanları seçmekteydi; ancak çağdaş savaş artık bu seçiciliği sergilememektedir. Önderleri keşfetmek için toplum bugün; üretim, bilim ve toplumsal kazanım olarak barışın fatihlerine yüzünü çevirmek zorundadır.
\usection{3.\bibnobreakspace Öncül İnsan Birliktelikleri}
\vs p070 3:1 En ilkel toplumda \bibemph{kitle topluluğu} her şeydi; çocuklar bile, bu kitle topluluğunun ortak mülkiyeti içindedir. Evrimleşen aile çocukların yetiştirilmesinde kitleden ayrılırken, ortaya çıkan kavimler ve kabileler toplum birimi olarak bu kitle topluluğunun yerini almıştır.
\vs p070 3:2 Cinsel açlık ve anne sevgisi aile kurumunu oluşturmaktadır. Ancak gerçek anlamda hükümet, aile üstü toplulukların oluşmaya başlamasına kadar ortaya çıkmamaktadır. Kitle topluluğunun aile öncesi zamanlarında önderlik, resmi olarak seçilmemiş bireyler tarafından sağlanmaktaydı. Afrikalı Buşmanlar, bu ilkel aşamanın ötesine geçen bir biçimde hiçbir zaman ilerleme kaydetmemişlerdir; onlar kitle toplulukları içinde önderlere sahip değillerdir.
\vs p070 3:3 Aileler, yakınlarının oluşturduğu topluluklar biçiminde kavimler içindeki kan bağları tarafından bir araya gelmişti; ve bu toplumsal birimler bir sonraki aşamada bölgesel halklar olarak kabilelere evirilmiştir. Savaşlar ve dışsal baskılar kabile örgütlenmesinin kan bağına dayalı kavimler temelinde kurulmasında baskı unsuru olmuştur; ancak bu öncül ve ilkel toplulukları bir ölçüde içsel barışla bir arada tutan etki alış\hyp{}veriş ve ticaret olmuştur.
\vs p070 3:4 Urantia’nın barışı, geleceğe dönük barış tasarımlarına ait duygusal savlarının tümüne kıyasla uluslararası ticaret örgütlenmeleri tarafından daha etkin bir biçimde sağlanacaktır. Ticari ilişkiler en başından beri, dilin gelişmesi ve daha iyi ulaşıma ek olarak iletişimin ilerlemiş yöntemleri tarafından sağlanmaktadır.
\vs p070 3:5 Ortak bir dilin yokluğu her zaman, barış topluluklarının gelişmesini engellemiştir; ancak para, çağdaş ticaretin evrensel dili haline gelmiştir. Çağdaş toplum büyük ölçüde, üretim pazarı tarafından bir arada tutulmaktadır. Kar gayesi, hizmet etme arzusu ile bütünleştiğinde kudretli bir medeniyetleştiricidir.
\vs p070 3:6 Öncül çağlarda her kabile, artan korku ve şüphenin iç içe geçmiş döngüleri tarafından çevrelenmişti; bu nedenle yabancıların tümünü öldürmek bir dönemde adet halinde iken, daha sonraki zamanlarda onların köleleştirilmesi gelenekselleşmişti. Arkadaşlığa dair var olan eski düşünce, diğer bir bireyin kavim ilişkileri içine alınması anlamına gelmekteydi; ve --- ebedi yaşamın en öncül kavramlarından biri olarak --- kavim üyeliğinin ölümden sonra bile varlığını devam ettirdiğine inanılmaktaydı.
\vs p070 3:7 Kavim üyelik töreni, toplum üyelerinin her birinin kanının içilmesinden oluşmaktaydı. Bazı topluluklar, toplumsal öpme uygulamasının tarihi kökeni olarak, kan yerine tükürük değiş tokuşunda bulundular. Ve, ister evlilik ister kabile üyeliğine alma uygulaması olsun birliktelik törenlerinin hemen arkasından her zaman yemek ziyafeti gelmekteydi.
\vs p070 3:8 Daha sonraki zamanlarda kanla karıştırılan kırmızı şarap kullanılmıştı; ve nihai olarak yalnızca şarap, kadehlere dokunulmasıyla ve bunun sonrasında içkinin içilmesiyle yerine getirilen kavim üyeliğine olan kabul törenlerini tamamlamaktaydı. İbraniler, bu üyelik törenlerinin değişime uğramış bir türünü uygulamaktaydı. Onların Arap ataları yemin törenlerini, üye adayının elini kabile atasının üreme organına koyarken gerçekleştirmişlerdir. İbraniler, kavim üyeliğine alınan yabancılara iyi ve kardeşçe davranmışlardı. “Sizlerle yaşayan yabancı aranızda doğmuşlar gibi kabul görmeli, ve siz onu kendiniz gibi sevmelisiniz.”
\vs p070 3:9 “Ziyaretçi arkadaşlığı” geçici misafirperverliğin bir ilişki türüydü. Ziyaretçiler ayrıldığı zaman bir tabak ortadan ikiye kırılır, ve tabağın bir yarısı evden ayrılan arkadaşa daha sonraki bir ziyarette beraberinde gelebilecek üçüncü bir kişi için tanışma aracılığı sağlaması amacıyla verilirdi. Ziyaretçilerin geçimlerini seyahatleri ve maceralarını anlatarak sağlaması alışılagelmiş bir durumdu. Eski dönemlerin meddahları öyle sevilen ve aranılan bir konumdaydı ki, adetler nihai olarak onların etkinliklerinin av veya hasat mevsimlerinde yapılmasını yasakladı.
\vs p070 3:10 Barışın ilk antlaşmaları “kan kardeşi” olmaktı. Savaşan iki kabilenin barış elçileri bir araya gelir, birbirlerine saygıda kusur etmez ve daha sonra derilerine kanayıncaya kadar sivri bir cisim batırırlardı; ve bunun üzerine onlar birbirlerinin kanını emer ve barışı ilan ederlerdi.
\vs p070 3:11 En eski barış heyetleri; cinsel arzunun savaş dürtüsüyle mücadele etmek için kullanıldığı şekliyle, bir zamanlar düşmanları olan kişilerin cinsel tatmini için seçmiş oldukları bakire kızları getiren erkek üyelerinden oluşmaktaydı. Kabile bu uygulamadan o kadar onur duyardı ki, kendi bakire kızlarıyla birlikte iadeyi ziyarette bulunurdu; bunun üzerine barış sağlam bir biçimde tekrar sağlanırdı. Ve yakın bir süre içinde kabile önderlerinin aileleri arasında karşılıklı evliliklere izin verilirdi.
\usection{4.\bibnobreakspace Kavimler ve Kabileler}
\vs p070 4:1 İlk barış topluluğu aile, sonra kavim, daha sonra kabile ve bu toplumsal birimlerin oluşumunu takiben toprak temelli çağdaş devlet haline gelen ulustu. Her ne kadar Urantia milletlerinin çok büyük miktardaki bütçelerini savaş hazırlıklarına aktardıkları gerçeğine rağmen, bugünün barış topluluklarının uzun bir süreden beri kan bağlarının ötesine geçip uluslar ile bütünleşmiş olduğu oldukça umut vericidir.
\vs p070 4:2 Kavimler; bir kabile içinde akraba toplulukları olup, mevcudiyetlerini şunlar gibi önem verdikleri belirli ortak değerlere borçlu olmuşlardır:
\vs p070 4:3 1.\bibnobreakspace Ortak bir ataya köklerinin dayanması.
\vs p070 4:4 2.\bibnobreakspace Ortak bir dini kabile simgesine olan bağlılık.
\vs p070 4:5 3.\bibnobreakspace Aynı lehçeyi konuşma.
\vs p070 4:6 4.\bibnobreakspace Ortak bir ikamet yerleşkesini paylaşma.
\vs p070 4:7 5.\bibnobreakspace Aynı düşmanlardan korkma.
\vs p070 4:8 6.\bibnobreakspace Beraberce deneyimlenmiş bir askeri geçmişe sahip olma.
\vs p070 4:9 Kavmin başı, kavimlerin yönetim bakımından içinde serbestliğe sahip olduğu idari birlik biçimindeki öncül kabile hükümetleri olarak, her zaman kabile önderine tabiydi. Yerli Avustralyalılar, hükümetin kabilesel bir türünü hiçbir zaman geliştirmişlerdir.
\vs p070 4:10 Kavim barış sorumluları genellikle anne tarafından gelen bireyler tarafından idare edilirdi; kabile savaş önderleri ise baba tarafından gelen bireylerin yönetimi altındaydı. Kabile önderleri ve ilkel dönem krallarının mahkeme üyeleri, kralın huzuruna yılda birkaç defa çıkmaları adet haline getirilmiş görevli kavim başlarından oluşmaktaydı. Bu uygulama kralın onları gözetlemesini ve aralarındaki eş\hyp{}güdümü daha iyi bir biçimde teminat altına almasını sağlamıştır. Kavimler, yerel özerk yönetim içinde değerli bir amacı yerine getirmişlerdir; ancak onlar, büyük ve güçlü milletlerin büyümesine ciddi bir biçimde engel olmuşlardır.
\usection{5.\bibnobreakspace Hükümetin İlk Adımları}
\vs p070 5:1 Her insan kurumu bir başlangıca sahiptir; ve bir sivil hükümet tıpkı evlilik, üretim ve din gibi ilerleyen evrimin bir sonucudur. Öncül kavimler ve ilkel kabilelerden, yirminci yüzyılın ikinci çeyreğini simgeleyen toplumsal ve sivil idari yapının bu türlerine kadar gelmiş ve değişime uğramış insan hükümetinin birbirlerini takip eden düzenleri kademeli olarak gelişmiştir.
\vs p070 5:2 Aile birimlerinin kademeli bir biçimde ortaya çıkışı ile birlikte hükümetin temelleri, aynı kökenden gelen ailelerin topluluğu biçiminde kavim örgütlenmesi üzerinde inşa edilmiştir. Gerçek anlamda ilk hükümet bünyesi \bibemph{ihtiyar heyetiydi}. Bu idari topluluk, belirli bir kabul edilmiş kıstasta kendilerini ispat eden yaşlı erkeklerden oluşmaktaydı. Bilgelik ve deneyim, yabansı insanlar tarafından bile öncül olarak takdir görmekteydi; ve bunun sonrasında orada, yaşlı insanların uzun bir süre boyunca devam eden egemenliği gerçekleşmişti. Bu yaş temelli zümresel egemenlik kademeli olarak ataerkil toplum düşüncesine doğru büyüdü.
\vs p070 5:3 Yaşlıların öncül heyeti içerisinde yürütme, yasama ve yargı olarak tüm hükümet faaliyetlerinin olanağı mevcut bir haldeydi. Heyet geçerli olan adetleri incelerken bir mahkeme görevi görmekteydi; toplum davranışlarının yeni türlerini oluştururken bir yasa koyucuydu; bu türden yönergelerin ve kararların uygulanması bakımından ise yürütme görevindeydi. Heyet başkanı, daha sonraki yönetim mevkisi olan kabile önderi makamının idari habercilerinden biriydi.
\vs p070 5:4 Bazı kabileler kadın heyet üyelerine sahipti, zaman zaman birçok kabile kadın yöneticilere sahip olmuştu. Kırmızı insanın belirli kabileleri, “yedi kişilik heyetin” oy birliği ile karara varma idari yapısını takip ederek Onamonalonton’un öğretisini korumuştur.
\vs p070 5:5 Ne barışın ne de savaşın yalnızca tartışan bir toplum tarafından yönetilemeyeceğini insan türünün öğrenmesi zor olmuştur. İlkel “lakırdılar” çok nadir koşullarda yarar sağlamaktaydı. Irk öncül bir biçimde, kavim önderlerinden oluşan bir topluluk tarafından emirle yönetilen bir ordunun güçlü tek bir önder tarafından yönlendirilen ordu karşısında herhangi bir şansının olmadığını öğrenmişti. Savaş her zaman siyasi önderliği belirleyen bir etkinlik olmuştur.
\vs p070 5:6 İlk başta kabile önderleri sadece askeri hizmet için seçilmekteydi; ve onlar, görevlerinin daha toplumsal bir niteliğe dönüştüğü dönemler olan barış zamanları boyunca yetkilerinin bazılarından vazgeçerlerdi. Ancak yavaş yavaş onlar, bir savaştan diğerine sürecek şekilde yönetimlerini devam ettirme eğilimi göstererek, barış dönemlerine bile yetkilerinden vazgeçmemeye başladılar. Onlar, bir savaşın diğerini takip etmesinin çok uzun bir süre almadığını sıklıkla gözlemlediler. Bu öncül savaş hâkimleri barış yanlıları değillerdi.
\vs p070 5:7 Daha sonraki dönemlerde bazı önderler, benzersiz vücut yapıları veya olağanüstü kişisel becerileri nedeniyle tercih edilerek, askeri hizmet dışındaki görevler için de seçilmişlerdi. Kırmızı insanlar sıklıkla, kabile reislerine veya diğer bir değişle barış önderlerine ek olarak babadan oğla geçen savaş önderleri biçiminde, iki çeşit yöneticiye sahiplerdi. Barış yöneticileri aynı zamanda hâkimler ve öğretmenlerdi.
\vs p070 5:8 Bazı öncül halklar, sıklıkla önderler olarak hareket eden sağlıkçılar tarafından yönetilmekteydi. Tek kişi din adamı, doktor ve baş yönetici olarak hareket etmekteydi. Oldukça sık bir biçimde kabile simgeleri kökensel olarak, bir zamanların din mensubu kıyafetlerinin işaretleri veya armalarıydı.
\vs p070 5:9 Ve bu aşamalardan geçerek hükümetin yönetim erki kademeli olarak mevcudiyetini kazanmıştır. Kavim ve kabile heyetleri danışman konumlarında görevlerine devam etmiş ve daha sonra açığa çıkmış yasama ve yargı erklerinin öncülleri olmuşlardır. Bugün Afrika’da ilkel hükümetin bu türleri, çeşitli kabileler arasında faal bir biçimde mevcutturlar.
\usection{6.\bibnobreakspace Monarşik Hükümet}
\vs p070 6:1 Etkin devlet yönetimi ancak, bütüncül yürütme yetkisine sahip bir önderin ortaya çıkmasıyla gelmiştir. İnsan etkin bir hükümete, bir fikri yaratarak değil ancak bir kişiliğe gücü teslim ederek sahip olunabileceğini keşfetmiştir.
\vs p070 6:2 Yönetim idaresi, aile yönetimi veya serveti fikrinden doğmuştur. Ataerkil bir kralcık gerçek bir kral haline geldiğinde, zaman zaman “topluluğunun babası olarak” adlandırılırdı. Daha sonra kralların kahramanlardan türediklerine inanılırdı. Ve çok daha sonraları yönetim idaresi, kralların kutsal kökenine dair inanış nedeniyle babadan oğla geçer hale geldi.
\vs p070 6:3 Saltanatsal krallık, kralın ölümü ile onun yerine geçecek bireyin seçileceği dönem arasında daha önceleri yaşanan yıkımlara sebep olan anarşiyi önledi. Aile biyolojik bir başa; kavim, seçilmiş doğal bir öndere sahipti; kabile ve daha sonraki devlet düzeni doğal bir öndere sahip değildi; ve bu durum, baş kralları babadan oğla geçen bir düzene oturtmak için ilave bir nedendi. Kraliyet aileleri ve soylular sınıfı fikri aynı zamanda, kavimler içindeki “soyadına sahip olmaya ” dair adetlere dayanmaktaydı.
\vs p070 6:4 Kralların saltanatı nihai olarak, kraliyet kanının Prens Caligastia’nın yeniden\hyp{}bedene kavuşturulmuş görevlileri zamanına kadar uzandığına inanılan bir biçimde, doğaüstü olarak görülmekteydi. Böylelikle krallar tapınan kişilikler haline gelip, kraliyeti ifade etmek için özel bir hitabet şeklinin uyarlandığı biçimde kendilerinden olağanüstü derecede korku duyulmaktaydı. Eski dönemlerde bile kralların dokunuşunun hastalıkları iyileştirdiğine inanılmaktaydı; ve bazı Urantia halkları hala yöneticilerinin kutsal bir kökene sahip olduklarını düşünmektedirler.
\vs p070 6:5 Öncül tapınan kral sıklıkla gözlerden uzak mekânlarda tutulmuştur; onun, şölen ve kutsal günler dışındaki zamanlarda görünmek için çok kutsal bir düzeyde olduğu düşünülmekteydi. Genellikle bir vekil kendisini temsil etmesi için seçilirdi; ve bu uygulama başbakanlık kurumunun kökenidir. İlk kabine görevlisi yiyecekten sorumlu bir idareciydi; diğer bakanlar kısa bir süre sonra bu bakanlığı izledi. Krallar yakın bir zaman içinde ticaret ve dinden sorumlu temsilcileri atadı; ve bir kabinenin gelişimi, yürütme yetkisinin kişiler temelli siyasetten arınmasında ilk doğrudan aşamaydı. Öncül kralların bu yardımcıları soyluluğa kabul edilmiş bireyler haline geldi; ve kralın eşi kademeli olarak, kendisine daha yüksek itibarla davranılan kadınlar biçiminde kraliçe soyluluğuna yükseldi.
\vs p070 6:6 Acımasız yöneticiler zehrin keşfiyle birlikte büyük bir güç elde ettiler. Öncül saltanat sihri şeytaniydi; kralın düşmanları yakın bir zaman içinde yaşamlarını yitirdi. Ancak en zalim hükümdar bile bazı sınırlamalara tabiydi; en azından bu yönetici, en başından beri varlığını sürdüregelmiş suikast korkusu ile kısıtlandırılmıştı. Sihirbaz doktorlar biçiminde sağlıkçılar ve din mensupları her zaman krallar üzerinde güçlü bir denetim unsuru olmuştur. Bunun sonrasında soylular sınıfı olarak toprak sahipleri, sınırlandırıcı bir etkiyi uygulamışlardır. Ve zaman zaman kavimler ve kabileler tamamen ayaklanıp, zorba hükümdarlarını yönetimden uzaklaştırmışlardır. Tahtan indirilen yöneticilere ölüm cezası verildiğinde intihar etme tercihi kendilerine sunulmuştur; bu tercih belirli durumlarda intiharın eski dönemlerdeki toplumsal rağbetine kaynaklık etmiştir.
\usection{7.\bibnobreakspace İlkel Cemiyetler ve Gizli Topluluklar}
\vs p070 7:1 Kan bağı, ilk toplumsal toplulukları belirlemiştir; birleşme kan bağına dayalı kavimleri genişletmiştir. Karşılıklı evlilik toplumun büyümesinde bir sonraki aşamaydı; ve bunun sonrasında açığa çıkan karmaşık kabile ilk gerçek siyasi bünyeydi. Toplumsal gelişmede bir sonraki ilerleme, dinsel inanışların ve siyasi cemiyetlerin evrimiydi. Bu cemiyetler ilk olarak gizli topluluklar biçiminde ortaya çıkmış olup, kökensel olarak bütünüyle dinsel nitelikteydi; daha sonra onlar idari niteliğe büründü. İlk başta onlar erkeklerin cemiyetleriydi; daha sonra kadınların toplulukları ortaya çıkmaya başladı. Daha sonra onlar, sosyo\hyp{}politik ve dini\hyp{}mistik olarak iki sınıfa ayrılmıştı.
\vs p070 7:2 Bu toplulukların gizliliği için şunlar gibi birçok neden mevcut bulunmaktaydı:
\vs p070 7:3 1.\bibnobreakspace Birtakım tabulara karşı gelmekten dolayı yöneticiler üzerinde hoşnutsuzluk yaratma korkusu.
\vs p070 7:4 2.\bibnobreakspace Azınlıkta bulunan dini törenleri yerine getirme amacı.
\vs p070 7:5 3.\bibnobreakspace Değerli “ruhaniyeti” veya ticaret sırlarını koruma gayesi.
\vs p070 7:6 4.\bibnobreakspace Birtakım özel büyü veya sihri memnuniyetle deneyimle arzusu.
\vs p070 7:7 Bu toplulukların bahse konu gizliliği tüm üyelerine kabilenin geri kalanları karşısında gizemlilik gücü vermiştir. Gizlilik aynı zamanda gösteriş duygusunu hoşnut eden bir içeriğe sahiptir; bu gizliliğin başlatıcıları, dönemlerinin toplumsal soylu sınıf üyeleriydi. Ergenlik dönemini geçen erkek çocuklar erişkin büyükleri ile birlikte avlanmaktaydı; ancak bu dönemden önce onlar kadınlar ile birlikte sebze toplamaktalardı. Ergenlik sınavlarını geçmede başarısız olmak ve böylece kadınsı olarak görüldüklerinden dolayı erkek yerleşkelerinin dışında kadınlar ve çocuklar ile birlikte kalmaya zorlanmak bir kabile utancı biçiminde olası en yüksek aşağılamaydı. Bu aşağılamanın dışında kadınsı olarak görülen erkeklerin evlenmelerine izin verilmemekteydi.
\vs p070 7:8 İlkel insanlar ergenlik döneminde bulunan çocuklarına çok önceden cinsel ilişki denetimini öğretmişlerdi. Erginlik ve evlilik dönemleri arasında erkek çocukları, eğitimleri ve hazırlanmaları erkeklerin gizli topluluklarına emanet edilen bir biçimde, ebeveynlerinden alıkoymak adet haline gelmişti. Ve bu cemiyetlerin başlıca faaliyetlerinden biri, ergenlik döneminde bulunan genç çocukları denetim altında tutmak, böylece gayrimeşru çocukların doğmasını engellemekti.
\vs p070 7:9 Ticarileşen fuhuş, erkeklerin sahip oldukları cemiyetlerin diğer kabilelerden olan kadınları para karşılığında kullanmasıyla başladı. Ancak daha önceki topluluklar dikkate değer bir biçimde cinsel sınırlamaları takip etmekteydi.
\vs p070 7:10 Erişkinliğe olan kabul töreni genellikle, beş yıllık bir döneme uzanmaktaydı. Bireyin kendine işkence etmesi ve bedenin bir parçasını acı dolu bir biçimde kesmesi bu törenlere girdi. Sünnet ilk olarak, bu gizli kardeşlik cemiyetlerinden biri için bir kabul ayini olarak uygulanmıştı. Kabile armaları, erişkinliğe kabulün bir parçası olarak beden üzerine kesim yoluyla işlenmekteydi; dövme uygulaması, üyeliğin bu türden bir nişanı olarak doğmuştu. Ciddi oranda yaratılan yoksunluk ile birlikte bu türden işkenceler, hayatın gerçekleri ve onun kaçınılmaz zorluklarıyla gençleri etkilemek amacıyla bu bireyleri sert bir biçimde pekiştirmek için tasarlanmıştı. Bu gaye, daha sonra ortaya çıkan atletizm oyunları ve fiziksel mücadeleler ile daha iyi bir biçimde yerine getirilmiştir.
\vs p070 7:11 Ancak gizli cemiyetler kesin bir biçimde ergenlik ahlakının gelişmesini amaçlamıştır; ergenlik törenlerinin başlıca gayelerinden biri, erkek çocuğu diğer erkeklerin kadınlarını hiçbir biçimde arzulamaması yönünde eğitmekti.
\vs p070 7:12 Evlilik öncesi gerçekleşen oldukça ciddi disiplin ve eğitimin bu yıllarından sonra genç erkekler genellikle, geri dönünce gerçekleştirecekleri evliliklerinden ve kendilerini kabile tabularına bir ömür boyu teslim etmelerinden önce kısa bir dönemliğine rahatlık ve özgürlük koşullarına bırakılmaktalardı. Ve eski adet, bekârların “sınırsız özgürlük dönemi” olarak bilinen budalaca uygulamaya uzanan bir biçimde çağdaş dönemlere kadar süregelmiştir.
\vs p070 7:13 Daha sonraki birçok kabile, ergenlik çağına girmiş kızları eşlik ve anneliğe hazırlamak amacıyla oluşturulmuş, kadınların gizli cemiyetlerinin kurulmasına izin verdi. Erişkinlik düzeyine kabulünden sonra kızlar evlilik için yetkin hale gelip, bu zamanların görücüye çıkma eğlencesi biçimindeki “gelin gösterisine” katılmalarına izin verilirdi. Evlilik için kadınların yarattığı düzenler öncül bir biçimde mevcudiyete kavuşmuştu.
\vs p070 7:14 Yakın bir süre sonra gizli olmayan cemiyetler, evlenmeyen erkekler ve herhangi bir erkeğe bağlı olmayan kadınlar kendi ayrı örgütlenmelerini gerçekleştirdiğinde ortaya çıkmıştır. Bu birliktelikler gerçek anlamdaki ilkokullardı. Ve, erkek ve kadınların cemiyetleri birbirlerine sıklıkla huzur vermese de, bazı ileri kabileler Dalamatia eğitmenleri ile iletişime geçtikten sonra her cinsiyetin aynı yerde öğrenimi için yatılı okullara sahip olarak ortak eğitimi deneyimlemişlerdir.
\vs p070 7:15 Gizli topluluklar, kabul düzenlerinin gizemli yapısıyla başlıca olmak üzere toplumun daha fazla katmanlı hale gelmesine neden oldular. Bu toplum üyeleri ilk olarak, --- atalara olan ibadet biçiminde --- matem törenlerine merak besleyen kişileri maske takıp korkutmuşlardır. Daha sonra bu korkutma, hayaletin ortaya çıktıklarına dair inancın temellendiği aldatıcı bir oturuma doğru gelişme gösterdi. “Yeniden doğumun” bu eski cemiyetleri simgeler ile anlaşıp, özel gizli bir dili kullandı; onlar aynı zamanda belirli yiyecek ve içecekleri tüketmemeye yemin ettiler.
\vs p070 7:16 Tüm gizli birliktelikler; istenen teminat biçiminde bir yemin etme uygulamasını zorunlu kılmış olup, sırların tutulmasını öğretmiştir. Bu düzenler, kitlelerin korkuyla karışık saygısını kazanmış ve onları denetim altına almıştır; onlar aynı zamanda, herhangi bir yanlış karşısında tetikte bekleyen cemiyetler olarak hareket etmiş olup, böylelikle linç kültürünü uygulamıştır. Onlar, kabile savaşlarında ilk hafiyeler ve kabile barışlarında ilk gizli polislerdi. Bu görevleri arasında onların neden oldukları en iyi şey, acımasız kralları endişeli bir biçimde makamlarında tutmalarıdır. Onların üstesinden gelebilmek için krallar kendilerine ait gizli polisleri öne sürmüştür.
\vs p070 7:17 Bu cemiyetler ilk siyasi partilerin oluşmasına kaynaklık etmiştir. İlk parti hükümeti “güçsüz” \bibemph{karşısında} “güçlü” yönetimiydi. Eski dönemlerde idari düzende yapılan bir değişiklik yalnızca sivil savaşa sebebiyet vermekteydi; bu bağlamda güçsüzün daha sonra güçlü hale geldiğine dair sayısız delil bulunmaktadır.
\vs p070 7:18 Bu cemiyetler; tüccarlar tarafından borçları tedarik etmek için, yöneticiler tarafından ise vergileri toplamak için kullanılmaktaydı. Vergi, av veya ganimetin onda biri biçiminde aşar vergisinin en ilk türlerinden biri olarak uzun bir mücadele alanıydı. Vergiler kökensel olarak, kralın evini idame etmek için zorla alınmaktaydı; ancak daha sonra, tapınak hizmetini desteklemek için bir bağış altında gösterildiğinde vergileri toplamanın daha kolay olduğu keşfedildi.
\vs p070 7:19 Bu gizli birliktelikler aşama aşama ilk yardım örgütlenmelerine doğru gelişmiş olup, daha sonra --- kiliselerin öncülleri olarak --- daha önceki din cemiyetleri haline evirildiler. Nihai olarak bu cemiyetlerden bazıları, ilk uluslararası kardeşlik toplulukları olarak, kabileler arası düzeye geldiler.
\usection{8.\bibnobreakspace Toplumsal Sınıflar}
\vs p070 8:1 İnsan varlıklarının akılsal ve fiziksel eşitsizliği, toplumsal sınıfların ortaya çıkışına temel teşkil etmektedir. Toplumsal tabakalaşmaya sahip olmayan sadece iki tür dünya vardır; bunlar en ilkel ve en gelişmiş dünyalardır. Doğmakta olan bir medeniyet toplumsal düzeylerin farklılaşması sürecine henüz başlamamıştır; bunun karşısında ışık ve yaşam altında istikrara kavuşturulmuş olan bir dünya ise, orta düzey evrimsel aşamaların tümünün oldukça temel niteliği olan insan türünün bu bölünmelerinin büyük ölçüde üstesinden gelmiştir.
\vs p070 8:2 Toplum hayvansı yaşamdan kurtulup vahşi yaşama gelirken onun insan öğeleri şu genel nedenlerden dolayı sınıflar halinde birliktelikler içinde bulma eğilimi gösterirler:
\vs p070 8:3 1.\bibnobreakspace \bibemph{Doğal} --- yakınlık, kan bağı ve evlilik; ilk toplumsal farklılaşmalar cinsiyet, yaş ve kabile önderiyle olan kan bağı ilişkisi şeklinde kandı.
\vs p070 8:4 2.\bibnobreakspace \bibemph{Kişisel} --- yetenek, dayanıklılık, beceri ve cesaretin tanınması; bu nitelikleri yakın bir zamanda dil üzerindeki ustalık, bilgi ve genel us takip etti.
\vs p070 8:5 3.\bibnobreakspace \bibemph{Şans} --- savaşlar ve göçler insan topluluklarının ayrılmasına sebebiyet verdi. Sınıf evrimi güçlü bir biçimde, galibin mağlup ile olan ilişkisi biçiminde fetihlerden etkilenirken; kölelik, özgür ve tutsak olarak toplumun ilk genel sınıflanışını beraberinde getirdi.
\vs p070 8:6 4.\bibnobreakspace \bibemph{Ekonomik} --- fakir ve zengin. Servet ve kölelerin sahipliği, toplumun bir sınıfının kalıtımsal temeliydi.
\vs p070 8:7 5.\bibnobreakspace \bibemph{Coğrafi} --- sınıflar şehir veya kırsal yerleşimlere bağlı olarak ortaya çıkmıştır. Şehir ve kır yaşamı sırasıyla, sürü ile uğraşan tarım insanları ve ticaret ile uğraşan üretim insanlarının değişik bakış açıları ve tepkileriyle birlikte ayrışmasına katkıda bulunmuştur.
\vs p070 8:8 6.\bibnobreakspace \bibemph{Toplumsal} --- sınıflar kademeli olarak, farklı toplulukların toplumsal değerinin dönemsel rağbeti ölçüsünde oluşmuştur. Bu türün en öncül bölünmesi; din mensupları ile öğretmenler, yöneticiler ile kahramanlar, sermaye sahipleri ile ticaret erbapları, olağan işçiler ile köleler arasındaki ayrımın kesinleşmiş hale gelmesiydi. Köle hiçbir zaman bir sermaye sahibi haline gelemezken, zaman zaman yevmiye ile çalışan birey sermaye sahipleri arasına girmeyi tercih edebilirdi.
\vs p070 8:9 7.\bibnobreakspace \bibemph{Mesleksel }--- meslekler çoğalınca toplumsal tabakalar ve loncalar kurma eğilimi gösterdiler. Çalışanlar; sağlıkçıları da içine alan bir biçimde uzman sınıflar, onlardan sonra gelen kalifiye işçiler ve en sonuncu olarak vasıfsız işçiler şeklinde üç topluluğa ayrılmıştı.
\vs p070 8:10 8.\bibnobreakspace \bibemph{Dinsel} --- öncül dini inanış cemiyetleri, kavimler ve kabileler içinde kendine ait sınıfları yaratmıştır; ve din mensuplarının takvası ve tasavvufu onların ayrı bir toplumsal zümre olarak uzun süreden beri varlıklarını devam ettirmelerini sağlamıştır.
\vs p070 8:11 9.\bibnobreakspace \bibemph{Irksal} --- iki veya daha fazla ırkın bir millet veya bölgesel birimdeki mevcudiyeti genellikle ırk sınıflarını yaratmaktadır. Hindistan’da tabakalaşmış özgün toplum düzeni, öncül Mısır’da olduğu gibi ten rengine dayanmaktaydı.
\vs p070 8:12 10.\bibnobreakspace \bibemph{Yaş} --- çocukluk ve olgunluk. Kabileler arasında erkek çocuk babanın yaşamı boyunca onun gözetimi altında kalmaya devam ederken, kız çocuk evlenene kadar annenin ilgisine bırakılmıştı.
\vs p070 8:13 Esnek ve değişken toplumsal sınıflar evrimleşen bir medeniyet için hayati öneme sahiptir; ancak \bibemph{sınıf} katı bir biçimde \bibemph{tabakalaştığı} zaman, toplumsal düzeylerin esneksel geçirgenliğini yitirdiği anda, toplumsal istikrarın gelişmesi bireysel özgür teşebbüsün azalmasıyla karşılanır. Katı toplumsal tabakalaşma, bir bireyin kendisine üretim alanında bir yer edinme sorununu çözer; ancak bu toplumsal yapı aynı zamanda bireysel gelişimi ciddi bir biçimde sekteye uğratarak toplumsal işbirliğini neredeyse tamamen engeller.
\vs p070 8:14 Doğal yollardan oluşan bir biçimde toplum içindeki sınıflar; ilerleyen bir medeniyetin biyolojik, ussal ve ruhsal kaynaklarının şu gibi biçimlerde akıllı bir biçimde kullanılmasıyla insanın bu sınıfların evrimsel yıkımlarına kademeli olarak erişimine kadar varlığını sürdürmeye devam edecektir:
\vs p070 8:15 1.\bibnobreakspace Irk kollarının biyolojik yenilenişi --- alt düzeyde bulunan insan kollarının seçici bir biçimde elenmesi. Bu uygulama, birçok fani eşitsizliği yok etme eğilimini ortaya çıkaracaktır.
\vs p070 8:16 2.\bibnobreakspace Bahse konu biyolojik gelişimden doğacak olan artan beyin gücünün eğitimsel hazırlanışı.
\vs p070 8:17 3.\bibnobreakspace Fani kan bağı ve kardeşliğe ait duyguların din tarafından hızlandırılışı.
\vs p070 8:18 Her ne kadar toplumsal ilerlemenin büyük bir kısmı, kültürel ilerlemenin bu hızlandırıcı etkenlerinin ussal, bilge ve \bibemph{sabırlı} kullanımı sonucunda gerçekleşecek olsa da; bahse konu bu çalışmalar gerçek meyvelerini yalnızca geleceğin uzak bin yıllarında verebilir. Din, medeniyeti karmaşadan çekip çıkaran kudretli bir kaldıraçtır; ancak din, güçlü ve sağlıklı kalıtım üzerine güvenli bir biçimde oturmuş güçlü ve sağlıklı aklın dayanak noktası olmadan güçsüz bir konumdadır.
\usection{9.\bibnobreakspace İnsan Hakları}
\vs p070 9:1 Doğa insanlara hiçbir hakkı sağlamamaktadır; doğa yalnızca bir yaşam ve içinde yaşanılabilecek bir dünya sunmaktadır. Silahsız bir insanın ilkel bir ormanda aç bir kaplan ile karşı karşıya gelmesiyle ortaya çıkacak olası sonuçlardan çıkarılabileceği gibi, doğa insanlara yaşama hakkı bile sunmamaktadır. Toplumun insana verdiği en temel hediye onun güvenliğidir.
\vs p070 9:2 Kademeli olarak toplum kendi haklarını belirlemiştir; ve mevcut zaman içerisinde bu haklar şunlardır:
\vs p070 9:3 1.\bibnobreakspace Yiyecek arzının güvence altına alınması.
\vs p070 9:4 2.\bibnobreakspace Askeri savunma --- hazırlıklı olma yoluyla sağlanan güvenlik.
\vs p070 9:5 3.\bibnobreakspace İç barış korunumu --- kişisel şiddet ve toplumsal düzensizliğin önlenmesi.
\vs p070 9:6 4.\bibnobreakspace Cinsel denetim --- aile kurumu olarak evlilik.
\vs p070 9:7 5.\bibnobreakspace Özel mülkiyet --- iyelik hakkı.
\vs p070 9:8 6.\bibnobreakspace Bireysel ve topluluk rekabetini destekleme.
\vs p070 9:9 7.\bibnobreakspace Gençliğin öğrenimi ve eğitiminin sağlanması.
\vs p070 9:10 8.\bibnobreakspace Ticaret ve alışverişi sağlama --- üretimsel gelişim.
\vs p070 9:11 9.\bibnobreakspace İş gücü şartlarındaki ve emeğin mükâfatlandırılmasındaki iyileştirme.
\vs p070 9:12 10.\bibnobreakspace Dinsel adetlerin yerine getirilmesine dair özgürlüğün, bahse konu diğer toplumsal etkinliklerin tümünün ruhsal bir dürtüyle etkinleşen hale gelerek yüceltilebileceği gayesiyle teminat altına alınması.
\vs p070 9:13 Haklar; bilginin açığa çıktığı dönemden bile eski olduğunda, sıklıkla \bibemph{doğal haklar} olarak adlandırılır. Ancak insan hakları, gerçekten, doğal değillerdir; onlar bütünüyle toplumsaldır. Onlar; insan rekabetinin sürekli değişin olgular bütününü idare eden ilişkilerin tanınmış düzenlemeleri biçiminde, oyunun kurallarından başka bir şey olmayarak göreceli bir nitelikte bulunup en başından beri dönüşüme uğramaktadır.
\vs p070 9:14 Bir çağda hak olarak değerlendirilen bir olgu, diğerinde bu yönde düşünülmeyebilir. Kusurlu ve yozlaşmış bireylerin geniş sayıdaki nüfusunun mevcut anın çağına kadar gelmesi, onların yirminci yüzyıl medeniyetini sekteye uğratma gibi doğal bir hakka sahip olması sonucunda gerçekleşmemektedir; onların varlığı sadece, çağın toplumunu oluşturan ahlaki değerler ve böylelikle kabul edilen hükümlerden kaynaklanmaktadır.
\vs p070 9:15 Avrupa’nın Orta Çağları’nda çok az sayıda insan hakkı tanınmıştır; bu dönemde her insan başka bir bireye ait bir konumdaydı; ve haklar yalnızca, devlet veya kilise tarafından verilen ayrıcalıklar veya iltimaslardı. Ve bu hatadan kaynaklanan ayaklanma, insanların tümünün eşit doğduğuna dair düşünceye yol açması bakımından eşit derecede hatalıydı.
\vs p070 9:16 Güçsüz ve alt düzeyde bulunun bireyler her zaman eşit haklar için mücadele etmişlerdir; onlar her zaman, güçlü ve üstün seviyede bulunan bireylerin devlet vasıtasıyla kendi ihtiyaçlarını karşılamaya zorlanmasını istemiş olup, aksi halde yetersizliklerini kötüye kullanmakla tehdit etmiştir; gerçekte onların bu yetersizlikleri oldukça sık gerçekleşen bir biçimde ilgisizlikleri ve tembelliklerinin doğal sonucundan kaynaklanmaktadır.
\vs p070 9:17 Ancak eşitliğe dair bu nihai amaç medeniyetin bir çocuğudur; bu amaç doğa içinde mevcut değildir. Kültürün kendisi bile, insanların eşit olmayan yetkinliklerinden kaynaklanan içkin eşitsizliklerini kesin olarak göstermektedir. Varsayılan doğal eşitliğin anlık ve evrimsel olmayan bir biçimde gerçekleştirilmeye çalışılması ilkel çağların gelişmemiş yaşamlarına medeni insanı hızlı bir biçimde geri atacaktır. Toplum herkese eşit haklar sunamaz; ancak bu oluşum adalet ve hakkaniyet uyarınca herkesin çeşitlilik gösteren haklarını gözetme sözü verebilir. Doğanın çoğunun kendi yaşam idaresini sağlaması, kendi nesillerini dünyaya getirmesi ve aynı zamanda bireysel tatminin bir düzeyini memnuniyetle deneyimlemesi için adil ve barışçıl bir olanak sunmak toplumun işi ve görevidir; birey için sağlanacak olan bu üç faaliyetin toplamı insan mutluluğunu oluşturmaktadır.
\usection{10.\bibnobreakspace Adaletin Evrimi}
\vs p070 10:1 Doğal adalet insanın yarattığı bir savdır; böyle bir adalet gerçekte bulunmamaktadır. Doğada adalet düşüncesi, bütünüyle bir kurgu olarak, tamamiyle insanın yarattığı bir kuramdan ibarettir. Doğa yalnızca, sonuçların nedenlere olan kaçınılmaz bağlılığı biçiminde, adaletin tek bir türünü sağlamaktadır.
\vs p070 10:2 İnsan tarafından algılandığı biçimde adalet, bir insanın haklarını elde etmesi ve bunun sonucunda onun ilerleyici evrimin bir parçası olmasıdır. Adalet kavramı, ruhaniyet bahşedilen bir akıl içinde oldukça kökleşmiş ve yaşamı belirleyen bir nitelikte bulunabilir; ancak adalet, mekânın dünyalarında uçsuz bucaksız bir enginlikte yayılma göstermemektedir.
\vs p070 10:3 İlkel insan olgular bütününün tamamını tek bir bireye yüklemiştir. Yaban bir bireyin ölümünde onu \bibemph{neyin} öldürdüğü yerine \bibemph{kimin} öldürdüğü sorulmuştur. Kaza eseri ölüm böylelikle tanınmadığı için, suçun cezalandırılmasında suçlunun temel gayesi bütünüyle göz ardı edilmiştir; yargı kararına açığa çıkan zarar karşılığında varılmıştır.
\vs p070 10:4 Diri diri yakarak cezalandırma bir dönem olağan bir uygulamaydı. Bu uygulama, Hammurabi ve Musa’yı içine alan bir biçimde birçok eski yönetici tarafından tanınmıştı; Musa, özellikle çok ciddi bir cinsel içeriğe sahip olan birçok suçun kazıklarda yakılarak cezalandırılmasını emretmiştir. Eğer “bir din mensubunun kızı” veya diğer bir önde gelen vatandaş açıkça fuhşa bulaşmışsa, bu kadını “ateş ile yakmak” İbrani âdetiydi.
\vs p070 10:5 Hayaletlerin sağlıkçılar ve din adamları vasıtasıyla adaleti yerine getirildiğine çok önceleri inanılmıştı; bu inanış, ilk suç belirleyicileri ve kanunu uygulayan görevlilerden oluşan toplum düzenlerini oluşturdu. Suçu tespit etmede onların öncül yöntemleri; zehir, ateş ve acının sınavlarından suçluları geçirmekti. Bu yabansı çetin sınavlar, suçun hükmüne karar vermede uygulanan gelişmemiş yöntemlerinden başka bir şey değildi; onlar bir anlaşmazlığı adil bir biçimde çözümlemek zorunluluğuna sahip değillerdi. Örneğin, zehir kullanıldığında eğer suçlanan kişi kusarsa onun suçsuz olduğuna inanılmaktaydı.
\vs p070 10:6 Eski Ahit bu çetin sınavlardan birisi olan bir evlilik suçunu açığa çıkarma yöntemini aktarmaktadır: Eğer bir erkek karısının onu aldattığından şüphe ederse, eşini din sorumlusuna götürüp kendisine şüphelerini aktarır; bunun sonrasında ilgili din mensubu kutsal sudan ve tapınak zemini tozundan meydana gelmiş bir karışımı hazırlar. Tehditkâr lanetleri içeren bir biçimde gereken ayin yapıldıktan sonra, suçlanan eş kötü zehri içmeye zorlanır. Eğer eş suçluysa, “kutsal su lanetin onun bünyesine girmesine ve daha keskin bir hale gelmesine neden olur; onun karın kısmı şişer ve kalçaları çürür; ve kadın böylece toplumu içinde lanetlenmiş bir hale gelir.” Eğer, nadiren bir kaza eseri, herhangi bir kadın bu çirkin karışımı yudumlar ve fiziksel hastalığın hiçbir belirtisini göstermezse; kıskanç kocası tarafından yapılan suçlamalardan aklanır.
\vs p070 10:7 Suçun tespit edilişine dair bu zalim yöntemler, neredeyse evrimleşen kabilelerin tümü tarafından bir biçimde uygulanmıştır. Düello, çetin sınavlar vasıtasıyla bireylerin yargılanmasının çağdaş dönemlere kadar gelmiş bir uygulamasıdır.
\vs p070 10:8 İbraniler ve diğer yarı\hyp{}medenileşmiş kabilelerin üç bin yıl önce adalet idaresinin bu türden ilkel yöntemlerini uygulamış olmalarında şaşılacak bir şey bulunmamaktadır; ancak bunların arasında en ilgi çekici olan şey, düşünen insanların kutsal yazıların bir derlemesine ait sayfalar içinde bu türden yabani uygulamaların bir kalıntısını tutmalarıdır. Sorgulayıcı düşünce; herhangi bir kutsal varlığın, şüphe duyulan evlik sadakatine dair suç tespiti ve hükmü ile ilgili bu türden haksız talimatları bir kere bile vermeyeceğinden emin olması gerekir.
\vs p070 10:9 Toplum öncül olarak, --- göze göz, yaşama yaşam şeklinde --- misillemenin bir intikam tavrını benimsemiştir. Evrimleşen kabilelerin tümü, kan intikamının bu hakkını tanımıştır. İntikam, ilkel yaşamın amacı haline gelmiştir; ancak din bahse konu dönemden bu yana, bu öncül kabile uygulamalarını büyük bir biçimde değişikliğe uğratmıştır. Açığa çıkarılmış dinin öğretmenleri her zaman, “Koruyucunun ‘İntikam bana aittir’ dediğini” bildirmişlerdir. Öncül zamanlar içinde intikam yoluyla öldürme, törelerin yalanları altında gerçekleştirilen bugünkü cinayetlerden çok da farklı değildi.
\vs p070 10:10 İntihar, misillemenin sıkça görülen bir türüydü. Eğer bir kişi yaşam sürecinde öcünü almaya yetkin değilse, bir hayalet olarak geri dönüp düşmanına gazabını göstereceği düşüncesiyle intihar ederdi. Ve bu düşünce oldukça genel bir kabul konumunda bulunduğu için, bir düşmanın kapısı önünde yapılan intihar tehdidi genellikle düşmanını yola getirmeye yeterdi. İlkel insan yaşamı çok kıymetli olarak görmedi; zorluk karşısında gerçekleştirilen intihar yaygındı; ancak Dalamatia unsurlarının öğretileri bu âdetin uygulanmasını giderek azaltırken, daha yakın zamanlarda boş zaman etkinlikleri, rahatlık sağlayan faaliyetler, din ve felsefe bütünleşerek yaşamı daha tatlı ve daha arzulanabilir kılmıştır. Açlık grevleri, buna rağmen, misillemenin bu eski yönteminin çağdaş bir örneğidir.
\vs p070 10:11 Gelişmiş kabile hukukunun en öncül tasarımlarından biri kan davasını bir kabile meselesi olarak üstlenmekti. Ancak her ne kadar benzerlik kurulması bakımından garip de olsa, bu dönemlerde bile bir erkek karısını tamamiyle başlık parası ödeyerek almışsa onu öldürdüğünde ceza almamaktaydı. Bugünün Eskimoları hala, buna rağmen, cinayet durumlarında bile bir suçun cezasına dair hükmü ve cezai yaptırımı haksızlığa uğramış aileye bırakmaktaydı.
\vs p070 10:12 Diğer bir gelişme, mal üzerinden ödenen ceza hükümleri biçiminde, tabulara karşı gelme durumları için maddi yaptırımların uygulanmasıydı. Bu maddi yaptırım cezaları, ilk kez kamu gelir düzenini oluşturmuştu. “Kan parası” ödeme uygulaması aynı zamanda, kan davalarının yerini alan gözde bir yöntem olmuştu. Bu türden hasarlar genellikle kadınlar veya büyük baş hayvanlar ile ödenmekteydi; parasal telafi biçimindeki bugünkü maddi yaptırımlardan çok daha uzun bir süre önce bu yöntem suçun cezası için belirlenmekteydi. Ve ceza fikri özü itibariyle tazminata dayandığı için, insan yaşamı dâhil olmak üzere her şey nihai olarak hasara karşılık gelen bir bedele sahip olur konuma gelmişti. İbraniler, kan parasını ödeme uygulamasını kaldıran ilk topluluktu. Musa, “ölümden suçlu olan bir katilin yaşamı üzerinden hiçbir çıkarın elde edilmemesi, bu katilin kesin bir biçimde öldürülmesi gerektiğini” Museviler’e öğretti.
\vs p070 10:13 Adalet böylece ilk başta aile, daha sonra kavim ve ondan sonra ise kabile tarafından dağıtılmıştır. Gerçek adaletin uygulanması, intikamın özel topluluklar ve kan bağı birlikteliklerinden alınıp devlet olarak toplumsal birliğin ellerine teslim edildiği zaman zarfından bugüne gelmektedir.
\vs p070 10:14 Diri diri yakarak cezalandırma bir dönem olağan bir uygulamaydı. Bu uygulama, Hammurabi ve Musa’yı içine alan bir biçimde birçok eski yönetici tarafından tanınmıştı; Musa, özellikle çok ciddi bir cinsel içeriğe sahip olan birçok suçun kazıklarda yakılarak cezalandırılmasını emretmiştir. Eğer “bir din mensubunun kızı” veya diğer bir önde gelen vatandaş açıkça fuhşa bulaşmışsa, bu kadını “ateş ile yakmak” Musevi âdetiydi.
\vs p070 10:15 “Satmak” veya bir kabilenin üyelerini aldatmak biçimindeki hıyanet, ilk ölüm cezası suçuydu. Büyük baş hayvan çalmak evrensel olarak doğrudan bir biçimde ve yargı kararı beklenmeden ölümle cezalandırılmaktaydı; yakın geçmişte bile at çalmak benzer bir biçimde cezalandırılmıştı. Ancak zaman geçtikçe cezanın şiddetine ek olarak onun kesinliği ve çabukluğunun suçun caydırıcılığında oldukça değerli bir etkiye sahip olmadığı öğrenildi.
\vs p070 10:16 Suçları cezalandırmada toplum başarısız olduğu zaman, topluluğun hıncı kendisini linç kültürü olarak göstermektedir; gözaltına alınmaya dair karar, bahse konu anlık topluluk öfkesinden kaçmanın bir aracıydı. Linç kültürü ve düellonun uygulanması, bireyin kendisine yapılan kişisel nitelikteki haksızlığın çözümünü devlet eline bırakmasındaki isteksizliği temsil etmektedir.
\usection{11.\bibnobreakspace Yasalar ve Mahkemeler}
\vs p070 11:1 Adetler ve yasalar arasında kesin çizgileri çizmek tıpkı gün doğumunda gecenin ne zaman günü takip ettiğini belirlemek gibi zordur. Adetler yazıya geçirildiklerinde kanunlar ve polis yönetmelikleridir. Varlıklarını uzun bir süre boyunca koruduğunda ismi konulmamış adetler, elle tutulur düzenlemeler şeklinde kesin kanunlara ek olarak oldukça iyi bir biçimde tanımlanmış toplumsal kabullere doğru yoğunlaşma eğilimi gösterirler.
\vs p070 11:2 Kanun ilk başta her zaman engelleyici ve yasaklayıcıdır; gelişme gösteren medeniyetlerde kanun artan bir biçimde olumlu ve yönlendiricidir. Öncül toplum bireye yaşam hakkını diğer tüm bireylere “öldürmeniz yasaktır” emrini dayatarak vermiştir. Bireye verilen hakların veya özgürlüğün her bir imtiyazı, diğer bireylerin tümünün özgürlüklerinde bir kısıtlamaya neden olmaktadır; ve bu durum, ilkel kanun olarak tabu tarafından belirlenmektedir. Tabunun bütüncül fikri, içkin olarak engelleyicidir, çünkü ilkel toplum, örgütlenmesi bakımından tamamen engelleyiciydi; ve adaletin öncül idaresi, tabuların uygulanmasından meydana gelmişti Ancak bu kurallar kökensel olarak, ayak takımından olanlarla ilişkilerinde farklı bir etik anlayışına sahip olmuş daha sonraki İbrani toplulukları tarafından sergilendiği gibi, yalnızca akran kabile üyelerine uygulanmaktaydı.
\vs p070 11:3 Şahitliği daha doğru kılma çabası olarak yemin, Dalamatia döneminde doğmuştur. Bu türden yeminler, kendisi üzerine bir lanet okuma ifadesinden oluşmuştu. Daha önceleri hiçbir birey, kendi özgün topluluğuna karşı şahitlik yapmazdı.
\vs p070 11:4 Suç, kabile tabularına karşı bir saldırıydı; günah, hayalet iznini memnuniyetle deneyimleyen bahse konu tabulara karşı gelmekti; ve orada, suç ile günahı birbirinden ayırt etme başarısızlığından doğan uzun süreli bir kafa karışıklığı yaşanmıştı.
\vs p070 11:5 Bireysel çıkar öldürme ile ilgili tabuyu oluşturmuş ve toplum bu türden bir yasaklamayı geleneksel adetler olarak kutsallaştırmışken; din toplumsal kabulü ahlaki yasa olarak kutsamış ve böylece onların üçü birden insan yaşamını daha güvenli ve daha kutsal kılmada birlik olmuşlardır. Toplum yasalara ve dinin yaptırımına sahip olmayan öncül dönemlerde bir arada tutulamazdı; hurafeler, uzun evrimsel dönemler boyunca ahlaki ve toplumsal polis kuvvetiydi; eski dönemlerde yaşayan bireylerin tümü, tabular olarak sahip oldukları tarihi kanunların atalarına tanrılar tarafından verilmiş olduğunu iddia etmişlerdi.
\vs p070 11:6 Yasa, kamuoyunun belirginleşmiş ve yasallaşmış hali olarak uzun süreler uygulanan insan deneyiminin yazıya geçirilmiş bir kaydıdır; adetler, daha sonra yöneten akılların oluşturdukları yazılı kanunlar şeklinde açığa çıkan, birikmiş deneyimin hammaddesidir. Eski dönemlerde bulunan hâkimler hiçbir kanuna sahip değillerdi. Bu hâkim bir karar vereceği zaman, yalnızca “adet böyledir” ifadesini kullanırdı.
\vs p070 11:7 Geçmişteki mahkeme kararlarına atıfta bulunma, hâkimlerin yazıya geçmiş hükümleri toplumun değişen koşullarına doğru uyarlama çabasını yansıtmaktadır. Bu uygulama, geleneğin devamının etkisiyle birleşen değişen toplumsal koşullara ilerleyici uyumu sağlamaktadır.
\vs p070 11:8 Özel mülkiyet sorunları şu gibi birçok şekilde çözümlenmekteydi:
\vs p070 11:9 1.\bibnobreakspace İtilaflı mülkiyeti yıkmak.
\vs p070 11:10 2.\bibnobreakspace Kuvvet kullanarak --- dava taraflarının fiziksel şiddete başvurması.
\vs p070 11:11 3.\bibnobreakspace Tahkim yoluyla --- üçüncü bir şahsın verdiği karar sonucu.
\vs p070 11:12 4.\bibnobreakspace İhtiyar heyetine yapılan başvuru --- daha sonrasında ise mahkemelere yapılan itiraz.
\vs p070 11:13 İlk mahkemeler yumruklu mücadeleyi düzenlemişti; hâkimler yalnızca müsabaka gözetmeni veya hakemler düzeyindeydi. Onlar, kavganın kabul edilmiş kurallara uygun gerçekleşip gerçekleşmediğini gözlemlemektelerdi. Bir mahkeme kavgasına girerken her taraf, rakibi tarafından yenilmesi durumunda ödemekle yükümlü olduğu masrafları ve cezaları hâkime önceden teslim etmekteydi. “Güçlü hala haklıydı.” Daha sonra sözlü münakaşalar fiziksel kavgaların yerini aldı.
\vs p070 11:14 İlkel adaletin bütüncül fikri, anlaşmazlığı gidermek ve böylece toplumsal kargaşayı ve bireysel şiddeti önlemek için adil olmayı başat bir biçimde temel almamaktaydı. Ancak ilkel insan, bugün haksızlık olarak addedilebilecek bir uygulama karşısında belirgin bir karşı çıkışı sergilememekteydi; güce sahip olanın bu gücü bencil bir biçimde kullanışı önemsenmiyordu. Yine de herhangi bir medeniyetin düzeyi, mahkemelerinin doğruluğu ve hakkaniyetine ek olarak hâkimlerinin dürüstlüğü ile oldukça kesin bir biçimde belirlenebilmektedir.
\usection{12.\bibnobreakspace Sivil Yönetimin Tahsisi}
\vs p070 12:1 Hükümetin evriminde büyük mücadele gücün toplanması ile ilgili olmuştur. Evren yöneticileri, yerleşik dünyalar üzerinde bulunan evrimsel toplulukların; oldukça iyi bir biçimde eş güdümsel hale getirilmiş yürütme, yasama ve yargı erkleri arasında doğru bir güç dengesi sağlandığında sivil hükümetin temsili türü tarafından en iyi bir şekilde idare edildiklerini deneyimleri vasıtasıyla öğrenmişlerdir.
\vs p070 12:2 İlkel yönetim, fiziksel güç olarak kuvvete dayalıyken; olası en yüksek hükümet olan temsili düzen içinde önderlik kabiliyete bağlıdır. Ancak yaban hayatın hüküm sürdüğü dönemlerde tamamiyle, temsili hükümetin etkin bir biçimde faaliyet göstermesine imkân vermeyecek derecede çok savaş gerçekleşmekteydi. Yönetim gücünün bölünmesi ve idarenin tek elde toplanması arasında gerçekleşen uzun mücadeleden diktatör galip çıkmıştı. İhtiyar heyetinin ilkel topluluğuna ait öncül ve paylaştırılmış güçler kademeli olarak mutlak monarşiyi elinde bulunduran bireyde toplanmıştı. Gerçek kralların ortaya çıkmasından sonra ihtiyar heyeti toplulukları, aslında var olmayan yasama ve yargı erkleri için danışma kurumları olarak varlıklarını sürdürmüştür; daha sonra eş güdümsel düzeyde yasama organları ortaya çıkmış, ve nihai olarak yargının en yüksek temyiz mahkemeleri bu organlardan bağımsız olarak kurulmuştur.
\vs p070 12:3 Kral, kökensel veya diğer bir değişle yazılmamış kanun biçimindeki örf ve adetlerin uygulayıcısıydı. Daha sonra kral, kamuoyunun belirginleşmiş hali olan yasama hükümlerini uyguladı. Kamuoyunun bir dışavurumu olarak bir genel meclis, her ne kadar yavaş yavaş ortaya çıkmış olsa da, büyük bir toplumsal gelişimi simgelemişti.
\vs p070 12:4 Öncül krallar büyük bir ölçüde --- gelenek veya kamuoyu biçimindeki --- adetler tarafından sınırlanmış bir haldeydi. Daha yakın zamanlarda bazı Urantia milletleri, bu adetleri hükümet yönetimi için yazılı düzene geçirmişlerdir.
\vs p070 12:5 Urantia fanilerinin özgür olmaları onların hakkıdır; onlar kendilerine ait hükümet yönetim düzenlerini yaratmalılardır; onlar, kendilerine ait anayasaları veya sivil yönetimin ve idari işleyiş düzeninin diğer tüzüklerini oluşturmalılardır. Ve bunu gerçekleştirdiklerinde onlar, baş yöneticiler olarak en yetkin ve en liyakat sahibi akranlarını seçmelilerdir. Yasama erkinde görev yapacak temsilciler için onlar sadece, bu türden kutsal görevleri yerine getirmek için ussal ve ahlaki olarak yetkin bireyleri seçmelilerdir. Yüksek ve en yüksek temyiz mahkemelerine atanacak hâkimler makamına yalnızca, doğal yetkinlik ile bahşedilmiş ve yeterli deneyimle bilge haline gelmiş kişiler seçilmelidir.
\vs p070 12:6 Eğer insanlar özgürlüklerini idame ettireceklerse, özgürlük sözleşmelerini tercih ettikten sonra, şu gibi şeylerin önlenmesi için onun bilge, ussal ve korkusuz yorumunu sağlamak zorundadırlar:
\vs p070 12:7 1.\bibnobreakspace Yönetim veya yargı erklerinin herhangi bir tarafından kanuna aykırı gücün elde edilmesi.
\vs p070 12:8 2.\bibnobreakspace Cahil ve hurafelere inanan tahrikçilerin fesatlıkları.
\vs p070 12:9 3.\bibnobreakspace Bilimsel gelişmeyi duraklatma.
\vs p070 12:10 4.\bibnobreakspace Sıradanlığın baskınlığının yarattığı açmaz.
\vs p070 12:11 5.\bibnobreakspace Fesat azınlıkların baskınlığı.
\vs p070 12:12 6.\bibnobreakspace Gelecekte diktatör olabilecek hırslı ve zeki kişilerin denetimi.
\vs p070 12:13 7.\bibnobreakspace Telaşın gidişatı yıkıcı bir biçimde bozması.
\vs p070 12:14 8.\bibnobreakspace Vicdansız kişiler tarafından sömürü.
\vs p070 12:15 9.\bibnobreakspace Devlet tarafından tüm vatandaşlığın vergilerle köleleştirilmesi.
\vs p070 12:16 10.\bibnobreakspace Toplumsal ve mali adaleti sağlayamama.
\vs p070 12:17 11.\bibnobreakspace Din ve devletin birleşmesi.
\vs p070 12:18 12.\bibnobreakspace Kişisel özgürlüğün kaybı.
\vs p070 12:19 Bahse konu bu durumlar, evrimsel bir dünya üzerinde temsili hükümetin çarkları üzerinde baş yöneticiler olarak faaliyet gösteren anayasa mahkemelerinin var oluş nedenleri ve gayeleridir.
\vs p070 12:20 İnsanlığın Urantia üzerindeki kusursuz hükümeti oluşturma mücadelesi; idare zincirini kusursuzlaştırmakla birlikte onu sürekli değişen mevcut ihtiyaçlara göre uyarlamakla, hükümet içindeki güç dağılımını geliştirme ve bunun sonrasında gerçekten bilge kişileri idari önderler olarak seçmekle ilgilidir. Her ne kadar orada hükümetin kutsal ve olası en yüksek türü mevcut bulunsa da bu türden bir hükümet açığa çıkarılamaz; ancak bu hükümet, zaman ve mekânın evrenleri boyunca her gezegenin erkek ve kadınları tarafından yavaşça ve emek sarf edilerek keşfedilmelidir.
\vs p070 12:21 [Nebadon’un bir Melçizedek unsuru tarafından sunulmuştur.]
