\upaper{78}{Âdem Döneminden Sonra Eflatun Irkı}
\vs p078 0:1 İkinci Cennet Bahçesi, neredeyse otuz bin yıl boyunca medeniyetin beşiğiydi. Mezopotamya içinde burada Âdem unsurları, dünyanın en uç noktalarına doğumlarını göndererek üstünlüklerini korudular; ve daha sonra Nod ve Sangik kabileleri ile bütünleşen bir biçimde onlar, And unsurları olarak tanındılar. Tarihi dönemlerin etkinliklerini başlatan ve Urantia üzerindeki kültürel ilerleyişi devasa bir biçimde hızlandıran bu erkek ve kadınlar bahse konu yerleşkeden gelmişlerdir.
\vs p078 0:2 Bu makale; yaklaşık M.Ö. 35.000 yılında Âdem’in görevindeki başarısızlığından hemen sonra başlayan, Nod’unkilere ek olarak Sangik ırkları ile karıştıkları yıl olan yaklaşık M.Ö. 15.000 boyunca devam eden ve yaklaşık M.Ö. 2000 yıl önce gerçekleşmiş And insan topluluklarının kurulduğu ve Mezopotamya’daki ana yerleşkelerinden sonsuza kadar ayrıldıkları tarihle sonlanan eflatun ırkının gezegensel hikâyesini tasvir etmektedir.
\usection{1.\bibnobreakspace Irksal ve Kültürel Dağılım}
\vs p078 1:1 Her ne kadar ırkların akıl ve ahlakları Âdem’in varış döneminde düşük bir düzeyde bulunmuş olsa da, fiziksel evrim Caligastia isyanının doğrudan zararlarından büyük ölçüde etkilenmeden ilerleyişini sürdürdü. Her ne kadar girişiminde kısmi bir biçimde başarısız olsa da, Âdem’in ırkların biyolojik düzeyine olan katkısı Urantia insan topluluklarını devasa bir biçimde üst düzeye yükseltmiştir.
\vs p078 1:2 Âdem ve Havva aynı zamanda; insan türünün toplumsal, ahlaki ve ussal ilerleyişi için önem taşıyan şeyleri gerçekleştirerek onlara büyük bir katkıda bulunmuştur; medeniyet, onların doğumlarının mevcudiyeti vasıtasıyla çok büyük bir ölçekte hızlanma göstermiştir. Ancak otuz beş bin yıl önce dünyanın büyük bir kesimi çok az bir kültürü bünyesinde barındırmaktaydı. Medeniyetin belirli merkezleri belli başlı birkaç yerde var olmuştu; ancak Urantia’nın büyük bir kısmı yabansı yaşamdan bitap düşmüştü. Irksal ve kültürel dağılımı şu biçimdeydi:
\vs p078 1:3 1.\bibemph{ Eflatun ırkı --- Âdem unsurları ve Âdemoğlu toplulukları}. Âdemsel kültürün başlıca merkezi, Fırat ve Dicle nehirlerinin oluşturduğu üçgen içinde konumlanan ikinci bahçe içerisindeydi; burası gerçekten de Batı ve Hint medeniyetlerinin beşiğiydi. Eflatun ırkının ikincil veya diğer bir değişle kuzey merkezi, Kopet dağları yakınındaki Hazar Gölü’nün güney kıyısının doğusuna doğru konumlanan bir biçimde Âdemoğlu ana yönetim yerleşkesiydi. Bu iki merkezden çevre yerleşkelere, ırkların tümünü tamamiyle anlık olarak hızlandırmış kültür ve yaşam plazması yayılmıştı.
\vs p078 1:4 2.\bibnobreakspace \bibemph{Sümer\hyp{}öncesi topluluklar ve Nod unsurları}. Orada aynı zamanda, ırmak ağızlarının bulunduğu yer yakınında Dalamatia dönemine ait tarihi kültürlerin kalıntıları da mevcuttu. İlerleyen bin yıllar boyunca bu topluluk kuzeydeki Âdem unsurları ile bütünüyle karışmış hale geldi; ancak onlar, Nod geleneklerini hiçbir zaman bütünüyle kaybetmediler. Doğu Akdeniz yerleşkeleri içinde ikamet eden çeşitli diğer Nod toplulukları, genel olarak, daha sonraki dönemlere ait genişleyen eflatun ırkına karışarak kimliklerini yitirdiler.
\vs p078 1:5 3.\bibnobreakspace \bibemph{Andon unsurları}, Âdemoğlu yönetim merkezinin kuzeyinde ve doğusunda bulunan beş ila altı tane çoğunlukta bulundukları temsili yerleşkeyi idare ettiler. Onlar da, ayrık topluluklar halinde Avrasya boyunca özellikle dağlık bölgelerde varlıklarını sürdürerek, Türkistan boyunca dağılmış bir haldelerdi. Bu yerliler hala, İzlanda ve Grönland’a ek olarak Avrasya kıtasının kuzey bölgelerini ellerinde bulundurmaktaydılar; ancak onlar uzunca bir süre öncesinden beri, Avrupa düzlüklerinden mavi ırk tarafından ve uzak Asya’nın nehir vadilerinden genişleyen sarı ırk tarafından uzaklaştırılmış bir konumda bulunmaktaydılar.
\vs p078 1:6 4.\bibnobreakspace \bibemph{Kırmızı ırk}, Âdem’in varışından elli bin yıldan fazla bir süre önce Asya’dan uzaklaştırılmış bir halde Amerika kıtalarında ikamet etmekteydiler.
\vs p078 1:7 5.\bibnobreakspace \bibemph{Sarı ırk}. Çin insan toplulukları, doğu Asya’nın denetiminde oldukça hâkim bir halde bulunmaktaydılar. Onların en gelişmiş yerleşim birimleri, Tibet’e komşu bölgeler içindeki bugünün Çin sınırlarının kuzeybatısında konumlanmıştı.
\vs p078 1:8 6.\bibnobreakspace \bibemph{Mavi ırk}. Mavi insanlar tüm Avrupa’ya yayılmışlardı; ancak onların daha yoğun kültür merkezleri bu dönemde, Akdeniz havzası ve kuzeybatı Avrupa’nın verimli vadilerinde konumlanmıştı. Neanderthaller’e olan baskın karışım, mavi ırkın kültürünü büyük ölçüde geriletmiş bir haldeydi; ancak buna rağmen bu ırk, Avrasya’nın evrimsel insan toplulukları içinde en savaşçı, maceraperest ve keşfedici birliktelikti.
\vs p078 1:9 7.\bibnobreakspace \bibemph{Dravid\hyp{}öncesi Hindistan}. Dünya üzerindeki her ırkı içine alan biçimde fakat özellikle yeşil, turuncu ve siyah olanların baskınlığında --- Hindistan içindeki ırkların karmaşık birlikteliği, diğer bölgelerde sahip olunanın biraz daha üst seviyesinde bulunan bir kültürü barındırdı.
\vs p078 1:10 8.\bibnobreakspace \bibemph{Sahara medeniyeti}. Çivit ırkının üstün kolları, şu an büyük Sahara çölü olarak adlandırılan bölge içerisinde en gelişmiş yerleşkelerine sahip oldular. Çivit\hyp{}siyah ırk topluluğu, sular altında kalmış turuncu ve yeşil ırkların geniş ırk kollarını taşımışlardı.
\vs p078 1:11 9.\bibnobreakspace \bibemph{Akdeniz Havzası}. Hindistan’ın dışarısında bulunan birbirine en çok karışmış ırk, bugün Akdeniz havzası olarak bilinen bölgede ikamet etmekteydi. Burada, kuzeydeki mavi insanlar ile güneydeki Sahara toplulukları buluşmuş ve doğudan katılan Nod ve Âdem unsurları ile karışmışlardı.
\vs p078 1:12 Bu anlatım, yaklaşık yirmi beş bin yıl öncesi olarak, eflatun ırkının büyük genişlemesinin başlangıcından önceki dünya resmiydi. Gelecek medeniyet ümidi, Mezopotamya ırmakları arasında ikinci bahçe içinde yatmaktaydı. Burada Güneybatı Asya içerisinde, Dalamatia zamanından ve Cennet Bahçesi döneminden kalabilmiş dünya düşüncelerinin ve olası en yüksek amaçlarının yayılma imkânı olarak büyük bir medeniyet olasılığı bulunmaktaydı.
\vs p078 1:13 Âdem ve Havva, kısıtlı ama yetkin bir soyu gerilerinde bırakmıştı; ve Urantia üzerindeki göksel gözlemciler tedirgin bir biçimde, hatalı Maddi Erkek ve Kız Evlat’ın bu soylarının kendilerini nasıl aklayacaklarını görmeye koyulmuşlardı.
\usection{2.\bibnobreakspace İkinci Bahçe İçindeki Âdem Unsurları}
\vs p078 2:1 Binlerce yıl süresince Âdem’in çocukları; güneydeki tarım ve sel\hyp{}denetim sorunlarını çözerek, kuzeydeki savunma düzenlerini kusursuzlaştırarak ve ilk Cennet Bahçesi’nin ihtişamından kalan geleneklerini korumaya çabalayarak Mezopotamya ırmakları boyunca emek vermişlerdir.
\vs p078 2:2 İkinci bahçenin önderliğinde sergilenen kahramanlık, Urantia tarihinin hayranlık uyandırıcı ve ilham verici destanlarından bir tanesini oluşturmaktadır. Bu muhteşem ruhlar, Âdemsel görevin amacı bir kez olsun dahi unutmadılar; ve böylelikle onlar, dünya ırklarına elçiler olarak düzenli bir akış içerisinde en seçkin erkek ve kız evlatlarını memnuniyetle gönderirken çevrelerinde bulunan ve alt düzeydeki kabilelerin etkilerine karşı cesurca göğüslerini siper etmişlerdir. Zaman zaman bu genişleme ana kültürleri için gerilemeyi beraberinde getirmekteydi; ancak bu üstün insan toplulukları her zaman kendilerini iyileştirmektelerdi.
\vs p078 2:3 Âdem unsurlarının medeniyet, toplum, ve kültürel düzeyi Urantia’nın evrimsel ırklarının genel seviyesinin çok üstündeydi. Yalnızca Van ve Amadon’a ek olarak Âdemoğlu unsurlarının eski yerleşkeleri arasında herhangi bir biçimde karşılaştırabilecek bir medeniyetin varlığı mevzu bahisti. Ancak ikinci Cennet Bahçesi’nin medeniyeti yapay bir yapıydı --- \bibemph{bu medeniyet evrimleşerek bu düzeye ulaşmamıştı} --- ve bu nedenle olağan bir evrimsel düzeye ulaşana kadar kötüleşmeye mahkûmdu.
\vs p078 2:4 Âdem ardında büyük bir ussal ve ruhsal kültürü bırakmıştı; ancak bu kültür mekanik araçlar bakımından gelişmemişti, çünkü her medeniyet, mevcut doğal kaynaklar, içkin ussal yetkinlik ve yaratıcı üretimi beraberinde getirecek yeterli düzeydeki dinlence ile sınırlıdır. Eflatun ırkının medeniyeti, Âdem’in mevcudiyetine ve ilk Cennet Bahçesi’nin geleneklerine dayanmaktaydı. Âdem’in ölümünden sonra ve bu geleneklerin her geçen bin yıl boyunca etkisinin azalma göstermesiyle beraber, Âdem unsurlarının kültürel düzeyi çevre insan toplulukları ve eflatun ırkının doğal bir biçimde evrimleşen kültürel yetkinlikleri ile birlikte karşılıklı bir dengeye erişene kadar sürekli olarak gerileme göstermiştir.
\vs p078 2:5 Ancak Âdem unsurları M.Ö. 19.000 yılı civarında, nüfusu dört buçuk milyona erişen bir biçimde gerçek bir milletti; ve onlar hali hazırda, çevre insan topluluklarına doğumlarının dört milyonluk nüfusunu göndermiş bir konumdaydılar.
\usection{3.\bibnobreakspace Âdem Unsurlarının Öncül Genişlemesi}
\vs p078 3:1 Eflatun ırkı, birçok bin yıl boyunca barışı amaçlayan Cennet Bahçesi geleneklerini sürdürmüştü; bu durum gerçekleştirdikleri toprak fetihleri arasındaki uzun süreli gecikmeyi açıklamaktadır. Onlar nüfus fazlalığından olumsuz bir biçimde etkilendikleri zaman, savaşta bulunup daha fazla toprağı ele geçirmek yerine sakinlerini diğer ırklara öğretmenler olarak göndermişlerdi. Bu öncül göçlerin kültürel etkisi kalıcı değildi; ancak Âdem unsurlarına ait öğretmenler, tüccarlar ve kâşiflerin çevre topluluklara olan karışımı onlar için biyolojik düzeyde canlandırıcı bir etkiye sahipti.
\vs p078 3:2 Âdem unsurlarının bazıları öncül bir biçimde, Nil vadisine doğru batı yönünde ilerledi; diğerleri ise Asya’ya doğru doğu yönünde mesafe kat ettiler, ancak bu unsurlar bir azınlık topluluğuydu. Daha sonraki dönemlerin geniş çaplı göçleri yaygın bir biçimde kuzeye ve buradan batıya doğru gerçekleşmişti. Çoğunlukla bu göçler kademeli olarak ortaya çıkmaktaydı; ancak onlar, daha fazla sayıdaki bireyin kuzeye hareketi ve daha sonra Hazar Gölü’nü etrafından dolaşarak Avrupa’ya doğru batı yönünde ilerlemesi biçiminde kademeli fakat sürekli bir biçimde kuzey yönünde gerçekleşen bir göçtü.
\vs p078 3:3 Yaklaşık yirmi beş bin yıl önce Âdem unsurlarının saf soylarının birçoğu tamamiyle kuzey göç yolu üzerinde bulunmaktalardı. Ve kuzey yönünde ilerlerken, Türkistan’ı fethettikleri dönem civarında özellikle Nod unsurları olarak diğer ırklar ile bütünüyle karışana kadar giderek azalan bir biçimde Âdemsel kaldılar. En başından beri saf ırk koluna ait eflatun topluluklarının çok azı Avrupa veya Asya’nın derinliklerine doğru ilerlemişti.
\vs p078 3:4 M.Ö. yaklaşık 30.000 ile 10.000 yılları arasında güneybatı Asya’nın bütününde çığır açıcı ırksal birliktelikler meydana gelmekteydi. Türkistan’ın dağ sakinleri yiğit ve cebbar bir insan topluluğuydu. Hindistan’ın kuzeybatısına uzanan bir biçimde Van döneminin kültürü varlığını devam ettirmişti. Ve kültür ve kişiliğin bu üstün iki ırkı da, kuzeye doğru hareket eden Âdem unsurlarının baskınlığında onlara karışmıştır. Bu birleşim, birçok yeni düşüncenin benimsenmesiyle sonuçlanmıştır; bu oluşum medeniyetin ilerleyişini kolaylaştırmış olup sanat, bilim ve toplumsal kültürün tüm fazlarını büyük ölçüde ilerletmiştir.
\vs p078 3:5 Öncül Âdemsel göçler dönemi yaklaşık olarak M.Ö. 15.000 yılında sona erdiğinde, Avrupa ve merkezi Asya içinde Mezopotamya’dan bile daha fazla sayıda Âdem soyu hali hazırda ikamet etmekteydi. Avrupalı mavi ırk, büyük ölçüde çözülmüştü. Bugün Rusya ve Türkistan olarak adlandırılan yerleşkeler; Nod, Andon unsurlarına ek olarak kırmızı ve sarı Sangikler’e karışan Âdem insanlarının büyük bir nüfusu tarafından güney istikametleri boyunca iskân edilmişti. Güney Avrupa ve Akdeniz havzası, Âdem ırk kolunun az sayıdaki bir topluluğuna ek olarak --- turuncu, yeşil ve çivit ırkları biçimindeki --- Andon ve Sangik insan topluluklarının karma bir ırkı tarafından iskân edilmekteydi. Ön Asya’ya ek olarak merkez ve doğu Avrupa yerleşkeleri, başat olarak Andon unsurlarından meydana gelen kabileler tarafından tutulmaktaydı.
\vs p078 3:6 Yaklaşık olarak bu dönemde Mezopotamya’dan gelenler tarafından büyük ölçüde güçlenmiş olan birbirine karışmış haldeki renkli bir ırk, Mısır’ı elinde bulundurup Fırat vadisinin yok olmaya yüz tutmuş kültürünü ele geçirmeye hazırlanmıştı. Siyah insan toplulukları Afrika içerisinde daha da güneye doğru ilerlemekte olup, kırmızı insanlar gibi neredeyse tecrit altında bulunmuş bir haldelerdi.
\vs p078 3:7 Sahara medeniyeti, kıtlık tarafından sekteye ve Akdeniz havzasının sebebiyet verdiği sel nedeniyle sekteye uğramış bir haldeydi. Mavi ırklar yine de, gelişmiş bir kültürü geliştirmekte başarısız oldular. Andon unsurları hala, Kuzey Kutbu ve merkezi Asya bölgelerine dağılmış bir haldeydi. Yeşil ve turuncu ırklar benzer bir biçimde yok olmuşlardı. Çivit ırkı, yavaş ancak uzun süren devamlı bir ırksal bozulmayı deneyimlemeye başlayacakları yer olan merkezi Asya’daki konumlarını güçlendirmektelerdi.
\vs p078 3:8 Hindistan’ın insan toplulukları ilerlemeyen bir medeniyet ile birlikte durağan bir konumda bulunurken, sarı insanlar merkezi Asya’daki elde ettikleri konumları sağlamlaştırmaktalardı; kahverengi ırk Büyük Okyanus’un yakın adaları üzerindeki medeniyet süreçlerine henüz giriş yapmamışlardı.
\vs p078 3:9 Baskın mevsim değişiklikleri etkisiyle de belirlenmiş olan bu ırk dağılımı, Urantia medeniyetinin And toplulukları döneminin başlangıcı için dünya zemini hazırladı. Bu öncül göçler, M.Ö. 25.000 ile 15.000 yıl arasında olmak üzere on bin yıllık bir süreçten daha fazlası bir döneme uzanmıştır. Daha sonraki göçler veya diğer bir değişle And göçleri, yaklaşık M.Ö. 15.000 ile 6.000 arasındaki bir süreci kaplamıştır.
\vs p078 3:10 Âdem unsurlarının öncül dalgalarının Avrasya üzerinden geçişleri o kadar uzun bir sürede gerçekleşti ki kültürleri geçiş sürecinde büyük ölçüde yok oldu. Sadece daha sonraki And unsurları, Mezopotamya’nın hangi ücra köşesinden olursa olsun Cennet Bahçesi kültürünü ellerinde bulundurmak için yeterli hızda göçlerini gerçekleştirmişti.
\usection{4.\bibnobreakspace And Unsurları}
\vs p078 4:1 And ırkları, evrimsel insanlara ek olarak eflatun ırkının saf ırk kolu üyeleri ve Nod unsurlarının başat karışımlarıydılar. And unsurlarının modern ırklara kıyasla Âdem kanının çok daha büyük bir oranını genel olarak taşımakta oldukları bilinmelidir. And unsurları terimi çoğunlukla, ırksal kökeni sekizde bir ila altı da bir arasında eflatun kalıtımından gelen insanları tanımlamak için kullanılır. Çağdaş Urantia unsurları, hatta kuzeydeki beyaz ırklar bile, Âdem kalıtımının bu oranının çok daha azını taşımaktadırlar.
\vs p078 4:2 Öncül And insan toplulukları; yirmi beş bin yıldan çok daha uzun bir süre önce Mezopotamya’ya komşu olan bölgelerde doğmuş olup, Âdem ve Nod unsurlarının bir karışımından meydana geldiler. İkinci bahçe, yok olmaktaki eflatun kökenine ait merkezi ortak farklı yerleşkeler tarafından çevrilmişti; ve bu ırksal kaynaşma merkezinin çevre kısımlarında And ırkı doğmuştu. Daha sonra, göç halindeki Âdem ve Nod unsurları Türkistan’ın bu dönemdeki verimli bölgelerine girdiklerinde, yakın zaman içerisinde buranın üst düzeyde bulunan sakinleriyle karışmışlardı; ve bunun sonucunda ortaya çıkan ırk karışımı, And türünü kuzey doğrultusunda genişletti.
\vs p078 4:3 And unsurları her bakımdan, eflatun insanlarının saf nesillerinin yaşadığı dönemden beri Urantia üzerinde ortaya çıkmış en iyi ırk koluydu. Onlar; Âdem ve Nod ırklarının geride kalan hayattaki üyelerinin en yüksek türlerinin çoğuna ek olarak daha sonraki dönemlerin sarı, mavi ve yeşil insanlarının en iyi ırk kollarının bazılarından meydana gelmişlerdi.
\vs p078 4:4 Bu öncül And unsurları Aryan ırk koluna ait değillerdi; onlar bu ırk kolunun öncül üyeleriydiler. Onlar beyaz değillerdi; onlar beyaz öncesi ırkın üyeleriydiler. Onlar ne Batı ne de Doğu insanlarıydı. Ancak And kökeni; Avrupalı olarak tanımlanan türdeşliği genelleştiren, tarafınızdan adlandırılmış beyaz ırkların çok dilli karışımına kaynaklık sağlamaktadır.
\vs p078 4:5 Eflatun ırkının daha saf ırk kolları Âdem’in barışı arzulayan geleneğini korumaya devam etmiş bir halde bulunmaktalardı; bu durum, daha önceki ırk hareketlerinin neden daha çok barışçıl göçler şeklinde gerçekleştiğini açıklamaktadır. Ancak And unsurları bu dönemde kavgacı bir ırk haline gelmiş Nod ırk kollarıyla birleştiğinde, onların And soyları kendi dönemlerinde Urantia üzerinde yaşamış en usta ve akıllı askerler haline gelmişti. Bu dönemden itibaren Mezopotamya unsurlarının göçleri artan bir biçimde askeri içerik kazanıp, mevcut fetihlere daha çok benzer bir hal almıştı.
\vs p078 4:6 Bu And unsurları maceraperestlerdi; onlar kürek çekme eğilimine sahiplerdi. Kalıtımlarındaki Sangik veya Andon ırk etkisindeki bir artış onları bu eğilimden uzaklaştırıp olağan bir konuma getirmişti. Ancak böyleyken bile onların daha sonraki soyları, dünya etrafını dolaşana ve en uzak kıtayı keşfedene kadar bu isteklerinden hiçbir şekilde vazgeçmemişlerdi.
\usection{5.\bibnobreakspace And Göçleri}
\vs p078 5:1 Yirmi bin yıl boyunca ikinci cennet bahçesi kültürü varlığını sürdürmeye devam etmişti; ancak bu kültür, Seth din adamlığı yenilendiğinde ve Amosad’ın önderliği parlak bir döneme giriş yaptığında, yaklaşık olarak M.Ö. 15.000’lere kadar düzenli bir gerileme süreci deneyimledi. Daha sonra Avrasya’nın tümüne yayılan medeniyetin büyük dalgaları, And topluluğunu meydana getiren komşu melez Nod unsurları ile Âdem insanlarının geniş ölçüdeki birlikteliği sonucunda gerçekleşmiş Cennet Bahçesi’nin büyük rönesansından hemen sonra ortaya çıktı.
\vs p078 5:2 Bu And unsurları, Avrasya ve Kuzey Afrika boyunca yeni gelişmeleri başlattı. Mezopotamya’dan Doğu Türkistan boyunca And unsurlarının kültürü baskın bir haldeydi; ve Avrupa’ya yapılan düzenli göç hareketi, Mezopotamya’dan gelen yeni üyeler tarafından sürekli bir biçimde telafi edilmekteydi. Ancak Âdem’in melez soylarının dönemsel göçlerinin yakın sürede başladığı vakte kadar, And unsurlarından Mezopotamya’da ikamet eden bütüncül bir ırk olarak bahsetmek doğru olmaz. Bu dönemde ikinci bahçe içerisindeki ırklar bile öyle bir düzeyde birbirine karışmış haldeydi ki, onlar artık Âdem unsurları olarak görülememekteydi.
\vs p078 5:3 Türkistan’ın medeniyeti, özellikle daha sonraki And atlıları olmak üzere Mezopotamya’dan yeni gelenler tarafından sürekli bir biçimde dirilmekte ve canlanmaktaydı. Hint\hyp{}Avrupa dili olarak adlandırdığınız lisan, Türkistan’ın dağlık bölgelerinde oluş süreci içerisindeydi; bu dil, Âdemoğlu ve daha sonraki And unsurlarının dili ile birlikte bölgenin Andon lehçesinin bir karışımıydı. Birçok çağdaş dil; Avrupa ve Hindistan’a ek olarak Mezopotamya düzlüklerinin kuzey kuşağını elinde bulundurmuş bu merkezi Asya kabilelerinin öncül dilinden türemiştir. Bu tarihi dil Batı dillerine, Ari olarak adlandırılan bütüncül benzerlik temelini sağlamıştı.
\vs p078 5:4 M.Ö. 12.000’lerde dünya And ırk kolunun üçte biri kuzey ve doğu Avrupa’da ikamet etmekteydi; ve Mezopotamya’dan yapılan ilerideki en son toplu göç gerçekleştiğinde bu hareketin son dalgalarının yüzde altmış beşi Avrupa’ya giriş yapmıştı.
\vs p078 5:5 And unsurları sadece Avrupa’ya değil kuzey Çin ve Hindistan’a da göç etmişlerdi; bunun yanı sıra birçok topluluk din elçileri, öğretmenler ve tüccarlar olarak dünyanın ücra köşelerine gitmektelerdi. Onlar dikkate değer bir ölçüde Sahara Sangik insanlarının kuzey topluluklarına katkıda bulunmuşlardı. Ancak en başından beri çok az sayıdaki öğretmen ve tüccar Afrika’da, Nil ırmak kaynaklarının ötesinde doğuya doğru hareket etmişti. Daha sonra melez And unsurları ve Mısırlılar, ekvator seviyesinin çok altında Afrika’nın doğu ve batı kıyılarının güneye uzanan sahillerini izlediler; ancak onlar Madagaskar’a ulaşamadılar.
\vs p078 5:6 Bu And unsurları, Hindistan’ın --- adlandırmış olduğunuz --- Dravid ve --- daha sonraki --- Ari fatihleriydiler; ve onların merkezi Asya’daki mevcudiyeti, Turan topluluklarının atalarını büyük ölçüde canlandırmıştır. Bu ırkın birçok üyesi Doğu Türkistan ve Tibet üzerinden Çin’e hareket edip, daha sonraki Çin ırk koluna arzu edilen nitelikleri kazandırdı. Zaman zaman küçük topluluklar Japonya, Formoza, Batı Hint Adaları’na ek olarak --- her ne kadar çok azı sahil yoluyla olsa da --- güney Çin’e ulaştı.
\vs p078 5:7 Bu ırkın yüz otuz iki üyesi Japonya’dan, küçük teknelerden bir araya gelmiş bir donanmayla yola çıkarak; nihai olarak Güney Amerika’ya varıp, Ant toplulukları yerlileri ile karışarak İnkalar’ın daha sonraki idarecilerinin kökenini oluşturmuşlardı. Onlar, karşılaştıkları birçok adada konaklayarak rahat geçen aşamalar sonucunda Büyük Okyanusu kat ettiler. Polinezya topluluk adaları bugünkünden daha fazla sayıda ve daha büyüktü; ve bu And denizcileri, kendilerini takip eden bazı diğer ırk üyeleri ile birlikte, geçiş sürecinde karşılaştıkları yerli toplulukları biyolojik olarak dönüşüme uğrattı. Medeniyetin gelişmekte olan birçok merkezi, And hareketinin bir sonucu olarak şu an sular altındaki adalarda yeşermişti. Paskalya Adası uzunca bir süre boyunca, bu kaybedilen topluluğun bir tanesinin dini ve idari bir merkeziydi. Ancak en başından beri, uzunca bir süredir Büyük Okyanus üzerinde seyahat eden And unsurlarının sadece yüz otuz ikisi Kuzey ve Güney Amerika’nın merkez bölgelerine ulaşabilmişti.
\vs p078 5:8 And unsurlarının göç fetihleri, M.Ö. 8.000 ile 6.000 yılları arasında, son dağılımlarına kadar devam etti. Mezopotamya’dan büyük topluluklar halinde dağılırlarken, bir yanda çevre insan topluluklarını dikkate değer bir biçimde güçlendirirken diğer bir yanda anavatanlarındaki biyolojik kaynağı sürekli bir biçimde azaltmaktalardı. Ve seyahat ettikleri her millete mizah, sanat, macera, müzik ve üretim kazandırmışlardır. Kısa bir süreliğine olsa da onların mevcudiyeti en azından, daha eski ırkların dini inançlarını ve ahlaki uygulamalarını sıklıkla geliştirmişti. Ve böylelikle Mezopotamya kültürü yavaşça; Avrupa, Hindistan, Çin, Kuzey Afrika ve Büyük Okyanus adalarının tamamına yayılmıştı.
\usection{6.\bibnobreakspace Son And Göçleri}
\vs p078 6:1 And unsurlarının son üç dalgası Mezopotamya’dan M.Ö. 8000 ile 6.000 yılları arasında büyük topluluklar halinde gerçekleşmişti. Bu üç büyük kültür dalgası; tepe kabilelerinin baskısı sonucunda doğu, düzlük sakinlerinin tacizleri sonucunda batı yönünde Mezopotamya’nın dışına itilmişti. Fırat nehri vadisi ve onun bitişiğindeki sakinler, birkaç doğrultuda gerçekleştirmiş oldukları en son toplu göçlerinde ilerlediler:
\vs p078 6:2 Onların yüzde altmış beşi onu elde etmek ve --- mavi insanların ve daha önceki And unsurlarının karışımı olan --- yeni yeni beliren beyaz ırklara karışmak için Avrupa’ya girdiler.
\vs p078 6:3 Seth din adamlığına ait bir topluluğa ek olarak onların yüzde onu, Elam yükseltileri boyunca İran yüksek düzlükleri ve Türkistan’a doğru doğu yönünde hareket etti. Onların soylarının birçoğu daha sonra, buradan kuzeye kadar uzanan bölgelerdeki Ari kardeşleri ile birlikte Hindistan’a itildi.
\vs p078 6:4 Mezopotamya unsurlarının yüzde onu, sarı\hyp{}And unsurları ile karıştıkları yer olan Doğu Türkistan’a girerek, kuzey göç hareketleri içinde doğuya doğru yönelmişlerdi. Bu ırksal birliktelikten meydana gelen yetkin doğumların çoğunluğu; daha sonra Çin’e girip, sarı ırkın kuzey biriminin doğrudan gelişimine büyük katkı sağlamıştır.
\vs p078 6:5 Bu kaçış halindeki And unsurlarının yüzde onu, Arabistan boyunca ilerleyip Mısır’a girdiler.
\vs p078 6:6 Komşu kabile üyeleri ile evlenme zorundalığından onları özgür bırakan Dicle ve Fırat nehir ağızlarının çevresinde oldukça üstün bir kıyı kültürünün üyeleri, And unsurlarının yüzde beşi evlerini terk etmeyi reddetti. Bu topluluk, birçok üstün Nod ve Âdem ırk kolunun kurtuluşunu temsil etmişti.
\vs p078 6:7 And unsurları M.Ö. 6000’li yıllarda bu bölgeyi neredeyse tamamen terk etmiş bir haldeydi; her ne kadar soyları büyük ölçüde komşu Sangik ırkları ve Anadolu And unsurları ile karışmış olsa da, çok daha sonraki bir dönemde kuzey ve doğu istilacıları ile savaşmak için onlar burada hazır bulunmuşlardı.
\vs p078 6:8 İkinci cennet bahçesinin kültürel çağı, buraya olan komşu alt düzey ırk kollarının kademeli nüfuzuyla sona ermişti. Medeniyet, Mezopotamya’daki yaşam pınarının kurumasından çok uzun bir süre sonra gelişmeye ve ilerlemeye devam ettiği Nil ve Akdeniz adalarına doğru batı yönünde ilerlemiştir. Ve alt düzey insan topluluklarının bu denetlenmemiş akınları, yetkin olan geride kalmış ırk kollarını Mezopotamya’nın tamamından uzaklaştırmış kuzey barbarları tarafından buranın daha sonraki fethinin zeminini hazırlamıştır. Daha sonraki yıllarda bile bu kültürel tarihin kalıntısı, bu cahil ve görgüsüz istilacıların varlığına karşı koymuştur.
\usection{7.\bibnobreakspace Mezopotamya’daki Seller}
\vs p078 7:1 Nehir sakinleri, belirli mevsimlerde nehir sularının kıyı şeritlerinde taşkınlığa sebebiyet vermesine alışkındı; bu dönemsel seller, yaşamlarında senelik olarak tekrar eden durumlardı. Ancak yeni tehlikeler, kuzeye doğru gerçekleşmekte olan ilerleyici yeryüzü değişikliklerinin bir sonucu olarak Mezopotamya vadisini tehdit etmekteydi.
\vs p078 7:2 İlk Cennet Bahçesi’nin sular altında kalışından sonra binlerce yıl boyunca, Akdeniz’in doğu sahili çevresindeki ve Mezopotamya’nın kuzeybatısında ve kuzeydoğusundaki dağlar yükselmeye devam etti. Bu dağlık alanların yükselişi yaklaşık olarak M.Ö. 5000’li yıllarda büyük ölçüde hızlanma gösterdi; ve kuzey dağlarındaki kar yağışının büyük oranda artış göstermesiyle birlikte bu durum, Fırat vadisi boyunca her ilkbaharda beklenmedik sellerin meydana gelmesine sebebiyet verdi. Bu ilkbahar selleri, nehir bölgesi sakinlerinin doğu yükseltilerine doğru nihai olarak itilmesi derecesinde artan bir biçimde kötüleşme gösterdi. Yaklaşık bin yıl boyunca birçok şehir, bu geniş çaplı sel baskınları nedeniyle neredeyse tamamen boşaltılmıştı.
\vs p078 7:3 Yaklaşık beş bin yıl sonra Babil esareti altındaki Musevi din adamları kendi topluluklarının kökenini Âdem’e dayandırmaya çalışırlarken, bu hikâyeyi doğrulamada büyük bir zorluk yaşadılar; ve onlardan bir tanesi, bu çabadan vazgeçip Nuh’un seli zamanında tüm dünyanın kendi günahı içinde boğulmasına müsaade ederek İbrahim’i böylelikle Nuh’un hayatta kalan üç evladından bir tanesine doğrudan bir biçimde dayandırmakla daha tutarlı bir anlatıma ulaşılacağını düşündü.
\vs p078 7:4 Dünya yüzeyinin tamamının bir dönem sular altında kaldığına dair tarihi anlatılar evrenseldir. Birçok ırk, eski çağlar boyunca dünya çapında gerçekleşmiş bir sele dair hikâyeye sığınmaktadır. Nuh’un, gemisinin ve selin İncil’deki hikâyesi, Babil esareti sürecinde gerçekleşen Musevi din adamlığının bir yaratımıdır. Urantia oluşturulduğundan beri tüm dünyayı içine alan bir sel hiçbir zaman ortaya çıkmamıştır. Dünya yüzeyinin tamamen sular altında kaldığı tek dönem, karanın ortaya çıkmaya başlamasından önceki Arkeozoyik çağlar sürecidir.
\vs p078 7:5 Ancak Nuh gerçekten yaşamıştır; o, Uyuk yakınındaki bir nehir yerleşkesi olan Aram’ın bir şarap ustasıydı. O, yıldan yıla nehrin taştığı dönemlerin kaydını tutmuştu. Nuh; tüm evlerin tekne biçiminde ahşaptan yapılmasını ve sel dönemi yaklaştığında aile hayvanlarının her gece bu yapılara konulmasını herkese öneren bir biçimde nehir vadisinden yukarı aşağı gidip gelerek fazla sayıdaki alaycı küçümsemeyi üzerine çekmiştir. O her yıl komşu nehir yerleşkelerine gidip sellerin yaklaşmakta olduğuna dair onları uyarırdı. Sonunda, senelik sellerin olağandışı bir sağanakla öyle büyük bir hale gelip suların beklenmedik taşkınının bütün köyü ortadan kaldırdığı bir yıl gelmişti; sadece Nuh ve onun birinci dereceden ailesi yüzen evlerinde hayatta kalmıştı.
\vs p078 7:6 Bu seller And medeniyetinin parçalanışını tamamlamıştır. Sel baskınlarının bu döneminin sona ermesiyle birlikte ikinci cennet bahçesi artık mevcut değildi. Sadece güneyde ve Sümer toplulukları arasında onların eski ihtişamına dair kalıntılar varlığını sürdürmekteydi.
\vs p078 7:7 En eski medeniyetlerden bir tanesi olan buranın kalıntıları, Mezopotamya’nın bu bölgelerinde ve onun kuzeydoğu ve kuzeybatı kesimlerine bulunabilir. Ancak Dalamatia dönemlerinin daha eski kalıntıları, Basra Körfezi’nin suları altında bulunmaktadır; ve ilk Cennet Bahçesi, Akdeniz’in doğu ucu altında deniz tabanında yatmaktadır.
\usection{8.\bibnobreakspace Sümerliler --- Son And Toplulukları}
\vs p078 8:1 Son And göçü Mezopotamya medeniyetinin biyolojik omurgasını kırdığında, bu üstün ırkın küçük bir azınlığı nehirlerin ağız kısımlarının yakınında ana yerleşkelerinde kalmaya devam etti. Bu topluluklar Sümerliler’di; ve her ne kadar onların kültürü kimlik bakımından daha baskın bir biçimde Nod aslına ait olsa da, M.Ö. 6000’li yıllarda özleri itibariyle And soyu haline gelmiş bir haldelerdi; ve onlar, Dalamatia’nın tarihi geleneklerine bağlı kaldılar. Yine de kıyı bölgelerin bu Sümer unsurları, Mezopotamya’daki en son And topluluklarının sonuncusuydu. Ancak Mezopotamya ırkları bu daha geç dönemde hali hazırda bütünüyle birbirine karışmış bir haldeydi; bu dönemin kalıntılarında bulunan kafatası türlerine bu durumu kanıtlamaktadır.
\vs p078 8:2 Susa’nın oldukça büyük bir oranda gelişmesi bu nehir taşkını süreçlerine denk gelmektedir. Birinci ve giriş şehri öyle bir düzeyde sele maruz kalmıştı ki ikinci veya diğer bir değişle yukarıdaki şehir bu dönemin görülmemiş el işlerinde alt şehri takip eden bir merkez haline gelmişti. Bu sellerin daha sonra azalışıyla birlikte Ur, çömlek imalatının ana yerleşkesi olmuştu. Ur, Basra Körfezi üzerinde bulunmaktaydı; nehir birikintileri bu dönemden beri karayı bugünkü sınırlarına ulaştıracak şekilde inşa etmişti. Bu yerleşkeler, nehirlerin daha iyi denetimi ve ağızlarının genişlemesi sebebiyle sellerden daha az zarar gördüler.
\vs p078 8:3 Fırat ve Dicle vadilerinin barışçıl buğday yetiştiricileri, Türkistan ve İran yüksek düzlüklerinden gelen barbarların akınlarıyla uzunca bir süredir tacize uğramaktaydı. Ancak dağlık otlaklarda yaşanan artan düzeydeki kuraklık bu aşamada, Fırat vadisine yapılan iyi tasarlanmış bir istilayı beraberinde getirdi. Ve bu istila çok daha ciddi bir bütünlük içerisindeydi, çünkü çevredeki sürü sahipleri ve avcılar geniş sayılarda evcilleştirilmiş atlara sahiplerdi. Güneydeki zengin komşuları üzerinde çok büyük bir askeri üstünlüğü kendilerine kazandıran şey onların atlara sahip olmalarıydı. Kısa bir süre içinde onlar, son kültür dalgalarını Avrupa, batı Asya ve kuzey Afrika’nın tamamına iten bir biçimde Mezopotamya’nın tümü üzerinde etkinlik kurdular.
\vs p078 8:4 Mezopotamya’nın bu fatihleri, Âdemoğlu ırk kollarının bazılarına ek olarak Türkistan’ın melez kuzey ırklarına daha iyi And kollarının birçoğunu soylarında taşıdılar. Kuzeyden gelen daha az gelişmiş ancak daha cesur olan bu kabileler; hızlı ve gönüllü bir biçimde Mezopotamya’nın arta kalan medeniyetine onun baskınlığı altında karışıp, yakın bir zaman içinde tarihi kayıtların giriş kısmında Fırat vadisinde bulunan bahse konu karma melez insanlara doğru gelişme gösterdiler. Onlar, vadi kabilelerinin sanatlarını öğrenip Sümer topluluklarının sahip oldukları kültürün büyük bir kısmına uyum sağlayarak Mezopotamya’da sona ermekte olan medeniyetin birçok fazını canlandırdılar. Onlar, Babil’in üçüncü kulesinin inşasını bile arzulayıp, daha sonra milli isimlerini ona vermişlerdir.
\vs p078 8:5 Kuzeydoğudan gelen bu barbar atlılar tüm Fırat vadisini elde ettiğinde, Basra Körfezi üzerindeki nehir ağzı etrafında ikamet eden And topluluklarının geride kalanlarını ele geçirmediler. Bu Sümerliler; üstün usla, daha iyi silahlarla ve birbirine bağlı havuzlardan oluşan sulama düzenlerinin yanında geniş çaplı askeri kanallardan oluşan bir sistemle kendilerini savunmaya yetkinlerdi. Onlar, tek\hyp{}tip bir topluluk dinine sahip oldukları için bütünleşmiş bir birliktelikti. Bu Sümerliler böylelikle, kuzeybatıdaki olan komşularının birbirinden kopuk şehir devletlerine bölünmelerinden çok daha uzun bir süre sonraya kadar ırksal ve milli bütünlüklerini muhafaza etmeye yetkinlerdi. Bu şehir topluluklarının hiçbiri bütünleşmiş Sümerler’i alt etmeye muktedir değildi.
\vs p078 8:6 Ve kuzeyden gelen istilacılar yakın bir zaman içerisinde, bu barışı seven Sümerler’i eğitmenler ve idareciler olarak güvenmeyi ve takdir etmeyi öğrendiler. Onlar fazlasıyla saygı görmüş ve kuzeydeki insanlara ek olarak batıda Mısır’dan doğuda Hindistan’a kadar tüm insan toplulukları tarafından sanat ve el işleri öğretmenleri, ticaret yöneticileri ve toplum idarecileri olarak aranılan bireyler olmuşlardır.
\vs p078 8:7 Öncül Sümer konfederasyonunun parçalanmasından sonra daha sonraki şehir devletleri, Seth din adamlığının bu inancı terk etmiş soyları tarafından idare edilmiştir. Yalnızca bu din adamları, komşu şehirleri ele geçirdikleri zaman kendilerini krallar olarak addetmektelerdi. Daha sonra şehir kralları Sargon döneminden önce ilahiyat kıskançlığı yüzünden güçlü konfederasyonları kurmada başarısız oldular. Her şehir kendi kent tanrısının tüm diğer tanrılardan daha üstün olduğuna inandı ve bu nedenle onlar kendilerini ortak bir öndere tabi kılmayı reddetti.
\vs p078 8:8 Şehir din adamlarının zayıf yönetimine ait bu uzun dönemin sonu, kendisini kral ilan eden ve Mezopotamya’nın tamamını yanındaki yerleşkeler ile birlikte ele geçirmeye girişmiş Kiş’in din adamı Sargon tarafından getirildi. Ve kısa bir süreliğine bu idare; her şehrin kendi sınırlarını temsil eden bir tanrıya ve onun törensel uygulamalarına sahip olduğu din adamlarının yönettiği ve onların hüküm sürdüğü şehir devletlerini sonlandırdı.
\vs p078 8:9 Bu Kiş konfederasyonunun parçalanmasını, bu vadi şehirleri arasında üstünlüğü ele geçirmek için sürekli verilen savaşların uzun bir dönemi izledi. Ve yönetim; Sümer, Akat, Kiş, Erek, Ur ve Susa arasında değişen sırayla el değiştirdi.
\vs p078 8:10 Yaklaşık M.Ö. 2500’lü yıllarda Sümerliler, kuzey Suit ve Guit topluluklarını karşısında ciddi kayıplara maruz kaldılar. Sel tepesi üzerine inşa edilen Sümer başkenti olan Lagaş yitirilmişti. Erek, Akat’ın çöküşünden sonra otuz yıl boyunca ayakta kaldı. Hammurabi idaresinin kurulması zamanında Sümerler, kuzey Sami topluluklarının kollarına onların baskınlığında karışmış bir haldeydi; ve Mezopotamyalı And unsurları tarih sayfasından silindiler.
\vs p078 8:11 M.Ö. 2500 ile 2000’li yıllar arasında göçebeler, Atlantik’den Pasifik’e kadar bir öfke nöbeti içerisindelerdi. Nerites, birbirlerine karışmış Andon ve And ırklarının Mezopotamyalı soylarına ait Hazar topluluklarının nihai patlamasını ortaya çıkardı. Barbarlar Mezopotamya’nın yerle bir edilmesinde hangi alanda başarısız oldularsa, daha sonraki iklim değişiklikleri bunları yerine getirdi.
\vs p078 8:12 Ve bu anlatım, Âdem’in döneminden sonra eflatun ırkının ve Fırat ve Dicle arasında kalan anavatanlarının kaderine dair hikâyesidir. Onların tarihi medeniyeti, üstün insan topluluklarının dış göçleri ve alt düzey komşularının iç göçleri nedeniyle nihai olarak çökmüştür. Ancak Barbar atlıların vadiyi ele geçirmelerinden uzun bir süre önce, Cennet Bahçesi kültürünün çoğu Asya, Avrupa ve Afrika’ya hali hazırda yayılmış bir haldeydi; bu kültür buralarda Urantia’nın yirminci yüzyıl medeniyetiyle sonuçlanan mayayı çalmıştı.
\vs p078 8:13 [Nebadon’un bir Başmelek unsuru tarafından sunulmuştur.]
