\upaper{1}{Kâinatın Yaratıcısı}
\vs p001 0:1 Kâinatin Yaratıcısı tüm yaratılmışların Tanrı’sı, tüm madde ve varlıkların İlk Kaynak ve Merkezi’dir. İlk önce Tanrı’yı bir yaratan olarak, daha sonra bir denetleyici olarak ve en sonunda sınırsız bir koruyucu olarak düşünün. Kâinatın Yaratıcısı hakkındaki gerçek peygamberin “Sen Tanrım teksin ve senin yanında hiçbir eşin benzerin yok. Cenneti, cennetlerin tümünü barınan tüm misafirleriyle yaratan, koruyan ve denetleyen sensin. Tanrı Evlatları tarafından evrenler inşa edildi ve Yaratıcı kendisini bir elbise gibi olan ışıkla kapladı ve tüm cennetler âlemine bir perde gibi yaydı” sözüyle insan aklına yerleşmeye başlamıştır. Sadece birçok tanrının bulunabileceği yerde tek Tanrı olarak hüküm süren Kâinatın Yaratıcısı kavramı fani insanın Yaratıcı’yı kutsal yaratan ve sınırsız denetleyici olarak kavramasına imkân sağladı.
\vs p001 0:2 Sayısız gezen sistemlerinin hepsi; onu tanıyacak, onun kutsal şefkatini algılayacak ve onu bunun karşılığında sevecek çeşitli birçok akıl sahibi yaratılmışlıklara sonunda ev sahibi yapması için onun tarafından yaratıldı. Kâinatın âlemlerinin tümü, Tanrı’nın bir yapıtı ve onun türlü yaratılmışlarının ikamet ettiği bir yaşam alanıdır. “Tanrı cenneti yaratmış ve yeryüzüne şekil vermiştir; evren düzenini kurmuş ve bu dünyayı sadece yaratmak için oluşturmamıştır, onu yaşama mesken oluşturması için belirli bir amaç dâhilinde şekillendirmiştir.
\vs p001 0:3 Aydınlanan dünyaların tümü Kâinatın Yaratıcısı’nı, ebedi yapıcı ve tüm yaratılmışlıkların sınırsız koruyucusu olarak tanır ve ona ibadet eder. Evren’in irade sahibi yaratılmışlıkları âlemler Yaratıcı olan Tanrı’ya ulaşmanın ebedi serüveninin cezp edici uğraşına çok uzun bir Cennet yolculuğuyla başlamışlardır. Zaman tarafından belirlenen bu çocukların bu aşkın hedefi ebedi Tanrı’yı bulmak, onun kutsal doğasını kavramak ve Kâinatın Yaratıcısı’nı tanımaktır. Tanrı\hyp{}bilen yaratılmışlar; onun kişiliğin Cennet mükemmeliyetinde ve haklı yüceliğin evrensel alanında olduğu gibi kendi yaşam alanlarında tıpkı onun gibi olmak gibi onları sürükleyen tek bir isteğe, sadece tek bir yüce amaca sahiptirler. Ebediyeti taşıyan Kâinatın Yaratıcısı’ndan “Kusursuz olun, hatta benim olduğum kadar kusursuz olun” biçiminde yüce bir buyruk çıktı. Cennet’in habercileri sevgi ve bağışlama içerisinde bu kutsal emri çağlar ve evrenler boyunca ve hatta en düşük seviyede hayvan\hyp{}kökenli Urantia’nın insan ırklarına varıncaya kadar taşıdı.
\vs p001 0:4 Kutsallığın kusursuzluğuna ulaşmak için verilen bu muhteşem ve evrensel hüküm, kusursuzluğun Tanrısı’nın tüm uğraş veren yaratılmışlarının en yüksek amacı olması gereken ilk emridir. Kutsal kusursuzluğa varışın olanaklılığı tüm insanlığın ebedi ruhsal gelişiminin kesin ve nihai kaderidir.
\vs p001 0:5 Urantia fanileri sınırsız bir biçimde kusursuzluk için neredeyse hiçbir umut besleyemezler, fakat bu kusursuzluğa erişim tüm insanoğlu için tamamiyle mümkündür. Sınırsız Tanrı’nın fani insan için belirlediği tanrısal ve kutsal hedefe ulaşmak için bu amaç doğrultusunda gezende faaliyetlerde bulunduklarında ve bu kaderi elde ettiklerinde, onların tüm kendilerini gerçekleştirmeleriyle ve akla ulaşımlarıyla Tanrı’nın kendisinin sınırsız ve ebedi alanında olduğu gibi kutsal kusursuzluğun onların yaşam alanlarındaki tamamlanmışlığına erişeceklerdir. Böyle bir kusursuzluk maddi bakımdan evrensel, idraki algılayışla sınırsız veya ruhani deneyimle nihai olmayabilir; fakat iradenin, kişiliğin kusursuzluk istencinin ve Tanrı\hyp{}bilincinin kutsallığının tüm sınırlı yönleriyle nihai ve eksiksizdir.
\vs p001 0:6 “Kusursuz olun, hatta benim olduğum kadar kusursuz olun” kutsal emrinin gerçek anlamı fani insanı onların bu uzun ve cezp edici çabalarında ruhsal değerlere ve gerçek evren anlamlarının daha üst düzeylerine erişimi için dışsal bir biçimde başından beri teşvik eder ve içsel bir biçimde onu sürekli kendisine doğru çağırır. Bu âlemlerin Tanrı’sı için gösterilen bu görkemli arayış zaman ve mekânın tüm dünyalarının sakinlerinin yüce serüvenidir.
\usection{I.\bibnobreakspace Yaratıcı’nın İsmi}
\vs p001 1:1 Yaratıcı olan Tanrı’nın tüm isimleri tüm evrenler boyunca bilinir, onların arasında onu İlk Kaynak ve Kâinatın Merkezi olarak tanımlayanlar en sıklıkla karşılaşılanlardır. Baş Yaratıcı farklı evrenlerde ve aynı evrenin farklı bölümlerinde bile çeşitli birçok isimlerde bilinir. Yaratılanın Yaratan’a atfettiği isimler yaratılanın Yaratıcı’yı algılayışından doğan kavramsallaştırmasına fazlasıyla bağlıdır. İlk Kaynak ve Kâinatın Merkezi kendisini doğasını dolaysız olarak açığa vurma dışında hiçbir zaman bir isim üzerinden açığa çıkarmamıştır. Eğer biz kendimizi bu Yaratıcı’nın çocukları olduğumuza inanıyorsak bizim onu Yaratıcı olarak adlandırmak zorunda oluşumuz sadece doğal bir gerekliliktir. Fakat bizim bu kendi tercihimiz sonucunda açığa çıkan isimdir, ve bu isim İlk Kaynak ve Merkez’le olan kişisel ilişkimizin tanınmasıyla doğmuş ve değişime uğramıştır.
\vs p001 1:2 Kâinatın Yaratıcısı hiçbir zaman keyfi bir tanınma biçimini, resmi ibadeti veya kölesel hizmeti evrenlerin akıl sahibi yaratılmışlarının iradesine dayatmaz. Zaman ve mekân dünyalarının evrimsel sakinleri kendi kalpleri doğrultusunda onu tanır, sever ve gönüllü olarak ona ibadet eder. Yaratıcı onun maddi yaratılmışlarının ruhani özgür iradelerinin teslimiyetini zorla veya baskıyla sağlamayı reddeder. Yaratıcı’nın istencini yapma konusunda insan iradesinin sevgi dolu bağlılığı insanın seçkin tercihinin Tanrı’ya sunulan bir hediyesidir; gerçekte yaratılan iradesinin böyle bir ithafı Cennet Yaratıcısı’na karşı verilebilecek gerçek değerin tek olası hediyesini oluşturur. Tanrı’da, insan yaşar, hareket eder ve kendi varlığına sahip olur; Yaratıcı’nın iradesine bağlı olmayı tercih etmekten başka insanın Tanrı’ya verebileceği hiçbir şey yoktur. Bu ilaveten bu kararlar evrenlerin akli iradeye sahip yaratılanlar tarafından etkileşim halinde olup Yaratıcı Yaratan’ın sevgi dolu baskın doğasını oldukça tatmin edecek gerçek ibadeti oluşturur.
\vs p001 1:3 Siz tam anlamıyla Tanrı\hyp{}bilincine sahip olduğunuz zaman, görkemli Yaratıcı’yı gerçekten keşfettiğinizde ve kutsal denetleyicinin nüfuz edici varlığının kendisini gerçekleştirmesini denetimlemeye başladığınızdan sonra, ve tüm bunların sonucunda kendi aydınlanmanızın ve Kutsal Evlatlar’ın Tanrı’yı aşığa çıkarış biçimiyle ve anlamınca siz İlk Muhteşem Kaynak ve Merkez’i yeterli bir biçimde tanımlayacak bir kavramsallaşmayla Kâinatın Yaratıcısı için bir isim bulacaksınız. Ve böylece, farklı dünyalarda ve çeşitli evrenlerde Yaratan bu isimlere karşılık gelen tüm ilişkiler bütününde birçok tanımlamalarla aynı anlamla bilinir hale gelir; fakat bu isimlerin ve sembollerin her birinin herhangi bir âlemdeki yaratılmışların kalplerinde kurduğu tahtın karşılık bulduğu derece ve derinlik farklılık gösterir.
\vs p001 1:4 Kâinatın âlemlerinin tümünün merkezinin yakınında Kâinatın Yaratıcısı genel olarak İlk Kaynak anlamına gelen adlandırmalarla bilinir. Uzay evrenlerinin çevrelerine doğru Evrensel Merkez anlamına gelen kavramlar sık bir biçimde kullanılır. Yıldızlar tarafından aydınlatılmış yaratılmışlıkların daha uzağında ise sizin yerel evreninizin yönetim merkezi olan dünyasında İlk Yaratıcı Kaynak ve Kutsal Merkez olarak bilinir. Buraya yakın bir yıldız kümesinde Tanrı, Âlemlerin Yaratıcısı olarak ifade edilir. Bir diğerinde ise, Sınırsız Koruyucu, ufuk noktasında ise Kutsal Denetleyici olarak bilinir. Kendisi aynı zamanda Aydınlığın Yaratıcısı, Yaşamın Bahşedişi ve Her şeye Gücü Yeten Birlik olarak tarif edilir.
\vs p001 1:5 Bir Cennet Evladı’nın bahşedilmiş bir yaşam yaşadığı bu dünyalar üzerinde Tanrı genel olarak kişisel ilişkilerin, duygusal sevginin, ve yaratıcı sadakatin tanımlayıcı sıfatlarıyla adlandırılır. Sizin yıldız kümenizin yönetim merkezinde Tanrı Kâinatın Yaratıcısı olarak kaynak gösterilir, ve yerleşik dünyaların üyesi olan sizin yerel güneş siteminin farklı gezegenlerinin üzerinde Yaratıcıların Yaratıcısı, Cennet Yaratıcısı, Havona Yaratıcısı ve Ruh Yaratıcısı olarak çeşitli biçimlerde bilinir. Cennet Evlatları’nın bahşedişlerinin açığa çıkardıklarıyla Tanrı’yı tanıyanlar sonunda Yaratan\hyp{}yaratılanın güçlü etkileyici ilişkisinin duygusal cazibesine erişebilir ve Tanrı’yı “bizim Yaratıcımız” biçiminde içselleştirebilirler.
\vs p001 1:6 Cinsiyete sahip olan yaratılmışların bir gezegeni üzerinde, aile duygusu uyarılarının akıl sahibi varlıkların kalplerinde bulunduğu bir dünyada, bir Baba olarak Yaratıcı kavramı ebedi Tanrı’yı tanımlamak için hayli tanımlayıcı ve yerinde bir isim haline gelir. Sizin gezegeniniz üzerinde, Urantia’da, en iyi bilinen biçimde ve evrensel olarak kabul ediliş biçimiyle kendisi \bibemph{Tanrı} olarak bilinir. Geçmişte sizin peygamberleriniz onu düzgün bir biçimde “sonsuza kadar var olan Tanrı” olarak adlandırıp onu “ebediyeti barındıran” olarak kaynak gösterdiler. Ona verilen isim aslında baskın bir öneme sahip değildir, aslında dikkate değer nokta sizin onu bilmenizin ve onun gibi mükemmel olmak için ondan feyz almanızın gerekliliğidir.
\usection{2.\bibnobreakspace Tanrı’nın Gerçekliği}
\vs p001 2:1 Tanrı ruhani dünyanın başat bir gerçekliği; akli alanın gerçeklik kökenidir; Tanrı’nın varlığı tüm maddi alanlar boyunca onların hepsini etkisi altına alacak biçimde hüküm sürer. Tüm yaratılmış akli varlıklara göre Tanrı bir kişiliktir, ve kâinatın âlemlerinin tümüne göre ise o ebedi gerçekliğin İlk Kaynak ve Merkezi’dir. Tanrı ne bir insan gibi ne de makine gibidir. Baş Yaratıcı evrensel bir ruhaniyet, ebedi doğru, sınırsız gerçeklik ve yaratıcı kişiliğidir.
\vs p001 2:2 Ebedi Tanrı sınırsız bir biçimde gerçekliğin olası en yüksek amaca getirilmesinden veya evrenin kişiselleştirmesinden her zaman çok daha fazlasıdır. Tanrı sadece basit bir tanımlamayla insanın en yüce isteği veya onun fani arayışının bir amacı değildir. Yine Tanrı ne sadece bir kavram, ne de doğruluğun olası\hyp{}kudreti değildir. Kâinatın Yaratıcısı doğayı tarif etmek için kullanılan ne eş anlamlı bir sözcük ne de doğa kanunlarının ete kemiğe büründüğü bir bünyeye için kullanılan bir anlamdır. Tanrı başlı başına aşkın bir gerçekliktir, sadece yüce değerler içerisinde insanın geleneksel kavramlaştırmalarından ibaret değildir. Ne Tanrı ruhsal anlamların bir psikolojik odaklanması, ne de “insanın en soylu yapıtı”dır. Tanrı insanların akıllarında bu kavramların herhangi biri veya hepsi bile olabilir, fakat gerçekte kendisi her zaman bütün bunlardan daha fazlasıdır. Tanrı, ruhani barışı dünya üzerinde sevinçle karşılayan ve kişiliğin ölümde hayatta kalışını deneyimlemeyi arzulayan herkes için bir kurtarıcı kişilik ve sevgi dolu bir Yaratıcı’dır.
\vs p001 2:3 Tanrı’nın varlığının mevcudiyeti, Cennet’ten fani insanın aklında yaşaması ve ebedi varlığını sürdürmesinin evrimleşen ölümsüz ruhuna yardımcı olmak için gönderilen ruh Gözlemleyicisi olan kutsal varlığın nüfuzu tarafından insan deneyimlemesinde gösterilmiştir. İnsan aklında bu kutsal Düzenleyici’nin varlığı üç deneysel olgular bütününde açığa çıkar:
\vs p001 2:4 1.\bibnobreakspace Tanrı’yı kavramak için akli kabiliyet --- Tanrı\hyp{}bilinci.
\vs p001 2:5 2.\bibnobreakspace Tanrı’yı bulmak için ruhani arzu --- Tanrı\hyp{}arayışı.
\vs p001 2:6 3.\bibnobreakspace Tanrı gibi olmak için duyulan özlem kişiliği --- Yaratıcı’nın iradesini gerçekleştirmek için duyulan samimi istek.
\vs p001 2:7 Tanrı’nın varlığı hiçbir zaman bilimsel deneylerle veya mantıksal çıkarsamalardan biri olan saf akılla ispatlanamaz. Bunun yerine Tanrı sadece insan deneyimlemelerinin yaşam alanlarında açığa çıkabilir; yine de Tanrı’nın gerçekliğinin asli kavramı mantığa uygunluk, felsefeye yatkınlık, dine özsellik, ve herhangi bir kişiliğin varlığını sürdürmesi umuduna karşı olmazsa olmazlık gösterir.
\vs p001 2:8 Tanrı’yı gerçekten bilenler onun varlığının gerçekliğini deneyimlemiş olanlardır; bunun gibi Tanrı\hyp{}bilen faniler deneyimlemelerinde sadece insanoğlunun diğerleriyle paylaşabileceği yaşayan Tanrı’nın varlığının olumlu kanıtını taşırlar. Tanrı’nın varlığı; insan aklının Tanrı\hyp{}bilinci ve onun fani düşünme gücünde ikamet eden, Kâinatın Yaratıcısı tarafından ona karşılıksız olarak verilen hediye olan Düşünce Denetleyicisi’nin Tanrı\hyp{}varlığı arasındaki bütünlük dışında tüm olası görüngülerin ötesindedir.
\vs p001 2:9 Kuramsal bir biçimde Tanrı’yı Yaratan olarak düşünebilir, ve onu Cennet’in kişisel yaratanı ve kusursuzluğun merkezi evreni olarak görebilirsiniz, fakat zaman ve mekân evrenlerinin tümü Yaratan Evlatlar’ın Cennet birliği tarafından yaratılmış ve düzenlenmiştir. Kâinatın Yaratıcısı Nebadon’un yerel evreninin kişisel yaratanı değildir; sizin yaşadığınız evren onun Mikâil olan Ebadı’nın yapıtıdır. Yaratıcı evrimsel evrenleri kişisel olarak yaratmadıysa bile; onları fiziksel, akli ve ruhani enerjilerin dışavurumlarında ve onların evresel ilişkilerinin birçoğunda kontrol eder. Yaratıcı olarak Tanrı Cennet âleminin kişisel yaratıcısıdır, ve Ebedi Evlat ile etkileşim halinde tüm diğer kişisel evren Yaratanları’nın yaratıcısıdır.
\vs p001 2:10 Maddi olan kâinatın âlemlerinin tümü içinde bir fiziksel denetleyici olarak İlk Kaynak ve Merkez, ebedi Cennet Adası’nın işleyişleri dâhilinde faaliyetlerde bulunur, ve bu mutlak yerçekimi merkeziyle ebedi Tanrı kâinatın âlemlerinin tümü boyunca ve merkezi evrende kozmik fiziksel düzeyin üst denetimini eşit bir biçimde uygular. Akıl olarak Tanrı, Sınırsız Ruh’un İlahiyatı’nda faaliyet gösterir; ruh olarak Ebedi Evlat’ın ve onun kutsal çocuklarının kişiliğinde açığa çıkmış bir biçimde bulunur. Bu İlk Kaynak ve Merkez’in Cennet Kişilikleri ve Mutlaklıkları ile eş güdümünün ilişkisi, Kâinatın Yaratıcısı’nın tüm yaratılanlar ve onların bütün düzeyleri üzerindeki \bibemph{doğrudan} kişisel faaliyetlerini hiçbir biçimde küçükte olsa engellemez. Onun katmanlı ruhunun varlığıyla Yaratan Yaratıcı onun yaratılmış çocuklarıyla ve kendi yarattığı evrenleriyle aracısız bir ilişkiyi yönetir.
\usection{3.\bibnobreakspace Evrensel Bir Ruh Olarak Tanrı}
\vs p001 3:1 “Tanrı ruhaniyettir.” O evrensel bir ruhani mevcudiyettir. Kâinatın Yaratıcısı sınırsız bir ruhani gerçekliktir; O “egemen, ebedi, ölümsüz, gözle görülemez, ve tek gerçek Tanrı”dır. Siz “Tanrı’nın doğumu” olsanız bile, Yaratıcı’yı sizin gibi sizin yapınızda veya fiziğinizde düşünmemeniz gerekir, çünkü onun ebedi mevcudiyetinin yerleşkesinden gönderilen Gizem Görüntüleyiciler’i tarafından yerleştirilmiş “onun görüntüsü içinde” yaratılmanız gerektiği söylenmiştir. Ruhani varlıklar her ne kadar insan gözlerine karşı görünmez ve ete ve kana sahip olmasalar da gerçektirler.
\vs p001 3:2 Geçmişin kâhini zamanında “Bak işte, Tanrı benimle birlikte hareket ve ben onu göremiyorum; o benden bağımsız olarak da devinim halinde, fakat ben bunu algılayamıyorum” dedi. Biz Tanrı’nın yapıtlarını sürekli bir biçimde gözlemleyebiliriz, ve aynı zamanda onun görkemli idaresinin maddi kanıtları hakkında yüksek bir bilince sahip olabiliriz, fakat bırakın insan nüfuzunda onun temsilci ruhunun farkına varmayı onun kutsallığının açığa çıkmış halinin gözle görülebilen niteliklerine nadir bir biçimde gözlerimizi yöneltebiliriz.
\vs p001 3:3 Kâinatın Yaratıcısı kısıtlı ruhani niteliklere ve maddi engellerin düşük seviyedeki yaratılmışlarından kendisini sakladığı için görülmez değildir. Onun görünmezliği hususunda bu durumun yerine: “Siz benim yüzümü göremezsiniz, hiçbir fani için benim varlığımın görülmesi ve bire bir biçimde benimle bir yaşanması söz konusu değildir.” Hiçbir maddi insan Tanrı’nın ruhaniyetini dolaysız olarak gözle göremez ve onlar kendilerinin fani varlığını koruyamazlar. Kutsal kişiliğin mevcudiyetinin ruhani mükemmelliği ve ihtişamı alt sınıf ruhani varlıklar veya maddi kişiliklerin herhangi bir düzeyi tarafından erişimi imkânsızdır. Yaratıcı’nın kişisel mevcudiyetinin ruhani ışıltısı “hiçbir fan, insanın yaklaşamayacağı; hiçbir maddi yaratılmışın gördüğü veya göreceği bir ışıktır.” Fakat ruhanileştirilmiş aklın inanç\hyp{}gözüyle Tanrı’yı anlamak için bedenin gözlerinden Tanrı’yı görüp görememek hiçbir önem arz etmez.
\vs p001 3:4 Kâinatın Yaratıcısı’nın ruhani doğası onun ortak varlığı olan Cennetin Ebedi Evladı tarafından tamamen paylaşılır. Yaratıcı ve Evlat onların birleşik kişilik eş güdümü olan Sınırsız Ruhla birlikte aynı ölçüde evrensel ve ebedi ruhaniyeti bütünüyle ve koşulsuz olarak paylaşır. Tanrı’nın kendisinin ve kendi içinde bulunan ruhu mutlak, Evlat’ta koşulsuz, Ruhaniyet’te evrensel, ve bunların hepsinde ve bunlar tümüyle birlikte sınırsızdır.
\vs p001 3:5 Tanrı evrensel bir ruhaniyet; Tanrı evrensel bir kişiliktir. Sınırlı yaratılmışların yüce kişisel gerçekliği ruhtur; kişisel kainatın nihai gerçekliği ise absonit ruhtur. Sadece sınırsızlığın dereceleri mutlaktır, ve sadece bu seviyeler üzerinde madde, akıl ve ruh arasında kesinsel bir bütünlük mevcuttur.
\vs p001 3:6 Âlemler içinde Yaratıcı olan Tanrı potansiyel olarak maddenin, aklın ve ruhun üst deneticisidir. Sadece kendisinin ucu bucağı olmayan kişilik itibarı sayesinde Tanrı kendisinin ifade edilemeyecek ölçüde yarattığı irade sahibi yaratılmışlarının kişilikleriyle dolaysız olarak iletişim halindedir. Fakat, âlemlerdeki geniş bir alana yayılan Tanrı iradesi olan sadece onun birimlere ayrılmış oluşumlarında Cennet’in dışından ilişkiye geçilebilir. Bu Cennet ruhaniyeti zamanın fanilerinin akıllarında ikamet eder, ve burada Kâinatın Yaratıcısı’nın kutsallığı ve doğasının bir parçası olan ölümsüz ruhun varlığını sürdürmeye çalışan yaratılmışların evrimini geliştirir. Fakat bu evrimsel yaratılmışların akıllarının kökeni yerel evrenlerde başlar, ve cennete Yaratıcı’nın iradesini yerine getirmeyi tercih eden yaratılmışlarının bu ruhani erişiminin deneyimsel dönüşümlerinin nihai sonucu olan kutsal kusursuzluğu elde etmelidirler.
\vs p001 3:7 İnsan’ın içsel deneyimlerinde, akıl maddeyle bütünleşir. Böyle maddesel\hyp{}birlikteliğe sahip olan akıllar fani ölümden kurtulamazlar. Ölümsüz ruhun varlığını devam ettirmenin düzeni, böyle bir Tanrı\hyp{}bilincine sahip aklın yavaşça ruh öğretisine dönüşmesi ve sonunda ruhaniyetle yönetilen bir oluşuma sahip olması vasıtasıyla fani aklın içindeki bu dönüşümleri ve insan iradesinin bu türlü düzenlenmelerinde sağlanmış olur. İnsan aklının maddesel etkileşimden ruhsal bütünlüğe olan bu evrimi fani aklın potansiyel ruh fazlarının ebedi ruhun morantiya gerçekliklerine doğru hal değişimiyle sonuçlanır. Maddenin hükmü altında olan fani akıl gittikçe artan bir biçimde maddeleşmeye, ve sonuçta nihai kişilik tükenmesinden yoksunluk duymaya mecburdur. Fakat akıl tarafından zemini hazırlanan ruh, bu kişilik varoluşunun varlığını devam ettirmesi ve ebediyete ulaşmasının yoluyla artan bir biçimde ruhanileşerek sonunda benzersizliğe erişir.
\vs p001 3:8 Ben Ebediyet’ten geliyorum, ve ben birçok kere Kâinatın Yaratıcısı’nın varlığına dönüşlerde bulundum. Ebedi ve Evrensel Yaratıcı olan İlk Kaynak ve Merkez’in kişiliğini ve mevcudiyetinin bilgisine sahibim. Tanrı’nın mutlak, ebedi ve sınırsız olmasının yanı sıra kendisinin aynı zamanda doğası gereği iyi erdem sahibi, kutsal ve merhametli olduğunun bilincindeyim. “Tanrı ruhaniyettir” ve “Tanrı sevgidir” gibi büyük bildirilerin doğruluğunu, ve onun bu iki niteliğinin neredeyse tamamının bütünlüksel bir biçimde Ebedi Evladı’nın bünyesinde evrende açığa çıkarıldığını biliyorum.
\usection{4.\bibnobreakspace Tanrı’nın Gizemi}
\vs p001 4:1 Tanrı’nın kusursuzluğunun sınırsızlığı öyle bir derecededir ki bu sınırsızlık ebedi bir biçimde onun gizemini oluşturur. Ve Tanrı’nın idrak edilemeyen gizemlerinin en büyüğü fani akılların içlerinde kutsallığı barındırışının olgusallığıdır. Kâinatın Yaratıcısı’nın zamanın yaratılmışları arasındaki bu geçici ikamesinin anlamı kâinattaki tüm gizemlerin arasından en derinidir; insan aklındaki bu kutsal mevcudiyet gizemlerin gizemidir.
\vs p001 4:2 Fanilerin fiziksel bedenleri “Tanrı’nın tapınaklarıdır”. Egemen Yaratan Evlatlar kendilerinin yaşam verdikleri dünyaların yaratılmışlarının yakınlarına gelmelerine ve “tüm insanları kendilerine doğru çekmelerine”; onların bilinçlerinin “kapılarında durmalarına” “ve bu kapıları çalmalarına” bunların sonucunda “onlara kalplerinin kapılarını açanların” içeriye buyur ettiklerine karşı hoşnutluk duymalarına, Yaratan Evlatlar ve onların fani yaratılmışları arasında bu içten kişisel iletişimin olmasına rağmen yine de fani insanlar onların içinde ikamet eden Tanrı’dan bir nüveye sahip olduklarından onların bedenleri Tanrı’nın tapınaklarıdır.
\vs p001 4:3 Siz bizim bulunduğumuz yerin aşağısında ikamet ettiğinizde, hayatınızın gidişatı geçici bir biçimde dünya üzerinde harekete geçirildiğinde, bedendeki bu deneyimleme yolculuğu tamamlandığında, ruhun geçici kaldığı bu fani bedeni oluşturan tozun “geldiği dünyaya tekrar geri döndüğünde”; tüm bu gelişmelerin aydınlığında artık açığa çıkmıştır ki içinizde ikamet eden Ruh onu size veren Tanrı’ya geri dönecektir. Kutsallığın bir parçası ve bölümü olan Tanrı nüvesi bu gezegenin her fani varlığında geçici bir süreliğine barınır. Size verilen bu nüve kullanım hakkından doğan bir özel mülkiyet niteliği göstermez ve bu sebeple tamamen size ait değildir, bunun yerine siz fani varoluşunuzdan ruhani varlığınızı devam ettirmeyi başarırsanız bilinçli bir biçimde sizinle birlikte var olması için tasarlanmıştır.
\vs p001 4:4 Biz sürekli Tanrı’nın gizemiyle karşılaşmaktayız; onun sınırsız iyiliği, bitip tükenmek bilmeyen bağışlaması, kıyaslanamaz bilgeliği ve üstün karakteri gerçeğinin sınırsız tayfının sürekli artan bir biçimde kendisini ortaya çıkarmasıyla hayretler içinde kalmaktayız.
\vs p001 4:5 Kutsal gizem, sınırlı insan ve kusursuz Cennet İlahiyat’ı, maddi ve ruhsal, geçici ve ebedi, sonsuz ve sonlu arasındaki içsel farklılaşmanın kendisinden oluşur. Kâinat sevgisinin Tanrı’sı tüm yaratılmışları; kutsal gerçeklik, güzellik ve iyilik gibi değerleri ruhsal olarak algılamaları için onların kabiliyetlerinin en üst noktasına kadar kusursuz bir biçimde onlara kendisini açığa çıkartır.
\vs p001 4:6 Her ruhsal varlık ve her fani yaratılmışlık için kâinatın âlemlerinin tümünün her dünyası üzerinde ve her kâinat alanında Kâinatın Yaratıcısı kendi merhametli ve kutsal kişiliğinin bütününü bu tür ruh varlıklarının ve fani yaratılmışlarının algılayacağı ve kavrayacağı biçiminde kendisini açığa çıkarır. Tanrı ruhani veya maddi hiçbir bünyenin tarafını tutmaz. Evrenin herhangi bir evladının verilen bir zaman aralığında Kutsal mevcudiyetten hoşnutlukla yararlanması böyle bir yaratılmışın sadece yüce maddi dünyanın ruhani gerçekliklerini algılaması ve onu kavraması kabiliyetiyle sınırlıdır.
\vs p001 4:7 İnsanın ruhani deneyimlemelerinin bir gerçekliği olarak Tanrı bir gizem değildir. Fakat, maddi düzenin fiziksel akıllarına göre ruhani dünyanın gerçekliği hakkında tekdüze bir yargıya varılmaya teşebbüs edildiğinde bu gizem ortaya çıkar. Bu gizemler o kadar güç algılanır ve o kadar derindir ki; zaman ve mekânın maddi dünyalarının evrimleşen fanileri tarafından ebedi Tanrı’nın algılanışı ve Sınırsızın sınırlılar tarafından tanınışının felsefi mucizesine ulaşımı sadece Tanrı\hyp{}bilen fanilerinin inanç\hyp{}algısıyla mümkündür.
\usection{5.\bibnobreakspace Kainatın Yaratıcısı’nın Kişiliği}
\vs p001 5:1 Tanrı’nın büyüklülüğünün ve sınırsızlığının onun kişiliğinin açıkça görülmemesine ve onu perdelemesine izin vermeyin. “Kulağı tasarlayanın kendisi, duyamaz mı?” Gözleri biçimlendirenin kendisi, göremez mi?” Kâinatın Yaratıcısı kutsal kişiliğin en yüksek noktası; tüm yaratılmışlar boyunca kişiliğin kökeni ve kaderidir. Tanrı sınırsız ve kişiseldir; ve kendisi sınırsız bir kişiliktir. Yaratıcı’nın kişisel sınırsızlığı onu maddi ve sınırlı varlıkların bütüncül kavramalarının sonsuz ötesinde konumlandırsa da Yaratıcı tam anlamıyla bir kişiliktir.
\vs p001 5:2 Tanrı, insan aklının anladığı kişilik kavramından ve kendisi herhangi bir üstün kişilik kavramsallaşmasından bile çok daha fazlasıdır. Fakat, kişiliğin nihai amacının ve bilgisinin oluşturduğu en üst düzey varlık gerçekliği kavramına sahip maddi yaratılmışların akıllarıyla bu tür idrak edilemeyecek kavramları tartışmak gereksizdir. Maddi yaratılmışların Kâinatın Yaratıcısı için en nihai olası kavramı, kutsal kişiliğin engin bilgisinin ruhani nitelikteki olası en yüksek amacında bütünleşir. Bu sebeple her ne kadar Tanrı’nın kişiliğin insan kavramsallaşmasının ötesinde olduğunu biliyor olsanız bile, siz Kâinatın Yaratıcısı’nın ebediyetten, sınırsızlıktan, gerçeklikten, iyilikten ve güzel kişilikten çok daha azı anlamına gelemeyeceğini de eşit ölçüde bilmelisiniz.
\vs p001 5:3 Tanrı kendi yarattığı hiçbir yaratılmışlarından saklanmaz. Kendisi sadece “hiçbir maddi yaratılmışın yaklaşamayacağı bir ışığın içinde ikamet ettiğinden” dolayı birçok varlık düzeyi için erişilemezdir. Kutsal kişiliğin bu enginliği ve ihtişamı evrimsel fanilerin sınırlı akıllarının algılayışının çok ötesindedir. Tanrı “elinin derinliğiyle suların hacmini, ve yine elinin bir karışıyla evreninin büyüklüğünü ölçebilir. Dünyanın yörüngesinde oturup bir perde gibi cennetleri bir perde gibi kapatıp sonunda içinde yaşamın hüküm süreceği bir evreni oluşturan Tanrı’dır.” “Tüm bunları yaratana, onların dünyalarını sırayla getirip onları isimleriyle çağırana gözlerinizi dört açıp dikkatlice bakın”; ve göreceksiniz ki “Tanrı’nın görünmeyenleri onun tarafından var edilenler tarafından ancak kısmen anlaşılabilmesi doğruluk arz eder. Bugün, siz yaşayan bir varlık olarak; görünmez Yapıcı’yı onun çok katmanlı ve çeşitli yapıtlarıyla, ve aynı zamanda onun Evlatlarının ve sayılamayacak derecedeki yardımcılarının hizmetleri ve açığa çıkarışlarıyla kavramak zorundasınız.
\vs p001 5:4 Maddi faniler Tanrı’nın kişiliğini göremeseler de onlar kendisinin bir kişilik olduğu kesinliğinden memnuniyet duymalı; Kâinatın Yaratıcısı’nın kendi düşük seviyedeki sakinlerinin ebedi ruhani gelişimlerini sağlayacak kadar dünyayı çok sevdiği gerçeğini yansıtacak doğruları inançla kabul etmeli; onun “kendi çocuklarının varlığından büyük bir keyif aldığından emin olmalıdır. Bir kusursuz, ebedi, sevgi dolu ve sınırsız Yaratan kişiliği oluşturan bu insan\hyp{}üstü ve kutsal niteliklerin hiçbirinden Tanrı mahrum değildir.
\vs p001 5:5 Üstün evrenlerin çalışanları dışında yerel yaratılmışlarda Tanrı, yerel evrenlerin egemenleri ve imara açılmış dünyalardaki yaratıcılar olan Cennet Yaratıcı Evlatları’ndan başka kişilerden veya yerleşik bir durumdan oluşan hiçbir dışavuruma sahip değildir. Yaratılmışın inancı eğer kusursuz olsaydı kendi Yaratıcısı’nı ararken Yaratıcı Evladı ile karşılaştığı zaman Kâinatın Yaratıcısı’nı gördüğünden emin olacak, ve onun Evladı’ndan başka bir şeyi görmek için kanıt istemeyecek veya beklemeyecekti. Fani insan yalın bir değişle ruh değişimini tamamlamadan ve Cennet’e gerçekte ulaşmadan Tanrı’yı göremez.
\vs p001 5:6 Cennet Yaratıcı Evlatları’nın doğaları İlk Muhteşem Kaynak ve Merkezi’nin sınırsız doğalarının evrensel mutlaklığı içerisinde koşulsuz potansiyellerin hepsini kapsamaz, fakat Kâinatın Yaratıcısı her durum ve koşulda Yaratıcı Evlatlar’da \bibemph{kutsal bir biçimde} var olur. Mikâil’in emri altındaki bu Cennet Evlatları, Işıldayan Sabah Yıldızı’ndan ilerlemeci hayvan evriminin en düşük insan yaratılmışına kadar tüm yerel evren kişiliğinin yönetimi için bile kusursuz kişiliklerdir.
\vs p001 5:7 Tanrı ve onun muhteşem ve merkezi kişiliği olmadan, uçsuz bucaksız olan kâinatın âlemlerinin tümü boyunca hiçbir kişilik oluşmazdı. Bu bakımdan \bibemph{Tanrı bir kişiliktir}.
\vs p001 5:8 Her ne kadar Tanrı ebedi bir kudret, görkemli bir mevcudiyet, aşkın bir nihai amaç ve muhteşem bir ruh olsa bile, ve kendisi tüm bunların hepsinden sınırsız bir biçimde daha fazlası olsa da, yine de Tanrı sonsuza kadar sürecek kelimenin tam anlamıyla kusursuz Yaratan kişiliğidir. “Seven ve sevilen” bir kişi “bilebilir ve bilinebilir,” ve bu kişi bizimle arkadaş olabilir, ve böylece sizin diğer insanlar tarafından bilinişiyle ve diğer insanların bilindikleri biçimde Tanrı’nın bir dostu olarak tanınırsınız. Bunların ışığında Tanrı gerçek bir ruh ve ruhani bir gerçekliktir.
\vs p001 5:9 Kâinatının tümüne açığa çıkarıldığı şekliyle; Kâinatın Yaratıcısı’nı gördüğümüz, onun sınırsız sayıdaki çeşitli yaratılmışlarının içinde onun bulunduğunu anladığımız, onun Egemen Evlatları’nın kişiliklerinde ona dikkatle baktığımız, onun kutsal mevcudiyetini burada veya orada, yakında ve uzakta hissetmeye devam ettiğimizde onun kişiliğinin yüceliğinden ne kuşku duyalım ne de onu sorgulayalım. Tüm bu ucu bucağı olmayan yetki dağıtımına rağmen, Tanrı her zaman gerçek kişiliğini korur ve kâinatın âlemlerinin tümü boyunca değişik bölgelere ayrılmış yaratılmışlarının sayısız ev sahipleriyle kişisel iletişimini sonsuza kadar sürdürür.
\vs p001 5:10 Kâinatın Yaratıcısı’nın kişiliğinin bilgisi genişlemiş ve daha doğru bir Tanrı kavramı olarak akla başat biçimde vahiy yoluyla ulaşmıştır. Nedensellik, bilgelik ve dinsel deneyimin bütünü Tanrı’nın kişiliğini çağrıştırır ve onun yorumlanmasını sağlar, fakat hepsinin birleşimi yine de tamamiyle onun kişiliğini tanımlamaya yetmez. İkame eden Düşünce Denetleyicileri bile birey öncesi niteliğini taşır. Herhangi bir dinin olgunluğu ve taşıdığı doğruluk payı onun Tanrı’nın sınırsız kişiliğine dair kavramlaştırmasıyla ve İlahiyat’ın mutlak birlikteliğini algılayışıyla orantılıdır. Bununla birlikte Kişisel İlahiyat’ın bilgisi Tanrı’nın bütünlüğü kavramını ilk kez tasarlanışından sonra dinsel olgunluğun bir ölçüsü haline gelir.
\vs p001 5:11 İlk çağ dinleri birçok kişisel tanrılara sahiplerdi, ve onlar insan görüntüsü biçiminde yansıtılmışlardı. Açığa çıkarılan bilgi İlk Sebep’in bilimsel nedenselliğinde yalnızca mümkün olan Tanrı’nın kişilik kavramının doğruluğunu onaylar ve sadece Evrensel Bütünlük’ün nihai olmayan felsefi bilgisinde belirtilir. Sadece kişilik yaklaşımıyla herhangi bir insan Tanrı’nın birliğini kavramaya başlayabilir. İlk Kaynak ve Merkez’in kişiliğini reddetmek maddecilik ve panteizm arasında bir tercih yapılmasını gerektirecek iki felsefi karmaşayı beraberinde getirir.
\vs p001 5:12 İlahiyat’ın üzerine yapılacak tasavvurlarda kişilik kavramı bedensellik fikrinden yoksun bırakılmalıdır. Bir maddesel beden ne insan ne de Tanrı’daki karakter için hayatiyet arz etmez. Bedenselliğin hatası insan felsefesinin bu iki örneğinde görülebilir. Maddecilikte insan kendi bedenini ölüm esnasında kaybettiği için kişilik olarak varlığının sona geldiği varsayılır; panteizmde ise Tanrı herhangi bir bedene sahip olmadığı için bir kişilik olarak tasvir edilmez. İlerleyici kişiliğin insan\hyp{}üstü türü akıl ve ruhun bir bütünlüğünde faaliyette bulunur.
\vs p001 5:13 Kişilik, Tanrı’nın basit bir niteliği değildir; bunun yerine bu nitelik, kelimenin tam anlamıyla kusursuz dışavurumun ebediyeti ve evrenselliğinde gösterilen sınırsız doğanın ve bütünleştirilmiş kutsal iradenin birliksel eş güdümünü simgeler. Kişilik, yüce bir anlamda, kâinatın âlemlerinin tümü için Tanrı’nın açığa çıkarılışıdır.
\vs p001 5:14 Tanrı, ebedi, evrensel, mutlak ve sınırsız olarak ne bilgelikte ne de bilgide büyümeye devam eder. Tanrı sınırlı insanın eksik bir yargıyla veya algıyla düşünebileceği bir biçimde deneyim kazanmaz; fakat onun kendi ebedi kişiliğinin alanlarında, kendini gerçekleştirmenin devamlı ilerlemesinden faydalanması, evrimsel dünyaların sınırlı yaratılmışları tarafından yeni bir deneyim elde edilmesiyle belli bir biçimde karşılaştırılabilir ve benzetilebilir.
\vs p001 5:15 Engin evren içinde Kâinatın Yaratıcısının, kutsal yardımla ruhani bakımdan kusursuz dünyalara doğru yukarı çıkmayı arzulayan sınırlı her ruhun kişilik mücadelesinde dolaysız olarak katkıda bulunuşu bir gerçek olmasaydı, sınırsız Tanrı’nın mutlak kusursuzluğu, kusursuzluğun koşulsuz kesinliğinden kaynaklanabilecek istenmeyen bağlayıcılıklarının sebebini oluştururdu. Kâinatın âlemlerinin tümü boyunca her ruhani varlığın ve her ölümlü yaratılmışın ilerleyici deneyimi, sonu olmayan kendini gerçekleştirmenin bitmeyen kutsal çevresi içerisinde Tanrı’nın ezelden beri gelişen İlahi\hyp{}bilincinin bir parçasıdır.
\vs p001 5:16 “Tüm sıkıntılarınızda Tanrı’da acı çeker” söylemi kelimenin tam anlamıyla doğru bir yargıdır, tıpkı “tüm zaferlerinizde Tanrı’da sizinle beraber bir başarı elde eder” gerçeği gibi. Onun birey öncesi kutsal ruhu sizin gerçek bir parçanızdır. Cennet Adası kâinatın âlemlerinin tümünün tüm fiziksel başkalaşımlarıyla iç içedir; Ebedi Evlat tüm yaratılanların akli yansımalarını içine alır, ve Birleştirici Bünye tüm akli yansımaları ve genişleyen kâinatı kapsar ve bütünlüğü altına toplar. Zaman ve mekânın evrimsel yaratılmışlarının tümünün kişiliğinde, her varlığın ve birlikteliğin yükselen ruhlarının ve genişleyen akıllarının içerisinde ilerlemeci mücadelelerinin bireysel deneyimlerinin hepsinde Kâinatın Yaratıcısı kutsallık bilincinin bütününde kendisini gerçekleştirir. Bu sebeple “Onun bünyesinde biz hepimiz yaşar, hareket eder ve varlığımıza sahip oluruz” sözünün tüm anlamı kuşkuya yer bırakmayacak bir biçimde doğrudur.
\usection{6.\bibnobreakspace Kâinat İçindeki Kişilik}
\vs p001 6:1 İnsan kişiliği kutsal Yaratan kişiliği tarafından şekillendirilmiş zaman\hyp{}mekân ve ışık\hyp{}gölgenin bir ürünüdür. Bu bakımdan hiçbir gerçekliğin ışığı onun gölgesi incelenerek yeterli bir biçimde kavranamaz. Bunun yerine gölgeler onların esas kaynakları olan gerçek özü bakımından değerlendirilmelidir.
\vs p001 6:2 Tanrı bilime göre bir neden, felsefeye göre bir fikir, dinselliğe göre bir kişilik ve hatta şefkat sahibi cennetlik Yaratıcı’dır. Tanrı bilim adamına göre bir esas kuvvet, filozofa göre bütünlüğün bir hipotezi, din adamına göre yaşayan ruhani bir deneyimdir. İnsanın Yaratıcı Tanrı’nın kişiliği hakkında yeterli olmayan kavramsallaşması yalnızca evrende insanın ruhani gelişimiyle arttırılabilir ve bu kavramsallaşma, yalnızca zaman ve mekânda kutsal yolculuğu yapanların sonunda Cennet’te yaşayan Tanrı’nın kutsal karşılamasına ulaştığında gerçekten yeterli bir anlamsal bütünlüğü haline gelecektir.
\vs p001 6:3 Kişiliğin Tanrı ve insan tarafından birbirine taban tabana zıt idrak edildiği farklı bakış açılarının varlığını hiçbir zaman gözden kaçırmayınız. İnsan kişiliğe sınırlılıktan sınırsızlığa bakmaya çalışarak kişiliği görür ve onu algılar; bunun karşısında Tanrı sınırsızlıktan sınırlılığı görerek onu değerlendirir. İnsan kişiliğin olası en alt seviyesine sahiptir, Tanrı ise en yüksek, hatta en yüce, nihai ve mutlak kişiliği kendi içerisinde barındırır. Bu sebeple, kutsal kişiliğin daha yerinde kavramları insan kişiliğinin gelişmiş düşüncelerinin, özellikle Yaratan Evlat olan Mikâil’in Urantia’daki yaşamının bahşedilişinin insani ve kutsal kişiliğinin gelişmiş dışavurumunun ortaya çıkmasını sabırla beklemek zorunda kalmıştır.
\vs p001 6:4 Fani insan aklında ikame eden birey öncesi kutsal ruh, kendi mevcut varoluşunun geçerli kanıtını kendi mevcudiyetinde taşır, fakat kutsal kişiliğin kavramsallaşması samimi olan kişisel dini deneyimlerin ruhani içeriğiyle ancak algılanabilir. Herhangi bir insan olan veya kutsal olan kişilik, kutsallığın dışsal tepkilerden veya bu insanın maddi mevcudiyetinden bağımsız olarak bilinebilir ve kavranabilir.
\vs p001 6:5 Ahlaki yakınlığın ve ruhani uyumun belli bir derecedeki birlikteliği iki birey arasındaki arkadaşlık için olmazsa olmaz koşullardan biridir; sevgi dolu bir kişilik kendisini kalbinde sevgi taşımayan bireye çok nadiren kendisini anlatabilir. Kutsal bir kişiliğin bilgisine yaklaşmak için bile, insanın kişilik kazanımlarının tümü onun bu yöndeki çabalarına tamamen adanmalıdır; aksi halde yeteri kadar istekli olmayan kısmi bir sadakat boş bir uğraş olacaktır.
\vs p001 6:6 İnsan kendisini ve kendi türünün kişilik değerlerini daha fazla bir bütünlükte anladığı ve takdir ettiğinde kendisi Özgün Kişilik’i tanımak için daha büyük bir arzu duyacak, ve böyle bir Tanrı\hyp{}tanıyan insan daha samimi olarak Özgün Kişilik’e benzemek için yüksek bir gayret gösterecektir. Siz Tanrı hakkında kendi görüşlerinizi tartışabilirsiniz, fakat onunla ve onun içinde yaşanılan deneyim sade bir akli mantığın ve insanın tüm uzlaşılamaz tartışmalarının ötesindedir ve üstündedir. Tanrı\hyp{}tanıyan insan kendi ruhani deneyimlerini, inanmayanları ikna etmek için değil bunun aksine inananların karşılıklı memnuniyeti ve onların aydınlanmasını sağlamak için açıklar.
\vs p001 6:7 Evren bilgisinin bilinebileceğini ve anlaşılabileceğini varsaymak evrenin bir akıl tarafından inşa edildiğini ve bir kişilikle idare edildiğini farz etmektir. İnsan aklı sadece diğer insan veya insan\hyp{}üstü akılların akli olgusallığını algılayabilir. Eğer insanın kişiliği evreni deneyimleyebiliyorsa orada evrenin bir yerinde saklı kutsal bir akıl ve mevcut bir kişiliğin varlığı söz konusudur.
\vs p001 6:8 Tanrı ruh kişiliği olarak ruhaniyettir, ve insanda aynı zamanda bir ruhtur. Nasıralı İsa, insan deneyimlerinde bu ruh kişiliğinin olası kendini gerçekleştirmesinin tamamına ulaşmıştı, bu sebeple onun Tanrı’nın iradesini yerine getirdiği hayatı Tanrı’nın kişililiğinin insan için en nihai ve en gerçek açığa çıkışının temsilidir. Kâinatın Yaratıcısı’nın kişiliği bile sadece mevcut dinsel deneyimde anlaşılabilir, bu bakımdan İsa’nın dünyevi yaşamında, gerçek bir insan deneyimiyle Tanrı’nın kişiliğinin kendini böyle gerçekleştirmesinin ve onun açığa çıkışının kusursuz örneğinden ilham alırız.
\usection{7.\bibnobreakspace Kişilik Kavramının Ruhani Değeri}
\vs p001 7:1 İsa “yaşayan Tanrı” hakkında konuştuğu zaman cennet içindeki Yaratıcı olan bir kişisel İlahiyat’ı kaynak olarak gösterdi. İlahiyat’ın kişilik kavramı amaç birliğinin oluşmasına olanak sağlar; akli ibadeti destekler; ve ferahlatıcı doğruluğu beraberinde getirir. Etkileşim birey olmayan maddeler arasında söz konusu olabilir, fakat amaç birliği ilişkileri için aynı durum söz konusu değildir. Yaratıcı ve evladın Tanrı ve insan olarak amaç birliği ilişkileri ikisinin de kişisel olmayan bünyevi durumlarında mümkün değildir. Her ne kadar kişisel bütünlük sadece Düşünce Düzenleyicisi’nin birey dışı varlığı tarafından fazlasıyla sağlansa da, yalnızca kişilikler birbirleriyle bir bütünlük ve birliktelik oluşturabilirler.
\vs p001 7:2 İnsanın Tanrı’yla olan bütünleşmesi bir su damlacığının okyanustaki tamamlanmışlığını bulabilmesine benzetilemez. İnsan kutsal bütünlüğe ilerleyici olan karşılıklı ruhani bütünlükle, kişisel Tanrı ile kişilik temasıyla, tüm samimiyetle kutsal doğaya artan bir biçimde varışla ve kutsal iradeye akli benimseyişle erişir. Böyle bir yüce ilişki sadece kişilikler arasında gerçekleşebilir.
\vs p001 7:3 Doğrunun kavramsallaşması kişilikten ayrı, olası bir biçimde uygulanabilir, güzelliğin kavramsallaşması kişilikten bağımsız var olabilir, fakat kutsallığın kavramsallaşması sadece kişiliğin ilişkilerinde anlaşılabilir. \bibemph{Sadece kişiliğe sahip bir birey} sevip sevilebilir. Güzellik ve gerçeklik değerleri bile eğer onlar sevgi dolu Yaratıcı olan kişisel Tanrı’nın kendine ait niteliklerinden biri olmasaydı, onlar varlığını devam ettirme içinde olan ümidin kapsamından ayrılırdı.
\vs p001 7:4 Tanrı’nın nasıl ebedi, değişmeyen, her şeye gücü yeten ve kusursuz olduğunu, fakat aynı zamanda sürekli değişen ve açıkça gözle görülebilen yasayla\hyp{}sınırlandırılmış, göreceli sınırlılıkların evrimleşen evreni tarafından nasıl kuşatıldığını bizim tamamiyle anlamamız mümkün değildir. Fakat bünyelerimizin ve yaşadığımız çevrenin sürekli değişmesine rağmen hepimiz iradenin bütünlüğünün ve kişiselliğinin kimliğini idare ettiğimizden dolayı kendi kişisel deneyimlerimizde böyle bir doğruyu \bibemph{bilebiliriz}.
\vs p001 7:5 Nihai evren gerçekliği matematikle, mantıkla veya felsefeyle algılanamaz; ancak bu durum bir kişilik olan Tanrı’nın kutsal iradesine ilerleyici bir biçimde uyumla mümkündür. Ne bilim, ne felsefe, ve ne de din felsefesi Tanrı’nın kişiliğini başlı başına doğrulayabilir. Sadece Cennetsel Yaratıcı’nın inanç sahibi evlatlarının kişisel deneyimleri Tanrı’nın kişiliğinin mevcut ruhani gerçekleştirmeleri üzerinde etkide bulunabilir.
\vs p001 7:6 Evren kişiliğinin daha yüksek seviyedeki kavramsallaşmaları kişiliği, birey\hyp{}bilincini, birey\hyp{}iradesini ve bireyin kendisini gerçekleştirmesi için olasılık anlamlarını çağrıştırır. Ve bu karakteristik özellikler bu anlamlara ek olarak Cennet İlahiyatları’nın kişilik birlikteliğinde olan bu tür diğer ve eşit kişiliklerin amaç birlikteliği anlamını karşılar. Ve bu birlikteliklerin mutlak bütünlükleri o kadar kusursuzdur ki kutsallık tek bir bölünmez bütünlük olarak tanınmaya başlar. “Koruyucu Tanrı \bibemph{tek}dir.” İnsan olan baba figürünün bölünmez karakteri her nasıl fani kız ve erkek evlatların doğmasına engel olmuyorsa, kişiliğin bölünemezliği Tanrı’nın fani insanların kalplerinde yaşaması için kendi ruhunu bahşedişi onun kişiliğinin bölünmezliğine engel olmaz.
\vs p001 7:7 Birliktelik kavramıyla ilişkili bölünmezliğin bu kavramsallaşması İlahiyatın Nihayet’i tarafından zaman ve mekânın aşkınlığını yansıtır; bu sebeple ne zaman ne de mekân mutlak veya sınırsız olabilir. İlk Kaynak ve Merkez tüm akıl, madde ve ruhu aşan bir sınırsızlıktır.
\vs p001 7:8 Ne Cennetin Kutsal Üçlemesi kutsal birlikteliğin gerçekliğine herhangi bir biçimde zarar verebilir, ne de bu üç ebedi kişiliğin varlığından herhangi biri İlahiyat’ın bölünmezliği gerçekliğini değiştirebilir. Cennet İlahiyatı’nın bu üç kişiliği tüm evren gerçeklik yansımalarında ve tüm yaratılmışların ilişkilerinde tek bir bütünündür. Bu noktada, emrim altındaki hiçbir dilin evren hakkında bahsi geçen sorunlarının bize nasıl göründüğü hususunda fani canlı aklını yeterli bir derecede aydınlığa kavuşturmayacağının tamamen bilincindeyim. Fakat siz böyle bir sebepten ötürü güven kaybetmemelisiniz; Cennet varlıkları içerisinde benim topluluğuma ait yüksek kişilikler için bile bu kavramların tamamı apaçık bir aydınlığa kavuşmamıştır. İlahiyat ile alakalı bu derin ve karmaşık gerçeklerin Cennet’e olan uzun fani yükselişin başarılı evreleri süresince sizin aklınızın ilerleyici bir şekilde ruhaniyet kazanımıyla artan bir biçimde açıklık kazanacağını da buna ek olarak unutmayın.
\vs p001 7:9 [Yedinci aşkın\hyp{}evrenin yönetim merkezi olan Uversa üzerinde bulunan Zamanın Ataları tarafından görevlendirilen göksel kişiliklerin bir topluluğunun üyesi olan bir Kutsal Danışman tarafından Nebadon’un yerel evreninin sınırlarının ötesinde olan bu planlı açığa çıkarımların belirtilen kısımlarının denetlenmesi ve yönetilmesi için sunulmuştur. Tanrı’nın doğasını ve niteliklerini tasvir eden bu sayfaların yazımına destek olmak için görevlendirildim. Varlığım herhangi bir yerleşik dünya üzerinde bu görev için muktedir en yüksek bilgi kaynağının temsilcisidir. Yedi aşkın\hyp{}evrenin yedisinde de bir Kutsal Danışman olarak hizmet ve her şeyin merkezi olan Cennet’te yeteri kadar uzunlukta ikamet ettim. Tanrı’nın karşı koyulamaz hâkimiyetiyle birlikte onun doğası ve niteliklerinin gerçekliği ve doğruluğunu tasvir ediyorum; bu bağlamda ne söylediğimin tam anlamıyla bilincindeyim.]
