\upaper{80}{And Topluluklarının Batıdaki Genişlemesi}
\vs p080 0:1 Her ne kadar Avrupalı mavi ırk, kendi çabalarıyla büyük bir kültürel medeniyete erişmemiş olsa da; onun Âdemleşen kolları daha sonraki And istilacıları ile karıştıkları zaman, ilerlemeye arzusu taşıyan medeniyete erişim için eflatun ırkı ve onların And halefleri döneminden beri Urantia üzerinde ortaya çıkmış en yetkin ırk kollarından birini dünyaya getirerek onun biyolojik temeline katkıda bulunmuştur.
\vs p080 0:2 Çağdaş beyaz insan toplulukları, bazısının sarı ve kırmızı fakat daha baskın bir biçimde mavi olduğu Sangik ırkları ile karışmış hale gelen Âdem ırk kolunun varlığını sürdüren ırk kollarını içine almaktadır. Tüm beyaz ırklar içinde ve çok daha fazla olmak üzere öncül Nod ırk kollarında özgün And ırk kökeninin dikkate değer bir oranı bulunmaktadır.
\usection{1.\bibnobreakspace Âdem Unsurları’nın Avrupa’ya Girişi}
\vs p080 1:1 Sonuncu And topluluklarının Fırat vadisinden uzaklaştırılmalarından önce, kardeşlerinin çoğu Avrupa’ya maceraperestler, öğretmenler, tüccarlar ve muzaffer savaşçılar olarak çoktan girmiş bir konumdaydılar. Eflatun ırkının öncül dönemlerinde Akdeniz oluğu Cebelitarık boğazı ve Sicilya kara köprüsü tarafından korunmaktaydı. İnsanların oldukça öncül gerçekleştirdikleri deniz ticaret etkinliklerinin bazıları, kuzeyden gelen mavi toplulukların ve güneyden göç etmiş Sahra bireylerinin doğudan katılan Nod ve Âdem unsurları ile buluştukları kara içindeki bu göllerde kurulmuştu.
\vs p080 1:2 Mezopotamya’nın batı oluğunda Nod unsurları; en geniş kapsamlı kültürlerinden birini hali hazırda oluşturmuş bir konumda bulunarak, bu merkezlerden daha çok kuzey Afrika olmak üzere bir ölçüde güney Avrupa’ya ilerlemiş durumdalardı. Ufuk sahibi Nod\hyp{}Andon Suriyelileri çok önceden, yavaşça yükselmekte olan Nil deltası üzerindeki yerleşkeleri ile ilişki içerisinde çömlekçilik ve tarımı getirmişlerdi. Onlar aynı zamanda koyun, keçi, büyükbaş ve diğer evcilleştirilmiş hayvanları buraya ithal etmiş olup, madeni eşya yapımın fazlasıyla gelişmiş yöntemlerini bu bölgeye getirdiler. Suriye bu dönemlerin bahse konu alandaki üretim merkeziydi.
\vs p080 1:3 Otuz bin yıldan daha fazla bir süre boyunca Mısır, Nil vadisininkini geliştirmek için sanat ve kültürlerini beraberinde getiren Mezopotamya topluluklarının devamlı bir göç hareketine uğramıştır. Ancak geniş sayıdaki Sahra topluluklarının girişi Nil boyunca uzanan öncül medeniyeti, Mısır’ı yaklaşık on beş bin yıl öncesi kültür düzeyine indirecek kadar kötüleştirdi.
\vs p080 1:4 Ancak öncül dönemler boyunca, Âdem unsurlarının batı yönündeki göçünü engelleyecek çok az etki bulunmaktaydı. Sahra, sürü ve tarım toplulukları tarafından geniş sayılarla ikamet edilmiş açık bir otlak arazisiydi. Bu Sahra toplulukları hiçbir zaman el eşyası üretimine girişmemişlerdi; buna ek olarak onlar şehir kuran topluluklar da değillerdi. Onlar, yok olmuş durumdaki yeşil ve turuncu ırkların geniş ırk kollarını taşıyan bir çivit\hyp{}siyah topluluktu. Ancak onlar, yer kabuğu yükselişinin ve yağmur yüklü bulutların yön değiştirişinin bu bayındır ve barışçıl medeniyetin kalıntılarını etrafa saçmasından önce, eflatun kalıtımından oldukça küçük bir miktar aldılar.
\vs p080 1:5 Âdem’in kanı, birçok insan ırkı tarafından paylaşılmıştır; ancak onların bazıları bunu diğerlerinden daha fazla bünyelerine kattı. Hindistan’ın melez ırkları ve Afrika’nın daha koyu renkli toplulukları Âdem unsurlarına çekici gelmemekteydi. Kırmızı insanlar Amerika kıtalarında fazlasıyla tecrit edilmiş bir konumda bulunmuş olmasalardı onlar ile oldukça özgür bir biçimde karışmış olacaklardı; ve buna ek olarak onlar, sarı ırka sevecen bir biçimde yaklaşmaktalardı; ancak bu topluluğa Asya’nın ücra yerleşkelerinde erişmek benzer bir şekilde zordu. Bu nedenle onlar oldukça doğal bir biçimde; macera veya toplumsal fedakârlık dürtüsü ile harekete geçirildikleri zaman, veya Fırat nehrinden dışarı doğru itildikleri zaman, Avrupa’nın mavi ırkları ile bütünlük kurmayı tercih etmişlerdi.
\vs p080 1:6 Bu dönemde Avrupa’da baskın bir konumda bulunan mavi ırklar, göç etmekteki öncül Âdem unsurlarının hoşuna gitmeyecek hiçbir dini âdete sahip değildi; ve burada, eflatun ile mavi ırklar arasında büyük bir cinsel çekim bulunmaktaydı. Mavi insanların en iyileri, Âdem unsurları ile çiftleşmelerine izin verilmelerini büyük bir onur olarak addetti. Her mavi insan, belli başlı Âdem kadınlarının kalbini kazanabilmek için çok kabiliyetli ve sanatkâr hale gelme arzusunu taşımıştı; ve üstün bir mavi ırk kadının en yüksek gayesi, bir Âdem unsurunun kendisine ilgi beslemesiydi.
\vs p080 1:7 Kademeli bir biçimde Cennet Bahçesi’nin bu göç eden evlatları; Neanderthal ırk kökeninin hala varlığını sürdüren ırk kollarını acımasız bir biçimde ortadan kaldırırken, kültürel adetlerini canlandırarak mavi ırkın yüksek türleri ile bütünleştiler. Irkların karışımının bu işleyiş biçimi, alt düzey ırk kökenlerinin ortadan kaldırılışıyla bir araya geldiğinde, üstün mavi ırkların bir düzine veya daha fazla sayıdaki güçlü ve ilerleyici topluluğunu yarattı.
\vs p080 1:8 Bu ve diğer nedenlerden dolayı, yoksa göçün daha elverişli istikametleri buradan geçtiği için değil, Mezopotamya kültürünün öncül dalgaları neredeyse bütünüyle ayrıcalıklı bir biçimde yönünü Avrupa’ya çevirmişti. Ve bu koşullar, çağdaş Avrupa medeniyetinin kökensel tarihini belirlemişti.
\usection{2.\bibnobreakspace İklim ve Yeryüzü Oluşumlarındaki Değişiklikler}
\vs p080 2:1 Avrupa’ya olan eflatun ırkının öncül genişlemesi; belli başlı, daha çok ansızın gerçekleşmiş iklim ve yeryüzü değişiklikleri tarafından yarıda kesildi. Kuzey buz tabakalarının geri çekilmesiyle birlikte batıdan gelen su yüklü rüzgârlar, kademeli bir biçimde Sahra’nın açık otlak arazilerini çorak bir çöle dönüştürerek kuzeye doğru kaydı. Bu kuraklık, büyük Sahra yüksek düzlüğünün siyah gözlü ancak zeki sakinleri olarak buranın daha minyon kumral topluluklarını dağıttı.
\vs p080 2:2 Daha saf çivit unsurları, bu dönemden beri kalmakta oldukları merkezi Afrika’ya doğru güney yönünde hareket ettiler. Daha karma topluluklar üç yönde dağıldılar: batıdaki üstün kabileler İspanya’ya ve oradan da, daha sonraki Akdeniz’in zeki kumral ırklarının çekirdeğini oluşturan bir biçimde, Avrupa’nın komşu kısımlarına göç ettiler. Sahra yüksek düzlüğünün doğusunda bulunan en az gelişmiş topluluk Arabistan’a ve oradan da kuzey Mezopotamya ve Hindistan boyunca ilerleyerek Sri Lanka’nın derinliklerine göç etti. Merkezde bulunan topluluk kuzeye, sonra Nil vadisinin doğusuna ve oradan da Filistin’e hareket etti.
\vs p080 2:3 Deccan’dan İran, Mezopotamya ve Akdeniz’in iki kıyısı boyunca dağılmış olan çağdaş insan toplulukları arasında belirli düzeydeki mevcut akrabalığı bu ikincil Sangik alt\hyp{}kolu açıklamaktadır.
\vs p080 2:4 Afrika’daki bu iklim değişikliklerinin gerçekleştiği zaman zarfında; İngiltere kıtadan ayrılmış, Danimarka su yüzüne çıkmış, Akdeniz’in batı havzasını koruyan Cebelitarık boğazı ise bu kara gölünü hızlı bir biçimde Atlas Okyanusu seviyesine yükselten bir depremin sonucunda sular altında kaldı. Yakın bir zaman içerisinde Sicilya kara köprüsü, Akdeniz’e ait denizlerden bir tanesini yaratarak ve onu Atlas Okyanusu’na bağlayarak sular altında kaldı. Bu doğal afet; insan yerleşkelerinin birçoğunu sular altında bırakıp, tüm dünya tarihi içinde sele kurban verilmiş en büyük insan kaybına neden oldu.
\vs p080 2:5 Akdeniz havzasının bu sular altında kalışı Âdem unsurlarının batı yönündeki göçlerini doğrudan bir biçimde engellerken, Sahra topluluklarının büyük akınları gittikçe artan sayıdaki üyelerinin Cennet Bahçesi’nin kuzeyine ve doğusuna doğru kaçış yolu aramasına neden oldu. Âdem soyları Dicle ve Fırat vadilerinden kuzeye doğru ilerlerken, dağ setleri ve bu dönemde genişlemiş Hazar Denizi ile karşılaştılar. Ve birçok nesil boyunca Âdem soyları, Türkistan’a dağılan yerleşkeleri etrafında avlanıp, sürülerini güdüp topraklarını işlediler. Kademeli bir biçimde bu muhteşem insanlar yerleşkelerini Avrupa’ya doğru genişlettiler. Ancak bu aşamada Âdem unsurları; Avrupa’ya doğudan giriş yapmış olup, mavi insanların kültürünü Asya’dakinin binlerce yıl gerisinde bulmuşlardı; bu durumun nedeni bahse konu bölgenin Mezopotamya ile neredeyse hiçbir irtibatının bulunmayışıydı.
\usection{3.\bibnobreakspace Cro\hyp{}Magnon Mavi Irkı}
\vs p080 3:1 Mavi insanın tarihi kültür merkezleri, Avrupa’nın tüm nehirleri boyunca konumlanmıştı; ancak mevcut an içerisinde sadece Somme, buzul hareketleri öncesi dönemlerdeki yatağında akmaktadır.
\vs p080 3:2 Avrupa kıtasını kaplamış mavi insanlar hakkında söz ederken, içlerinde birçok farklı ırk türünün barınmış oldu belirtilmelidir. Otuz beş bin yıl öncesinde bile Avrupalı mavi insanlar hali hazırda, kırmızı ve sarı ırk kollarını taşıyan oldukça melez bir insan topluluğu konumundaydılar; bunun yanı sıra Atlas Okyanusu kıyı şeridindeki yerleşkelerde ve bugünün Rusya bölgesinde onlar, Andon kanının dikkate değer bir miktarını çoktan almış olup, güneyde Sahra toplulukları ile irtibat halindelerdi. Ancak, bu kadar çok ölçüdeki ırk topluluklarını teker teker sıralamaya girişmek beyhude bir çabadan ötesine geçmeyecektir.
\vs p080 3:3 Bu öncül Âdem\hyp{}sonrası dönemin Avrupa medeniyeti, Âdem unsurlarının yaratıcı hayal gücü ile birlikte mavi insanların zindeliği ve maharetinin benzersiz bir karışımıydı. Mavi ırklar büyük bir diriliğe sahip ırktı; ancak onlar Âdem unsurlarının kültürel ve ruhsal düzeyini fazlasıyla kötüleştirdiler. Birçoğunun eşlerini aldatma ve evlenmemiş kızları kötü yola düşürme eğilimi nedeniyle, onların daha sonra Cro\hyp{}Magnon unsurlarını dinleriyle etkilemeleri oldukça zordu. On bin yıl boyunca Avrupa’da din, Hindistan ve Mısır’da yaşanan gelişmelere kıyasla zayıflamış bir konumda bulunmaktaydı.
\vs p080 3:4 Mavi insanlar ilişkilerinde kusursuz bir biçimde dürüst olup, melez Âdem unsurlarının cinsel temelli kötülüklerinden tamamiyle uzaklardı. Onlar, yalnızca savaşlar erkek sayılarındaki bir nüfus azlığını yarattığında çok eşliliğe başvurarak, bekâretliğe saygı duydular.
\vs p080 3:5 Bu Cro\hyp{}Magnon insan toplulukları, cesur ve ileri görüşlü bir ırktı. Onlar, çocuk kültürünün etkin bir düzenini idare etti. Ebeveynlerin ikisi de bu çabalara katıldı; ve daha büyük çocuklardan tamamiyle faydalanıldı. Her çocuk mağaraların bakımı, sanat ve çakmaktaşı yapımında dikkatli bir biçimde eğitildi. Erken bir yaşta kadınlar ev sanatlarında ve ilkel tarımcılıkta oldukça hünerli iken, erkekler yetenekli avcılar ve cesur savaşçılardı.
\vs p080 3:6 Mavi insanlar avcılar, balıkçılar ve yiyecek toplayıcılardı; onlar tekne yapımı uzmanıydılar. Taş baltaları yapıp, ağaçları kesip, bir kısmı toprağın altında olmak üzere hayvan derilerinden çatılara sahip kütük kulübeleri diktiler. Ve Sibirya’da hala benzer kulübeleri inşa eden insan toplulukları bulunmaktadır. Güney Cro\hyp{}Magnon unsurları genellikle gerçek veya yapay mağaralarda yaşadılar.
\vs p080 3:7 Kışın çetin dönemlerinde korumalarının soğuktan donacak bir biçimde mağara girişlerinde nöbette beklemesi hiç de nadir bir durum değildi. Onlar cesarete sahiplerdi; ancak bunun üstünde onların tümü sanatkârdılar; Âdem topluluklarının karışımı birden bire yaratıcı hayal gücünü hızlandırdı. Mavi insanın sanatının doruk noktası, Afrika’dan İspanya boyunca daha koyu tenli ırkların geldiği dönemlerin evvelinde gerçekleşen bir biçimde, yaklaşık olarak on beş bin yıl önce yaşanmıştı.
\vs p080 3:8 Yaklaşık on beş yıl önce Alp ormanları geniş bir biçimde yayılmaktaydı. Avrupalı avcılar, dünyanın şen av bölgelerini kurak ve çorak çöllere çeviren aynı iklim baskısı tarafından nehir vadilerine ve deniz kıyılarına sürüklenmişlerdi. Yağmur rüzgârları kuzeye doğru kayarken, Avrupa’nın önü açık geniş otlak arazileri ormanlar ile kaplanmış hale gelmişti. Bu büyük ölçekli ve görece ani gerçekleşen iklim değişiklikleri Avrupa ırklarını açık arazi avcılarından sürü sahiplerine ve bir ölçüde toprak toplayıcısı ve ekicisine doğru değişmeye itmişti.
\vs p080 3:9 Bu değişiklikler, bir yanda kültürel gelişimlere yol açarken, belirli biyolojik gerilemeleri yaratmıştı. Geçmiş avlanma döneminde üstün olan kabileler, savaş esirlerinin daha yüksek türleri ile karşılıklı olarak evlenmiş ve alt düzeyde gördükleri unsurları kesin bir biçimde yok etmişlerdi. Ancak yerleşkeler kurup, tarım ve ticaret faaliyetine girişirlerken onlar; vasat düzeyde bulunan esirlerin birçoğunun yaşamını köleler olarak bağışlamaya başlamışlardı. Ve Cro\hyp{}Magnon türünün tamamını daha sonra oldukça fazla bir biçimde kötüleştiren bu kölelerin doğumuydu. Kültürün bu gerileyişi, Cro\hyp{}Magnon tür ve kültürünü hızlı bir biçimde kendi üstünlüğü altında içine katarak ve beyaz ırkların medeniyetini başlatarak Mezopotamyalılar’ın nihai ve topluca olarak gerçekleştirdikleri akınlar Avrupa’yı teslim alana kadar doğudan gelen yeni etkilerle şiddetlenmeye devam etti.
\usection{4.\bibnobreakspace Avrupa’nın And İstilaları}
\vs p080 4:1 And toplulukları Avrupa’ya düzenli bir seyirde hareket etmekte iken, burada yeni ana akın gerçekleşmişti; at sırtındaki son göçler üç büyük dalga halinde meydana gelmişti. Bazıları Avrupa’ya, Ege’nin adaları ve Tuna vadisinin kuzeyi üzerinden giriş yapmıştı; ancak daha önceki ve daha saf olan ırk kollarının çoğunluğu kuzeybatı Avrupa’ya, Volga ve Don’un otlak arazileri boyunca uzanan kuzey istikameti üzerinden göç etmişti.
\vs p080 4:2 Üçüncü ve dördüncü akınlar arasında Andon topluluklarının bir birliği, Sibirya’dan Rusya ırmakları ve Baltık bölgesi üzerinden gelerek, kuzey doğrultusundan Avrupa’ya giriş yapmışlardı. Onlar kısa bir süre içinde, kuzey And kabileleri üstünlüğünde onlara karışmışlardı.
\vs p080 4:3 Daha saf eflatun ırkının daha erken dönemdeki genişlemeleri, daha sonraki yarı\hyp{}askeri ve toprak kazanmayı seven And soylarınınkine kıyasla çok daha barışçıldı. Âdem unsurları barışçıldı; Nod unsurları kavgacılardı. Daha sonra Sang ırklarına da karışan bir biçimde bu ırk kökenlerinin birleşimi, mevcut askeri fetihleri gerçekleştirmiş olan, yetkin ve saldırgan And topluluklarını meydana getirdi.
\vs p080 4:4 Ancak at, Batı’daki And topluluklarının üstünlüğünü belirleyen evrimsel bir etkendi. At; And atlılarının son topluluklarının Hazar Denizi üzerinden çabucak geçerek Avrupa’nın tamamına yayılmalarını mümkün kılan bir biçimde, bu zamana kadar mevcut olmayan hareket kabiliyeti üstünlüğünü onların etrafa dağılan topluluklarına vermiştir. And topluluklarının daha önceki dalgalarının tümü o kadar yavaş bir biçimde hareket etmişlerdi ki, Mezopotamya’dan uzaklaşır uzaklaşmış birbirlerinden kopma eğilimi göstermişlerdi. Ancak daha sonraki dalgalar o kadar hızlı bir biçimde hareket ettiler ki, daha yüksek kültürün belli bir nüvesini hala ellerinde bulunduran bir biçimde bütüncül topluluklar halinde Avrupa’ya ulaştılar.
\vs p080 4:5 Çin ve Fırat bölgesi dışında yerleşik dünyanın tümü, çetin And atlılarının İsa’dan önceki altıncı ve yedinci bin yılda tarih sahnesine çıkışlarından önceki on bin yıl boyunca çok sınırlı bir ölçekte kültürel gelişimi gerçekleştirmişlerdi. Rusya düzlüklerinden batı yönünde ilerlerken, mavi ırkın en iyi unsurlarını içlerine katarak ve en kötülerini yok ederek, bir topluluk halinde bütünleştiler. Bu unsurlar; İskandinavya, Alman ve Anglosakson topluluklarının büyük dedeleri olan İskandinav ırkları olarak adlandırmakta olduğunuz topluluğun atalarıydılar.
\vs p080 4:6 Üstün mavi ırk kollarının kuzey Avrupa’nın tamamı boyunca And toplulukları tarafından onların üstünlüğü altında bütünüyle karışmalarının üstünden çok uzun bir süre geçmemiştir. Yalnızca Sami (ve bir ölçüye kadar Breton için de dâhil olmak üzere) eski Andon toplulukları kimliksel görünüşlerini bile korudular.
\usection{5.\bibnobreakspace Kuzey Avrupa’nın And Fethi}
\vs p080 5:1 Kuzey Avrupa kabileleri, Türkistan güneyindeki Rus bölgeleri boyunca Mezopotamya’dan gelen devamlı göç hareketleri tarafından sürekli bir biçimde güçlendirilmekte ve ilerleyici bir biçimde canlandırılmaktaydı; ve And atlılarının son dalgaları Avrupa’yı sardığında, bu bölgede hali hazırda; dünyanın geri kalan kısmının tümünde rastlanabilecek olandan daha çok sayıda kişi And kökenine sahipti.
\vs p080 5:2 Üç bin yıl boyunca kuzey And topluluklarının askeri yönetim merkezi Danimarka’daydı. Bu merkezi yerleşkeden fetih hareketlerinin birbirlerini takip eden dalgaları hareket etti; bu dalgalar, ilerleyen çağlar Mezopotamya’lı fatihlerin elde edilen yerleşke insanlarıyla nihai karışımlarına şahit olurken azalan bir biçimde And ve artan bir ölçüde beyaz hale gelmişti.
\vs p080 5:3 Her ne kadar mavi insan, kuzeyde bulunan toplulukların üstünlüğünde onlara karışmış ve güneye ilerleyen beyaz atlı akıncılarına nihai olarak karşı koyamamış olsa da; melez beyaz ırkın genişleyen kabileleri, Cro\hyp{}Magnon topluluklarının inatçı ve uzun süreler devam etmiş direnciyle karşılaştı; ancak üstün us ve sürekli artan biyolojik köken unsurları eski ırkı ortan kaldırmakta kendilerini yetkin kıldı.
\vs p080 5:4 Beyaz ve mavi ırk arasında meydana gelen belirleyici mücadeleler Somme vadisinde verildi. Burada mavi ırkın yetkin soyları güneye doğru hareket eden And topluluklarıyla kıyasıya savaştı; ve beş yüzden fazla yıl boyunca bu Cro\hyp{}Magnon unsurları, beyaz istilacılarının üstün askeri stratejilerine teslim olmadan önce topraklarını başarıyla savundular. Somme’de verilen son savaşta kuzey ordularının muzaffer kumandanı olan Tor; kuzeyli beyaz kabilelerinin kahramanı haline gelip, daha sonra bazıları tarafından bir tanrı olarak kutsandı.
\vs p080 5:5 Mavi insanın varlığını en uzun süre devam ettirmiş kaleleri güney Fransa’da bulunmaktaydı; ancak sonuncu en büyük askeri direnç Somme boyunca kırılmıştı. Daha içerilerin fethi; ticari faaliyetler, nehirler arasındaki nüfus baskısı ve alt düzeyde bulunan unsurların acımasız bir biçimde ortadan kaldırılışıyla beraber üstün olanlar ile karşılıklı olarak evlenilmeye devam edilişi vasıtasıyla ilerledi.
\vs p080 5:6 And unsurlarının büyüklerinden oluşan kabile heyeti alt düzeyde bulunan bir esirin yetersiz olduğuna hüküm verdiğinde özenle hazırlanmış merasim eşliğinde bu kişi, kendisini nehre kadar eşlik eden ve --- ölümcül boğma faaliyeti olan --- “mutlu av alanlarına” kabul törenini uygulayan şaman din adamlarına teslim edilmekteydiler. Böylelikle Avrupa’nın beyaz istilacıları, kendilere ait düzeylere kolay bir biçimde karışmayan karşılaştıkları her topluluğu ortadan kaldırmışlardı; ve böylelikle mavi insanlar --- hem de hızlı bir biçimde --- sona yaklaşmışlardı.
\vs p080 5:7 Cro\hyp{}Magnon mavi insanlar topluluğu, çağdaş Avrupa ırklarının biyolojik temelini oluşturmuştu; ancak onlar, anavatanlarının daha sonraki kudretli fatihleriyle onların üstünlüğünde karıştıkları bir biçimde varlıklarını devam ettirmişlerdi. Bu mavi ırk kolu, Avrupa’nın beyaz ırklarına birçok sağlam karakter ve fazla fiziksel kudreti sağlamıştı; ancak melez Avrupa ırklarının mizah ve hayal gücü And topluluklarından gelmekteydi. Kuzeyli beyaz ırkların meydana gelişi ile sonuçlanan bu And\hyp{}mavi birliktelik, geçici bir doğanın kötüleşmesi olarak And medeniyetinde gerçekleşen doğrudan bir duraklamayı açığa çıkarmıştı. Bu kuzeyli barbarların gün yüzüne çıkmamış üstünlüğü kendisini nihayeten ortaya çıkarmış ve bugünün Avrupa medeniyetinin oluşumuyla sonuçlanmıştır.
\vs p080 5:8 M.Ö. 5000’li yıllarda evrim halindeki beyaz ırklar; kuzey Almanya, kuzey Fransa ve Britanya Adaları’nı içine alan bir biçimde, kuzey Avrupa’nın tümü boyunca baskın bir halde bulunmaktaydı. Merkezi Avrupa bir süreliğine, mavi insan ve yuvarlak kafatasına sahip Andon unsurları tarafından yönetilmekteydi. Daha sonrakiler başlıca olarak Tuna vadisinde konumlanmış olup, And toplulukları tarafından hiçbir zaman tümüyle yerlerinden edilmediler.
\usection{6.\bibnobreakspace Nil Irmağı Boyunca Yerleşik And Toplulukları}
\vs p080 6:1 Dönemsel And göçlerinin gerçekleştiği zamanlardan beri Fırat vadisinde kültür düzeyi düşüş gösterdi; ve medeniyetin ana merkezi Nil vadisine kaydı. Mısır, dünya üzerindeki en gelişmiş topluluğun yönetim merkezi olarak Mezopotamya’nın halefi haline geldi.
\vs p080 6:2 Nil vadisi, Mezopotamya vadilerinden kısa bir süre önce sellerden zarar görmeye başladı; ancak bu vadi bahse konu durumla daha iyi başa çıkabildi. Bu öncül gerilme, And göçmenlerinin devamlı gerçekleşen akımı tarafından telafi edilmekten öte bir duruma ulaştı; böylelikle Mısır kültürü, gerçekte her ne kadar kökenini Fırat bölgesinden alsa da, diğerinin önüne geçen bir görünüş sergiledi. Ancak M.Ö. 5000’li yıllarda Mısır’da, Mezopotamya’daki sel dönemi boyunca, yedi farklı insan topluluğu bulunmaktaydı; bir tanesi dışında onların hepsi, Mezopotamya’dan gelmişti.
\vs p080 6:3 Fırat vadisinden gerçekleşen son büyük göç ortaya çıktığında Mısır, en hünerli sanatkâr ve zanaatkârların birçoğunu elde etmede şanlıydı. Bu And zanaatkârları; nehir yaşamına, taşkınlarına, tarımsal sulamada kullanılmasına ve kurak mevsimlerine tümüyle aşina olmaları bakımından kendilerini oldukça evlerinde hissetmişlerdi. Onlar, Nil vadisinin tecritsel konumunu memnuniyetle deneyimlemişlerdi; onlar, Fırat nehri boyunca gerçekleşen düşmansı saldırı ve yağmalara çok daha az maruz kalmaktaydılar. Ve onlar, Mısırlılar’ın madeni eşyaları işleme hünerlerine fazlasıyla katkıda bulundular. Burada onlar, Karadeniz bölgelerininki yerine Sina Dağı’ndan gelen demir cevherlerini işlediler.
\vs p080 6:4 Mısırlılar oldukça öncül bir biçimde, yerel ilahiyatlarını tanrıların detaylı bir milli sistemi altına birleştirdiler. Onlar geniş bir tanrı bilimi geliştirmiş olup, yine bu ölçüde geniş ancak ağır bir din adamlığı düzenini inşa ettiler. Birkaç farklı önder, Seth topluluklarının öncül dinsel öğretilerinden geriye kalanları tekrar canlandırma peşine düştü; ancak bu çabalar kısa ömürlü olmuştu. And toplulukları Mısır’da ilk taş yapılarını inşa etti. Taş piramitlerin ilk ve en seçkin olanı bir mimari And dehası olan İmhotep tarafından başbakanlığı döneminde dikilmişti. Daha önceki yapılar tuğlalardan inşa edilmekteydi; ve birçok taş bina hali hazırda dünyanın farklı bölgelerinde dikilmiş bir konumda iken, bu türden bir yapı Mısır’da ilk kez meydana gelmekteydi. Ancak bu büyük mimarın döneminden itibaren yapı sanatı sürekli bir biçimde gerileme gösterdi.
\vs p080 6:5 Kültürün bu parlak çağı, Nil boyunca verilen iç savaşlar tarafından yarıda kesildi; ve ülke yakın bir zaman içerisinde, Mezopotamya’da gerçekleşmiş olduğu gibi, konuksever olmayan Arabistan’dan gelen alt düzey kabileler ve güneyden gelen siyah topluluklar tarafından kaplandı. Sonuç olarak toplumsal ilerleme, beş yüz yıldan daha fazla bir süre boyunca sürekli bir biçimde azalma gösterdi.
\usection{7.\bibnobreakspace Akdeniz Adaları’nın And Toplulukları}
\vs p080 7:1 Mezopotamya’daki kültürün çöküşü boyunca üstün bir medeniyet, doğu Akdeniz’in adaları üzerinde belli bir süre boyunca varlığını sürdürmeye devam etti.
\vs p080 7:2 Yaklaşık olarak M.Ö. 12.000’li yıllarda And topluluklarının muhteşem bir kabilesi Girit’e göç etti. Burası, bu türden üstün bir topluluk tarafından bu kadar öncül bir biçimde yerleşilen tek adaydı; ve bu hareket, bahse konu denizcilerin sahip oldukları soyların komşu adalara olan yayılışlarından neredeyse iki bin yıl önce gerçekleşmişti. Bu topluluk, kuzey Nod unsurlarının Van birimleri kesimi ile karşılıklı olarak bütünleşmiş bir konumda bulunan küçükbaşlı minyon And unsurlarıydılar. Onların tümü uzunluk bakımından bir seksen beş santimetrenin altında olup, daha büyük sayılarda bulunan alt düzeyde akranları tarafından kelimenin tam anlamıyla karadan sürülen insanlardı. Girit’e hareket eden bu göçmenler; dokuma, maden, çömlekçilik ve tesisat işlerine ilaveten yapı malzemesi olarak taşın kullanılmasında oldukça hünerlilerdi. Onlar yazıya girişmiş olup, sürü sahipleri ve tarımla uğraşan bireyler olarak yaşamlarını sürdürmüşlerdi.
\vs p080 7:3 Girit’in yerleşik hale gelişinden yaklaşık olarak iki bin yıl sonra Âdemoğlu’nun uzun soyları, Mezopotamya’nın kuzeyindeki dağ evlerinden neredeyse doğrudan bir biçimde gelerek kuzey adalar boyunca Yunanistan’a ulaşmışlardı. Yunanlılar’ın bu ataları, Âdemoğlu ve Ratta’nın doğrudan bir soy üyesi olan Sata tarafından batı istikametinde yönlendirilmişti.
\vs p080 7:4 Yunanistan’da nihai olarak yerleşen bu topluluk, Âdemoğlu topluluklarının ikinci medeniyetinin sonunu oluşturan üç yüz yetmiş beş seçilmiş ve üstün insandan meydana gelmişti. Âdemoğlu’nun bu daha sonraki soyları, ortaya çıkmaktaki beyaz ırkların bu dönemlerdeki en değerli ırk kollarını taşımaktaydı. Onlar yüksek bir us düzeyine ait olup, fiziksel bakımdan ilk Cennet Bahçesi döneminden beri ortaya çıkmış en güzel insanlardı.
\vs p080 7:5 Yakın bir zaman içerisinde Yunanistan ve Ege Adalar bölgesi, Batı’nın ticaret, sanat ve kültür merkezi olarak Mezopotamya ve Mısır’ın yerine geçti. Ancak Mısır’da olduğu gibi, Ege dünyasının tüm sanat ve bilimi Yunanlılar’ın öncülleri olan Âdemoğlu topluluk kültürü dışında neredeyse tamamen Mezopotamya’dan elde edilmişti. Bu daha sonraki insanların sanat ve dehalarının tümü, Âdem ve Havva’nın ilk oğlu olan Âdemoğlu ve Prens Caligastia’nın öz Nod çalışanlarının saf bir soy kulundan gelen bir kız evladı olarak onun olağanüstü ikinci karısının doğrudan bir soy mirasıydı. Yunanlılar’ın, kendilerinin tanrılar ve insan\hyp{}üstü varlıklarından doğrudan türemiş olduklarını anlatan tarihi mitolojik anlatılara sahip olmaları şaşılası güç bir durum değildir.
\vs p080 7:6 Ege denizi, her biri diğerinden daha az ruhsal olan beş farklı kültür aşamasından geçmişti; ve çok geçmeden sanatın son muhteşem çağı, Yunanlılar’ın daha sonraki nesillerinin dışarıdan getirmiş olduğu Tuna kölelerinin hızlıca çoğalan vasat soylarının etkisi altında ortadan kaybolmuştu.
\vs p080 7:7 Kabil’den gelen soylara ait olan \bibemph{anne inancı} en büyük ilgisine bu dönemde ulaşmıştı. Bu inanç, “yüce anne” ibadeti içerisinde Havva’yı yüceltmişti. Havva’nın temsilleri her yerdeydi. İnsanların topluca kullandığı binlerce tapınak, Girit ve Ön Asya boyunca dikilmişti. Ve bu anne inancı, İsa’nın dünyadaki annesi olan Meryem’in yüceltilmesi ve ona ibadet biçimi altında öncül Hıristiyan dini içinde daha sonra eklemlenmiş hale gelerek İsa’nın yaşadığı dönemlere kadar varlığını sürdürdü.
\vs p080 7:8 M.Ö. Yaklaşık 6500’lü yıllarda, And topluluklarının ruhsal mirasında büyük bir gerileme gerçekleşmişti. Âdem’in soyları geniş çaplı bir biçimde etrafa yayılmış ve eski ve sayıca daha fazla olan insan ırkları tarafından neredeyse tamamen yutulmuştu. Ve And medeniyetinin bu soyu, dini ortak kabullerinin ortadan kalkması ile birlikte, dünyanın ruhsal olarak fakirleşmiş ırklarını acınası bir konumda bıraktı.
\vs p080 7:9 M.Ö. 5000’li yıllarda Âdem soylarının en saf üç ırk kolu Sümer yerleşkesinde, kuzey Avrupa ve Yunanistan’da bulunmaktaydı. Mezopotamya’nın tamamı, Arabistan’dan kademeli bir biçimde gelen karma ve koyu tenli ırkların akımı tarafından yavaşça kötüleşmekteydi. Ve bu alt düzey toplulukların gelişi, And unsurlarının biyolojik ve kültürel kalıntısının daha fazla dışarı doğru dağılmasına katkıda bulundu. Verimli hilal bölgenin tamamından daha maceraperest topluluklar batı yönünde adalara doğru göçte bulundu. Bu göçmenler hem tahıl hem de sebzeler ektiler; ve onlar kendileriyle birlikte evcilleştirilmiş hayvanları getirdiler.
\vs p080 7:10 M.Ö. 5000’li yıllarda Mezopotamyalılar’ın büyük bir topluluğu, Fırat vadisinden çıkıp, Kıbrıs adasına yerleşti; bu medeniyet, kuzeyden gelen barbar topluluklardan sonraki yaklaşık iki bin yıl içerisinde ortadan kaldırıldı.
\vs p080 7:11 Başka bir büyük topluluk Mezopotamya üzerinde Kartaca’nın daha sonraki yerleşkesi yakınında yerleşmişti. Ve bu kuzey Afrika bölgesinden And topluluklarının geniş sayıdaki unsurları İspanya’ya girip daha sonra, öncesinde Ege Adaları üzerinden İtalya’ya gelmiş olan, kardeşleri ile İsviçre’de bir araya gelmişlerdi.
\vs p080 7:12 Mısır Mezopotamya’yı kültürel çöküşte takip ettiği zaman, daha yetkin ve gelişmiş ailelerin birçoğu, hali hazırda var olan gelişmiş bir medeniyet düzeyini böylelikle fazlasıyla arttıran bir biçimde, Girit’e göçtü. Ve Mısır’dan gelen alt düzey toplulukların varışı daha sonra Girit medeniyetini tehdit ettiği zaman, daha kültürlü aileler batı yönünde Yunanistan’a göç etti.
\vs p080 7:13 Yunanlılar yalnızca büyük öğretmen ve sanatkâr değillerdi, onlar aynı zamanda dünyanın en büyük tüccarları ve sömürgeleştiren bireyleriydiler. Sanatını ve ticaretini nihai bir biçimde yok eden alt düzeyin seline teslim olmadan önce onlar, öncül Yunan medeniyetinin güney Avrupa’nın daha sonraki topluluklarında varlığını sürdüren oldukça fazla sayıdaki gelişmelere sebebiyet veren bir biçimde birçok kültür merkezini batıya yeşertmede başarılı olmuşlardı; ve bu Âdemoğlu topluluklarının karma soylarının çoğu, komşu kıtaların kabileleriyle bütünleşmiş bir konuma gelmişlerdi.
\usection{8.\bibnobreakspace Tuna Andon Toplulukları}
\vs p080 8:1 Fırat vadisinin And toplulukları; mavi insanlar ile karışan bir biçimde kuzey doğrultusunda Avrupa’ya, Sahra ve güney mavi ırklarının karışımının geride kalanlarıyla bütünleşen bir biçimde batıda Akdeniz bölgelerine göç etmişlerdi. Ve beyaz ırkın bu iki kolu, eskiden ve şimdi de, bu merkezi bölgelerde uzun süreler boyunca ikamet etmiş bir konumda hali hazırda bulunan öncül Andon kabilelerinin geniş ve yassı şekilli kafatası yapısına sahip yükseltilerde yaşayan hayatta kalmış unsurları tarafından geniş bir biçimde ayrılmışlardı.
\vs p080 8:2 Andon’un bu soyları, merkezi ve güneydoğu Avrupa’nın dağlık bölgelerinin birçoğu boyunca dağılmışlardı. Onlar sıklıkla, dikkate değer bir ölçüde güç ile ikamet ettikleri Ön Asya’dan gelen göçmenler tarafından güçlenmişti. Tarihi Hitit toplulukları doğrudan bir biçimde Andon ırk kökeninden gelmekteydi; onların beyaz tenleri ve geniş yassı kafa yapıları bu ırkın başat özellikleriydi. Bu ırk kolu; İbrahim’in soyunda taşınmış olup, her ne kadar And topluluklarından elde ettikleri bir kültür ve dine sahip olsalar da oldukça farklı bir dili konuşmuş olan daha sonraki Musevi soylarının yüz hat özelliklerine oldukça büyük katkı sağlamıştı. Onların dili oldukça ayırt edilebilen bir biçimde Andon’du.
\vs p080 8:3 İtalya, İsviçre ve güney Avrupa’nın gölleri üzerinde kazıklar veya kütük iskeleler üstüne diktikleri evlerde ikamet eden kabileler; Afrika, Ege ve, daha belirgin bir biçimde, Tuna göçmenlerinin genişleyen uçlarıydı.
\vs p080 8:4 Tuna toplulukları, Andon unsurlarıydılar; bu bireyler, Balkan yarımadası üzerinden Avrupa’ya hali hazırda girmiş olan çiftçi ve sürü sahipleri olup, kademeli bir biçimde Tuna vadisi üzerinden kuzeye doğru hareket etmektelerdi. Onlar, vadilerde yaşamayı tercih eden bir biçimde çömlek yapıp, toprağı işlemişlerdi. Bu kabileler, kültürlerinin merkez ve kaynağından uzaklaştıklarında hızlı bir biçimde gerileme gösterdiler. Şu ana kadar gelmiş geçmiş en iyi çömlekçilik, bu öncül yerleşkelerinin eserleridir.
\vs p080 8:5 Tuna toplulukları, Girit’den gelen din yayıcılarının çalışmalarının bir sonucu olarak anneye ibadet eden bireyler haline gelmişlerdi. Bu kabileler daha sonra, Ön Asya sahilinden teknelerle gelen ve aynı zamanda anne inancına sahip olan Andon denizci toplulukları ile karışmışlardı. Merkezi Avrupa’nın büyük bir kısmı böylelikle erkenden, anne inancını yerine getiren ve ölü yakma dini törenlerini uygulayan geniş ve yassı kafatası yapısına sahip beyaz ırkların bu melez türleri tarafından yerleşik hale geldi; bu âdetin uygulanma sebebi, anneye ibadet etme inancına sahip bireylerin ölülerini taş barakalarda yakma âdetiydi.
\usection{9.\bibnobreakspace Üç Beyaz Irk}
\vs p080 9:1 And göçlerinin sonlanmasına yakın Avrupa’da bulunan ırk karışımları, şu üç beyaz ırkta sınıflanabilir hale geldi:
\vs p080 9:2 1.\bibemph{ Kuzey beyaz ırkları}: İskandinav olarak adlandırdığınız ırk, başlıca olarak And toplulukları ve mavi insandan oluşmuştur; ancak bu unsurlar aynı zamanda, kırmızı ve sarı Sang topluluklarının küçük düzeydeki karışımıyla birlikte And kanının dikkate değer bir kısmını bünyesinde barındırmaktalardı. Kuzey beyaz ırkı böylelikle, en fazla arzu edilen insan ırk kökenlerinin bu dördünü taşımaktaydı. Ancak onların en büyük orandaki kalıtımı mavi insandan gelmekteydi. Öncül İskandinav insanının temel nitelikleri uzun kafatası yapısına sahip, boyu yüksek ve sarışın olmasıydı.
\vs p080 9:3 İstila içerisindeki İskandinav topluluklar tarafından karşılaşılan Avrupa’nın ilkel kültürü, mavi insanla karışan gerilemekteki Tuna topluluklarına aitti. İskandinav\hyp{}Danimarkalı ve Tuna\hyp{}And topluluk kültürleri, bugünün Almanya’sı içinde bulunan iki ırk topluluğunun mevcudiyetinde gözlenebileceği gibi, Ren’de buluşmuş ve birbirine karışmıştır.
\vs p080 9:4 İskandinav toplulukları, Brenner geçidi vasıtasıyla Tuna vadisinin geniş ve yassı kafatasına sahip topluluklar ile büyük bir ticaret ilişkisi kuran bir biçimde Baltık sahilinden kehribar ticaretine devam etmişlerdi. Tuna toplulukları ile bu genişlemiş ilişki, bahse konu kuzey unsurlarının anne inancını benimsemesine yol açmıştı; ve birkaç bin yıl boyunca ölülerin yakılması neredeyse İskandinavya’nın tamamı boyunca herkes tarafından uygulanan bir haldeydi. Bu durum, her ne kadar Avrupa’nın tamamına yayılmış olsa da öncül beyaz ırkların kalıntılarının bulunamamasını açıklamaktadır --- yalnızca onların külleri, içlerinde saklandıkları özel taş ve çömlek kaplarda bulunabilir. Ve yine bu durum, her ne kadar bir önceki Cro\hyp{}Magnon türü tek parça halinde yer üstü ve yer altı mağaralarında tabakalaşmış biçimde oldukça iyi bir şekilde korunmuş olsa da, beyaz insanın öncül kültürüne ait neden çok az sayıda kanıtın mevcut bulunduğunu açıklamaktadır. Bahse konu bu tarihi uygulama; sanki bir gün kuzey Avrupa’da gerileme içindeki Tuna toplulukları ve mavi ırkın ilkel bir kültürünün var olduğunu ve ertesi gün ise kıyaslanamayacak düzeydeki üstün beyaz insanın birden bire ortaya çıkan kültürünün mevcudiyetini göstermektedir.
\vs p080 9:5 2.\bibemph{ Merkezi beyaz ırk}. Bu topluluk her ne kadar beyaz, sarı ve And ırk kökenlerini içinde barındırsa da, başat bir biçimde Andon kalıtımına aitti. Bu topluluklar geniş ve yassı kafatasına sahip, esmer ve tıknaz insanlardı. Onlar, Baltık ve Akdeniz ırkları arasında ortada sıkışıp kalmışlardı; onların bir ucu Asya’da iken diğer bir ucu doğu Fransa’ya hareket etmekteydi.
\vs p080 9:6 Neredeyse yirmi bin yıl boyunca Andon unsurları, And toplulukları tarafından merkezi Asya’nın kuzeyinin giderek daha derinliklerine sürülmüş bir konumdaydılar. M.Ö. 3000’li yıllarda artan kuraklık bu Andon topluluklarını Türkistan’a geri sürüklemekteydi. Güneye doğru gerçekleşen bu Andon hareketi bin yıldan daha fazla bir süre boyunca devam etmiş olup, Hazar Denizi ve Karadeniz etrafında ikiye ayrılıp hem Balkanlar hem de Ukrayna üzerinden Avrupa’ya girmişti. Bu istila, Âdemoğlu soylarının geride kalan topluluklarından da meydana gelmekteydi; ve istila döneminin daha sonraki yarısı boyunca, Seth din adamları soylarının birçoğuna ek olarak İranlı And topluluklarının dikkate değer sayıdaki üyelerini barındırmıştır.
\vs p080 9:7 M.Ö. 2000’li yıllarda Andon topluluklarının batı yönündeki hareketi Avrupa’ya ulaşmıştı. Ve Mezopotamya, Ön Asya ve Tuna havzasının Türkistan tepelerinden gelen barbarlar tarafından istila edilmesi, bu döneme kadar tüm kültürel gerilemeler içinde en ciddisi ve en uzun süreli olanı meydana getirmişti. Bu istilacılar, nitelik olarak bu dönemden beri Alp kökenini sürdüren merkezi Avrupa ırklarının temel niteliğini kesin bir biçimde Andonsal hale getirdi.
\vs p080 9:8 \bibemph{3. Güneyli beyaz ırk}. Bu kumral Akdeniz ırkı, kuzeylilere kıyasla daha az düzeyde bir Andon ırk koluyla birlikte And toplulukları ve mavi ırkın bir karışımından meydana gelmişti. Bu topluluk aynı zamanda, Sahra toplulukları üzerinden ikincil Sang kanının dikkate değer bir düzeyini bünyelerine katmışlardı. Daha sonraki dönemlerde beyaz ırkın güney birimi, doğu Akdeniz’den gelen güçlü And kökenleri ile bütünleşmişti.
\vs p080 9:9 Akdeniz sahil şeridi, buna rağmen, M.Ö. 2500’lü yıllardaki büyük göçebe saldırıları dönemine kadar And toplulukları tarafından nüfuz edilmiş bir konuma gelmemişti. Kara üzerinden gerçekleşen seyahatler ve ticaret faaliyetleri, göçebelerin doğu Akdeniz bölgelerini işgal ettikleri bu çağlar boyunca neredeyse tamamen durmuştu. Kara ulaşımının bu şekilde engellenişi, deniz ulaşımı ve ticaretinde gerçekleşen büyük bir büyümeyi beraberinde getirdi; Akdeniz deniz ticareti, yaklaşık kırk beş bin yıl önce en yüksek hacminde bulunmaktaydı. Ve deniz ulaşımının bu gelişimi, Akdeniz havzasının bütüncül sahil bölgesi boyunca And topluluk soylarının ansızın gerçekleşen bir genişlemesiyle sonuçlandı.
\vs p080 9:10 Bu ırksal karışımlar, tüm zamanların en yüksek düzeyde birbirine karışmış unsuları halindeki güney Avrupa ırkının temelini atmıştı. Ve bu dönemden beri bahse konu ırk, özellikle Arabistan’ın mavi\hyp{}sarı\hyp{}And topluluklarıyla birlikte daha da karışım sürecinden geçmiştir. Bu Akdeniz ırkı gerçekte, çevre topluluklar ile o kadar özgür bir biçimde karışmıştır ki ayrı bir ırk olarak neredeyse ayırt edilebilir olmaktan çıkmıştır; ancak genellikle onun üyeleri kısa boylu, uzun kafatasına sahip ve kumral insanlardır.
\vs p080 9:11 Kuzeyde And toplulukları, savaşlarla ve evliliklerle, mavi insanları ortadan kaldırmışlardı; ancak güneyde mavi insanlar büyük nüfuslar halinde varlıklarını sürdürmeye devam ettiler. Bask ve Berberi toplulukları, bu ırka ait iki kolun kurtuluşunu temsil etmektedir; ancak bu topluluklar bile fazlasıyla Sahra unsurlarıyla karışmışlardı.
\vs p080 9:12 Bu anlatım, M.Ö. 3000’li yıllarda merkezi Avrupa’da ortaya çıkan ırk karışımının bir resmidir. Kısmi Âdemsel kusura rağmen yüksek türler birbirlerine karışmıştı.
\vs p080 9:13 Bu dönemler, devam etmekteki Tunç Çağı’nın Yeni Taş Çağ’ı ile çakıştığı süreçlere rastlamaktaydı. Güney Fransa ve İspanya’da Yeni Taş Çağı güneşe olan ibadet ile nitelendirilir hale gelmişti. Bu dönem, dairesel ve tavanı olmayan tapınak yapılarının bulunduğu süreçtir. Avrupa’lı batı ırkları, tıpkı bir sonraki dönemdeki soylarının Stonehenge’de yaptıkları gibi, güneş için simgeler biçiminde büyük taşları bir araya getirmekten büyük bir haz duyan bir biçimde durmak bilmeyen yapı ustalarıydılar. Güneş ibadetine olan rağbet, bu dönemin güney Avrupa’da tarımın önemli bir süreci olduğuna işaret etmektedir.
\vs p080 9:14 Bu görece yakın zaman içerisinde gerçeklemiş güney ibadet dönemi hurafeleri mevcut an içerisinde bile Breton’un adetlerinde varlığını sürdürmektedir. Her ne kadar bin beş yüz yıldan beri Hıristiyanlaşmış bir konumda olsalar da bu Breton toplulukları, nazarı kovmak için Yeni Taş Çağı’nın büyü etkisine sahip olduğuna inanılan maddeleri kullanmaya devam etmektedirler. Breton unsurları İskandinav kökenli Kuzey Avrupalı topluluklar ile hiçbir zaman karışmamışlardı. Onlar, Akdeniz ırk kökeni ile karışmış bir haldeki batı Avrupa’nın özgün Andon sakinlerinin hayatta kalmış unsurlarıdır.
\vs p080 9:15 Ancak beyaz toplulukları İskandinav, Alp ve Akdeniz sınıflandırmaları altında değerlendirmeye çalışmak yanlış bir inançtır. Orada bütüncül bir biçimde bu türden bir sınıflandırmaya izin veremeyecek kadar çok sayıda karışım meydana gelmişti. Bir zamanlar bu türden topluluklar altında toplanabilecek beyaz ırkın sınırları oldukça belirgin bir sınıflandırılışı mevcuttu; ancak bahse konu dönemden beri geniş çaplı bütünleşmeler ortaya çıkmıştır; ve bu farklılıkları kesin bir biçimde gösterebilmek artık mümkün değildir. M.Ö. 3000’li yıllarda bile tarihi toplumsal bütünlükler, Kuzey Amerika’nın mevcut sakinlerini gibi artık tek bir ırkın türden üyelerinden meydana gelmemekteydi.
\vs p080 9:16 Bu Avrupa kültürü beş bin yıl boyunca büyümeye ve bir ölçüde kendi içinde bütünleşmeye devam etti. Ancak dil engeli, çeşitli Batı milletlerinin bütüncül etkileşimi engelledi. Geçmiş yüzyıldan beri bu kültür, Kuzey Amerika’nın çok uluslu nüfusu içinde birbiriyle bütünleşmek için sahip olduğu en iyi imkânı değerlendirmektedir; ve bu kıtanın geleceği, idare edilen toplumsal kültür seviyesine ek olarak mevcut ve gelecekteki nüfusunun bir parçası olmasına izin verilen ırk etkenlerinin niteliğine göre belirlenecektir.
\vs p080 9:17 [Nebadon’un bir Başmelek unsuru tarafından sunulmuştur.]
