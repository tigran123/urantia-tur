\upaper{123}{İsa’nın Öncül Çocukluğu}
\vs p123 0:1 Beytüllahim’deki konaklamalarının getirdiği belirsizlikler ve endişeler nedeniyle, Meryem; ailenin olağan bir yaşam için yerleşmeye yetkin olduğu yer olan İskenderiye’ye güvenli bir biçimde ulaşana kadar, bebeği sütten kesmemişti. Onlar, kan bağı insanlarıyla beraber yaşamışlardı; ve, Yusuf oldukça yetkin bir biçimde, varışından kısa bir süre sonra işini bulmuş bir olarak, ailesini destekleyebilmekteydi. O birkaç ay boyunca bir marangoz olarak işe alınmıştı, ve daha sonra, bu dönemde yapım aşamasında olan kamu binalarının bir tanesinde işe alınmış işçilerin büyük bir topluluğunun işçi şefliği konumuna yükseltilmişti. Bu yeni deneyim ona, Nasıra’ya geri dönmelerinden sonra bir yapı çalışanı ve ustası haline gelme fikrini verdi.
\vs p123 0:2 İsa’nın yardıma muhtaç bebekliğinin tüm bu öncül yılları boyunca, Meryem; refahını tehlike altına alabilecek ve dünya üzerindeki gelecek görevine herhangi bir biçimde müdahalede bulunabilecek herhangi bir şeyin çocuğunun başına gelmemesi için, uzun ve sürekli ibadetsel gece nöbetinde bulundu; başka hiçbir anne bu zamana kadar çocuğuna daha fazla adanmış olmamıştı. İsa’nın şans eseri bulunduğu evde, yaşına yakın diğer iki çocuk bulunmaktaydı; ve, yakın komşularda, kendilerini makul oyun arkadaşları haline getirecek düzeyde yaşları İsa’nınkine yeteri kadar yakın olan altı diğer çocuk bulunmaktaydı. İlk başta Meryem, İsa’yı yanı başında tutma eğilimi gösterdi. O, İsa’nın diğer çocuklar ile bahçede oynamasına izin verilirse başına bir şey gelebileceğinden korkmuştu; ancak, akraba insanlarının yardımıyla Yusuf Meryem’i, böyle bir gidişatın İsa’yı, kendi yaşındaki çocuklara nasıl uyum göstermesi gerektiğini öğrenmenin yararlı deneyiminden mahrum bırakacağına ikna edebilmişti. Ve, Meryem, nihai olarak, gereksiz kollamanın ve olağandışı korumanın bu türden bir izlencesinin onu, öne çıktığını bilerek utangaç ve bir ölçüde benmerkezci yapma eğilimi gösterebileceğinin farkına vararak, söz verilmiş evladın tıpkı diğer her bir çocuk gibi büyümesine izin vermenin tasarımına rıza gösterdi; ve, her ne kadar bu karara itaat etmiş olsa da, Meryem her zaman, evin yakınlarında veya bahçede küçük çocuklar oynarken gözünün İsa’nın üzerinde oluşunu işi haline getirmişti. Yalnızca şefkatli bir anne, İsa’nın bebekliği ve öncül çocukluğunun bu yılları boyunca oğlunun güvenliği için kalbinde taşıdığı yükü bilebilir.
\vs p123 0:3 İskenderiye’deki iki yıllık konaklamaları boyunca, İsa, iyi sağlığı keyifle deneyimledi ve olağan bir biçimde büyümeye devam etti. Birkaç arkadaştan ve akrabadan başka hiç kimseye, İsa’nın bir “söz verilmiş çocuk” olduğunu söylenmemişti. Yusuf’un akrabalarından bir tanesi bunu, tarihi Akhenaton’un soyları olarak, Memfis’deki birkaç arkadaşa açığa çıkardı; ve, İskenderiye inananlarının küçük bir topluluğu ile birlikte onlar, Nasıra ailesine iyi dileklerde bulunmak ve çocuğa olan saygılarını yerine getirmek için Filistin’e geri dönmelerinden kısa bir süre önce, Yusuf’un yardımda bulunan akrabasının görkemli evinde bir araya geldi. Bu olayda, toplanmış arkadaşlar İsa’ya, Musevi yazıtlarının Yunanca çevirisinin bütüncül bir nüshasını sundular. Ancak, kutsal Musevi yazılarının bu nüshası; hem Yusuf hem de Meryem’in nihai bir biçimde Memfisli ve İskenderiyeli arkadaşlarının Mısır’da kalma davetlerini reddedişlerine kadar, Yusuf’un ellerine verilmemişti. Bu inananlar nihai sonun evladının; İskenderiye’nin bir sakini olarak, Filistinde belirlenmiş herhangi bir bölgedekinden daha büyük bir dünya etkisinde bulunabileceğinde ısrarcı oldular. Onların iknası, Hirodes’in ölüm haberini almalarından sonra ailenin biraz daha Filistin için ayrılıklarını ertelemelerine neden oldu.
\vs p123 0:4 Yusuf ve Meryem nihai olarak; M.Ö. 4.yılda Ağustos ayının sonlarına doğru limana ulaşan bir biçimde, Yafa için arkadaşları Ezraeon’a ait bir tekne ile İskenderiye’den ayrıldılar. Onlar doğrudan bir biçimde; orada mı kalmaları yoksa Nasıra’ya geri dönmeleri mi gerektiği hususunda Eylül ayının tamamını arkadaşlarına ve akrabalarına danışarak geçirdikleri yer olan, Beytüllahim’e gittiler.
\vs p123 0:5 Meryem hiçbir zaman; İsa’nın, Davud’un Şehri olan Beytüllahim’de büyümesi gerektiği düşüncesini terk etmemişti. Yusuf gerçekte, oğullarının İsrail’in bir kralsı koruyucusu haline gelecek oluşuna inanmamaktaydı. Bunun yanı sıra, o, kendisinin Davud’un gerçek bir soyu olmadığını bilmekteydi; Davud’un soyu arasında sayılmasının, akrabalarından birinin Davudi soy kolu içinde evlatlık edilmesinden kaynaklanışı bilmekteydi. Meryem, tabiî ki de, Davud’un Şehri’nin, Davud’un krallığı için yeni adayın yetişebileceği en uygun yer olduğunu düşünmekteydi; ancak, Yusuf şansını, kardeşi Hirodes Archelaus yerine Antipa ile denemeyi tercih etmişti. O, çocuğun Beytüllahim’deki veya Yahudiye’deki herhangi bir şehir içindeki güvenliğine dair büyük korkulara sahipti; ve, Archelaus’un, Celiledeki Antipa’ya nazaran daha muhtemel bir biçimde babası Hirodes’in korkutucu siyasalarını izleyeceği fikrini yürüttü. Ve, tüm bu nedenlerin yanı sıra, Yusuf, çocuğu yetiştirmenin ve eğitmenin daha iyi bir mekânı olarak Celile’den yana olan tercihini açık açık belirtmekteydi; ancak, Meryem’in itirazlarına karşı gelmek üç hafta almıştı.
\vs p123 0:6 Ekim’in ilk günü, Yusuf; Meryem ve tüm diğer arkadaşlarını, Nasıra’ya geri dönmelerinin onların için en iyisi olduğuna ikna etmiş konumdaydı. Bunun uyarınca, M.Ö. 4.yılda Ekim ayının başında, onlar, Lod ve Bet Şean üzerinden, Nasıra için Beytüllahim’den ayrıldılar. Onlar; Meryem ve çocuk yeni alınmış yük hayvanı üzerinde giderken, Yusuf’un ve ona eşlik eden beş erkek akrabasının yürüyerek ilerlediği biçimde, bir Pazar sabahı yola erken vakit yola çıktılar; Yusuf’un akrabaları, Nasıra’ya tek başlarına seyahat etmelerine izin vermeyi reddetmişlerdi. Onlar Celile’ye, Kudüs ve Ürdün vadisi üzerinden gitmekten korkmaktaydılar; ve, batı yolları hiç de, hassas dönemlerin bir çocuğu ile iki yalnız yolcu için hiç de güvenli değildi.
\usection{1.\bibnobreakspace Nasıra’ya Geri Dönüş}
\vs p123 1:1 Yolculuğun dördüncü günü, kafile, istikametine güvenli bir biçimde ulaştı. Onlar, kendilerini görünce gerçekten şaşkınlığa düşmüş olan, Yusuf’un evli kardeşlerinin bir tanesi tarafından üç yıldan fazla bir süredir ikamet edilmekte olan Nasıra evine, haber verilmeden ulaştılar; onlar oldukça sessiz bir biçimde işleri yürütmekteydeler ki, ne Yusuf’un ailesi ne de Meryem’inkiler, çiftin İskenderiye’den ayrılmış olduklarını bile bilmemekteydiler. Bir sonraki gün Yusuf’un ailesi, ailesini taşıdı; ve, Meryem, İsa’nın doğumundan beri ilk kez, kendine ait evde yaşamı keyifle deneyimlemek için kendi küçük ailesiyle yerleşti. Bir haftadan daha az bir süre içinde, Yusuf, bir marangoz olarak iş bulmuş olup, onlar en yüksek derecede mutlulardı.
\vs p123 1:2 İsa, Nasıra’ya geri döndüklerinde yaklaşık olarak üç yıl iki aylıktı. O; tüm bu yolculuklara çok iyi dayanmış, ve mükemmel sağlığa sahip olmuş, ve bütünüyle, etrafında koşacağı ve keyifle yaşayacağı kendisine ait alanlara sahip olmanın çocuksu neşesiyle ve heyecanıyla doluydu. Ancak, o fazlasıyla, İskenderiyeli oyun arkadaşlarıyla olan birlikteliğini özlemişti.
\vs p123 1:3 Nasıra’ya olan yolculuklarında, Yusuf Meryem’i; Celile arkadaşları ve akrabaları arasında İsa’nın söz verilmiş bir evlat olduğu sözünü yaymanın bilgece olmayacağına ikna etmişti. Onlar, herhangi bir kişiye bu hususlarda herhangi bir şeyden bahsetmekten tamamiyle kaçınmaya karar vermişlerdi. Ve, onların her ikisi de, bu sözü tutmada oldukça sadıklardı.
\vs p123 1:4 İsa’nın dördüncü yaşının tamamı, olağan fiziksel gelişim ve olağandışı ussal etkinliğin bir süreciydi. Bu arada, o, Yakup isminde yaklaşık olarak kendi yaşında bulunan bir komşu çocuğuna oldukça yakın bir bağlılık kurmuştu. İsa ve Yakup, oyunlarından her zaman mutlulardı; ve, onlar, büyük arkadaşlar ve sadık dostlar haline gelen bir biçimde büyüdüler.
\vs p123 1:5 Bu Nasıra ailesinin yaşamındaki bir sonraki önemli olay, M.Ö. 3.yılda Nisan’ın ikisinin erken sabah saatlerinde James olarak ikinci çocuğun doğumuydu. İsa, bir erkek bebek kardeşe sahip olma düşüncesinden çok büyük heyecan duymaktaydı; ve, o, sadece bebeğin öncül olarak yaptıkları şeyleri gözlemlemek için saatlerce öyle durup izlemekteydi.
\vs p123 1:6 Yusuf, köy pınarına yakın ve kervan bekleme bağlama yerinin yakınında küçük bir atölyeyi aynı yılın yaz ortasında inşa etmişti. Bundan sonra o, her gün çok az marangoz işinde bulundu. O; atölyede boyunduruk ve saban yapmak ve diğer ahşap işlerinde bulunmak için kaldığında çalışmaya gönderdiği, kardeşlerinin ikisinden ve başka birkaç zanaatkârdan oluşan yardımcılara sahipti. O aynı zamanda, deri üzerinde ve ip ve branda bezi ile bazı işlerde bulunmuştu. Ve, İsa, büyürken, okulda olmadığında; zamanını eşit bir biçimde ev işlerinde annesine yardım etmeyle ve atölyede babasını gözlemlemeyle geçirirken, bu arada da, dünyanın dört bir köşesinden gelen kervancı başlarının ve yolcularının konuşmalarına ve dedikodularına kulak kabartmaktaydı.
\vs p123 1:7 Bu yılın Temmuz ayında, İsa’nın dördüncü yaşına girmesinden bir ay önce, kervan yolcularıyla olan etkileşimden zararlı bağırsak rahatsızlığının bir salgını tüm Nasıra’ya yayılmıştı. Meryem; bu salgın hastalığa İsa’nın maruz kalabilme tehlikesinden o kadar endişelenmiş hale gelmişti ki, her iki çocuğunu da sımsıkı giydirip, Sarid yakınındaki Megiddo yolu üzerinde Nasıra’nın birkaç mil güneyinde kalan bir biçimde, abisinin şehir dışındaki evine kaçmıştı. Onlar Nasıra’ya iki aydan daha fazla bir süre boyunca geri dönmedi; İsa, bundan fazlasıyla keyif duydu, bir çiftlikte bu onun ilk deneyimiydi.
\usection{2.\bibnobreakspace Beşinci Yıl (M.Ö. 2)}
\vs p123 2:1 Nasıra’ya geri dönmelerinden bir yıldan biraz daha fazla bir süre içinde, erkek çocuk İsa, ilk kişisel ve samimi nitelikteki ahlaki kararının yaşına gelmişti; ve, oraya, Maçiventa Melçizedeği’ne daha öncesinden hizmet etmiş, böylece fani bedeninin suretinde yaşayan bir fani\hyp{}ötesi varlığın vücutlaşımı ile birliktelik içinde faaliyet gösterme deneyimini elde etmiş, Cennet Yaratıcısı’nın kutsal bir hediyesi olarak bir Düşünce Düzenleyicisi kendisiyle kalmak için gelmişti. Bu olay, M.Ö. 2.yılda Şubat ayının 11’nde gerçekleşmişti. İsa; akıllarında ikamet etmek, ve, bu akılların nihai ruhanileşimi ve onların evrimleşen ölümsüz ruhlarının ebedi kurtuluşu için görevde bulunmak amacıyla, benzer bir biçimde bu Düşünce Düzenleyicileri’ni, bugünden önce ve bu günden sonra almış milyonlarca diğer çocuktan daha fazla bir biçimde bu kutsal Görüntüleyici’nin farkında değildi.
\vs p123 2:2 Şubat ayında bu gün, Mikâil’in çocuksu bahşedilişinin doğruluğundan sorumlu olan Kâinat Yöneticileri’nin doğrudan ve kişisel yüksek denetimi sonlandırılmıştı. Bu andan itibaren, bahşedilişin insan gerçekleşimi boyunca İsa’yı kollama görevi; zaman zaman, gezegensel üstlerinin yönergesi uyarınca bir takım sınırları belirlenmiş sorumluluklarını yerine getirmek amacıyla görevlendirilmiş yarı\hyp{}ölümlü yaratılmışların hizmeti ile desteklenen bir biçimde, bu ikamet eden Düzenleyici’nin muhafazasına ve ilgili yüksek meleksel koruyuculara bırakılma nihai sonuna sahipti.
\vs p123 2:3 İsa, bu yılın Ağustos ayında beş yaşındaydı; ve, bizler bu nedenle bunu, onun beşinci (takvim) yılı olarak adlandıracağız. M.Ö. 2.yıl olarak bu yılda, beşinci doğum gününün yıldönümüne bir aydan biraz daha kısa bir süre önce, İsa; Temmuz’un 11’inin gecesi doğmuş olan kız kardeşi Miryam’ın gelişiyle oldukça mutlu olmuştu. Ertesi günün akşamı boyunca İsa babasıyla; yaşayan şeylere ait çeşitli topluluklarının dünyaya nasıl ayrı bireyler olarak doğduklarına dair uzun bir konuşmada bulundu. İsa’nın öncül eğitiminin en değerli kısmı, derin ve arayış halindeki sorularına karşılık ebeveynlerinden elde edilmişti. Yusuf hiçbir zaman; zahmete katlanıp, oğlunun sayısız sorusuna cevap vererek vaktini harcama görevini eksiksiz yerine getirmede başarısız olmamıştı. Beş yaşındayken on yaşına kadar, İsa, devamlı bir soru işaretiydi. Her ne kadar Yusuf ve Meryem her zaman İsa’nın sorularına cevap veremeseler de, onlar hiçbir zaman, İsa’nın sorgularını bütünüyle tartışmada ve meraklı aklının getirdiği sorunun tatminkâr bir çözümüne ulaşmadaki onun çabalarında kendisine başka her biçimde yardımcı olmada başarısız olmadılar.
\vs p123 2:4 Nasıra’ya geri dönmelerinden beri, kendilerininki meşgul bir ev yaşamı olmuştu; ve, Yusuf olağanın dışında bir biçimde, yeni atölyesini inşa etmekle ve işini tekrar yoluna koymakla meşgul haldeydi. Tamamiyle o kadar yoğundu ki, James için bir beşik yapacak vakit bulamamıştı; ancak bu durum Miryam’ın gelişinden uzunca bir süre önce düzeltildi, ve Miryam, ailesi ona hayran hayran bakarken huzurla barınacağı oldukça rahat bir bebek yatağına sahip olmuştu. Ve, çocuk İsa tam da en içinden, tüm bu doğal ve olağan ev deneyimlerine girmişti. O küçük erkek kardeşinden ve bebek kız kardeşinden fazlasıyla keyif duymuş, her ikisinin de bakımında Meryem’e büyük yardımı dokunmuştu.
\vs p123 2:5 Bu dönemlerin Musevi\hyp{}olmayan dünyasında; bir çocuğa, Celile’nin Musevi evlerinden daha iyi nitelikli ussal, ahlaki ve dini hazırlanmayı verebilecek çok az ev bulunmaktaydı. Bu Museviler, çocuklarını yetiştirmek ve eğitmek için düzenli hale getirilmiş bir eğitime sahipti. Onlar, bir çocuğun yaşamını yedi aşamaya ayırmıştı:
\vs p123 2:6 1.\bibnobreakspace Birinci günden sekizinci güne kadar olan süre biçiminde, yeni\hyp{}doğan çocuk.
\vs p123 2:7 2.\bibnobreakspace Emzirilen çocuk.
\vs p123 2:8 3.\bibnobreakspace Anne sütünden kesilen çocuk.
\vs p123 2:9 4.\bibnobreakspace Beşinci yılın sonuna kadar süren bir biçimde, anneye bağımlılığın süreci.
\vs p123 2:10 5.\bibnobreakspace Çocuğun bağımsızlığının başlangıcı, ve erkek çocuklar ile, babanın eğitim sorumluluğunu üstlenişi.
\vs p123 2:11 6.\bibnobreakspace Ergenlik sürecindeki erkek ve kız çocuklar.
\vs p123 2:12 7.\bibnobreakspace Genç erkek ve kız çocuklar.
\vs p123 2:13 Beşinci doğum gününe kadar annenin bir çocuğun hazırlanışı için sorumluluk üstlenişi, ve bunun sonrasında, çocuk eğer erkek ise gencin eğitimi için artık babanın sorumlu tutuluşu Celile Musevileri’nin âdetiydi. Bu yıl, bu nedenle, İsa; bir Celile Musevi çocuğu sürecinin beşinci aşamasına girmişti, ve bu böylece M.S. 2.yılda Ağustos’un 21’nde gerçekleşmişti. Meryem resmi bir biçimde İsa’yı, ilave eğitimi için Yusuf’a yönlendirmişti.
\vs p123 2:14 Her ne kadar Yusuf bu aşamada artık İsa’nın ussal ve dini eğitimi için doğrudan sorumluluğu üstlenmekte olsa da, annesi hala İsa’nın ev eğitimi ile doğrudan ilgilenmekteydi. Meryem İsa’ya, ev yerleşke arazisini tamamiyle çevreleyen bahçe duvarları etrafında büyüyen sarmaşıkları ve çiçekleri tanımayı ve onlara bakmayı öğretmişti. Meryem aynı zamanda evin çatısına (yaz yatak odasına); üzerinde, İsa’nın haritalara çalıştığı ve zaman içinde Arami, Yunan ve daha sonrasında İbrani olmak üzere üç dilde de yetkin bir biçimde okumayı, yazmayı ve konuşmayı öğrendiği küçük kum kutuları koymuştu.
\vs p123 2:15 İsa; fiziksel olarak neredeyse kusursuz bir çocuk halinde gelişmiş olup, zihinsel ve hissel olarak olağan ilerleyişinde bulunmaya devam etmekteydi. O; bu, beşinci (takvim) yılının son kısmında, ilk küçük çaplı hastalığı olarak, hafif bir sindirim sorunu yaşadı.
\vs p123 2:16 Her ne kadar Yusuf ve Meryem sıklıkla en büyük çocuklarının geleceği hakkında konuşmakta olsalar da, eğer siz orada bulunmuş olsaydınız yalnızca, olağan, sağlıklı, rahat ancak bu zaman ve mekânın aşırı derece sorgulayıcı olan bir çocuğunu gözlemlemiş olurdunuz.
\usection{3.\bibnobreakspace Altıncı Yılın Yaşanılmışlıkları (M.Ö. 1.yıl)}
\vs p123 3:1 Annesinin yardımı vasıtasıyla, İsa hali hazırda; Arami dilinin Celil aksanı üzerinde üstün hale gelmiş konumdaydı; ve, bu aşamada babası ona Yunanca öğretmeye başladı. Meryem çok az derecede Yunanca konuşabilmekteydi; ancak, Yusuf, hem Aramice ve hem de Yunanca’yı akıcı bir biçimde konuşabilmekteydi. Yunan dilini öğrenmek için ana çalışma kitabı, Mısır’dan ayrılırlarken kendilerine sunulmuş olan --- Mezmurlar’ı da içine alan bir biçimde, kanun ve peygamberlerin bütüncül bir metin hali olarak --- İbrani yazıtlarının nüshasıydı. Tüm Nasıra içinde yalnızca, Yunan dilinde Yazıtlar’ın tamamlanmış iki nüshası bulunmaktaydı; ve, marangozun ailesi tarafından onlardan birine sahip olmak, Yusuf’un evini fazlasıyla aranılan bir ev haline getirmiş olup, İsa’yı, o büyürken, dürüst öğrencilerin ve içten doğruluk arayıcılarının neredeyse sonu gelmez bir akışıyla tanışmasını sağlamıştır. Bu yıl sonlanmadan önce, İsa; altıncı doğum gününde kendisine, bu kutsal kitabın kendisine İskenderiye arkadaşları ve akrabaları tarafından sunulmuş olduğu söylenmiş bir biçimde, bu paha biçilemez el yazmasının koruyuculuğunu üstlenmişti. Ve, oldukça kısa bir zaman zarfı içinde İsa, onu hali hazırda okuyabilmekteydi.
\vs p123 3:2 İsa’nın genç yaşamındaki ilk büyük şaşkınlık, altı yaşını dolduruşundan biraz önce ortaya çıkmıştı. Bu gence daha öncesinde, babasının --- en azından annesi ve babasının beraberce --- her şeyi bildiği görünmüştü. Bu nedenle; bu sorgulayıcı çocuğun babasına, hemen öncesinde gerçekleşmiş hafif bir depremin nedenini sorduğu zaman, Yusuf’un “Oğlum, bunun cevabını gerçekten bilmiyorum” dediğinde yaşadığı şaşkınlığı bir hayal edin. Böylelikle, İsa’nın; süreç içinde dünyasal ebeveynlerinin her şeyin bilgeliğine sahip ve her şeyi bilen konumda bulunmadıklarını öğrendiği, gerçekleri görüşün bu uzun ve şaşkına çeviren süreci başlamıştı.
\vs p123 3:3 Yusuf ilk önce; İsa’ya, depreme Tanrı’nın neden olduğunu söylemeyi düşündü; ancak, bir anlık üzerinde düşünce kendisini, böyle bir cevabın derhal ilave ve daha utandırıcı sorguları tetikleyeceği konusunda onu uyardı. Erken yaşında bile, İsa’nın fiziksel veya toplumsal olgular ile ilgili sorularını, ya Tanrı yâda şeytanın neden olduğu gibi üzerinde düşünmeden cevaplamak çok zordu. Musevi topluluğunun bu zamanlarda hükmünü sürdüren inanışıyla uyumlu bir biçimde, İsa uzunca bir süredir, zihinsel ve ruhsal olguların olası açıklaması olarak iyi ruhaniyetlerin ve kötü ruhaniyetlerin inanış savını kabul etmeye gönüllüydü; ancak, o çok önceden, bu türden görülmemiş etkilerin doğal dünyanın fiziksel gelişimlerinden sorumlu oluşundan kuşku duyar hale gelmişti.
\vs p123 3:4 İsa altıncı yaşına girmeden, M.S. 1.yılda yazın başında, Zekeriya ve Elizabet ve onların oğlu Yahya, Nasıra ailesini ziyaret etmeye geldi. İsa ve Yahya; bu, hatırlayabildikleri ilk ziyaretleri boyunca neşeli bir vakit geçirdiler. Her ne kadar ziyaretçiler yalnızca bir kaç gün kalabilseler de, ebeveynler, evlatlarına dair gelecek tasarımlarını da içine alan bir biçimde birçok şey hakkında konuştular. Her ne kadar onlar böyle konuşmalar içinde olsalar da, gençler evin tavanı üzerinde kum içindeki taşlarla oynayıp, birçok açıdan gerçek erkek çocuklar gibi zevk alarak bunu gerçekleştirdiler.
\vs p123 3:5 Kudüs’ün yakınından gelmekte olan Yahya ile tanışarak, İsa; İsrail’in tarihine dair olağandışı bir ilgi sergilemiş olup, Şabat adetlerine, sinagog vaazlarına ve anma törenlerinin tekrarlanan ziyafetlerinin anlamına dair büyük çaplı bir sorguda bulunmuştu. Onun babası, bu dönemlerin anlamına ona açıklamıştı. Bu, ilk gece bir mum yakarak ve tekrar eden her gece bir tanesini daha ekleyerek devam eden bir şekilde, sekiz gün süren, kış ortası gerçekleşmekte olan ışık kutlamalarıydı; bu, Yehuda Makabi tarafından Musasal hizmetlerin eski haline getirilişi sonrasında mabedin adanışının anma törenidir. Bunun sonrasında Purim’in erken bahar kutlaması, Ester ziyafeti ve İsrail’in kendisinden olan kurtuluşu gelmektedir. Daha sonra, ebeveynlerin her ne zaman müsait olduklarında Kudüs’de kutlamalarını gerçekleştirirken, evde geride kalan çocukların bu bütün hafta boyunca maya ile yapılmış hiçbir ekmeğin yenmemesi gerektiğini hatırladıkları, önemli Hamursuz izledi. Sonrasında, hasat toplanışı olan ilk\hyp{}meyvelerin ziyafeti geldi; ve, en sonunda, en önemlisi olarak, günahlardan arınmanın günü olarak, yeni yılın ziyafeti geldi. Her ne kadar bu kutlamaların ve adetlerin bazıları İsa’nın genç aklının anlaması için zor olmuş olsa da, o; onlar hakkında ciddi bir biçimde düşünmüş olup, bunun sonrasında, dışarıda yapraklardan çadırlarda konakladıkları ve kendilerini eğlence ve keyfe verdikleri zaman olarak bütün Musevi topluluğunun yıllık tatil dönemi halindeki, mişkanların ziyafetinin neşesine tamamiyle katılmıştı.
\vs p123 3:6 Bu yıl boyunca, Yusuf ve Meryem, duaları ile ilgili İsa ile sorun yaşamışlardı. İsa, tıpkı dünyasal babası olan Yusuf ile konuşur gibi, cennetsel Yaratıcısı ile konuşmada ısrar etmişti. İlahiyat ile olan daha ciddi ve derin saygı duyan iletişim biçimlerinden bu ayrılık, özellikle annesi olmak üzere, ebeveynlerini biraz şaşkına çevirmişti; İsa dualarını, “cennetteki Yaratıcısı ile küçük bir konuşmada” ısrar ettikten sonra, öğretildiği gibi yerine getirirdi.
\vs p123 3:7 Bu yılın Haziran ayında, Yusuf; Nasıra’daki atölyesini kardeşlerine devredip, bir inşaatçı olarak işine resmi bir biçimde başlamıştı. Bu yıl sona ermeden, ailenin geliri üç katından fazlasına yükselmişti. İsa’nın ölümüne kadar Nasıra ailesi bir daha hiçbir zaman, fakirliğin etkisini hissetmemişti. Aile gittikçe büyüdü, ve onlar, ilave eğitim ve seyahate daha fazla para harcadılar; ancak, Yusuf’un artan geliri her zaman, büyüyen giderlerle aynı hızda ilerledi.
\vs p123 3:8 Sonraki birkaç yıl içinde Yusuf; Cen, (Celile’nin şehri olan) Beytüllahim, Mecdel, Nain, Seforis, Kapernahum ve Endor’da dikkate değer ölçüde çalışmış olup, buna ek olarak, Nasıra’nın içinde ve yakınında birçok inşaat içinde bulundu. James, genç çocukların ev ödevleri ve bakımında annesine yardım edecek kadar büyüdüğünde, İsa babası ile birlikte evden ayrılarak bu çevre kasabaları ve köylerine sık ziyaretlerde bulundu. İsa keskin bir gözlemci olup, evden ayrılarak gerçekleştirdiği bu ziyaretlerden birçok yararlı bilgi edinmişti; o, insan ve onun dünya üzerinde yaşama biçimi hakkında kararlı ve devamlı bir biçimde bilgi biriktirmekteydi.
\vs p123 3:9 Bu yıl, İsa, aile işbirliğinin ve ev disiplinin talepleri karşısında güçlü duygularını ve istekli dürtülerini uyumlaştırmada büyük ilerleme kaydetmişti. Meryem sevgi dolu bir anneydi, ancak o oldukça katı bir disiplin gözeticiydi. Birçok şekilde, buna rağmen, Yusuf; küçük çocuk ile oturup, ailenin bütününün refahı ve huzurunu göz önünde bulundurarak kişisel arzuların disiplinsel azaltılışının gerekliliğine dair gerçek ve altında yatan nedenleri bütünüyle açıklamak onun âdeti olduğu için, İsa üzerinde daha büyük bir etkide bulunmuştu. Durum İsa’ya açıklandığında, o her zaman, ussal ve gönüllü bir biçimde ebeveynsel arzular ve aile yönergeleri ile işbirliğinde bulunmaktaydı.
\vs p123 3:10 Annesinin ev ile ilgi yardımına ihtiyaç duymadığı zamanlar olarak --- boş zamanının büyük bir bölümü, gündüz çiçekler ve bitkiler üzerinde ve akşam ise yıldızlar üzerinde çalışarak geçmişti. O; bu çok düzenli Nasıra hanesinde olağan yatak vaktinden çok sonra, sırtının üzerine uzanıp, yıldızlar ile kaplı göklere doğru bakmanın kendisine sorun çıkaran ama hoşlandığı bir eğilimini sergilemişti.
\usection{4.\bibnobreakspace Yedinci Yaş (M.S. 1.yıl)}
\vs p123 4:1 Bu, gerçekten de, İsa’nın yaşamında önemli olaylara sahne olmuş bir yıldı. Ocağın başında, büyük bir kar fırtınası Celile’de ortaya çıkmıştı. Kar; yaşamı boyunca İsa’nın gördüğü en yoğun kar yağışı ve yüzlerce yıllık bir süre içinde Nasıra’da en derinlerinden bir tanesi olarak, yaklaşık altmış metreyi geçmişti.
\vs p123 4:2 Musevi çocuklarının oyun yaşamı İsa’nın zamanında daha kısıtlıydı; haddinden fazla bir biçimde çocuklar, büyüklerini yaparken gözlemledikleri daha ciddi şeyleri oynamaktaydılar. Onlar fazlasıyla; oldukça sık bir biçimde gördükleri ve oldukça dikkate çekici olan biçimindeki törenler olarak, düğünler ve cenazelerde oynadılar. Onlar dans edip şarkı söylediler, ancak, daha sonraki çocukların fazlasıyla keyif alarak gerçekleştirdikleri düzenli oyunların çok azına sahiptiler.
\vs p123 4:3 İsa, bir komşu çocuğunun ve daha sonra ise kardeşi James’in eşliğinde; talaşlar ve ahşap parçaları ile büyük zevk aldıkları yer olan, ailenin marangoz atölyesinin uzak köşesinde oynamaktan çok mutlu olmaktaydı. İsa için her zaman, Şabat günü yasaklanmış olan belirli oyun türlerinin yaratabileceği zararı kavramak zordu; ancak o hiçbir şekilde, ebeveynlerinin isteklerine uymada başarısız olmadı. O, dönemi ve neslinin sahip olduğu çevre içinde dışa vurulması için çok az olanak sağlayan mizahın ve oyunun bir yetkinliğine sahipti; ancak, on dört yaşına kadar o çoğu zaman, güler yüzlü ve tasadan uzaktı.
\vs p123 4:4 Meryem, evin bitişindeki hayvan barınağının çatısında bir güvercinlik bakmaktaydı; ve, onlar, İsa’nın yüzde onunu kesip sinagogun görevli kişisine verdiği, güvercinlerin satışından gelen karı özel bir yardım hesabı olarak kullandılar.
\vs p123 4:5 Bu zamana kadar İsa’nın yaşadığı tek gerçek kaza, branda beziyle kaplı yatak odasına çıkan arka bahçe taş merdivenlerinden bir düşüşüydü. Bu, doğudan gelen beklenmemiş bir Temmuz kum fırtınası süresince gerçekleşmişti. İnce kum tanelerinin şiddetli fırtınasını taşıyan sıcak rüzgârlar, özellikle Mart ve Nisan ayında gerçekleşen bir biçimde, genellikle yağmurlu dönem boyunca esmekteydi. Bu türden bir fırtınayı Temmuz’da yaşamak olağandışı bir durumdu. Fırtına geldiğinde, İsa alışkanlığı olarak çatı üzerinde oyun oynamaktaydı; zira burası onun, kuru mevsimin büyük bir kısmı boyunca onun alışılageldik oyun odasıydı. Merdivenlerden aşağıya inerken kum gözlerini tamamen kapamıştı, ve o düştü. Bu kazadan sonra, Yusuf, merdivenlerin her iki tarafına da merdiven korkulukları inşa etti.
\vs p123 4:6 Bu kaza hiçbir biçimde engellenemezdi. Bu, gencin gözetimine verilmiş olan bir birincil ve bir ikincil yarı\hyp{}ölümlü olarak, yarı\hyp{}ölümlü geçici koruyucuların ilgisizliğine yüklenebilecek bir durum söz konusu değildi; bu ne de, koruyucu yüksek meleğe yüklenebilecek bir durumdu. Bu olay tamamiyle kaçınılamaz bir yaşanmışlıktı. Ancak, Yusuf’un Endor’da olduğu bir vakit yokluğunda gerçekleşen bir biçimde, bu küçük kaza; Meryem’in aklında, birkaç ay boyunca İsa’yı oldukça yakın gözetim altında yanı başında bilgece olmayan bir biçimde tutmaya çalışır ölçüde büyük bir endişenin büyümesine neden oldu.
\vs p123 4:7 Fiziksel doğanın yaygın gelişmeleri olarak maddi kazalara keyfi bir biçimde, göksel kişilikler tarafından müdahalede bulunmamaktadır. Olağan koşullar altında yalnızca yarı\hyp{}ölümlü yaratılmışlar, nihai sona ait erkek ve kadın bireyleri kollamak için maddi durumlara müdahalede bulunabilir; ve, ruhsal durumlarda bile bu varlıklar bu şekilde, üstlerinin belirli emirlerine uyarak hareket edebilirler.
\vs p123 4:8 Ve, bu sorgulayıcı ve maceraperest gencin daha sonra başına gelen bu türden küçük çaplı geniş sayıdaki kazaların yalnızca bir tanesiydi. Eğer fazlasıyla hareketli bir çocuğun ortalama çocukluğunu ve gençliğini gözünüzün önünde canlandıracak olursanız, İsa’nın gençlik sürecine dair oldukça iyi bir fikre sahip olursunuz; ve, sizler, özellikle annesi olmak üzere, ebeveynlerine ne kadar büyük bir endişe kaynağı olabileceğini hayal edebilirsiniz.
\vs p123 4:9 Nasıra ailesinin dördüncü üyesi olarak, Yusuf, M.S. 1.yılda Mart’ın 16’sında Çarşamba sabahı doğmuştu.
\usection{5.\bibnobreakspace Nasıra’daki Okul Günleri}
\vs p123 5:1 İsa, bu aşamada; Musevi çocuklarının sinagog okullarında resmi eğitimlerine başlamalarının beklendiği yaş olarak, yedinci yaşındaydı. Bunun uyarınca, bu yılın Ağustos ayında o, Nasıra’daki birçok önemli yaşanmışlıklara sahne olmuş okul yaşamına adım atmıştı. Hâlihazırda bu genç; Arami ve Yunan dilleri olarak, iki dili yetkin bir biçimde okumakta, yazmakta ve konuşmaktaydı. O bu aşamada, İbrani dilinde okumayı, yazmayı ve konuşmayı öğrenme görevi ile kendisini aşina edecekti. Ve, o gerçekten de, önünde uzanan bu yeni okul yaşamını derinden arzulamaktaydı.
\vs p123 5:2 Üç yıl boyunca --- onuncu yaşına kadar --- Nasıra sinagogunun ilkokuluna gitmişti. Bu üç yıl boyunca, İbrani dilinde kayıt altına alınmış haliyle Kanun Kitabı’nın o döneme kadar ulaşabilmiş kalıntılarını çalışmıştı. Takip eden üç yıl boyunca, o; yüksek düzey okulda çalışmış olup, kutsal kanunun daha derin olan öğretilerini sesli bir biçimde tekrar etme yönetimiyle hafızasına almıştı. O, bu sinagog okulundan on üçüncü yaşı içinde mezun olmuş olup, taşıyan herkesi Kudüs’de Hamursuz’a katılmaya hak kazandıran İsrail ulusunun artık sorumlu bir vatandaşı niteliğinde --- eğitilmiş bir “emrin evladı” olarak, sinagog yöneticileri tarafından ebeveynlerine teslim edilmişti; bunun uyarınca o ilk Hamursuzu’na, babası ve annesinin eşliğinde o yıl katılmıştı.
\vs p123 5:3 Nasıra’da öğrenciler yarı daire halinde yerde otururlarken, bir sinagog görevlisi halindeki hazzan olarak, onların öğretmeni bu öğrencilere bakan konumda oturmaktaydı. Levililer kitabından başlayarak onlar, kanunun diğer kitaplarının, ve onun ardından da, Peygamberlerin ve Mezmurlar’ın irdelenişine geçmekteydiler. Nasıra sinagogu, İbrani dilinde Yazıtlar’ın bütüncül bir nüshasına sahipti. On ikinci yıla kadar Yazıtlar’dan başka hiçbir şey çalışılmamaktaydı. Yaz aylarında, okul saatleri fazlasıyla kısaltılmaktaydı.
\vs p123 5:4 İsa öncül bir biçimde, İbrani dilini üstünlükle kullanan biri haline geldi; ve, Nasıra’da yüksek derecede bir ziyaretçinin bulunmadığı zamanlarda, bir genç birey olarak ondan sıklıkla, olağan Şabat hizmetlerinde sinagogda toplanmış olan inanç sahiplerine İbrani yazıtlarını okuması istenirdi.
\vs p123 5:5 Bu sinagog okulları, tabiî ki de, herhangi bir ders kitabına sahip değildi. Eğitimde, hazzan bir ifade de bulunurken, öğrenciler hep birlikte onun arkasından tekrar ederlerdi. Kanunun yazılı kitaplarına başvurmak zorunda kaldıkları zaman, öğrenci dersini, sesli bir biçimde okuyarak ve sürekli olarak tekrar ederek öğrenirdi.
\vs p123 5:6 Daha sonra, daha fazla resmi eğitimine ek olarak, İsa; babasının tamir atölyesine giren çıkan, birçok yerden gelen insanlar olarak dünyanın dört bir yanından insan doğasıyla iletişimde bulunmaya başladı. O daha da büyüdüğünde, istirahat ve beslenmek için pınar yakınında dururlarken kervan yolcuları ile özgür bir biçimde kaynaştı. Yunan dilini akıcı bir biçimde konuşur halde o, kervan yolcuları ve kervanbaşlarının büyük bir çoğunluğu ile sohbet ederken çok az sorun yaşadı.
\vs p123 5:7 Nasıra bir kervan yolu durağı ve seyahatin kesişim noktası olup, nüfus bakımından fazlasıyla Musevi\hyp{}olmayanlardan oluşmaktaydı; aynı zamanda o yaygın bir biçimde, geleneksel Musevi kanununun özgürlükçü yorumunun bir merkezi olarak bilinmekteydi. Celile’de Museviler, Yahudiye’deki topluluklarının gerçekleştirdiklerine kıyasla daha özgür bir biçimde Musevi\hyp{}olmayanlar ile içli dışlı olmaktaydı. Ve, Celile’nin tüm şehirleri içinde, Nasıra’nın Musevileri; Musevi\hyp{}olmayanlar ile iletişimlerinin bir sonucu olarak, kirlenme korkularına dayanan toplumsal kısıtlılıklara dair yorumlarında en özgürlükçü olanlarıydı. Ve, bu koşullar, Kudüs’de “Nasıra’dan hiç iyi bir şey çıkar mı?” gibi ortak söyleme sebebiyet vermişti.
\vs p123 5:8 İsa, ahlaki eğitimini ve ruhsal kültürünü başlıca bir biçimde evinde almıştı. O, ussal ve din\hyp{}kuramsal eğitiminin çoğunu hazzandan elde etmişti. Ancak, yaşamın zor sorunları ile mücadele vermenin mevcut sınavı için aklının ve kalbinin başlıca aracı olarak --- gerçek eğitimini, akran insanları ile işli dışlı olmaktan elde etmişti. İnsan ırkını tanımanın olanağını kendisine sağlayan şey, Musevi ve Musevi\hyp{}olmayan, genç ve yaşlı fark etmeksizin akran insanları ile olan bu yakın iletişim olmuştu. İsa, bütünüyle insanları anlar ve adanmış bir biçimde onlara derin sevgi besler derecede, oldukça eğitimliydi.
\vs p123 5:9 Sinagogdaki yılları boyunca o; üç dil bildiği için büyük bir üstünlüğü elinde bulunduran bir biçimde, muhteşem bir öğrenciydi. İsa’nın okulunda dersini tamamladığı bir seferinde Nasıra hazzanı, Yusuf’a; “gence öğretmeye yetkin olduğundan” daha fazla şeyi, tam tersine, “İsa’nın arayış içindeki sorularından öğrenmiş olduğuna” dair endişesini özellikle belirtmişti.
\vs p123 5:10 Her ne kadar eğitim süreci boyunca İsa, sinagogdaki düzenli Şabat vaazlarından çok şey öğrenmiş ve onlardan büyük bir ilham elde etmişti. Nasıra’da Şabat zamanı duran seçkin ziyaretçilerden sinagoga seslenmesini istemek adettendi. İsa büyürken; görüşlerini sunmuş olan Musevi dünyasının tamamına ait büyük düşünürlerin büyük bir kısmını, ve aynı zamanda, Nasıra sinagogu İbrani düşünce ve kültürünün gelişmiş ve özgürlükçü bir merkezi olduğu için neredeyse hiçbir şekilde köktenci Musevi sayılamayacak bireylerin çoğunu dinlemişti.
\vs p123 5:11 Yedi yaşında okula başlayarak (bu dönemde Museviler daha yeni bir zorunlu eğitim yasasını yürürlülüğe koymuştu) öğrencilerin; bir dönemler uygulandığı gibi on üç yaşına geldiklerindeki mezuniyetlerinde sıklıkla okumakta oldukları metin olarak, eğitimleri boyunca onları yönlendirecek altın kuralın bir türü niteliğindeki “doğum günü metnini” seçmeleri adettendi. İsa’nın metni İşaya Peygamber’dendi: “Koruyucu Tanrı’nın ruhaniyeti benim üzerimdedir, zira Koruyucu beni kutsadı; o beni, ezilenlere iyi haberleri getirmek, kalbi kırılmışın gönlünü almak, esirlere özgürlüğü duyurmak ve ruhsal mahkûmları serbest bırakmak için gönderdi.
\vs p123 5:12 Nasıra, İbrani milletinin yirmi\hyp{}dört din\hyp{}adamı merkezinden bir tanesiydi. Ancak, Celile din\hyp{}adamlığı, Yahudiye yazıcıları ve hahamlarına kıyasla geleneksel kanunların yorumunda daha özgürlükçüydü. Ve, Nasıra’da onlar aynı zamanda, Şabat’ın yerine getirilmesinde daha özgürlükçüydü. Tüm Celile’yi bir ucundan diğerine gören bir manzarayı elde edebilecekleri, evlerinin yakınındaki yüksek tepeye tırmanmanın onların gözde kısa keyifli gezintilerinden bir tanesi olan bir biçimde, Şabat öğleden sonraları yürüyüş yapmak için İsa’yı dışarı çıkarmak bu nedenle Yusuf’un âdetiydi. Havanın açık olduğu günlerde kuzeybatıya doğru onlar, Karmel Dağı’nın uzun dağ sırasının denize doğru uzandığını görebilmektelerdi; ve, birçok kez İsa Yusuf’un, Ahav’ı kınamış ve Baal’in din adamlarının foyasını ortaya çıkarmış olan İbrani din adamlarının uzun koluna ait ilk üyelerden bir tanesi olarak İlyas’ın hikâyesini anlatışını dinlemişti. Kuzey’de Hermon Dağı, sürekli yağan karla bembeyaz parıldayan 900 metreden fazla yükseklikteki yukarı yamaçlarıyla, karlı doruğunu ihtişamlı ışıltısıyla yükselmekte ve tek başına göğü kaplamaktaydı. Doğu ucunda onlar, Ürdün vadisini, ve onun çok uzağında, Moav’ın taş tepelerini seçebilmekteydiler. Aynı zamanda güney ve doğu yönünde, güneş mermer duvarları üzerinde ışıdığı zaman, onlar; amfi\hyp{}tiyatroları ve gösterişli mabetleriyle, Dekapolis’in Yunan\hyp{}Romalı şehirlerini görebilmektelerdi. Ve, güneşin batışına doğru bakışlarını uzun süre yönelttiklerinde, batıya doğru çok uzaktaki Akdeniz üzerindeki ilerlemekte olan deniz araçlarını görebilmektelerdi.
\vs p123 5:13 Dört bir yandan, İsa, Nasıra’ya girerken ve ondan ayrılırken yollarında ilerlemekte olan kervan kafilelerini gözlemleyebilmekteydi; ve, güneye doğru o, Gilboa Dağı ve Samaria Dağı’na doğru uzanmakta olan, Esdraelon’un geniş ve verimli ovasından meydana gelen şehrini üzerinden görebilmekteydi.
\vs p123 5:14 Uzak manzarayı görmek için yükseklere tırmanmadıkları zaman, onlar; şehrin dışına doğru yürüyüşte bulunup, mevsimlere bağlı olarak çeşitli dönemleri içinde doğayı incelediler. Ev ocağınki dışında İsa’nın en öncül eğitimi, doğa ile derin saygısal ve anlayışlı bir iletişim üzerine gerçekleşmişti.
\vs p123 5:15 Sekiz yaşından önce, İsa; evinden çok uzakta bulunmayan ve kasabanın bütünü için iletişimin ve dedikodunun toplumsal merkezlerinden bir tanesi olmuş olan pınarda, kendisiyle tanışmış ve konuşmuş Nasıra’nın tüm anneleri ve genç kadınları tarafından bilinmekteydi. Bu yıl, İsa, aile ineğini sağmayı ve diğer hayvanlara bakmayı öğrendi. O on yaşındayken, dokuma tezgâhını çalıştırmada usta olmuştu. Yaklaşık olarak bu dönemde, İsa ve komşu çocuğu Yakup, akan pınarın yanında çalışmış olan çömlekçinin çok iyi arkadaşları haline gelmişti; ve, onlar Nathan’ın mahir parmaklarını çömlekçi çarkında kile şekil verirken izlediklerinde, birçok kez her ikisi de, büyüdüklerinde çömlekçi olmaya karar vermişlerdi. Nathan bu gençleri çok sevmekte olup, sıklıkla onlara, çeşitli nesneleri ve hayvanları yapmada onları yarıştıran çabaları teşvik ederek yaratıcı hayal güçlerini harekete geçirmeyi arzulayan bir biçimde, oynamaları için kil vermekteydi.
\usection{6.\bibnobreakspace Onun Sekizinci Yaşı (M.S. 2.yıl)}
\vs p123 6:1 Bu yıl, okulda ilgi çekici bir seneydi. Kararlı bir öğrenci olan ve sınıfın daha fazla ilerleme kaydeden üçte birlik kısmında bulunan bir biçimde her ne kadar İsa olağandışı bir öğrenci olmasa da, çalışmasını o kadar güzel bir biçimde yerine getirmekteydi ki, her ayın bir haftası okula katılmaktan muaf tutulmaktaydı. Bu haftayı o genellikle; ya Mecdel yakınındaki Celile Denizi’nin kıyılarında balıkçılık yapmakta olan amcasıyla, veya, Nasıra’nın sekiz kilometre güneyindeki (annesinin abisi olan) diğer amcasının çiftliğinde geçirmekteydi.
\vs p123 6:2 Her ne kadar annesi öncesinden İsa’nın sağlığına ve güvenliğine dair haddinden fazla endişeli hale gelmiş bulunsa da, kademeli bir biçimde evden gerçekleştirdiği bu yolculukları kabul eder hale geldi. İsa’nın amacı ve teyzelerinin hepsi onu çok sevmekteydiler; ve orada, bu yıl ve bunun hemen ardındaki yıllar boyunca bahse konu aylık ziyaretlerde onu konuk etmek için araklarında etkili bir rekabet ortaya çıkmıştı. Onun (bebekliğinden beri) amcasının çiftliğindeki ilk haftalık konukluğu bu yılın Ocak ayındaydı; Celile Denizi üzerindeki balıkçılıkta ilk haftalık deneyim, Mayıs ayında gerçekleşmişti.
\vs p123 6:3 Yaklaşık olarak bu zaman zarfında İsa, Şam’dan gelen bir matematik öğretmeni ile tanıştı; ve, sayıların yeni belirli yöntemlerini öğrenen bir biçimde, zamanının büyük bir kısmını bir kaç yıl boyunca matematik üzerine harcadı. O; sayılara, uzaklıklara ve oranlara dair keskin bir anlayış geliştirdi.
\vs p123 6:4 İsa; kardeşi James’in varlığından o kadar büyük keyif almaya başlamış olup, bu yılın sonunda hali hazırda ona alfabeyi öğretmeye girişmişti.
\vs p123 6:5 Bu yıl İsa, arp üzerine ders alma karşılığında süt ürünlerini takas etmenin anlaşmalarında bulundu. O, müzik ile ilgili her şeye karşı olağandışı bir beğeniye sahipti. Daha sonra o, genç birlikteleri arasında ses müziğine bir ilgi yaratmak için fazlasıyla uğraş verdi On bir yaşında o; yetenekli bir arpist olup, hem ailesini hem de arkadaşlarını olağanüstü yorumlarıyla ve yetkin doğaçlamalarıyla eğlendirmekten fazlasıyla keyif aldı.
\vs p123 6:6 İsa, okulda kıskanılacak derecede gelişimde bulunmaya devam ederken, her şey ne ebeveynleri ne de öğretmenleri için kolay bir biçimde ilerlemedi. İsa, özellikle coğrafya ve gök bilimi ile olmak üzere, hem bilim hem de din ile ilgili birçok utandırıcı soru sormaya devam etmişti. O özellikle, Filistin’de neden kuru bir mevsimin ve yağmurlu bir mevsimin bulunduğunu anlamada ısrarcıydı. Tekrar eden bir biçimde, o, Nasıra ve Ürdün vadisi arasındaki büyük sıcaklık farklılığının açıklamasını arzuladı. O tek kelimeyle, bu tür ussal ancak kafa karıştırıcı soruları sormaya hiçbir zaman ara vermedi.
\vs p123 6:7 Onun üçüncü kardeşi, Şimon, M.S. 2.yıl olan bu yılın Nisan ayının 14’nde, Cuma akşamı doğmuştu.
\vs p123 6:8 Şubat ayında, hahamlara ait bir Kudüs akademisinde öğretmenlerden biri olarak Nahor, Kudüs yakında Zekeriya’nın evine gerçekleştirdiği benzer bir görevde bulunmuş olarak, İsa’yı gözlemlemek için Nasıra’ya gelmişti. O Nasıra’ya, Yahya’nın babası nedeniyle gelmişti. Her ne kadar Nahor ilk başta, İsa’nın dini şeyler ile kendisini ilişkilendirişinin açıklığı ve olağanın dışındaki biçimi nedeniyle bir ölçüde derin şaşkınlık içine düşmüşse de, bunu, Celile’nin İbrani öğrenim ve kültür merkezlerinden olan uzaklığına bağladı; ve, Nahor Yusuf ve Meryem’e, Musevi kültürünün merkezinde eğitim ve hazırlanmanın üstünlüklerini elde edebileceği yer olan Kudüs’e İsa’yı beraberinde götürmesine izin vermelerini tavsiye etti. Meryem, razı göstermeye yarı ikna olmuştu; o, en büyük oğlunun Musevi kurtarıcısı olarak Mesih haline gelecek oluşundan emindi; Yusuf, ikna olmada; o, Meryem ile eşit bir ölçüde, İsa’nın nihai sona ait bir insan haline gelen bir biçimde büyüyeceğine ikna olmuştu; ancak, bu nihai sonun onu tam olarak kim haline getireceğinde çok büyük ölçüde kararsızdı. Ancak, o gerçekten de hiçbir zaman, evladının dünya üzerinde belli bir büyük görevi yerine getirecek oluşundan kuşku duymamıştı. Nahor’un tavsiyesi üzerinde daha fazla düşündüğünde, Kudüs’de önerilen konukluğa dair bilgeliği daha fazla sorgular hale geldi.
\vs p123 6:9 Yusuf ile Meryem arasındaki bu görüş farklılığı nedeniyle, Nahor, tüm bu konuyu İsa’nın önüne sermek için izin istedi. İsa dikkatli bir biçimde dinledi; ve, Yusuf, Meryem, ve, çocuğu en gözde oyun arkadaşı olan taş ustası Yakup isminde bir komşu ile konuştu; ve, daha sonra, iki günün sonrasında, ebeveynleri ve danışmanları arasında bu türden bir görüş farklılığının mevcut bulunması, ve, lehte veya aleyhte kesin bir görüşe sahip olmayan bir biçimde bu türden bir karar için sorumluluğu almaya yetkin hissetmediği için, durumun bütünlüğünü göz önüne alarak nihai bir biçimde “cennetteki Yaratıcım ile konuşmaya” karar verdiğini bildirdi; ve, o her ne kadar aldığı cevap hakkında kusursuz bir biçimde emin olamasa da, “sadece bedenime bakabilmeye ve aklımı gözlemleyebilmeye yetkin olan ancak beni gerçekten bilebilmeye neredeyse hiçbir biçimde yetkin olamayacak yabancılara kıyasla, beni bu kadar derinden seven onların benim için daha fazlasını yerine getirebileceğini ve beni daha güvenli bir biçimde yönlendirebileceğini” ekleyerek, bunun yerine “annem ve babam ile” evde kalmaya devam etmesi gerektiğini hissetti. Onların hepsi şaşkına döndü, ve Nahor Kudüs’e geri olan yoluna devam etti. Ve, birçok sene boyunca, İsa’nın evden ayrılma konusu, değerlendirilmesi için tekrar gündeme gelmedi.
