\upaper{65}{Evrim’in Yüksek Denetimi}
\vs p065 0:1 Akil\hyp{}öncesi yaşam olarak evrimsel maddi yaşamın temeli; görevlendirilen Yaşam Taşıyıcıları’nın etkin hizmetleri ile ortak bir biçimde, Üstün Fiziksel Düzenleyicileri’nin tasarımı ve Yedi Üstün Ruhaniyet’in yaşam\hyp{}aktarım hizmetidir. Bu üç katmanlı yaratıcılığın eş güdümsel faaliyetinin bir sonucu olarak orada; --- ilk önce dışsal çevre uyarılara karşı ve daha sonra organik bütünlüğün aklı içinde gerçekleşen etkiler biçimindeki içsel uyarılara karşı ussal tepkileri gerçekleştiren maddi işleyiş düzenleri olarak --- akıl için organik bütünlük içinde faaliyet gösteren fiziksel yetkinlik gelişmiştir.
\vs p065 0:2 Bunun sonucunda orada, yaşam üretimi ve evriminin üç farklı düzeyi bulunmaktadır:
\vs p065 0:3 1.\bibnobreakspace Akıl\hyp{}yetkinlik üretimi olarak fiziksel\hyp{}enerji nüfuz alanı.
\vs p065 0:4 2.\bibnobreakspace Ruhaniyet yetkinliğini belirleyen bir biçimde emir\hyp{}yardımcı ruhaniyetlerinin akıl hizmeti.
\vs p065 0:5 3.\bibnobreakspace Düşünce Düzenleyicisi’nin bahşedilişi ile sonuçlanan bir biçimde fani aklın ruhani kazanımı.
\vs p065 0:6 Organik bütünlüğün çevresel tepkilerini oluşturan işleyiş düzeninin bir parçası niteliğinde kendiliğinden gerçekleşen düzeyler, fiziksel düzenleyicilerin nüfuz alanlarına girmektedir. Emir\hyp{}yardımcı akıl\hyp{}ruhaniyetleri; deneyimlerden öğrenmeye yetkin olan organizmaların bu tepkisel işleyiş düzenleri olarak, aklın uyumsal veya diğer bir değişle işleyişsel düzeninin bir parçası olmayan öğretilebilir türlerini etkinleştirip onları düzenler. Ruhaniyet emir\hyp{}yardımcıları böylelikle akıl yetkinliklerini değiştirirken, Yaşam Düzenleyicileri ise; Tanrı’yı tanıma yetkinliği ve onun için ibadet etme gücü olarak, insan iradesinin ortaya çıkış zamanına kadar evrimsel süreçlerin çevresel nitelikleri üzerinde dikkate değer takdir yetkilerini kullanan biçimde denetimlerini uygularlar
\vs p065 0:7 Bu düzen; --- yerleşik dünyalar üzerinde organik evrimin ilerleyişini belirleyen --- Yaşam Taşıyıcılar, fiziksel düzenleyiciler ve ruhaniyet emir\hyp{}yardımcılarının bütünleşmiş faaliyetidir. Ve bu nedenle --- Urantia üzerinde veya herhangi bir yerleşkede --- evrim her zaman bir amaç dâhilinde gerçekleşmiş olup, hiçbir biçimde şans eseri meydana gelmemiştir.
\usection{1.\bibnobreakspace Yaşam Taşıyıcı Faaliyetleri}
\vs p065 1:1 Yaşam Taşıyıcıları, yaratılmışlarının çok az sayıdaki düzeyinin sahip olduğu kişilik başkalaşımının yetkinliği ile donatılmıştır. Yerel evreninin bu Evlatları, varlığın üç farklı fazı içinde faaliyet göstermeye yetkindir. Onlar genellikle, kökenlerinin düzeyi olarak ara\hyp{}faz Evlatları biçiminde görevlerini yerine getirir. Ancak bir Yaşam Taşıyıcısı mevcudiyetin bu türden bir aşaması içinde, fiziksel enerjileri ve maddi parçacıkları yaşam mevcudiyetinin birimlerine dönüştüren bir üretim varlığı biçiminde elektr0kimyasal nüfuz alanlarında hiçbir biçimde faaliyet gösteremezdi.
\vs p065 1:2 Yaşam Taşıyıcıları, şu üç düzey içinde faaliyet göstermeye yetkin olup, bu faaliyetleri düzenli bir biçimde yerine getirirler:
\vs p065 1:3 1.\bibnobreakspace Elektrokimyanın fiziksel düzeyi.
\vs p065 1:4 2.\bibnobreakspace Morontiyal mevcudiyet görünümünün olağan ara\hyp{}fazı.
\vs p065 1:5 3.\bibnobreakspace Gelişmiş yarı\hyp{}ruhaniyet düzeyi.
\vs p065 1:6 Yaşam Taşıyıcıları; yaşam aktarımına katılmak için hazır hale geldiğinde, ve bu türden bir teşebbüs için yerleşkelerini seçtiklerinde, Yaşam Taşıyıcı başkalaşım heyetinin baş meleğini çağırırlar. Bu topluluk; fiziksel düzenleyiciler ve onların birlikteliklerini içine alan bir biçimde farklı kişiliklerin on düzeyinden meydana gelmekte olup, Cebrail’in görevlendirilmesi ve Zamanın Ataları’nın izni ile bu yetkinlik içerisinde faaliyet gösteren baş meleklerin yöneticisi tarafından idare edilir. Bu varlıklar gerektiği gibi konumlandırıldıkları zaman, elektrokimyanın fiziksel düzeyleri içinde Yaşam Taşıyıcıları’nın derhal faaliyet gösterebilmesini mümkün kılan bu tür değişikleri sağlayabilirler.
\vs p065 1:7 Yaşam işleyiş yöntemleri tasarlandığında ve maddi düzenlemeler gerektiği gibi yerine getirildiğinde, yaşam çoğalımı ile ilgili aşkın\hyp{}maddi kuvvetler derhal etkin hale gelir, ve yaşam kendi mevcudiyetine kavuşur. Bu gelişim üzerine Yaşam Taşıyıcıları; --- yarama biçiminde --- yaşayan maddenin yeni işleyiş biçimlerini düzenlemeye dair tüm yetkinliklerinden mahrum kalsalar da yaşayan birimler üzerinde değişikliği sağlayacakları ve evrimleşen organizmaların gidişatını belirleyecekleri konum olan kişilik mevcudiyetinizin olağan ara\hyp{}fazına, derhal geri dönüştürülürler.
\vs p065 1:8 Belirlenmiş bir gidişat içinde organik evrimin seyrine başlamasından ve insan türünün özgür iradesi evrimleşen en yüksek organizmalar içinde ortaya çıktıktan sonra, Yaşam Taşıyıcıları ya gezegeni terk etmek zorunda veya görevlerinden ayrılış yeminlerini etmek durumundadır; bu yemin, organik evriminin gidişatını ilave bir biçimde etkilemelerine dair tüm girişimlerinden kendilerini mahrum bırakan bir taahhüttür. Ve bu türden yeminler, yeni evrimleşen irade yaratılmışlarının gelişimi için güvenilebilecek gelecekteki danışmanlar olarak gezegen üzerinde kalmayı tercih eden bu Yaşam Taşıyıcıları tarafından gönüllü bir biçimde yerine getirildiği zaman; Sistem Egemeni’nin yönetim yetkisi adına ve Cebrail’in izni ile hareket eden Akşam Yıldızları’nın baş yöneticisinin başkanlığında on iki unsurdan oluşan bir heyet toplanır; ve bunun üzerine bu Yaşam Taşıyıcıları derhal, varlığın yarı\hyp{}ruhsal düzeyi olan kişilik mevcudiyetinin üçüncü fazına dönüştürürler. Ve Urantia üzerinde ben, Andon ve Fonta zamanından beri bu üçüncü faz içerisinde faaliyet göstermekteyim.
\vs p065 1:9 Bizler; içinde bütünüyle ruhsal hale gelebileceğimiz olası bir dördüncü düzey biçiminde, evrenin ışık ve yaşam altında istikrara kavuşturulabileceği bir zaman zarfını dört gözle beklemekteyiz; ancak bu türden arzulanan ve gelişmiş düzeye hangi işleyiş biçimleri ile ulaşabileceğimiz tarafımıza hiçbir zaman açığa çıkarılmamıştır.
\usection{2.\bibnobreakspace Evrimsel Bütünlüğün Görünümü}
\vs p065 2:1 İnsanın su yosunundan dünyasal yaratım hâkimiyetine kadar yükselişi gerçek anlamıyla biyolojik mücadele ve akıl kurtuluşuna ait destansı bir serüvendir. İnsanın ilkel ataları, gerçek anlamıyla; Yaşam Taşıyıcıları’nın Urantia üzerinde üç bağımsız yaşam aktarımını konumladıkları yer olan, tarihi iç denizlerinin fazlasıyla geniş kıyı şeritlerine ait durgun sıcak\hyp{}su körfezleri ve gölcükleri içinde okyanus tabanının balçıkları ve çamurlarıydı.
\vs p065 2:2 Sınırda bulunan hayvansı organizmalarla sonuçlanan tarihi değişiklikleri beraberinde getirmiş deniz bitki yaşamının öncül türlerinin çok azı mevcut zaman zarfında varlığını devam ettirmektedir. Süngerler; bitkiden hayvana olan \bibemph{kademeli} geçişin konumsal olarak meydana geldiği varlıklar olarak, öncül ara türlerinden birinin varlığını devam ettirebilmiş canlılarıdır. Bu öncül geçiş türleri, her ne kadar çağdaş süngerler ile özdeş olmasalar da, onlara çok benzemekteydiler; onlar gerçekten, ne bitkiye ne de hayvana benzemeyen bir biçimde --- sınır organizmalarıydılar; ancak onlar nihai olarak, yaşamın gerçek hayvan türlerinin gelişimine öncülük etmişlerdir.
\vs p065 2:3 En ilkel doğanının basit bitki organizmaları olarak bakteriler, yaşamın öncül doğuş döneminden itibaren çok az değişiklik göstermiştir; onlar, parazit davranışı bakımından kısmı bir gerileme bile sergilemişlerdir. Aynı zamanda mantarların birçoğu; klorofil üretme yetkinliklerini kaybederek ve neredeyse parazitsel bir nitelik kazanan bitkiler olarak, evrim bakımından bir geri yönlü ilerleme sergilemişlerdir. Hastalığa neden olan bakterilerin büyük bir çoğunluğu ve onların uzuv bedenleri, gerçekten de gerilemiş parazitsel mantarların bu topluluğuna aittir. Ara çağlar boyunca bitki yaşamının geniş krallığının bütünü, bakterilerin aynı zamanda türemiş olduğu atalardan evirilmiştir.
\vs p065 2:4 Hayvan yaşamının daha yüksek tek hücreli türü yakın zamanda ve \bibemph{ansızın} ortaya çıkmıştır. Ve bu uzak zamanlardan tipik tek hücreli hayvan organizması olan amip varlığını çok az değişikliğe uğramış bir biçimde sürdürmüştür. Bu canlı, yaşam evrimi içinde en son ve en büyük kazanım olarak ortaya çıktığında gerçekleştirdiği gibi mevcut an içerisinde hareketine devam etmektedir. Bu ufak canlı ve onun tek hücreli kuzenleri, bakteriler bitki krallığı içinde nasıl bir konumda bulunuyorsa, hayvan yaratımı içinde o konumda ikamet etmektedir; onlar, \bibemph{ileri süreçlerdeki gelişimin başarısızlığına} ek olarak yaşam farklılaşması içinde ilk öncül evrimsel aşamaların kurtuluşunu temsil etmektedir.
\vs p065 2:5 Öncül tek hücreli hayvan türlerinin kendilerini topluluklar halinde bir araya getirdikleri süreçten sonra yakın zaman içerisinde, ilk olarak Volvoks birimleri ve onların ardından Polipler ve denizanalarının kolları bütünlük kurdular. Bu gelişmeler sonrasında orada; denizyıldızları, kaya zambakları, denizkestaneleri, denizhıyarları, çıyanlar, böcekler, örümcekler, kabuklular ve yeryüzü solucanlarının ve sülüklerin yakından ilişkili topluluklarını sonradan takip eden --- istiridye, ahtapot ve salyangoz olarak --- yumuşakçalar evirilmiştir. Yüzlerce tür bu zaman zarfında ortaya çıkmış ve yok olmuştur; bu anlatımda yalnızca uzun mücadeleler sonrasında hayatta kalmış türlere yer verilmiştir. Daha sonra ortaya çıkan balık ailesi ile birlikte bu türden ilerleme göstermeyen canlı örnekleri bugün, gelişimde başarısız yaşam ağaçlarının kolları olarak öncül ve daha alt düzey hayvanların sabit türlerini temsil etmektedirler.
\vs p065 2:6 Bu gelişim, balıklar olarak omurgalı hayvanlarının ortaya çıkışına böylelikle zemin hazırlamıştır. Balık ailesinden, kurbağa ve semender canlıları olarak iki benzersiz dönüşüm açığa çıkmıştır. Ve, insanın ortaya çıkışı ile nihai olarak sonlanan hayvan yaşamı içerisinde ilerleyici farklılaşma dizisini başlatan kurbağa canlısı olmuştur.
\vs p065 2:7 Kurbağa, hayatta kalan insan\hyp{}ırk atalarının en öncül varlıklarından bir tanesidir; ancak aynı zamanda o, bu uzak zamanlardakine oldukça benzer bir biçimde bugün mevcudiyetine sahip olarak, gelişmede başarısız olmuştur. Kurbağa, dünya yüzeyi üzerinde mevcut an içerisinde yaşayan öncül doğum ırklarının ataları olan tek varlıktır. İnsan ırkı, kurbağa ve Eskimo arasında hayatta kalmış hiçbir ataya sahip değildir.
\vs p065 2:8 Kurbağalar, neredeyse tamamen nesli tükenmiş bir büyük hayvan ailesi olan Sürüngenleri dünyaya getirmiştir; yok olmalarından önce bu aile, bütüncül kuş ailesi ve memelilerin sayısız düzeylerinin kökenini oluşturmuştur.
\vs p065 2:9 Muhtemelen, insan\hyp{}öncesi evrimin tamamına ait bu en büyük nitelikteki tek adım, sürüngen kuş haline geldiğinde atılmıştır. Kartallar, ördekler, güvercinler ve deve kuşları olarak bugünün kuş türlerinin tümü, devasa sürüngenlerden çok uzun bir süre önce türemiştir.
\vs p065 2:10 Kurbağa ailesinden türemiş olan sürüngenlerin krallığı mevcut an içerisinde varlığını devam ettiren dört kol tarafından temsil edilmektedir: yılanlar ve kertenkelelere ilaveten timsahlar ve kaplumbağalar olarak onların kuzenleri biçiminde gelişmeyen iki kol; kuş ailesi olarak kısmi gelişim halindeki bir kol; ve memelilerin atalarına ek olarak insan türlerinin doğrudan nesilleri halindeki dördüncü kol mevcut bulunmaktadır. Her ne kadar nesilleri uzun süre önce yok olsa da Sürüngenler ailesi fil ve mastodon canlıları içinde özelliklerini sergilemiştir; bunun karşısında ise onların özel türleri, sıçrayan kanguruların ortaya çıkmasına zemin hazırlamıştır.
\vs p065 2:11 Urantia üzerinde yalnızca on dört hayvan soyu ortaya çıkmıştır; balıklar bu soyların en sonuncusu olarak, kuşlar ve memelilerden beri hiçbir yeni canlı sınıfı gelişmemiştir.
\vs p065 2:12 Et ile beslenen çevik özellikli minyon yapılı ama göreceli olarak büyük bir beyne sahip olan sürüngen dinozorundan karın\hyp{}bağı memelileri \bibemph{ansızın} türemiştir. Yalnızca sıkça görülen çağdaş memeli çeşitlerinin ortaya çıkmasına zemin hazırlamayan aynı zamanda balinalar ve ayıbalıkları gibi deniz türlerine ek olarak yarasa ailesi gibi uçan canlılara da türeyen bir biçimde bu memeliler, hızlı bir şekilde ve çok çeşitli biçimlerde gelişmişlerdir.
\vs p065 2:13 İnsan böylelikle başlıca olarak, doğu\hyp{}batı kuşağında bulunan kapalı tarihi denizlerdeki \bibemph{batı yaşam aktarımından} türemiştir. Yaşayan organizmaların \bibemph{doğu} ve \bibemph{merkezi} \bibemph{toplulukları}, hayvan mevcudiyetinin insan\hyp{}öncesi düzeylerine erişim için daha elverişli bir biçimde öncül olarak ilerlemekteydi. Ancak çağlar ilerledikçe yaşam aktarımının doğu odağı; insan yetkinliklerini canlandırma gücünü sonsuza kadar ondan mahrum bırakan çekirdek plazmasının en yüksek türlerine ait tekrar eden ve geri döndürülemez kayıpları deneyimleyerek, insan\hyp{}öncesi ussal düzeyin tatmin edici bir düzeyine erişmede başarısız olmuştur.
\vs p065 2:14 Bu doğu topluluğu içinde akıl yetkinliğinin kalitesi gelişim için diğer iki topluluktan kesin bir biçimde o kadar alçak bir düzeyde bulunmaktaydı ki, Yaşam Taşıyıcıları üstlerinin rızasını alarak evrimleşen yaşamın bu alçak düzey insan\hyp{}öncesi ırk kollarını ilave bir biçimde sınırlandırmak için çevre koşullarında değişiklikte bulunmuşlardı. Yaratılmışların bu alçak düzey topluluklarının ortadan kalkışının tümü dışarıdan bakıldığında şans eseri gerçekleşen bir görünüme sahipti, ancak gerçekte bu olay tamamiyle amaç dâhilinde gerçekleşmiştir.
\vs p065 2:15 Usun evrimsel ortaya çıkışı içinde daha sonra, insan türlerinin lemur ataları diğer bölgelere kıyasla Kuzey Amerika’ya oldukça etraflı bir biçimde yayılmıştır; ve onlar bu şartlar nedeniyle, batı yaşam aktarım bölgesinden hareket ederek Bering kara köprüsü üzerinden geçip, evirilmeye devam ettikleri ve merkezi yaşam topluluğunun belirli ırk kollarının eklemlenmesinden yarar sağladıkları yer olan, güneybatı Asya sahiline göç etmeye yönlendirilmişlerdir. İnsan böylelikle, belirli batı ve merkezi yaşam ırk kollarından evrimleşmiş olup ancak yakın doğu bölgelerin merkezinde gelişmiştir.
\vs p065 2:16 Bu anlatıldığı biçimiyle Urantia’ya aktarılan yaşam, insanın ortaya çıktığı ve kendisinin dikkate değer gezegensel sürecine başladığı çağ olan buz devrine kadar evirilmişti. Ve buz dönemi boyunca ilkel insanın bu ortaya çıkışı, bir şans eseri sonucu meydana gelmemiştir; bu oluşum bir tasarım uyarınca gerçekleşmiştir. Buzul döneminin sıkıntıları ve mevsimsel zorlukları her bakımdan, güçlü bir insan türünün devasa bir kurtuluş kazanımı ile yetişmesini desteklemek amacına göre uyarlanmıştır.
\usection{3.\bibnobreakspace Evrimin Desteklenişi}
\vs p065 3:1 Bugünün insan aklına öncül evrimsel ilerleyişin tuhaf ve görünüşte anlamsız gelen garip olayların çoğunu açıklamaya çalışmak neredeyse imkânsız olacaktır. Yaşayan varlıkların bu görünüşteki tuhaf evrimlerinin tümü boyunca amacı olan bir tasarı faaliyet göstermekteydi; ancak yaşam işleyiş biçimleri bir kez işlerlik kazandıktan sonra bizlerin onlara keyfi bir biçimde müdahale etme yetkimiz bulunmamaktadır.
\vs p065 3:2 Yaşam Taşıyıcıları, yaşam deneyiminin gelişimsel ilerleyişini destekleyecek olası her doğal kaynağı kullanabilir ve rastlantısal olayların herhangi birinden veya hepsinden faydalanabilir; ancak bizlerin, bitki veya hayvan evriminin işleyişi ve gidişatı üzerinde anlık müdahalede bulunmamıza veya keyfi bir biçimde değişikliğe gitmemize izin verilmemektedir.
\vs p065 3:3 Urantia fanilerinin ilkel kurbağa gelişimi vasıtasıyla evirildiği ve belirli bir olayda yok olmaktan kıl payı kurtulan tek bir kurbağa içinde potansiyel olarak taşınan bu yükseliş kolu hakkında bilgilendirilmiş bir konumda bulunmaktasınız. Ancak insan türü evriminin bu dönüm noktasında bir kaza sonucu tamamiyle sonlanabilecek olma ihtimalinizin çıkarımında bulunulmamalıdır. Bu zaman zarfında bizler, insan\hyp{}öncesi gelişimin farklılaşmış çeşitli doğum biçimleriyle sonuçlanabilecek olan binden fazla farklı ve birbirlerinden ayrı konumda ikamet eden yaşamın başkalaşan kollarını gözlemlemekte ve onları desteklemekteydik. Bu özel atasal kurbağa bizim üçüncü tercihimizi simgelemektedir; bundan önceki iki yaşam kolu, korunumu için tüm çabalarımıza rağmen yok olmuştur.
\vs p065 3:4 Doğumlarından önce Andon ve Fonta’nın kaybı bile, her ne kadar insan evrimini geciktirecek olsa da, onu tamamiyle engellemiş olmayacaktı. Andon ve Fonta’nın ortaya çıkışından sonra ve hayvan yaşamının başkalaşan insan potansiyellerin gerçekleşmesinden önce orada, gelişimin benzer insan türüne erişebilecek yedi binden fazla elverişli yaşam kolu gelişmişti. Ve bu daha iyi yaşam ırk kollarının çoğu, genişleyen insan türlerinin çeşitli dalları tarafından ilerleyen zamanlarda ortadan kaldırılmıştır.
\vs p065 3:5 Biyolojik canlandırıcılar olan Maddi Erkek ve Kız Evladı’nın ortaya çıkışından çok uzun zaman önce, evrimleşen hayvan türlerine ait insan potansiyellerinin hepsi denenmiştir. Hayvan yaşamının bu biyolojik düzeyi; insan\hyp{}öncesi bireylerin başkalaşıma uğramış potansiyellerine kaynak sağlamak amacıyla hayvan yaşamının tümünün yetkinliğinin denenmesiyle eş zamanlı bir biçimde kendiliğinden ortaya çıkan, emir yardımcı ruhaniyet ilerleyişinin üçüncü fazına ait oluşumla Yaşam Taşıyıcıları için apaçık bir hal almıştır.
\vs p065 3:6 Urantia üzerinde insan türü, sahip olduğu insan ırk kollarıyla birlikte fani gelişimine ait sorunlarını kendi başına çözmek durumundadır --- insan\hyp{}öncesi kaynaklardan ilave ırklar bir daha sonsuza kadar evirilmeyecektir. Ancak bu gerçek, fani ırklar içinde hâlihazırda barınmakta olan evrimsel potansiyellerin ussal bir biçimde gelişimi vasıtasıyla insan ilerleyişinin daha yüksek engin düzeylerine olan erişime engel olmamaktadır. Yaşam Taşıyıcıları olarak bizlerin gerçekleştirdiği insan iradesinin ortaya çıkışından önce yaşamı desteklemeye ve onu korumaya dönük çabalarımız gibi, bu türden oluşum sonrasında ve evrime olan etkin katılımımızı sonlandırmamızın peşi sıra insan, bu görevleri kendisi için yerine getirmek durumundadır. Genel anlamda insanın evrimsel kaderi kendi ellinde olup, düzenlenmemiş doğal seçilim ve şanssal kurtuluşun rastgele gerçekleşen faaliyetinin yerini bilimsel us er veya geç almak zorundadır.
\vs p065 3:7 Çok ileriki zamanlarda belirli bir süreç içerisinde Yaşam Taşıyıcılar’ın bir birliğine bağlı hale geldiğiniz zaman, yaşam idaresi ve aktarımının tasarımı ve işleyiş biçimi konularda tavsiyelerde bulunmak ve olası her tür ilerleyici düzenlemeleri gerçekleştirmek için çok sayıda imkâna sahip olacağınızın evrimin desteklenmesi hususunda belirtilmesi yanlış olmaz. Sabırlı olun! Evrensel nüfuz alanlarının herhangi bir parçasına ait idarenin daha iyi yöntemlerini geliştirme bakımından akıllarınızın verimli olduğu durum biçiminde iyi fikirlere sahip olursanız, gelecek çağlar boyunca birliktelik içinde bulunduğunuz unsurlara ve görevdaş idarecilerinize bu fikirleri sunmak için bir imkâna kesin bir biçimde sahip olacaksınız.
\usection{4.\bibnobreakspace Urantia Serüveni}
\vs p065 4:1 Urantia’nın bir yaşam\hyp{}deneyim dünyası olarak bizlere görev yerleşkesi olarak verildiği gerçeğini gözden kaçırmayın. Bizler, Nebadon yaşam tasarımlarının Satania uyumluluğunu değiştirmek ve mümkünse onu geliştirmek için altmışıncı girişimimizi bu gezegen üzerinde gerçekleştirdik; ortak yaşam işleyiş biçimlerinin sayısız yararlı değişikliğini başarıyla gerçekleştirdiğimiz kayıtlara girmiştir. Daha detaylı ifadeyle, Urantia üzerinde bizler; bundan sonraki gelecek zamanın tümü boyunca Nebadon’un tamamı için hizmete hazır halde bulunacak yaşam değişiminin yirmi sekiz çeşidini denemiş ve onları başarıyla göstermiş bulunmaktayız.
\vs p065 4:2 Ancak yaşam oluşumu hiçbir dünya üzerinde, denenmemiş bir şeyin denendiği veya bilinmeyen bir şeyin gerçekleştirilmeye kalkışıldığı anlamda deneyimsel değildir. Yaşamın evrimi; her zaman ilerleyici, farklılaşan ve çeşitlilik gösteren bir işleyiş biçimidir; ve bu evrim hiçbir zaman rastgele, düzenlenmemiş veya şans eseri gerçekleşen anlamda tamamiyle deneyimsel değildir.
\vs p065 4:3 İnsan yaşamının birçok özelliği; organik evrimin yalnızca kâinatsal bir kaza eseri gerçekleşmediği biçiminde, fani mevcudiyet olgusunun ussal bir biçimde tasarlandığına işaret eden bolca kanıtı sunmaktadır. Yaşayan bir hücre yaralandığı zaman, yara içinde iyileşme süreçlerini başlatacak belirli özlerin derhal salgılanışını sağlamak için sağlıklı komşu hücreleri uyarma ve onları etkinleştirme gücünü barındıran belirli kimyasal maddeleri harekete geçirme yetkinliğine sahiptir; ve bu zaman zarfında bahse konu sağlıklı ve zarar görmemiş hücreler, gerçekte kaza sonucu zarar görmüş olabilecek olası her akran hücreyi değiştirmek için yeni hücreleri üretmeye başlayarak --- çoğalmaya koyulurlar.
\vs p065 4:4 Yaranın iyileşmesi ve hücre yenilenmesindeki bu kimyasal etki ve tepki, olası kimyasal tepkimeler ve biyolojik sonuçların yüz bin fazı ve özelliği arasındaki bir formülü Yaşam Taşıyıcılar’ın tercih edişini yansıtmaktadır. Bu formül Urantia yaşam deneyimi içinde nihai olarak yürürlüğe konmadan önce, yarım milyondan fazla özel deney Yaşam Taşıyıcılar tarafından kendi laboratuarlarında gerçekleştirilmiştir.
\vs p065 4:5 Urantia bilim adamları bu iyileştirici kimyasallar hakkında daha fazla bilgiye sahip olduğunda, yaraların iyileşmesinde daha etkin hale gelecekler; ve dolaylı bir biçimde onlar, belirli ciddi hastalıkların denetim altına alınmasına dair daha fazla bilgiye sahip olacaklar.
\vs p065 4:6 Urantia üzerinde yaşam oluşturulduğundan beri Yaşam Taşıyıcıları, diğer Satania dünyasına aktarılmış bu iyileştirme biçimini geliştirmişlerdir; bu gelişim içerisinde iyileştirme biçimi acıyı daha fazla dindirişi sunmakta ve ilgili sağlıklı hücrelerin çoğalım yetkinliği üzerinde daha iyi denetimi sağlamaktadır.
\vs p065 4:7 Urantia yaşam deneyimine ait orada birçok benzersiz özellik bulunmaktaydı; ancak onların arasında iki olağanüstü gelişim, altı renkli ırkın evriminden önce Andonsal ırkın ortaya çıkışı ve daha sonrasında ise Sangik başkalaşımların tek bir aile içinde eş zamanlı olarak belirişidir. Urantia, aynı insan ailesinden altı renkli ırkın türediği Satania üzerindeki ilk dünyadır. Bu ırklar genellikle; insan\hyp{}öncesi hayvan yaşam kolu içerisinde bağımsız başkalaşımlardan çeşitli ırk kolları halinde ortaya çıkmakta, kırmızı ırktan başlayarak her dönemde tek yeni bir ırkın çivit ırkına kadar inen sıralı ilerleyişi uzun süreçler boyunca sıklıkla gerçekleşir.
\vs p065 4:8 İşleyiş düzenine ait bir diğer olağanüstü çeşitlilik Gezegensel Prens’in varışının geç bir zaman zarfında gerçekleşmiş olmasıydı. Bir kural olarak prens, idare gelişimin gerçekleştiği zaman sularında bir gezegen üzerinde ortaya çıkar; ve eğer bu tasarı takip edilseydi, altı Sangik ırkının ortaya çıkmasıyla eş zamanlı olarak Andon ve Fonta’nın yaşadığı dönemden neredeyse beş yüz bin yıl sonra Urantia’ya gelmek yerine bu ilk iki insan yaşam döneminde bile Caligastia bu gezegene varmış olabilirdi.
\vs p065 4:9 Olağan bir yerleşik dünya üzerinde bir Gezegensel Prens, Andon ve Fonta’nın ortaya çıkışı üzerine veya bu gelişimden kısa bir süre sonra Yaşam Taşıyıcıları’nın talebiyle görevlendirilmiş olurdu. Ancak Urantia bir yaşam\hyp{}dönüşüm gezegeni olarak sınıflandırıldığı için, Gezegensel Prens’in gezegene olan varışına kadar sayıca on iki Melçizedek gözlemcisinin Yaşam Taşıyıcıları’na danışmanlar olarak bağlanması ve gezegenin yüksek denetimcileri olarak görevlendirilmesi önceden karara bağlanmış bir durumdu. Bu Melçizedekler; Andon ve Fonta zamanında gezegene ulaşmış olup, Düşünce Düzenleyiciler’in bu iki insanın fani akıllarında ikamet edişini etkin hale getiren kararları almışlardır.
\vs p065 4:10 Yaşam Taşıyıcıları’nın Satania yaşam işleyiş biçimlerini Urantia üzerinde geliştirme çabaları, geçiş yaşamının görünüşte kullanışsız birçok türünün ortaya çıkışını ister istemez beraberinde getirmiştir. Ancak elde edilen hâlihazırdaki kazanımlar, ortak yaşam tasarımlarının Urantia değişikliklerini haklı çıkaracak yeterli düzeyde bulunmaktadır.
\vs p065 4:11 Urantia’nın evrimsel yaşamı içinde iradenin öncül dışavurumunu üretmek bizim başlıca gayemizdi, ve biz bu amacı yerine getirdik. Genellikle irade; çoğu kez kızıl ırkın üstün türleri arasında ilk kez ortaya çıkan bir biçimde, renkli ırkların uzun süreler boyunca yaşamalarına kadar ortaya çıkmamaktadır. Sizin dünyanız, iradenin insan türünün renkli ırk\hyp{}öncesi bir kökende ortaya çıktığı Satania’da ki tek gezegendir.
\vs p065 4:12 Ancak, insan ırkının memeli atalarına nihai olarak köken sağlayacak kalıtımsal etkenlerin karışımı ve birlikteliğini sağlamak için gerçekleştirdiğimiz çabalarda, bu kalıtımsal etkenlerin yüz binlerce görece yararsız karışımlarının ve birlikteliklerinin ortaya çıkmasına izin verme zorunluluğu ile karşılaştık. Çabalarımızın bu görünüşte tuhaf yan sonuçlarının çoğu, gezegensel tarihin derinliklerine indiğinizde gözünüze kesin bir biçimde çarpacaktır; ve ben, kısıtlı insan bakış açısı için bu türden şeylerin ne kadar kafa karıştırıcı olduğunu oldukça iyi bir biçimde anlayabilmekteyim.
\usection{5.\bibnobreakspace Yaşam Evrimi’nin Beklenmemiş Talihsizlikleri}
\vs p065 5:1 Urantia üzerinde ussal yaşam üzerinde değişikliklerde bulunmak için özel çabalarımızın, Caligastia ihaneti ve Âdemsel bozulma olarak denetimimizin ötesinde bulunan vahim sapkınlıklar tarafından oldukça engellenmiş bir hale gelmiş olması Yaşam Taşıyıcıları olarak bizler için derin bir üzüntü kaynağıdır.
\vs p065 5:2 Ancak bu biyolojik serüvenin tamamı boyunca yaşadığımız en büyük hayal kırıklığı; belirli ilkel bitki yaşam türünün geniş ve beklenmeyen ölçüde parazit bakterisinin klorofil\hyp{}öncesi düzeylerine geri dönüşüydü. Bitki yaşamı evrimi içindeki bu nihai gelişim, başlıca daha kırılgan olan insan varlıkları olmak üzere üst düzeydeki memeliler içinde birçok acı hastalığa neden oldu. Biz bu kafa karıştırıcı durumla karşılaştığımız zaman, bir şekilde ortaya çıkan zorlukları fazla önemsemedik; çünkü bizler bilmekteydik ki, organizmanın bitkisel türü tarafından üretilen tüm hastalıklara karşı insanı neredeyse tamamen bağışık hale getirecek bir biçimde birbirine karışan ırkların meydana getireceği dirençsel güçleri Âdemsel yaşam plazmasının ilerleyen süreçlerdeki birleşiminin yeniden güçlendireceğini bilmekteydik. Ancak bizim bu umutlarımız, Âdemsel yoldan çıkışın talihsiz gelişimi nedeniyle boş çıkmaya mahkûm hale gelmişti.
\vs p065 5:3 Urantia olarak adlandırılan bu küçük dünyayı içine alan kâinat âlemlerinin tümü, yalnızca bizlerin onayından geçmesi için veya sadece beklentilerimizi karşılaması için idare edilmemektedir; ve bu erekler bir yana dursun, onların idaresi hiçbir biçimde heveslerimizi tatmin etmesi veya merakımızı gidermesi için gerçekleşmemektedir. Evren idaresi için sorumlu olan bilge ve her şeye gücü yeten varlıklar kuşkusuz, neler ile karşılaşacaklarını çok iyi bir biçimde bilmektedir; ve bu durum Yaşam Taşıyıcıları için gerçeklik taşırken, fani akılların sabırlı bekleyiş içinde ve içten eş güdüm halinde bilgeliğin iradesi, gücün hâkimiyeti ve ilerleyişin adımlarına olan katılımlarını gerektirmektedir.
\vs p065 5:4 Tabii ki orada Mikâil’in bahşedilişi gibi yaşanan felaketlerin telafisi için belirli olumlu gelişmeler gerçekleşmiştir. Ancak bu türden olumlu olumsuz tüm gelişmelerden bağımsız olarak bu gezegenin daha sonraki göksel yüksek denetimcileri, insan ırkının nihai evrimsel zaferine ve Yaşam Taşıyıcıları olarak bizlerin özgün tasarımları ve yaşam işleyiş biçimlerinin er ya da geç haklı çıkacağına dair eksiksiz güveni dile getirmektedirler.
\usection{6.\bibnobreakspace Yaşamın Evrimsel İşleyiş Biçimleri}
\vs p065 6:1 Hareket eden bir eşyanın kesin yer ve hızını doğru bir biçimde eş zamanlı olarak belirlemek imkânsızdır; yer veya hızın ölçümüne dair herhangi bir girişim kaçınılmaz olarak bir diğerinde açığa çıkacak değişikliği beraberinde getirecektir. Bu türden bir çelişki, fani insanın protoplazmanın kimyasal analizini yapmaya giriştiğinde karşılaştığı bir durumdur. Kimyager, \bibemph{ölü} protoplazmanın kimyasını açıklayabilir; ancak kendisi, \bibemph{yaşayan} protoplazmanın ne fiziksel düzenlenişini ne de devinimsel oluşum hareketini algılayamaz. Bu bilim adamı her zaman yaşamın sırlarına gittikçe yaklaşacaktır, ancak kendisi bu sırlara hiçbir zaman erişemeyecek ve protoplazmayı öldürüp incelemekten başka bir yola sahip olamayacaktır. Ölü protoplazma yaşayan plazma ile aynı ağırlığa sahiptir, ancak onlar birbirine eş değillerdir.
\vs p065 6:2 Yaşayan şeyler ve varlıklar içinde uyumun özgün bir kazanımı mevcut bulunmaktadır. Maddi veya ruhsal oluşundan bağımsız olarak, her \bibemph{yaşayan} organizma biçiminde her \bibemph{yaşayan} bitki veya hayvan hücresi içerisinde; çevreye alışmaya, organizmasal uyuma ve bütünleşerek değişen yaşam koşullarına ait sürekli artan kusursuzluğa erişim için tatmin edilemez bir arzu bulunmaktadır. Tüm yaşayan varlıkların bu bitmek tükenmek bilmeyen çabaları, kusursuzluğa erişim için kendilerinde içkin bir biçimde barınan arzunun mevcudiyetine kanıt teşkil etmektedir.
\vs p065 6:3 Bitki evrimi içerisinde en önemli aşama, klorofil\hyp{}üretim yetkinliğinin gelişimiydi; ve ikincil en büyük gelişme, sporların karmaşık bitki tohumlarına olan evrimiydi. Bitki sporları, çoğalımı gerçekleştiren bir birim olarak oldukça etkindir; ancak onlar, bitki tohumları içerisinde içkin bir biçimde var olan çeşitlilik ve çok yönlülüğün potansiyellerinden yoksunlardır.
\vs p065 6:4 Hayvanların daha yüksek türlerinin evrimi içinde en yararlı ve en karmaşık gelişmelerden biri; dolaşım halindeki kan hücreleri içinde oksijen taşıma ve karbondioksiti arındırma biçiminde çifte görevi yerine getirecek demir yetkinliğinin gelişimiydi. Ve kırmızı kan hücrelerinin bu görevi, evrim halindeki organizmaların farklı ve değişen koşullar karşısında işlevlerini nasıl da uyumlu hale getirmeye yetkin olduklarını göstermektedir. İnsanı içine alan bir biçimde daha üst düzey hayvanlar; oksijeni yaşayan hücrelere taşıyan ve aynı verimlilik içerisinde karbondioksiti bu hücrelerden arındıran, kırmızı kan hücrelerine ait demir faaliyeti vasıtasıyla dokularına oksijen alınımını gerçekleştirirler. Ancak diğer metaller de aynı amaca hizmet etmek için kullanılabilir. Mürekkepbalığı bu faaliyet için bakırı kullanmakta olup, deniz üzümü vanadyumdan faydalanır.
\vs p065 6:5 Bu türden biyolojik uyumun devamı, daha yüksek Urantia memelilerinde dişlerin evrimi tarafından sergilenmiştir; bu canlılar, insanın uzak akrabalarında otuz altı dişe erişen canlılar olup, bunun sonrasında öncül insan ve onun yakın akraba türleri içinde otuz ikiye doğru yeniden bir uyum sürecini gerçekleştirmeye başlamıştır. Bugün insan varlıkları yavaş bir biçimde, yirmi sekiz dişe sahip olmaya eğilimi göstermektedir. Evrimin bu ilerleyişi etkin ve uyumluluk içerisinde bu gezegen içerisinde hali hazırda faaliyet halindedir.
\vs p065 6:6 Ancak yaşayan organizmaların gizemli görünen uyumlarının çoğu, bütünüyle fiziksel bir biçimde katışıksız nitelikte kimyasaldır. Herhangi bir an içerisinde, bir insan varlığının kan dolaşımında; bir düzine iç salgı bezinin hormonsal üretimi arasında 15.000.000 kimyasal tepkimeye varan miktarda olası oluşum gerçekleşmektedir.
\vs p065 6:7 Bitki yaşamının daha alt düzey türleri; fiziksel, kimyasal ve elektriksel çevreye bütünüyle duyarlıdır. Ancak yaşam ölçeğinde kademe yükseldikçe, yedi emir\hyp{}yardımcı ruhaniyetin akıl hizmetkârları birer birey etkin hale gelmektedir; ve akıl artan bir biçimde uyumlu, yaratıcı, eş güdümsel ve baskınlık kuran hale gelmektedir. Hayvanların kendilerini hava, su ve karaya uyumlu hale getirme yetkinliği doğa\hyp{}ötesi bir kazanım değildir; bu uyum, aşkın\hyp{}fiziksel uyumun bir parçasıdır.
\vs p065 6:8 Fizik ve kimya, öncül denizlere ait ilk çağ protoplazmasından bir insan varlığının nasıl evirildiğini tek başına açıklayamaz. Çevreye karşı deneyimsel ve farklılaşan duyarlılık olarak öğrenme yetisi, aklın bir kazanımıdır. Eğitim ve yeterli ölçüde hazırlanma fizik kurallarının üstesinden gelememektedir; zira bu kurallar sabit ve değişmezdir. Kimyanın tepkimeleri, eğitimle değişikliğe uğratılamaz; onlar tek\hyp{}tip ve her zaman doğruluk gösteren oluşumlardır. Koşulsuz Mutlaklık’ın mevcudiyeti dışında elektriksel ve kimyasal tepkiler tahmin edilebilir nitelikte bulunmaktadır. Ancak akıl, gelişen dışsal etkenlerin tekrarlanışına karşı gösterilen tepkisel davranış alışkanlıklardan öğrenmeye yetkin biçimde, deneyimden yarar sağlayabilmektedir.
\vs p065 6:9 Us\hyp{}öncesi organizmalar, çevresel dış etkenlere tepki göstermektedirler; fakat akıl hizmetine karşı duyarlı olan bu organizmalar çevreye uyum sağlayıp, kendisi için çevre koşulları üzerinde değişiklikte bulunmaya yetkindirler.
\vs p065 6:10 Kendisine atanan sinir sistemi ile birlikte fiziksel beyin; tıpkı bir kişiliğin gelişme gösteren aklının ruhaniyet duyarlılığı için belirli bir içkin yetkinliğe sahip olması gibi, akıl hizmeti için içkin yeteneği elinde bulundurmaktadır; ve böylelikle bahse konu bu oluşum, ruhsal gelişim ve erişimin potansiyellerini beraberinde taşımaktadır. Ussal, toplumsal ve ruhsal evrim; yedi emir\hyp{}yardımcı ruhaniyet ve onların aşkın\hyp{}fiziksel birlikteliklerine ait akıl hizmetine bağlıdır.
\usection{7.\bibnobreakspace Evrimsel Akıl Düzeyleri}
\vs p065 7:1 Yedi emir\hyp{}yardımcı akıl\hyp{}ruhaniyeti, yerel bir evrenin sahip olduğu daha alt düzey ussal mevcudiyetler için çok yönlü akıl hizmetkârlıdır. Aklın bu düzeyi, yerel evren yönetim merkezinden veya ona bağlanan bir dünyadan idare edilmektedir; ancak orada, sistem başkentlerinden idare edilen daha alt düzey akıl faaliyetinin etkili yönlendirişi mevcut bulunmaktadır.
\vs p065 7:2 Evrimsel bir dünya, bu yedi emir\hyp{}yardımcının görevine oldukça fazla bir biçimde bağımlıdır. Ancak onlar akıl hizmetkârlarıdır; onlar, Yaşam Taşıyıcılar’ın nüfuz alanı olan fiziksel evrim ile ilgili değillerdir. Yine de, Yaşam Taşıyıcıları’nın kendisini açığa çıkaran ve içkin düzenine ait hükmedilen nitelikteki doğal işleyiş ile bu ruhsal kazanımların kusursuz birleşimi; her ne kadar zaman zaman sizler madde ile birliktelik halinde bulunan aklın doğal tepkileri ile ilişkili her şeyi açıklamada bir şekilde kafa karışıklığına sahip olsanız da, tamamiyle doğanın elinde olan ve doğal süreçlerin dışavurumları biçiminde akıl olguları içinde fani insanın kavrayabilme yetkisizliğine yol açmaktadır. Ve eğer Urantia özgün tasarımlara daha bağımlı bir biçimde ilerlemeye gösterseydi, akıl olgusu içinde sizler daha da az şeyi kavrama süzgecinden geçirecek olurdunuz.
\vs p065 7:3 Yedi emir\hyp{}yardımcı ruhaniyeti, birimsel olmaktan ziyade daha döngüsel bir niteliktedir; ve olağan dünyalar üzerinde onlar, yerel evren boyunca diğer emir\hyp{}yardımcı faaliyetler ile birlikte bağımlı\hyp{}döngüsel bir konuma yerleştirilmişlerdir. Yaşam\hyp{}deneyim gezegenleri üzerinde, buna rağmen, onlar göreceli olarak tecrit edilmiş bir konumda bulunmaktadır. Ve yaşam işleyiş biçimlerinin benzersiz doğası nedeniyle Urantia üzerinde daha alt düzey emir\hyp{}yardımcıları, yaşam kazanımının daha ortak türüne kıyasla evrimsel organizmalar ile iletişim kurmada daha büyük bir zorluğu deneyimlemiştir.
\vs p065 7:4 Bir kez daha altını çizersek: olağan bir evrimsel dünya üzerinde yedi emir\hyp{}yardımcı ruhaniyet, Urantia üzerindeki iletişimlerine kıyasla hayvan gelişiminin ilerleyen düzeyleri ile çok daha iyi bir biçimde uyumlu hale gelmiştir. Tek bir istisna dışında, emir\hyp{}yardımcılar; Nebadon evreni boyunca faaliyetlerinin tümü içinde en büyük zorluğu Urantia organizmaların evrimsel akılları ile olan iletişimlerinde deneyimlediler. Bu dünya üzerinde, kendiliğinden\hyp{}gerçekleşen öğrenmeye\hyp{}müsait\hyp{}olmayan ve kendiliğinden\hyp{}gerçekleşmeyen\hyp{}fakat\hyp{}öğrenebilen organizmasal tepkilerin kafa karıştırıcı bütünlüğü biçiminde, sınır olgularının birçok türü gelişmiştir.
\vs p065 7:5 Yedi emir yardımcı ruhaniyet, organizmasal çevre tepkilerinin bütünüyle kendiliğinden gerçekleşen düzeyleri ile ilişkide bulunmamaktadır. Yaşayan organizmaların bu türden akıl\hyp{}öncesi tepkileri tamamen; güç merkezleri, fiziksel düzenleyiciler ve onların birlikteliklerine ait enerji nüfuz alanları ile ilgilidir.
\vs p065 7:6 Deneyimden öğrenme yetkinliğine ait potansiyelin erişimi, emir\hyp{}yardımcı ruhaniyet faaliyetinin başlangıcını simgelemektedir; ve onlar, insan varlıklarının evrimsel ölçeği içerisinde ilkel ve görünmez olan başlangıçsal mevcudiyetlerin en alt düzey akıllarından en yüksek türlere uzanan bir kapsamda faaliyet gösterir. Onlar, şimdiye kadar bahsi geçen durumlar haricinde, aklın maddi çevreye olan neredeyse gizemli ve tamamen anlaşılmayan çabuk tepkilerinin kaynağı ve işleyiş biçimidir. Bu sadık ve her zaman güvenilebilir etkenler, hayvan aklı ruhaniyet duyarlılığının insan düzeylerini elde etmesinden önce başlangıçsal hizmetlerini uzun süreler boyunca ilerletmelidir.
\vs p065 7:7 Emir\hyp{}yardımcılar, deneyimleyen aklın evirilişinden başlayarak ibadet ruhaniyeti olarak altıncı düzeye kadar ayrıcalıklı bir biçimde faaliyet gösterir. Bu düzeyde, gelişimin ileri düzeylerine gelecekte meydana gelecek erişimin farkındalığı içerisinde yüksek emir\hyp{}yardımcıların daha altta bulunanlar ile eş güdümlü hale gelmek için birleşmesi biçiminde, hizmetin kaçınılmaz bir uyumu ortaya çıkar. Ve buna ek olarak ilave ruhaniyet hizmeti, bilgeliğin ruhaniyeti olarak son ve yedinci emir\hyp{}yardımcısının faaliyetine eşlik eder. Ruhani dünyanın hizmeti boyunca birey hiçbir zaman, ruhaniyet iş birliğinin kesintili geçişlerini deneyimlememektedir; bu değişiklikler her zaman kademeli ve karşılıklı bir biçimde meydana gelir.
\vs p065 7:8 Çevresel etkenlere karşı gösterilen (elektromekaniksel olarak) fiziksel ve akılsal tepkiye ait nüfuz alanları her zaman birbirinden ayrılmalıdır; ve sonuç olarak onların hepsi, ruhsal etkinliklerden ayrı olgular biçimi olarak tanınmalıdır. Fiziksel, akılsal ve ruhsal çekimin nüfuz alanları, her ne kadar birbirine yakın karşılıklı ilişkilere sahip olsalar da, kâinatsal gerçekliğin farklı âlemleridir.
\usection{8.\bibnobreakspace Zaman ve Mekân içerisinde Evrim}
\vs p065 8:1 Zaman ve mekân ayrışmaz bir biçimde birbiriyle ilişkilidir; orada yakın bir birliktelik mevcuttur. Zamansal gecikmeler, belirli mekân koşullarının mevcudiyeti içerisinde kaçınılmazdır.
\vs p065 8:2 Eğer yaşam gelişimine ait evrimsel değişikliklerin gerçekleşmesinde çok fazla harcanan zaman kafa karışıklığına sebep oluyorsa, bir gezegenin müsaade edeceği fiziksel başkalaşımlardan daha hızlı bir biçimde yaşam süreçlerinin hayat buluşunu belirleyemeyeceğimizi söylemeliyim. Bizler, bir gezegenin fiziksel gelişimi biçiminde doğal gelişimi beklemek durumundayız; bizlerin yeryüzü olaylarının evrimi üzerinde kesinlikle hiçbir denetim gücü bulunmamaktadır. Eğer fiziksel koşullar izin vermiş olsaydı, bizler yaşamın tamamlanmış evrimini yarım milyon yıldan çok daha az sürecek bir biçimde halde tasarlardık. Ancak hepimiz, Cennet’in Yüce İdareciler’in karar yetkisi altında bulunmaktayız; ve Cennet üzerinde zaman mevhumu bulunmamaktadır.
\vs p065 8:3 Bireyin zaman ölçümü, kendi yaşamanın uzunluğu kadardır. Yaratılmışların tümü böylece zaman bakımından belirlenmiştir; ve bu nedenle onlar evrimi sonu gelmez bir süreç olarak görmektedirler. Yaşam ömrü geçici bir mevcudiyet ile kısıtlı olmayan bizim gibi varlıklar için evrim, bu gibi gereğinden fazla süren bir geçiş etkileşimi olarak görülmemektedir. Zamanın var olmadığı Cennet üzerinde bu şeylerin tamamı Sınırsızlık’ın aklında ve Ebediyet’in eylemlerinde \bibemph{mevcuttur}.
\vs p065 8:4 Akıl evrimi, fiziksel koşulların gelişimine bağlı olup onlar tarafından gecikmeye uğrarken; benzer bir biçimde ruhsal gelişim akılsal genişlemeye bağlı olup, ussal gerilik tarafından kesin bir biçimde gecikmeye uğramaktadır. Ancak bu durum ruhsal evrimin eğitime, kültüre veya bilgeliğe bağlı olduğu anlamına gelmemektedir. Ruh, akılsal kültürün yokluğunda evirilebilir; fakat bu evrimini --- kurtuluşa erme tercihi ve sürekli artan kusursuzluğa erişme kararı olarak --- cennet içindeki Yaratıcı’nın iradesini yerine getirmeye dair akılsal yetkinlik ve istek yokluğunda gerçekleştiremez. Her ne kadar kurtuluş bilgi ve bilgeliğin iyeliğine bağlı olmasa da, ilerleme oldukça kesin bir biçimde bu iki niteliğe bağlıdır.
\vs p065 8:5 Kâinatsal evrim laboratuarları içerisinde akıl her zaman maddeden üstündür; ve ruhaniyet her zaman akıl ile ilişkilidir. Bu farklı kazanımları uyumlu ve eş güdümsel hale getirmedeki başarısızlık gecikmelere neden olabilir; ancak birey gerçekten Tanrı’yı bilir ve onu bulmaya ek olarak onun gibi olmayı arzularsa, kurtuluşun teminatı zamanın engellerinden bağımsız olarak ona verilir. Fiziksel düzey akıl üzerinde engel yaratabilir; ve akılsal sapkınlık, ruhsal erişimi geciktirebilir; ancak bu zorlukların hiçbiri, iradenin bütün ruhu içine alan tercihin üstesinden gelemez.
\vs p065 8:6 Fiziksel koşullar olgunlaştığında, akılsal evrimler \bibemph{ansızın} gerçekleşebilir; akıl düzeyi zenginleştiğinde, \bibemph{ansızın} gerçekleşen ruhsal değişimler ortaya çıkabilir; ruhsal değerler uygun tanınışı aldığı zaman, bunun sonucunda kâinatsal anlamlar anlaşılır hale gelebilir; ve artan bir biçimde kişilik, zamanın engellerinden bağımsız, ve mekânın kısıtlılıklarından özgürleştirilmiş hale gelir.
\vs p065 8:7 [Bu anlatım, Urantia üzerinde ikamet etmekte olan Nebadon’un bir Yaşam Taşıyıcısı tarafından sağlanmıştır.]
