\upaper{35}{Tanrı’nın Yerel Evren Evlatları}
\vs p035 0:1 Önceki anlatımlarda tanıtılan Tanrı’nın Evlatları, bir Cennet kökenine sahiptir. Onlar, evrensel nüfuz alanlarına ait kutsal Yöneticiler’in doğumlarıdır. Yaratan Evlatlar biçiminde evlatlığın bir Cennet düzeyi içerisinde, Nebadon’da Mikâil olarak sadece bir tek evren yaratıcısı ve egemeni bulunmaktadır. Avonal veya Hakimane Evlatlar biçimindeki Cennet evlatlığının ikinci düzeyi içerisinde, Nebadon kendisine ayrılmış sınırın tamamına 1.062 sayıdaki unsura sahip olarak erişmiş bir halde bulunmaktadır. Buna ek olarak bahse konu bu “daha alt düzeyde bulunan Hazreti unsurları”; Urantia üzerinde Yaratan ve Üstün Evlat’ın gerçekleştirdikleri gibi, gezegensel bahşedilmişlikleri içerisinde tıpkı onlar kadar etkin ve bütünüyle güç sahibidir. Kutsal Üçleme kökeninin varlığı olarak üçüncü düzey, yerel bir evrende kayıtlı değildir; fakat Nebadon içinde, tutulan kayda ait 9.642 yaratılmış biçimindeki kutsal üçleme haline getirilmiş olan varlığın dışında, on beş bin ila yirmi bin arasında değişen bir nüfusa sahip Kutsal Üçleme Eğitmen Evlatları’nın bulunmakta olduğunu tahmin etmekteyim. Bu Cennet Daynalları, ne hâkim ne de idarecilerdir; onlar aşkın eğitmenlerdir.
\vs p035 0:2 İrdelenecek olan Evlatlar’ın türleri, yerel evren kökenine aittir; onlar, tamamlayıcı Evren Ana Ruhaniyeti ile birlikte değişen birliktelikler içerisinde bir Cennet Yaratan Evladı’nın doğumudur. Bu anlatımlarda ifade edilen yerel evren evlatlığının düzeyleri şunlardır:
\vs p035 0:3 1.\bibnobreakspace Melçizedek Evlatları.
\vs p035 0:4 2.\bibnobreakspace Vorondadek Evlatları.
\vs p035 0:5 3.\bibnobreakspace Lanonandek Evlatları.
\vs p035 0:6 4.\bibnobreakspace Yaşam Taşıyıcı Evlatları.
\vs p035 0:7 Cennet Üçleme Bütünlüğü İlahiyat’ı, evlatlığın üç düzeyinin yaratılması için faaliyette bulunmaktadır. Bu düzeyler; Mikâiller, Avonallar ve Daynallar’dır. Evlat ve Ruhaniyet olarak yerel evren içerisinde bulunan Çifte İlahiyat aynı zamanda, Evlatlar’ın üç düzeyinin yaratılmasında faaliyette bulunur. Bu düzeyler; Melçizedekler, Vorondadekler ve Lanonandekler’dir; ve üç katmanlı dışavuruma erişmiş olarak onlar, Yaşam Taşıyıcılar’ın çok yönlü düzeyinin yaratılmasında Yedi Katmanlı Tanrı’nın bir sonraki düzeyi ile işbirliğinde bulunurlar. Bahse konu bu varlıklar, Tanrı’nın alçalan Evlatları ile birlikte sınıflandırılmaktadır; fakat onlar, evren yaşamının benzersiz ve özgün bir biçimidir. Onlar hakkındaki inceleme, bir sonraki makalenin bütününü teşkil edecektir.
\usection{1.\bibnobreakspace Yaratıcı Melçizedek}
\vs p035 1:1 Bu tür Berrak ve Sabah Yıldızı ve diğer idari kişilikler biçimindeki kişisel yardımın varlıklarını mevcut hale getirdikten sonra, ilgili evrene ait kutsal amacın ve yaratıcı tasarıların uyarınca; Sınırsız Ruhaniyet’in yerel evren Kız Evladı olarak Yaratan Evlat ve Yaratıcı Ruhaniyet arasında yaratıcı birlikteliğin yeni bir türü açığa çıkar. Bu yaratıcı birliktelikten ortaya çıkan kişilik doğumu, Yaratıcı Melçizedek biçimindeki özgün Melçizedek’dir. Bu benzersiz varlık bunun sonrasında, bu isim altındaki bütüncül topluluğu mevcut hale getirmek için Yaratan Evlat ve Yaratıcı Ruhaniyet ile birlikte işbirliğinde bulunmaktadır.
\vs p035 1:2 Nebadon evreni içerisinde Yaratıcı Melçizedek, Berrak ve Sabah Yıldızı’nın ilk yönetici birlikteliği olarak hareket eder. Cebrail, daha çok evren siyasaları ile ilgili bir konumda bulunurken; Melçizedekler, işleyişsel yönetmeliklerden sorumludur. Cebrail, Nebadon’a ait düzenli bir biçimde oluşturulan mahkemeler ve kurullar üzerinde hâkimiyete sahipken; Melçizedek özel, olağanüstü ve acil kurullara ek olarak danışma birimleri üzerinde iradeye sahiptir. Cebrail ve Yaratıcı Melçizedek hiçbir zaman aynı anda Salvington’dan ayrı bir konumda bulunmamaktadır; çünkü Cebrail’in yokluğunda Yaratıcı Melçizedek, Nebadon’un baş idareci yöneticisi olarak faaliyet gösterir.
\vs p035 1:3 Bizim evrenimize ait Melçizedekler’in tümü; Yaratıcı Melçizedek ile birliktelik halinde Yaratan Evlat ve Yaratıcı Ruhaniyet tarafından, ortak zamana göre bir bin yıllık süreç içerisinde yaratılmıştır. Nüfusları içerisinde bir unsurun eş güdüm yaratıcısı olarak faaliyet gösterdiği, evlatlığın bir düzeyi olarak Melçizedekler; oluşum bakımından kısmen benliksel kökene sahip olup, bu nedenle öz yönetimin tanrısal bir türünün gerçekleştirilmesine adaylardır. Her ne kadar özgün Melçizedek belirli içkin eş\hyp{}ebeveynsel ayrıcalıklara sahip olmasa da, Melçizedekler; ortak zamana göre yedi yıllık bir dönem için kendilerinin idari baş yöneticisini dönemsel olarak seçmekte olup, bunun dışında kendiliğinden düzen sağlayıcı düzey olarak faaliyet gösterir. Zaman zaman bu Yaratıcı Melçizedek, kendi düzeyine ait belirli bireyleri; Urantia üzerinde henüz açığa çıkarılmamış olan yerleşik gezegenin bir türü biçimindeki midsonit dünyaları için, özel Yaşam Taşıyıcıları olarak faaliyet göstermesi için atamaktadır.
\vs p035 1:4 Melçizedekler; aşkın\hyp{}evrenin mahkemelerinden önce beklemekte olan olaylarda şahitler olarak çağrıldıkları ve özel elçiler olarak aynı aşkın\hyp{}evren içinde bir evreni diğeri için temsil etme göreviyle atandıkları zaman zarfı haricinde, yerel evrenin dışında geniş bir ölçüde faaliyet göstermemektedir. Her evrenin özgün veya ilk doğan Melçizedek’i her zaman, kendi düzeyinin çıkarları ve sorumlulukları ile ilgili olan görevler üzerinde komşu evrenlere veya Cennet’e seyahat etme özgürlüğüne sahiptir.
\usection{2.\bibnobreakspace Melçizedek Evlatları}
\vs p035 2:1 Melçizedekler; ete kemiğe bürünmenin gerekliliğine ihtiyaç duymadan evrimsel ırklara hizmet eden bir biçimde, fani yükselişin hizmeti içinde doğrudan faaliyet göstermeye yetkin hale gelebilmek amacıyla alt düzey yaratım yaşamına yeterli bir biçimde yakınlaşmada kutsal Evlatlar’ın ilk düzeyidir. Bu Evlatlar doğal bir biçimde; irade bahşedilişine ait en yüksek Kutsallık ile en alçak düzeyde bulunan yaratılmış yaşamı arasında yaklaşık olarak ara bir konumda bulunan köken varlığı aracılığıyla büyük kişilik kökeninin orta noktasında bulunmaktadır. Onlar böylelikle; evrimsel dünyalar üzerindeki yaşam türleri olarak maddi yaşamı bile içine alan bir biçimde, yaşam varoluşunun yüksek ve kutsal seviyeleri ile alt düzeyleri arasında doğal aracılar haline gelmektedir. Melekler olarak yüksek melek düzeyleri, Melçizedekler ile beraber görev yapmaktan büyük mutluluk duymaktadır; gerçekte ussal yaşamın bütün türleri, bu Evlatları anlayış sahibi arkadaşlar, duygudaş eğitmenler ve bilge danışmanlar olarak değerlendirmektedir.
\vs p035 2:2 Melçizedekler, öz idareye sahip bir düzeydir. Bu benzersiz topluluk ile birlikte biz; yerel evren varlıkların bir kısmı üzeride özerkliğin öncül girişimiyle karşılaşmakta olup, gerçek öz iradenin en yüksek biçimini gözlemlemekteyiz. Bu Evlatlar; birliktelik içinde bulunan altı âlem ve onların alt dünyalarına ek olarak, topluluklarına ait kendi öz işleyiş biçimini ve bulundukları gezegen idaresini düzenlemektedir. Tüm bunlara ek olarak, onların hiçbir zaman sahip oldukları ayrıcalıkları kötüye kullanmamış olmaları belirtilmelidir; Orvonton’un aşkın\hyp{}evreninin tümü boyunca bahse konu bu Melçizedek Evlatları, kendilerine duyulan güveni bir kere bile olsun boşa çıkarmamışlardır. Onlar, öz iradeye sahip olmayı arzulan her evren topluluğunun umudu olmaktadır; onlar, Nebadon’a ait âlemlerin tümü için öz idarenin yöntemi ve eğitmenleridir. Kendilerinin alt düzeyleri için üstün bir konumda bulunan ve aynı zamanda üst düzeyleri karşısında ast bir mevkie sahip olan ussal varlıkların düzeylerinin tümü, Melçizedekler’in hükümetini takdir etmede oldukça samimidir.
\vs p035 2:3 Evlatlığın Melçizedek düzeyi; geniş bir aile içinde yaşça en büyük olan evladın mevkisini elinde bulundurup, bunun sorumluluğunu almaktadır. Onların görevlerinin büyük bir çoğunluğu, düzenli ve bir ölçüde sürekli tekrar eden etkinliklerden oluşmaktadır; fakat yine bunların büyük bir çoğunluğu gönüllü olup, bütünüyle bireysel olarak belirlenen faaliyetlerdir. Zaman zaman Salvington üzerinde toplanan özel meclislerin büyük bir kısmı, Melçizedekler’in kararı üzerine bir araya gelmesi için çağrılır. Kendi öz kararlarına bağlı olarak bu Evlatlar, özgün evrenleri üzerinde hareket eder. Onlar; âlemin olağan idaresi ile ilgili düzenli bir biçimde faaliyet gösteren birimler boyunca evren yönetim merkezine gelen bilgilerin tümünden bağımsız olarak, Yaratan Evlat’a düzenli bir biçimde bildirimde bulunarak evren usu için adanmış özerk bir idareyi sağlamaktadır. Onlar doğalarından kaynaklanan bir biçimde, ön yargıya sahip olamayan gözlemcilerdir; onlar, ussal varlıkların tüm sınıfları için bütüncül inanca sahiptirler.
\vs p035 2:4 Melçizedekler, âlemlerin hareketli bir biçimde bulunan ve danışma niteliğinde olan inceleme mahkemeleridir; bu evren Evlatları danışma kurulları biçiminde hizmet vermek, mahkemeleri için ifadeleri dinlemek, tavsiye almak ve danışmanlar olarak faaliyet göstermek amacıyla küçük topluluklar halinde bu dünyalara hareket etmektedir; böylelikle onlar, zaman zaman evrimsel nüfuz alanları içerisinde açığa çıkan olaylardaki büyük zorlukları gidermek ve ciddi farklılıkları düzeltmek amacıyla yardımda bulunmaktadır.
\vs p035 2:5 Bir evrenin en kıdemli Evlatları, Yaratan Evlat’ın emirlerini yerine getirmede Berrak ve Sabah Yıldızı’nın baş yardımcılarıdır. Bir Melçizedek, Cebrail adına uzak bir dünyaya hareket ettiği zaman; bahse konu bu özel görevin amaçları için kendisini görevlendiren unsur adına vekâlet edip, bu etkinlik içerisinde Berrak ve Sabah Yıldızı’nın bütüncül yönetim yetkisiyle birlikte görevlendirildiği gezegen üzerinde ortaya çıkar. Özellikle bahse konu bu durum, daha yüksek olan Evlat’ın âlemin yaratılmışların suretinde henüz ortaya çıkmadığı bu âlemler için doğruluk teşkil etmektedir.
\vs p035 2:6 Bir Yaratan Evlat, evrimsel bir dünya üzerinde bahşedilme sürecine giriş yapmasından itibaren bu yükümlülüğü tek başına yerine getirir; fakat bir Avonal Evladı biçiminde onun Cennet kardeşlerinden biri bir bahşedilme sürecine girdiği zaman, bahşedilme görevinin başarısına oldukça etkin bir şekilde katkıda bulunan sayıca on iki unsurdan oluşan Melçizedek destekçileri tarafından eşlik edilir. Onlar aynı zamanda, yerleşik dünyalar için hakimane görevleri üzerinde Cennet Avonalları’na yardım etmektedir; eğer Avonal Evladı belirgin bir görünüme sahip bir konumda bulunuyorsa, bunun sonucunda bu görevleri içinde Melçizedekler aynı zamanda fani göz için görünebilen bir niteliktedir.
\vs p035 2:7 Gezegensel ruhsal ihtiyaca ait onun hizmetini gerektirmeyen hiçbir faz bulunmamaktadır. Onlar; Yaratan Evlat ve onun Cennet Yaratıcısı’nın nihai ve bütüncül olan tanınması sağlamak amacıyla, gelişmiş yaşamın bütüncül dünyalarının desteğini oldukça sık bir biçimde kazanan eğitmenlerdir.
\vs p035 2:8 Melçizedekler, bilgelik bakımından neredeyse kusursuz bir niteliğe sahiptir; fakat kararda bulunma bakımından onlar hatasız değillerdir. Gezegensel görevler üzerinde ayrılmış ve yalnız bir konumda bulunduğu zaman onlar; yüksek denetimcilerinin onayına sunulmayan belirli sorumlulukları yerine getirmek amacıyla seçildikleri, ufak çaplı yükümlülüklerde zaman zaman yanılgıya düşmektedir. Karara varmada bu türden bir hata; bir Melçizedek’in Salvington’a hareket etmesine ve burada Yaratan Evlat’ın yönetimsel sorgusu içinde kendi yoldaşları arasında anlaşmazlığa sebep olan uyuşmazlığı etkin bir biçimde düzeltecek talimatı almasına kadar, onu yetki dışı bırakır; bu durumun sonrasında düzeltici istirahatı takiben üçüncü gün içerisinde, görevine dair hak kendisine tekrar teslim edilir. Fakat Nebadon üzerindeki Melçizedek faaliyeti içerisinde bahse konu bu tür küçük çaplı uyum bozuklukları nadiren ortaya çıkmıştır.
\vs p035 2:9 Bu Evlatlar, sayıları bakımından artan nitelikte bulunan bir düzey değildir; her ne kadar her yerel evren için farklılık gösterse de onların nüfusu sabittir. Nebadon içinde kendilerine ait yönetim merkezi üzerinde kaydedilen Melçizedek nüfusu, on milyondan fazladır.
\usection{3.\bibnobreakspace Melçizedek Dünyaları}
\vs p035 3:1 Melçizedekler, evren yönetim merkezi olarak Salvington yakınlarında kendilerine ait bir dünyada ikamet ederler. Melçizedek adı altındaki bu âlem; her birinin özelleşmiş etkinliklere adanan altı bağlı âlem tarafından çevrildiği yetmiş ana âleme ait Salvington döngüsünün öncül dünyasıdır. Yetmiş ana ve 420 bağlı yerleşkeden oluşan bu muazzam âlemler, sıklıkla Melçizedek Üniversitesi olarak adlandırılır. Nebadon’un takımyıldızlarının tümünden gelen yükseliş fanileri, Salvington üzerinde yerleşik düzeyin elde edilmesinde 490 dünyanın hepsi üzerinden eğitimden geçerler. Fakat yükseliş halinde bulunan unsurların eğitimi, mimari dünyaların Salvington kümelenmesi üzerinde gerçekleşen çok katmanlı etkinlere ait aşamalardan yalnızca bir tanesidir.
\vs p035 3:2 Salvington döngüsüne ait 490 âlem, her biri yedi ana ve kırk iki bağlı âlemi içinde barındıran on topluluğa ayrılmıştır. Bu topluluklardan her biri, kâinat yaşamının ana düzeylerinin herhangi birine ait olan genel yüksek denetimi altında bulunmaktadır. Çevreleyen gezegensel dönüş hareketi içinde öncül dünya ve ona komşuluk eden altı ana âlem ile bütünleşen ilk topluluk, Melçizedekler’in yüksek denetimi altındadır. Bahse konu bu Melçizedek dünyaları şunlardır:
\vs p035 3:3 1.\bibnobreakspace Öncül Dünya --- Melçizedek Evlatları’nın ana dünyası.
\vs p035 3:4 2.\bibnobreakspace Fiziksel\hyp{}yaşam okullarının ve yaşayan enerjilere ait laboratuarların dünyası.
\vs p035 3:5 3.\bibnobreakspace Morontia yaşamının dünyası.
\vs p035 3:6 4.\bibnobreakspace Başlangıçsal ruhani yaşamın âlemi.
\vs p035 3:7 5.\bibnobreakspace Orta\hyp{}ruhani yaşamın dünyası.
\vs p035 3:8 6.\bibnobreakspace Gelişmekte olan ruhani yaşamın âlemi.
\vs p035 3:9 7.\bibnobreakspace Eş güdümsel ve yüce olan benliğin gerçekleştirilmesinin nüfuz alanı.
\vs p035 3:10 Bu Melçizedek âlemlerinin her birine ait altı bağlı dünya, birliktelik içinde bulunan ana âlemin görevine ilişkin etkinliklere adanmıştır.
\vs p035 3:11 \bibemph{Melçizedek} âlemi olan öncül dünya, zaman ve mekânın yükseliş fanilerinin eğitimine ve ruhsallaşmasına katılan varlıkların tümü için ortan buluşma noktasıdır. Yükseliş halindeki bir unsur için bu dünya muhtemelen Nebadon’un tümü içerisinde bulunan en ilginç yerleşkedir. Takımyıldız eğitimlerinden mezun olan evrimsel fanilerin tümü, Salvington eğitim sisteminin davranışsal ve ruhani ilerleme düzenine adım atarlar. Ve siz; Cennet varış istikametinize ulaştıktan sonra bile, bu benzersiz dünya üzerindeki yaşamın ilk gününe karşı göstermiş olduğunuz tepkileri hiçbir zaman unutmayacaksınız.
\vs p035 3:12 Yükseliş fanileri; özelleşmiş eğitimin çevreleyen altı gezegeni üzerinde eğitimlerine devam ederken, Melçizedek dünyası üzerinde yerleşimlerini sürdürür. Ve bahse konu bu yöntem, Salvington döngüsünün ana dünyaları biçimindeki yetmiş kültürel dünya üzerindeki onların kısa süreli ikametleri boyunca aynı biçimiyle kalmaya devam eder.
\vs p035 3:13 Çeşitli birçok etkinlik, Melçizedek âleminin altı bağlı dünyası üzerinde ikamet eden sayısız varlığın zamanını teşkil etmektedir; fakat yükseliş fanileri ile ilgili bu uydular, çalışmanın şu özel fazlarına adanmıştır:
\vs p035 3:14 1.\bibnobreakspace Birinci âlem, yükseliş fanilerinin başlangıçsal gezegen yaşamının incelenmesine ayrılmıştır. Bu görev, fani kökene ait ilgili bir dünyadan gelen bahse konu unsurlardan oluşmuş sınıflar tarafından yerine getirilir. Urantia’dan gelen unsurlar, bu türden bir deneyimsel incelenmesini hep birlikte yerine getirirler.
\vs p035 3:15 2.\bibnobreakspace İkinci âlemin özel görevi, yerel sistem yönetim merkezine ait baş uyduyu çevreleyen malikâne dünyalar boyunca gerçekleşmekte olan deneyimlerin benzer bir incelenişinden oluşmaktadır.
\vs p035 3:16 3.\bibnobreakspace Üçüncü âlem olarak bu âleme ait inceleme; yerel sistemin başkenti üzerindeki kısa süreli ikamet ile ilgili olup, sistem yönetim merkezi kümelenmesine ait mimari dünyaların geride kalan etkinlikleriyle bütünleşir.
\vs p035 3:17 4.\bibnobreakspace Dördüncü âlem, takımyıldızlarının ve onların birliktelik içinde bulundukları âlemlerin yetmiş bağlı dünyasına ait deneyimlerin bir incelenmesine ayrılmıştır.
\vs p035 3:18 5.\bibnobreakspace Beşinci âlem üzerinde, takımyıldız yönetim dünyasında kısa süreli yükseliş ikamesinin incelenmesi gerçekleştirilir.
\vs p035 3:19 6.\bibnobreakspace Altıncı âlem üzerinde zaman; bahse konu beş çağı bağdaştırma amacıyla girişilen bir teşebbüse adanmış olup, evren eğitiminin Melçizedek ana okullarına olan giriş için hazırlık niteliğindeki deneyiminin eş güdümünü böylece elde eder.
\vs p035 3:20 Evren idaresi ve ruhsal bilgeliğin okulları; enerji, madde, işleyişsel düzenleme, iletişim, kayıt, etik ve karşılaştırmalı yaratılmış mevcudiyeti biçimindeki araştırmanın özelleşmiş bir koluna ayrılmış bahse konu okullarının da aynı zamanda bulunduğu, Melçizedek ana dünyası üzerinde konumlandırılmıştır.
\vs p035 3:21 Ruhsal Kazanımın Melçizedek Üniversitesi içinde, Cennet düzeylerini bile içine alan Tanrı’nın Evlatları’nın tüm düzeyleri; evrenin uzak dünyalarına bile ruhsal özgürlüğü ve kutsal evlatlığı duyuran bir biçimde nihai sonun müjdecileri olarak hareket eden ev sahiplerinin eğitilmesinde, Melçizedekler ve yüksek melek eğitmenleri ile iş birliği yapmaktadır. Melçizedek Üniversitesi’nin bahse konu bu özelleşmiş okulu, ayrıcalıklı bir üniversite kurumudur; diğer âlemlerden gelen ziyaretçi öğrenciler buraya kabul edilmemektedir.
\vs p035 3:22 Evren idaresinde eğitimin en yüksek dersi, ana dünyaları üzerinde Melçizedekler tarafından verilmektedir. Yüksek Etik’in bu Üniversitesi, özgün Yaratıcı Melçizedek tarafından idare edilmektedir. Çeşitli evrenlerin değişim öğrencileri gönderdiği eğitim kurumları bu okullardır. Nebadon’un genç evreni, ruhsal erişim ve yüksek etik gelişimi bakımından evrenler ölçeğinde düşük bir konumda bulunsa da; yine de bizim idari sorunlarımız, ziyaretçi öğrencilerin ve diğer âlemlerden katılan gözlemcilerin akın ettiği Melçizedek üniversiteleri biçimindeki, yakın bir konumda bulunan diğer yaratılmışlar için bütün evreni böylece geniş bir tedavi mekânı haline dönüştürmüştür. Kayıt altındaki yerel katılımcıların engin topluluğunun yanı sıra, Melçizedek okullarına katılımda bulunan sayıca yüz binden fazla yabancı öğrenci her zaman mevcut bulunmaktadır; çünkü Nebadon içindeki Melçizedekler’in düzeyi, Splandon’un tümü boyunca tanınmış bir niteliğe sahiptir.
\usection{4.\bibnobreakspace Melçizedekler’in Özel Görevi}
\vs p035 4:1 Melçizedek etkinliklerinin oldukça özelleşmiş bir kolu, yükseliş halindeki fanilerin ilerleyici morontia sürecinin yüksek denetimi ile ilgilidir. Bu eğitimin büyük bir kısmı; evren erişiminin görece yüksek düzeylerine yükselmiş olan faniler tarafından desteklenen, sabırlı ve bilge yüksek melek hizmetkârları vasıtasıyla yerine getirilir; fakat bu eğitimsel görevin bütünü, Kutsal Üçleme Eğitmen Evlatları ile birliktelik halinde bulunan Melçizedekler’in genel yüksek denetimi altındadır.
\vs p035 4:2 Melçizedek düzeyleri; yerel evrenin geniş eğitim sistemine ve deneyimsel hazırlanış düzenine ayrılmış bir niteliğe sahipken, aynı zamanda benzersiz görevler içinde ve olağan dışı koşullar altında faaliyet göstermektedir. Nihai olarak bütünleşen yaklaşık olarak on milyon yerleşik dünya biçimindeki evrimleşen bir evren içinde, olağanın dışında mevcut olan birçok şeyin gerçekleşmesi kaçınılmazdır; ve bu tür acil durumlarda Melçizedekler hareket etmektedir. Sizin takımyıldızınızın yönetim merkezi olarak Edentia üzerinde onlar, acil durum Evlatları olarak bilinir. Onlar her zaman; bir gezegen, bir sistem, bir takımyıldızı veya bir evren fark etmeksizin fiziksel, ussal veya ruhsal biçimdeki acil eylem gerektiren tüm durumlarda hizmet vermeye hazırdır. Her ne zaman ve her nerde olursa olsun özel bir yardıma ihtiyaç duyulduğunda, burada siz bir veya daha fazla Melçizedek Evladı’nı mevcut bir halde bulacaksınız.
\vs p035 4:3 Yaratan Evlat’a ait tasarımın bazı nitelikleri başarısızlığa uğrama olasılığıyla yüzleştiği zaman, bir Melçizedek yardım sağlamak için derhal harekete geçer. Fakat Satania içinde ortaya çıktığı gibi kötülüğün başkaldırısının mevcudiyeti durumunda onlar, faaliyette bulunmak amacıyla sıklıkla çağrılmamaktadır.
\vs p035 4:4 Melçizedekler, idare sahibi yaratılmışların ikamet ettiği dünyaların tümü üzerindeki doğalardan bağımsız bir biçimde acil durumların hepsinde ilk olarak harekete geçen unsurlardır. Onlar zaman zaman; yükümlülüğünü yerine getirmeyen bir gezegensel hükümetin teslim alıcıları biçiminde hizmet ederek, geçici koruyucular olarak düzensiz gezegenler üzerinde faaliyette bulunur. Bir gezegensel buhranda bahse konu bu Melçizedek Evlatları, birçok benzersiz yetkinlik içerisinde hizmet vermektedir. Bu türden bir Evlat’ın kendisini fani varlıklar için görünebilir halde kılması şüphesiz mümkündür, ve hatta zaman zaman bu düzeyin bir unsuru fani beden sureti içinde ete kemiğe bile bürünmüştür. Nebadon içinde bir Melçizedek yedi kez, fani bedenin benzerliği içinde bir evrimsel dünya üzerinde hizmet vermiştir; buna ek olarak sayısız birçok durumda bu Evlatlar, evren yaratılmışlarının diğer düzeylerinin sureti içinde ortaya çıkmışlardır. Onlar gerçekten; evren akli varlıklarının tüm düzeylerine ek olarak, dünyaların ve onlara ait olan sistemlerin hepsi için çok yönlü ve gönüllü acil durum hizmetkârlarıdır.
\vs p035 4:5 İbrahim döneminde Urantia üzerinde yaşamış olan Melçizedek, yerel olarak Salem’in Prensi biçiminde tanınmaktaydı; çünkü o, Salem olarak adlandırılan bir yerleşkede ikamet eden doğruyu arayanların küçük bir eşlenik topluluğu üzerinde yöneticilik yapmaktaydı. O; fani bedenin sureti içinde ete kemiğe bürünmeye gönüllü olmuş, ve artan ruhsal karanlığın dönemi boyunca yaşamın ışığının söneceğinden korku duyan gezegenin Melçizedek teslim alıcılarına ait onayla bu vücuda bürünmeyi gerçekleştirmiştir. Buna ek olarak o; var olduğu zamana ait doğruluğun mevcudiyetini güçlendirmiş, ve güvenli bir biçimde bu doğruyu İbrahim’e ve onun birlikteliklerine aktarmıştır.
\usection{5.\bibnobreakspace Vorondadek Evlatları}
\vs p035 5:1 Kişisel yardımcıların ve çok yönlü Melçizedekler’in ilk topluluğunun yaratımından sonra, Yaratan Evlat ve yerel evren Yaratıcı Ruhaniyeti; Vorondadekler biçimindeki evren evlatlığının ikinci büyük ve çeşitli düzeyini tasarlayıp, onları mevcut hale getirmiştir. Onlar, daha genel olarak Takımyıldız Yaratıcıları olarak bilinirler; çünkü bu düzeyin bir Evladı, her yerel evren içindeki her takımyıldız hükümetinin başında her koşulda aynı niteliğe sahip bir halde bulunmaktadır.
\vs p035 5:2 Vorondadekler’in nüfusu, her yerel evren içinde değişkenlik göstermektedir; sadece Nebadon içinde kayıt altına alınmış bir milyon Vorondadek bulunmaktadır. Melçizedekler olarak onların eş güdüm sağlayıcıları gibi bu Evlatlar, çoğalımsal doğumun hiçbir gücünü ellerine bulundurmamaktadır. Nüfuslarını arttırabilecekleri bilinen hiçbir yöntem bulunmamaktadır.
\vs p035 5:3 Birçok bakımdan bu evlatlar, bir öz iradeye bünyesidir; bireyler ve hatta bir bütün halinde topluluklar olarak onlar, büyük bir ölçüde tıpkı Melçizedekler kadar özerktir; fakat Vorondadekler, etkinliklerin bu türden geniş bir kapsamı dâhilinde faaliyet göstermemektedir. Onlar, harikulade bir nitelikte olan çok yönlülük bakımından Melçizedek kardeşlerine denk değillerdir; fakat onlar, yöneticiler ve ileri görüşlü idareciler olarak daha bile güvenilir ve etkin niteliğe sahiptir. Buna ek olarak onlar, Lanonandek Sistem Egemenleri olarak emirleri altında bulunan unsurlarının tam anlamıyla idari denkleri değildir; yine de onlar, amacın istikrarı ve yargının kutsallığı bakımından evren evlatlığının tüm düzeylerinden üstündür.
\vs p035 5:4 Her ne kadar Evlatlar’ın bu düzeyine ait kararlar ve yönetimler, her zaman kutsal evlatlığın ruhaniyeti uyarınca gerçekleşse ve Yaratan Evlat’ın siyasaları ile uyum içinde bulunsa da; onlar, Yaratan Evlat’ın mevcudiyeti karşısında hatalı bulunan durumlara sahip olmuştur, ve işleyiş biçiminin ayrıntılarında onların kararları zaman zaman evrenin yüksek mahkemelerine yapılan itirazlar sonunda geri alınmıştır. Yine de bu Evlatlar; nadiren hataya düşmekte olup, hiçbir zaman emirlere karşı gelmemişlerdir; Nebadon’un tüm tarihi boyunca bir Vorondadek, evren hükümetinin nefretinde hiçbir zaman yer teşkil etmemiştir.
\vs p035 5:5 Vorondadekler’in yerel evrenler içindeki hizmeti, geniş olup değişkenlik arz eder. Onlar, diğer evrenler için elçiler olarak ve özgün evrenleri içinde takımyıldızlarını temsil eden konsoloslar biçiminde hizmet eder. Yerel evren evlatlığının tüm düzeyleri içinde onlar çok sık bir biçimde, ciddi evren durumları içinde uygulanacak egemen güçlerinin bütüncül temsiliyle görevlendirilmektedir.
\vs p035 5:6 Başkaldırı ve görevini yerine getirmeme nedeniyle gezegensel tecritten mustarip olan bu tür âlemler biçimindeki ruhsal karanlık içinde tecrit edilmiş bahse konu dünyalar üzerinde bir gözlemci Vorondadek çoğunlukla, olağan düzey yeniden sağlanana kadar mevcut bir biçimde bulunmaktadır. Belirli acil durumlar içinde bu En Yüksek gözlemci, bahse konu gezegen için görevlendirilmiş her göksel varlık üzerinde mutlak ve isteğe bağlı olan yetkisini kullanabilir. Vorondadekler’in bu türden gezegenlerin En Yüksek vekilleri olarak bahse konu yetkiyi zaman zaman uygulandıkları Salvington üzerinde tutulmuş kayda aittir. Buna ek olarak, başkaldırının rastlanmadığı yerleşik dünyalar için bile bu durum aynı zamanda doğruluk arz eder.
\vs p035 5:7 On iki veya daha fazla Vorondadek Evladı’nın bir birliği sıklıkla; incelemenin bir yüksek mahkemesi olarak jüri koltuğunda oturmakta, ve bir gezegenin veya sistemin durumu ile ilgili olan özel davalar hakkında itirazda bulunmaktadır. Fakat onların görevi daha geniş bir biçimde, takımyıldız hükümetlerine özgü olan yasama faaliyetleri ile ilgilidir. Tüm bu hizmetlerin bir sonucu olarak Vorondadek Evlatları, yerel evrenlerin tarihçileri haline gelmişlerdir; onlar kişisel olarak, yerleşik dünyaların siyasi mücadelelerinin ve toplumsal karmaşalarının bütünü ile aşinadır.
\usection{6.\bibnobreakspace Takımyıldız Yaratıcıları}
\vs p035 6:1 En az üç Vorondadek, yerel bir evrenin yüz takımyıldızının her birine ait olan yöneticiliğe atanır. Bu Evlatlar; Yaratan Evlat tarafından seçilip, Urantia zamanına göre 50.000 yıla karşılık gelen 10.000 ortak zaman yılı biçimindeki dekamilenyum boyunca hizmet vermek için takımyıldızlarının \bibemph{En Yüksek Unsurları} olarak Cebrail tarafından görevlendirilir. Takımyıldız yaratıcısı biçimindeki egemen olan En Yüksek, bir kıdemli ve bir kıdemsiz unsurdan oluşan iki birlikteliğe sahiptir. Salvington dünyaları üzerinde ikamet eden görevlendirilmemiş Vorondadekler, kendi unsurlarından birini kıdemsiz birlikteliğin sorumluluklarını üstlenmesi için aday olarak gösterirken; idarenin her değişiminde kıdemli olan unsur hükümetin başı haline gelmekte, kıdemsiz olan unsur ise kıdemli olan unsurun bırakmış olduğu sorumlulukları üstlenmektedir. Mevcut olan siyasa uyarınca En Yüksek yöneticilerin her biri böylece, Urantia zamanına göre yaklaşık olarak 150.000 yıl süren üç dekamilenyum boyunca bir takımyıldızın yönetim merkezi üzerinde bir hizmet dönemine sahiptir.
\vs p035 6:2 Takımyıldız hükümetlerinin mevcut olarak yönetimde bulunan baş unsurları biçimindeki yüz Takımyıldız Yaratıcısı, Yaratan Evlat’ın yüce danışma kabinesini oluşturmaktadır. Bu kurul; evren yönetim merkezinde sürekli bir biçimde toplantı halinde olup, karara alma süreçlerinin kapsamı ve çeşitliliği bakımından kısıtlandırılmamıştır; fakat bu kurul başlıca olarak, takımyıldızlarının refahı ve yerel evrenin tümüne ait idarenin bütüncül hale gelmesi ile ilgilenmektedir.
\vs p035 6:3 Bir Takımyıldız Yaratıcısı; sıklıkla gerçekleştirdiği biçimiyle evren yönetim merkezi üzerindeki sorumluluklarına katıldığı zaman, birliktelik içinde bulunduğu kıdemli unsur takımyıldız olaylarının faal yöneticisi haline gelir. Kıdemsiz birliktelik kişisel olarak takımyıldızının fiziksel refahından sorumlu iken, kıdemli birlikteliğin olağan faaliyeti ruhsal olayların gözetimidir. Buna rağmen geniş kapsamlı hiçbir siyasa bir takımyıldızında, En Yüksek Üç Unsur’un uygulamanın detayları üzerine anlaşmaya varmamaları durumunda hiçbir zaman uygulanmamaktadır.
\vs p035 6:4 Ruhani usa ve iletişim kanalarına ait işleyiş biçiminin bütünü, takımyıldız En Yüksekleri’nin yetkisi altındadır. Onlar; Salvington üzerindeki üstlerine ek olarak, yerel sistemlerin egemenleri biçimindeki doğrudan emirleri altında bulunan unsurlar ile kusursuz iletişime sahiptir. Onlar sıklıkla, takımyıldızının durumu üzerine karara varmak için bahse konu bu Sistem Egemenleri ile bir kurul altında toplanır.
\vs p035 6:5 En Yüksekler; takımyıldız yönetim merkezindeki çeşitli toplulukların mevcudiyetine ve yerel şartların değişmesine bağlı olarak, zaman zaman nüfus ve çalışan görevli bakımından farklılık gösteren danışmanların bir birliğiyle kendisini çevrelemiştir. Kargaşanın süreci içerisinde onlar; idari göreve desek sağlaması için Vorondadekler’in ilave Evlatları’nı çağırabilir, ve bu durumda onları hızlı bir biçimde teslim alır. Sizin takımyıldızınız olan Norlatiadek, mevcut an içerisinde on iki Vorondadek Evladı tarafından idare edilmektedir.
\usection{7.\bibnobreakspace Vorondadek Dünyaları}
\vs p035 7:1 Salvington’u çevreleyen yetmiş ana âlemine ait döngü içerisinde yedi dünyanın ikinci topluluğu, Vorondadek gezegenlerini bir araya getirir. Bu âlemlerin her biri onu çevreleyen altı uydu ile birlikte, Vorondadek etkinliklerinin özel bir fazına adanmıştır. Bu kırk dokuz âlem üzerinde yükseliş fanileri, evren yasaması ile ilgili eğitimlerinin doruk noktasını tamamlar.
\vs p035 7:2 Yükseliş fanileri, takımyıldızlarına ait yönetim merkezi dünyaları üzerinde faaliyet gösteren bir biçimde bulunan yasama meclislerini gözlemlemişlerdir; fakat burada Vorondadek dünyaları içinde onlar, kıdemli Vorondadekler’in vesayeti altında yerel evrenin genel mevcut yasamasının uygulamasına katılmaktadır. Bu tür uygulamalar, yüz takımyıldızına ait özerk yasama meclislerinin değişen karar beyanlarını eş güdümsel hale getirmek için tasarlanmıştır. Vorondadek okulları içinde bilgilendirilecek olan yönerge, Uversa üzerinde bile eşine rastlanılmayan bir niteliğe sahiptir. Bu eğitim; ilk âlemden başlayarak onun altı uydusunun tamamlayıcı görevi ile birlikte, geride kalan altı âleme ve onun birliktelik halindeki uydu topluluklarına kadar uzanan bir biçimde ilerleyici özellik taşımaktadır.
\vs p035 7:3 Yükseliş halinde bulunan kutsal yolcular, eğitimin ve işlevsel görevin bu dünyaları üzerinde sayısız yeni etkinlik ile tanıştırılacaktır. Bizim, bahse konu bu yeni ve hayal edilmemiş uğraşın açığa çıkarılışına girişmemiz engellenmemiştir; fakat fani varlıkların maddi aklı için bu girişimleri temsil etmeye dair yetkiliğimizden kuşku duymaktayız. Bu tanrısal etkinliklerin anlamını taşımada yeterli kelime bulamaktayız, ve yükseliş fanileri bu kırk dokuz dünya üzerinde çalışmalarına devam ederlerken onların bu yeni görevlerinin temsilleri olarak kullanılabilecek hiçbir şey benzer insan yükümlülüğü bulunmamaktadır. Buna ek olarak yükseliş düzeninin bir parçası olmayan birçok diğer etkinlik, Salvington döngüsüne ait bu Vorondadek dünyaları üzerinde merkezi bir biçimde konumlanmıştır.
\usection{8.\bibnobreakspace Lanonandek Evlatları}
\vs p035 8:1 Vorondadekler’in yaratımından sonra Yaratan Evlat ve Evren Ana Ruhaniyeti, Lanonandekler biçimindeki evren evlatlığının üçüncü düzeyini mevcut hale getirme amacıyla bir araya gelir. Her ne kadar sistem idaresi ile ilişkili birçok görevle yükümlü hale getirilmiş olsa da onlar en yaygın şekliyle, yerel sistemlerin idarecileri olarak Sistem Egemenleri ve yerleşik dünyaların idari başları biçimindeki Gezegensel Prensler bilinir.
\vs p035 8:2 Kutsallığın seviyeleri bakımından evlatlığın yaratımlarına ait daha sonra ve daha alt düzey olarak bu varlıkların; Melçizedek dünyaları üzerindeki eğitimin belirli derslerinden, takip eden hizmete hazırlık amacıyla geçmeleri gerekmekteydi. Onlar; Melçizedek Üniversitesi’nin ilk öğrencileri olup, yetkinlik, kişilik ve kazanımlarına göre Melçizedek eğitmenleri ve sınayıcıları tarafından sınıflandırılmış ve onaylanmışlardır.
\vs p035 8:3 Nebadon evreni mevcudiyetine, tam olarak on iki milyon Lanonandek unsuru ile başlamıştır; ve onlar Melçizedek âlemini geçtikleri zaman, nihai sınayış içinde üç sınıfa ayrılmıştır:
\vs p035 8:4 1.\bibnobreakspace \bibemph{Birinci Derece Lanonandekler}. En yüksek düzeye ait olarak onların nüfusu 709.841’di. Bu unsurlar, Sistem Egemenleri’ne ek olarak takımyıldızlarının yüce kurullarına olan yardımcılar ve evrenin yüksek idari görevi içindeki danışmanlar olarak tasarlanan Evlatlar’dır.
\vs p035 8:5 2.\bibnobreakspace \bibemph{İkinci Derece Lanonandekler}. Melçizedeklerden türeyen düzey onların nüfusu 10.234.601’di. Onlar, Gezegensel Prensler olarak ve bahse konu düzeyin yedek birliklerine atanmışlardır.
\vs p035 8:6 3.\bibnobreakspace \bibemph{Üçüncü Derece Lanonandekler}. Bu topluluk 1.055.558 unsuru içinde barındırmaktadır. Bu Evlatlar; emir altında bulunan yardımcılar, ileticiler, koruyucular, delegeler ve gözlemciler olarak faaliyette bulunup, bir sitemin ve onun bağlı dünyalarının çeşitli görevlerini yerine getirir.
\vs p035 8:7 Evrimsel varlıkların gerçekleştirebildiklerinin aksine bu Evlatlar’ın bir topluluktan diğerine ilerlemesi mümkün değildir. Sınandıkları ve sınıflandırıldıkları an olan Melçizedek eğitimine bağlı olduklarında onlar, görevlendirildikleri sıralama içinde sürekli bir biçimde hizmet verirler. Benzer bir biçimde bu Evlatlar da üreme biçimde doğum faaliyetine katılmazlar; onların nüfusu evren içinde sabittir.
\vs p035 8:8 Evlatlar’ın Lanonandek düzeyi, Salvington üzerinde yuvarlak rakamlar halinde şu biçimde sınıflandırılmıştır:
\vs p035 8:9 Evren Eş Güdüm Sağlayıcıları ve Takımyıldız Danışmanları.
\vs p035 8:10 Sistem Egemenleri ve Yardımcıları.……………………\bibdf 600.000
\vs p035 8:11 Gezegensel Prensler ve Yedek Birlikler.……………… 10.000.000
\vs p035 8:12 İletici Birlikler…………………………………………………… 400.000
\vs p035 8:13 Koruyucular ve Kaydediciler….………………………………. 100.000
\vs p035 8:14 Yedek Birlikler………………….………………………………. 800.000
\vs p035 8:15 Lanonandekler’in Melçizedekler ve Vorondadekler’den bir ölçüde daha alt bir düzey olmasından dolayı, onlar evrenin bağımlı alt birimleri içinde daha bile büyük bir hizmetin parçasıdır; çünkü onlar, ussal ırkların daha alt düzeyde bulunan yaratılmışlarına daha fazla yakınlaşma yetisine sahiptir. Aynı zamanda onlar; evren hükümetinin kabul edilen işleyiş biçiminden uzaklaşma biçiminde, yanlış yola sapma bakımından daha büyük bir tehlike altında bulunmaktadır. Yine de özellikle birinci derece düzeye ait olan bahse konu Lanonandekler, tüm yerel evren idarecileri arasında en yetkin ve en çok yönlü olan unsurlardır. Yönetim yetisinde bakımından sadece Cebrail ve onun açığa çıkarılmamış birliktelikleri, bahse konu bu unsurlardan üstün bir konumda bulunmaktadır.
\usection{9.\bibnobreakspace Lanonandek Yöneticileri}
\vs p035 9:1 Lanonandekler; gezegenlerin devamlı yöneticileri olup, sistemlerin dönüşümlü egemenleridir. Bu türden bir Evlat mevcut an içerisinde, yerleşik dünyalar içindeki sizin yerel sisteminize ait yönetim merkezi olan Jerusem üzerinde yönetimde bulunmaktadır.
\vs p035 9:2 Sistem Egemenleri, yerleşik dünyalara ait her sistemin yönetim merkezi üzerinde ikişerli veya üçerli heyetler halinde idarede bulunmaktadır. Takımyıldız Yaratıcısı, her dekamilenyumun baş yöneticisi olarak bahse konu bu Lanonandekler’den birini tayin etmektedir. Zaman zaman bahse konu üçlemenin başı içinde hiçbir değişiklik yapılmamaktadır; bu durum bütünüyle takımyıldız yöneticilerinin tercihine bağlıdır. Sistem hükümetleri, bir takım felaket ortaya çıkmadığı takdirde görevlileri bakımından ani değişikliğe uğramamaktadır.
\vs p035 9:3 Sistem Egemenleri veya yardımcıları geri çağrıldıklarında, onların boşalan mevkileri; Edentia üzerinde belirtilmiş olağan topluluktan daha büyük sayıda bulunan bu düzeyin yedek birliklerinden, takımyıldızı yönetim merkezi üzerinde konumlanan yüce kurul tarafından yapılan seçimler aracılığıyla doldurulur.
\vs p035 9:4 Yüce Lanonandek kurulları, birçok farklı takımyıldız yönetim merkezi üzerinde konumlanmıştır. Bu türden bir bünye, Takımyıldız Yaratıcısı’nın kıdemli En Yüksek birlikteliği tarafından yönetilirken; kıdemsiz birliktelik, ikinci derece düzeye ait yedek birliklerin yüksek denetimini yapmaktadır.
\vs p035 9:5 Sistem Egemenleri, isimlerinin çağrıştırdığı anlamı karşılayan bir niteliğe sahiptir; onlar, yerleşik dünyalara ait yerel olaylara neredeyse bütünüyle egemendir. Onlar; Gezegensel Prensler’i, Maddi Evlatlar’ı ve hizmetkâr ruhaniyetleri yönlendirmeleri bakımından adeta ebeveynsel bir niteliğe sahiptir. Egemene ait olan kişisel kavrayış neredeyse tamamlanmış bir haldedir. Bahse konu bu yöneticiler, merkezi evrenden olan Kutsal Üçleme gözlemcileri tarafından yüksek denetime tabi değildir. Onlar, yerel evrene ait yönetim kolu olup; yasama emirlerinin uygulama sorumluları ve yargısal kararların yerine getirilmesinin idarecileri olarak onlar, evren idaresinin tümü içinde Mikâil Evladı’nın yönetimine olan kişisel sadakatsizliğin en kolay ve en hazır bir biçimde kendisini savunmak için konumlandıracağı ve gücünü kanıtlamak amacıyla fırsat kollayacağı yer olan bir konumu temsil etmektedir.
\vs p035 9:6 Bizim yerel evrenimiz, evren hükümetine karşı başkaldırıda bulunmuş Lanonandek düzeyine ait yedi yüz Evlat bakımından talihsiz bir konumda bulunmuştur; bahse konu durum bu nedenle, birkaç sistem içinde ve sayısız gezegen üzerinde kafa karışıklığına sebebiyet vermektedir. Sayılı olan başarısızlığın tümü içinde sadece üçü Sistem Egemenleri unsurları ile ilgili olmuştur; gerçekte bahse konu bu Evlatlar’ın tümü, Gezegensel Prensler ve üçüncü derece Lanonandekler olarak ikinci ve üçüncü düzeye ait bir halde bulunmuştur.
\vs p035 9:7 Doğruluktan ayrılan bahse konu bu Evlatlar’ın geniş nüfusu, yaratımdan kaynaklanan hiçbir hatanın varlığını işaret etmemektedir. Onlar, kutsal bir biçimde kusursuz olarak yaratılabilirlerdi; fakat onlar, zaman ve mekânın dünyaları üzerinde ikamet eden evrimsel yaratılmışları daha iyi anlamaları ve onlara yakınlaşmaları için bu biçimde yaratılmışlardır.
\vs p035 9:8 Orvonton içinde yerel evrenler arasında Henselon’un haricinde bizim evrenimiz, Evlatlar’ın bu düzeyinin en geniş nüfusunu yitirmiştir. Uversa üzerindeki fikir birliğine göre; Nebadon içinde bu kadar idari soruna sahip olmamızın sebebi, bize ait Lanonandek Evlatları’nın tercihte ve tasarımda kişisel özgürlüğün bu türden geniş bir kapsamıyla yaratılmış olmasıdır. Ben bahse konu bu gözlemi, eleştirme biçiminde gerçekleştirmemekteyim. Evrenimizin Yaratan’ı, bu durumu gerçekleştirmekte bütüncül yönetime ve güce sahiptir. Bizim yüksek yöneticilerimizin iddiasına göre; bahse konu evrenin öncül çağları içinde bu tür özgür tercih Evlatlar’ı aşırı bir biçimde sorun yaratırken, olaylar ve durumlar bütüncül olarak incelendiğinde ve nihai olarak düzene kavuşturulduğunda, bahse konu bu tamamiyle sınanmış Evlatlar’ın yüksek sadakatinin ve bütüncül özgür irade hizmetinin kazanımları öncül zamanların kafa karışıklığını ve sıkıntılarını telafi etmeden çok daha fazlasını gerçekleştirecektir.
\vs p035 9:9 Bir sistem yönetim merkezi üzerindeki başkaldırı olayında, çoğunlukla yeni bir egemen göreceli olan kısa bir süre zarfında atanmaktadır; fakat böyle bir durum, bireysel gezegenler üzerinde bu şekliyle gerçekleşmemektedir. Onlar, maddi yaratımın tamamlayıcı birimleridir; buna ek olarak yaratılmış özgür iradesi, bu türden sorunların tümüne ait nihai yargı kararı içinde bir etkendir. Varis Gezegensel Prensleri, yanlış yola sapmış yetkinin prenslerinin ait olduğu gezegenler biçimindeki tecrit edilmiş dünyalar için atanır; fakat onlar, ayaklanmanın sonuçlarının Melçizedekler ve diğer hizmetkâr kişilikleri tarafından uygulanan çözüme yönelik önlemleri tarafından kısmi bir biçimde üstesinden gelinmesine ve ortadan kaldırılmasına kadar bu türden dünyaların etkin idareciliğini üstlenmemektedir. Bir Gezegensel Prens’in başkaldırısı, kendisine ait gezegenin derhal tecrit altına alınmasına sebebiyet vermektedir; buna ek olarak yerel ruhsal döngüleri eş zamanlı olarak diğerlerinden ayrılmaktadır. Sadece bahşedilmiş bir Evlat, bu türden ruhsal bir biçimde tecrit edilmiş dünya üzerinde iletişimin gezegenler arası hatlarını yeniden oluşturmaya yetkindir.
\vs p035 9:10 Bahse konu dik başlı ve bilgeliğe sahip olmayan bu Evlatlar için bir tasarım bulunmakta olup, onların bir çoğu bu merhametli hükümden olumsuz bir biçimde çıkar sağlamaktadır; fakat onlar, yükümlülüklerini yerine getirmedikleri bu mevkiler içinde artık hiçbir biçimde yeniden faaliyet gösteremeyebilirler. İyileşme sürecinin sonunda onlar, emanet görevlerine ve fiziksel idarenin dairelerine atanmaktadır.
\usection{10.\bibnobreakspace Lanonandek Dünyaları}
\vs p035 10:1 Yetmiş gezegenin Salvington döngüsü içinde yedi dünyanın bu üç topluluğu, kendilerine ait ilgili kırk iki uydu ile birlikte, idari âlemlerin Lanonandek kümelenişini bir araya getirmektedir. Bu âlemler üzerinde eski\hyp{}Sistem Egemen birliklerine ait olan deneyim kazanmış Lanonandekler, yükseliş halindeki kutsal yolcuların idari öğretmenleri ve yüksek meleksel ev sahipleri olarak resmi bir biçimde görev vermektedir. Evrimsel faniler, sistem başkentleri üzerinde görev halindeki sistem idarecilerini gözlemlemektedir; ancak burada onlar, on bin yerel sistemin idari beyanlarının mevcut eş güdümüne katılmaktadır.
\vs p035 10:2 Yerel evrenin bu idari okulları, Sistem Egemenleri ve takımyıldız danışmanları olarak geniş bir deneyime sahip olan Lanonandek Evlatları’nın bir birliğinin yüksek denetimi altındadır. Bu yönetim üniversitelerine kıyasla sadece Ensa’nın idari okulları daha üstün bir konumda bulunmaktadır.
\vs p035 10:3 Lanonandek dünyaları; bir taraftan yükseliş fanileri için eğitim âlemleri olarak hizmet ederken, diğer bir taraftan evrenin olağan ve alışılagelmiş faaliyetleri ile ilgili olan geniş yükümlülüklerin merkezidir. Cennet’e olan doğrultunun bütünü içinde yükseliş halindeki kutsal yolcuları; eğitildikleri hususları gerçek anlamıyla yerine getirmedeki mevcut eğitim olarak, uygulamalı bilginin işlevsel okulları içinde çalışmalarına devam ederler. Melçizedekler tarafından sağlanan evren eğitim sistemi; işlevsel, ilerleyici, anlamlı ve deneyimseldir. Bahse konu bu eğitim sistemi; maddi, ussal, morontial ve ruhsal olan hususlardaki eğitim ile bütünleşir.
\vs p035 10:4 Lanonandekler’in bahse konu bu idari âlemleri ile bağlantılı bir biçimde, bu düzeye ait kurtarılmış Evlatlar’ın büyük bir çoğunluğu; gezegensel olaylarının koruyucuları ve idarecileri olarak hizmet vermektedir. Ve önerilen iyileştirme sürecini kabul etmeyi tercih etmiş olan görevlerini yerine getirmemiş bu Gezegensel Prensler ve başkaldırıda bulunmuş onların birliktelikleri; en azından Nebadon evreninin ışık ve yaşam içinde düzene tekrar kavuşturulmasına kadar bahse konu bu alışılagelmiş yetkinlikler içinde hizmet vermeye devam etmektedir.
\vs p035 10:5 Buna rağmen, daha eski sistemlerde bulunan Lanonandek Evlatları’nın birçoğu; hizmet, idare ve ruhsal erişimin muazzam başarılarına imza atmışlardır. Her ne kadar kişisel özgürlüğün yanılsaması ve özerkliğin yanlış inancı boyunca hataya düşme eğilimine sahip olsalar da, onlar soylu, inançlı ve sadık bir topluluktur.
\vs p035 10:6 [Bu anlatım, Salvingtonlu Cebrail’in yetkilendirmesi vasıtasıyla hareket eden Baş Melekler’in Baş İdareci’si tarafından sağlanmıştır.]
