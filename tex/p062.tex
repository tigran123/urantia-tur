\upaper{62}{İlk İnsanın Doğuş Irkları}
\vs p062 0:1 Yaklaşik bir milyon yıl önce, karın\hyp{}bağı memelilerine ait lemur türünün öncül nesil kolundan kaynağını alan bir biçimde birbirini takip eden ve ansızın gerçekleşen üç başkalaşım sonrasında insan türünün birincil ataları ilk kez ortaya çıkmışlardır. Bu öncül lemurların baskın nitelikleri, evrimleşen yaşam plazmasına ait batı veya diğer bir değişle sonraki Amerikan topluluğundan elde edilmiştir. Ancak insan atalarının birincil kökeninin oluşturulmasından önce bu ırk kolu, Afrika içerisinde evrimleşen merkezi yaşam aktarımının katkılarıyla güçlendirilmiştir. Doğu yaşam topluluğu, insan türlerinin mevcut üretimine ya çok az bir yardımda bulunmuş veya hiçbir katkı sağlamamıştır.
\usection{1.\bibnobreakspace Öncül Lemur Türleri}
\vs p062 1:1 İnsan türlerinin atalarıyla ilgili olan öncül lemurlar; mevcut ana kadar kökenleri varlığını sürdürmüş, Avrasya ve kuzey Afrika’da bahse konu zaman zarfında yaşamakta olan asya\hyp{}şebekleri ve maymunlara ait önceden var olan kabileler ile doğrudan bağlantılı bir konumda bulunmamaktaydılar. Buna ek olarak onlar; her ne kadar bu iki türün atası olan fakat bahse konu zaman zarfından bu yana nesli tükenmiş bir türden kökenini alsa da, lemurun çağdaş türünün doğumu değillerdi.
\vs p062 1:2 Bu öncül lemurlar her ne kadar Batı Yarımküre içerisinde evrimleşmişse de; insan türünün doğrudan memeli soyunun oluşumu, merkezi yaşam aktarımının özgün yerleşkesi içinde fakat buranın doğu sınırları üzerinde, güneybatı Asya’da gerçekleşmiştir. Birkaç milyon yıl önce Kuzey Amerika lemur türleri; Bering kara köprüsü üzerinden batıya doğru göç edip, istikametlerini yavaşça gerçekleşen bir biçimde Asya sahili boyunca güneybatı doğrultusuna yöneltmişlerdir. Bu göç eden kabileler nihai olarak, bahse konu zaman zarfında Akdeniz’in genişleyen suları ve Hint yarımadasının yükselen dağ bölgeleri arasında kalan sağlıklı bölgeye ulaşmışlardır. Bu kara bölgelerinden Hindistan’ın batısına kadar onlar; diğer ve elverişli nesil kolları ile bütünleşip, böylelikle insan ırkının soyunu oluşturmuşlardır.
\vs p062 1:3 Zamanla, Hindistan’ın güneybatısında bulunan dağlara ait sahil şeridi kademeli olarak, bu bölgeyi tamamen yalıtan bir biçimde, sular altında kalmıştır. Bu Mezopotamya yerleşkesi veya diğer bir değişle Fars yarımadasına kuzey doğrultusu yönünden erişilebilecek hiçbir bağlantı, veya buradan dışarıya gerçekleştirilebilecek hiçbir bir kaçış noktası bulunmamıştır; bu kuzey bağlantı noktası ise buzların güney istilası tarafından sürekli bir biçimde kesilmiştir. Ve burası; bahse konu zaman zarfında neredeyse cennetsel bir yerleşke halinde bulunmakta olup, burası içerisinde memelilerin bu lemur türünün baskın atalarından, çağdaş zamanın maymun kabileleri ve mevcut insan türleri biçiminde, iki büyük topluluk türemiştir.
\usection{2.\bibnobreakspace İlkel Memeliler}
\vs p062 2:1 Bir milyon yıldan biraz daha fazla bir zaman sürecinden önce, karın\hyp{}bağı memelilerin Kuzey Amerika lemur türünün doğrudan ataları olan Mezopotamyalı ilkel memeliler, \bibemph{ansızın} ortaya çıkmışlardır. Onlar, neredeyse üç fit uzunluğunda bulunan hareketli küçük yaratılmışlardı; ve onlar alışkanlıkla arka ayakları üzerinde yürümezken, dik bir konumda kolaylıkla durabilmekteydiler. Onlar, maymunsu bir biçimde kıllı, çevik ve konuşkandılar; ancak onlar, maymun kabilelerinin aksine etle beslenmekteydiler. Onlar, oldukça yararlı bir niteliğe sahip kavrayıcı ayak parmağına ek olarak kıvrılıp dönebilen ilkel bir baş olanağına sahiptiler. Bu noktadan itibaren insan\hyp{}öncesi türler bahse konu ilkel başparmağı geliştirirken, büyük ayak parmağının kavrayıcı gücünü kademeli bir biçimde yitirmişlerdir. Daha sonraki maymun kabileleri, kavrayıcı ayak parmağını ellerinde bulundurmuşlardır; ancak onlar, insan türünün sahip olduğu baş varmağını hiçbir zaman geliştirmemişlerdir.
\vs p062 2:2 Bu ilkel memeliler; yaklaşık olarak yirmi yıllık olası bir yaşam ömrüne sahip bir biçimde, üç veya dört yaşında bütüncül gelişimlerini kazanmaktaydılar. Bir kural olarak, her ne kadar nadiren ikizler dünyaya gelse de, doğumlar tekil olarak gerçekleşmekteydi.
\vs p062 2:3 Bu yeni türlerin üyeleri, bahse konu zaman zarfına kadar dünya üzerinde mevcut bulunan herhangi bir hayvanın sahip olduğu en büyük beyne sahipti. Onlar; oldukça meraklı olan ve girişimlerinde başarılı olduklarında dikkate değer sevinci gösteren bir biçimde, daha sonra ilkel insanı belirleyecek birçok duyguyu deneyimlemiş ve sayısız içgüdüyü paylaşmıştır. Yiyecek kıtlığı ve cinsel arzu oldukça iyi bir biçimde gelişme göstermiş, buna ek olarak kur yapmanın ve eş bulmanın ilkel bir türü içinde belirli bir cinsiyet tercihi sergilenmiştir. Onlar; kendi türlerinin korunması için oldukça çetince savaşabilecek varlıklar olup, utanç ve pişmanlığa yakın alçak gönüllülüğün bir duygusunu taşıyan bir biçimde, aile ilişkileri içinde fazlasıyla şefkatlilerdi. Onlar, eşlerine karşı oldukça sevgi besleyip ve etkileyici bir biçimde sadıklardı; ancak mevcut koşullar onları sevdiklerinden ayırınca, yeni eşlerini seçmekteydiler.
\vs p062 2:4 Kısa bedenlere ve orman yaşam alanları içinde mevcut tehlikelerin farkında olabilecek keskin akıllara sahip olan bir biçimde onlar; kara yaşamının birçok tehlikesini saf dışı bırakan ağaç tepelerinde ilkel barınakları inşa etmek gibi, varoluşlarına oldukça büyük bir biçimde katkı sağlayan akıl dolu tedbir önlemlerini almaya onları iten olağandışı bir korku duyusunu geliştirmişlerdir. İnsan türünün korku eğilimlerinin başlangıcı kökenini daha çok bu zamanlardan almaktadır.
\vs p062 2:5 Bu ilkel memeliler, en başından beri daha önce sergilenenden çok daha güçlü bir kabile ruhunu geliştirmişlerdir. Onlar gerçekten oldukça toplumsal varlıklardı, ancak buna rağmen herhangi bir biçimde günlük yaşamlarının alışılagelmiş faaliyetlerinden alıkonulduklarında aşırı bir biçimde kavgacı olmaktaydılar; buna ek olarak onlar, bütünüyle sinirli olduklarında durdurulamaz bir öfke selini sergilemişlerdir. Onların kavgacı doğaları, buna rağmen, yarlı bir amaca hizmet etmiştir; üstün topluluklar alt düzeyde bulunan komşuları ile mücadele etmekten çekinmemişlerdir; ve böylelikle, seçici evrim vasıtasıyla, bu türler ilerleyen bir biçimde gelişme göstermişlerdir. Onlar yakın bir zaman zarfı içerisinde; bu bölgenin daha küçük yaratılmışlarının yaşamlarında üstünlük kurmuş olup, eskiden kalan etobur olmayan maymunsu kabilelerin çok azı varlığını sürdürebilmiştir.
\vs p062 2:6 Bu saldırgan küçük hayvanlar; fiziksel tür ve genel us bakımından sürekli gelişme gösteren bir biçimde, bin yıldan daha fazla bir süreç boyunca çoğalmış ve Mezopotamya yarımadasının tamamına yayılmıştır. Bu yeni kabilenin, lemur soyunun en yüksek türünden kökenini aldığı sürecin yalnızca yetmiş nesil sonrasında; Urantia üzerinde insan varlıklarının evrimleşmesi içinde bir sonraki hayati aşamaya ait ataların ansızın meydana gelen farklılaşması biçiminde, bir diğer çığır açıcı gelişme ortaya çıkmıştır.
\usection{3.\bibnobreakspace Ara\hyp{}Memeliler}
\vs p062 3:1 İlkel memelilerin oluşum süreçlerinin başında, bu çevik yaratılmışlarının üstün bir çiftinin sahip olduğu ağaç tepesi yerleşkesi içinde erkek ve dişi olarak ikiz canlı dünyaya gelmiştir. Ataları ile karşılaştırıldıklarında onlar gerçekten güzel küçük yaratılmışlardı. Onlar bedenleri üzerinde çok az miktarda kıla sahipti; ancak bu oluşum, ılıman ve düzenli bir iklimde yaşamaları nedeniyle kendilerinin gelişimi bakımından bir kısıtlılık niteliğinde bulunmamaktaydı.
\vs p062 3:2 Bu çocuklar, dört fitin biraz üstünde bir uzunluğa gelen bir biçimde büyümüşlerdir. Onlar her bakımdan, daha uzun bacaklara ve kısa kollara sahip bir biçimde ebeveynlerinden büyük varlıklardı. Onlar, çeşitli uğraşlar için oldukça iyi bir biçimde uyum sağlamış mevcut insan başparmağına benzer bir biçimde, kıvrılabilen ve dönebilen neredeyse kusursuz başparmaklarına sahiplerdi. Onlar, neredeyse daha sonraki insan ırklarının yürümeleri için oldukça elverişli uzuvlarına benzer bacaklara sahip olarak, dik bir biçimde hareket etmişlerdir.
\vs p062 3:3 Onların beyinleri; insan varlıklarına kıyasla daha alt bir düzeyde olup onlarınkinden daha küçüktü, ancak atalarına kıyasla daha üstün ve göreceli olarak daha büyüktü. İkizler; ilk olarak daha üstün usu sergilemiş olup, toplumsal düzenin ilkel bir türünün ve henüz ilkel bir iş bölümünün temellerini atarak ilkel memelilere ait kabilenin bütününün başları olarak yakın bir zaman içerisinde tanınmışlardır. Bu erkek ve dişi kardeş çiftleşmiş ve yakın bir zaman içinde, hepsinin dört fitten daha uzun olduğu ve her bakımdan köken türlerinden daha üstün niteliklerde bulunduğu, kendileri gibi yirmi bir çocuktan oluşan bir topluluk içinde yaşamışlardır. Bu yeni topluluk, ara\hyp{}memelilerin çekirdeğini oluşturmuştur.
\vs p062 3:4 Bu yeni ve üstün topluluğun nüfusu artınca, acımasız mücadeleler biçiminde savaşlar çıkmıştır; ve bu korkunç mücadele sonlandığında, ilkel memelilerin daha önceden mevcut olan türlerine ve köken ırklarına ait hiçbir birey hayatta kalmamıştır. Türlerin sayıca daha az olanları ancak daha güçlü ve daha fazla usu barındıran ırk kolları, atalarının yok oluşu pahasına hayatta kalmışlardır.
\vs p062 3:5 Ve bu aşamada (altı yüz nesillik bir süreçte) yaklaşık olarak on beş bin yıllık bir süre zarfı boyunca bu yaratılmış, dünyanın bu kısmının dehşet yayan varlığı haline gelmiştir. Daha önceki zamanların büyük ve hırçın hayvanlarının tümü ortadan yok olmuşlardır. Bu bölgelere özgü geniş yapılı canlılar etçil değillerdi; buna ek olarak, aslanlar ve kaplanlar biçimindeki kedi ailesinin daha iri türleri dünya yüzeyinin bu özel yalıtılmış ücra noktasını henüz istila etmemişti. Böylelikle bu ara\hyp{}memeliler cesur bir biçimde ortaya çıkmış ve yaratımlarının bu köşesinin bütününü egemenlikleri altına almıştır.
\vs p062 3:6 Atasal türlerine kıyasla ara\hyp{}memeliler, her bakımdan bir gelişim unsurlarıydı. Onların olası yaşam ömürleri bile, yirmi beş yılla yakın bir biçimde, daha uzundu. Gelişmemiş insan niteliklerinin bir miktarı bu yeni türler içinde ortaya çıkmıştır. Ataları tarafından sergilenen içkin eğilimlere ek olarak bu ara\hyp{}memeliler, belirli itici durumlarda iğrenme belirtisi göstermeye yetkinlerdi. Onlar daha ileri bir biçimde, oldukça belirgin bir biriktirme içgüdüsüne sahip olmuşlardır; onlar, yiyeceklerini daha sonraki kullanımlar için saklayacak olup, savunma ve saldırı için elverişli olan yuvarlak çakıl birikintilerine ve pürüzsüz taşların belirli türlerine fazlasıyla sahiplerdir.
\vs p062 3:7 Bu ara\hyp{}memeliler; ağaç tepesi evlerinin ve toprak altı çok geçitli sığınaklarının inşasında gösterdikleri rekabette olduğu gibi, belirgin bir inşa eğilimi gösteren ilk türlerdir; onlar, ağaç ve yeraltı barınakları içinde güvenliği ilk kez sağlayan memelilerin en öncül türleriydi. Onlar, yer üstünde gündüz yaşayan ve ağaç tepelerinde geceleri uyuyan bir biçimde, ağaçları yerleşkeleri olarak tercih etmeyi büyük ölçüde bıraktılar.
\vs p062 3:8 Zaman geçtikçe nüfuslarındaki doğal artış, bütün türleri neredeyse tamamen yok eden öldürücü çatışmaların bir dizisiyle sonuçlanan, ciddi bir yiyecek rekabetine ve cinsel ilişki çekişmesine nihai olarak sebep olmuştur. Bu mücadeleler, yüz bireyden biraz daha az üyeye ait bir tek topluluk hayatta kalana kadar devam etmiştir. Ancak barış bir kez daha hüküm sürmüştür; ve bu varlığını devam ettiren yalnız kabile kendisine ait ağaç tepesi yatak odalarını yeniden inşa etmiş olup, olağan ve yarı\hyp{}huzurlu mevcudiyetlerine kaldıkları yerden devam etmişlerdir.
\vs p062 3:9 Sizler, insan\hyp{}öncesi atalarınızın zaman zaman nasıl küçük farklarla yok olmaktan kurtulduklarını neredeyse hiçbir biçimde algılamazsınız. İnsanlığın tümünün kaynağını aldığı atasal kurbağa belirli bir durumda iki inç daha az zıplasaydı, evrimin bütün gidişatı dikkate değer bir biçimde değişirdi. İlkel memelilerin birincil lemursal annesi, yeni ve daha yüksek memeli düzeyinin babasını doğurmadan önce ölümünden beş kez kıl payı farkla kurtuldu. Ancak bunların ölüme en yakın olanı, primat ikizlerinin olası annesinin uyuduğu ağaca yıldırımın düşmesiydi. Bu ara\hyp{}memelilerinin ebeveynlerinin ikisi de, ciddi bir biçimde elektrikle çarpılmış ve kötü bir biçimde yanıklara sahip olmuştur; onların yedi çocuğundan üçü, gökten gelen bu yıldırım sonucunda ölmüştür. Bu evrimleşen hayvanlar neredeyse hurafelere sahip varlıklardı. Ağaç tepesi evine yıldırım düşen bu çift, ara\hyp{}memeli türlerinin daha ilerleyici topluluğunun gerçek anlamıyla liderleriydi; ve bu çiftin yaşadığı hadiseden sonra daha ussal ailelere sahip olan kabilenin yarıdan fazlası bahse konu yerleşkeden yaklaşık iki mil daha uzağa taşınıp, beklenmedik tehlike anında geçici barınakları olan yeni ağaç tepeleri ve yeni yer altı sığınaklarının inşasına başlamışlardır.
\vs p062 3:10 Evlerinin tamamlanmasından sonra yakın bir zaman içerisinde, birçok mücadeleden sağ kurtulan bu çift kendilerini; insan\hyp{}öncesi evrim içerisinde bir sonraki hayati aşamayı oluşturan \bibemph{Primatlar’ın} yeni türlerinin ilki olan, bu zamana kadar yeryüzünde doğmuş en ilgi çekici ve en önemli hayvanlar biçiminde, ikizlerin gururlu ebeveynleri olarak buldular.
\vs p062 3:11 Bu Primat ikizlerinin doğum sürecine yakın bir zaman içerisinde, --- ara\hyp{}memeli kabilenin özellikle alt seviye düzey aklına sahip bir erkek ve dişi olarak --- akıl ve fizik bakımından aşağı konumda bulunan bir çift başka tür ikizleri dünyaya getirmiştir. Bir erkek ve dişiden oluşan bu ikiz, diğer türler karşında baskınlık kurmaya çalışmamaktaydı; onlar yalnızca yiyecek elde etmekle ilgilenmişti, ve etçil olmadıkları için yakın bir zaman içerisinde avlarını elde etmedeki tüm isteklerini kaybetmişlerdi. Bu us bakımından düşük düzeyde bulunan ikizler, çağdaş maymun kabilelerinin kurucuları haline geldiler. Onların soyları; asya şebekleri ve diğer maymunların öncül türleri ile çiftleşen ırk kolları dışında bahse konu zaman zarfında yaşadıkları gibi ve bunun sonucunda nesillerinin zarar gördüğü yerler olan, güney bölgelerinin ılık iklimlerini ve sıcak iklim meyvelerinin bol olduğu yerleşkeleri tercih etmişlerdir.
\vs p062 3:12 Ve bu anlatımlar sonucunda açık bir biçimde gözlenebilir ki, insan ve maymun yalnızca; ikizlerin iki çiftinin eş zamanlı doğumu ve bunun sonrasında ayrışmasının içinde gerçekleştiği bir kabile olarak, ara\hyp{}memelilerden çıktıkları köken bakımından birbirleriyle ilişkilidir; aşağı düzeyde bulunan çift günümüz maymununu, babununu, şempanzesini ve gorilini üretme nihai sonuna sahip olurken; üstün olan çift ise, insanın kendisine olan evrimin yükseliş doğrultusu boyunca ilerleme nihai sonuna sahip olmuştur.
\vs p062 3:13 Çağdaş insan ve maymunlar aynı kabile türlerinden türemişlerdir, aynı ebeveynlerden değil. İnsanın ataları, ara\hyp{}memeliler kabilesinin seçilmiş kalıntılarına ait üstün ırk kollarından gelmektedirler; bunun karşısında ise çağdaş maymunlar (lemurlar, asya şebekleri, diğer maymunlar ve onlara benzer yaratılmışların dışında), yalnızca düşmanlıklar bütünüyle sona erdiğinde kabilelerin en son şiddetli savaşı boyunca iki haftadan daha uzun bir süre boyunca bir toprakaltı besin sığınağı içinde kendilerini saklayarak hayatta kalabilen bir çift biçimindeki, bu ara\hyp{}memeli topluluğunun en alt düzeyde bulunan çiftinin soyundan gelmektedirler.
\usection{4.\bibnobreakspace Primatlar}
\vs p062 4:1 Ara\hyp{}memeli kabilesinin iki önde gelen üyesi biçiminde bir erkek ve bir dişi olarak üstün ikizlerin doğumuna gelecek olursak: Bu hayvan bebekleri benzersiz bir düzeye aitlerdi; onlar ebeveynlerinden bile daha az kılı taşımakta olup, çok genç yaşta dik bir biçimde yürümekte ısrar ettiler. Onların ataları her zaman, arka ayakları üzerinde yürümeyi öğrenmişlerdi; fakat bu Primat ikizleri en başından beri dik bir duruşa sahip olmuştur. Onlar; beş fitin üstünde bir yüksekliğe erişmiş olup, kafaları kabilenin diğer üyelerine kıyasla daha büyük bir haldeydi. İşaretler ve sesler aracılığıyla birbirleriyle konuşmayı önceden öğrenirlerken, bu yeni sembollerin ne anlama geldiğini kendi topluluk üyelerine bir türlü anlatamadılar.
\vs p062 4:2 Yaklaşık olarak on dört yaşında, ailelerini büyütmek ve Primatlar’ın yeni türlerini oluşturmak amacıyla kabilelerinden ayrıldılar. Ve bu yeni yaratılmışlar; insan ailesinin doğrudan ve birincil hayvan ataları oldukları için, oldukça yerinde bir biçimde \bibemph{Primatlar} olarak adlandırılmaktadır.
\vs p062 4:3 Daha az ussa sahip ve birbirleri ile yakın bir biçimde ilişki halinde olan kabileler yarımada merkezinde ve onun üstündeki doğu sahil hattı civarında yaşarken; Primatlar, bahse konu zaman zarfında güney denizine doğru bakan Mezopotamya yarımadasının batı sahili üzerinde bir bölgeye böylece yerleşmek için geldiler.
\vs p062 4:4 Primatlar, ana\hyp{}memeli soylarına kıyasla daha çok insan ve daha az hayvandı. Bu yeni türlerin iskelet oranları, ilkel insan ırklarının sahip olduklarına çok benzemekteydi. İnsan türünün el ve ayağı bütünüyle gelişmişti; ve bu yaratılmışlar, daha sonraki insan soylarının herhangi biri gibi yürüyebilmekte ve hatta koşabilmekteydi. Onlar büyük bir ölçüde ağaç yaşamını geride bırakmış olsa da, geceleri bir güvenlik önlemi olarak ağaç tepelerine ikamet etmeye devam ettiler; çünkü önceki ataları gibi onlar fazlasıyla korku duyan türlerdi. Ellerinin artan oranda gerçekleşen kullanımı, içkin beyin gücünün gelişmesine katkıda bulunmuştur; ancak onlar henüz, insan olarak gerçek anlamıyla adlandırılabilecek akıllara sahip değillerdi.
\vs p062 4:5 Her ne kadar duygusal doğa bakımından Primatlar atalarından çok az ölçüde farklı olsalar da, sahip oldukları niteliklerin tümü bakımından daha çok bir insan eğilimi sergilediler. Onlar, gerçekten de, yaklaşık olarak on yaşında ergenliğe erişerek ve kırk yaşa varan bir yaşam ömrüne sahip olarak, harika ve üstün hayvanlardı. Bu yaşam ömrü, doğal ölümlerinin gerçekleştiği zamana kadar yaşamalarını varsayan bir ölçektir; ancak bu öncül zamanlarda çok az hayvan doğal bir ölüm sonucunda hayatını kaybetmiştir; mevcudiyet için verilen mücadele bütünü bakımından fazlasıyla yoğundu.
\vs p062 4:6 Ve bu aşamada, ilkel memelilerin doğumundan itibaren yirmi bir bin yılı kaplayan bir süreç olarak gelişimin neredeyse dokuz yüz neslinden sonra Primatlar, \bibemph{ansızın}, ilk gerçek insan varlıkları olarak iki dikkate değer yaratılmışı dünyaya getirdiler.
\vs p062 4:7 Kuzey Amerikalı lemur türünden doğan, ara\hyp{}memelilere kaynağını veren, ve bu yaratılmışlardan ilkel insan ırkının doğrudan ataları haline gelen üstün Primatlar’ı dünyaya getiren canlılar böylelikle ilkel memelilerdir. Primatlar, insanın evrimi içinde en son hayati zincirdi; ancak beş bin yıldan daha az bir süre içerisinde bu olağanüstü kabilelerden tek bir birey bile hayatta kalmamıştır.
\usection{5.\bibnobreakspace İlk İnsan Varlıkları}
\vs p062 5:1 M.S. 1934 yılından ilk iki insan varlığının doğuşuna kadar yalnızca 993.419 yıl geçmiştir.
\vs p062 5:2 Bu iki dikkate değer yaratılmış, gerçek insan varlıklarıydı. Onlar, birçok atasının sahip olduğu gibi, kusursuz insan başparmaklarını ellerinde bulundururken; günümüz insan ırklarının tıpkı sahip olduğu gibi kusursuz ayaklara sahiptiler. Onlar yürüyücü ve koşuculardı, tırmanıcı değillerdi; büyük ayak tırnağının kavrayıcı faaliyeti bulunmamaktaydı, bu nitelikten tamamen yoksunlardı. Tehlike onları ağaç tepelerine sürüklediği zaman, tıpkı bugünün insanlarının yapacağı gibi tırmandılar. Onlar, bir ağacın tepesine bir ayı gibi tırmanacak, bir şempanzenin veya gorilin yapacağı gibi daldan dala atlamayacaktır.
\vs p062 5:3 Bu ilk insan varlıkları (ve onların soyları), on iki ergenliklerini tamamlamış olup, yaklaşık olarak yetmiş beş yıllık bir olası yaşam ömrünü sahiptiler.
\vs p062 5:4 İnsan ikizlerinde yeni birçok duygu öncül bir biçimde ortaya çıkmıştır. Onlar; eşyalar ve varlıkların ikisi içinde besleyebildikleri beğeniyi deneyimlemiş olup, dikkate değer kibri sergilediler. Ancak duygusal gelişim bakımından en ilgi çekici gelişme; ibadet bütünlüğü olarak --- saygı, ağırbaşlılık ve hatta minnettarlığın ilkel bir türünü içine alan bir biçimde --- gerçek insan hislerinin yeni bir topluluğunun ansızın ortaya çıkmasıydı. Doğal olgulara karşı duyulan önemsemezlik ile birleşince korkunun ilkel dini dünyaya getirmesi yakındı.
\vs p062 5:5 Bahse konu ilkel insanlarda; yalnızca bu türden insan hisleri değil, aynı zamanda oldukça yüksek bir biçimde evirilmiş duyguların olgunlaşmamış türü de mevcut bulunmaktaydı. Onlar; orta düzeyde acımanın, utancın ve tiksintinin hissiyatında olup, derin bir biçimde sevginin, nefretin, intikamın ve ayrıca kıskançlığın belirgin duygularının da bilincindelerdi.
\vs p062 5:6 İkizler olarak bu ilk iki insan, Primat ebeveynleri için büyük bir zorluk teşkil etmekteydi. Onlar; o kadar meraklılar ve o kadar maceraperesttiler ki, sekiz yaşına gelmeden önce sayısız birçok durumda neredeyse yaşamları kaybetme gerçeğiyle yüz yüze gelmişlerdir. Böyle olduğu için onlar, on iki yaşında oldukça korku duyan bir konuma geldiler.
\vs p062 5:7 Çok küçük yaşlarda onlar sözlü iletişim kurmayı öğrendiler; on yaşında onlar, neredeyse elli düşüncelik bir işaret ve kelime dilini geliştirip, atalarının ilkel iletişim tekniğini genişlettiler. Ancak onlar ne kadar çabalarlarsa çabalasınlar, kendilerine ait yeni işaret ve simgelerin sadece çok az bir miktarını ebeveynlerine öğretebildiler.
\vs p062 5:8 Yaklaşık olarak dokuz yaşında onlar, açık bir gün vakti nehre doğru hareket edip çok önemli bir görüşme düzenlediler. Urantia üzerinde konumlanmış olan her göksel us, buna ben dâhil olmak üzere, bu öğle vakti gerçekleşen toplanım görüşmelerinin bir gözlemcisi olarak orada hazır bulunduk. Bu önemli günde onlar, kendi aralarında ve birbirleri için yaşama amacında bir anlaşmaya vardılar; ve bu sözleşme, insan ırkının böylece kuruluşunu gerçekleştireceklerinden habersiz bir biçimde, aşağı düzeyde bulunan hayvan birlikteliklerinden nihai olarak ayrılma ve kuzeye doğru hareket etme kararlarıyla sonuçlanacak bir dizi anlaşma zincirlerinin ilkiydi.
\vs p062 5:9 Her ne kadar hepimiz, bu iki ilkel varlığın tasarımlarından bütünüyle korku duysak da; karar alış süreçlerini denetlemede güçsüz bir konumda bulunmaktaydık; bizler --- yetkin olmayan bir biçimde --- onların kararlarına keyfi olarak müdahalede bulunmadık. Ancak gezegensel faaliyetin müsaade eden sınırları içinde, birlikteliklerimiz ile birlikte Yaşam Taşıyıcıları olarak hepimiz; insan ikizlerinin kuzeye ve kıllı olmalarına ek olarak kısmi olarak ağaçta barınan topluluklarından uzağa doğru hareket edişlerini destekler tasarımlarda bulunduk. Ve böylelikle, kendilerinin ussal tercihi sonucunda bu ikizler \bibemph{göçlerini} gerçekleştirmişlerdir; ve bizim yüksek denetimimiz vasıtasıyla onlar, Primat kabilelerinin aşağı düzeyde bulunan akrabaları ile karışmaları neticesinde ortaya çıkabilecek biyolojik gerilemenin olasılığından uzaklaştıkları tecrit edilmiş bir bölgeye doğru \bibemph{kuzey doğrultusunda} hareket etmişlerdir.
\vs p062 5:10 Ev ormanlarından ayrılışlarından kısa bir süre önce onlar, bir asya şebeği saldırısında annelerini kaybetmişlerdir. Her ne kadar bu anne çocuklarının taşıdığı ussu elinde bulundurmamış olsa da, çocukları için yüksek bir düzeye karşılık gelen bir memeli sevgisine sahipti; ve o korkusuz bir biçimde kendi canını, bu muhteşem çiftin hayatını koruma girişiminde feda etmiştir. Bu fedakârlık amaçsız bir biçimde gerçekleşmemiştir; çünkü o düşmanı, çocuklarının babası takviye güçler ile gelinceye ve saldırganları bozguna uğratıncaya kadar, oyalamıştır.
\vs p062 5:11 Bu genç çiftin birlikteliklerini geride bırakıp insan ırkının temellerini atmasından yakın bir zaman sonra, babaları --- kalbi kırık bir biçimde --- kederli hale gelmiştir. Bu varlık yemeği, diğer çocukları tarafından kendisine yemek sunulduğunda bile, reddetmiştir. Onun muhteşem çocukları ortadan kaybolmuş, bu yüzden sıradan türleri arasında yaşamak onun için anlamsız hale gelmişti; bunun sonrasında o ormanın derinliklerine doğru uzaklaşmış, düşman asya şebekleri tarafından pusuya düşmüş ve yaşamını yitirinceye kadar dövülmüştü.
\usection{6.\bibnobreakspace İnsan Aklının Evrimi}
\vs p062 6:1 Urantia üzerinde Yaşam Taşıyıcıları olarak bizler, gezegensel sularda yaşam plazmasını ilk kez aktardığımız günden beri dikkatli gözlemimizin uzun bekleyiş süreci boyunca ilerlemekteydik; ve ilk gerçek us ve irade sahibi varlıkların ortaya çıkışı doğal olarak bizlere büyük bir neşe ve yüce tatmini beraberinde getirmiştir.
\vs p062 6:2 Bizler, gezegene varışımızda Urantia için görevlendirilmiş olan yedi emir\hyp{}yardımcı akıl\hyp{}ruhaniyetinin dikkatle gözlediğimiz faaliyeti içinde ikizlerin akılsal gelişimini takip etmekteydik. Gezegensel yaşamın uzun evrimsel gelişimi boyunca bu yorulmak bilmez akıl hizmetkârları, gelişimsel olarak üstün hayvan yaratılmışlarının genişleyen beyin yetkinlikleri ile peşi sıra bir şekilde iletişimde bulunuşlarının sürekli artan kabiliyetlerini bizlere bildirdiler.
\vs p062 6:3 İlk başta yalnızca \bibemph{içgüdü ruhaniyeti} temel hayvan yaşamının sezgisel ve tepkisel davranışı içinde faaliyet gösterebilmekteydi. Daha yüksek türlerin farklılaşması ile birlikte \bibemph{anlayış ruhaniyeti}, düşüncelerin eş zamanlı birlikteliğini bu tür yaratılmışlara kazandırmaya yetkin hale geldi. Daha sonrasında ise bizler, \bibemph{cesaret ruhaniyetinin} faal olduğunu gözlemledik; evrimsel faniler gerçek anlamıyla koruyucu öz benliğin ilkel bir türünü geliştirdiler. Memeli topluluklarının ortaya çıkışından sonra bizler \bibemph{bilgi ruhaniyetinin} artan bir ölçekte kendisini dışavurumunu dikkatle gözlemledik. Ve daha yüksek memelilerin evrimi, sürü içgüdüsünün büyümesi ve ilkel toplumsal gelişimin başlaması ile birlikte \bibemph{danışma ruhaniyetinin} faaliyetini beraberinde getirmiştir.
\vs p062 6:4 İlkel memeliler, ara\hyp{}memeliler ve Primatlar’ın peşi sıra bizler artan bir biçimde ilk beş emir\hyp{}yardımcı niteliğin bütünleşen hizmetini gözlemledik. Ancak en yüksek akıl hizmetkârları biçiminde geride kalan iki nitelik, Urantia türünün evrimsel aklı içinde bu dönemde faaliyet göstermeye yetkin bir konumda bulunmamaktaydı.
\vs p062 6:5 İkizler yaklaşık olarak on yaşına geldiğinde, \bibemph{ibadet ruhaniyetinin} ikizlerin kadın olan bireyinin aklı ile erkeğinkinin hemen ardından ilk iletişime geçtiği gün bizin duyduğumuz mutluluğu hayal edin. Bizler insan aklının en yüksek düzeyine erişiminin çok yakın bir süre içinde gerçekleşeceğini bilmekteydik; ve bir yıl sonra onlar, irdeleyici düşünüş ve amaçsal kararlarının bir sonucu olarak evlerinden ayrılıp güneye doğru hareket etmeleri gerektiğini nihai olarak tasarladıklarında, bunun sonrasında\bibemph{ bilgeliğin ruhaniyeti}, Urantia üzerinde ve bu aşamada artık tanınmış olan iki insan beyni içerisinde faaliyet göstermeye başlamıştır.
\vs p062 6:6 Orada, yedi emir\hyp{}yardımcı akıl\hyp{}ruhaniyetinin işlerlik kazanmasının doğrudan ve yeni bir düzeyi bulunmaktaydı. Bizler bütünüyle arzu dolu beklenti içerisindeydik; uzun sürelerdir beklediğimiz anın yaklaşmakta olduğunun farkına vardık; Urantia üzerinde irade sahibi yaratılmışlarının evirilişi için uzun süredir gösterdiğimiz çabaların meyvesini toplamaya başlama anının geldiğinin bilincindeydik.
\usection{7.\bibnobreakspace Yerleşik bir Dünya Olarak Tanınma}
\vs p062 7:1 Bizler bu aşama sonrasında çok uzun bir süre beklemedik. İkizlerin evlerinden kaçtıkları günün sonrasında öğlen vakti, Urantia’nın gezegensel alıcı\hyp{}odağında evren döngüsünün başlangıçsal deneme ışıması gerçekleşmiştir. Tabii ki hepimiz büyük bir gelişmenin yaklaşmakta olduğunun farkındalığında heyecanlıydık; ancak bu dünya yaşam denemelerinin gerçekleştiği bir hazırlanış yerleşkesi olduğu için, gezegen üzerinde ussal yaşamın tanınmasının tam olarak nasıl haberdar edileceğine dair en ufak bir fikre sahip değildik. Ancak biz bu belirsizlik içinde çok kalmadık. İkizlerin kaçışının üçüncü gününde, ve Yaşam Taşıyıcı birliklerinin ayrılışından önce, Nebadon baş meleğine ait başlangıçsal gezegensel döngü oluşumu gezegene ulaşmıştır.
\vs p062 7:2 Uzay iletişiminin gezegensel havuzu yakınında bizim küçük topluluğumuz bir araya geldiği ve gezegenin yeni oluşturulmuş akıl döngüsü üzerinden Salvington’un ilk iletisini aldığımız gün, Urantia üzerinde oldukça önemli bir gündü. Ve baş melek birliklerinin önderi tarafından yazdırılmış ilk ileti şunları ifade etmekteydi:
\vs p062 7:3 “Urantia üzerindeki Yaşam Taşıyıcıları’na --- Selamlar! Akıl ve irade soyluluğunun Urantia üzerindeki mevcudiyetine dair işareti Nebadon yönetim merkezi üzerinde almamızın onuru taşıyarak Salvington, Edentia ve Jerusem üzerindeki derin memnuniyetin teminatını sizlere iletmekteyiz. İkizlerin kuzeye doğru hareket etmelerine ve aşağı düzey atalarından kendi doğumlarını ayırmalarına dair amaçsal kararları tarafımızdan kayda geçirilmiştir. Bu düşünce; --- insan türünün aklı olarak --- aklın ilk kararı olup, tanınmanın bu başlangıçsal iletisinin üzerinde taşındığı iletişim döngüsünü kendiliğinden kurmaktadır.”
\vs p062 7:4 Bu yeni döngü üzerinde bu iletiyi takiben, oluşturduğumuz yaşamın işleyiş biçimine karışmamamızı yasaklayıcı ikamet halindeki Yaşam Taşıyıcılar için yönergeleri taşıyan Edentia’nın En Yüksek Unsurları’nın selamları ulaşmıştır. Bizlerin insan ilerleyişinin durumlarına karışmamamız emredilmiştir. Yaşam Taşıyıcıları’nın en başından beri keyfi olarak ve her aşamada gezegensel evrim tasarımlarının doğal işleyişlerine karışmakta olduğu bu yönergeden çıkarılmamalıdır; çünkü biz böyle bir şeyi hiçbir zaman gerçekleştirmemekteyiz. Ancak bu zaman zarfına kadar bizlerin özel bir biçimde çevre koşulları üzerinde değişiklerde bulunmamıza ve korumamıza izin verilmiştir; ancak bu aşamada, bütünüyle doğal olan yüksek denetimin varlığı sona erdirilmiştir.
\vs p062 7:5 En Yüksek Unsurlar yönetimi güzel bir ileti ile Lucifer’e bırakır bırakmaz, Satania sistem egemeninin gezegenleştirme süreci başlamıştır. Bu aşamada Yaşam Taşıyıcıları önderinin karşılayıcı sözlerini duymuş, ve Jerusem’e dönmek için gereken izni almışlardır. Lucifer’den gelen bu ileti; Yaşam Taşıyıcıları’nın Urantia üzerindeki çalışmasının kabulünü taşımakta olup, Satania sistemi içinde oluşturulmuş olan Nebadon yaşam biçimlerine dair herhangi bir çabamıza karşı getirilecek gelecekteki tüm eleştirilerden bizleri muaf kılmıştır.
\vs p062 7:6 Salvington ve Jerusem’den gelen bu iletiler, Yaşam Taşıyıcılar’ın gezegen üzerindeki çağlar süren yüksek denetiminin tamamlanışını resmi bir biçimde simgelemiştir. Çağlar boyunca bizler, yalnızca yedi emir\hyp{}yardımcı akıl\hyp{}ruhaniyeti ve Üstün Fiziksel Düzenleyiciler tarafından yardım görerek görevimiz üzerinde çalışmaktaydık. Ve bu aşamada; ibadet ve yükselişi tercih etme gücü olarak iradenin gezegenin evrimsel yaratılmışları içinde ortaya çıkmasıyla bizler, görevimizin tamamlandığının farkına vardık, ve geride iki kıdemli Yaşam Taşıyıcı’sı ile on iki yardımcıyı bu gezegende bırakma iznine sahip olduk; ve ben bu topluluğun bir üyesi olup, Urantia üzerinde bu zaman sürecinden beri ikamet etmekteyim.
\vs p062 7:7 Nebadon evreni içinde insan yaşam alanının barındığı bir gezegen olarak Urantia’nın resmi olarak tanınmasından (M.S. 1934 yılından bu yana) yalnızca 993.408 sene geçmiştir. Biyolojik evrim, irade soyluluğunun insan düzeylerine bir kez daha erişmiş; insan, Satania’nın 606’ıncı gezegeni üzerine ulaşmıştır.
\vs p062 7:8 [Bu anlatım, Urantia üzerinde ikamet etmekte olan Nebadon’un bir Yaşam Taşıyıcısı tarafından sağlanmıştır.]
