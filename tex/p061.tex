\upaper{61}{Urantia üzerindeki Memeliler Dönemi}
\vs p061 0:1 Memeliler dönemi, elli milyon yıldan biraz daha az bir süreci kapsayarak, karınbağı memelilerinin ortaya çıkış zamanlarından buz devrinin sonuna kadar uzanmaktadır.
\vs p061 0:2 Bu Senozoyik devir boyunca dünyanın kara tabiatı --- engebeli tepeler, geniş vadiler, büyük nehirler ve muazzam orman toplulukları biçiminde --- ilgili çekici bir görünüm sunmuştur. Bu süreç boyunca iki kez Panama Kanalı yükselip alçalmıştır; üç kez Bering Boğazı kara köprüsü aynı gelişimi göstermiştir. Hayvan türleri çok ve çeşitli bir düzeyde bulunmaktaydı. Ağaçlar kuşlarla dolup taşmış, dünyanın tümü, üstünlük için evrimleşen hayvan türlerinin sonu gelmez mücadelesi dışında, bir hayvan cennetine dönüşmüştü.
\vs p061 0:3 Bu elli milyon yıllık devrin beş dönemine ait tortullaşmış birikintiler, birbirlerini takip eden memeli soylarından insanın mevcut olarak ortaya çıkışına kadar olan zamanlara kadar uzanan fosil kayıtlarını taşımaktadır.
\usection{1.\bibnobreakspace Yeni Kıtasal Kara Aşaması\\Öncül Memeli Çağı}
\vs p061 1:1 \bibemph{50.000.000} yıl önce dünyanın kara bölgeleri, oldukça yaygın bir biçimde suyun üstünde veya çok az bir şekilde onun altında bulunmaktaydı. Bu dönemin oluşumları ve birikintileri karasal ve denizsel niteliğin ikisine de sahip olup, karasal ağırlıkta bulunmaktadır. Dikkate değer bir zaman süreci boyunca kara kademeli olarak yükselmiştir, ancak su baskını etkisiyle eş zamanlı olarak alçalmış ve deniz seviyelerine doğru inmiştir.
\vs p061 1:2 Bu dönemin başında ve Kuzey Amerika’da memelilerin karınbağı türleri \bibemph{ansızın} ortaya çıkmıştır; ve bu canlılar, bu zaman zarfına kadar olan ki en önemli evrimsel gelişmeyi meydana getirmiştir. Karınbağı\hyp{}olmayan memelilerin önceki düzeyleri var olmuştu; ancak bu yeni tür doğrudan ve \bibemph{ansızın}, dinozorların azalış dönemi boyunca varlığını sürdüren mevcut sürüngen soylarından türemiştir. Karınbağı memelilerinin atası; küçük, oldukça hareketli, etçil, dinozordan türeyen bir cinsti.
\vs p061 1:3 Temel memeli içgüdüleri, bu ilkel memeli türleri içinde sergilenmeye başlamıştır. Memeliler, hayvan yaşamının tüm diğer türleri üzerinde çok büyük bir kurtuluş üstünlüğünü elinde bulundurmaktadır; onların bu üstünlükleri şu özelliklerden kaynaklanmaktadır:
\vs p061 1:4 1.\bibnobreakspace Göreceli olgun ve oldukça iyi gelişmiş nesilleri dünyaya getirmek.
\vs p061 1:5 2.\bibnobreakspace Doğumlarını sevecen bir tutumla beslemek, büyütmek ve korumak.
\vs p061 1:6 3.\bibnobreakspace Üstün beyin güçlerini yaşamlarını sürdürmek için kullanmak.
\vs p061 1:7 4.\bibnobreakspace Düşmanlarından kaçmak için artan çevikliklerinden faydalanmak.
\vs p061 1:8 5.\bibnobreakspace Çevresel şartlara alışma ve uyum için üstün aklı kullanmak.
\vs p061 1:9 \bibemph{45.000.000} yıl önce kıtasal omurgalar, kıyı şeritlerinin oldukça sık görülen bir biçimde batışıyla ilgili olarak yükselmiştir. Memeli yaşamı hızlı bir biçimde evrim göstermekteydi. Memelilerin yumurtlayan türü biçimindeki küçük bir sürüngen aile üyesi gelişti, ve daha sonraki kanguruların bu ataları Avustralya’ya yayıldı. Yakın bir zaman içinde orada; küçük atlar, düzayaklı gergedanlar, hortumlu tapirler, ilkel domuzlar, sincaplar, lemurlar, keseli sıçanlar ve maymuna benzer hayvanların birkaç kabilesi mevcut bulunmuştur. Onların tümü; küçük, ilkel ve dağ bölgelerinin ormanları arasındaki yaşama en iyi uyum sağlayan canlılardı. Geniş devekuşu\hyp{}vari karakuşu; on fit gibi bir yüksekliğe kadar büyüyecek ve on üç inç çapında dokuz yumurta dünyaya getirecek bir biçimde gelişme gösterdi. Bu canlılar, oldukça ussal ve bir zamanlar insanları havada taşıyan bir niteliğe sahip, daha sonraki devasa ulaşım kuşlarının atalarıydı.
\vs p061 1:10 Senozoyik devir başı memelileri; kara üstünde, su altında ve ağaç tepelerinde yaşamışlardı. Onlar bir çiften on bir çifte kadar değişen miktarlarda meme\hyp{}vari bezlere sahiplerdi; ve onların tümünün vücudu ciddi oranda da kıllarla kaplanmıştı. Daha sonra ortaya çıkan düzeyler ile ortak bir biçimde onlar; birbirini tamamlayan iki diş setine, ve bedenlerine kıyasla büyük beyinlere sahip oldular. Ancak onlar arısında bugünkü türlerin hiçbiri var olmamıştır.
\vs p061 1:11 \bibemph{40.000.000} yıl önce Kuzey Yarımküre’nin kara bölgeleri yükselmeye başlamıştır; ve bu yükselişi, geniş ölçekli yeni kara birikintilerine ek olarak lav akıntılarını, kabuk bükülmelerini, göl oluşumlarını ve toprak aşınmalarını içine alan diğer yeryüzü etkinlikleri takip etmiştir.
\vs p061 1:12 Bu çağın daha sonraki dönemi boyunca Avrupa’nın büyük bir kısmı sular altında kalmıştır. Küçük çaplı bir kara yükselişini takiben kıta, göller ve körfezler ile kaplanmıştı. Kuzey Buz Denizi, Ural çöküntüsü boyunca; deniz adaları olarak su seviyesinin üstünde bulunan Alpler, Karpatlar, Apennineler ve Piranalar’ın yükseklerine doğru kuzey yönlü genişlemiş bir biçimde bu zaman zarfında mevcut bulunan Akdeniz ile birleşmek için güneye doğru uzanmıştır. Panama Kanalı yükselmişti; Atlas ve Büyük Okyanus birbirinden ayrılmıştı. Kuzey Amerika Asya ile birlikte Bering Boğazı kara köprüsü, Avrupa ile Grönland ve İzlanda kanalıyla bağlanmıştır. Kuzey enlemlerde karanın dünya bağlantı hattı yalnızca, genişleyen Akdeniz ile birleşmiş olan kutup denizlerinin altında kalan Ural Boğazı tarafından çöküntüye uğramıştır.
\vs p061 1:13 Ciddi ölçekte foraminiferitsel kireçtaşı Avrupa suları içinde birikmiştir. Mevcut an içinde bahse konu taş; Alpler’de 10.000 fit, Himalayalar’da 16.000 fit ve Tibet’te 20.000 fitlik bir yüksekliğe çıkmıştır. Bu dönemin tebeşir kalıntıları; Afrika ve Avustralya’nın sahilleri boyunca, Güney Amerika’nın batı sahilinde, ve Batı Hint Adaları etrafında bulunmaktadır.
\vs p061 1:14 \bibemph{Eosen }devri olarak sizler tarafından adlandırılan bu süreç boyunca memelilerin ve yaşamın diğer ilgili türlerinin evrimi neredeyse hiçbir kesintiye uğramadan gelişimine devam etmiştir. Kuzey Amerika bu zaman zarfında kara vasıtasıyla Avustralya haricinde her kıtaya bağlı bir konumda bulunmaktaydı; ve dünya kademeli olarak, çeşitli türlerin ilkel memeli hayvanatları ile kaplanmaktaydı.
\usection{2.\bibnobreakspace Yeni Su Baskını Aşaması\\Gelişmiş Memelilerin Çağı}
\vs p061 2:1 Bu süreç, bu çağlar boyunca niteliği artan memeli yaşamının daha gelişimsel türleri biçimindeki karınbağı memelilerinin daha ileri ve hızlı evrimi tarafından simgelenmekteydi.
\vs p061 2:2 Her ne kadar öncül karınbağı memelileri etçil atalarından türemiş olsalar da, oldukça yakın bir zaman zarfı içinde onların otçul türleri gelişmiş olup, buna ek olarak çok geçmeden hem etçil hem otçul memeli aileleri aynı zamanda evirilmiştir. Kapalı tohumlu bitkiler; daha önceki süreçlerde ortaya çıkmış bir biçimde bugünkü bitki ve ağaçların büyük bir çoğunluğunu içine alan çağdaş kara bitkisi örtüsü olarak, sayıları hızla artan memelilerin temel besin kaynağıydı.
\vs p061 2:3 \bibemph{35.000.000} yıl öncesi, karınbağı memeli dünyasının sahip olduğu hâkimiyet çağının başlangıcını simgelemektedir. Güney kara köprüsü, bu zaman zarfında devasa konumda bulunan Güney Kutup karasını Güney Amerika, Güney Afrika ve Avustralya ile yeniden bağlayan bir biçimde oldukça geniş bir konumda bulunmaktaydı. Yüksek enlemlerde kara kütlesinin aksine dünya iklimi, sıcak iklim denizlerinin büyüklüğünde olan devasa artış nedeniyle göreceli olarak ılık kalmaya devam etmiştir; buna ek olarak kara, buzul oluşumunu meydana getirecek kadar yükselmemiştir. Geniş lav akıntıları, Grönland ve İzlanda’da meydana gelmiştir; az miktarda kömür bu akıntıların tabakaları arasında tortullaşmıştır.
\vs p061 2:4 Ciddi ölçekteki değişiklikler, gezegenin hayvan yaşamı içinde gerçekleşmekteydi. Deniz yaşamı, büyük bir değişikliğe uğramaktaydı; deniz yaşamının sahip olduğu bugünkü türlerinin birçoğu bu zaman zarfında mevcut olup, forminiferler bu yaşamda önemli bir rol oynamaya devam etti. Böcek yaşamı, bir önceki dönemdeki yaşamına oldukça benzer bir konumdaydı. Colorado’nun Florissant fosil yatakları, bu uzak zamanların son dönemlerine aittir. Yaşayan böcek ailelerinin büyük bir kısmı, kökensel olarak bu döneme uzanmaktadır; ancak her ne kadar fosilleri kalmış olsa da bu zaman zarfında mevcut olanların birçoğunun nesli tükenmiştir.
\vs p061 2:5 Kara üzerinde bu dönem, memeli değişimi ve gelişiminin başat çağıdır. Daha önceki ve daha ilkel memeliler arasında yüz türden fazlasının nesli bu dönem sona ermeden tükenmiştir. Geniş büyüklüğe ve küçük beyne sahip memeliler bile yakın zaman içerisinde yol olmuştur. Beyin ve çeviklik, hayvan yaşam mücadelesinin ilerleyişinde zırh ve beden büyüklüğünün yerini almıştır. Ve dinozor ailesinin çöküşü ile birlikte, sürüngen atalarının kalıntılarını hızlı ve bütüncül bir biçimde yok ederek memeliler yavaş bir biçimde dünyanın hâkimiyetini ele geçirmişlerdir.
\vs p061 2:6 Dinozorların ortadan yok oluşu ile birlikte, sürüngen ailesinin çeşitli kollarında farklı ve büyük değişiklikler meydana geldi. Öncül sürüngen ailelerinin hayatta kalan üyeleri; insanın öncül atalarının geriden kalan tek temsilci topluluğu olarak, saygın kurbağalar ile birlikte kaplumbağalar, yılanlar ve timsahlardır.
\vs p061 2:7 Memelilerin çeşitli toplulukları kökenlerini şu an nesli tükenmiş bir hayvandan almışlardır. Bu etçil yaratılmış, bir kedi ile bir ayı balığının birleşime benzer bir canlıydı; kara ve su üzerinde yaşabilmekte olup, oldukça akıllı ve etkindi. Avrupa’da bu köpek cins ailesinin atası, yakın bir zaman içinde küçük köpeklerin birçok türünün ortaya çıkmasına sebep olan bir biçimde evirilmiştir. Yaklaşık olarak yine bu zaman zarfında; kunduzlar, sincaplar, yer sincapları, fareler ve tavşanları içine alan aşındırıcı kemirgenler ortaya çıkmıştır; ve yakın bir zaman içinde bu canlılar, bu aile içinde çok az bir değişikliğin gerçekleştiği biçimde, yaşamın dikkate değer bir türü haline gelmiştir. Bu sürecin daha sonraki birikintileri; kökensel türleri halinde köpeklerin, kedilerin, rakunların ve sansarların fosil kalıntılarını taşımaktadır.
\vs p061 2:8 \bibemph{30.000.000} yıl önce memelilerin bugünkü türleri ortaya çıkışlarını gerçekleştirmiştir. Daha önceden memeliler dağ türleri olarak büyük ölçüde tepelerde yaşamışlardır; \bibemph{ansızın} orada, beden ile beslenen pençeli canlılardan farklılaşan bir biçimde otlayan canlılar olarak bitki beslenicilerinin veya diğer bir değişle toynak türlerinin evrimi başlamıştır. Bu otlakçılar, bu çağın sona ermesinden önce ortadan kaybolmuş bir biçimde, beş toynağa ve kırk dört dişe sahip farklılaşmamış bir atadan türemiştir. Bu süreç boyunca ayak parmaklarının evrimi, üç\hyp{}ayak\hyp{}parmak aşamasının ötesine geçmemiştir.
\vs p061 2:9 Evrimin olağanüstü bir örneği olarak at; her ne kadar ileriki buz devrine kadar bütünüyle gelişimi tamamlanmasa da, bu zaman zarfı içerisinde Kuzey Amerika ve Avrupa’da yaşamıştır. Gergedan ailesi bu dönemin sonunda ortaya çıkarken, bunun sonrasında kendilerine ait en büyük büyümeyi deneyimlemişlerdir. Yabandomuz\hyp{}vari küçük bir canlı; domuzlar, göbekli domuzlar ve hipopotamların birçok türünün atası haline gelerek, aynı zamanda bu dönemde gelişme göstermiştir. Develer ve Lamalar bu sürecin ortalarına doğru Kuzey Amerika’da ortaya çıkmış, ve bu kıtanın batı kesimlerine yayılmıştır. Daha sonra, Lamalar Güney Amerika’ya, Develer Avrupa’ya göç etmiştir; her ne kadar birkaç deve buz devrine kadar hayatta kalsa da, yakın zaman içerisinde Kuzey Amerika’da bu iki türünde nesillerinin tükendiği bir konuma gelinmiştir.
\vs p061 2:10 Yine yaklaşık olarak bu süre zarfında Kuzey Amerika’nın batı kesiminde dikkate değer bir gelişme meydana gelmiştir: tarihi lemurların öncül ataları ilk kez ortaya çıkmıştır. Her ne kadar bu aile gerçek lemurlar olarak gösterilemese de, onların ortaya çıkışı takip eden dönemlerdeki gerçek lemurların türemiş olduğu evrim hattının kuruluşunu simgelemiştir.
\vs p061 2:11 Koşulların kendilerini denize ittiği bir önceki çağın karayılanları gibi bu aşamada karınbağı memelilerin bütüncül bir kabilesi, karayı terk edip yerleşkelerini okyanuslarda kurmuştur. Ve onlar bu zaman zarfından beri; bugünün balinaları, yunusları, domuzbalıkları, ayıbalıkları, ve denizaslanlarının türemesine sebebiyet veren bir biçimde denizde kalmaya devam etmişlerdir.
\vs p061 2:12 Gezegenin kuş yaşamı gelişmeye devam etmiştir; ancak birkaç önemli evrimsel değişiklikler ile bu gelişme meydana gelmiştir. Bugünkü kuşların büyük bir çoğunluğu; martılar, balıkçıllar, flamingolar, doğanlar, şahinler, kartallar, baykuşlar, bıldırcınlar ve deve kuşlarını içine alan bir biçimde, mevcut bulunmuştu.
\vs p061 2:13 On mil yılı kaplayan bir biçimde bu \bibemph{Oligosen} döneminin sonuna doğru, deniz yaşamı ve kara hayvanları ile birlikte bitki yaşamı; büyük ölçüde evcilmiş olup, bugüne benzer bir biçimde dünya üzerinde var olmuştur. Dikkate değer özelleşme bunun sonrasında ortaya çıkmıştır; ancak yaşayan varlıkların büyük bir kısmına ait kökensel türler bu zaman zarfında hayatta bulunmaktaydı.
\usection{3.\bibnobreakspace Çağdaş Çağ Aşaması\\Fillerin ve Atların Çağı}
\vs p061 3:1 Kara yükselişi ve denizin karadan ayrışımı, kademeli olarak soğutan bir biçimde dünyanın havasını yavaşça değiştirmekteydi; ancak iklim hala ılık bir düzeyde bulunmaktaydı. Kızılağaçlar ve manolyalar Grönland’de yetişmişti; ancak bu dönence altı bitkileri, güneye doğru göç etmekteydi. Bu sürecin sonunda bahse konu sıcak\hyp{}iklim bitkileri ve ağaçları, daha sert bitkiler ve yaprak döken ağaçlar tarafından yerleri alınarak kuzey enlemlerde büyük ölçüde ortadan kayboluşlardır.
\vs p061 3:2 Çayır çeşitlerinde oldukça büyük bir artış bulunmaktaydı; ve birçok memeli türünün dişleri kademeli olarak bugünkü otçul türe uyum sağlayacak ölçüde değişime uğramıştır.
\vs p061 3:3 \bibemph{25.000.000} yıl önce kara yükselişinin uzun çağını takip eden küçük çaplı bir kara batışı meydana gelmiştir. Kayalık Dağ bölgesi, aşınan toprak birikiminin doğunun düzlükleri boyunca devam etmesine sebebiyet verecek ölçüde, oldukça yükselmiş bir konumda kalmaya devam etmiştir. Sierra Dağları ciddi bir biçimde yeniden yükselmeye uğramıştır; gerçekte onlar en başından beri yükselişlerine kesintisiz olarak devam etmekteydi. Kaliforniya bölgesi içinde dört millik büyük dikey fay kırılışı kökenini bu zaman zarfından almaktadır.
\vs p061 3:4 \bibemph{20.000.000} yıl öncesi gerçek anlamıyla memelilerin altın çağıydı. Bering Boğazı kara köprüsü su seviyesinin üstündeydi; ve dört uzun dişe sahip mastodonları, kısa bacaklı gerdanları ve kedi ailesinin birçok türünü içine alan bir biçimde hayvanların birçok topluluğu Asya’dan Kuzey Amerika’ya göç etmiştir.
\vs p061 3:5 İlk geyik ortaya çıkmış, ve Kuzey Amerika yakın bir zaman içerisinde --- geyikler, öküzler, develer, bizonlar ve gergedanların birkaç türü biçiminde --- gevişgetirenler tarafından kaplanmıştır; ancak altı fitten daha fazla uzunluğa sahip olan dev domuzların nesli tükenmiştir.
\vs p061 3:6 Bu ve daha sonraki dönemlerin dev filleri, bedenlerine ek olarak büyük ölçekteki beyinlere sahip olmuşlardır; ve onlar yakın zaman içerisinde Avustralya haricinde tüm dünyayı kaplamıştır. Bu sefer dünya, bedenini yeterli ölçüde idame ettirecek beyne sahip olan dev bir hayvanın baskın olduğu bir konumda bulunmuştur. Bu çağların yüksek ussal yaşamı ile bakımından fil büyüklüğünde hiçbir hayvan, geniş bir beyne ve üstün niteliğe sahip olmadan hayatta kalamazdı. Us ve uyum bakımından file yalnızca at yaklaşmaktadır; ve filin üstünlüğü sadece insan geçebilmektedir. Böyle bir durumda bile, bu dönemin başında var olan fillerin elli türü arasında yalnızca ikisi varlığını devam ettirebilmiştir.
\vs p061 3:7 \bibemph{15.000.000} yıl önce Avrasya’nın dağ bölgeleri yükselmekteydi; ve bu bölgeler boyunca bazı volkanik hareketler gerçekleşmişti; ancak bu hareketler hiçbir biçimde Batı Yarımküre’nin lav akıntıları ile karşılaştırılabilecek bir düzeyde bulunmamaktaydı. Bu düzensiz koşullar dünyanın tümü üzerinde varlığını sürdürdü.
\vs p061 3:8 Cebelitarık Boğazı kapandı; ve İspanya Afrika ile eski kara köprüsü vasıtasıyla bağlandı; ancak Akdeniz Atlas Okyanusu’na Fransa boyunca uzanan küçük bir kanal vasıtasıyla bağlandı; dağ tepeleri ve yükseltiler bu tarihi deniz üzerinde adalar gibi görünmekteydi. Daha sonra bu Avrupa suları çekilmeye başladı. İlerleyen zamanlarda ise Akdeniz Hint Okyanusu’na bağlanırken, bu sürecin sonunda Suez bölgesi Akdeniz’i ilk defa bir tuzlu iç deniz haline getirecek ölçüde yükselişe uğradı.
\vs p061 3:9 İzlanda kara köprüsü sular altında kaldı, ve kuzey kutup suları Atlas Okyanusu’nun suları ile birleşti. Kuzey Amerika’nın Atlas Okyanus sahili hızlı bir biçimde soğudu; ancak Büyük Okyanus sahili, bugünkünden daha sıcak olarak kalmaya devam etti. Büyük okyanus akıntıları faaliyet altında olup, bugünküne benzer bir biçimde iklimi etkilemekteydi.
\vs p061 3:10 Memeli yaşamı evirilmeye devam etti. Atların dev sürüleri, Kuzey Amerika’nın batı düzlükleri üzerinde develere katıldı; bu dönem gerçek anlamıyla, fillere ek olarak atların çağıydı. Atın beyni, hayvan ölçüsü içerisinde file çok yakındı; ancak bir nitelik bakımından onun beyni daha alt düzeydeydi, çünkü at korktuğu zaman derininde oluşan kaçma eğiliminin üstesinden hiçbir zaman gelememiştir. At, filin sahip olduğu duygusal denetime sahip değilken; fil büyüklüğü ve çevik olmayışı engeliyle kısıtlanmış bir haldeydi. Bu süreç boyunca bir hayvan, at ve filin karışımına benzer bir cinste evirilmiştir; ancak bu tür, hızla çoğalan kedi ailesi tarafından yok edilmiştir.
\vs p061 3:11 Tarafınızdan “atsız çağ” olarak adlandırılan döneme Urantia giriş yaparken, bu hayvanın atalarınız için ne anlama geldiğini bir durup düşünmelisiniz. İnsan ilk olarak atları kendi besin kaynakları olarak, daha sonra taşıma amacıyla ve bunun sonrasında ise tarımda ve savaşta kullandı. At uzun yıllar boyunca insan türüne hizmet etmiş ve insan medeniyetinin gelişiminde önemli bir rol oynamıştır.
\vs p061 3:12 Bu sürecin biyolojik gelişmeleri, insanın daha sonraki ortaya çıkışı için koşulların hazırlanması yönünde katkıda bulunmuştur. Merkezi Asya içinde, mevcut an içerisinde nesli tükenmiş ortak bir kökenden gelerek, ilkel maymunlar ve gorillaların gerçek türleri evirilmiştir. Ancak bu türlerin hiçbirinin, insan ırkının daha sonra ataları haline gelecek olan yaşayan varlıkların evrim hattı ile ilgisi bulunmamaktadır.
\vs p061 3:13 Köpek ailesi, kurtlar ve tilkiler başta olmak üzere birkaç topluluk tarafından temsil edilmiştir; panterler ve büyük kılıç dişli kaplanlar tarafından temsil edilen kedi kabilesi, daha sonra ilk olarak Kuzey Amerika’da evirilmekteydi. Çağdaş kedi ve köpek aileleri, dünyanın tümü üzerinde artış göstermişti. Gelincikler, sansarlar, su samurları ve rakunların sayıları kuzey enlemler boyunca artmış ve gelişme göstermiştir.
\vs p061 3:14 Her ne kadar birkaç ciddi değişiklik meydana gelse de kuşlar evirilmeye devam etmiştir. Sürüngenler --- yılanlar, timsahlar ve kaplumbağalar olarak --- bugünkü türlerine benzer bir konumda bulunmuşlardır.
\vs p061 3:15 Böylelikle, dünya tarihinin çarpıcı ve ilgili çekici bir süreci sona ermiştir. Fillerin ve atların bu çağı, \bibemph{Miyosen} devri olarak bilinmektedir.
\usection{4.\bibnobreakspace Yeni Kıta\hyp{}Yükseliş Aşaması\\Son Büyük Memeli Göçü}
\vs p061 4:1 Bu çağ; Kuzey Amerika, Avrupa ve Asya içindeki buzul\hyp{}öncesi kara yükselişinin dönemidir. Kara, topoğrafik anlamda büyük ölçüde değişikliğe uğramıştır. Dağ sıraları doğmuş, nehir akımlarının yönü değişmiş, ve bağımsız volkanların dünyanın tümü boyunca patlamaya başlamıştır.
\vs p061 4:2 \bibemph{10.000.000} yıl önce, geniş çaplı yerel kara birikintileri kıtaların düzlüklerinde başlamıştır; ancak bu tortullaşmaların büyük bir kısmı daha sonra ortadan kalkmıştır. Bu zaman zarfında İngiltere, Belçika ve Fransa’nın belirli kesimlerini içine alan bir biçimde Avrupa’nın büyük bir kısmı sular altında kalmıştır; ve Akdeniz kuzey Afrika’nın büyük bir kısmını kaplamıştır. Kuzey Amerika içerisinde dağ tabanlarında, göllerde ve büyük kara havzalarında geniş birikintiler oluşmuştur. Bu birikintilerin kalınlığı yaklaşık olarak yalnızca iki yüz fit kalınlığında olup, neredeyse renkli bir biçimde bulunup, fosilleri nadiren taşımaktadır. İki büyük tatlı su gölü, Kuzey Amerika’nın batı kesiminde var olmuştur. Sierra Dağları yükselmiştir; Shasta, Hood ve Rainier dağları kendilerine ait yükselti süreçlerine başlamaktaydı. Ancak bir sonraki buz devrine kadar Kuzey Amerika’nın Atlas Okyanusu’na doğru olan batışı gerçekleşmeyecektir.
\vs p061 4:3 Kısa bir süreliğine, Avustralya haricinde dünya karalarının tümü tekrar birbirine bağlanmıştır; ve bu aşamada son büyük dünya çaplı hayvan göçü gerçeklenmiştir. Kuzey Amerika, Güney Amerika ve Asya ile bağlanmıştır; ve bu kıtalar arasında bağımsız bir hayvan yaşam değiş tokuşu gerçekleşmiştir. Asya kökenli tembelhayvan cinsi, armadillolar, antiloplar ve ayılar Kuzey Amerika’ya girerken; Kuzey Amerika develeri Çin’e gitmiştir. Gergedanlar, Avustralya ve Güney Amerika dışında tüm dünyaya göç etmiştir; ancak onların nesilleri Batı Yarımküre’de bu dönemin sonuna doğru tükenmiştir.
\vs p061 4:4 Genel olarak bir önceki dönemin yaşamı evirilmeye ve yayılmaya devam etmiştir. Kedi ailesi hayvan yaşamı üzerinde baskın bir konuma gelmiş, ve deniz yaşamı neredeyse değişmez bir konumda bulunmuştur. Atların birçoğu hala üç toynaklıydı; ancak bugünkü atlar gelmeye başlamaktaydı; lamalar ve zürafa\hyp{}vari develer, otlak düzlüklerinde atlara katıldılar. Zürafa, bugünkü uzunluğundan yalnızca bir boyun daha büyük bir bedene sahip olarak, Afrika’da ortaysa çıkmıştır. Kuzey Amerika tembelhayvanları, armodilloları ve karıncayiyenlerine ek olarak ilkel maymunların Güney Amerika türü evirilmiştir. Kıtaların nihai olarak birbirlerinden kopmalarından önce, mastodonlar olarak bu devasa hayvanlar Avustralya dışında her yere göç etmiştir.
\vs p061 4:5 \bibemph{5.000.000} yıl önce at, bugünkü bedenine evcilmiş ve Kuzey Amerika’dan tüm dünyaya göç etmiştir. Ancak köken kıtası üzerinde atın nesli, kızıl insanın ortaya çıkışından çok uzun bir süre önce tükenmiştir.
\vs p061 4:6 İklim kademeli olarak soğumaktaydı; kara bitkileri yavaş bir biçimde güneye doğru hareket etmekteydi. İlk başta kuzeyde soğukluğun artışı hayân göçlerinin kuzey kanallarına doğru olan yönelimini durdurmuştur; bunu takiben bahse konu Kuzey Amerika kara köprüleri sular altında kalmıştır. Bunun sonrasında yakın bir süre zarfında Afrika ve Güney Amerika arasındaki kara bağlantısı nihai olarak batmış, ve Batı Yarımküresi bugünküne çok benzer bir biçimde bağımsız hale gelmiştir. Bu zaman zarfından itibaren yaşamın birbirinden farklı türleri Doğu ve Batı Yarımküreler içinde gelişmeye başlamıştır.
\vs p061 4:7 Ve yaklaşık olarak on milyon yıllık bir süreci kaplayan bu dönem böylece sona yaklaşmış, ve insanın hiçbir atası henüz ortaya çıkmamıştır. Bu dönem, genel olarak \bibemph{Pliyosen} devri olarak adlandırılan süreçtir.
\usection{5.\bibnobreakspace Öncül Buz Devri}
\vs p061 5:1 Bir önceki dönemin sonunda Kuzey Amerika’nın kuzeydoğu kısmına ve kuzey Avrupa’ya ait kara; Kuzey Amerika’da geniş alanların 30.000 fitten fazla bir düzeye kadar yükseldiği biçimde, geniş bir ölçekte oldukça yükselmiştir. Ilık iklimler önceden bu kuzey bölgeleri içinde baskın bir konumda bulunmaktaydı; ve kuzey buz deniz sularının tümü buharlaşmaya maruz kalmış olup, buzul sürecinin yaklaşık olarak sonuna doğru buzdan yoksun bir konumda kalmayı sürdürdü.
\vs p061 5:2 Bu kara yükseltileri ile birlikte eş zamanları olarak okyanus akıntıları yönlerini değiştirdi ve mevsimsel rüzgârların doğrultuları farklılaşmaya uğradı. Bu koşullar nihai olarak, kuzey yükseltiler üzerinde oldukça doygun atmosferin hareketinden meydana gelen neredeyse sürekli olarak gerçekleşen bir nem yağışını yarattı. Kar, bu yüksek ve dolayısıyla soğuk bölgeler üzerinde yağmaya başladı; ve 20.000 fitlik bir derinliğe kadar yağışını sürdürdü. Yükseklik ile birlikte karın en yüksek derinlikte bulunduğu bölgeler, bir sonraki buzul basınç değişimlerinin merkezi noktalarını belirledi. Ve buz devri bu aşırı miktarda gerçekleşen yağış; yakın bir zaman içinde başkalaşıp katı ama sürünen buz biçimine gelecek olan, karın devasa kabuğuyla birlikte bu kuzey yükseltilerini kaplar hale gelene kadar yeryüzüne düşmeye devam etti.
\vs p061 5:3 Bu sürecin büyük buz tabakalarının tümü, yükselmiş dağlarda konumlanmıştı; onlar, bugünkü bulundukları yerlerdeki dağ bölgelerinde oluşmamışlardı. Buzul çağının yarısı Kuzey Amerika’da, dörtte biri Avrasya’da, ve diğer çeyreği ise başlıca Güney Kutbu olmak üzere her yerde gerçekleşmiştir. Afrika buzdan çok az bir ölçüde etkilenmiştir; ancak Avustralya güney kutup buz yorganıyla neredeyse tamamen kaplanmıştır.
\vs p061 5:4 Her ne kadar her buz tabakasının hareketi ile ilişkili sayısız ilerleme ve gerileme bulunsa da, bu dünyanın kuzey bölgeleri altı ayrı ve farklı buz istilasını deneyimlemiştir. Kuzey Amerika içindeki buz, önceki iki daha sonra üçüncü merkez etrafında toplanmıştır. Grönland’ın tamamı kaplanmış, ve İzlanda bütünüyle buz akışının altına gömülmüştür. Avrupa içinde buz farklı dönemlerde; güney İngiltere dışında Britanya Adaları’nı kaplamış, ve batı Avrupa’dan Fransa’ya kadar yayılmıştır.
\vs p061 5:5 \bibemph{2.000.000} yıl önce ilk Kuzey Amerika buzulu kendisine ait güney yönlü hareketine başladı. Buz çağı bu aşamada faaliyet halinde olup, bu buzul kuzey basınç merkezlerine doğru ilerleyişini ve gerileyişini neredeyse bir milyon yıl boyunca sürdürdü. Merkezi buz tabakası güneyde Kansas’a kadar genişledi; doğu ve batı bu merkezleri bahse konu bu zaman zarfında çok geniş değildi.
\vs p061 5:6 \bibemph{1,500.000} yıl önce ilk büyük buzul kuzey doğrultuda geri çekilmekteydi. Bu zaman zarfında devasa kar yağışları Grönland’e ve Kuzey Amerika’nın kuzeybatı kesimlerine düşmekteydi; ve yakın bir süre zarfında bu doğu buz kütlesi güneye doğru kaymaya başlamıştı. Bahse konu bu oluşumlar, buzun ikinci istila dönemidir.
\vs p061 5:7 Bu ilk iki buz istilası Avrasya’da çok geniş ölçekte gerçekleşmemekteydi. Buz devrinin bu öncül çağları boyunca Kuzey Amerika; mastodonlar, yünlü mamutlar, atlar, develer, geyikler, misk öküzleri, bizonlar, yeraltı tembelhayvanları, dev kunduzlar, kılıç dişli kaplanlar, filler kadar büyük yerüstü tembelhayvanlarına ek olarak kedi ve köpek ailelerinin birçok topluluğu ile kaplanmıştı. Ancak bu zaman zarfından beri onlar, buz döneminin artan soğukluğu nedeniyle sayı bakımından hızlı bir biçimde azalma göstermiştir. Buz devrinin sonuna doğru bu hayvan türlerinin büyük bir çoğunluğunun nesli Kuzey Amerika’da tükenmiştir.
\vs p061 5:8 Buz dışında dünyanın kara ve su yaşamı çok az değişiklik göstermiştir. Buz istilaları arasında iklim mevcut andaki gibi ılıman, muhtemelen biraz daha sıcaktı. Her ne kadar devasa alanları kaplayacak ölçüde yayılmış olsalar da buzullar nihayeten yerel olgulardı. Kıyı çevresi iklimi; buzul hareketsizlik zamanları arasında oldukça büyük ölçüde çeşitlilik göstermiştir; ve bu zamanlar devasa buz dağlarının, Maine sahilinden Atlas Okyanusu’na doğru kaydığı, Puget Sound’dan Büyük Okyanus’a sürüklendiği, ve Norveç fiyortlarını Kuzey Denizi’ne doğru ittiği zamanlarıdır.
\usection{6.\bibnobreakspace Buz Devri’nde İlkel İnsan}
\vs p061 6:1 Bu buzul döneminin en büyük etkinliği, ilkel insanın evrimidir. Hindistan’ın biraz batısına doğru şu an su altında bulunan kara üzerinde ve daha eski Kuzey Amerika lemur türlerine ait Asya göçmenlerinin doğumları arasında ilk memeliler \bibemph{ansızın} ortaya çıktı. Bu küçük hayvanlar büyük bir çoğunlukla arka ayakları üzerinde yürümüşlerdir; ve onlar, bedenlerine ve diğer hayvanların beyinlerine kıyasla büyük beyinler taşıdılar. Yaşamın bu türünün yedinci neslinde, yeni ve daha yüksek bir hayvan topluluğu \bibemph{ansızın} farklılaşma gösterdi. Atalarının neredeyse iki katı uzunluğa ve genişliğe sahip ve orantısal olarak artış göstermiş beyin gücünü elinde bulunduran bir biçimde bu yeni yarı\hyp{}memeliler kendi oluşumlarını daha yeni yerine getirmişken, üçüncü hayati başkalaşım olarak Primatlar \bibemph{ansızın} ortaya çıkmıştır. (Yine bu zaman aralığında, maymun soyuna kökenini veren yarı\hyp{}memeli nesil kolu içinde geri yönde bir gelişim meydana gelmiştir; ve bu zaman zarfından bugüne kadar insan kolu ilerleyici bir evrim gösterirken, maymun kabileleri sabit kalmış veya gerçek anlamda geriye gitmiştir.)
\vs p061 6:2 \bibemph{1.000.000} yıl önce Urantia \bibemph{ikamet edilmiş} bir dünya olarak kaydedilmiştir. İlerleyen Primatlar’ın nesil kolu içinde bir başkalaşım \bibemph{ansızın}, insan türünün mevcut ataları olan iki ilkel insanı dünyaya getirmiştir.
\vs p061 6:3 Bu gelişim yaklaşık olarak üçüncü buzul ilerleyişinin başlayış zamanında açığa çıkmıştır; böylelikle sizin öncül atalarınızın ilgi çekici, güçlendirici ve zor bir çevre içinde dünyaya geldiği ve yetiştiği gözlenebilir. Ve Eskimolar olarak bu Urantia aborjinleri mevcut zaman zarfında bile dondurucu kuzey iklimlerinde yaşamayı tercih etmektedirler.
\vs p061 6:4 İnsan varlıkları, buz devrinin sonlanmasına yakın Batı Yarımküre içinde bulunmamaktaydı. Ancak bu buzul\hyp{}içi\hyp{}dönem çağları boyunca onlar Akdeniz etrafından batıya doğru geçip, yakın zaman içerisinde Avrupa’ya yayıldılar. Batı Avrupa’nın mağaralarında; ilerleyen ve geri çekilen buzulların daha sonraki çağları boyunca bu bölgeler içerisinde insanın yaşadığını kanıtlar nitelikteki, sıcak iklim ve kutup hayvanlarının kalıntıları ile birleşmiş insan kemikleri bulunabilir.
\usection{7.\bibnobreakspace Devam Eden Buzul Çağı}
\vs p061 7:1 Buzul dönemi boyunca diğer etkinlikler faaliyet içerisindeydi; ancak buzun faaliyeti, kuzey enlemlerinde tüm diğer oluşumları gölgelemekteydi. Hiçbir kara hareketi bu türden özel bir kalıntıyı topoğrafya üzerinde bırakmamaktadır. Kaya çukuru, göller, yerinden uzaklaşmış kaya ve kaya unu gibi örneklerdeki ayırt edici kaya parçaları ve kabuk kırılmaları, buzullar haricinde doğa içerisindeki hiçbir oluşum sonucunda meydana gelmemektedir. Buz aynı zamanda, drumlin olarak bilinen küçük yüzey kabartılarından veya diğer bir değişle kabuk kıvrımlarından da sorumludur. Ve bir buzul ilerlerken nehirleri yerlerinden alarak dünyanın bütün yüzeyini değiştirmektedir. Tek başları buzullar --- yüzey, ensi ve dönemsel biçimlerdeki buzul taşları olarak --- bu belirteç kıvrımları bırakmaktadırlar. Özellikle yüzey buzul taşları olarak bu kıvrımlar; Kuzey Amerika’nın kuzey ve batı kesimlerine doğru batı kıyı havzasından uzanmakta olup, Avrupa ve Sibirya’da bulunabilirler.
\vs p061 7:2 \bibemph{750.000} yıl önce Kuzey Amerika merkez ve doğu buz alanlarının bir birliği olarak dördüncü buz tabakası, güneye doğru kararlı bir doğrultuda hareket etmekteydi; zirve noktasında bu tabaka Mississippi Nehri’ni elli mil batıya kaydırarak güney Illinois’ye ulaşmış olup, doğuda en fazla Ohio Nehri ve Pennsylvania’nın merkezine uzanmıştır.
\vs p061 7:3 Asya içinde Sibirya buz tabakası kendisinin en güneydeki istilasını gerçekleştirirken, Avrupa içinde ilerleyen buz Alpler’in dağ duvarında takılı kalmıştır.
\vs p061 7:4 \bibemph{500.000} yıl önce, buzun beşinci ilerleyişi boyunca, yeni bir gelişim insan evriminin ilerleyişini hızlandırmıştır. \bibemph{Ansızın}, ve bir nesil içerisinde, altı renk ırk yerli insan nesil kolundan başkalaşıma uğramıştır. Bu gelişim iki kat öneme sahiptir, çünkü o aynı zamanda Gezegensel Prens’in varışını simgelemektedir.
\vs p061 7:5 Kuzey Amerika içinde ilerleyiş halindeki beşinci buzul, üç buz merkezinin bir araya gelen bir istilasından kaynaklanmıştır. Doğu kol, buna rağmen, St. Lawrence vadisinin yalnızca küçük bir seviye altına kadar genişleme göstermiştir; ve batı buz tabakası, güney ilerleyişinde çok az mesafe kat edebilmiştir. Ancak merkezi kol, Iowa Eyaleti’nin tamamını kaplayacak bir biçimde güneye ulaşmıştır. Avrupa içinde buzun bu istilası, bir önceki kadar geniş çaplı değildi.
\vs p061 7:6 \bibemph{250.000} yıl önce altıncı ve son buzul hareketi başlamıştır. Ve kuzey yükseltilerin az miktardaki batışına rağmen bu gelişim, kuzey buz bölgeleri üzerinde en büyük kar birikintisinin yaşandığı dönem olmuştur.
\vs p061 7:7 Bu istila içinde bahse konu üç büyük buz tabakası, tek bir geniş buz kütlesi içinde kümelenmiştir; ve batı dağlarının tümü bu buzul etkinliğine katılmıştır. Bu oluşum, Kuzey Amerika içindeki buz istilalarının tümü içinde en büyük olanıydı; buz basınç merkezinden bin beş yüz milden fazla bir uzaklıkta doğuya doğru hareket etmiş, ve Kuzey Amerika sahip olduğu en düşük sıcaklıkları deneyimlemiştir.
\vs p061 7:8 \bibemph{200.000} yıl önce bu son buzulun ilerleyişi sırasında, --- Lucifer isyanı olarak --- Urantia üzerinde gerçekleşen etkilikler ile oldukça iniltili bir gelişme yaşanmıştır.
\vs p061 7:9 \bibemph{150.000} yıl önce altıncı ve son buzul tabakası, batı bu tabakasının Kanada sınırının tam üzerinden geçerek, güney doğrultudaki en geniş sınırlarına ulaşmıştır; merkezi buz tabakası Kansas, Missouri ve Illinois’nin güneyine doğru ilerlemiş; doğu buz tabakası güneye doğru genişlemiş, Pennsylvania ve Ohio’nun geniş kesimlerini kaplamıştır.
\vs p061 7:10 Bu oluşum; büyük ve küçük ölçekli bugünkü gölleri şekillendiren, birçok buz dil\hyp{}uzantısının veya diğer bir değişle buz kolunun oluşumuna sebebiyet veren buzuldur. Onun çekilişi sırasında Büyük Göller bölgesinin Kuzey Amerika sistemi meydana gelmiştir. Ve Urantia yeryüzü bilimcileri; oldukça doğru bir biçimde bu gelişmenin çeşitli aşamalarını tespit edip, bu su kütlelerin farklı zamanlarda Mississippi vadisine, daha sonra doğuya doğru Hudson vadisine ilerleyip son olarak bir kuzey bağlantısı vasıtasıyla St. Lawrence yerleşkesine boşaldığının doğru çıkarımında bulundular. Birbiri ile bağlantılı Büyük Göller’in bugünkü Niyagara hattı üzerinden boşalmaya ilk olarak başlaması sürecinden otuz yedi bin yıl geçmiştir.
\vs p061 7:11 \bibemph{100.000} yıl önce bu son buzulun çekiliş süreci boyunca, geniş kutup buz tabakaları oluşmaya başlamıştır; ve buz birikiminin merkezi ciddi bir ölçüde kuzeye doğru hareket etmiştir. Ve kutup bölgeleri buz ile kaplanmaya devam ettikçe, gelecekte meydana gelecek kara yükselişlerinden veya okyanus akıntılarının değişimlerinden bağımsız bir biçimde, başka bir buzul çağının meydana gelmesi neredeyse olanaksızdır.
\vs p061 7:12 Bu son buzulun ilerleyişi yüz bin yıl sürmüştür; ve onun kuzey çekilişi için benzer bir süre zarfının geçmesi gerekmiştir. Ilıman bölgeler, elli bin yıldan biraz daha fazla süren bir süre zarfı boyunca hiçbir buz oluşumunu taşımamışlardır.
\vs p061 7:13 Bu ciddi buzul dönemi birçok türü ortadan kaldırmış olup, köklü bir biçimde onların diğer sayısız türünü değişikliğe uğratmıştır. Birçoğu, ilerleyen ve geri çekilen buzun zorundalık yarattığı göç süreçlerinin gidiş ve dönüş istikametlerinde acı bir biçimde telef olmuşlardır. Buzulların ilerleme ve gerileme oluşumlarını kara üzerinde takip eden hayvanlar; ayılar, bizonlar, rengeyikleri, misk öküzleri, mamutlar ve mastodonlar olmuştur.
\vs p061 7:14 Mamutlar açık çayırları tercih etmiş olup, bunun karşısında mastodonlar orman bölgelerinin korunaklı sınırlarını aramaya koyulmuşlardır. Geç bir zaman sürecine kadar mamutlar, Meksika’dan Kanada’ya kadar uzanmaktaydı; Sibirya türü yün ile kaplanmıştı. Mastodonlar Kuzey Amerika’da, kızıl ırk tarafından katledilene kadar varlığını sürdürmüş olup, benzer bir biçimde bizonlar beyaz ırkın bu nesli tüketişine kadar var olmuşlardır.
\vs p061 7:15 Kuzey Amerika içinde son buzul oluşumu boyunca; atlar, tapirler, lamalar, kılıç dişli kaplanların nesli tükenmiştir. Onların yerlerini Güney Amerika’dan gelen tembelhayvanları, armadillolar ve yaban su domuzları almıştır.
\vs p061 7:16 İlerleyen buzun öncesinde yaşamın zorunlu göçü, bitkiler ve hayvanların olağanüstü bir birleşimine sebebiyet vermiştir; ve buzun son istilasından çekilişiyle birlikte bitkilerin ve hayvanların birçok kutup türü, buzulun tahribatından kurtulmak için neresi elverişli olursa oraya kaçan bir biçimde, belirli dağ tepeleri üzerinde sıkışmış bir konumda kalmışlardır. Ve bu nedenle bahse konu yerlerinden edilmiş bitkiler ve hayvanlar mevcut an içerisinde, Avrupa Alpleri’nin tepelerinde ve hatta Kuzey Amerika’nın Appalachian Dağları’nda bile bulunabilir.
\vs p061 7:17 Bu buz devri, iki milyon yıldan daha fazla süren bir zaman aralığını kaplayan bir biçimde, tarafınızdan \bibemph{Pleistosen} devri isminde adlandırılmakta olan, yeryüzü sürecinin tamamlanmış en son zaman zarfıdır.
\vs p061 7:18 \bibemph{35.000} yıl öncesi, gezegenin kutup bölgeleri dışında büyük buz devrinin sonlanışını simgelemektedir. Bu zaman zarfı aynı zamanda; \bibemph{Holosen} veya diğer bir değişle buzul\hyp{}sonrası dönemin başına yaklaşık olarak denk düşen, bir Maddi Erkek ve Kız Evlat’ın varışına ve Âdemsel yazgı döneminin başlangıcına yaklaşması bakımından önemlidir.
\vs p061 7:19 Memeli yaşamının yükselişinden buzun çekilişine ve ilkel çağların sonuna kadar uzanan bu anlatım, yaklaşık olarak elli milyon yıllık bir zaman zarfını kaplamaktadır. Bu devir en son --- ve şimdiki --- yeryüzü dönemi olup, araştırmacılarınız tarafından \bibemph{Senozoyik} veya mevcut yeryüzü devri olarak bilinmektedir.
\vs p061 7:20 [Bu anlatım, bir Yerleşik Yaşam Taşıyıcısı tarafından sunulmuştur.]
