\upaper{82}{Evliliğin Evrimi}
\vs p082 0:1 Evlilik --- çiftleşme olarak --- iki karşıt cinsiyet doğasından türemektedir. Evlilik, insanın bu türden bir cinsel doğaya olan verdiği tepkisel uyumken, aile yaşamı bu türden evrimsel ve uyumsal düzenlemelerin tümünün ortaya çıkardığı bütüncül bir sonuçtur. Evlilik varlığını sürdürmektedir: biyolojik evrim içerisinde içkin bir nitelikte bulunmamaktadır, ancak toplumsal evrimin tamamının temeli ve bu nedenle onun bir bütünlük içinde devam eden mevcudiyetinin ana kaynağıdır. Evlilik insan türüne ev kurumunu kazandırmıştır, ve ev uzun ve çetin evrimsel mücadelenin tamamının en yüksek övünç kaynağıdır.
\vs p082 0:2 Her ne kadar dini, toplumsal ve eğitimin kurumlarının tümü kültürel medeniyetin varlığını sürdürebilmesi için hayati derecede öneme sahip olsa da, \bibemph{aile en üstün uygarlaştırıcıdır}. Bir çocuk, hayata dair temel bilgilerin çoğunu ailesinden ve komşularından öğrenmektedir.
\vs p082 0:3 Eski dönemlerin insanları, oldukça zengin bir toplumsal medeniyete sahip değillerdi; ancak sahip oldukları şeyleri aslına bağlı kalarak ve etkin bir biçimde bir sonraki nesle aktardılar. Ve sizler; geçmişe ait bu medeniyetlerin çoğunun, ev kurumunun etkin bir biçimde faaliyet göstermesi sonucunda diğer kurumsal etkilerin yeterli en alt düzeyiyle birlikte evirilmeye devam ettiğini görmelisiniz. Bugün insan ırkları, zengin bir toplumsal ve kültürel mirasa sahiptirler; ve bu miras, bilge ve etkin bir biçimde sonraki nesillere aktarılmalıdır. Bir eğitim kurumu olarak ailenin muhafaza edilmesi zorunludur.
\usection{1.\bibnobreakspace Çiftleşme İçgüdüsü}
\vs p082 1:1 Erkeler ile kadınlar arasındaki kişilik uçurumuna rağmen cinsel dürtü, türlerini dünyaya getirmek için bir araya gelmelerini teminat altına alacak kadar yeterlidir. Bu içgüdü, insanların aşk, bağlılık ve aile sadakati olarak sonradan adlandırılan niteliklerin çoğunu deneyimlemelerinden uzun bir süre önce etkin bir biçimde faaliyet göstermişti. Çiftleşme içkin bir eğilim, aile kurumu ise onun evrimsel gerçekleşen toplumsal sonucudur.
\vs p082 1:2 Cinsel ilgi ve arzu, ilkel insan topluluklarının baskın tutkuları değildi; yalın bir değişle, onlar bu dürtüleri önemsemedi. Üremeyi sağlayan deneyimin bütünü yaratıcı güzelleştirmeden yoksundu. Daha yüksek medeniyete ait insan topluluklarının sahip oldukları bütünüyle odaklaşmış cinsel arzu; evrimsel doğanın Nod ve Âdem unsurlarının ilişkilendirici hayal güçleri ve güzellik takdirleri ile değişikliğe uğradığı alanlar başta olmak üzere, temelde ırk karışımlarından kaynağını almaktadır. Ancak bu And kalıtımı; hayvan kökenli arzular üzerinde bireyin öz denetimini yeterli ölçüde sağlamasında başarısız olmasına neden olacak bir biçimde kısıtlı ölçekte karışmış, bunun sonucunda daha keskin cinsel bilincin ve daha güçlü çiftleşme arzusunun kazanılmasıyla hemen harekete geçirilen ve uyarılan hale gelmişlerdi. Evrimsel ırklar arasında kırmızı insanlar en üstün cinsel yasaya sahip topluluktu.
\vs p082 1:3 Cinsel ilişkinin evlilik ile ilgili düzenlenişi şunlara işaret etmektedir:
\vs p082 1:4 1.\bibnobreakspace Medeniyetin görece ilerleyişi. Medeniyet artan bir biçimde, cinsel arzunun yararlı biçimler içerisinde ve ahlak kuralları uyarınca tatmin edilişini talep etmektedir.
\vs p082 1:5 2.\bibnobreakspace Herhangi bir topluluk içerisindeki And ırk kökeninin düzeyi. Bu türden topluluklar içinde cinsel ilişki, hem fiziksel hem de duygusal doğalar bakımından hem en yüksek hem de en düşük düzeyleri belli eden hale gelmiştir.
\vs p082 1:6 Sang toplulukları olağan düzeyde bulunan hayvan kökenli arzuya sahiptiler; ancak onlar çok az düzeyde, karşı cinsin güzelliği ve fiziksel çekiciliğinin takdirini ve hayal gücünü sergilediler. Cinsel cazibe olarak adlandırılan nitelik, bugünün ilkel kabilelerinde bile neredeyse hiçbir biçimde bulunmamaktadır; bu karışmamış insan toplulukları kesin bir çiftleşme içgüdüsüne sahiplerdir; ancak yetersiz cinsel cazibe, toplumsal denetimi gerektiren ciddi sorunları yaratmaktadır.
\vs p082 1:7 Çiftleşme içgüdüsü, insan varlıklarının yönlendirici bir biçimde harekete geçirici en baskın fiziksel kuvvetlerinden biridir; bireyin tatmini kılıfı altında o, ırk refahını ve onun devamlılığını bireyin rahatı ve onun bireysel sorumluluktan bağımsızlığının çok üzerinde bir yere koymasında bencil insanı etkin bir biçimde kandıran bir duygudur.
\vs p082 1:8 Bir kurum olarak evlilik, ilk başlangıç günlerinden çağdaş dönemlere kadar, birey\hyp{}devamlılığı için biyolojik eğilimin toplumsal evrimini temsil etmektedir. Evirilen insan varlıklarının devamlılığı, geniş bir ölçüde cinsel çekim olarak adlandırılan bir dürtü olarak, bu ırksal çiftleşme uyarımının mevcudiyeti tarafından mümkün kılınmıştır. Bu büyük biyolojik dürtü, --- fiziksel, ussal, ahlaki ve toplumsal olarak --- ilgili içgüdüler, duygular ve ortak uygulamaların tüm çeşitleri için uyarım merkezi haline gelmektedir.
\vs p082 1:9 Yaban yaşamında yiyecek temini, harekete geçiren amaçtı; ancak medeniyet bol miktarda yiyeceği teminat altında alınca cinsel dürtü başat bir uyarım haline gelmekte ve böylece sürekli bir biçimde toplumsal denetim ihtiyacını gerektirmektedir. Hayvanlarda içgüdüsel dönemlilik çiftleşme eğilimini denetlemektedir; ancak insan oldukça fazla bir biçimde öz denetimsel varlık olduğu için cinsel arzu tamamiyle dönemsel değildir; bu nedenle toplumun birey üzerinde öz denetim zorunluluğunu kılması gerekli hale gelmektedir.
\vs p082 1:10 Hiçbir insan duygusu veya dürtüsü, dizginlenmediğinde ve haddinden fazla teslim olunduğunda bu güçlü cinsel uyarım kadar büyük zarara ve üzüntüye sebep olamaz. Bu uyarımın toplum düzenlemelerine olan ussal uyumu, her medeniyetin mevcut düzeyini gösteren en büyük sınavdır. Gittikçe kapsamı ve yoğunluğu genişleyen özdenetim, gelişen insan türünün sürekli bir biçimde artan talebidir. Gizlilik, samimiyetsizlik ve iki yüzlülük cinsellikten doğan sorunların üstünü örtebilir; ancak onlar ne çözümleri beraberinde getirir, ne de etik kuralları geliştirir.
\usection{2.\bibnobreakspace Kısıtlayıcı Tabular}
\vs p082 2:1 Evliliğin evrim hikâyesi, yalın bir değişle; toplumsal, dini ve kamu kısıtlamalarının baskısıyla gerçekleşmiş cinsel ilişki denetimi tarihidir. Doğa neredeyse hiçbir biçimde bireyleri tanımamaktadır; onun için ahlaki değerler olarak adlandırdığınız nitelikler umurunda değildir; doğa, yalnızca ve özellikle, türlerin çoğalımı ile ilgilenmektedir. Doğa; zorlayıcı bir biçimde çoğalımda ısrar etmektedir, ancak bunun sonrasında açığa çıkan sorunları çözülmesi için topluma aldırış etmeden devretmekte, böylece evrimsel insan türü için sürekli mevcut ve büyük bir sorunu yaratmaktadır. Bu toplumsal çatışma, temel içgüdüler ile evirilen etik kurallar arasında gerçekleşen sonu gelmez savaştır.
\vs p082 2:2 Öncül ırklar arasında, cinsel türlerin ilişkileri için neredeyse hiçbir düzenlenme bulunmamaktaydı. Bugün, Pigme toplulukları ve diğer geri kalmış birliktelikler hiçbir evlilik kurumuna sahip değillerdir; bu topluluklar üzerinde gerçekleştirilecek bir araştırma, ilkel topluluklar tarafından uygulanan basit çiftleşme adaletlerini açığa çıkarmaktadır. Ancak eskiçağ topluluklarının tümü her zaman, kendi dönemlerinin adetlerine ait ahlaki ölçütlerinin ışığında irdelenmeli ve yargılanmalıdır.
\vs p082 2:3 Evliliksiz sürdürülen birlikteliğe, ilkel insan düzeyi ölçeğinin üstünde hiçbir zaman iyi gözle bakılmamaktaydı. Toplumsal birlikteliklerin oluşmaya başladığı an, aile kuralları ve evlilik sınırlamaları gelişmeye başladı. Çiftleşme böylelikle; neredeyse bütüncül cinsel ehliyet düzeyinden, göreceli olarak bütüncül cinsel ilişki kısıtlamasının yirminci yüzyıl ortak ölçütlerine birçoklu geçişler süreci boyunca ilerledi.
\vs p082 2:4 Kabilesel gelişmenin en ilk aşamalarında adetler ve sınırlayıcı tabular oldukça ilkeldi; ancak onlar, kadın ve erkekleri birbirinden ayırmışlardı, --- bu durum huzuru, düzeni ve üretimi desteklemişken --- evlilik ve ev kurumunun uzun evrimi başlamış oldu. Giyim, kuşam ve dinsel uygulamalara ait cinsel adetler; cinsel özgürlükler sınırını belirleyen ve böylece nihai olarak kötülük, suç ve günah kavramlarını yaratan bahse konu öncül tabulardan kaynağını almaktaydı. Ancak, özellikle Bahar Bayramı günü gerçekleştirilen, kendinden geçiren şenlik günlerinde tüm cinsel düzenlemelerin askıya alınması uzun bir süre boyunca kullanılan uygulamaydı.
\vs p082 2:5 Kadınlar her zaman erkeklere nazaran daha sınırlayıcı tabulara maruz kalmışlardır. Öncül adetler, evlenmemiş kadınlara erkeklere verileninkine eşit düzeydeki cinsel özgürlüğü sağlamıştı; ancak kadınların eşlerine sadık olmaları her zaman şart koşulan bir durum olmuştur. İlkel evlilik, erkeğin cinsel özgürlüklerini çok fazla kısıtlamamıştır; ancak kadın eşe ilave cinsel ehliyet tabusu dayatmıştır. Evlenmiş kadınlar her zaman; saç şekli, kıyafet, örtü, ayrı mekânları kullanma, süsleme ve yüzükler tarafından başlı başına farklı bir sınıf olarak kendilerini ayıran birtakım işaretleri taşımışlardır.
\usection{3.\bibnobreakspace Öncül Evlilik Adetleri}
\vs p082 3:1 Evlilik, --- bireyin kendi devamlılığı olarak --- insanın aralıksız ortaya çıkan çoğalım dürtüsünün yarattığı her daim mevcut biyolojik gerilime toplum işleyiş bütünlülüğünün verdiği kurumsal cevaptır. Çiftleşme evrensel bir biçimde doğaldır; ve toplum basit düzeyden katmanlaşmış bütünlüğe evrilirken, evlilik kurumunun doğumu olan çiftleşme adetlerinin ilgili bir evrimi bulunmaktaydı. Toplumsal evrim her ne zaman adetlerin yaratıldığı düzeye ilerlerse, evlilik evirilmekte olan bir kurum halinde bulunacaktır.
\vs p082 3:2 Evliliğin her zaman iki farklı alanı bulunmuş olup, onlar ileride de varlıklarını bu biçimde sürdüreceklerdir: bunlardan ilki çiftleşmenin dış yönlerini düzenleyen yasalar olarak adetler, diğeri ise kadınlar ve erkekler arasındaki geride kalan gizli ve bireysel ilişkilerdir. Birey her zaman, toplum tarafından zorunlu kılınan cinsel düzenlemelere karşı isyankâr olmuştur; ve bu durum çağlar boyu süre gelen şu cinsel sorunun kaynağını teşkil etmektedir: bireyin kendisini tek başına idare edişi kişisel olarak gerçekleştirilir, ancak topluluk tarafından beraberce uygulanmalıdır; bireyin kendi devamlılığını sağlaması toplumsaldır, ancak bireysel uyarım tarafından teminat altına alınmalıdır.
\vs p082 3:3 Adetler, onlara saygı duyulduklarında, tüm ırklar arasında sergilendiği gibi cinsel dürtüyü kısıtlama ve onu denetim altına almada büyük güce sahiptir. Evliliğin ortak ölçütleri her zaman, adetlerin mevcut gücünün ve toplum idaresinin sahip olduğu işleyişsel doğruluğun gerçek bir göstergesi olmuştur. Ancak öncül cinsel ilişki ve çiftleşme adetleri, bir tutarsız ve ilkel düzenlemeler bütünüydü. Ebeveynler, çocuklar, akrabalar ve toplumun tümü, evlilik düzenlemelerinde çatışan çıkarlara sahipti. Ancak tüm bunlara rağmen, evliliği yücelten ve onu uygulayan ırklar daha yüksek düzeylere evirilmiş olup artan sayılarda yaşam savaşından galip çıktılar.
\vs p082 3:4 İlkel dönemlerde evlilik, toplumsal saygınlığın bir emaresiydi; bir kadın eşe sahip olmak bir ayrıcalık nişanıydı. İlkel insanlar evlilik günlerini, sorumluluğa ve erkekliğe olan ilk adımlarını simgeleyen milat olarak gördüler. Bir çağda evlilik toplumsal bir görev; ötekisinde bir dini ödev; ve bir diğerinde ise devlete vatandaşlar kazandıran bir toplumsal zorunluluk olarak değerlendirilmektedir.
\vs p082 3:5 Birçok öncül kabile, bir evlilik yeterliliğini elde etmek için çalma hünerlerini şart koydu; diğer topluluklar ise bunun yerine saldırgan yağmaları, atletizm yarışmalarını ve rekabete dayalı oyunları getirdi. Bu yarışmaların galiplerine, mevsimin gelinleri arasında tercihte bulunma olan birincilik ödülü verilmekteydi. Kafa\hyp{}avcıları topluluğu içinde bir genç, her ne kadar zaman zaman satın alınabilir olsalar da, en az bir kafaya sahip olmadığını müddetçe evlenememekteydi. Kadın eşlerin satın alınması azalma gösterdikçe onlar, siyah ırkın birçok topluluğu arasında hala varlığını sürdürmekte olan bir uygulama halindeki bilmece yarışmalarıyla kazanılmaktalardı.
\vs p082 3:6 Gelişen medeniyet ile birlikte belirli kabileler, erkek dayanıklılığını ölçen ciddi evlilik sınavlarını kadınların idaresine vermişlerdi; böylelikle onlar tercih ettikleri erkekleri seçebilmeye yetkin haldelerdi. Bu evlilik sınavları; avcılıkta ve kavgada hüner ve bir aileyi destekleme kabiliyetiydi. Damat uzun bir süre boyunca, en az bir yıllık gelinin ailesine katılmakla ve orada arzuladığı kadın eşe layık olduğunu ispat edecek bir biçimde yaşamakla ve çalışmakla yükümlüydü.
\vs p082 3:7 Bir kadının vasıfları ağır işin üstesinden gelme ve çocuklara bakabilme yetisiydi. O, belirli bir zaman aralığı içinde çiftçilik görevinin bir parçasını yerine getirmekle yükümlüydü. Ve o evlilik öncesinde bir çocuk dünyaya getirmişse, çok daha fazla değerliydi; onun verimliliği böylelikle kanıtlanmış haldeydi.
\vs p082 3:8 İlkçağ insan topluluklarının evlenilmemeyi bir utanç, ve hatta bir günah olarak, değerlendirme gerçeği çocuk evliliklerinin kökenini açıklamaktadır; birey evlenmek zorunda olduğunu için, ne kadar erken olursa o kadar iyiydi. Evlenmemiş bireylerin ruh\hyp{}yerleşkesine girememesi aynı zamanda yaygın bir inanıştı; ve bu inanış doğum anında bile, ve zaman zaman, doğacak çocuğun cinsiyetine bağlı olarak, doğumdan önce çocuk evliliklerinin gerçekleştirilmesi için ilave bir teşvikti. İlkçağ insanları ölülerin bile evlenmek zorunda olduklarına inandılar. Özgün çöpçatanlar, merhum bireyler için evliliklerde arabuluculuk yapması amacıyla kullanılmıştı. Bir ebeveyn, diğer ailenin ölü kızıyla kendi yitirdiği oğlunun evliliğini gerçekleştirmesi için bu aracılara başvururdu.
\vs p082 3:9 Daha sonraki insan toplulukları arasında ergenlik, evliliğin ortak yaşıydı; ancak bu durum, medeniyetin ilerlemesiyle doğru orantılı olarak gelişti. Toplumsal evrimin başında erkek ve kadınların çoğunluktan ayrılan, bekâr düzeyleri ortaya çıktı; bu topluluklar neredeyse hiçbir biçimde olağan cinsel yarıma sahip olmayan bireyler tarafından oluşturuldu ve idare edildi.
\vs p082 3:10 Birçok kabile, yönetim topluluğu üyelerinin erkek eşine verilmesinden hemen önce gelinle cinsel ilişkiye girmesine izin verdi. Bu erkeklerin her biri evlenecek genç kadına bir hediye vermekte olup, bu adet evlilik hediyeleri verme geleneğinin başlangıcıydı. Bazı topluluklar arasında genç bir kızın çeyiz parası kazanması beklenmekteydi; bu beklenti, gelinin sergi salonunda cinsel hizmet karşılığında ödül olarak hediyeleri kazanmasından meydana gelmekteydi.
\vs p082 3:11 Bazı kabileler, genç erkekleri dullar ve yaşlı kadınlar ile evlendirmişti; ve bunun sonrasında bahse konu eşler ileride kocalarını yalnız bıraktıklarında, onların genç kızlar ile evlenmeleri mümkün hale gelmekteydi; bu durum, ifade ettikleri gibi, iki ebeveyninde olgunlaşmasını teminat altına almaktaydı; çünkü onlar iki gencin doğrudan bir biçimde evlenmesine izin verilmesinin bunun aksi bir durumu beraberine getireceğini düşünmektelerdi. Diğer kabileler çiftleşmeyi benzer yaş topluluklarıyla sınırladı. Belirli yaş topluluklarıyla evlenmeye dair sınırlılık öncül bir biçimde, akrabalar arasında yapılan cinsel ilişki hakkındaki düşünceleri doğurmuştu. (Hindistan içinde bugün bile evlilik için hiçbir yaş sınırlaması bulunmamaktadır.)
\vs p082 3:12 Belirli adetler içinde dulluk fazlasıyla korkulan bir şeydi; dullar ya öldürülmekteydi veya erkek eşlerinin mezarlarında intihar etmelerine izin verilmekteydi; bu durumun nedeni, onların eşleri ile birlikte ruh\hyp{}yerleşkesine gideceklerinin düşünülmesiydi. Hayatta kalan kadın dul eş neredeyse hiç değişmeyen bir biçimde kocasının ölümünden suçlanmaktaydı. Bazı kabileler bu dul eşleri diri diri yakmışlardı. Eğer bir dulun yaşamasına izin verilirse, onun yaşamı sürekli bir elem ve katlanılmaz toplumsal kısıtlama hayatından biri olurdu; çünkü yeniden evlenme çoğunlukla uygun görülmemekteydi.
\vs p082 3:13 Şimdi ahlaksızlık olarak görülen birçok uygulama eski dönemlerde teşvik edilmekteydi. İlkel dönemdeki kadın eşler hiç de nadir bir biçimde gerçekleşmeyen bir biçimde, kocalarının diğer kadınlar ile olan evlilik dışı ilişkilerinden büyük gurur duymuşlardı. Genç kadınlardaki iffet evlilik üzerinde büyük bir kısıtlamaydı; evlilik öncesi bir çocuğa sahip olma, bir genç kadının bir eş olarak arzu edilebilirliğini fazlasıyla artırmaktaydı; çünkü erkek doğurgan bir eşe sahip olduğundan emin olmaktaydı.
\vs p082 3:14 Birçok ilkel kabile; kadın hamile kalana kadar, olağan evlilik töreni gerçekleştirilinceye kadar, deneme evliliğini zorunlu kıldı; diğer topluluklar arasında evlilik ilk çocuğun doğumuna kadar kutlanılmamaktaydı. Eğer bir kadın eş çocuk veremez bir durumda ise, ebeveynleri tarafından onun yetkin olmayışı telafi edilmek durumunda olup, evlilik iptal edilirdi. Adetler, her çiftin çocuklara sahip oluşunu zorunlu kılmaktaydı.
\vs p082 3:15 Bu ilkel deneme evlilikleri, ehliyete benzeyen her türden bütünüyle uzaktı; onlar yalın bir değişle, samimi doğurganlık denemeleriydi. Doğurganlık sağlanır sağlanmaz sözleşen bireyler kalıcı bir süreliğine evlenmektelerdi. Çağdaş çiftler akıllarında yatan evlilik yaşamından bütünüyle memnun olmadıklarında kolay boşanma düşüncesiyle evlendiklerinde, gerçekte bir çeşit deneme evliliğine girmektelerdir; ve bu evlilik, daha az medeni atalarının atıldıkları dürüst maceralarının ait olduğu düzeyden çok daha alttadır.
\usection{4.\bibnobreakspace Özel Mülkiyet Adetleri altındaki Evlilik}
\vs p082 4:1 Evlilik her zaman özel mülkiyet ve din ile yakın bir biçimde ilişkilendirilmişti. Özel mülkiyet evliliğin istikrarlaştırıcısı; din ise onun ahlaki hale getiricisidir.
\vs p082 4:2 İlkel evlilik, ticari bir kazan faaliyeti olarak, bir yatırımdı; evlilik, bir cilveleşme durumundan çok bir ticaret olayıydı. İlkel insanlar, topluluğun yararı ve refahı için evlenmişti; bu nedenle onların evlilikleri, ebeveynleri ve yaşlı atalarından meydana gelen bir biçimde, toplulukları tarafından tasarlanmakta ve düzenlenmekteydi. Ve özel mülkiyet adetlerinin evlilik kurumunu istikrarlı hale getirmede etkin olduğu, birçok çağdaş insan topluluğu arasındaki yerine kıyasla öncül kabileler arasında evliliğin daha kalıcı olduğu gerçeğinden kaynaklanmaktadır.
\vs p082 4:3 Medeniyet geliştikçe ve özel mülkiyet adetler içinde daha fazla bir biçimde tanındığında, çalma büyük bir suç haline geldi. Zina, erkek eşin özel mülkiyet hakkına bir müdahale olarak, bir çeşit çalma türü olarak tanınmıştı; daha eski yasa ve adetlerde bahse konu durumun özel olarak bahsedilmeme nedeni budur. Kadın babasının özel mülkiyeti görülmeye başlayıp, evlendiğinde bu mülkiyet atasından eşine aktarılmaktaydı; ve tüm yasallaşmış cinsel ilişkiler bu hâlihazırda mevcut özel mülkiyet haklarından doğmuştu. Eski Ahit, kadınları bir özel mülkiyet olarak bahsetmekte; Kuran ise onların erkeklere nazaran daha alt bir düzeyde olduğunu öğretmektedir. Erkek karısını bir arkadaşına veya misafirine ödünç verme hakkına sahipti; ve bu gelenek belirli topluluklar arasında hala varlığını sürdürmektedir.
\vs p082 4:4 Cinsel ilişkilerden doğan kıskançlık içkin değildir; bu durum, evrimleşen örf ve adetlerin bir ürünüdür. İlkel insan kadınını kıskanmamaktaydı; o sadece özel mülkiyetini muhafaza etmekteydi. Kadın eşi erkek eşten daha katı bir cinsel gözetim altına alma nedeni soyunun ve kalıtımının neden olduğu sadakatsizliğiydi. Medeniyetin ilerleyişinin başında gayri meşru çocuk itibarsız konuma düşmüştü. İlk başta sadece kadın zina için cezalandırılmıştı; daha sonra örf ve adetler onun erkek eşinin de cezalandırılışına hüküm vermişti; ve uzun çağlar boyunca incinen erkek eş veya koruyucu baba başkasının hakkına tecavüzde bulunan erkeği bütünüyle öldürme hakkına sahipti. Çağdaş insan toplulukları, yazılmamış kanunlar uyarınca töre cinayetleri olarak adlandırdığınız uygulamalara izin veren bu adetleri korumaya devam etmektedir.
\vs p082 4:5 Namus tabusu kökenini özel mülkiyet örf ve adetlerinin bir aşamasından aldığı için, ilk başta, evlenmemiş genç kızlar yerine evlenmiş kadınlara uygulanmıştı. Daha sonraki yıllarda namus, talip olan erkekten çok baba tarafından talep edilmişti; bakir bir kız, baba için ticari bir varlıktı --- o daha büyük gelir getirmekteydi. Namus daha fazla talep edilir bir konuma gelince, koca olacak kişi için iffetli bir gelini uygun bir şekilde yetiştirme hizmetinin tanınması altında babaya bir gelin ücreti ödenmesi uygulaması ortaya çıkmıştı. Bu uygulama bir kez geçerli olduğunda, kadın iffetine dair düşünce o kadar geçerli hale geldi ki, bekâretini teminat altına almak için, seneler boyunca fiilen esir kılan bir biçimde, gerçek anlamıyla kızları hapsetmek ortak adet haline gelmişti. Ve böylece daha ileri dönemlerdeki ortak ölçütler ve bekâret testleri kendiliğinden, mesleksel olarak gerçekleştirilen hayat kadınlığı sınıflarının doğuşuna kaynaklık etmişti; onlar, damatların anneleri tarafından bekâr olmadıkları belirlenen kadınlar olarak reddedilmiş gelinlerdi.
\usection{5.\bibnobreakspace Aile İçi ve Dışı Evlilik}
\vs p082 5:1 Çok öncesinden ilkel insanlar, ırk karışımının doğumların kalitesini arttırdığını gözlemlemişti. Bu gözlem; aile içi çoğalımın her zaman kötü olduğu anlamına gelmemekteydi, bunun yerine aile dışı yapılan çoğalımın her zaman göreceli olarak daha iyi olduğunu işaret etmekteydi; bu nedenle adetler, yakın akrabalar arasındaki cinsel ilişkilerin kısıtlanmasının yönünde belirginleşme eğilimi gösterdiler. Aile dışı ilişkilerle doğan bireyler daha becerikli olup, düşmancıl bir dünyada hayatta kalmada daha büyük yetiye sahiplerdi; aile içi ilişki sonrasında ortaya çıkmış bireyler adetleri ile birlikte kademeli bir biçimde ortadan kayboldu. Bu sürecin tümü yavaş bir gelişmeydi; ilkel insanlar bu tür sorunlara bilinçli bir biçimde neden aramadılar. Ancak daha sonraki ve gelişmiş insan toplulukları bunu gerçekleştirdi; ve onlar aynı zamanda, olağan zayıflığın zaman zaman fazlasıyla gerçekleşen aile işi çoğalımdan kaynaklandığı gözleminde de bulundular.
\vs p082 5:2 Her ne kadar iyi ırk türünün kendi içindeki çoğalımı, güçlü kabilelerin inşası ile sonuçlanmış olsa da; kalıtımsal bir biçimde gelen eksik gelişimde bulunan bireylerin aile içi çoğalımlarının kötü sonuçlarının yarattığı dehşet verici vakalar insan aklında daha fazla yer teşkil etmişti; bu sonuçtan yola çıkarak gelişen örf ve adetler artan bir biçimde, yakın akrabalar arasında gerçekleşen tüm evliliklere karşı koyan tabular inşa etmişlerdi.
\vs p082 5:3 Din, aile dışı evlilikler karşısında uzun bir süreden beri güçlü bir engel görevi görmektedir; birçok dini öğreti inancın dışında gerçekleştirilen evliliği yasaklamaktadır. Kadın aile içi evlilik için uygun görülürken, erkek için bu durum aile dışıdır. Özel mülkiyet her zaman evliliği etkilemiştir; ve zaman zaman, bir kavim içindeki mülkiyeti koruma amacıyla, örf ve adetler kadınları, erkekleri babalarının kabileleri arasından tercih etmeye zorlamaktadır. Bu türden hükümler, kuzen evliliklerinin büyük bir artışına sebebiyet vermişti. Aile içi çiftleşme aynı zamanda, zanaat sırlarını muhafaza etmeye dair bir çaba içinde uygulanmıştı; hünerli zanaatkârlar, zanaatlarına dair bilgilerini aile içinde tutmaya çalıştılar.
\vs p082 5:4 Üstün topluluklar, tecrit altına alındıklarında, her zaman akraba evliliklerine geri dönmüşlerdi. Nod toplulukları yüz elli bin yıldan fazla bir süre boyunca, aile içi çoğalan büyük topluluklardan bir tanesiydi. Daha sonraki aile içi evlilik adetleri, eflatun ırkına dair tarihi anlatılardan etkilenmişti; bu ırkta çiftleşmeler, ilk başta, erkek ile kız kardeş arasında olmak üzere sorunluydu. Mısırlı topluluklar uzun bir süre boyunca, Fars içinde daha da uzun bir süre boyunca varlığını sürdüren bir gelenek biçiminde, soylu saf kanı korumaya dair bir çaba içerisinde erkek ve kız kardeş evliliklerini uygulamıştı. Mezopotamyalılar arasında, İbrahim döneminden önce, kuzen evlilikleri zorunluydu; kuzenler öz kuzenleri karşısında daha öncül evlilik haklarına sahiplerdi. İbrahim’in kendisi, yarı kız kardeşi ile evlenmişti; ancak bu türden birlikteliklere, Museviler’in daha sonraki adetleri uyarınca izin verilmemişti.
\vs p082 5:5 Erkek ve kız kardeş evliliklerinden ilk uzaklaşma, çok eşlilik adetleri zamanında başladı; çünkü kız kardeş olan kadın eş, diğer eş veya eşler üstünde kibirli bir biçimde baskınlık kurmaktalardı. Bazı kabile adetleri hayatını yitirmiş bir erkek kardeşin eşiyle yapılacak evliliği yasakladı, ancak yaşayan erkek kardeşin hayatını yitirmiş olan için dul karısından çocuklar dünyaya getirmesini şart koydu. Aile içi evliliğin herhangi bir düzeyi için hiçbir biyolojik içgüdü bulunmamaktadır; bu türden kısıtlamalar tamamiyle bir tabu durumudur.
\vs p082 5:6 Aile dışı evlilikler sonunda baskın bir konuma geldi, çünkü onlar erkekler tarafından tercih edilmekteydi; aile dışından bir kadın eş seçmek aile içi yasalardan büyük bir özgürlüğü beraberinde getirdi. Çok yakın aşinalık nefrete sebebiyet vermektedir; bu nedenle, bireysel tercih çiftleşmede baskın bir konuma gelmeye başlayınca, kabile dışından eşleri seçmek adet haline gelmişti.
\vs p082 5:7 Birçok kabile nihai olarak kavim içi evlilikleri yasakladı; diğerleri çiftleşmeyi belirli toplumsal tabakalarla sınırladı. Birinin kendi toteminden olan bir kadınla evlenmesine karşı olan tabu, komşu kabilelerden kadınların kaçırılması âdetine zemin hazırladı. Daha sonra evlilikler, akrabalık yerine daha çok bölgesel ikamet uyarınca düzenlendi. Aile içi evliliğin aile dışı evliliğin çağdaş uygulamasına olan evriminde birçok aşama bulunmaktaydı. Halk içinde aile içi evliliklere karşı tabunun hüküm sürmesine rağmen, kabile önderleri ve kralların soylu kanı bir bütün ve saf halde tutabilmek için yakın kan bağından olanlar ile evlenmelerine izin verilmişti. Adetler genellikle, egemen yöneticilerin cinsel konularda belirli ehliyetleri belirlemelerine izin vermektedir.
\vs p082 5:8 Daha sonraki And toplulukların mevcudiyeti, Sang topluluklarının kendi kabilelerinin dışındaki bireyler ile olan artan çiftleşme arzularına sebebiyet vermişti. Ancak topluluk dışından bireylerle çiftleşmenin, komşu topluluklar göreceli bir barış ortamında yaşamaya öğrenene kadar yaygın hale gelmesi mümkün değildi.
\vs p082 5:9 Aile dışı evlenme kendi başına bir barış sağlayıcısıydı; kabileler arası evlilikler düşmanlıkları azaltmıştı. Aile dışı evlilik, kabilesel eş güdüme ve askeri birliklere yol açmıştı; bu tür evlilikler baskın hale geldi çünkü onların güçlerini arttırmıştı; ülkeyi inşa eden etmenlerden biriydi. Aile dışı evlilik aynı zamanda, artan ticaret ilişkileri tarafından fazlasıyla olumlu karşılanmıştı; macera ve keşif çiftleşme sınırlarının gelişmesine katkı sağlayıp, ırksal kültürlerin karşılıklı birleşimini fazlasıyla kolaylaştırmıştı.
\vs p082 5:10 Irksal evlilik adetlerinin bunun dışında kalan açıklanamaz tutarsızlıkları büyük ölçüde, yabancı kabilelerden kadın eşlerin kaçırılması veya çalınmasını içeren bu aile dışı evlilik geleneğinden kaynaklanmaktadır; bütün bunların hepsi farklı kabile adetlerinin bir birleşimi ile sonuçlanmıştı. Aile içi evliliği önemseyen bu tabular toplumsaldı; hiçbir kan ilişkisinin bulunmadığı durumlar halindeki evliliğin getirdiği karşı aile akrabalıklarının birçok düzeyini içine alan kan bağı evlilikleri üzerine olan tabularda oldukça iyi sergilendiği gibi biyolojik değildi.
\usection{6.\bibnobreakspace Irksal Karışımlar}
\vs p082 6:1 Bugün dünyada hiçbir saf ırk bulunmamaktadır. Ayrı renklere sahip olan öncül ve özgün evrimsel insan toplulukları, sarı ve siyah ırk olmak üzere, dünyada yalnızca iki temsili ırka sahiptir; ve bu iki ırk bile, nesli tükenmiş diğer renk toplulukları ile fazlasıyla karışmış haldedir. Beyaz ırk olarak adlandırdığınız topluluk; başat bir biçimde ilkçağın mavi insan soyundan gelmekte olup, Amerika kıtalarının mavi insanlarına ek olarak neredeyse bütün diğer ırklar ile karışmış haldedir.
\vs p082 6:2 Altı renkli Sang ırkları içinde onların üçü birincil ve diğer üçü ikincil topluluktur. Her ne kadar --- mavi, kırmızı ve sarı olarak --- birincil ırklar üç ırktan oluşan ikincil topluluğa göre birçok açıdan üstün olmuşsa da; bu ikincil ırkların, daha iyi ırk kolları birincil topluluklara karışabilseydi onları ciddi ölçüde geliştirebilecek olan birçok arzulanan niteliğe sahip oldukları hatırlanmalıdır.
\vs p082 6:3 “Melez,” “karma” ve “katışık” topluluklara dair beslenen bugünün önyargısı; ırkların karşılıklı giriştikleri günümüz birleşimlerinin, büyük ölçüde, ilgili ırkların oldukça alt düzeydeki ırk kolları arasında gerçekleşmesinden doğmaktadır. Sizler, ırk evliliğinin aynı ırk içinde yozlaşmış ırk kolları arasında gerçekleştiği durumlarda da tatmin edici olmayan doğumlar ile karşılaşmaktasınız.
\vs p082 6:4 Eğer Urantia’nın bugününki ırkları; kötüleşmiş, toplum karşıtı, iradesiz ve toplumdan yetersiz olduğu gerekçesiyle dışlanmış insan türlerinden meydana gelen en alt düzey katmanının verdiği kalıcı zarardan kurtulabilirse, sınırlı bir ırk bütünleşmesi karşısında daha az itiraz oluşacaktır. Ve eğer bu türden ırk karışımları, birkaç ırkın en yüksek türleri arasında gerçekleşebilirse, çok daha az miktarda itiraz sebebi var olacaktır.
\vs p082 6:5 Üstün ve birbirine benzemeyen ırk kollarının karışımı, yeni ve daha kudretli ırk kollarının yaratılma sırrıdır. Ve bu durum bitkiler, hayvanlar ve insan türleri için doğruluk taşımaktadır. Çeşitli toplulukların orta düzey veya üstün tabakasının ırk karışımları; Kuzey Amerika’nın Amerika Birleşik Devletler ülkesindeki mevcut nüfusunda görüldüğü gibi, \bibemph{yaratıcılık} potansiyelini fazlasıyla arttırmaktadır. Bu türden çiftleşmeler daha alt veya fazlasıyla düşük tabaka arasında gerçekleştiğinde, güney Hindistan’ın mevcut topluluklarında görüldüğü gibi, yaratıcılık azalmaktadır.
\vs p082 6:6 Irk karışımı, \bibemph{yeni} niteliklerin anlık ortaya çıkışına fazlasıyla katkıda bulunmaktadır; ve eğer bu tür karışım üstün ırk kollarının birlikteliği olursa, bunun sonrasında yeni nitelikler aynı zamanda \bibemph{üstün} özellikler haline gelecektir.
\vs p082 6:7 Bugünün ırkları alt düzey ve yozlaşmış ırk kolları ile çok fazla dolup taşan bir konumda bulunmayı sürdürdükçe, büyük çapta bir ırk karışımı en zararlı sonuçları doğuracaktır; ancak bu türden denemelere karşı itirazların çoğu, biyolojik nedenler yerine toplumsal ve kültürel ön yargılar üzerine dayandırılmaktadır. Alt düzey ırklar arasında gerçekleşen karışım olarak melez topluluklar sıklıkla, ataları karşısında bir ilerlemedir. Irkların birbirine karışımı, \bibemph{baskın genlerin} etkisi vasıtasıyla türleri geliştirmektedir. Irkların kendi aralarındaki karışımlar, karma topluluk içerisinde mevcut olan arzu edilen konumdaki\bibemph{ baskın öğelerin} geniş bir sayısının ortaya çıkma ihtimalini arttırmaktadır.
\vs p082 6:8 Geçmiş yüz yıl boyunca Urantia üzerinde, binlerce yıl öncesinde ortaya çıkana kıyasla daha fazla ırksal karışım gerçekleşmektedir. İnsan ırk kökenlerinin karşılıklı birleşiminin bir sonucu olarak öne sürülen büyük uyumsuzlukların tehlikesi fazlasıyla abartılmaktadır. “Melez” olarak adlandırılan topluluklara dair başlıca sorunlar toplumsal önyargılardan kaynağını almaktadır.
\vs p082 6:9 Beyaz ve Polonez ırklarının karışımı olan Pitcairn deneyiminin oldukça yararlı olduğu ortaya çıktı; çünkü beyaz erkekler ve Polonez kadınlar oldukça iyi ırk kollarından gelmekteydiler. Beyaz, kırmızı ve sarı ırkların en yüksek türleri arasında gerçekleşen karşılıklı birliktelik doğrudan bir biçimde birçok yeni ve biyolojik olarak etkin nitelikleri beraberinde getirecekti. Bu üç topluluk birincil Sang ırklarına aittir. Beyaz ve siyah ırkların karışımları doğrudan sonuçlarında o kadar arzu edilen sonuçları ortaya çıkarmamaktadır; ne de bu türden melez tenli doğumlar toplumsal ve ırksal itirazların göstermeye çalıştığı kadar olumsuzdurlar. Fiziksel olarak bu türden beyaz\hyp{}siyah ırk karışımları, bir takım yönlerden küçük çaplı geriliklerine rağmen, insanlığın mükemmel türleridir.
\vs p082 6:10 Birincil bir Sang ırkı ikincil bir Sang ırkı ile bütünleştiğinde, sonuncusu ilkinin zararına gerçekleşse de ciddi bir biçimde gelişmiş olur. Ve küçük bir ölçekte --- zamanın uzun süreçlerine yayılan bir biçimde --- ikincil toplulukların gelişimi için birincil ırkların bu türden bir fedakâr katkısı karşında itiraz edilebilecek çok az şey bulunabilir. Biyolojik olarak irdelendiğinde, ikincil Sang toplulukları, bazı açılardan birincil ırklar karşısında üstünlerdi.
\vs p082 6:11 Sonuçta insan türlerini bekleyen gerçek tehlike, onların ırklar arası çoğalımları karşısında varsayılan tehditten çok, medenileşmiş çeşitli topluluklar içindeki alt düzey ve yozlaşmış ırk kollarının sınırlanmamış çoğalımında yatmaktadır.
\vs p082 6:12 [Urantia üzerinde konumlanan bir Yüksek Melek Önderi tarafından sunulmuştur.]
