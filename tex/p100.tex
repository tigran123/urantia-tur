\upaper{100}{İnsan Deneyimi İçindeki Din}
\vs p100 0:1 Canli dini yaşamın deneyimi, ortalama düzeydeki insanı idealist gücün bir kişiliğine dönüştürür. Din, her bireyin ilerleyişini teşvik ederek her şeyin ilerleyişine hizmet eder; ve her bir şeyin ilerleyişi, her şeyin gerçekleştirdiği kazanımla çoğalır.
\vs p100 0:2 Ruhsal büyüme, diğer dindarlarla olan yakın birliktelikle karşılıklı olarak harekete geçer. Derin sevgi; olası en yüksek düzeyde bireysel tatmini açığa çıkarırken, --- bireysel tatmin aracılığı arkasında bir amaçsal çekim olarak --- dini büyüme için zemin hazırlamaktadır. Ve, din, günlük yaşamın sıradan angaryasına soyluluk kazandırmaktadır.
\usection{1.\bibnobreakspace Dini Büyüme}
\vs p100 1:1 Din anlamların derinleşmesini ve değerlerin gelişmesini ortaya çıkarırken, kötülük her zaman, tamamiyle kişisel olan değerlendirmeler mutlaklıkların düzeyine çıkarıldığında gerçekleşir. Bir çocuk deneyimi, içerdiği haz ölçüsünde değerlendirir; olgunluk, kişisel haz ile daha yüksek anlamların yer değiştirilişiyle, hatta çeşitli yaşam durumlarına ve kâinatsal ilişkilere dair en yüksek kavramsallaşmalara olan bağlıkla doğru orantılıdır.
\vs p100 1:2 Bazı kişiler büyümek için haddinden fazla meşgul olup, bu bedenle ruhsal sabitleştirmenin çok büyük tehlikesi içindedirler. Farklı yaşlardaki, birbirini takip eden kültürlerdeki ve medeniyetin ilerleyen aşamalarındaki anlamların gelişimi için hazırlığın yapılması zorunludur. Büyümenin ana engelleyicileri önyargı ve bilgisizliktir.
\vs p100 1:3 Her gelişen çocuğa, kendi dini deneyimini geliştirmesi için bir şans verin; hazır bir erişkin deneyimini ona dayatmayın. Yerleşik bir eğitim düzeni boyunca gerçekleşen senelik ilerleyişin doğrudan bir şekilde, bırakınız ruhsal gelişme, ussal deneyim haline gelmeyeceğini unutmayın. Kelime dağarcığındaki gelişim, karakterdeki büyümeye işaret etmemektedir. Büyüme gerçek anlamıyla sırf ürünlerden oluşan şeylerle gösterilmez; o, ilerlemeyle gösterilir. Gerçek eğitimsel büyüme; ideallerin gelişimi, değerlerin artan takdiri, değerlerin yeni anlamları ve yüce değerlere olan artmış bağlılıkla gösterilir.
\vs p100 1:4 Çocuklar kalıcı bir biçimde sadece, erişkin birlikteliklerinin bağlılıklarından etkilenirler; davranışsal kural veya örnek bile uzunca bir süreliğine etkili değildir. Sadık bireyler büyüyen kişilerdir; ve büyüme etkileyici ve iham verici bir gerçekliktir. Bugünü sadık bir biçimde yaşayın --- büyüyün ---, ve yarın onu takip edecektir. Bir iribaşın bir kurbağa haline gelmesinin en hızlı yolu, her anı bir iribaş olarak sadık bir şekilde yaşamaktır.
\vs p100 1:5 Dini gelişim için temel derecede önemli olan toprak; bireyin kendisini gerçekleştirişinin ilerleyici bir yaşamını, doğal eğilimlerin eş\hyp{}güdümünü, merakın sürekli yaşanan hissini ve makul serüvenden duyulan keyfini, memnuniyet duygularının deneyimini, dikkatli olma halini ve farkındalığı harekete geçiren korku uyarımının işlevini, ilgi çekici şeyler ve alçak gönüllülük olarak küçük olmanın olağan bir bilincini gerekli kılmaktadır. Büyüme aynı zamanda; --- vicdan olarak, zira vicdan gerçekten de kişisel idealler olarak bir kişinin sahip olduğu değer\hyp{}alışkanlıklarıyla kendisinin eleştirisidir --- özeleştiri tarafından eşlik edilen birey olmanın keşfidir.
\vs p100 1:6 Dini deneyim belirgin bir biçimde; fiziksel sağlık, içkin mizaç ve toplumsal çevre tarafından etkilenir. Ancak bu geçici koşullar, gökteki Tanrı’nın iradesini yerine getirmeye adanan bir ruh tarafından gerçekleştirilen içsel ruhsal ilerlemeyi engellemez. Özel olarak engellenmediği takdirde işlevini gerçekleştiren, olağan fanilerin tümünde büyümeye ve bireyin kendisini açığa çıkarmasına yönelik belirli içkin güdüler mevcuttur. Ruhsal büyümenin sahip olduğu bu temel potansiyel kazanımını teşvik etmenin bir yöntemi, yüce değerlere olan içten bağlılığın bir tutumunu sürdürmektedir.
\vs p100 1:7 Din bahşedilemez; başkasından alınamaz, bir başkasına verilemez, öğrenilemez veya kaybedilemez. O; nihai değerler için artan arayış arzusuyla orantılı bir biçimde büyüyen bir kişisel deneyimdir. Kâinatsal büyüme böylelikle, anlamların birikimini ve değerlerin sürekli genişleyen yükselişini beraberinde getirmektedir.
\vs p100 1:8 Dini düşünme ve eylem alışkanlıkları, ruhsal büyümenin dikkatli idaresine katkı sağlar niteliktedir. Bir kişi; şartlandırılmış nitelikteki anlık gerçekleşen bir ruhsal tepki biçiminde ruhsal uyarıma elverişli tepki doğrultusunda dini eğilimler geliştirebilir. Dini büyümeyi destekleyen alışkanlıklar; dini değerlere olan geliştirilmiş hassasiyeti, diğerleriyle olan dini yaşamın tanınmasını, kâinatsal anlamlar üzerinde irdeleyici düşünmeyi, ibadetsel sorun çözümünü, bir kişinin sahip olduğu ruhsal yaşamı diğer akranlarıyla paylaşmayı, bencillikten kaçınmayı, kutsal bağışlamaya haddinden fazla güvenmeyi reddetmeyi, Tanrı’nın mevcudiyeti içinde yaşanmayı içerir. Dini büyümenin etkenleri bilinç dâhilinde olabilir, ancak büyümenin kendisi değişiklik göstermeyen bir biçimde bilinç dışı gelişir.
\vs p100 1:9 Ruhsal büyümenin bilinç\hyp{}dışı doğası, buna rağmen, insan usunun varsayılmakta olan alt\hyp{}bilinç âlemlerinde faaliyet gösteren bir etkinliğe işaret etmemektedir; bunun yerine o, fani aklın bilinç\hyp{}ötesi düzeylerindeki yaratıcı etkinliklere işaret eder. Biliç\hyp{}dışı dini büyüme gerçekliğinin yerine getirilme deneyimi, bilinç\hyp{}ötesi düzeyin işlevsel mevcudiyetinin bir olumlu kanıtıdır.
\usection{2.\bibnobreakspace Ruhsal Büyüme}
\vs p100 2:1 Ruhsal gelişim, ilk olarak, gerçek ruhsal kuvvetler ile canlı bir ruhsal iletişimin idaresine dayanmaktadır; daha sonra, ruhsal meyvenin şu devamlı üretimine dayanmaktadır: bir kişinin sahip olduğu ruhsal bağışçılarından aldığı şeyi diğer akranlarının hizmetine sunmak. Ruhsal ilerleme; gökteki Tanrı’nın iradesini yerine getirmek için candan amaç olarak Tanrı’yı bilme ve onun gibi olma arzusu biçimindeki kusursuzluk açlığının öz bilinciyle birleşen ruhsal açlığın ussal tanınmasına dayanmaktadır.
\vs p100 2:2 Ruhsal büyüme ilk olarak, ihtiyaçların bir farkındalığıdır; daha sonra, anlamların bir kavrayışıdır; ve, bunun sonrasında, değerlerin bir keşfidir. Gerçek ruhsal gelişimin kanıtı; derin sevgi güdüsüyle hareket eden, fedakâr hizmet ile etkinleşen, ve, kutsallığın kusursuzluk ideallerine olan candan ibadetin üzerinde hâkim olduğu bir insan kişiliğinin sergilenişinden meydana gelmektedir. Ve, bu bütüncül deneyim, yalnızca din\hyp{}kuramsal nitelikli inanışlara zıt bir biçimde dinin gerçekliğini oluşturmaktadır.
\vs p100 2:3 Din; üzerinde, evrene olan aydınlanmış ve bilge bir ruhsal tepki yöntemi haline geldiği deneyim seviyesine kadar ilerleyebilir. Bu türden yüceltilmiş bir din, insan kişiliğinin şu üç düzeyinde faaliyet gösterebilir: ussal, morontial, ve ruhsal; akıl üzerinde, evrimleşen ruhta ve ikamet eden ruhaniyet ile birlikte.
\vs p100 2:4 Ruhsallık, bir seferde; bir kişinin Tanrı’ya olan yakınlığının göstergesi ve akran varlıklarına olan yararlığının ölçüsüdür. Ruhsallık; nesnelerde güzelliği keşfetme, anlamlarda gerçekliği tanıma ve değerlerde iyiliği bulma yetkinliğini geliştirir. Ruhsal gelişme, bu amaç için var olan yetkinlikle belirlenmekte olup, doğrudan bir şekilde, derin sevginin bencil niteliklerinin saf dışı bırakılmasıyla doğru orantılıdır.
\vs p100 2:5 Mevcut ruhsal düzey, Düzenleyici’ye olan uyum niteliğindeki İlahiyat erişiminin ölçüsüdür. Ruhsallığın nihai seviyesinin elde edilişi, Tanrı\hyp{}gibi\hyp{}olmanın en yüksek düzeyi olarak gerçekliğin doruk noktasına olan erişime eş değerdir. Ebedi yaşam, sınırsız değerlerin sonu gelmez arayışıdır.
\vs p100 2:6 İnsanın kendisini gerçekleştirişinin hedefi ruhsal olmalıdır, maddi değil. Uğruna çaba sarf edilmesi değer gerçeklikler sadece kutsal, ruhsal ve ebedidir. Fani insan, fiziksel hazların memnuniyetle deneyimini ve insansı sevgi duygularının tatminini yaşamakla yükümlüdür; o, insan birlikteliklerine ve geçici kurumlara olan bağlılıktan yarar sağlamaktadır; ancak bunlar mekânı aşmak, zamanı yenmek ve kutsal kusursuzluğun ve kesinlik unsur hizmetinin ebedi nihai sonunu elde etmek zorunda olan ölümsüz kişiliğin üzerine inşa edileceği ebedi temeller değildir.
\vs p100 2:7 İsa, şunları söylediğinde Tanrı\hyp{}bilen faninin derin olan kendinden eminliğini temsil etmişti: “Tanrı\hyp{}bilen bir krallık inananı için, dünyasal her şey yıkılsa ne olur?” Geçici teminatlar kırılgandır, ancak ruhsak kesinlikler yıkılmazdır. İnsansı düşmanlığın, bencilliğin, kabalığın, nefretin, kötü niyetin ve kıskançlığın taşkın dalgaları fani ruhunu dövdüğünde, mutlak bir biçimde yok edilemez olan ruhaniyetin hisarı biçimindeki içsel bir burcun orada var oluşunun güvencesinde huzur bulabilirsiniz; en azından bu durum, ruhunun idaresini ebedi Tanrı’nın ikamet eden ruhaniyetine adamış olan her insan varlığı için doğrudur.
\vs p100 2:8 Bu türden ruhsal kazanımdan sonra, ister kademeli büyümeyle ister belli bir buhranla teminat altına alınmış olsun, değerlerin yeni bir ortak ölçüsünün gelişimine ek olarak kişiliğin yeni bir yönelimi ortaya çıkar. Bu türden ruhaniyet\hyp{}doğumu bireyleri yaşam içinde bir amaç için yeniden o kadar bütünleşir ki, en fazla arzuladıkları gayeleri yok olurken ve en düşkün oldukları ümitleri boşa çıkarken, sakince hiçbir şey olmamış gibi bekleyebilirler; onlar olumlu bir biçimde bu türden felaketlerin, evren kazanımının yeni ve daha yüce bir düzeyine ait daha soylu ve daha kalıcı gerçekliklerin yetişmesine hazırlık niteliğindeki bir kişinin geçici yaratımlarına zarar veren yeniden yönlendirici ani karışıklardan başkası olmadığını bilirler.
\usection{3.\bibnobreakspace Yüce Değer’in Kavramsallaşmaları}
\vs p100 3:1 Din, durağan ve şen bir iç huzura erişme yöntemi değildir; o, devinimsel hizmet amacıyla ruhun düzenlenişi için bir dürtüdür. Sevgi dolu Tanrı ve hizmet eden insana olan sadık görevde, birey bütünlüğünün sağlanışıdır. Din, ebedi ödül olan yüce amacın elde edilişinde hayati derecede önemli her bedeli ödemektedir. Muhteşem bir biçimde ulvi olan dini sadakatte kutsanmış bir bütüncüllük bulunmaktadır. Ve bu bağlılıklar, toplumsal olarak etkin ve ruhsal olarak ilerleyici niteliktedir.
\vs p100 3:2 Dindar kişi için Tanrı kelimesi; yüce gerçekliğin yaklaşılmasını ve kutsal değerin tanınmasını gösteren bir simge haline gelmektedir. İnsanın beğendikleri ve hoşlanmadıkları şeyler, iyiliği veya kötülüğü belirlememektedir; ahlaki değerler, arzuların tatmininden veya duygusal hoşnutsuzluktan doğmamaktadır.
\vs p100 3:3 Değerlerin düşünülüşünde sizler, \bibemph{değerin kendisi} ile \bibemph{değerli} olanı ayırmak zorundasınız. Hoşnutluk veren etkinlikler ile onların insan deneyiminin sürekli olarak ilerleyen bir biçimde yükselen düzeyleri üzerindeki anlamlı bütünleşmelerini ve gelişmiş dışavurumları arasındaki ilişkiyi tanımak zorundasınız.
\vs p100 3:4 Anlam, deneyimin değere kattığı bir şeydir; o, değerin takdirsel bilincidir. Soyutlanmış ve tamamiyle bencil olan bir haz, göreceli kötülüğe yaklaşan anlamsız bir eğlence niteliğindeki anlamların neredeyse bütüncül bir değersizleşimini simgeleyebilir. Değerler; gerçeklikler anlamlı ve akılsal olarak ilişkili olduğunda, bu türden ilişkiler akıl tarafından tanındığında ve takdir edildiğinde deneyimseldir.
\vs p100 3:5 Değerler hiçbir zaman durağan olamaz; gerçeklik, büyüme olarak değişime işaret etmektedir. Anlamın genişlemesi ve değerin yükselişi olarak büyümenin yoksunluğundaki değişim --- olası kötülük niteliğinde --- değersizdir. Kâinatsal uyumun niteliği arttıkça, herhangi bir deneyimin sahip olduğu anlam artmaktadır. Değerler, kavramsal aldanmalar değillerdir; onlar gerçektir, ancak her zaman ilişkilerin durumuna bağlıdırlar. Değerler her zaman, hem mevcut hem de olasıdır --- geçmişte olan değil, şimdi var olan ve gelecekte gerçekleşecek olan şeylerdir.
\vs p100 3:6 Mevcut ve olası olan şeylerin ilişkilendirilmesi, değerlerin deneyimsel gerçekleştirilmesi olarak büyümeye karşılık gelmektedir. Ancak büyüme, yalnızca gelişme değildir. İlerleme her zaman anlamlıdır; ancak büyüme olmadan göreceli olarak değersizdir. İnsan yaşamının yüce değeri; değerlerin büyümesinden, anlamlardaki ilerlemeden ve bu iki deneyimin de kâinatsal düzeydeki karşılıklı ilişkiselliğinin gerçekleştirilmesinden meydana gelmektedir. Ve bu türden bir deneyim, Tanrı\hyp{}bilincinin dengidir. Bu türden bir fani, her ne kadar doğa\hyp{}ötesi olmasa da, gerçekten insan\hyp{}ötesi hale gelmektedir; ölümsüz olan bir fani evirilmektedir.
\vs p100 3:7 İnsan, büyümeyi tek başına gerçekleştirmez; ancak o, elverişli şartları sağlayabilir. İster fiziksel, ister ussal ve ister ruhsal olsun büyüme her zaman bilinçdışıdır. Sevgi bu nedenle her zaman büyümektedir; o yaratılamaz, imal edilemez veya satın alınamaz; o büyümek zorundadır. Evrim, büyümenin kâinatsal bir yöntemidir. Toplumsal büyüme, yasamayla teminat altına alınamaz; ve ahlaksal büyümeye, gelişmiş idare ile sahip olunamaz. İnsan bir makineyi imal edebilir; ancak onun gerçek değeri, insan kültüründen ve kişisel takdirden elde edilmek zorundadır. İnsanın büyümeye olan tek katkısı, yaşayan inanç olarak --- kendi kişiliğinin bütüncül güçlerini harekete geçirmektedir.
\usection{4.\bibnobreakspace Büyümenin Sorunları}
\vs p100 4:1 Dini yaşam, adanmış yaşamdır; ve adanmış yaşam, özgün ve kendi kendine gerçekleşen yaratıcı yaşamdır. Yeni dini kavrayışlar, eski ve alt düzeydeki tepki yöntemleri yerine yeni ve daha iyi tepki alışkanlıklarını tercih etme sürecini başlatan çatışmalardan doğmaktadır. Yeni anlamlar yalnızca, çatışmanın ortasında ortaya çıkmaktadır; ve çatışma yalnızca, üstün anlamlar içinde çağrıştırılan yüksek değerleri benimsemenin reddi karşısında varlığını sürdürebilir.
\vs p100 4:2 Dini kafa karışıklıkları kaçınılmazdır; orada, öngörüsel çatışma ve ruhsal hoşnutsuzluk olmadan hiçbir büyüme mevcut olamaz. Yaşama ait felsefi ortak bir ölçütünün düzenlenişi, aklın felsefi alanları içinde dikkate değer düzeydeki akılsal karışıklığını beraberinde getirmektedir. Bağlılıklar; bir mücadele olmadan büyük, iyi, gerçek ve soylu adına yerine getirilmez. Çaba, ruhsal görüşün belirginleşmesi ve kâinatsal kavrayışın gelişmesini doğurur. Ve insan usu, geçici mevcudiyetin ruhsal\hyp{}olmayan enerjilerinden kıt kanaat geçinmekten alı konulmaya karşı çıkar. Miskin hayvan aklı, kâinatsal sorun çözümü ile cebelleşmek için gereken çabaya isyan eder.
\vs p100 4:3 Ancak dini yaşamın büyük sorunu, kişiliğin ruh güçleri ile birlikte SEVGİ’nin baskınlığını birleştirme görevinden oluşmaktadır. Sağlık, zihinsel verimlilik ve mutluluk; fiziksel sistemlerin, akılsal sistemlerin ve ruhaniyet sistemlerin bütünleşmesinden doğmaktadır. Sağlık ve sıhhat hakkında insan birçok şeyi anlamaktadır; ancak mutluluk hakkında o, gerçekte çok az şeyin farkına varmıştır. En yüksek mutluluk, ruhsal ilerleme ile ayrışmaz bir biçimde ilişkilidir. Ruhsal büyüme; her anlayışın ötesine geçen uzun ömürlü neşeyi, huzuru açığa çıkarır.
\vs p100 4:4 Fiziksel yaşam içinde duyular, maddelerin mevcudiyetini anlatır; akıl, anlamların gerçekliğini keşfeder; ancak ruhsal deneyim, yaşamın gerçek değerlerini birey için açığa çıkarır. İnsan yaşamının bu yüksek düzeylerine, Tanrı’nın yüce sevgisinde ve insanın fedakâr sevgisinde erişilir. Eğer siz akranınız olan insanları severseniz, onların sahip oldukları değeri çoktan keşfetmiş bir durumda olursunuz. İsa insanları çok sevmişti, çünkü o onlara çok büyük bir değer atfetmişti. Birliktelik içinde bulunduğunuz kişilerin sahip oldukları değeri en iyi, onların güdülerini keşfederek bulabilirsiniz. Eğer biri sizi sinirlendiriyorsa, hınç duygularına sebebiyet veriyorsa, siz; bu türden uygunsuz davranış için onun nedenleri biçiminde onun bakış açısını duygudaş bir biçimde algılamaya çabalamalısınız. Eğer bir kez olsun komşunuzu anlarsanız, hoşgörülü hale geleceksiniz; ve bu hoşgörü arkadaşlığa doğru büyüyüp, sevgiye doğru olgunlaşacaktır.
\vs p100 4:5 Aklın gözünde, tam da karşısına şiddetle bakarken ayakta, bacakları açık, sopası havada, nefret ve düşmanlık soluyan kısa, çirkin, pasaklı, homurdanan bir insan gövdesi biçiminde --- mağara\hyp{}ikameti dönemlerindeki ilkel atalarınızın birine ait bir resim canlanır. Bu türden bir resim, insanın kutsal soyluluğunu neredeyse hiçbir şekilde tasvir etmez. Ancak bu resmi genişletmemize izin verin. Bu canlandırılmış insanın karşısında, kılıç\hyp{}dişli bir kaplan oturmaktadır. Onun arkasında ise, bir kadın ve iki çocuk. Sizler derhal, bu türden bir resmin, insan ırkı içindeki çoğu güzel ve soylu şeyin başlangıcını işaret ettiğinin ayrımına varırsınız; ancak insan bu iki resimde de aynıdır. Yalnızca, ikinci tasvirde size genişlemiş bir bakış açısı sunulmuştur. Siz onun içinde, bu evrimleşen faninin güdüsünü algılarsınız. Onun tutumu övülmeye değer bir konuma gelmektedir, çünkü siz onu anlamaktasınız. Eğer yalnızca birliktelik içinde bulunduğunuz bireylerin güdülerini derinliğine kavrayabilseydiniz, onları ne kadar da iyi bir şekilde anlayabilirdiniz. Eğer yalnızca akranlarınızı tanıyabilseydiniz, nihai olarak onlara çok derin bir sevgi besleyebilirdiniz.
\vs p100 4:6 Sizler akranlarınızı, tek başına bir irade eylemiyle gerçek anlamıyla sevemezsiniz. Sevgi yalnızca, komşunuzun güdüleri ve duyuşlarının en ince ayrıntısına kadar anlaşılmasından doğmaktadır. Her gün bir tane daha insan varlığını sevmeyi öğrenmek karşısında bugünün tüm insanlarını sevmek çok önemli değildir. Eğer her gün veya her hafta akranlarınızdan bir tanesine dair bir anlayışa erişirseniz, ve bu yetkinliğinizin ölçütü olursa, bunun sonucunda siz kesin bir biçimde toplumsal hale gelip, gerçek anlamıyla kişiliğinizi ruhsallaştırırsınız. Sevgi yayılıcıdır; ve insan sadakati ussal ve bilge olduğunda, sevgi nefretten daha çekicidir. Yalnızca içten ve fedakâr sevgi gerçek anlamıyla bulaşıcıdır. Eğer yalnızca her fani devinimsel şefkatin bir odağı haline gelebilseydi, sevginin bu iyi huylu virüsü yakın zaman içinde, tüm medeniyetin sevgiyle kaplanacağı ve onun insanlığın kardeşliğinin gerçekleşmiş hali olacağı bir düzeyde insanlığın duygusal duygu\hyp{}akışını kaplardı.
\usection{5.\bibnobreakspace Dönüştürme ve Gizemcilik}
\vs p100 5:1 Dünya, kayıp ruhlar ile doludur; bu kayıplık din\hyp{}kuramsal olarak değil, hayal kırıklığına uğramış bir felsefi dönemin ana öğreti akımları arasında kafa karışıklığı içinde amaçsızca dolaşan bir biçimde istikametsel anlamdadır. Oldukça az sayıdaki kişi, yönetim yetkisini elinde bulunduran dinsel kurumun yerine bir yaşam felsefesinin nasıl yerleştirileceğini öğrenmiştir. (Her ne kadar nehir yatağı nehrin kendisi olmasa da, toplumsallaşmış dinin simgeleri büyümenin kanalları olarak hor görülmemelidir.)
\vs p100 5:2 Dini büyümenin ilerleyişi; duraklamadan çatışma boyunca eş\hyp{}güdüme, güvensizlikten şüphe duyulmayan inanca, kâinatsal bilince dair kafa karışıklığından kişiliğin bütünleşmesine, geçici amaçtan ebedi olana, korkunun esaretinden kutsal evlatlığın özgürlüğüne doğru hareket eder.
\vs p100 5:3 Tanrı\hyp{}bilincinin zihinsel, duygusal ve ruhsal farkındalığı olarak --- yüce ideallerine olan bağlılığın resmi ifadeleri doğal ve kademeli bir büyüme şeklinde veya zaman zaman, bir buhranda olduğu gibi, belirli olaylarda deneyimlenebilir. Aziz Pavlus, Şam yolundaki o önemli günde bu türden bir anlık ve hayret verici dönüşümü deneyimlemişti. Gotama Sidarta, yalnız başına oturduğu ve nihai gerçekliğin gizemine girmeye çalıştığı gece benzer bir deneyime sahip olmuştu. Diğer birçokları benzer deneyimlere sahip olmuşlardır; ve birçok gerçek inanan, anlık dönüşüm olmadan ruhaniyet içinde ilerlemiştir.
\vs p100 5:4 Dinsel dönüşümler olarak adlandırdığınız olaylar ile ilişkili hayret verici olguların çoğu, özü bakımından tamamiyle psikolojiktir; ancak zaman zaman orada, kökeni itibariyle ruhsal da olan deneyimler ortaya çıkmaktadır. Akılsal hareketlenme, ruhaniyet kazanımına olan yukarı yönlü zihinsel erişiminin herhangi bir düzeyi üzerinde mutlak olarak bütüncül olduğu zaman, orada kutsal düşünceye olan sadakatlerin insan güdüsü türünün kusursuzluğu mevcut olduğunda, bunun sonucunda orada oldukça sık bir biçimde; ikamet eden ruhaniyetin, inanan faninin bilinç\hyp{}ötesi aklının yoğunlaşmış ve kutsanmış amacıyla eş zamanlı hale gelmek için anlık bir aşağı yönlü kavrayışı ortaya çıkar. Ve, tamamiyle psikolojik olan katılımın üzerinde ve ötesindeki etkenlerden meydana gelen dönüşümü mevcut kılan şey, bütünleşmiş ussal ve ruhsal olguların bu türden deneyimleridir.
\vs p100 5:5 Ancak, duygu tek başına sahte bir dönüşümdür; bir kişi, duyguya ek olarak inanca sahip olmak zorundadır. O kadar ki; bu türden zihinsel hareketlenme yarımdır, ve insan\hyp{}bağlılık güdüsü tamamlanmamıştır, dönüşümün deneyimi iç içe geçmiş bir ussal, duygusal ve ruhsal gerçeklik olmalıdır.
\vs p100 5:6 Eğer bir kişi; başka bir biçimde bütünleşmiş ussal yaşam içinde işlevsel olarak geçerlilik gösteren bir varsayım biçiminde kuramsal bir alt\hyp{}bilinç aklını tanımak istiyorsa, bunun sonucunda, tutarlılığını korumak için, kişi, Düşünce Düzenleyicisi olan ikamet eden ruhaniyet birimi ile doğrudan ilişkinin bölgesi olarak bilinç\hyp{}ötesi düzey niteliğindeki yükselen ussal etkinliğin benzer ve ilgili bir âlemi üzerinde düşünmek zorundadır. Tüm bu zihinsel varsayımlar içindeki büyük tehlike; olağanüstü rüyalara ek olarak geleceğe dair görülerin ve gizemli olarak adlandırdığınız diğer deneyimlerin insan aklına yapılan kutsal iletişimler biçiminde görülebilecek oluşudur. Geçmişteki dönemlerde kutsal varlıklar kendilerini belirli Tanrı\hyp{}bilen kişilere açığa çıkarmışlardı; onlar bunu, bahse konu kişilerin gizemli iletişim biçimleri veya kötümser gelecek görüşleri nedeniyle değil, tüm bu olgulara rağmen gerçekleştirmişlerdi.
\vs p100 5:7 Dönüşme\hyp{}arzusu karşısında, Düşünce Düzenleyicisi ile olası ilişkinin morontia alanlarına olan yaklaşımın daha iyisi, içten ve fedakâr dua olarak yaşayan inanç ve samimi ibadet aracılığıyla olacaktır. İnsan aklının bilinç\hyp{}dışı düzeylerine ait hafızaların haddinden fazla bir biçimde deneyimlenen anlık heyecan hissi, kutsal açığa çıkarılışlar ve ruhani yönlendirmeler ile karıştırılmıştır.
\vs p100 5:8 Orada, dini hayalciliğin alışkanlıksal uygulamasıyla iniltili büyük tehlike bulunmaktadır: gizemcilik, her ne kadar zaman zaman içten ruhsal birlikteliğin bir aracı olmuş olsa da, gerçeklikten kaçınmanın bir yöntemi haline gelebilir. Yaşamın yoğun olarak aktığı yerlerden kısa süreli çekilme ciddi bir biçimde tehlike arz etmeyebilse de, kişiliğin haddinden fazla uzun süren tecridi en istenilmeyen şeydir. Hiçbir koşul altında geleceği görmeye yönelen bilincin kendinden geçme düzeyi, dini bir deneyim olarak işlenilemez.
\vs p100 5:9 Gizem düzeyinin temel nitelikleri; bilincin, görece eylemsiz olan bir us üzerinde faaliyet gösteren merkezi odaklanmanın keskin adalarıyla dağılımıdır. Tüm bunların hepsi, bilinci; bilinç\hyp{}ötesi düzey olan ruhsal iletişiminin alanı doğrultusu yerine bilinç\hyp{}altına doğru çekmektedir. Birçok gizemci, akılsal ayrışımlarını olağandışı akılsal dışavurumlarının düzeyine taşımıştır.
\vs p100 5:10 Ruhsal derin düşünmenin daha sağlıklı tutumu, düşünceli ibadette ve şükranlık duasında bulunabilir. Beden içindeki İsa’nın yaşamının daha sonraki yıllarında ortaya çıkmış olduğu gibi bir kişinin Düşünce Düzenleyicisi ile olan doğrudan birlikteliği, bu gizemli olarak adlandırılan deneyimler ile karıştırılmamalıdır. Gizemli birlikteliğe olan girişe katkı sağlayan etkenler, bu türden zihin düzeylerinin sahip olduğu tehlikeye işaret etmektedir. Gizemsel düzey şu gibi şeyler tarafından tetiklenir: fiziksel yorgunluk, oruç, zihinsel ayrışma, derin estetik deneyimler, keskin cinsel dürtüler, korku, endişe, hiddet ve vahşi dans etme. Bu türden başlangıçsal hazırlanmanın bir sonucu olarak anlık gerçekleşen maddi nitelikli derin duygu etkileşimlerinin büyük bir kısmı kaynağını bilinç\hyp{}altı akıldan alır.
\vs p100 5:11 Gizem olguları için şartlar her ne kadar elverişli olursa olsun, Nasıralı İsa’nın; Cennet Yaratıcısı ile olan bütünlüğü için bu türden yöntemlere hiçbir zaman başvurmamış olduğu kesin bir biçimde anlaşılmalıdır. İsa hiçbir alt\hyp{}bilinç yanılsaması veya bilinç\hyp{}ötesi aldanması yaşamamışlardı.
\usection{6.\bibnobreakspace Dini Yaşamın İşaretleri}
\vs p100 6:1 Evrimsel dinler ve açığa çıkarımsal dinler yöntem bakımından dikkate değer bir biçimde farklılık gösterebilir; ancak güdü bakımından büyük benzerlik bulunmaktadır. Din, yaşamın özel bir işlevi değildir; bunun yerine o bir yaşam biçimidir. Gerçek din, dindar bireyin kendisi ve tüm insanlık için yüce değerde gördüğü belirli bir gerçeklik için gösterdiği içten bir bağlılıktır. Ve, tüm dinlerin oldukça belirgin olan nitelikleri şunlardır: yüce değerlere olan şüphe duyulmayan bağlılık ve içten sadakat. Yüce değerlere olan bu dini bağlılık, varsayıldığı şekliyle dindar olmayan annenin çocuğuyla olan ilişkisinde ve dindar olmayan bireylerin benimsenmiş bir amaca yönelik coşkun bağlılığında sergilenmektedir.
\vs p100 6:2 Dindar bireyin kabul edilmiş yüce değeri bayağı veya yanlış bile olabilir; ancak o yine de dinseldir. Bir din, tam da; yüce olarak gördüğü değerin, gerçekten de, özgün ruhsal değerdeki bir kâinatsal gerçekliği olduğu ölçüde içtendir.
\vs p100 6:3 Dini uyarıma olan insan karşılığının işaretleri, soyluluk ve ihtişamın niteliklerini içine alır. Samimi dindar; evren vatandaşlığının bilincinde olup, Tanrı’nın evlatlarının üstün ve soylulaştırılmış bir birliktelik aidiyetinin üyesi olma güvencesiyle heyecanlanır ve canlanır. Bireyin kendisine beslediği güvenin bilinci, yüce hedefler olarak --- en yüksek evren amaçlarının peşine düşme uyarımı tarafından çoğalmış hale gelmiştir.
\vs p100 6:4 Birey; gelişmiş öz\hyp{}denetimi zorunlu kılan, duygusal çatışmayı düşüren ve fani yaşamı gerçek anlamıyla yaşamaya değerli kılan her şeyi içine alan bir güdünün ilgi çekici etkisine kendisini teslim etmiştir. İnsan sınırlılıklarının kötümser tanınışı; en yüksek evren ve aşkın\hyp{}evren amaçlarına erişmek için ahlaki kararlılık ve ruhsal arzu ile ilişkili halindeki fani yetersizliklerinin doğal bilincine dönüşmüştür. Ve, fani\hyp{}ötesi ideallere olan erişim için bu güçlü arzu her zaman; artan sabır, müsamaha, cesaret ve hoşgörü tarafından nitelenir.
\vs p100 6:5 Ancak, gerçek din, bir hizmet yaşamı olarak yaşayan bir sevgidir. Dindar bireyin, büyük bir kısmı tamamiyle geçici ve boş olan şeylerden ayrılışı hiçbir zaman toplumsal tecride götürmemektedir; ve o hiçbir zaman mizah anlayışına zarar vermemelidir. Gerçek din, insan mevcudiyetinden hiçbir şey almamaktadır; ancak o, yaşamın tümüne yeni anlamlar kazandırmaktadır. O; ruhsal kavrayışa ek olarak insan bağlılıklarının ortak toplumsal yükümlülüklerine olan sadık bağlılık tarafından denetlenmediğinde daha tehlikeli olan, katı inancı için sonuna kadar mücadele eden birinin ruhaniyetini bile ortaya çıkarabilir.
\vs p100 6:6 Dini yaşamın en hayret verici belirleyici işaretlerinden bir tanesi; tüm insan anlayışının ötesine geçen, kuşku ve kargaşanın her türünün yokluğunu simgeleyen kâinatsal dinginlik olarak, devinimsel ve ulvi barıştır. Ruhsal istikrarın bu türden düzeyleri, hayal kırıklığından etkilenmez. Bu gibi dindarlar şunları söylemiş olan Aziz Pavlus gibidir: “Ben; ne ölümün, ne yaşamın, ne meleklerin, ne prensliklerin, ne güçlerin, ne şimdiki şeylerin, ne gelecek olan şeylerin, ne yüksekliğin, ne derinliğin, ne de başka bir şeyin bizleri Tanrı’nın sevgisinden ayırmaya yetkin olamayacağına kani oldum.”
\vs p100 6:7 Orada; Yüce’nin gerçekliğini kavramış ve Nihayet’in amacının peşine düşmüş olan dindarın bilinci içinde barınır halde bulunan, galip gelen ihtişamın kendisini gerçekleşmesi ile ilişkili bir güvenlik hissi bulunmaktadır.
\vs p100 6:8 Evrimsel din bile, içten bir deneyim olduğu için bağlılık ve ihtişam içinde tüm bunlara karşılık gelmektedir. Ancak, açığa çıkarılmış din, içten olmasına ek olarak \bibemph{muhteşemdir}. Genişlemiş ruhsal duyuşun yeni bağlılıkları, hizmet ve aidiyet birlikteliğine ait olarak sevgi ve sadakatin yeni düzeylerini yaratmaktadır; ve, tüm bu gelişmiş toplumsal bakış, Tanrı’nın Yaratıcılığına ek olarak insanın kardeşliğinin gelişmiş bir bilincini üretmektedir.
\vs p100 6:9 Evrimsel din bile, içten bir deneyim olduğu için bağlılık ve ihtişam içinde tüm bunlara karşılık gelmektedir. Ancak, açığa çıkarılmış din, içten olmasına ek olarak muhteşemdir. Genişlemiş ruhsal duyuşun yeni bağlılıkları, hizmet ve aidiyet birlikteliğine ait olarak sevgi ve sadakatin yeni düzeylerini yaratmaktadır; ve, tüm bu gelişmiş toplumsal bakış, Tanrı’nın Yaratıcılığına ek olarak insanın kardeşliğinin gelişmiş bir bilincini üretmektedir.
\usection{7.\bibnobreakspace Dini Yaşamın Zirve Noktası}
\vs p100 7:1 Her ne kadar Urantia’nın ortalama fanisi; Nasıralı İsa’nın beden içinde kısa süreli ikametinde elde etmiş olduğu, karakterdeki yüksek kusursuzluğa erişmeyi hayal bile edemese de, her fani inananın, İsa kişiliğinin kusursuzlaşmış doğrultusu boyunca güçlü ve bütünleşmiş bir kişiliği geliştirmesi tamamiyle mümkündür. Hâkim’in kişiliğine ait benzersiz nitelik çok da, seçkin ve dengeli bütünlüğü olarak onun uyumundaki kusursuzlukta değildi. İsa’nın en etkili sunumu, suçlayıcıları karşısında Hâkim’i işaret ederken “İnsana bakın!” sözünü söyleyen birinin örneğinin takip edilişinden meydana gelmektedir.
\vs p100 7:2 İsa’nın hatasız iyiliği, insanların kalplerine dokunmuştu; ancak onun güvenilir karakter kuvveti, takipçilerini büyülemişti. O gerçekten içtendi; onun içinde ikiyüzlü nitelikte olan hiçbir şey yoktu. O, gösterişten tamamiyle uzaktı; o her zaman canlandırıcı bir biçimde samimiydi. O, başkalarının gerçek olmayan bir şeye inanmalarını için rol yapacak kadar hiçbir zaman alçalmadı; ve o hiçbir zaman, gerçek olmayan bir şeyi gerçeklik olarak sunmaya başvurmadı. O gerçekliği yaşadı, hatta bunu öğrettiği şekliyle bile gerçekleştirdi. O gerçeklikti. Her ne kadar bu tür içtenlik zaman zaman acı çekmesine neden olduysa da o, kurtarıcı gerçekliği kendi nesline duyurmakla sınırlandırılmıştı. O şüphesiz bir biçimde gerçekliğin tümüne sadıktı.
\vs p100 7:3 Ancak Hâkim, oldukça erişilebilir bir biçimde fazlasıyla makuldü. Tasarımlarının tümü bu türden kutsanmış ortak duyuşla nitelenirken, hizmetinin tümünde oldukça işlevseldi. O oldukça garip, tutarsız ve alışılmadık şeyleri dışa vuran eğilimlerden fazlasıyla uzaktı. O hiçbir zaman değişken, tuhaf veya kendini kaybetmiş nitelikte değildi. Öğretisinin tümünde ve yaptığı her şeyde her zaman, ölçülülüğün olağanüstü bir duyuşuyla ilişkili seçkin bir ayırt etme bulunmaktaydı.
\vs p100 7:4 İnsanın Evladı her zaman oldukça dingin bir kişilikti. Onun düşmanları bile, kendisi için bütüncül bir saygı beslediler; onlar, mevcudiyetinden bile korku duymuşlardı. İsa korkusuzdu. O her şeye ek olarak kutsal coşku ile doluydu; ancak o hiçbir zaman yobaz hale gelmedi. Duygusal olarak etkindi, ancak hiçbir zaman uçarı değildi. O hayalciydi, ancak her zaman işlevseldi. O dürüst bir biçimde yaşamın gerçeklikleriyle yüzleşti; ancak o hiçbir zaman sıkıcı veya sıradan olmadı. O cesurdu, ancak hiçbir zaman dikkatsiz değildi; tedbirliydi, ancak bunu hiçbir zaman korkakça yapmamaktaydı. Duygudaştı, ancak aşırı derecede duygusal değildi; benzersizdi, ancak tuhaf değildi. Dindardı, ama sofu değildi. Ve o, bu derecede fazlasıyla dingindi, çünkü kusursuz bir biçimde bütünleşmişti.
\vs p100 7:5 İsa’nın özgünlüğü engellenememişti. O; geleneğe bağlı kalmamış, dar ortak kabullere olan kölelikle kısıtlanmamıştı. İsa; kuşku duyulmayan güvenle konuşup, mutlak uzmanlıkla öğretti. Ancak onun muhteşem özgünlüğü, selefleri ve çağdaşlarının sahip oldukları öğretilerinde gerçeklik mücevherlerini görmezden gelmesine neden olmamıştı. Ve onun öğretilerinin en özgün olanı, korku duyma ve feda verme yerine sevgi ve bağışlamanın vurgusuydu.
\vs p100 7:6 İsa, bakış açısı bakımından oldukça geniş görüşlüydü. Takipçilerinden müjdeyi tüm insanlara duyurmalarını ısrarla öğütledi. O, her türlü dar görüşlülükten uzaktı. Onun anlayışlı kalbi, tüm insanlığı, hatta bir evreni kucakladı. Onun daveti her zaman “Her kim olursa, gelmesine izin verin” olmuştu.
\vs p100 7:7 İsa hakkında söylenen “Tanrı’ya güvendi” sözü doğruydu. İnsanlar arasında bir insan halinde, olabilecek en ulvi biçimde gökteki Yaratıcı’ya inandı. O Yaratıcısı’na, küçük bir çocuğun dünyasal ebeveynine güvendiği gibi güvendi. Onun inancı kusursuzdu, ancak hiçbir zaman aşırı bir biçimde kendinden emin değildi. Acımasız doğa nasıl kendini gösterirse göstersin, ve dünya üzerinde insanın refahına karşı ne kadar vurdumduymaz olursa olsun, İsa hiçbir zaman inancında bocalamaya düşmedi. O hayal kırıklığından etkilenmemekteydi, ve idamına karşı kayıtsızdı. O, belirgin başarısızlık karşısında değişmemekteydi.
\vs p100 7:8 O, doğuştan gelen yeteneklerinde ve elde ettikleri niteliklerinde ne kadar farklı olduklarını aynı zamanda tanıyarak insanları kardeşleri olarak sevmişti. “O iyi şeyler yapmak için uğraşmıştı.”
\vs p100 7:9 İsa, olağandışı biçimde neşeli bir insandı; ancak o, gözleri görmez ve sorgulamaz bir iyimser değildi. Onun ısrarlı tavsiyesindeki sürekli adı geçen sözcük “Neşenizi kaybetmeyin” olmuştu. O, Tanrı’ya olan şaşmaz güveni ve insana olan sarsılmaz inancı nedeniyle bu kendine güvenen tutumu koruyabilmişti. O her zaman etkileyici bir biçimde tüm insanları düşünmekteydi, çünkü o, onları sevmiş ve onlara inanmıştı. Buna rağmen o her zaman; yargılarında doğru olup, Yaratıcı’nın iradesini yerine getirmeye olan bağlılığında muhteşem bir biçimde kararlıydı.
\vs p100 7:10 Hâkim her zaman cömertti. O hiçbir zaman şunu söylemekten yorulmadı: “Vermek almaktan daha kutludur.” O, “Özgürce aldınız, özgürce verin” demişti. Ve yine de, tüm sınırsız cömertliğine rağmen, o hiçbir zaman savurgan veya müsrif olmadı. O, kurtuluşu alabilmeniz için inanmanız gerektiğini öğretti. “Zira, peşine düşen herkes onu alacaktır.”
\vs p100 7:11 O dürüsttü, ancak her zaman nazikti. “Eğer böyle olmasaydı, sana söylerdim” derdi. O açık sözlüydü, ancak her zaman candandı. O, günah işleyene olan sevgisinde ve günaha olan nefretinde sözünü sakınmayan biriydi. Ancak tüm bu hayranlık verici dürüstlüğü boyunca o yanılsamaz bir biçimde \bibemph{adildi}.
\vs p100 7:12 İsa, her ne kadar insan kederinin kadehinden zaman zaman uzun uzun içmiş olsa da, tutarlı bir biçimde neşeliydi. O korkusuz bir biçimde mevcudiyetin gerçeklikleriyle yüzleşmişti; yine de krallığın müjdesi için coşkuyla doluydu. Ancak o coşkusunu denetledi; bu coşku onu hiçbir zaman denetlemedi. O koşulsuz bir biçimde “Yaratıcı’nın işine” adanmıştı. Bu kutsal coşku, ruhsal olmayan kardeşlerinin onun kendinden geçmiş olduğunu düşünmelerine neden olmuştu; ancak seyirci olan evren kendisini, aklıbaşındalılığın örneği ve ruhsal yaşamın yüksek ölçütlerine olan yüce fani bağlılığın emsali olarak değerlendirmişti.
\vs p100 7:13 Celile’nin bu insanı, kederlerin bir insanı değildi; o, mutluluğun bir ruhuydu. O her zaman “Neşelenin ve fazlasıyla mutlu olun” demekteydi. Ancak, sorumluluğu gerektirdiğinde, “ölümün gölgesi vadisi” boyunca cesaretle yürümeye gönüllüydü. O sevinçliydi, ancak aynı zamanda alçak gönüllüydü.
\vs p100 7:14 Onun cesareti ancak sabrıyla dengelenmekteydi. Vaktinden önce hareket etmesi için üstüne gelindiğinde, sadece “Benim vaktim henüz gelmedi” şeklinde cevap vermekteydi. O hiçbir zaman bir acelecilik içerisinde değildi; onun sakinliği ulviydi. Ancak, o sıklıkla, kötülüğe öfkelenir, günahı hoş görmez bir tutum sergilemişti. O çoğu kez, dünya üzerindeki çocuklarının refahına zarar verecek olan şeye karşı koyacak kadar çok etkilenmişti. Ancak, onun günaha olan öfkesi hiçbir zaman günahı işleyene karşı yönelmemişti.
\vs p100 7:15 Onun cesareti muhteşemdi; ancak o hiçbir zaman çılgınca hareket etmedi. Onun düsturu “Korku duyma” olmuştu. Onun mertliği üstün, onun cesareti çoğu zaman kahramancaydı. Ancak onun cesareti tedbirle ilişkili olup, nedensellikle denetlenmişti. O, inançtan doğan cesaretti, gözü görmez cürete ait dikkatsizlik değil. O, tam anlamıyla mertti; ancak hiçbir zaman düşüncesizce cüretkâr değildi.
\vs p100 7:16 Hâkim bir hürmet emsaliydi. Onun gençlik duası bile “Gökte olan Tanrımız, ismin kutsansın.” O, akranlarının yanlış ibadetine bile saygılıydı. Ancak bu kendisini, dini geleneklerini eleştirmekten ve insan inanışının hatalarını sert bir biçimde hedef almaktan alıkoymamıştı. O, gerçek kutsallığa hürmetkârdı; ve yine de şunu söyleyerek akranlarının ilgisini yerinde bir biçimde çekebilirdi: “Aranızdan kim benim günah işlediğimi ispat edebilir?”
\vs p100 7:17 İsa, iyi olduğu için büyüktü; ve yine de küçük çocuklar ile birlikte kardeşçe bütünleşmişti. O kişisel yaşamında nazik ve gösterişsizdi; ve yine de, bir evrenin kusursuzlaştırılmış insanıydı. Onun birliktelikleri kendisini, kendiliğinden gelen olarak adlandırmışlardı.
\vs p100 7:18 qwİsa, kusursuzca bütünleşmiş insan kişiliğiydi. Ve bugün 0, Celile’de olduğu gibi, fani deneyimini bütünleştirmeye ve insan çabalarını eş\hyp{}güdümsel hale getirmeye devam etmektedir. O; yaşamı bütünleştirmekte, kişiliği soylulaştırmakta ve deneyimi yalınlaştırmaktadır. O insan aklına; yüceltmek, dönüştürmek ve güzelleştirmek için girmektedir. “Herhangi bir kişi İsa Mesih’i yanına alırsa, o yeni bir varlıktır; eski şeyler gelip geçmektedir; bakın, her şey yenilenmektedir.” sözü kelimenin tam anlamıyla doğrudur.
\vs p100 7:19 [Nebadon’un bir Melçizedek unsuru tarafından sunulmuştur.]
