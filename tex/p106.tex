\upaper{106}{Gerçekliğin Kâinat Düzeyleri}
\vs p106 0:1 Yükseliş fanisinin, İlahiyat’ın ilişkilerinden kâinatsal gerçekliğin doğumu ve dışavurumuna kadar belirli bir bilgiye sahip olması yeterli değildir; o aynı zamanda, kendisi ve, potansiyel ve mevcut gerçekliklere ait olan, var oluşsal ve deneyimsel gerçekliklerin sayısız düzeyleri arasında bulunan ilişkilere dair belirli bir kavrayışa sahip olmalıdır. İnsanın mekânsal yöneliminin, onun kâinatsal kavrayışının ve onun ruhsal yönlenişinin tümü; kâinat gerçekliklerinin daha iyi bir kavranışına ek olarak onların karşılıklı\hyp{}ilişkilenim, birleşim ve bütünleşim yöntemleri tarafından gelişir.
\vs p106 0:2 Mevcut asli evren ve ortaya çıkış halindeki üstün evren; işlevsel etkinliğin belirli seviyeleri üzerinde karşılıksal olarak mevcut olan, gerçekliğin birçok türü ve fazı tarafından meydana gelmiştir. Bu çok katmanlı mevcudiyetlere ve potansiyellere bu makaleler içinde daha önce değinilmiştir; ve, onlar şimdi, şu sınıflandırmalar altında kavramsal kolaylık için toplanmıştır:
\vs p106 0:3 1.\bibnobreakspace \bibemph{Tamamlanmamış sınırlılıklar}: Bu faz, Urantia fanilerinin mevcut düzeyi olan asli evrenin yükseliş yaratılmışlarının mevcut düzeyidir. Bu seviye; gezegensel insandan başlayarak, nihai sona erişmiş unsurlara kadar, onları içine almayan bir biçimde, yaratılmış mevcudiyetinden oluşur. O; öncül fiziksel başlangıçlardan başlayarak, ışık ve yaşam altında istikrara kavuşma düzeyine kadar, onu içine almayan bir biçimde, evrenlerle ilgilidir. Bu seviye, zaman ve mekân içerisinde yaratılmış etkinliğinin mevcut çepersel konumunu oluşturur. Bu faz, Cennet’den dışa doğru hareket eder görünümdedir; zira, asli evrenin ışık ve yaşama olan erişimine tanık olacak mevcut evren çağının kapanışı, aynı zamanda ve kesin bir biçimde, ilk dışsal mekân düzeyindeki gelişimsel büyümenin belirli bir düzeyinin ortaya çıkışına tanıklık edecektir.
\vs p106 0:4 2.\bibnobreakspace \bibemph{Nadir sınırlılıklar}. Bu faz, mevcut evren çağının kapsamı içinde açığa çıkarılmış biçimdeki nihai son olarak --- nihai sona erişen tüm deneyimsel yaratılmışların mevcut düzeyidir. Evrenler bile, hem ruhsal hem de fiziksel olarak nadir düzeye erişebilir. Ancak, neye göre nadir? olarak --- “nadir” kavramının kendisi göreceli bir terimdir. Ve, mevcut evren çağında nihai olarak görünen bir biçimde nadir, gelecekteki çağlar bakımından gerçek bir başlangıçtan daha fazlası olmayabilir. Havona’nın bazı fazları, nadir düzeyde bulunan görünüme sahiptir.
\vs p106 0:5 3.\bibnobreakspace \bibemph{Aşkınlar}. Bu sınırlılık\hyp{}ötesi düzey (varlığını sınırlı\hyp{}öncesinden alan bir biçimde), sınırlılık ilerleyişini takip etmektedir. O, sınırlı başlangıçların sınırlılık\hyp{}öncesi doğumuna ek olarak gözle görülen tüm sınırlı sonların veya nihai sonların sınırlılık\hyp{}sonrası önemi anlamına gelmektedir.
\vs p106 0:6 4.\bibnobreakspace \bibemph{Nihailer}. Bu seviye; üstün evrenin önemi düzeyini içine almakta olup, tamamlanmış üstün evrenin nihai son düzeyi üzerinde etkiye sahiptir. Cennet\hyp{}Havona (özellikle Yaratıcı’nın dünyalarının döngüsü olarak), birçok açıdan nihai önem içerisinde bulunmaktadır.
\vs p106 0:7 5.\bibnobreakspace \bibemph{Ortak\hyp{}mutlaklıklar}. Bu seviye, deneyimliliklerin yaratılmış dışavurumunun bir üstün\hyp{}ötesi evren düzlemine olan uygulanışı anlamına gelmektedir.
\vs p106 0:8 6.\bibnobreakspace \bibemph{Mutlaklıklar}. Bu seviye, yedi varoluşsal Mutlaklık’ın ebediyet mevcudiyeti anlamına gelmektedir. O aynı zamanda, ilişkilenimsel nitelikli deneyimsel erişimin belirli bir aşamasını içine alabilir; ancak, eğer bu olasılık gerçeklik kazanırsa, biz bunun nasıl oluştuğunu anlamamaktayız, o galiba kişiliğin sahip olduğu iletişimsel potansiyel vasıtasıyla gerçekleşmektedir.
\vs p106 0:9 7.\bibnobreakspace \bibemph{Sonsuzluk}. Bu seviye, varoluşsal\hyp{}öncesi ve deneyimsel\hyp{}sonrasıdır. Sonsuzluğun koşulsuz bütünlüğü, tüm başlangıçların öncesini ve tüm nihai sonlardan sonrasını kapsayan bir varsayımsal gerçekliktir.
\vs p106 0:10 Gerçekliğin bu seviyeleri; mevcut evren çağının simgeleşmiş kavramları ve fani bakış açısı için elverişli bir ortak noktadır. Fani\hyp{}bakış\hyp{}açısından\hyp{}farklı\hyp{}olarak ve başka evren çağlarının bakış açısından gerçekliğe bakmanın birçok diğer yolu bulunmaktadır. Bu nedenle, bu makalede sunulmuş kavramsallaşmalar, tamamiyle şu etkenler tarafından belirlenmiş ve kısıtlanmış olarak, görecelidir:
\vs p106 0:11 1.\bibnobreakspace Fani dilinin sınırlılıkları.
\vs p106 0:12 2.\bibnobreakspace Fani aklın sınırlılıkları.
\vs p106 0:13 3.\bibnobreakspace Yedi aşkın\hyp{}evrenin kısıtlı gelişimi.
\vs p106 0:14 4.\bibnobreakspace Cennet’e olan fani yükselişle ilgisi bulunmayan aşkın\hyp{}evren gelişiminin altı başat amacı hususundaki bilgisiz konumunuz.
\vs p106 0:15 5.\bibnobreakspace Kısmi bir ebediyet bakış açısını kavramaya yetkin olmayan doğanız.
\vs p106 0:16 6.\bibnobreakspace Yalnızca, yedi aşkın\hyp{}evrene ait evrimsel gerçekleşimin mevcut çağı ile ilgili olmayan bir biçimde; tüm evren çağlarıyla ilişkili olarak kâinatsal evrim ve nihai sonu tasvir etmedeki imkânsızlık.
\vs p106 0:17 7.\bibnobreakspace Başlangıçlardan öncesi ve nihai sonlardan sonrası dönem olarak --- her yaratılmışın, mevcudiyet\hyp{}önceleri veya mevcudiyet\hyp{}sonralarının gerçekte ne anlama geldiğini kavramadaki yetkinsizliği.
\vs p106 0:18 Gerçeklik büyümesi, birbirini takip eden evren çağlarının sahip olduğu koşullar tarafından belirlenir. Merkezi evren, Havona çağı içerisinde hiçbir evrimsel değişiklik sürecinden geçmemiştir; ancak, aşkın\hyp{}evren çağının mevcut çağları içerisinde, evrimsel aşkın\hyp{}evrenlerin eşgüdümünün gerekli kıldığı belirli ilerlemesel değişiklik süreçlerinden geçmektedir. Şu an evrimleşmekte olan yedi aşkın\hyp{}evren bir zaman zarfında, mevcut evren çağının büyüme sınırına erişen bir biçimde ışık ve yaşamın istikrar düzeyine erişecektir. Ancak, kuşkusuz bir biçimde, ilk dışsal mekân düzeyinin çağı olan bir sonraki çağ; aşkın\hyp{}evrenleri mevcut çağın nihai son sınırlılıklarından kurtaracaktır. Her tamamlanış düzeyinin üzerine sürekli olarak, ilave doygunluk bütünlüğü eklenmektedir.
\vs p106 0:19 Bunlar; gerçekliğin sürekli yükselen seviyeleri üzerinde, nesnelerin, anlamların ve değerlerin kâinatsal büyümesine ve onların bir araya gelmesine dair bütünleşmiş bir kavramsallaşmayı sunmada karşılaştığımız kısıtlılıklardan bazılarıdır.
\usection{1.\bibnobreakspace Sınırlı İşlevliliklerin Başat İlişkilenimi}
\vs p106 1:1 Sınırlı gerçekliğin başat veya diğer bir değişle ruhani\hyp{}köken fazları doğrudan dışavurumunu, kusursuz kişilikler olarak yaratılmış düzeylerinde ve kusursuz Havona yaratımı olarak evren düzeylerinde bulur. Deneyimsel İlahiyat bile, böylelikle, Havona içindeki Yüce olan Tanrı’nın ruhaniyet kişiliğinde dışa vurulur. Ancak, sınırlı olanın ikincil, evrimsel, zaman\hyp{}ve\hyp{}madde\hyp{}tarafından\hyp{}belirlenmiş fazları, kâinatsal bir biçimde, sadece büyüme ve erişimin bir sonucu olarak bütünleşmiş hale gelir. Nihai olarak tüm ikincil veya diğer bir değişle kusursuzlaşmakta olan sınırlılıklar, başat kusursuzluğunkine eşit olan bir seviyeye erişme potansiyelindedirler; ancak, bu türden nihai son, merkezi yaratım içinde kalıtımsal bir biçimde bulunmayan bir yapıcı aşkın\hyp{}evren niteliği olarak bir zaman gecikmesine tabidir. (Bizler üçüncü düzey sınırlılıkların mevcudiyetini bilmekteyiz, ancak onların birleşim yöntemi henüz açığa çıkarılmamıştır.)
\vs p106 1:2 Kusursuzluk erişimine olan bu engel biçiminde bu aşkın\hyp{}evren zaman gecikmesi, evrimsel büyümeye olan yaratılmış katılımını sağlamaktadır. O böylelikle, bahse konu yaratılmışın evrimi sürecinde yaratılmışın Yaratan ile olan ortak\hyp{}birlikteliğine girmesini mümkün kılar. Ve, genişleyen büyümenin bu dönemleri boyunca, tamamlanmamışlık; Yedi Katmanlı Tanrı’nın hizmeti vasıtasıyla kusursuz ile ilişik hale gelir.
\vs p106 1:3 Yedi Katmanlı Tanrı, mekânın evrimsel evrenleri içinde Cennet İlahiyatı’nın zamanın engellerini tanıyışını simgelemektedir. Cennet’den ne kadar uzak, uzayda ne kadar derin bir mekânda bir maddi kurtuluş kişiliği kökenine sahip olursa olsun; Yedi Katmanlı Tanrı orada mevcut bulunup, bu tür tamamlanmamış, mücadele halindeki ve evrimsel bir yaratılmış için gerçeklik, güzellik ve iyiliğin sevgi dolu ve bağışlayıcı hizmetine katılmış bulunacaktır. Yedi Katmanlı’nın kutsallık hizmeti; içe yönelik doğrultuda, Ebedi Evlat vasıtasıyla Cennet Yaratıcısı’na, ve, dışa yönelik doğrultuda, Zamanın Ataları vasıtasıyla --- Yaratan Evlatlar --- olarak evren Yaratılmışları’na doğru ulaşır.
\vs p106 1:4 Kişisel nitelikte ve ruhsal ilerleme vasıtasıyla yükseliş halinde bulunan insan, Yedi Katmanlı İlahiyat’ın kişisel ve ruhsal kutsallığını bulmaktadır; ancak, orada, kişiliğin ilerleyişi ile ilgili olmayan Yedi Katmanlı’nın diğer fazları bulunmaktadır. Bu İlahiyat’ın kutsallık nitelikleri, Yedi Üstün Ruhaniyet ve Bütünleştirici Bünye arasındaki irtibat içinde mevcut bir biçimde bütünleşmiş haldedir; ancak, onlar, Yüce Varlık’ın ortaya çıkmakta olan kişiliği içinde ebedi bir biçimde bütünleşme nihai sonuna sahiplerdir. Yedi Katmanlı İlahiyat’ın diğer fazları çeşitli bir biçimde, mevcut evren çağı içerisinde bir araya gelmiştir; ancak, onların tümü benzer bir biçimde, Yüce içinde bütünleşme nihai sonuna sahiptir. Yedi Katmanlı, tüm fazları içerisinde, mevcut asli evrenin işlevsel gerçekliğine ait göreceli birliğin kökenidir.
\usection{2.\bibnobreakspace İkincil Yüce Sınırlılık Bütünleşimi}
\vs p106 2:1 Yedi Katmanlı Tanrı işlevsel bir biçimde sınırlı evrimi eşgüdümsel hale getirirken, Yüce Varlık benzer bir biçimde nihai son erişimini nihai olarak bir bütün hale getirir. Yüce Varlık, ruhani bir çekirdek etrafındaki fiziksel evrime ek olarak fiziksel evrimin döngüsel ve burgaçsal âlemleri üzerinde ruhaniyet çekirdeğinin nihai üstünlüğü biçiminde --- asli evren evriminin ilahiyat sonudur. Ve, tüm bunların hepsi, kişiliğin emirleri uyarınca gerçekleşmektedir: Cennet kişiliği en yüksek anlamda, Yaratan kişiliği evrensel anlamda, fani kişiliği insansı anlamda, Yüce kişiliği sonuçsal veya deneyimsel bütünlük anlamında.
\vs p106 2:2 Yüce’nin kavramsallaşması; evrimsel gücün ruhaniyet kişiliği ile, ve onun üstünlüğüyle, olan bütünleşimi olarak --- ruhaniyet bireyinin, evrimsel gücün ve güç\hyp{}kişilik bileşiminin farklılaşan tanıyışını sağlamak zorundadır.
\vs p106 2:3 Ruhaniyet, son kertede, Cennet’den Havona vasıtasıyla gelmektedir. Enerji\hyp{}maddesi; göründüğü biçimiyle uzayın derinlerinde evirilmekte olup, Tanrı’nın Yaratan Evlatları ile ilişki dâhilinde Sınırsız Ruhaniyet’in çocukları tarafından güç olarak düzenlenir. Ve, tüm bunların hepsi deneyimseldir; o, Yaratan kutsallıklarını ve evrimsel yaratılmışları bile içeren bir biçimde yaşayan varlıkların geniş bir kapsamını içine alan zaman ve mekândaki etkileşimdir. Yaratan kutsallıklarının asli evren içindeki güç üstünlüğü, yavaş bir biçimde, zaman\hyp{}mekân yaratılmışlarının evrimsel nitelikli denge ve istikrar sürecini içine alacak şekilde genişlemektedir. O, Kâinatın Yaratıcısı’nın Düzenleyici bahşedilişinden Cennet Evlatları’nın yaşam bahşedilişine kadar zaman ve mekân içinde kutsallık erişiminin bütüncül kapsamını içine alır. Bu kazanılmış güç, sergilenmiş güç, deneyimsel güçtür; o, Cennet İlahiyatları’nın ebediyet gücünün, anlaşılamaz gücünün ve deneyimsel gücünün zıddı konumundadır.
\vs p106 2:4 Yedi Katmanlı Tanrı’nın kutsallık kazanımlarından doğan bu deneyimsel gücün kendisi; evrimleşen yaratılmışların elde edilmiş deneyimsel üstünlüğünün her\hyp{}şeye\hyp{}muktedir gücü olarak --- bütünleştirici nitelikteki --- birleştirme vasıtasıyla kutsallığın bir araya getirici niteliklerini sergiler. Ve, bu her\hyp{}şeye\hyp{}muktedir güç, sonuçsal bir biçimde, Yüce olarak Tanrı’nın Havona mevcudiyetine ait ruhaniyet kişiliği ile bütünlük içerisinde Havona dünyalarının dışsal kemerinin yönlendirici ana âlemi üzerinde ruhaniyet\hyp{}kişilik birleşimini bulur. Böylelikle, deneyimsel İlahiyat; zaman ve mekânın güç ürünü ile ruhaniyet mevcudiyetine ek olarak merkezi yaratım içinde ikamet eden kutsal kişiliği birleştirerek uzun evrimsel mücadelesini tamamlar.
\vs p106 2:5 Böylelikle Yüce Varlık nihai olarak, bu nitelikler ile ruhaniyet kişiliğini birleştirirken, zaman ve mekân içinde evrimleşen her şeyi içine alan konuma erişir. Yaratılmışlar, faniler bile, bu görkemli etkileşimin katılımcıları oldukları için, Yüce’yi tanıma ve Yüce’yi bu türden bir evrimsel İlahiyat’ın gerçek çocukları olarak algılama yetkinliğine kesin bir biçimde erişir.
\vs p106 2:6 Nebadon’un Mikail’i, Cennet Yaratıcısı gibidir, çünkü kendisi onun Cennet kusursuzluğunu paylaşmaktadır; böylelikle, evrimsel faniler bir zaman zarfında deneyimsel Yüce ile olan içkin yakınlığı elde edeceklerdir, zira onlar gerçek anlamıyla onun sahip olduğu evrimsel kusursuzluğu paylaşacaktır.
\vs p106 2:7 Yüce olan Tanrı, deneyimseldir; bu nedenle, o, tamamiyle deneyimlenebilir. Yedi Mutlak’ın deneyimsel gerçeklikleri, deneyim yöntemiyle algılanabilir nitelikte değildir; yalnızca, Yaratıcı, Evlat ve Ruhaniyet’in\bibemph{ kişilik nitelikleri}, dua\hyp{}ibadet tutumu içinde sınırlı yaratılmışın kişiliği tarafından kavranabilir.
\vs p106 2:8 Yüce Varlık’ın tamamlanmış güç\hyp{}kişilik bileşimi içinde, benzer bir biçimde ilişkilendirilebilecek olan yedi üçlüklerin sahip olduğu mutlaklığın tümü ilişkilenecektir; ve, evrimin bu görkemli kişiliği, tüm sınırlı kişilikler tarafından deneyimsel olarak erişilebilir ve anlaşılabilir hale gelecektir. Yükseliş unsurları ruhaniyet mevcudiyetinin düşünülen yedinci aşamasına eriştiklerinde, onun içinde, deneyimlenebilir nitelikte bulunan Yüce Varlık içerisinde alt\hyp{}mutlak seviyelerinde açığa çıkarıldığı haliyle üçlüklerin ve sonsuzluğuna ait yeni bir anlam\hyp{}değerinin farkındalığını deneyimleyeceklerdir. Ancak, olası en yüksek gelişimin bu aşamalarına olan erişim, muhtemel bir biçimde, ışık ve yaşam altında bütüncül asli evrenin eşgüdümsel olarak istikrara kavuşmasını bekleyecektir.
\usection{3.\bibnobreakspace Aşkın Üçüncül Gerçeklik İlişkilenimi}
\vs p106 3:1 Absonit mimarları tasarımı mevcut kılmaktadırlar; Yüce Yaratanlar, onu yerine getirmektedirler; Yüce Varlık, Yüce Yaratanlar tarafından yaratıldığı şekliyle ve Üstün Mimarlar tarafından mekân içinde öngörüldüğü biçimiyle, onun bütünlüğünü tamamlayacaklardır.
\vs p106 3:2 Mevcut evren çağı boyunca, üstün evrenin idari eşgüdümü, Üstün Evren’in Mimarları’nın faaliyetidir. Ancak, Her\hyp{}Şeye\hyp{}Muktedir Yüce’nin mevcut evren çağının sonlanış anında ortaya çıkışı; evrimsel sınırlılığın, deneyimsel nihai sonun ilk aşamasına eriştiğini simgeleyecektir. Bu gelişme kesin bir biçimde; Yüce Yaratanlar’ın, Yüce Varlık’ın ve Üstün Evren’in Mimarları’nın ortak birliği olarak --- ilk deneyimsel Kutsal Üçleme’nin tamamlanmış faaliyetiyle sonuçlanacaktır. Bu Kutsal Üçleme, üstün yaratımın daha ileri evrimsel bütünleşimini yerine getirme nihai sonuna sahiptir.
\vs p106 3:3 Cennet Kutsal Üçlemesi, gerçekten de sonsuzluk unsurlarından bir tanesidir; ve, hiçbir Kutsal Üçleme, bu özgün Kutsal Üçlemeyi içermeden muhtemel bir biçimde sonsuz olamaz. Ancak, özgün Kutsal Üçleme, mutlak İlahiyatlar’ın ayrıcalıklı ilişkileniminin bir nihai sonucudur; alt\hyp{}mutlak varlıklar, bu başat ilişkilem ile hiçbir bağlantıya sahip bulunmamaktaydı. İlerleyen süreçlerde ortaya çıkmakta olan ve deneyimsel nitelikli Kutsal Üçlemeler, yaratılmış kişiliklerinin bile katkılarını içine almaktadır. Kesin bir biçimde bu durum; bünyesinde, Yüce Yaratan üyeleri arasında Üstün Yaratan Evlatları’nın bireysel mevcudiyetinin bu Kutsal Üçleme ilişkilemi \bibemph{içerisinde} mevcut ve özgün yaratılmış deneyiminin eş zamanlı gerçekleşen mevcudiyetinin habercisi olduğu, Nihai Kutsal Üçleme için doğrudur.
\vs p106 3:4 İlk deneyimsel Kutsal Üçleme, en yüksek nihayetliklerin topluluksal erişimini sağlamaktadır. Topluluk ilişkilenimleri, bireysel yetkinlikleri öngörmeye, ve hatta onları aşmaya, yetkin hale getirilmişlerdir; ve, bu durum, sınırlı düzeyin ötesinde bile doğrudur. Gelecek çağlar içerisinde, yedi aşkın\hyp{}evren ışık ve yaşam altında istikrara kavuştuğunda, Kesinliğin Birlikleri, kuşkusuz bir biçimde; Nihai Kutsal Üçleme tarafından salık verildiği biçimiyle, ve, Yüce Varlık içinde güç\hyp{}kişilik bütünleşmişlikleri olarak, Cennet İlahiyatları’nın amaçlarını duyuruyor olacaklardır.
\vs p106 3:5 Geçmiş ve gelecek ebediyetin devasa evren gelişmelerinin tümü boyunca bizler, Kâinatın Yaratıcısı’nın kavranabilen niteliklerinin genişlemesini tespit etmekteyiz. BEN olarak, bizler felsefi bir biçimde onun bütüncül sonsuzluğunun yayılımını düşünmekteyiz; ancak, hiçbir yaratılmış, bu tür bir düşünceyi bütünüyle deneyimsel bir biçimde kavramaya yetkin değildir. Evren genişledikçe, ve, çekim ve sevgi, düzenlenmekte olan\hyp{}zaman mekânına ulaştıkça, bizler, İlk Kaynak ve Merkez’e dair daha çok şeyi anlamaya yetkin hale gelmekteyiz. Bizler, çekim etkisinin, Koşulsuz Mutlak’ın mekân mevcudiyetine girmekte olduğunu gözlemlemekteyiz; ve, bizler, ruhani yaratılmışların İlahi Mutlak’ın kutsallık mevcudiyeti içinde evrimleşip genişlerken, hem kâinatsal hem de ruhsal evrimin akıl ve deneyim vasıtasıyla Yüce Varlık olarak sınırlı ilahiyat düzeylerinde bütünleştiğini ve Nihai Kutsal Üçleme olarak aşkın seviyelerde eşgüdüm halinde olduğunu tespit etmekteyiz.
\usection{4.\bibnobreakspace Nihai Dördüncül Bütünleşme}
\vs p106 4:1 Cennet Kutsal Üçlemesi, kesin bir biçimde, olası en yüksek anlamda eşgüdümde bulunur; ancak o, bu anlamda, kendi kendisini sınırlayan bir mutlak olarak faaliyet gösterir; deneyimsel Nihai Kutsal Üçleme, bir aşkın olarak aşkınlığı eşgüdümsel hale getirir. Ebedi gelecekte bu deneyimsel Kutsal Üçleme iradesi, çoğalan bütünlük vasıtasıyla, Nihai İlahiyat’ın mevcut hale gelen varlığını ileri bir biçimde etkinleştirir.
\vs p106 4:2 Her ne kadar Nihai Kutsal Üçleme üstün yaratımı eşgüdümsel hale getirmenin nihai sonuna sahipse de, Nihai olarak Tanrı, asli evrenin bütünün yönlendirilişine ait aşkın güç\hyp{}kişileşmesidir. Nihai’nin tamamlanmış mevcut hale gelişi; üstün yaratımın tamamlanışı anlamına gelip, bu aşkın İlahiyat’ın bütüncül ortaya çıkışını simgelemektedir.
\vs p106 4:3 Nihai’nin bütüncül ortaya çıkışı tarafından hangi değişikliklerin gerçekleşmeye başlayacağını bilmemekteyiz. Ancak, Yüce şimdi ruhsal ve kişisel bir biçimde Havona içerisinde mevcut haldeyken, benzer bir biçimde, Nihai de orada mevcut haldedir, ama bu mevcudiyet absonit ve kişisel\hyp{}ötesi anlamdadır. Ve, sizler, her ne kadar bulundukları mevcut konum veya sahip oldukları faaliyet hakkında bilgilendirilmemiş olsanız da, Nihai’nin Sınırlı Vekilleri’nin mevcudiyeti hakkında bilgilendirilmiş konumda bulunmaktasınız.
\vs p106 4:4 Ancak, Nihai İlahiyat’ın ortaya çıkışının beraberinde getirdiği idari sonuçlardan bağımsız olarak, onun aşkın kutsallığına ait kişisel değerler, bu İlahiyat seviyesinin gerçekleşimine katılmış tüm kişilikler tarafından deneyimlenebilir olacaktır. Sınırlılığın aşkınlığı, yalnızca, nihai kazanıma götürebilir. Nihai olarak Tanrı, zaman ve mekânın aşkınlığında mevcuttur; ancak o yine de, her ne kadar mutlaklıklar ile işlevsel ilişkilem için içkin yetiye sahipse de, alt\hyp{}mutlaktır.
\usection{5.\bibnobreakspace Ortak\hyp{}Mutlak veya Beşinci\hyp{}Faz İlişkilemi}
\vs p106 5:1 Yüce evrimsel\hyp{}deneyimsel gerçekliğin tamamlayıcısıyken bile, Nihai aşkın gerçekliğin zirve noktasıdır. Ve, bu iki deneyimsel İlahiyat’ın mevcut ortaya çıkışı, ikinci deneyimsel Kutsal Üçleme’nin temelini oluşturur. Bu; Yüce olarak Tanrı’nın, Nihai olarak Tanrı’nın ve açığa çıkarılmamış Kâinat Nihai Sonu’nun Tamamlayıcısı’nın birliği olarak Mutlak Kutsal Üçleme’dir. Ve, bu Kutsal Üçleme kuramsal olarak; İlahiyat, Kâinatsal ve Koşulsuz olarak --- kişiliğin Mutlaklıkları’nı etkinleştirme yetisine sahiptir. Ancak, bu Mutlak Kutsal Üçleme’nin tamamlanmış bütünlüğü, yalnızca; Havona’dan dördüncü ve en uç mekân düzeyine kadar bütün üstün evrenin tamamlanmış evriminden sonra gerçekleşebilir haldedir.
\vs p106 5:2 Bu deneyimsel Kutsal Üçlemeler’in; sadece deneyimsel İlahiyat’ın kişilik nitelikleriyle değil, aynı zamanda, onların eriştiği İlahiyat bütünlüğünü niteleyen kişisel\hyp{}niteliklerden\hyp{}başka her şey ile ilişkili olduğunun altı çizilmelidir. Her ne kadar bu sunum başat bir şekilde kâinatın bütünleşmesine ait kişisel fazları merkezine alırken, kâinat âlemlerinin tümünün kişilik\hyp{}dışı yönleri benzer bir biçimde; güç\hyp{}kişilik bileşiminin mevcut an içinde Yüce Varlık’ın evrimi ile ilişkili biçimde ilerlemesiyle sergilendiği gibi, bütünleşme sürecinden geçme nihai sonuna sahiptir. Yüce’nin ruhaniyet\hyp{}kişisel nitelikleri, Her\hyp{}Şeye\hyp{}Muktedir’in güç ayrıcalıklarından ayrıştırılamaz niteliktedir; ve, onun her ikisi de, Yüce aklın bilinmeyen potansiyeli tarafından tamamlanır. Buna ek olarak ne de, bir birey niteliğinde Nihai olarak Tanrı, Nihai İlahiyat’ın kişisel\hyp{}olmayan\hyp{}diğer\hyp{}niteliklerinden ayrı bir biçimde düşünülebilir. Ve, mutlak düzeyde İlahiyat ve Koşulsuz Mutlak, Kâinatsal Mutlak’ın mevcudiyetinde ayrılamaz ve ayrıştırılamaz konumdadır.
\vs p106 5:3 Kutsal Üçlemeler, kendileri içinde ve bütünlükleri bakımından, kişisel değillerdir; ancak, buna ek olarak onlar ne de kişiliğe karşı gelmektedir. Bunun yerine onlar, kişiliği kapsamakta ve onu, topluca gerçekleştirilen bir anlamda, kişisel\hyp{}olmayan faaliyetlerle ilişkili hale getirmektedirler. Kutsal Üçlemeler, buna ek olarak, her zaman \bibemph{ilahiyat} gerçekliğidir, hiçbir zaman \bibemph{kişilik} gerçekliği değildir. Bir Kutsal Üçleme’nin kişilik yönleri, bireysel üyelerini içinde içkindir; ve, bireysel kişiler olarak onlar, bu kutsal üçlemenin \bibemph{kendisi değillerdir}. Yalnızca ortak bir biçimde onlar kutsal üçlemedir; bu kutsal üçlemenin \bibemph{tam da kendisidir}. Ancak, kutsal üçleme her zaman, içine aldığı ilahiyatın tümünü temsil etmektedir; kutsal üçleme ilahiyat birlikteliğidir.
\vs p106 5:4 İlahiyat, Kâinatsal ve Koşulsuz olarak --- üç Mutlak kutsal üçleme değildir; zira, onların tümü, ilahiyat değildir. Sadece ilahlaştırılmış olan, kutsal üçleme haline gelebilir; tüm diğer ilişkilenimler üçleme birlikleri veya üçlükleridir.
\usection{6.\bibnobreakspace Mutlak veya Altıncı\hyp{}Faz Bütünleşimi}
\vs p106 6:1 Üstün evrenin mevcut potansiyeli, her ne kadar nihaiye oldukça yakın olsa da, neredeyse hiçbir biçimde mutlak değildir; ve, bizler, bir alt\hyp{}mutlak kâinatın kapsamı içinde mutlak anlam\hyp{}değerlerinin bütüncül açığa çıkarılışını elde etmeyi imkânsız olarak görmekteyiz. Bizler, böylelikle; üç Mutlak’ın sonsuz olasılıklarının bütüncül bir dışavurumunu düşünmeye veya İlahi Mutlak’ın şimdiki kişilik\hyp{}dışı düzeyi üzerinde Mutlak olarak Tanrı’nın deneyimsel kişilikleşmesini hayal etmeye çalışmada bile ciddi zorlukla karşılaşmaktayız.
\vs p106 6:2 Üstün evrenin mekân\hyp{}aşaması; Yüce Varlık’ın gerçekleşimi, Nihai Kutsal Üçleme’nin bütünleşimi ve bütünsel faaliyeti, Nihai olarak Tanrı’nın mevcut hale gelişi, ve hatta, Mutlak Kutsal Üçleme’nin başlangıçsal oluşumu için bile için yeterli görünmektedir. Ancak, bu ikincil deneyimsel Kutsal Üçleme’nin bütüncül faaliyeti ile ilgili bizlerin kavramsallaşmaları, oldukça engin üstün evrenin bile ötesinde bir şey anlamına gelir görüme sahiptir.
\vs p106 6:3 Eğer bizler; üstün evrenin ötesinde sınırlanamaz bir kâinat olarak --- bir kâinat\hyp{}sonsuzluğunu varsayarsak, ve, Mutlak Kutsal Üçleme’nin nihai gelişimlerinin eylemin bu türden uzak bir nihai\hyp{}ötesi aşamasında gerçekleşeceğini düşünürsek, bunun sonucunda, Mutlak Kutsal Üçleme’nin tamamlanmış faaliyetinin, sonsuzluğun yaratımlarında nihai dışavurumunu elde edeceğini ve \bibemph{tüm} potansiyellerin mutlak gerçekleşimini tamamlayacağını varsaymak mümkün hale gelmektedir. Gerçekliğin sürekli\hyp{}genişleyen birimlerinin bütünleşimi ve ilişkilenimi, bu şekilde ilişkilenen birimler içinde tüm gerçekliğin dâhil edilişiyle orantılı biçimde mutlaklık düzeyine yaklaşacaktır.
\vs p106 6:4 Başka bir şekilde ifade edildiği şekliyle: Mutlak Kutsal Üçleme, isminin de çağrıştırdığı biçimiyle, bütüncül faaliyeti içinde gerçekten de mutlaktır. Mutlak bir faaliyetin sınırlandırılmış, kısıtlı veya farklı bir biçimde engellenmiş bir temelde bütüncül dışavurumunu nasıl elde ettiğini bilmemekteyiz. Bu nedenle bizler, bu türden her bütünlük faaliyetinin (potansiyel bakımından) koşulsuz olacağını varsaymak zorundayız. Ve, her ne kadar bizler niceliksel ilişkiler hususunda çok emin olmasak da, en azından niceliksel bir bakış açısından, koşulsuzun aynı zamanda sınırlandırılmamış olacağı da görülecektir.
\vs p106 6:5 Tüm bunlar içinde bizler, buna rağmen, şundan eminiz: Varoluşsal Cennet Kutsal Üçlemesi sonsuzken, ve deneyimsel Nihai Kutsal Üçleme alt\hyp{}mutlakken, Mutlak Kutsal Üçleme bu kadar kolay bir biçimde sınıflandırabilen konumda bulunmamaktadır. Her ne kadar doğum ve oluşum bakımından deneyimsel olsa da, o kesin bir biçimde, potansiyelliğin varoluşsal Mutlaklıkları’na bağlıdır.
\vs p106 6:6 İnsan aklı için bu türden uzak ve insan\hyp{}ötesi kavramsallaşmaları kavramaya çalışmak neredeyse hiçbir biçimde faydalı olmasa da, Mutlak Kutsal Üçleme’nin ebediyet faaliyetinin, potansiyelliğin Mutlaklıkları’nın bir çeşit deneyimlenişiyle sonuçlanabileceğini önermekteyiz. Bu, Koşulsuz Mutlak ile ilgili olmasa da, Kâinatsal Mutlak ile ilgili bir makul çıkarım olarak görünecektir; en azından bizler, Kâinatsal Mutlak’ın yalnızca durağan ve potansiyel olmayıp aynı zamdan bu niteliklerin bütüncül İlahiyat anlamında da ilişkilenimsel olduğunu bilmekteyiz. Ancak, kutsallık ve kişiliğin düşünebilen değerleri ile ilgili olarak, bu varsayımsal gelişimler; İlahi Mutlak’ın kişisel hale gelişi ve bu kişilik\hyp{}ötesi değerlerin ortaya çıkışı ve --- deneyimsel İlahiyatlar’ın üçüncü ve sonuncusu olarak --- Mutlak olarak Tanrı’nın kişilik tamamlanışında içkin olan nihai\hyp{}kişisel anlamları çağrıştırmaktadır.
\usection{7.\bibnobreakspace Nihai Son’un Kesinliği}
\vs p106 7:1 Sonsuz gerçeklik bütünleşimine ait kavramsallaşmaları oluşturmadaki zorluklardan bazıları, içkin olarak; bu türden düşüncelerin tümünün, şimdiye kadar gerçekleşebilecek olan her şeyin bir tür deneyimsel gerçekleşimi olarak, kâinatsal gelişimin kesinliğine dair bir şeyi taşıması gerçekliğinden kaynaklanmaktadır. Ve, niceliksel sonsuzluğun nihai biçimde tamamiyle gerçekleşebilecek oluşu düşünülemez niteliktedir. Üç potansiyel Mutlaklık içinde her zaman, deneyimsel gelişime ait hiçbir niceliğinin hiçbir zaman tamamiyle tüketemeyeceği sayıda keşfedilmemiş olasılık bulunmaya devam edecektir. Ebediyetin kendisi, her ne kadar mutlak olsa da, mutlaklıktan fazlası değildir.
\vs p106 7:2 Nihai bütünleşmenin kesin olmayan bir kavramsallaşması bile, koşulsuz ebediyetin meyvelerinden ayrılamaz niteliktedir; ve, böylelikle neredeyse tamamen, düşünülebilen herhangi bir gelecek zaman içerisinde gerçekleştirilemez niteliktedir.
\vs p106 7:3 Nihai son, Cennet Kutsal Üçlemesi’ni oluşturan İlahiyatlar’ın özgür iradesel eylemi tarafından kurulmuştur; nihai son, mutlaklığı tüm gelecek gelişiminin olasılıklarını kapsayan üç büyük potansiyelin enginliği tarafından kurulmuştur; nihai son, muhtemel bir biçimde, Kâinat Nihai Son Tamamlayıcısı’nın eylemi tarafından tamamlanmaktadır, ve, bu eylem, muhtemel bir biçimde, Mutlak Kutsal Üçleme içinde Yüce ve Nihai’yi içine almaktadır. Her bir deneyimsel nihai son, en azından kısmi bir biçimde, deneyimleyen yaratılmışlar tarafından kavranabilir; ancak, sınırlı mevcudiyetliklere bağlı olan nihai bir son, neredeyse hiçbir şekilde kavranılmayan niteliktedir. Kesinlik nihai sonu, İlahi Mutlak’ı içine alır görünümdeki bir varoluşsal\hyp{}deneyimsel kazanımdır. Ancak, İlahi Mutlak, Kâinatsal Mutlak nedeniyle Koşulsuz Mutlak ile ebediyet ilişkisinde bulunur. Ve, olasılık bakımından deneyimsel olan bu üç Mutlaklık; gerçekte varoluşsal olup, gerçekten sonsuz bir biçimde --- kısıtlamasız, zamansız, mekânsız, sınırsız ve engelsiz olarak bundan daha fazlasıdır.
\vs p106 7:4 Hedefe erişimin imkânsızlığı, buna rağmen, bu türden varsayımsal nihai sonlar hakkında felsefi kuram geliştirmeyi engellememektedir. İlahi Mutlak’ın erişilebilir mutlak bir Tanrı olarak gerçekleşimi, farkındalık bakımından neredeyse imkânsız bir şey olabilir; yine de, bu türden bir kesinlik etkinlik süreci, kavramsal bir olasılık olarak varlığını korumaktadır. Koşulsuz Mutlak’ın düşünülemez belirli bir kâinat\hyp{}sonsuzluğuna olan katılımı, sonsuz ebediyetin geleceği içinde ölçütler ile tarif edilemez bir biçimde uzak olabilir; ancak, bu türden bir varsayım, yine de, geçerlidir. Faniler, morontialılar, ruhaniyetler, kesinlik unsurları, Aşkınlar, ve diğerleri, evrenlerin kendileri ile birlikte ve gerçekliğin tüm diğer fazlarıyla, kesin bir biçimde, \bibemph{değer bakımından mutlak olan bir nihai sona potansiyel olarak} sahiplerdir; ancak, bizler, herhangi varlığın veya evren iradesinin bir kez bile, bu türden bir nihai sonun tüm niteliklerine bütüncül bir biçimde erişeceğinden kuşku duymaktayız.
\vs p106 7:5 Yaratıcı kavrayışında ne kadar büyüyebileceğinizden bağımsız olarak aklınız her zaman; ebediyetin tüm döngüleri boyunca sürekli olarak düşünülemez ve kavranılamaz nitelikte kalmaya devam edecek keşfedilmemiş enginliği biçimindeki Yaratıcı\hyp{}BEN’in açığa çıkarılmamış sonsuzluğu tarafından şaşkınlığa uğrayacaktır. Tanrı’nın ne kadarına erişirseniz erişin, orada her zaman, aklınızdan bile geçirmeyeceğiniz mevcudiyeti olarak ondan çok daha fazlası varlığını sürdürmeye devam edecektir. Ve, bizler bunun, sınırlı mevcudiyetin nüfuz alanlarında olduğu gibi aşkın düzeylerde de aynı gerçekliği taşıdığına inanmaktayız. Tanrı’nın arayışının sonu yoktur!
\vs p106 7:6 Kesinsel bir anlamda Tanrı’ya erişmedeki bu türden yetkinsizlik hiçbir biçimde evren yaratılmışlarının cesaretini kırmamalıdır; gerçekten de, sizler, ebediyet mevcudiyeti içindeki mutlak düzeylerinde Yaratıcı olarak Tanrı’nın sonsuz gerçekleşimi Ebedi Evlat için ve Bütünleştirici Bünye için ne anlama geldiyse sizler için aynı anlama gelecek Yedi Katmanlı, Yüce ve Nihai’nin İlahiyat düzeylerine erişebilmekte olup, bunu gerçekleştirmektesiniz. Yaratılmışın şevkini kırmanın tam tersine Tanrı’nın sonsuzluğu; tüm sonsuz gelecek boyunca bir yükseliş kişiliğinin önünde, ebediyetin bile tüketemeyeceği veya sonlandıramayacağı kişilik gelişiminin ve İlahiyat ilişkileminin olasılıklarına sahip olacağının yüce güvencesi olmalıdır.
\vs p106 7:7 Asli evrenin sınırlı yaratılmışları için üstün evrenin kavramsallaşması neredeyse sonsuz olarak görülür; ancak, kuşkusuz bir biçimde, onların absonit mimarları bu kavramın gelecekle olan ilişkini ve sonu gelmez BEN içindeki hayal edilemez gelişmelerini kavramaktadır. Uzayın kendisi bile, ara\hyp{}uzayın sessiz bölgelerinin görece mutlaklığı \bibemph{içinde} bir kısıtlılık durumu olarak nihai bir durumdan başkası değildir.
\vs p106 7:8 Bütüncül asli evrenin nihai tamamlanışına ait düşünülemeyecek kadar uzak ebediyet anında, kuşkusuz olarak bizler geriye dönüp onun bütüncül tarihine; yalın bir değişle, yaşanmamış sonsuzluk içinde daha da büyük ve etkileyici başkalaşımlar için belirli bir sınırlı ve aşkın temellerin yaratımı olarak, yalnızca bir başlangıç olarak bakacağız. Bu türden bir gelecek ebediyet anında, üstün evren hala genç olarak görünecektir; o her zaman, sonu gelmez ebediyetin sınırsız olasılıkları karşısında her zaman genç olacaktır.
\vs p106 7:9 Sonsuz nihai son erişiminin olanaksızlığı, bu türden nihai son hakkında düşüncelerin yürütülmesini en ufak bir biçimde bile engellememektedir; ve, bizler, eğer üç mutlak potansiyel bir kez olsun tamamiyle gerçekleşir konuma gelirse, bütüncül gerçekliğin nihai bütünleşimini düşünmenin mümkün hale geleceğini tereddüt etmeden söyleyebiliriz. Bu gelişimsel gerçekleşme; birliktelikleri, ebediyetin askıya alınmış gerçeklikleri, tüm geleceğin bekler konumdaki olasılıkları ve daha fazlası biçimindeki BEN’in eylemsizliğini oluşturan üç potansiyel olarak Koşulsuz, Kâinatsal ve İlahiyat Mutlak’ın tamamlanmış gerçekleşimine dayanmaktadır.
\vs p106 7:10 Bu tür nihailikler, en hafif tabirle çok uzaktırlar; yine de, işleyiş biçimlerinde, kişiliklerinde ve üç Kutsal Üçleme’nin ilişkilenimlerinde bizler, Yaratıcı\hyp{}BEN’in yedi mutlak fazının yeniden birleşimi için kuramsal olasılığın varlığını tespit etmiş olduğumuza inanmaktayız. Ve, bu durum; varoluşsal düzeye ait Cennet Kutsal Üçlemesi’ne ek olarak deneyimsel doğa ve kökene ait birbirinin peşi sıra açığa çıkan iki Kutsal Üçleme’yi çevreleyen üç katmanlı Kutsal Üçleme’nin kavramsallaşmasıyla bizleri karşı karşıya getirir.
\usection{8.\bibnobreakspace Kutsal Üçlemeler’in Kutsal Üçlemesi}
\vs p106 8:1 Kutsal Üçlemeler’in Kutsal Üçlemesi’nin doğasının insan aklına tasvir edilmesi zordur; o, ebediyet gerçekleşimine ait kuramsal bir sonsuz içinde sergilendiği haliyle, deneyimsel sonsuzluğun bütünlüğüne ait mevcut toplamıdır. Kutsal Üçlemeler’in Kutsal Üçlemesi içerisinde, deneyimsel sonsuzluk, varoluşsal sonsuzluk ile tanımlanmaya erişmektedir; ve, her ikisi de, deneyimsel\hyp{}öncesi, mevcudiyet\hyp{}öncesi BEN içinde bir tekdir. Kutsal Üçlemeler’in Kutsal Üçlemesi, on beş üçlü birlik ve onunla ilişkili üçlükler içinde ima edilenlerin tümünün nihai dışavurumudur. Kesinlikler, ister varoluşsal ister deneyimsel olsun, göreceli varlıkların kavrayışı için zordur; bu nedenle, onlar her zaman, görecelikler olarak sunulmak zorundadır.
\vs p106 8:2 Kutsal Üçlemeler’in Kutsal Üçlemeleri, yedi faz içinde mevcuttur. O; insan düzeyinin çok üstündeki varlıkların hayal gücünü şaşkınlığa uğratan olasılıkları, olanaklılıkları ve kaçınılmazlıkları taşımaktadır. O, göksel filozoflar tarafından muhtemel bir biçimde kuşku duyulmamış sonuçlara sahiptir; zira, onun sonuçları üçleme birlikleri içindedir, ve, kutsal üçleme birlikleri, son kertede, düşünülemez niteliktedir.
\vs p106 8:3 Kutsal Üçleme Birlikleri’nin Kutsal Üçlemesi’nin tasvir edilebileceği birçok yol bulunmaktadır. Bizler, şu üç\hyp{}seviye kavramsallaşmasını sunmayı tercih etmekteyiz:
\vs p106 8:4 1.\bibnobreakspace Üç Kutsal Üçleme’nin seviyesi.
\vs p106 8:5 2.\bibnobreakspace Deneyimsel İlahiyat’ın seviyesi.
\vs p106 8:6 3.\bibnobreakspace BEN’in seviyesi.
\vs p106 8:7 Bunlar, artan birleşimlerin seviyeleridir. Gerçekte Kutsal Üçlemeler’in Kutsal Üçlemesi ilk seviyeyken, ikinci ve üçüncü seviyeler ilkinin birleşim\hyp{}türevleridir.
\vs p106 8:8 İLK SEVİYE: İlişkilenimin bu başlangıçsal düzeyinde, üç Kutsal Üçleme’nin; farklı İlahiyat kişilikleri topluluklarının kusursuz bir biçimde karşılıklı uyumlu hale gelmiş bütünlükte faaliyet gösterdiğine inanılmaktadır.
\vs p106 8:9 1.\bibnobreakspace \bibemph{Cennet Kutsal Üçlemesi}, Yaratıcı, Evlat ve Ruhaniyet olarak --- üç Cennet İlahiyatı’nın ilişkilenimi. Cennet Kutsal Üçlemesi’nin; mutlak bir işlev, aşkın bir işlev (Nihayet’in Kutsal Üçlemesi), ve sınırlı bir işlev (Yüceliğin Kutsal Üçlemesi) olarak --- üç katmanlı bir işlev anlamına geldiği hatırlanmalıdır. Cennet Kutsal Üçlemesi, her bir ve tüm zamanlarda tüm bunların her biri ve bütünüdür.
\vs p106 8:10 2.\bibnobreakspace \bibemph{Nihai Kutsal Üçleme}. Bu; Yüce Yaratanlar, Yüce olarak Tanrı ve Üstün Evren’in Mimarları’nın ilahiyat ilişkilemidir. Bu, bahse konu Kutsal Üçleme’nin kutsallık yönlerinin yeterli bir sunumu olsa da; buna rağmen kutsallık yönleri ile kusursuz bir biçimde eşgüdüm haline bulunur görünen, bu Kutsal Üçleme’nin diğer fazlarının da bulunduğunun altı çizilmelidir.
\vs p106 8:11 3.\bibnobreakspace \bibemph{Mutlak Kutsal Üçleme}. Bu, tüm kutsallık değerleri ile ilişkili Yüce olarak Tanrı’nın, Nihai olarak Tanrı’nın ve Kâinat Nihai Son Tamamlayıcısı’nın topluluğudur. Bu üçleme birlik topluluğunun diğer fazları, genişleyen kâinat içindeki kutsallıktan\hyp{}başka değerler ile ilgilidir. Ancak, bunlar; tıpkı deneyimsel İlahiyat’ın güç ve kişilik yönlerinin mevcut an içerisinde deneyimsel bileşim içinde olduğu gibi, kutsallık fazları ile bütünleşmektedir.
\vs p106 8:12 Kutsal Üçlemeler’in Kutsal Üçlemeleri içinde bu üç Kutsal Üçleme’nin ilişkilenimi, gerçekliğin olası bir sınırsız bütünleşimini sağlamaktadır. Bu topluluk nedenler, araçlar ve kesinlikleri taşımaktadır; düzensel oluşumları, gerçekleştirmeler ve tamamlamaları; başlangıçları, mevcudiyetleri ve nihai sonları. Yaratıcı\hyp{}Evlat ortak birlikteliği; Evlat\hyp{}Ruhaniyet haline, daha sonra Ruhaniyet\hyp{}Yüce haline, onu takiben Yüce\hyp{}Nihai ve Nihai\hyp{}Mutlak haline ve hatta --- gerçekliğin döngüsünün tamamlanışı olarak --- Mutlak ve Yaratıcı\hyp{}Sonsuzluk haline bile gelmiştir. Benzer bir biçimde, kutsallık ve kişilik ile oldukça doğrudan bir biçimde ilişkili olmayan diğer fazlar içinde, İlk Büyük Kaynak ve Merkez; varoluşsallıkların mutlaklığından deneyimliliklerin kesinliğine kadar --- kendiliğinden\hyp{}mevcudiyetin mutlaklığından kendini\hyp{}açığa\hyp{}çıkarışın sonsuzluğu boyunca kendini\hyp{}gerçekleştirmenin kesinliğine kadar, ebediyetin döngüsü etrafında gerçekliğin sınırsızlığını kendi\hyp{}kendine\hyp{}deneyimler.
\vs p106 8:13 İKİNCİ SEVİYE: Üç Kutsal Üçleme’nin eşgüdümü kaçınılmaz bir biçimde; köken olarak bu Kutsal Üçlemeler ile ilişkilenmiş olan, deneyimsel İlahiyatlar’ın ilişkilenimsel birliğini içine alır. Bu ikinci seviyenin doğası, zaman zaman şunlar olarak sunulmuştur:
\vs p106 8:14 1.\bibnobreakspace \bibemph{Yüce}: Bu; Cennet İlahiyatları’nın Yaratan\hyp{}Yaratıcı çocukları ile deneyimsel irtibat içindeki Cennet Kutsal Üçlemesi’nin birlikteliğine ait ilahiyat sonucudur. Yüce, sınırlı evrimin ilk aşamasının tamamlanışının ilahiyatsal bünyelişimidir.
\vs p106 8:15 2.\bibnobreakspace \bibemph{Nihai}. Bu; kutsallığın aşkın ve absonit kişilikleşimi olarak, ikinci Kutsal Üçleme’nin mevcut hale gelmiş birliğinin ilahi sonucudur. Nihai, birçok niteliğin çeşitli biçimlerde değerlendirilmiş bir birliğinden oluşmaktadır; ve, buna dair insan kavramsallaşması düzen yönlendiren, kişisel olarak deneyimlenebilen ve devinimsel gerilimi bütünleştiren nihayetin en azından bu fazlarını içermelidir; ancak, orada, mevcut hale gelmiş İlahiyat’ın açığa çıkarılmamış birçok diğer yönü de bulunmaktadır. Nihai ve Mutlak karşılaştırılabilir niteliğe sahipken, onlar özdeş değillerdir; buna ek olarak, ne de Nihai, Yüce’nin yalnızca bir genişlemiş bütünlüğü değildir.
\vs p106 8:16 3.\bibnobreakspace \bibemph{Mutlak}. Kutsal Üçlemeler’in Kutsal Üçlemesi’ne ait ikinci seviyenin üçüncü üyesinin kişiliğine dair geliştirilmiş birçok kuram bulunmaktadır. Mutlak olarak Tanrı, kuşkusuz bir biçimde, Mutlak Kutsal Üçleme’nin nihai işlevinin kişilik sonucu olarak bu ilişkileme katılmıştır; yine de, İlahi Mutlak, ebedi düzeye ait bir deneyimsel gerçekliktir.
\vs p106 8:17 Bu üçüncü üye ile ilgili kavramsal zorluk kökenini, bu tür bir üyeliğin varsayımının gerçekten yalnızca tek bir Mutlak anlamına geldiği gerçeğinden almaktadır. Kuramsal olarak; eğer bu türden bir etkinlik ortaya çıkacak olursa, bizler, üç Mutlak’ın bir tek olarak \bibemph{deneyimsel} bütünleşimine tanıklık etmeliyiz. Ve, bizlere, sonsuzluk içerisinde ve \bibemph{varoluşsal olarak}, yalnızca tek Mutlak’ın varlığı öğretilmiştir. Her ne kadar üçüncü üyenin hangi unsur olacağı hiçbir biçimde kesin olmasa da, onun, hayal edilemez nitelikte irtibat ve kâinat dışavurumunun bir türü içinde İlahiyat, Kâinatsal ve Koşulsuz Mutlak’dan meydana gelebileceğine dair düşünce sıklıkla yürütülmektedir. Kesinlikle, Kutsal Üçlemeler’in Kutsal Üçlemesi, neredeyse hiçbir biçimde, üç Mutlak’ın tamamlanmış birleşimi olmadan bütüncül faaliyete neredeyse herhangi bir biçimde erişecek konumda bulunmamaktadır; ve üç Mutlak, tüm sonsuz potansiyellerin bütüncül gerçekleşimi olmadan neredeyse hiçbir biçimde bütünleşemez.
\vs p106 8:18 Kutsal Üçlemeler’in Kutsal Üçlemesi’ne ait üçüncü üyeyi Kâinatsal Mutlak olarak düşünmek; bu kavramsallaşmanın, Kâinatsal’ı yalnızca durağan ve potansiyel değil aynı zamanda ilişkilendiren bir nitelikte tahayyül etmesi şartıyla, muhtemel bir biçimde gerçekliğin olası en düşük düzeydeki bir bozulumunu sunacaktır. Ancak, bizler, hala, bütüncül İlahiyat’ın işlevine ait yaratıcı ve evrimsel yönleri ile olan ilişkiyi kavrayamamaktayız.
\vs p106 8:19 Her ne kadar Kutsal Üçlemeler’in Kutsal Üçlemesi’ne ait tamamlanmış bir kavramsallaşma oluşturması bakımından zor olsa da, sınırlı bir kavramsallaşma o kadar zor değildir. Kutsal Üçlemeler’in Kutsal Üçlemesi’ne ait ikinci seviye temel bir biçimde kişisel olarak düşünülürse, Yüce olarak Tanrı’nın, Nihai olarak Tanrı’nın ve Mutlak olarak Tanrı’nın birliğini bu deneyimsel İlahiyatlar’ın kökensel ataları olan kişisel Kutsal Üçlemeler’in birliğine ait kişisel sonuç olarak düşünmek oldukça mümkün hale gelir. Bizler, bu üç deneyimsel İlahiyat’ın kesin bir biçimde; ilk seviyeyi oluşturan, atasal ve nedensel Kutsal Üçlemeleri’ne ait büyüyen birlikteliğin doğrudan sonucu olarak bu üç deneyimsel İlahiyat’ın ikinci seviye üzerinde kesin bir biçimde bütünleşeceğine dair düşünceyi varsaymaktayız.
\vs p106 8:20 İlk seviye, üç Kutsal Üçleme tarafından meydana gelmektedir; ikinci seviye, deneyimsel\hyp{}evirilmiş, deneyimsel\hyp{}mevcut\hyp{}hale\hyp{}gelmiş ve deneyimsel\hyp{}varoluşsal İlahiyat kişiliklerinin kişilik ilişkilenimi olarak mevcuttur. Ve, Kutsal Üçlemeler’in bütüncül Kutsal Üçlemesi’ne dair anlayıştaki her türlü kavramsal zorluktan bağımsız olarak, bu üç İlahiyat’ın ikinci düzey üzerindeki kişisel ilişkilenimi bizim evren çağımızda; Nihai vasıtasıyla ve Yüce Varlık’ın başlangıçsal nitelikli yaratıcı emrine karşılık hareket eden İlahi Mutlak tarafından ikinci düzey üzerinde gerçekleştirilmiş, Majeston’un ilahlaştırılmasına ait olgu içinde gözle görülür hale gelmiştir.
\vs p106 8:21 ÜÇÜNCÜ SEVİYE: Kutsal Üçlemeler’in Kutsal Üçlemesi’ne ait ikinci düzeyin koşulsuz bir varsayımda, sonsuzluğun bütünlüğünde mevcut olan veya geçmişte mevcut bulunmuş veya gelecekte mevcut bulunabilecek gerçekliğin her türüne ait her fazın birbiriyle olan karşılıklı ilişkilenimi söz konusudur. Yüce Varlık yalnızca ruhaniyet değil, aynı zamanda akıl ve güç ve deneyimdir. Nihai, bunların tümü ve daha fazlasıdır; buna ek olarak, İlahi, Kâinatsal ve Koşulsuz Mutlak’ın bir tekliğine ait bütünleşmiş kavramsallaşma içinde, tüm gerçeklik gerçekleşimin mutlak kesinliği dâhildir.
\vs p106 8:22 Yüce, Nihai ve tamamlanmış Mutlak’ın birlikteliği içinde; kökensel olarak BEN tarafından birimselleştirilmiş ve Sonsuzluk’un Yedi Mutlaklığı’nın ortaya çıkışı ile sonuçlanmış, sonsuzluğun bu yönlerinin işlevsel nitelikli yeniden bir araya gelişi gerçekleşebilir. Her ne kadar kâinatsal filozofları bunun oldukça uzak bir olasılık olduğunu düşünce de, yine de, bizler şu soruyu sıklıkla sormaktayız: Eğer Kutsal Üçlemeler’in Kutsal Üçlemesi’nin ikinci düzeyi bir kez bile olsun kutsal üçleme birlikteliğine erişebilirse, bu türden ilahiyat bütünlüğünün bir sonucu olarak ne açığa çıkacaktır? Biz bu sorunun cevabını bilmemekteyiz, ancak, bizler onun, deneyimsel olarak bir erişilebilen niteliğinde BEN’in gerçekleşimiyle doğrudan bir biçimde sonuçlanacağından eminiz. Kişisel varlıkların bakış açısından bu, bilinmez olan BEN’in hali hazırda Yaratıcı\hyp{}Sonsuzluk olarak deneyimlenebilir hale geldiği anlamına gelebilir. Bu mutlak nihai sonların kişisel\hyp{}olmayan bir bakış açısından ne anlama geleceği başka bir konu olup, sadece ebediyetin muhtemel bir biçimde aydınlatabileceği husustur. Ancak, bizler; bu uzak nihaililikleri kişisel yaratılmışlar olarak görürken, tüm kişiliklerin kesin nihai sonunun, bu bahse konu kişiliklerin Kâinatsal Yaratıcısı’na dair kesin bilinci olacağı çıkarımında bulunmaktayız.
\vs p106 8:23 Felsefi bir biçimde BEN’i geçmiş ebediyet içinde düşünürken, kendisi tek başına, yanında kimsenin olmadığı bir konumda görünmektedir. Gelecek ebediyete doğru ileri yönlü bakarken, BEN’in bir varoluşsal olarak değişeceğini öngörmemekteyiz; ancak bizler, çok büyük çaplı bir deneyimsel farklılığı öngörme eğilimdeyiz. BEN’in bu türden bir kavramsallaşması bütüncül bireysel\hyp{}kendini\hyp{}gerçekleştirme anlamına gelmektedir --- o, BEN’in kendisini\hyp{}açığa\hyp{}çıkarışı içinde özgür iradesel katılımcıları haline gelen ve, ebedi bir biçimde, mutlak Yaratıcı’nın kesin evlatları olarak sonsuzluğun bütünlüğünün mutlak nitelikli özgür iradesel parçaları halinde kalmaya devam eden kişiliklerin sınırsız galaksisinden meydana gelmektedir.
\usection{9.\bibnobreakspace Deneyimsel Sonsuz Bütünleşme}
\vs p106 9:1 Kutsal Üçleme’nin Kutsal Üçlemesi’ne dair kavramsallaşmalar içinde bizler, sınırsız gerçekliğin olası deneyimsel birleşimi üzerinde düşünmekteyiz; ve, bizler, zaman zaman, tüm bunların hepsinin çok uzak ebediyetin tamamiyle bilinmeyen bir uç noktasında gerçekleşebileceğini kurgulamaktayız. Ancak, tüm geçmiş ve gelecek evren çağlarında olduğu gibi içinde yaşadığımız tam da bu çağda, yine de, sonsuzluğun mevcut ve yaşanmakta olan bir birleşimi bulunmaktadır; bu türden birleşme, Cennet Kutsal Üçlemesi içinde deneyimseldir. Sonsuzluk birleşimi bir deneyimsel gerçeklik olarak düşünülemez derecede uzaktır; ancak, sonsuzluğun koşulsuz bir bütünlüğü şu an içerisinde, kâinat mevcudiyetinin şu anında üstün olup, kendisi \bibemph{mutlak} olan deneyimsel bir görkem ile tüm gerçekliğin farklılaşmalarını birleştirmektedir.
\vs p106 9:2 Sınırlı yaratılmışlar; tamamlanmış ebediyetin kesinlik düzeyleri üzerinde sonsuzluk bütünleşimini düşünmeye çalıştıklarında, sınırlı mevcudiyetleri içinde içkin olan ussal sınırlılıklar ile karşılaşırlar. Zaman, mekân ve deneyim yaratılmış kavramsallaşması için engelleri oluşturur; ve, yine de, zaman olmadan, mekânın dışında, ve deneyim bulunmadan hiçbir yaratılmış, kâinat gerçekliğine ait sınırlı bir kavramsallaşmaya bile erişemez. Zamanın algısı olmadan herhangi bir yaratılmışın mümkün bir biçimde birbirini takip eden ilişkileri algılayabilmesi imkânsızdır. Zaman algısı olmadan herhangi bir yaratılmış, eş zamanlılığın ilişkilerini kavrayamaz. Deneyim olmadan herhangi bir evrimsel yaratılmışın mümkün bir biçimde mevcut oluşu bile imkânsızdır; sadece Sonsuzluk’un Yedi Mutlak’ı gerçekten deneyimi aşabilir, ve bunlar bile bazı fazlarda deneyimsel nitelikte bulunmaktadır.
\vs p106 9:3 Zaman, mekân ve deneyim göreceli gerçeklik algısı için insanın en büyük yardımcılarıdır; ve, yine de onlar, bütüncül gerçeklik algısı karşısında onun sahip olduğu en korkulu engelleridir. Faniler ve birçok diğer evren yaratılmışları potansiyelleri, mekân içinde gerçekleşmekte ve zaman için sonuçlanmak için evirilmekte olarak düşünmeyi gerekli görmektedirler; ancak, bu bütüncül süreç, Cennet üzerinde ve ebediyet içinde mevcut bir biçimde gerçekleşmeyen bir zaman\hyp{}mekân olgusudur. Mutlak düzeyde, ne zaman ne de mekân bulunur; tüm potansiyeller orada mevcudiyetler olarak algılanabilir.
\vs p106 9:4 Tüm gerçekliğin bütünleşimine dair kavramsallaşma, ister bu çağ içinde ister herhangi başka bir evren çağında olsun, temel olarak iki katmanlıdır: varoluşsal ve deneyimsel. Bu türden bir bütünlük, Kutsal Üçleme’nin Kutsal Üçlemesi içinde deneyimsel gerçekleşimin süreci içindedir; ancak, bu üç katmanlı Kutsal Üçleme’nin görünen gerçekleşiminin düzeyi, doğrudan bir biçimde, kâinat içindeki sınırlılıkların giderilişiyle ve gerçekliğin kusursuzluklarıyla orantılıdır. Ancak, gerçekliğin bütüncül birleşimi; tam da bu evren anında sonsuz gerçekliğin bünyesinde mutlak bir biçimde bütünleştiği, Cennet Kutsal Üçlemesi içinde koşulsuz ve ebedi ve deneyimsel olarak mevcuttur.
\vs p106 9:5 Deneyimsel ve varoluşsal bakış açıları tarafından yaratılmakta olan çelişki; kaçınılmaz olup, kısmi bir biçimde, Cennet Kutsal Üçlemesi ve Kutsal Üçlemeler’in Kutsal Üçlemesi’nin her birinin, fanilerin yalnızca bir zaman\hyp{}mekân göreceliliği olarak algılayabildiği bir ebediyet ilişkisi olduğu gerçekliğinden kaynaklanmaktadır. Zaman\hyp{}bakış\hyp{}açısı olarak --- Kutsal Üçlemeler’in Kutsal Üçlemesi’ne ait kademeli deneyimsel gerçekleşime dair insan kavramsallaşması, ebediyet bakış\hyp{}açısı\hyp{}olarak --- hali hazırda bir \bibemph{gerçekleştirme halindeki} ilave düşünce ile desteklenmek zorundadır. Ancak, bu iki bakış\hyp{}açısı nasıl birleştirilebilir? Sınırlı faniler için bizler; Cennet Kutsal Üçlemesi’nin, sonsuzluğun varoluşsal bütünleşimi olduğu, ve, Kutsal Üçlemeler’in deneyimsel Kutsal Üçlemesine ait mevcut mevcudiyeti ve tamamlanmış dışavurumu tespit etmedeki yetkinsizliğin kısmi olarak şu nedenlerden dolayı karşılıklı bozulma altında bulunduğu gerçekliğinin kabulünü sizlere tavsiye etmekteyiz:
\vs p106 9:6 1.\bibnobreakspace Koşulsuz ebediyetin kavramsallaşmasını kavrama yetkinsizliği olarak sınırlı insansal bakış açısı.
\vs p106 9:7 2.\bibnobreakspace Deneyimselliklerin mutlak düzeyinden uzaklık olarak kusursuz insan düzeyi.
\vs p106 9:8 3.\bibnobreakspace Deneyim yöntemi vasıtasıyla insanlığın evrimleşmek için tasarlandığı, ve, bu nedenle, içkin ve yapıcı bir biçimde deneyime bağlı olmak zorunluluğu biçimde insan mevcudiyetinin amacı. Yalnızca bir Mutlak hem varoluşsal hem de deneyimsel olabilir.
\vs p106 9:9 Cennet Kutsal Üçlemesi içinde Kâinatın Yaratıcısı, Kutsal Üçlemeler’in Kutsal Üçlemesi’nin BEN’idir; ve, Yaratıcı’yı sonsuz olarak deneyimlemedeki başarısızlık sınırlılığın sınırlılıklarından kaynaklanmaktadır. \bibemph{Varoluşsal}, tekil ve Kutsal Üçleme\hyp{}öncesi erişilemez BEN’in kavramsallaşmasına ek olarak Kutsal Üçlemeler’in \bibemph{deneyimsel} sonraki\hyp{}Kutsal Üçleme’si ve erişilebilir BEN’in düşüncesi, tek ve aynı varsayımdır; mevcut hiçbir değişiklik Sonsuz içinde gerçekleşmemiştir; görünen tüm gelişmeler, gerçeklik algısı ve kâinatsal takdir için artan yetiler sebebiyle gerçekleşmektedir.
\vs p106 9:10 BEN, son kertede, tüm varoluşsallıklardan \bibemph{önce} ve tüm deneyimselliklerden \bibemph{sonra} mevcut olmalıdır. Bu düşünceler; insan aklı içinde ebediyet ve sonsuzluğun karmaşalıklarını aydınlığa kavuşturamayabilirken, en azından, Salvington üzerinde ve daha sonra kesinlik unsurları olarak ve uçsuz bucaksız evrenler içindeki ebedi süreçlerinizin sonu gelmez geleceği boyunca sizlerin ilgisini çekmeye devam edecek sorunlar olarak, hiç tükenmeyen bu sorunlar ile bu türden sınırlı usların yeniden uğraşmasını sağlamalıdır.
\vs p106 9:11 Er ya da geç tüm evren kişilikleri, ebediyetin nihai arayışının; İlk Kaynak ve Merkez’in mutlaklığına olan hiç bitmeyen keşif seyahati olarak sonsuzluğun sonu gelmez araştırması olduğunu anlamaya başlayacaklardır. Er ya da geç hepimiz, tüm yaratılmış büyümesinin Yaratıcı tanımlaması ile orantılı olduğunun farkına varacağız. Bizler, Tanrı’nın iradesini yaşamanın; sonsuzluğun kendisinin sahip olduğu sonsuz olasılık için ebedi giriş kartı niteliğinde bulunduğu anlayışına varacağız. Faniler gelecekte bir zaman içinde; Sonsuz’un arayışındaki başarının Yaratıcı\hyp{}gibi\hyp{}olmaya erişimle doğru orantılı olduğunun, ve, bu evren çağı içinde Yaratıcı’nın gerçekliklerinin kutsallığın nitelikleri içinde açığa çıkarıldığının farkına varacaklardır. Ve, kutsallığın bu niteliklerine kişisel olarak, kutsal bir biçimde yaşama deneyimi içinde kâinat yaratılmışları tarafından sahip olunabilmektedir; ve, kutsal bir biçimde yaşamak, gerçekte, Tanrı’nın iradesini yerine getirmektedir.
\vs p106 9:12 Maddi, evrimsel ve sınırlı yaratılmışlar için, Yaratıcı’nın iradesini yaşama temeli üzerine oturtturulmuş bir yaşam; doğrudan bir biçimde, kişilik alanında ruhaniyet yüceliğine olan erişime götürmekte olup, bu türden yaratılmışları Yaratıcı\hyp{}Sonsuzluk’un kavrayışında bir adım yaklaştırır. Bu türden bir Yaratıcı yaşamı; güzelliğe olan duyarlılık biçiminde gerçeklik üzerine dayanmakta olup, iyilik üstünlüğü altındadır. Bu türden bir Tanrı\hyp{}bilen birey; içe\hyp{}dönük bir biçimde, ibadet tarafından aydınlanmakta olup, dışa dönük bir biçimde, bağışlamayla dolu ve derin sevgiyle güdülenen bir hizmet yardımı olarak tüm kişiliklerin kâinatsal kardeşliğine ait içten hizmete adanmıştır; bunun karşısında, tüm bu yaşam nitelikleri kâinatsal bilgeliğin, bireyin kendisini gerçekleştiriminin, Tanrı’yı\hyp{}bulmanın ve Yaratıcı ibadetinin sürekli yükselen düzeylerinde evrimleşen kişilik içinde bütünleşmektedir.
\vs p106 9:13 [Nebadon’un bir Melçizedek unsuru tarafından sunulmuştur.]
