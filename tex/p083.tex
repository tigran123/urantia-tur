\upaper{83}{Evlilik Kurumu}
\vs p083 0:1 Bu anlatım, evlilik kurumunun öncül adımlarının öyküsüdür. Evlilik; topluluğun içindeki serbest ve ayrım gözetmeksizin gerçekleştirilen çiftleşmelerden birçok çeşitlilik ve uyum boyunca, en yüksek toplumsal düzeyi temsil eden bir evi oluşturmak için bir erkek ve kadının birlikteliği biçiminde, iki eşin çiftleşmelerinin gerçekleşmesiyle nihai olarak sonuçlanan evlilik ortak ölçütlerinin ortaya çıkışına varıncaya gelişme göstermiştir.
\vs p083 0:2 Evlilik birçok kez tehlike altına girmiştir; ve evlilik adetleri yoğun bir biçimde, özel mülkiyet ve dinin desteğine ihtiyaç duymuştur; ancak evliliği ve onun sonucunda aileyi sonsuza kadar teminat altına alan gerçek etki, ister en ilkel insanlar olsun ister en kültürlü faniler olsun erkek ve kadınların birbirleri olmadan kesin bir biçimde yaşayamamasının yalın ve içkin gerçeğidir.
\vs p083 0:3 Cinsel dürtüsü nedeniyle bencil insan kendi kökeninden bir hayvandan daha iyi bir şeyi yaratmakla cezp edilmektedir. Bireyin kendisi ve onun tatmini için gerçekleştirilen cinsel ilişki; bireyin kendisini reddeden belirli sonuçları beraberinde getirip, toplumsal fedakârlık görevlerine ek olarak ırka yarar sağlayan sayısız ev sorumluluklarının üstlenilmesini teminat altına almaktadır. Bu açıdan cinsel ilişki, ilkel dönem insanlarının farkında olmadıkları ve ummadıkları medenileştiricisi olmuştur; çünkü bu aynı cinsel dürtü kendiliğinden ve sürekli bir biçimde insanı \bibemph{düşünmeye zorlamakta} ve nihai olarak onu \bibemph{sevmeye itmektedir}.
\usection{1.\bibnobreakspace Bir Toplum Kurumu olarak Evlilik}
\vs p083 1:1 Evlilik, iki cinsin fiziksel gerçekliğinden doğan birçok insan ilişkisini düzenlemek ve denetmek için tasarlanmış toplumun işleyiş düzenidir. Bu türden bir kurum olarak evlilik iki yönde faaliyet göstermektedir:
\vs p083 1:2 1.\bibnobreakspace Kişisel cinsel ilişkilerin düzenlenmesi.
\vs p083 1:3 2.\bibnobreakspace Eski ve kökensel faaliyeti olarak soy, miras, vesayet ve toplumsal düz
\vs p083 1:4 Evlilikten doğan ailenin kendisi, özel mülkiyet adetleri ile birlikte evlilik kurumunun bir istikrarlaştırıcı etmenidir. Evlilik istikrarı içindeki diğer güçlü etkenler; gurur, gösteriş, kahramanlık, görev ve dini inançlardır. Ancak evlilik yüksek bir biçimde onaylansa veya reddedilse de, cennetten indirilmemiştir. İnsan ailesi, evrimsel bir gelişim olarak kesin bir biçimde insan kurumudur. Evlilik toplumun bir kurumudur, dinin bir birimi değil. Dinin güçlü bir biçimde onu etkileme zorunluluğunun doğruluk payı bulunmaktadır; ancak onu ayrıcalıklı bir biçimde denetleme ve düzenleme girişiminde bulunmamalıdır.
\vs p083 1:5 İlkel evlilik başat bir biçimde üretimsel ilişkiler ile ilgiliydi; ve çağdaş dönemlerde bile sıklıkla toplumsal veya ticari bir olaydır. And ırk kökeninin karışımlarının etkisi vasıtasıyla ve gelişen medeniyetin örf ve adetlerinin bir sonucu olarak evlilik kademeli bir biçimde karşılıklı, duygusal, ebeveynsel, şiirsel, sevgisel, etik ve hatta idealist bir hale geldi. Tercih ve adlandırdığınız şekliyle duygusal aşk, buna rağmen, ilkel çiftleşme içinde en alt düzeyde bulunmaktaydı. Öncül dönemler boyunca erkek ve kadın eş fazlasıyla birlikte değildi; çok sık bir biçimde beraber bile yemek yememektelerdi. Ancak ilkel insan toplulukları arasında kişisel sevgi güçlü bir biçimde cinsel cazibe ile ilişkili değildi; onlar büyük ölçüde, birlikte yaşadıkları ve çalıştıkları için birbirlerine sevgi besler hale gelmişlerdi.
\usection{2.\bibnobreakspace Kur ve Nişan}
\vs p083 2:1 İlkel evlilikler her zaman, erkek ve kızın ebeveynleri tarafından tasarlanmıştı. Bu gelenek ve özgür tercih arasındaki geçiş aşaması, evlilik aracısı veya uzman çöpçatan tarafından doldurulmuştu. Bu çöpçatanlar, ilk başta berberlerdi; daha sonra ise din adamlarıydı. Evlilik kökensel olarak bir topluluk olayıydı; sonrasında bir aile olayı oldu; ve sadece yakın dönemler içinde bir bireysel serüven haline geldi.
\vs p083 2:2 Cazibe değil zorlama ilkel evliliğe olan yaklaşımdı. Öncül dönemlerde kadın hiçbir cinsel ilgisizliğe sahip değildi; yalnızca cinsel ilişkinin bayalığı adetler tarafından kendisine aşılanmıştı. Yağma yerini ticarete bırakınca, gasp ile gerçekleşen evlilik yerini sözleşme ile yapılana bıraktı. Bazı kadınlar, kabilelerinin üyeleri olan kıdemli erkeklerin baskısından kaçmak için kaçırılmaya göz yummaktalardı; onlar, diğer bir kabileden olan aynı yaştaki erkeklerin ellerine düşmeyi tercih etmekteydiler. Bu görünüşte kaçırma olan olgu, kuvvet ile gerçekleşen gasp ile daha sonraki işveli kur arasındaki geçiş aşamasıydı.
\vs p083 2:3 Düğün töreninin öncül bir türü, bir zamanlar ortak uygulama haline gelen bir çeşit kaçırma provası olarak kaş\hyp{}göz oyunuydu. Daha sonra şakacı kaçırma hareketi, olağan düğün töreninin bir parçası haline geldi. Evliğe karşı ketumluluk olarak, yüksek perdeden “kaçırmaya” itiraz eden çağdaş bir genç kızın iddialarının tümü eski dönem geleneklerinin kalıntılarıdır. Gelinin kucakta taşınması, diğerleri ile birlikte, kadın eşin kaçırılması dönemine ait ilkel dönem uygulamalarından birçoğunun kalıntısıdır.
\vs p083 2:4 Kadın uzunca bir süre, evlilik içinde kendi yaşamını idare etmenin bütüncül özgürlüğünden mahrum bırakılmıştır; ancak daha ussal kadınlar her zaman, kıvrak zekâlarının usta uygulamalarıyla bu kısıtlamaların üstesinden gelmede yetkin olmuşlardır. Erkek genellikle kurda öncülük etmiştir, ancak her zaman değil. Kadın zaman zaman resmi olarak, zaman zaman ise gizli olarak, evliliği başlatmaktadır. Ve medeniyet ilerledikçe, kadınlar kur ve evliliğin tüm aşamalarında artan bir paya sahip olmaktadırlar.
\vs p083 2:5 Evlilik öncesi kurdaki, duygusallık olarak, artan aşk ve kişisel tercih, dünya ırklarına bir And topluluğu katkısıdır. Cinsler arasındaki ilişkiler daha iyiye giden bir biçimde evirilmektedir; birçok gelişmiş insan topluluğu kademeli olarak, kullanışlılık ve sahipliğin eski güdülerini cinsel cazibenin bir tür idealleşmiş kavramlarıyla değiştirmektedir. Cinsel uyarım ve sevgi duyguları, yaşam eşlerinin tercihindeki soğuk hesaplılığın yerini almaya başlamaktadır.
\vs p083 2:6 Nişan ilk olarak evliliğin dengiydi; ve öncül insan toplulukları arasında nişan süreci boyunca cinsel ilişkilerin gerçekleştirilmesi adetti. Yakın zamanlarda din, nişan ve evlilik süreci arası için bir cinsel tabu yaratmıştır.
\usection{3.\bibnobreakspace Satın Alma ve Çeyiz}
\vs p083 3:1 İlkel insan toplulukları, aşk ve sözlere karşı güvensizlik besledi; onlar, kalıcı birlikteliklerin özel mülkiyet gibi bir takım elle tutulur güvenceler ile teminat altına alınması gerekliliğini düşündüler. Bu nedenle bir kadının alım değeri, boşanma veya terk etme durumunda erkek eşin tamamen yitirmeye mahkûm olduğu bir ceza bedeli veya emanet ücreti olarak düşünülmüştü. Bir gelinin alım değeri ödendikten sonra birçok kabile, kocasının damgasının kadının üzerine işlenmesine izin verdiler. Afrikalı topluluklar hala kadınlarını satın almaktadırlar. Bir aşk kadınını, veya bir beyaz erkeğin kadınını, hiçbir fiyatı olmaması nedeniyle bir kediyle kıyaslamaktadırlar.
\vs p083 3:2 Gelin sergileri, kadın eşler olarak daha fazla fiyat getireceği düşüncesiyle kamuya sergilemek için kız çocuklarının giydirilmesi ve süslendirilmesi etkinlikleriydi. Ancak onlar hayvanlar olarak satılmamışlardı --- daha sonraki kabileler arasında bu türden bir kadın eş devredilemezdi. Ne de onun satın alınışı her zaman sadece duygusuz bir para ilişkisiydi; hizmet, bir kadın eşin alımında harcanan paraya denkti. Başka durumlarda eğer arzu edilen bir erkek kadın eşi için ödeyecek durumda bulunmuyorsa, kız babası tarafından bir evlat olarak kabul edilip, daha sonra evlenmesine izin verilir. Ve eğer fakir bir adam eşini arzulamışsa ve açgözlü babası tarafından istenen fiyatı karşılayamıyorsa, yaşlılar heyeti çoğu kez babası üzerinde taleplerinde değişikliğe yol açacak etkileyici baskılarda bulunur, veya aksi halde kaçırma durumu gerçekleşebilir.
\vs p083 3:3 Medeniyet ilerledikçe babalar kızlarını satan görünümlerinden hoşnut kalmadılar, ve böylece gelin alım fiyatını kabul etmeye devam ederken, satın alınan fiyata yaklaşık olarak denk gelen değerli hediye çiftlerini verme âdetini başlattılar. Ve gelin için yapılan ödemenin daha sonraki sonlanışı üzerine, bu hediyeler gelinin çeyizi haline geldi.
\vs p083 3:4 Bir çeyiz düşüncesi, köle kadın eşleri ve özel mülkiyet dostları dönemlerinden çok uzaklaşıldığını gösteren bir biçimde, gelinin bağımsızlığına dair izlenimi yaratma amacıyla gerçekleşmişti. Bir erkek, çeyiz parasının tamamını geri ödemeden bir çeyizli eşini boşayamamaktaydı. Bazı kabileler arasında karşılıklı bir emanet ücreti, birinin diğerini yalnız bırakması durumunda ceza bedeli olarak yitirilmesi amacıyla, hem gelinin hem de damadın ebeveynleri tarafından birbirlerine ödenmekteydi; aslında bu uygulama bir evlilik bağıydı. Alımdan çeyize olan geçiş süreci boyunca eğer kadın eş satın alınmışsa, çocuklar babaya aitti; eğer alınma gerçekleşmemişse onlar kadının ailesine aitti.
\usection{4.\bibnobreakspace Düğün Töreni}
\vs p083 4:1 Düğün töreni, yalnızca iki bireyin verdikleri bir karar sonucu değil, evliliğin ilk olarak bir toplumsal birliktelik olayı olması gerçeğinden doğdu. Çiftleşme, kişisel faaliyete ek olarak topluluk çıkarını ilgilendirmekteydi.
\vs p083 4:2 Büyü, ayin ve törenler ilkel dönem insan topluluklarının bütün yaşamlarını çevreledi; ve evlilik bir istisna değildi. Medeniyet geliştikçe, düğün töreni artan bir biçimde gösterişli hale gelerek ciddi bir biçimde değerlendirilen hal kazandı. Bugün bile olduğu gibi öncül evlilik, mülkiyet çıkarlarında bir etkendi; ve bu nedenle, daha sonra ortaya çıkacak olan çocukların toplumsal düzeyi olası en yüksek tanınmaya ihtiyaç duyarken yasal bir törenin yapılması gerekliydi. İlkel insan hiçbir kayda sahip değildi; böylelikle, evlilik töreninin birçok birey tarafından gözlenmesi zorunluydu.
\vs p083 4:3 İlk başta düğün töreni bir nişan düzeninden daha fazlası olup, sadece, beraber yaşama arzunun halka bildiriminden ibaretti; daha sonra bu durum resmi bir biçimde beraber yemek yemeden meydana geldi. Bazı kabileler arasında ebeveynler doğrudan bir biçimde kızlarını erkek eşlere götürdü; diğer durumlarda ise, gelin babasının kızını damada sunmasından sonra hediyelerin resmi değiş tokuşuydu. Birçok Levant topluluğu arasında, evliliğin cinsel ilişkiler ile tamamlanması biçiminde resmiyetten kurtulmak bir adetti. Kırmızı insan, düğünlerin daha detaylı kutlanılmasını ilk olarak geliştiren topluluktu.
\vs p083 4:4 Çocuksuzluk fazlasıyla korku duyulan bir durumdu; ve çocuğa sahip olamama ruhaniyetin gizli kapaklı emelleriyle ilişkilendirildiği için, doğurganlığı sağlamak için çabalar aynı zamanda belirli büyüsel veya dinsel törenler ile evliliğin birleşmesine yol açtı. Buna ek olarak, mutlu ve verimli evliliği teminat altına almadaki bahse konu çaba içerisinde birçok büyü eşyası kullanıldı; yıldız falcılarına bile, bir araya gelen çiftlerin doğum yıldızlarının belirlenmesi için danışılmıştır. Bir dönem insanların kurban edilmesi, hali vakti yerinde insanlar arasındaki düğünlerin tümünün olağan bir parçasıydı.
\vs p083 4:5 Perşembe en olumlu değerlendirilen olarak, şanslı günler aranmaktaydı; ve dolunay döneminde düğünlerin kutlanılmasının olası en yüksek derecede şansı getirdiği düşünülürdü. Birçok Yakın Doğu insan topluluklarının yeni evli çiftlere tahıl taneleri atması adetti; bu davranış, doğurganlığı sağladığına inanılan bir büyüsel ayindi. Belirli Doğu toplulukları bu amaç için pirinci kullanmıştı.
\vs p083 4:6 Ateş ve su her zaman, hayaletlere ve düşman ruhaniyetlere karşı koymada en iyi araçlar olarak düşünülmüştü; bu nedenle mihrap ateşleri ve yakılmış mumlara ek olarak kutsal suyun vaftizsel serpintisi, genellikle düğünlerde hazır bulunmaktaydılar. Uzun bir zaman boyunca, sahte bir düğün günü seçmek ve daha sonra hayaletleri ve ruhaniyetleri saptırmak için bu etkinliği aniden ertelemek adetti.
\vs p083 4:7 Yeni evlilere muziplik yapmanın ve balayı çiftlerine şakalar düzenlemenin tümü; kıskançlıklarını çekmekten kaçınmak için ruhaniyetlerin gözünde zavallı ve endişeli görünmenin en iyi olduğunun düşünüldüğü bu çok eski günlerin kalıntılarıdır. Gelin duvağı giyinmek, hayaletlerin onu tanımasını engellemek ve aynı zamanda güzelliğini diğer kıskanç ve çekemez ruhaniyetlerin bakışlarından saklamak amacıyla gelini gizlemenin gerekli olduğunun düşünüldüğü dönemlerin bir kalıntısıdır. Gelinin ayaklarının törenden hemen önce toprağa değmemesi zorunludur. Yirminci yüzyılda bile Hıristiyan adetleri uyarınca, taşıyıcılardan kilise mihrabına kadar halıları germek adettir.
\vs p083 4:8 Düğün töreninin en eski türlerinden bir tanesi, birlikteliğin doğurganlığını sağlamak için düğün yatağını bir din adamına kutsatmaktı; bu uygulama, herhangi bir resmi düğün âdetinin oluşumundan çok uzun bir süre önce uygulanmıştı. Bu süreç boyunca evlilik adetlerinin evrimi içinde düğün davetlilerinin yatak odasından bir sıra halinde yürüyerek ilerlemeleri, böylece evliliğin tamamlanışına yasal bir biçimde şahitlik etmeleri gerçekleşmişti.
\vs p083 4:9 Evlilik öncesi sınavların tümüne rağmen belirli evliliklerin kötü sonuçlanması biçiminde, şans etkeni ilkel insanı evlilik başarısızlığından korunmanın yollarını aramaya itti; onu din adamaları ve büyüye yöneltti. Ve bu hareket doğrudan bir biçimde çağdaş kilise evlilikleriyle sonuçlandı. Ancak uzun bir süre boyunca evlilik, çoğunlukla sözleşen ebeveynlerin --- daha sonra ise çiftlerin --- kararlarından meydana gelen bir biçimde tanınırken; son beş yüz yıl boyunca kilise ve devlet, karar yetisini üstlenmiş olup evliliğin resmi bildirileri yapmaktadır.
\usection{5.\bibnobreakspace Çoklu Evlilikler}
\vs p083 5:1 Evliliğin öncül tarihinde evlenmemiş kadınlar kabile erkeklerine aitti. Daha sonra bir kadın bir seferde tek bir eşe sahipti. Bu \bibemph{bir\hyp{}seferde\hyp{}bir\hyp{}erkek} uygulaması, sürü içinde sınırlandırılmamış cinsel ilişkilerden uzaklaşmanın ilk adımıydı. Bir kadın bir erkek ile sınırlandırılmışken, onun kocası bu türden geçici ilişkileri idaresi dâhilinde bitirebilmekteydi. Ancak bu gevşek bir biçimde düzenlenen birliktelikler, sürü gibi yaşamaktan ayrı bir biçimde çiftler halinde yaşama yolunda ilk adımdı. Evlilik gelişiminin bu aşamasında çocuklar çoğunlukla anneye aitlerdi.
\vs p083 5:2 Çiftleşme evrimi içindeki bir sonraki aşama \bibemph{topluluk evliliğiydi}. Evliliğin bu ortak aşaması, açığa çıkan aile yaşamı süreci arasına girmek zorundaydı, çünkü evlilik adetleri henüz çift birlikteliklerini kalıcı kılmaya yetecek kadar güçlü değildi. Erkek ve kız kardeş evlilikleri bu topluluğa aitti; bir ailenin beş erkek kardeşi, diğerinin beş kız kardeşi ile evlenirdi. Dünyanın tümü üzerinde ortak evliliğinin gevşek türleri kademeli bir biçimde topluluk evliliğinin çeşitli türlerine doğru evirildi. Ve bu topluluk birlikteliği geniş bir biçimde totem adetleri tarafından düzenlenmekteydi. Aile yaşamı yavaş ve kesin bir biçimde gelişmişti, çünkü cinsel ilişki ve evlilik düzenlemesi daha fazla sayıda çocuğun hayatta kalışını teminat altına alan bir biçimde kabilenin varlığını devam ettirişine yaramaktaydı.
\vs p083 5:3 Topluluk evlilikleri kademeli olarak --- çok kadın ve erkek eşlilik biçiminde --- çok eşliliğin ortaya çıkma uygulamalarından önce daha gelişmiş kabileler arasında ortadan kayboldu. Ancak çok erkek eşlilik, genellikle kraliçelerle ve zengin kadınlarla sınırlı olan bir biçimde, hiçbir zaman çoğunlukta bulunmamaktaydı; ayrıca, bir kadın eşin birkaç erkek kardeş tarafından paylaşılması geleneksel olarak bir aile olgusuydu. Toplumsal tabakalaşma ve ekonomik kısıtlılıklar, birkaç erkeğin tek bir kadın eşle yetinmesini gerekli kılmıştı. Bu dönemde bile kadın sadece tek biriyle evlenmekteydi; diğerlerine ortak soyun “amcaları” olarak gevşek bir biçimde müsamaha gösterilirdi.
\vs p083 5:4 Bir erkeğin “kardeşinin tohumunu yetiştirmek” amacıyla ölen abisinin dul eşiyle evlenmesini zorunlu kılan Musevi âdeti, ilkçağ dönem dünyasının yarısından fazlasının ortak geleneğiydi. Bu uygulama, ailenin bireysel bir birliktelik yerine bir aile olayı olduğu dönemin bir kalıntısıydı.
\vs p083 5:5 Çok kadın eşliliğin kurumu farklı dönemlerde dört farklı eşlilik türünü tanımıştı:
\vs p083 5:6 1.\bibnobreakspace Törensel veya diğer bir değişle yasal kadın eşler.
\vs p083 5:7 2.\bibnobreakspace Sevgiyle ve izinle elde edilen kadın eşler.
\vs p083 5:8 3.\bibnobreakspace Anlaşmalı kadın eşler olarak cari
\vs p083 5:9 4.\bibnobreakspace Köle kadın eşler.
\vs p083 5:10 Tüm kadın eşlerin eşit düzeye sahip olduğu ve çocukların hepsinin aynı seviyede bulunduğu biçimde gerçek kadın çok eşliliği, oldukça nadir görülen bir durum olagelmiştir. Çoğunlulukla, çoklu evliliklerde bile, ev temsili eş olan baş kadın eşin baskınlığında idare edilmekteydi. Sadece ona töresel düğün töreni yapılmış olup, temsili eş ile özel düzenlemelerde bulunulmadıkça bu türden alınmış veya çeyiz verilmiş eşin çocukları mirastan faydalanabilmekteydi.
\vs p083 5:11 Temsili eş doğrudan bir biçimde sevgi duyulan eş anlamına gelmemekteydi; öncül dönemlerde o çoğunlukla bu konumda bulunmamaktaydı. Kendisine sevgi beslenen kadın eş, veya sevgili; özellikle Nod ve Âdem toplulukları ile evrimsel kabilelerin karışımlarından sonra olmak üzere, ırklar ciddi ölçüde gelişene kadar ortaya çıkmamıştı.
\vs p083 5:12 Yasal düzeydeki tek kadın eş olarak kadın eş tabusu, cariye adetlerini yarattı. Bu adetler uyarınca bir insan sadece tek bir kadın eşe sahip olabilirdi; ancak o, sınırsız sayıda cariye ile cinsel ilişkisini idare edebilirdi. Cariyelik, dürüst çok kadın eşlilikten ilk ayrılış olarak, tekeşliliğe geçişte bir basamak noktasıydı. Musevi, Romalı ve Çin toplulukların cariyeleri sıklıkla kadın eşin hizmetçileriydi. Daha sonra, Museviler arasında görüldüğü gibi, yasal kadın eşe kocanın sahip olduğu tüm çocukların annesi olarak bakılmıştı.
\vs p083 5:13 Bir hamile veya emziren kadın ile olan cinsel ilişkilere dair eski dönemlerin tabuları fazlasıyla çok kadın eşliliği destekleme eğilimi göstermekteydi. İlkel kadınlar, sık çocuk dünyaya getirmekle beraber ağır çalışma sonrasında oldukça erken bir biçimde yaşlanmaktaydılar. (Bu türden ağır yükün altındaki kadın eşler yalnızca, hamile olmadıkları zamanlarda her ayın bir haftası yalnız bırakılmaları gerçeği sayesinde hayatta kalmayı başardılar.) Bu türden bir kadın eş sıklıkla; çocuk doğurmaktan yorulup, hem çocukların dünyaya getirilmesinde ve hem de ev işlerinde yardım etmeye yetkin ikinci ve genç bir kadın eş almasını kocasından talep ederdi. Yeni kadın eşler böylelikle, eski kadın eşler tarafından sevinçle karşılanırdı; bu dönemde, cinsel kıskançlığa benzer hiçbir şey bulunmamaktaydı.
\vs p083 5:14 Kadın eşlerin sayısı yalnızca, erkeğin onları destekleme yetkinliğiyle sınırlıydı. Zengin ve yetkin erkekler, çok fazla sayıda çocuğa sahip olmak istediler; ve bebek ölüm oranları çok yüksek olduğu için, büyük bir aile elde etmek için bir kadın topluluğunun varlığı gerekmekteydi.
\vs p083 5:15 İnsan adetleri evirilmektedir, ancak bu evirilme oldukça yavaş bir biçimde gerçekleşmektedir. Bir haremin var olma amacı, saltanatı desteklemek için kan bağından gelen güçlü ve çok sayıda bireyden oluşan bir birliktelik yaratmaktı. Bir zamanlar belirli bir önder artık hareme sahip olmaması gerektiğine, tek bir kadın eş ile yetinmesi gerektiğine ikna oldu; böylece o, kendi haremini derhal dağıttı. Memnun olmayan kadın eşler evlerine geri döndü, ve onların incinmiş akrabaları öndere kinle koşup, onu oracıktı öldürdüler.
\usection{6.\bibnobreakspace Gerçek Tekeşlilik --- Çift Evliliği}
\vs p083 6:1 Tekeşlilik tekel düzenidir; bu arzu edilebilen düzeye ulaşanlar için iyidir, ancak oldukça talihsiz olanlar için biyolojik bir zorluk çıkarma eğilimi göstermektedir. Ancak birey üzerindeki etkilerden oldukça bağımsız olarak, tekeşlilik kesinlikle çocuklar için en iyi olandır.
\vs p083 6:2 Öncül tekeşlilik, açlık olarak koşulların zorlamasıyla gerçekleşmişti. Tekeşlilik, yapay ve doğal olmayan bir biçimde kültürel ve toplumsaldır; yani evrimsel insan için doğal değildir. Tekeşlilik; saf Nod ve Âdem toplulukları için tümüyle doğal olup, gelişmiş tüm ırklar için büyük bir kültürel değer taşımaktaydı.
\vs p083 6:3 Keldani kabileleri, bir kadın eşin kocasına ikinci eşi veya cariyeyi almaması için evlilik öncesi bir anlaşma maddesi koyma hakkını tanıdı; hem Yunanlı hem de Romalı topluluklar tekeşli evliliği tercih ettiler. Atalara olan ibadet her zaman tekeşliliği desteklemiştir; bu bağlamda evliliği bir kutsallık olarak görmek Hıristiyanlığın hatasıdır. Yaşam koşullarının artması bile tutarlı bir biçimde çoklu kadın eşliliğinin aleyhine gelişmişti. Mikâil’in Urantia’ya olan varışı döneminde neredeyse medenileşmiş dünyanın tümü kuramsal tekeşlilik düzeyine erişmiş haldelerdi. Ancak bu edilgen tekeşlilik, insan türünün gerçek çift evliliği uygulamasına hâlihazırda uyum sağladığı anlamına gelmemişti.
\vs p083 6:4 Nihayetinde tekeşli nitelikteki bir cins birlikteliği benzerliğinde bulunan olası en yüksek çift evliliğinin tekeşlilik hedefinde ilerlerken; şartlarını takip etmek ve onların bir parçası olmak için her şeyi yaptıkları zaman bile, bu yeni ve geliştirilmiş toplumsal düzende bir yer edinmede başarısız olmuşmuş talihsiz erkek ve kadınların hoş gözle bakılmayan durumlarına toplum tepeden bakmamalıdır. Rekabetsel toplumsal alan içinde çift bulmadaki başarısızlık, mevcut adetlerin zorunlu kıldığı üstesinden gelinemeyecek zorluklar veya pek çok kısıtlama sonrasında gerçekleşmiş olabilir. Gerçekten de içinde bulunanlar için tekeşlilik idealdir; ancak yalnızlığın soğuk mevcudiyetine terk edilenler için büyük zorlukları kaçınılmaz bir biçimde beraberinde getirmektedir.
\vs p083 6:5 Her zaman talihsiz olan azınlıklar, evrimleşen medeniyetin gelişen adetleri uyarınca çoğunluğun gelişebilmesi için sıkıntı çekmek zorunda kalmışlardı; ancak kendisine iltimas geçilen çoğunluk her zaman, ilerleyen toplumsal evrimin en yüksek adetlerinin izinleri altında tüm biyolojik dürtülerin tatminini sağlayan bu nihai cinsel birlikteliğin düzeylerinde üyeliği elde etmedeki başarısızlığının cezasını çekmekle yükümlü daha az talihli akranlarına iyi gözle ve hoşgörüyle bakmak zorundadır.
\vs p083 6:6 Tekeşlilik şimdi ve her zaman olduğu gibi ve gelecekte olacağı gibi, insanın cinsel evriminin olası en yüksek hedefidir. Gerçek çift evliliğin bu ideali bireyin kendisini reddini beraberinde getirmektedir; ve bu nedenle o sadece, çetin öz denetimin olarak tüm insan erdemlerinin bütünlüğünde bir veya iki birliktelik üyesinin eksiklik taşıması yüzünden oldukça sık bir biçimde başarısız olmaktadır.
\vs p083 6:7 Tekeşlilik, saf biyolojik evrimden farklılaşmış bir biçimde toplumsal medeniyetin gelişimini ölçen ölçü çubuğudur. Tekeşliliğin gerçekliği, onun doğrudan bir biçimde biyolojik ve doğal düzeyler ile ilişkili olduğu anlamına gelmemektedir; ancak tekeşlilik, toplumsal medeniyetin doğrudan idaresi ve daha ileri gelişimi için hayati öneme sahiptir. Tekeşlilik, ahlaki kişiliğin bir gelişimi olarak bir duygu hassasiyetine ve çok eşlilikte gelişmesi kesinlikle mümkün olmayan ruhsal bir gelişime katkıda bulunmaktadır. Bir kadın, kocasının sevgisini kazanmada çetin rekabete girişmeye zorlanmadığı bir süreci deneyimlemedikçe hiçbir zaman ideal bir anne olamaz.
\vs p083 6:8 Çift evliliği; ebeveynsel mutluluk, çocuk refahı ve toplumsal verimlilik için en iyisi olan sevgi dolu duygudaşlığı ve etkin iş birliğini tercih eder ve destekler. İlkel zorlama ile başlayan evlilik; bireyin kendi kendisini kültürlü hale getirdiği, üzerinde öz denetim uyguladığı, kendisini ifade ettiği ve kendi devamlılığını gerçekleştirdiği mükemmel bir kuruma doğru kademeli bir biçimde evirilmektedir.
\usection{7.\bibnobreakspace Evlilik Bağının Kopuşu}
\vs p083 7:1 Evlilik adetlerinin öncül evriminde evlilik, irade dâhilinde sonlanabilecek olan gevşek bir birliktelik olup, çocuklar her zaman anneyi takip etmekteydi; anne\hyp{}çocuk bağı içgüdüsel olup, adetlerin gelişimsel aşamasından bağımsız olarak faaliyet göstermiştir.
\vs p083 7:2 İlkçağ topluluklar arasında evliliklerin yaklaşık sadece yarısı başarılıydı. En sık karşılaşılan neden, her zaman kadının suçlandığı bir biçimde, çocuğa sahip olamamaktı; ve çocuksuz kadın eşlerin ruhani dünyada yılanlar haline geleceklerine inanılmaktaydı. Daha ilkel adetler uyarınca boşanma sadece erkeğin bir tercihiydi, ve bu ortak ölçütler yirminci yüzyılın birçok topluluğu arasında varlığını korumaya devam etmektedir.
\vs p083 7:3 Adetler evrimleştikçe, belirli kabileler iki tür evlik geliştirdi: boşanmaya izin veren sıradan olanlar, ayrılmaya izin vermeyen din adamı evlilikleriydi. Kadın eşin satın alınması ve kadın eş çeyizinin başlaması, evlilik başarısızlığı için mülksel bir cezanın getirilmesi ile birlikte, ayrılığını azalması üzerinde büyük etkiye sahip oldu. Ve gerçekten de birçok çağdaş birliktelikler bu ilkçağ mülk etkeni vasıtasıyla istikrarlı hale gelmektedir.
\vs p083 7:4 Toplumdaki düzeyin ve mülkiyet ayrıcalıklarının getirdiği dair toplumsal baskı, evlilik tabuları ve adetlerinin idare edilmesinde her zaman etkili olmuştur. Her ne kadar --- yeni bir özgürlük olarak --- bireysel tercihin fazlasıyla yüksek olduğu yerlerdeki insan toplulukları arasındaki geniş çaplı memnuniyet tarafından tehditkâr bir biçimde saldırıya uğrasa da, çağlar boyunca evlilik istikrarlı ilerlemede bulunmuş olup, bugünün dünyasında gelişmiş bir temel üzerinde durmaktadır. Uyumun bu kargaşaları aniden hızlanan toplumsal evrimin bir sonucu olarak daha ilerici ırklar arasında ortaya çıksa da, daha az gelişmiş insan toplulukları arasında evlilik yeşermekte ve daha eski adetlerin rehberliği altında yavaşça gelişmektedir.
\vs p083 7:5 Eski ve uzun bir süreden bu yana varlığını oluşturmuş mülkiyet güdüsüyle evlilik içerisindeki daha ideal fakat aşırı derecedeki bireysel aşk güdüsünün yeni ve ani değişimi kaçınılmaz bir biçimde, evlilik kurumunun geçici bir süreliğine istikrarsız hale gelişine neden olmaktadır. İnsanın evlilik güdüleri her zaman evliliğin mevcut ahlaki kurallarının çok ötesine geçmiştir; ve on dokuzuncu ve yirminci yüzyılın Batı evlilik ideali ansızın bir biçimde birey merkezli olan ama kısmen denetlenen ırkların cinsel uyarımlarının ötesine geçmiştir. Herhangi bir toplumda bulunan evli olmayan bireylerin geniş sayılarının varlığı, geçici başarısızlık süreçlerine veya adetlerin geçiş dönemlerine işaret etmektedir.
\vs p083 7:6 Çağların en başından bu döneme kadar evliliğin gerçek sınavı, aile yaşamının tümü içinde kaçınılmaz haldeki devamlılığı olan içten yakınlıktır. Gösteriş ve benliğin her hoşgörüsünü ve bütüncül tatminini bekleyecek bir biçimde eğitilen pohpohlanmış ve şımartılmış iki genç --- ağırbaşlılık, uzlaşma ve sadakatin ömür boyu süren bir yaşam birlikteliğine ek olarak çocuk yetiştirilimine olan fedakâr bağlılık biçiminde --- evlilik ve ev inşasının büyük başarısını gerçekleştirmeyi neredeyse hayal bile edemez.
\vs p083 7:7 Kur dönemine girerken yüksek derecede hayal ve gerçek dışı duygusallık büyük ölçüde, çağdaş Batılı topluluklar arasındaki artan boşanma eğilimlerinden sorumludur; bütün bunların hepsi kadının daha fazla kişisel özlüğü ve artan ekonomik bağımsızlığı ile daha karmaşık kale gelmiştir. Öz denetim eksikliğinin veya olağan kişilik uyumundaki başarısızlığın gerçekleşmesi sonucunda ortaya çıkan kolay boşanma, insanın çok yakın bir süre içinde ve oldukça fazla sayıdaki kişisel ızdıraplarının ve ırksal sıkıntıların sonucunda üstüne çıktığı ilkel toplumsal aşamalara doğrudan bir biçimde geri götürmektedir.
\vs p083 7:8 Ancak işte toplum çocukları ve gençleri uygun bir biçimde eğitmede başarısız oldukça, toplumsal düzen yeterli aile öncesi hazırlığı sağlamada başarısız oldukça ve bilge olmayan, olgunlaşmamış gençlik idealizmi evliliğe giriş yapmada belirli oldukça; boşanma işte böyle uzun bir süre yaygın olmaya devam edecektir. Ve toplumsal birliktelik gençlerin evliliğe hazırlanmasını sağlamada başarısız oldukça boşanma, evrimleşen adetlerin hızlı büyüdüğü çağlar boyunca daha da kötü durumların ortaya çıkmasını engelleyen bir biçimde toplumun emniyet sibobu olarak faaliyet göstermek zorundadır.
\vs p083 7:9 İlkçağ toplulukları evliliği, bugünün bir takım topluluklarının gördüğü kadar ciddi olarak değerlendiren görünüme sahiptirler. Ve çağdaş dönemlerin hızlıca yapılmış ve başarısız olan birçok evliliğinin yetkin genç erkek ve kadınların ilkçağ çiftleşme uygulamaları üstüne büyük bir gelişim olmadığı görünmektedir. Çağdaş toplumun büyük tutarsızlığı, ikisinin de bütüncül irdeleyişini reddederken aşkı yüceltmesi ve evliliği idealleştirmesidir.
\usection{8.\bibnobreakspace Evliliğin İdealleştirilmesi}
\vs p083 8:1 Evin oluşumuyla sonuçlanan evlilik gerçekten de insanın en yüce kurumudur; ancak bu kurum özü itibariyle insan kaynaklıdır; bu kurum, hiçbir şekilde bir kutsal oluşum olarak değerlendirilmemeliydi. Seth din adamları evliliği dini bir ayin haline getirmişti; ancak Cennet Bahçesi’nden binlerce yıl sonra çiftleşme tamamiyle toplumsal ve kamusal bir kurum olarak varlığını sürdürdü.
\vs p083 8:2 İnsan birlikteliklerini kutsal birlikteliklerine özleştirmeye çalışmak en talihsiz olan şeydir. Erkek ve kadın eşin evlilik\hyp{}ev ilişkisi içindeki birlikteliği, evrimsel dünyaların fanilerine ait maddi bir faaliyettir. Ruhsal ilerleyişin büyük bir kısmının erkek ve kadın eşin ilerleme gerçekleştirdikleri içten insan çabalarına bağlı olduğu gerçekten de doğrudur; ancak bu gerçeklik evliliğin doğrudan bir biçimde kutsal olduğu anlamına gelmemektedir. Ruhsal ilerleme, insani çabanın diğer alanlarına yapılan içten bir biçimde adanmışlığa bağlıdır.
\vs p083 8:3 Ne evlilik gerçek anlamıyla Düzenleyici’nin insan ile olan ilişkisiyle karşılaştırılabilir ne de Hazreti Mikâil’in insan kardeşleri ile olan bütünlüğüyle. Hiçbir biçimde hiçbir düzeyde bu türden ilişliler erkek ve kadın eşin birlikteliğiyle karşılaştırılamazlar. Ve bu ilişkilere dair insanların kavram yanılgısının evliliğin düzeyi ile ilgili bu kadar kafa karışıklığını yaratması en talihsiz olanıdır.
\vs p083 8:4 Fanilerin belirli toplulukların evliliğin kutsal bir eylem ile tamamlandığını düşünmesi başka bir talihsizliktir. Bu türden inanışlar doğrudan bir biçimde, koşullardan ve birliktelik üyelerinin isteklerinden bağımsız bir biçimde evlilik düzeyinin parçalanamazlığına dair kavrama yol açmıştır. Ancak evliliğin parçalanmasına dair gerçekliğin tam da kendisi, İlahiyat’ın bu türden birlikteliklerde etkin bir üye olmadığını göstermektedir. Eğer Tanrı herhangi iki şeyi ve bireyi bir araya getirirse, onlar kutsal iradenin ayrılmalarına hüküm verdiği süreye kadar bu şekilde birlikte olmaya devam edecektir. Ancak bir insan kurumu olan evlilik hususunda; evliliklerin, doğası ve kökeni itibariyle tamamiyle insan olanların tam karşısında bulunan evrenin yüksek denetleyicileri tarafından onaylanabilen birliktelikler olduğunu söyleyen bir biçimde kim hükümde bulunulduğunu farz edebilir?
\vs p083 8:5 Yine de, yüksek âlemlerde evliliğin bir ideali bulunmaktadır. Her yerel sistemin başkentinde Tanrı’nın Maddi Erkek ve Kız Evlatları, evlilik bağlarına ek olarak doğum ve çocukların yetiştirilmesinde erkek ve kadının birlikteliğine dair en yüksek idealleri sergilemektedir. Nihayetinde, olası en yüksek fani evlilik \bibemph{insana özgü bir biçimde} kutsaldır.
\vs p083 8:6 Evlilik her zaman insanın dünyevi idealizmine ait yüce hayal olmuştur, ve bu durum halen böyledir. Her ne kadar bu güzel hayal yaratıldığından beri nadiren gerçekleşmişse de; insan mutluluğu için daha büyük çabalara doğru insan türünü sürekli cezbeden bir biçimde, ihtişamlı bir ülkü olarak varlığını sürdürmektedir. Ancak genç erkek ve kadınlara, aile yaşamının karşılıklı birlikteliklerinin zorlayıcı taleplerine dalmalarından önce evliliğin gerçeklerine dair bir şeyler öğretilmelidir; gençliksel idealleştirme, evlilik öncesinde gerçeklerle karşılaşılması sonucu yaşanacak belirli düzeydeki hayal kırıklıklarıyla seyreltilmelidir.
\vs p083 8:7 Evliliğin gençliksel idealleştirilişi, buna rağmen, caydırılmamalıdır; bu türden hayaller, aile yaşamının gelecekteki hedefinin tahayyül edilişidir. Bu tutum; evlilik ve ilerideki aile yaşamının gündelik ve olağan zorundalıklarının yerine getirilmesi karşısında bir duyarsızlık yaratmaması koşuluyla, hem ilham verici hem de yardımcıdır.
\vs p083 8:8 Evliliğin idealleri, yakın dönemlerde büyük gelişme gösterdi; bazı topluluklar arasında kadın, eşinin sahip olduklarına neredeyse eşit hakları memnuniyetle deneyimlemektedir. Kavramsal olarak, en azından, aile; cinsel sadakatle beraber doğumların yetiştirilmesi için sadık bir birliktelik haline gelmektedir. Ancak evliliğin bu yeni türünün bile, her kişilik ve bireyin ortak zorunluluğuymuş gibi düşünülen bir biçimde aşırı derecede uç boyutlara çekilmesine gerek yoktur. Evlilik sadece bir kişilik ülküsü değildir; evlilik, mevcut adetler uyarınca var olan ve faaliyet gösteren, tabular tarafından sınırlanan ve toplumun yasaları ve düzenlemeleri tarafından üzerinde yaptırım uygulanan bir biçimde bir erkek ve kadının evrimleşen toplumsal birlikteliğidir.
\vs p083 8:9 Yirminci yüzyıl evlilikleri; geçmiş nesillerin sahip oldukları adetlerin geç kalmış evrimi sonucunda kendisine uzun bir süre boyunca verilmemiş haklar olarak kadın özgürlüklerinin çok çabuk artışı nedeniyle toplum düzeni üstüne çok ansızın binen yükün yarattığı sorunlar nedeniyle ev kurumu mevcut an içerisinde her ne kadar ciddi bir sınavdan geçse de, geçmiş çağlara kıyasla yüksek bir konumda bulunmaktadır.
\vs p083 8:10 [Urantia üzerinde konumlanan bir Yüksek Melek Önderi tarafından sunulmuştur.]
