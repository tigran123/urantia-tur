\upaper{103}{Dini Deneyimin Gerçekliği}
\vs p103 0:1 İnsan’ın gerçekten dinsel olan tepkilerinin tümü; ibadet emir\hyp{}yardımcısının öncül hizmeti tarafından sağlanmakta olup, bilgelik emir\hyp{}yardımcısı tarafından denetlendir. İnsanın ilk akıl\hyp{}ötesi kazanımı, Kâinat Yaratıcı Ruhaniyeti’ne ait Kutsal Ruhaniyet içinde döngüleştirilmiş kişiliğe aittir; ve kutsal Evlatlar’ın bahşedilişlerinden veya Düzenleyiciler’in kâinatsal bahşedilişlerinden çok önce bu etki, insanın etik kurallar, din ve ruhsallığa dair bakış açısını genişletmek için faaliyet gösterir. Cennet Evlatları’nın bahşedilişlerini takiben özgürleştirilmiş Gerçekliğin Ruhaniyeti, dini gerçeklikleri algılamak için insan yetisinin genişlemesine çok büyük katkıda bulunmaktadır. Evrim yerleşik bir dünyada ilerlerken Düşünce Düzenleyicileri artan bir biçimde, insanın dini kavrayışının daha yüksek türlerinin gelişimine katılır. Düşünce Düzenleyicisi; sınırlı yaratılmışın aracılığıyla, Kâinatın Yaratıcısı olarak sınırı bulunmayan İlahiyat’ın kesinliklerine ve kutsallıklarına inanç\hyp{}bakışıyla göz atabileceği kâinatsal penceredir.
\vs p103 0:2 İnsan ırkının dini eğilimleri içkindir; onlar evrensel olarak dışa vurulmakta olup, gözle görülür biçimde doğal kökene sahiptir; ilkel dinler doğuşlarında her zaman evrimseldir. Doğal dini deneyim ilerlemeye devam ederken gerçekliğin dönemsel açığa çıkarılışları, aksi halde gezegensel evrimin yavaş hareket eden gidişatına aralıklar getirebilir.
\vs p103 0:3 Urantia üzerinde, bugün, dört tür din bulunmaktadır:
\vs p103 0:4 1.\bibnobreakspace Doğal veya evrimsel din.
\vs p103 0:5 2.\bibnobreakspace Doğa\hyp{}üstü veya açığa çıkarımsal din.
\vs p103 0:6 3.\bibnobreakspace Doğal ve doğa\hyp{}üstü dinlerin çeşitli düzeylerdeki karışımı olan işleyiş halindeki veya diğer bir değişle mevcut din.
\vs p103 0:7 4.\bibnobreakspace Din\hyp{}kuramsal inanç savları ve nedensellikle yaratılmış dinler üzerindeki felsefi düşünceden doğan veya diğer bir değişle insan yapımı dinler olarak felsefi dinler.
\usection{1.\bibnobreakspace Din’in Felsefesi}
\vs p103 1:1 Toplumsal veya ırksal bir topluluk içinde dini deneyimin bütünlüğü özünü, birey içinde ikamet eden Tanrı nüvesinin özdeş doğasından almaktadır. İnsan içindeki bu kutsallık, diğer insanların refahına olan fedakâr ilgiye kaynaklık etmektedir. Ancak, kişilik benzersiz olduğu için --- hiçbir iki faninin birbirine benzememesi olarak, herhangi bir iki insan varlığının; akılları içinde yaşayan kutsallığın ruhaniyetine ait yönelimleri ve uyarımları benzer bir şekilde yorumlamayacağı kaçınılmaz olarak ortaya çıkar. Fanilerin bir topluluğu ruhsal bütünlüğü deneyimleyebilir; ancak onlar hiçbir zaman, felsefi tek\hyp{}tipliliğe erişemezler. Ve, dini düşünce ve deneyimin yorumlanışındaki bu çeşitlilik; yirminci yüzyıl din\hyp{}kuramcıları ve filozoflarının dini beş yüzden fazla farklı tanımla tarif etmeleri gerçeğiyle sergilenmektedir. Gerçekte, her insan varlığı dini, kendisinde ikamet eder konumda bulunan Tanrı ruhaniyetinden doğan kutsal uyarımları kendi deneyimsel yorumuyla tanımlar; ve, bu nedenle, böyle bir yorum benzersiz ve tüm diğer insan varlıkların sahip olduğu dini felsefeden tamamiyle farklı olmak zorundadır.
\vs p103 1:2 Bir fani, bir akran faninin dini felsefesiyle tamamiyle hemfikir olduğunda, bu olgu, felsefi nitelikli dini yorumlarının benzerliğiyle ilgili hususlarda benzer bir\bibemph{ dini deneyime} sahip olduklarını gösterir.
\vs p103 1:3 Her ne kadar dininiz bir kişisel deneyim olayıyken, dini yaşamınızın --- soyutlanmış, bencil ve toplumsal\hyp{}olmayan bir biçimde --- sadece\hyp{}birey\hyp{}merkezli hale gelmesini engelleyecek düzeyde sizin diğer dini deneyimlerin geniş bir sayısının bilgisine açık olmanız en önemli şeydir.
\vs p103 1:4 Nedensellik dini, ilk başta bir şeye olan ilkel inanış olarak varsaydığında ve değerlerin arayışının daha sonra onu takip ettiğini düşündüğünde hata yapmaktadır. Din, öncül bir biçimde değerlerin bir arayışıdır; ve orada daha sonra, yorumsal inanışların bir sistemi oluşur. İnsanların; yorumlar olarak --- inanışlardan ziyade --- amaçlar biçiminde --- dini değerler üzerinde anlaşmaları çok daha kolaydır. Ve, bu durum; mezhepler olarak --- çatışma halindeki inanışların yüzlercesini içinde barındıran tek bir inanışı sürdürmedeki kafa karışıklığına sebebiyet veren olguyu yansıtırken, değerler ve amaçlar üzerinde dinin nasıl hemfikir olduğunu açıklamaktadır. Bu, aynı zamanda, herhangi bir insanın; dini inanışlarının birçoğunu bırakmasına veya onları değiştirmesine rağmen, dini deneyimine nasıl devam edebildiğini açıklamaktadır. Din, dini inanışlardaki devrimsel değişikliklere rağmen varlığını sürdürmektedir. Din\hyp{}kuramı dini yaratmamaktadır; din, din\hyp{}kuramsal felsefeyi ortaya çıkarmaktadır.
\vs p103 1:5 Dindarların çok fazla yanlış olan şeye inanmış olmaları dini geçersiz kılmamaktadır; çünkü din, değerlerin tanınması temeli üzerine kurulmuş olup, geçerliliğini, kişisel nitelikli dini deneyimin sahip olduğu inançtan almaktadır. Din, bunlardan sonra, deneyim ve dini düşünceye dayanmaktadır; dinin felsefesi olarak din\hyp{}kuramı, bu deneyimi yorumlamak için içten bir girişimdir. Bu türden yorumlayıcı inanışlar yanlış olabilir, veya gerçeklik ve hatanın bir karışımı olabilir.
\vs p103 1:6 Ruhsal değerlerin tanınmasının gerçekleşimi, düşünsel\hyp{}ötesi konumdaki bir deneyimdir. Herhangi bir insan dili içerisinde, Tanrı\hyp{}bilincine çağrışımda bulunmak amacıyla tercih ettiğimiz bu “duygu”, “his”, “içgüdü” veya “deneyimi” adlandırmak için kullanılabilecek hiçbir kelime bulunmamaktadır. İnsan içinde ikamet eden Tanrı’nın ruhaniyeti kişisel değildir --- Düzenleyici kişilik\hyp{}öncesidir, ancak Görüntüleyici, en yüksek ve sınırsız düzeyde kişisel olan, kutsallığın bir hoş kokusunu yayan biçimde bir değeri sunmaktadır. Eğer Tanrı kişisel bile olmasaydı, bilinç sahibi olamazdı; bilinç sahibi olmasaydı, bunun sonrasında insan düzeyinin altında olurdu.
\usection{2.\bibnobreakspace Din ve Birey}
\vs p103 2:1 Din; insan aklında işlevsel olup, insan bilincindeki ortaya çıkışından önce gerçekleştirilmektedir. \bibemph{Doğumu} deneyimlemesinden dokuz ay önce bir çocuk mevcudiyet halinde bulunmaktadır. Ancak, dinin “doğumu” anlık değildir; o bunun yerine, kademeli bir ortaya çıkıştır. Yine de, elinde sonunda bir “doğum günü” gerçekleşir. Siz cennetin krallığına “tekrar doğmadan” --- Ruhaniyet’den doğmadan --- girmemektesiniz. Birçok ruhsal doğuma; tıpkı fiziksel doğumların çoğunun bir “fırtınalı doğum” ve “doğurmanın” diğer olağandışılıkları ile nitelenişi gibi, ruhaniyetin fazlasıyla yaşanan ızdırabı ve ciddi derecede psikolojik rahatsızlıklar eşlik etmektedir. Diğer ruhsal doğumlar; her ne kadar hiçbir dini deneyim, bilinçsel çabaya ek olarak olumlu ve bireysel kararlıklar olmadan ortaya çıkmasa da, dini deneyimdeki bir gelişimle beraber yüce değerlerin tanınışındaki doğal ve olağan bir gelişimidir. Din hiçbir zaman, olumsuz bir tutum niteliğindeki durağan bir deneyim değildir. “Dinin doğumu” olarak tanımlanan şey, doğrudan bir biçimde; genellikle, akılsal çatışmanın, duygusal baskının ve anlık mizaç değişimlerinin bir sonucu olarak yaşamda daha sonra ortaya çıkan dini yaşanmışlıkları niteleyen din değiştirme olarak adlandırdığınız olguyla doğrudan bir biçimde ilişkili değildir.
\vs p103 2:2 Ancak, ebeveynleri tarafından, sevgi dolu cennetsel bir Yaratıcı’nın çocukları olmanın bilinci içinde büyüyen bir biçimde yetiştirilmiş kişiler; yalnızca, anlık duygusal bir değişim olarak bir psikolojik buhranla Tanrı ile olan aidiyetsel birlikteliğin bu türden bir bilincine erişebilmiş akran fanilerine şüpheci gözle bakmamalıdırlar.
\vs p103 2:3 İçinde açığa çıkarılmış dine ait tohumunun filiz verdiği insan aklındaki evrimsel toprak, çok önceden toplumsal bir bilince kaynaklık eden ahlaki doğadır. Bir çocuğun ahlaki doğasının ilk yönergelerinin cinsel ilişkiyle, suçluluk duygusuyla veya kişisel gurur ile hiçbir ilişkisi yoktur; ancak onlar gerçekte, adiliyet olarak adaletin uyarımları olup, bir kişinin akranlarına olan yardımcı hizmeti biçimindeki --- iyiliğe sevk eder. Ve, bu türden öncül ahlaki uyanışlar beslendiğinde, orada, çatışmalardan, büyük ani değişikliklerden ve buhranlardan görece uzak dini yaşamın kademeli bir gelişimi ortaya çıkar.
\vs p103 2:4 Her insan varlığı oldukça öncül bir biçimde, kendisini bulma ile topluma hizmet etme uyarımları arasında bir çatışmaya benzer bir şeyi deneyimler; ve, birçok kez Tanrı\hyp{}bilincinin ilk deneyimi, bu tür ahlaki çatışmaları çözme amacı için insan\hyp{}ötesi yardımı aramanın sonucu olarak elde edilebilir.
\vs p103 2:5 Bir çocuğun psikolojisi doğal olarak olumludur, olumsuz değil. Çok fazla sayıdaki fani, öyle yetiştirildiği için olumsuz bir bakışa sahiptir. Çocuğun olumlu bakış açısına sahip olduğu söylendiğinde, bahsi geçen şey, ortaya çıkışı Düşünce Düzenleyicisi’nin varışını haber veren akıl güçleri olarak, onun ahlaki uyarımları ile ilgilidir.
\vs p103 2:6 Yanlış eğitimin olmadığı durumda olağan çocuğun aklı; olumsuz bakış açısı yerine, günah ve suçluluktan uzaklaşarak ahlaki doğruluk ve toplumsal hizmete doğru, dini bilincin ortaya çıkışıyla, olumlu bir şekilde hareket eder. Dini deneyimin gelişimi içinde çatışma olabilir de, olmayabilir de; ancak orada her zaman, insan iradesinin kaçınılmaz kararları, çabası ve faaliyeti bulunmaktadır.
\vs p103 2:7 Ahlaki tercihe genel olarak, belli bir düzeyde ahlaki çatışma eşlik eder. Ve, çocuğun aklı içindeki tam da bu çatışma, bencilliğin arzuları ile fedakârlığın uyarımları arasındadır. Düşünce Düzenleyicisi, ben\hyp{}merkezci güdümde olan kişilik değerlerini önemsiz görmemektedir; ancak, insan mutluluğunun amacına ve cennetin krallığının neşelerine götürürken fedakârlık uyarımı üzerinde biraz daha fazla tercih hakkı kullanarak işlerliğini gösterir.
\vs p103 2:8 Ahlaki bir varlık bencil olmanın arzusu ile karşılaştığında bencil olmamayı tercih ettiğinde, bu durum, ilkel dini deneyimdir. Hiçbir hayvan böyle bir tercihte bulunmamaktadır; bu türden bir karar, hem insani hem de dinidir. Tanrı\hyp{}bilinci gerçeğini bünyesinde barındırıp, insan kardeşliğinin temeli olarak toplumsal hizmetin uyarımını göstermektedir. Akıl özgür iradenin bir eylemi vasıtasıyla doğru bir ahlaki yargıda bulunmayı tercih ederse, bu türden bir karar bir dini deneyimi meydana getirmektedir.
\vs p103 2:9 Ancak, bir çocuğun toplumsal yetkinliği elde etmek için yeterli bir biçimde gelişiminden ve böylece fedakâr hizmeti tercih etmeye yetkin hale gelişinden önce, o hali hazırda, güçlü ve oldukça bütünleşmiş bir ben\hyp{}merkezci d0ğayı geliştirmiş haldedir. Ve, gerçek olan bu durum, “günahın yaşlı adamıyla” şükranın “yeni doğası” arasında olmak üzere, “daha yüksek” ve “daha alt düzey” doğalar arasındaki mücadeleye dair kuramın doğuşuna kaynaklık etmektedir. Yaşamda çok öncül bir biçimde olağan çocuk, “vermenin almaktan daha mukaddes” olduğunu öğrenmeye başlar.
\vs p103 2:10 İnsan, kendisine hizmet etme arzusunu --- kendisi olarak --- sahip olduğu benlik ile tanımlama eğilimine sahiptir. Bunun karşısında, fedakâr alma iradesini --- Tanrı olarak --- kendisinin dışındaki belli bir etkiyle tanımla yönelimine sahiptir. Ve, gerçekten de bu türden bir yargı doğrudur; zira, benin merkezinden olmadığı arzuların tümü kökenini, özünde, ikamet eden Düşünce Düzenleyicisi’nin yönlendirmelerinden almaktadır; ve, bu Düzenleyici, Tanrı’nın bir nüvesidir. Ruhaniyet Görüntüleyicisi’nin uyarımı, akran\hyp{}yaratılmışı önemseyen bir biçimde fedakâr olma arzusu olarak insan bilincinde gerçekleştirilir. En azından bu, çocuk aklının öncül ve temel deneyimidir. Büyüyen çocuk kişilik bütünleşmesinde başarısız olduğu zaman, fedakâr güdü bireyin refahı üzerinde ciddi şekilde zarar verebilecek düzeyde haddinden fazla gelişebilir. Yanlış yönlendirilmiş vicdan; fazla sayıdaki çatışmanın, üzüntünün ve sonu gelmez insan mutsuzluğunun nedeni hale gelebilir.
\usection{3.\bibnobreakspace Din ve İnsan Irkı}
\vs p103 3:1 Ruhaniyetlere, rüyalara ve çeşitli düzeydeki diğer hurafelere olan inanışın tümü ilkel dinlerin evrimsel kökeninde bir rol almış olsa da; bütünlüğün kavimsel veya kabilesel ruhunu önemsiz görmemelisiniz. Topluluk ilişkisinde, öncül insan aklının ahlaki doğasında gerçekleşen ben\hyp{}merkezci ve fedakâr olma çatışmasını sarsan bir doğa sunan toplumsal durum ortaya çıkmıştı. Ruhaniyetlere olan inanışlarına rağmen Avusturyalılar, hala, dinlerini kavimleri üzerinde odaklamaktadırlar. Zaman içerisinde bu türden dini kavramsallaşmalar, ilk başta hayvanlar olarak ve daha sonra insan\hyp{}ötesi bir varlık veya bir Tanrı olarak kişileşme eğilimi gösterir. İnanışlarında totemsel bile olmayan Afrikalı Buşmanlar olarak bu türden alt düzeydeki ırklar dahi; din\hyp{}dışı ve kutsal olanın değerleri arasındaki ilkel bir farklılaşma niteliğindeki birey\hyp{}çıkarı ile toplum\hyp{}çıkarı arasındaki farklılığın bir tanınışına sahiptir. Ancak toplumsal topluluk, dini deneyimin kaynağı değildir. İnsanın öncül dinine olan tüm bu ilkel katkıların etkisinden bağımsız olarak, gerçek dini uyarımın kökenini, bencil\hyp{}olmama iradesini etkinleştiren özgün ruhaniyet mevcudiyetlerinden alması gerçekliğini korumaktadır.
\vs p103 3:2 Daha sonra dinin habercisi, kişilik\hyp{}dışı mana biçimindeki büyük doğa olayları ve gizemlerine duyulan ilkel inanış oldu. Ancak, er veya geç evrimleşen din, diğer insanları daha mutlu ve iyi kılmak için bir şeyler yapma gerekliliği olarak, kendi toplumsal topluluğunun yararı için kişisel fedakârlıkta bulunma gereksinimini şart koşar. Nihai olarak din, Tanrı ve insanın hizmeti haline gelecektir.
\vs p103 3:3 Din, insanın çevresini değiştirmek için tasarlanmıştır; ancak, bugün faniler arasında bulunan dinin büyük bir kısmı bunu gerçekleştirmeye aciz hale gelmiştir. Çevre, haddinden fazla sıklıkla gerçekleşen bir biçimde, dini yönete gelmiştir.
\vs p103 3:4 Her çağda ortaya çıkmış din içinde en önemli şey olan deneyimin; ahlaki değerler ve toplumsal anlamlara dair his olduğunu, din\hyp{}kuramsal dogmalar veya felsefi kuramlar ile ilgili düşünce olmadığını unutmayın. Din, büyü unsurunun yerini ahlaki değerlerin kavramsallaşmasının alırken yararlı bir biçimde evirilme gösterir.
\vs p103 3:5 İnsan; mana hurafeleri, büyü, doğa ibadeti, ruhaniyet korkusu ve hayvan ibadeti boyunca, aracılığıyla bireyin dini tutumunun kavimin topluluk tepkileri haline geldiği çeşitli resmi törenlere doğru evirilmişti. Ve, bu resmi törenler kabile inanışlarına doğru yoğunlaşmış ve katı kalıplara dönüşmüş olup, nihai olarak bu korku ve inançlar tanrılara doğru kişilikleşmiştir. Ancak, bu dini evrimin tümü içerisinde, ahlaki etken hiçbir zaman tamamiyle mevcut olmayan bir konumda bulunmamıştı. İnsan içindeki Tanrı’nın uyarımı her zaman güçlüydü. Ve, bu güçlü etkiler --- biri insan kökenli ve diğeri kutsal olarak; çağların anlık değişimleri boyunca ve bin bir yıkıcı eğilim ve düşmancıl karşıtlık tarafından çok sıkça yok edilmekle tehdit edilmiş olmasına rağmen, dinin kurtuluşunu teminat altına aldı.
\usection{4.\bibnobreakspace Ruhsal Birliktelik}
\vs p103 4:1 Ahlaki bir etkinlik ile dini bir toplanış arasındaki ayırt edici nitelik; din\hyp{}dışı olana kıyasla dini olanda, \bibemph{birliktelik} duygusunun yarattığı havanın hâkim oluşudur. Böylelikle insan birlikteliği kutsal olan ile gerçekleştirilen aidiyet birlikteliğine ait bir his yaratmaktadır; ve bu, topluluk ibadetinin başlangıcıdır. Ortak bir yemeğe katılma, toplumsal birlikteliğin en öncül türüydü; ve, bu nedenle öncül dinler, ibadet edenler tarafından yenilmesi gereken törensel kurbandan bir parça alınmasını şart koydu. Hristiyanlık içinde bile, Koruyucu’nun Akşam Yemeği bu birliktelik türünü korumaktadır. Birlikteliğin yarattığı hava, ikamet eden ruhaniyet Görüntüleyicisi’nin fedakâr arzusu ile benliğini amaçlayan benlik arasında gerçekleşen çatışma içinde ateşkesin yarattığı canlandırıcı ve huzur verici bir dönemi sağlamaktadır. Ve, bu, insanlığın kardeşliğinin ortaya çıkışını mevcut kılan Tanrı’nın mevcudiyetinin uygulaması olarak --- gerçek ibadetin giriş kısmıdır.
\vs p103 4:2 İlkel insan Tanrı ile olan birlikteliğinin kesintiye uğradığını hissettiğinde, arkadaşça ilişkisini eski haline getirmek amacıyla kefarette bulunan bir çaba içerisinde herhangi bir fedada bulunma eylemine başvurdu. Doğruluk için çekilen açlık ve susuzluk gerçekliğin keşfine götürmekte, gerçeklik idealleri çoğaltmakta, ve bu durum, bireysel dindarlar için yeni sorunları yaratmaktadır; zira, ideallerimiz geometrik artışla çoğalırken, onlara uygun bir şekilde yaşama yetimiz sadece aritmetik artışla gelişmektedir.
\vs p103 4:3 Suçluluk duygusu (günah bilinci değil) ya kesintiye uğramış ruhsal birliktelikten doğar, veya bir kişinin ahlaki ideallerinin ölçüsünde yaşamamasından doğar. Bu tür zorlu bir durumdan kurtulma yalnızca; bir kişinin, doğrudan bir biçimde Tanrı’nın iradesiyle aynı düzlemde bulunduğu anlamına gelmeyen, en yüksek ahlaki ideallerinin gerçekleşmesiyle ortaya çıkar. İnsan, en yüksek idealleri uyarınca yaşamayı hayal dahi edemez; ancak, o, Tanrı’yı bulma ve giderek onun gibi olma amacına sadık kalabilir.
\vs p103 4:4 İsa, feda ve kefaretin tüm resmi dini törenlerini çöpe attı. O; insanın Tanrı’nın bir evladı olduğunu duyurarak, tüm bu hayali suçluluğun ve dışlanmışlığın hissini yok etti; yaratılmış\hyp{}Yaratan ilişkisi, bir evlat\hyp{}ebeveyn temeline oturtturulmuştu. Tanrı, fani kız ve erkek evlatları için sevgi dolu bir Baba haline gelmektedir. Bu tür içten bir aile ilişkinin makul görülen bütünlüğünün dışındaki tüm törenler sonsuza kadar sonlandırılmıştır.
\vs p103 4:5 Yaratıcı olan Tanrı, yaratılmışın amacı ve kastı olarak --- evladın güdüsünü tanıma temelinde evladıyla ilişki kurar, mevcut erdem veya liyakat ölçütüne göre değil. Bu ilişki; ebeveyn\hyp{}çocuk birlikteliğinin bir türü olup, kutsal sevgi tarafından gerçekleştirilir.
\usection{5.\bibnobreakspace İdeallerin Kökeni}
\vs p103 5:1 Öncül evrimsel akıl, toplumsal görevin bir hissine kaynaklık etmektedir; ve, ahlaki ödev başat olarak duygusal korkudan kaynağını almıştı. Toplumsal hizmetin ve fedakârlık idealizminin daha olumlu dürtüsü, insan aklı içinde ikamet eden kutsal ruhaniyetin doğrudan uyarımından kökenini almaktadır.
\vs p103 5:2 Başkalarına iyilik yapmanın bu düşünce\hyp{}ideali --- komşusunun yararına olacak bir şey için bir kişinin kendi benliğinin uyarımını reddedişi olarak --- ilk başta çok kısıtlıdır. İlkel insan komşusunu, yalnızca, kendisine arkadaşça davrananlar olarak kendisine en yakın olanları görür; dini medeniyet ilerledikçe bir kişinin komşusu kavramsal olarak kavimi, kabileyi ve milleti içine alacak bir biçimde genişlemektedir. Ve, daha sonra İsa, düşmanlarımızı bile sevmemiz gerektiğini söyleyerek insanlığın tamamını içine alacak şekilde komşu kapsamını genişletti. Ve, olağan her insan varlığı içinde bu öğretinin --- doğru olarak --- ahlaki olduğunu söyleyen bir şey vardır. Bu ideali en az uygulayanlar bile, bunun kuramsal olarak doğru olduğunu kabul etmektedirler.
\vs p103 5:3 İnsanların tümü, fedakâr olma ve topluluğun yararına hizmet etmek için mevcut bu evrensel insan uyarımın ahlaki niteliğini tanımaktadır. İnsancıl görüşü savunan bir kişi, bu uyarımın kökenini maddi aklın doğal çalışımına atfeder; dindar kişi, fani aklın gerçekten fedakâr olan güdüsünün, Düşünce Düzenleyicisi’nin içsel nitelikli ruhani yönlendirmelerine verilen karşılık olarak ortaya çıktığını daha doğru bir biçimde tanır.
\vs p103 5:4 Ancak, insanın benlik\hyp{}iradesi ile benliğin\hyp{}dışındaki\hyp{}irade arasındaki bu öncül çatışmalara dair getirdiği yorum her zaman güvenilir değildir. Yalnızca oldukça yerinde bir biçimde bütünleşmiş kişilik, benlik arzuları ile gelişen toplumsal bilinci arasında yaşanan çok türlü uyuşmazlıklar arasında arabuluculuk yapabilir. Bu sorunu çözmedeki başarısızlık, insansı suçluluk duygularının en öncül türüne kaynaklık etmektedir.
\vs p103 5:5 İnsan mutluluğu yalnızca; bireyin benlik arzusu ile daha yüksek bireyin (kutsal ruhaniyetin) fedakâr uyarımı, bütünleşen ve yüksek bir biçimde denetleyen kişiliğin bir bütün hale gelmiş iradesi tarafından eşgüdümsel hale getirilince ve uzlaştırılınca elde edilir. Evrimsel insanın aklı; duygusal uyarımların doğal gelişimi ile --- özgün dini görüş niteliğindeki --- ruhsal kavrayışa dayanmakta olan fedakâr arzuların ahlaki büyümesi arasındaki mücadeleye hakemlik etmenin bu karmaşık sorunu ile sürekli olarak karşılaşmaktadır.
\vs p103 5:6 Birey ve diğer bireylerin olabilecek en fazla sayıdaki bütünlüğü için eşit düzeyde iyiliği sağlama çabası, bir zaman\hyp{}mekân düzleminde her zaman başarılı bir biçimde çözümlenemeyecek bir sorunu sunmaktadır. Ebedi bir yaşam içinde bu türden karşıtlıklar giderilebilir; ancak, kısa bir insan yaşamı içinde onlar çözülemez niteliktedir. İsa şunları ifade ettiğinde bu türden bir çıkmaza işaret etmişti: “Kim yaşamını kurtaracak olursa, onu kaybedecektir; ancak, kim yaşamını krallık için kaybedecek olursa, onu bulacaktır.”
\vs p103 5:7 Tanrı gibi olmayı arzulama olarak --- idealin arayışı, hem ölümden öncesini hem de sonrasını içine alan devamlı bir çabadır. Ölümden sonraki yaşam, temel nitelikleri bakımından fani mevcudiyetten hiç de farklı değildir. Şimdiki yaşamda iyi niteliğe sahip olan yaptığımız her şey, doğrudan bir biçimde, gelecek yaşamın gelişimine katkı sağlamaktadır. Gerçek din; doğal ölümün kapılarından geçişin bir sonucu olarak bir kişi üzerine soylu bir karakterin tüm erdemlerinin bahşedileceğine dair içi boş ümidi destekleyerek, ahlaki eylemsizliği ve ruhsal tembelliği teşvik etmez. Gerçek din, yaşam üzerindeki fani kiracılığı boyunca insanın ilerleme çabalarını anlamsız görmez. Her fani kazanımı, ölümsüz kurtuluş deneyiminin ilk aşamalarının zenginleşmesine doğrudan bir katkıdır.
\vs p103 5:8 Fedakâr uyarımlarının tümünün yalnızca onun doğal sürü içgüdülerinin gelişimi olduğunun kendisine öğretilmesi, insanın idealizmi için ölümcüldür. Ancak, o, ruhunun bu daha yüksek uyarımlarının fani aklında ikamet eden ruhsal güçlerden kaynaklandığını öğrendiğinde soylulaşmakta ve içi aşırı derecede yaşam enerjisiyle dolmaktadır.
\vs p103 5:9 Kendisi içinde yaşayanın, ebedi ve kutsal olan bir şeyi arzulayanın tamamiyle farkına bir kez vardığında, bu insanı kendisi içinde uyandırmakta ve ötesine taşımaktadır. Ve böylece, hepimizin Tanrı’nın evlatları olduğumuza dair inanışımızı doğrulayan ve insanın kardeşliğinin hisleri olarak fedakâr nitelikli yargılarımızı gerçek kılan şey, ideallerimiz içindeki insan\hyp{}ötesi kökene beslenen bu yaşayan inançtır.
\vs p103 5:10 Ruhsal nüfuz alanında insan, kesin bir biçimde, bir özgür iradeye sahiptir. Fani insan; ne her\hyp{}şeye\hyp{}gücü\hyp{}yeten bir Tanrı’nın katı egemenliğine olan çaresiz bir köledir, ne de, tamamiyle sebep\hyp{}sonuçsal nitelikte bulunan mekanik bir kâinatsal işleyiş düzeninin kaçınılmaz sonunun kurbanıdır. İnsan, olası en gerçek anlamıyla, kendi ebedi nihai sonunun mimarıdır.
\vs p103 5:11 Ancak, insan, baskıyla kurtarılmış veya diğer bir değişle soylu hale getirilmemiştir. Ruhsal büyüme, evrimleşen ruhtan doğmaktadır. Baskı kişilik üzerinde değişiklikte bulunabilir, ancak o hiçbir zaman büyümeyi harekete geçirmez. Eğitimsel baskı bile yalnızca, yıkıcı deneyimlerin önlenmesine destek olabilecek bir biçimde önleyici nitelikte yardımcıdır. Ruhsal büyüme, tüm çevre baskıları en alt düzeyde olduğunda en üst konumundadır. “Koruyucu’nun ruhaniyeti neredeyse, özgürlük oradadır.” İnsan; evin, cemiyetin, dini kurumun ve devletin baskıları en alt düzeyde olduğunda en iyi şekilde gelişmektedir. Ancak, bu durumdan, ilerleyici bir toplum içerisinde eve, toplum kurumlarına, din kurumuna ve devlete hiçbir yer olmadığı anlamı çıkarılmamalıdır.
\vs p103 5:12 Toplumsal bir dini topluluğun bir üyesi bu türden bir topluluğun şartlarına uyum sağladığında, bu kişinin; dini inanışın gerçeklerine ek olarak dini deneyimin gerçekliklerine dair kendi bireysel yorumunu bütüncül bir biçimde ifade etmedeki dini serbestliği memnuniyetle deneyimlemesi teşvik edilmelidir. Dini bir topluluğun güvencesi ruhsal birlikteliğe bağlıdır, din\hyp{}kuramsal özdeşliğe değil. Bir dini topluluk, “reddediciler” haline gelme zorunluluğuna düşmeden özgür düşünce serbestliğini memnuniyetle deneyimlemeye yetkin olmalıdır. Yaşayan Tanrı’ya ibadet eden, insanlığın kardeşliğini onaylayan ve üyeleri üzerindeki tüm öğretisel baskıyı kaldırmaya cesaret eden her din kurumda büyük umut vardır.
\usection{6.\bibnobreakspace Felsefi Eşgüdüm}
\vs p103 6:1 Din\hyp{}kuramı alanı, insan ruhaniyetinin etkileri ve tepkilerinin çalışmasıdır; o hiçbir zaman bir bilim haline gelemez, çünkü o her zaman, kişisel nitelikli ifadesinde psikolojiye ek olarak düzenli tasvirinde felsefeyi içine almak zorundadır. Din\hyp{}kuramı alanı her zaman, \bibemph{sizin} dininizin çalışmasıdır; başkasının dininin çalışması psikolojidir.
\vs p103 6:2 İnsan evrenini incelemeye ve irdelemeye \bibemph{dışarıdan} yaklaşınca, çeşitli fiziksel bilimleri ortaya çıkarmaktadır; kendini ve evreni \bibemph{içeriden} araştırmaya yaklaşınca, din\hyp{}kuramını ve metafiziği yaratmaktadır. Daha sonraki felsefe sanatı; nesne ve varlıkların evrenine yaklaşımda bu iki taban tabana zıt yolun bulguları ve öğretileri arasında, ilk başta ortaya çıkması kesin olan birçok ayrışmayı uyumlu hale getirmenin bir çabası içinde gelişir.
\vs p103 6:3 Din, insan deneyiminin \bibemph{içselliğinin} farkındalığı olarak ruhsal bakış açısıyla ilgilidir. İnsanın ruhsal doğası ona, evrenin dışını içe çevirme olanağı sunar. Bu nedenle, tamamiyle kişisel deneyimin içselliğinden bakıldığında, yaratımın tümünün özünde ruhsal olduğu görünür.
\vs p103 6:4 İnsan fiziksel duyularına ve onunla ilişkili akıl algısına ait maddi kazanımları aracılığıyla evreni mantıksal bir biçimde çözümlemeye çalıştığında, kâinatın mekanik ve enerji\hyp{}maddesi halinde görünür. Gerçekliği irdelemenin bu türden bir yöntemi, evrenin içini dışına çevirmekle gerçekleşir.
\vs p103 6:5 Evrenin mantıksal ve tutarlı bir felsefi kavramsallaşması, ne tamamiyle maddecilik ne de tamamiyle ruhsalcılık varsayımları üzerine inşa edilemez; zira bu düşünce sistemlerinin ikisi de, evreni bir bütün olarak ele aldığında, kâinatı gerçeğinden uzaklaşarak görmeye mahkûmdur; maddecilik bir evrenin içini dışına çeviren bir biçimde irdelerken, ruhsalcılık bir evrenin doğasını dışının içine çevirmiş halde görür. Böyle olduğu müddetçe ne bilim ne de din hiçbir zaman, tek başına, kendileri içinde ve kendileri aracılığıyla insan felsefesinin rehberliği ve kutsal açığa çıkarılışın aydınlatımı olmadan evrensel gerçeklikler ve ilişkilerin yeterli bir anlayışını elde etmeyi hayal dahi edemez.
\vs p103 6:6 İnsanın içsel ruhaniyeti her zaman, kendisini ifade edişi ve kendisini gerçekleştirişi için aklın işleyiş düzenine ve yöntemine dayanmak zorundadır. Benzer bir biçimde, insanın maddi gerçekliğe ait dışsal deneyimi, deneyimleyen kişiliğin akıl bilincini temel almak zorundadır. Bu nedenle, içsel ve dışsal olarak ruhsal ve maddi insan deneyimleri her zaman akıl faaliyeti ile ilişkili olup, bilinçli gerçekleşimleri bakımından akıl etkinliği tarafından belirlenir. Us, fani deneyimin bütününün uyumlaştırıcısı, her daim belirleyicisi ve seçicisidir. Hem enerji\hyp{}maddeleri hem de ruhaniyet değerleri, bilincin akıl aracılığıyla gerçekleşen yorumlarıyla şekillenir.
\vs p103 6:7 Bilim ve din arasında daha uyumlu bir eşgüdümü elde etmede yaşadığınız zorluk, morontia dünyasına ait nesneler ve varlıkların arada kalan bölgesine dair bütüncül bilgisizliğinizden kaynaklanmaktadır. Yerel evren, gerçekliğin dışavurumunun şu üç düzeyi, veya başka bir değişle, aşamasından, meydana gelmektedir: madde, morontia ve ruhaniyet. Morontia bakış açısı, fiziksel bilimlerin bulguları ile dinin ruhaniyet faaliyeti arasındaki tüm farklılığı ortadan kaldırır. Nedensellik, bilimlerin anlama yöntemidir; inanç, dinin kavrayış yöntemidir; mota, morontia düzeyinin yöntemidir. Mota; niteliğini bilgi temelli nedensellikten ve kökenini inanç temelli kavrayıştan alan bir biçimde, tamamlanmamış büyümeyi telafi etmeye başlayan bir madde\hyp{}ötesi gerçeklik duyarlılığıdır. Mota, maddi kişilikler tarafından erişilemez nitelikte bulunan, ayrık gerçeklik algısının bir felsefe\hyp{}ötesi birleşimidir; bu birleşim, bir ölçüde, bedenin maddi yaşamından kurtulmuş olmanın deneyimine bağlıdır. Ancak, birçok fani, bilim ve dinin geniş ölçüde ayrılmış alanları arasındaki karşılıklı etkiler ağını birleştiren belirli bir yöntemi bulmanın olumluluğunu tanımışlardır; ve, metafizik, bu oldukça iyi tanınan uçurumun aşılmasında insanın başarıyla sonuçlanmamış çabasının ürünüdür. Ancak, insan metafiziğinin aydınlatmadan çok kafa karıştırıcı olduğu kendisini göstermiştir. Metafizik, morontia motasının yokluğunu telafi etmek için insanın iyi niyetli ancak faydasız çabası anlamına gelmektedir.
\vs p103 6:8 Metafiziğin bir başarısızlık olduğu kendisini göstermiştir; motayı insan algılayacak yetkinlikte değildir. Açığa çıkarılış, maddi bir dünyada motanın gerçeklik duyarlılığının yokluğunu telafi edebilecek tek yöntemdir. Açığa çıkarılış, yekin bir biçimde, evrimsel bir dünya üzerinde nedensellikle gelişmiş metafiziğin yarattığı kafa karışıklığını gidermektedir.
\vs p103 6:9 Bilim, enerji\hyp{}maddesinin dünyası olarak fiziksel çevresine dair insanın giriştiği çalışmadır; din, ruhaniyet değerlerinin kâinatı ile insanın sahip olduğu deneyimdir; felsefe, bu geniş bir biçimde ayrılmış kavramsallaşmaların bulgularını kâinatın tamamına yönelik kabul edilebilir ve bütüncül bir tutuma doğru düzenleme ve ilişkilendirme için verilen insanın akıl çabası tarafından geliştirilmiştir. Felsefe, açığa çıkarılış tarafından daha açık hale getirilmiş haliyle; motanın yokluğuna ilaveten --- metafizik olarak --- motanın yerine insanın getirdiği nedensellik denginin iflası ve başarısızlığının olduğu durumda kabul edilebilir bir biçimde faaliyet göstermektedir.
\vs p103 6:10 Öncül insan, enerji düzeyini ve ruhaniyet düzeyini birbirinden ayırt edemedi. Matematiksel olanı iradesel olandan ayırmaya ilk kez girişmiş olanlar, eflatun ırkı ve onların And soylarıydı. Artan bir biçimde medeni insan, cansız olan ile canlı olanı birbirinden ayırt etmiş öncül Yunanlılar ve Sümerler’in ayak izini takip etmiştir. Ve, medeniyet ilerledikçe felsefe, ruhaniyet kavramsallaşması ile enerji kavramsallaşması arasındaki sürekli genişleyen uçurumları birleştirmek zorunda olacaktır. Ancak, mekânın zamanı içerisindeki bu ayrımlar, Yüce’de bir bütündür.
\vs p103 6:11 Her ne kadar hayal etme ve bütüncül bilgiye dayanmayan düşünce sınırlarını genişletmeye yardımcı olsa da, bilim, her zaman, temelini nedensellikten almalıdır. Her ne kadar nedensellik istikrarlaştırıcı bir etki ve yardımcı bir etkense de, din, sonsuza kadar, inanca bağlıdır. Ve, orada her zaman, yanlış bir biçimde bilimler ve dinler olarak adlandırılmakta olan hem doğal hem de ruhsal dünyaların olgularına ait yanlış yönlendirici yorumlar bulunmuştur ve bulunacaktır.
\vs p103 6:12 Bilimi bütüncül bir biçimde kavrayamayışından, dine zayıf bir biçimde bağlanışından ve metafizikteki başarısız girişimlerinden insan, felsefe düşüncelerini inşa etmeye girişmiştir. Ve, fiziksel ve ruhsal olan arasındaki morontia uçurumunu birleştirmede metafiziğin başarısızlığı olarak, insanın giriştiği, madde ve ruhaniyetin dünyaları arasındaki hayati derecede önemli ve olmazsa olmaz metafiziksel birleşimin iflası olmasaydı, çağdaş insan kendisi ve evreni ile ilgili gerçekten de değerli ve etkili bir felsefe inşa etmiş olacaktı. Fani insan, morontia aklı ve maddesinin kavramsallaşmasından yoksundur; ve, \bibemph{açığa çıkarılış}, evrene dair mantıklı bir felsefe inşa edebilmeye ek olarak bu evren içindeki kesin ve yerleşik konumunun tatminkâr bir anlayışına erişebilmek için oldukça hayati bir biçimde ihtiyaç duyduğu, kavramsal bilgiler içindeki zafiyetini gidermedeki tek yöntemdir.
\vs p103 6:13 Açığa çıkarılış, morontia uçurumunu birleştirmek için evrimsel insanın tek ümididir. Mota tarafından desteklenmemiş haldeki inanç ve nedensellik, mantıklı bir evreni düşünemez ve onu inşa edemez. Mota’nın kavrayışı olmadan fani insan, maddi dünyanın olgularında iyiliği, sevgiyi ve gerçekliği kavrayamaz.
\vs p103 6:14 İnsanın felsefesi madde dünyasına aşırı bir biçimde eğildiğinde, mantıksal veya \bibemph{doğasal} hale gelmektedir. Felsefe özellikle ruhsal düzeye yöneldiğinde, \bibemph{idealist} veya gizemci bile olmaktadır. Felsefe metafizikten temelini alacak kadar talihsiz olduğunda, hataya yer bırakmayan bir biçimde, kafası karışmış biçimde \bibemph{kuşkucu} olmaktadır. Geçmiş çağlarda, insanın bilgisi ve ussal değerlendirmelerinin çoğu, algının bu üç bozuluşundan bir tanesine düşmüştür. Felsefe, gerçekliğe dair yorumlarını mantığın doğrusal doğrultusuna uyarlamaya cüret etmemelidir; o hiçbir zaman, gerçekliğin elipsel simetrisini ve tüm ilişki kavramsallaşmalarının sahip olduğu temel eğik düzlemi hesaba katmada başarısız olmamalıdır.
\vs p103 6:15 Fani insanın erişebileceği en yüksek felsefe, mantıklı bir biçimde; bilimin nedenselliğine, dinin inancına ve açığa çıkarılış tarafından sağlanan gerçeklik kavrayışına dayanmak zorundadır. Bu birliktelikle insan, yeterli bir metafizik geliştirememedeki başarısızlığını ve morontianın motasını kavramadaki yetkinsizliğini bir ölçüde telafi edebilir.
\usection{7.\bibnobreakspace Bilim ve Din}
\vs p103 7:1 Bilim, nedensellikle beslenir; din, inançla. İnanç, her ne kadar nedenselliğe dayanmasa da, makuldür; her ne kadar mantıktan bağımsız olsa da, yine de güçlü mantık tarafından desteklenmektedir. İnanç, bir ideal felsefe ile bile beslenemez; gerçekten, o, bilimle birlikte, bu türden bir felsefenin çıktığı kaynağın tam da kendisidir. İnsanın dinsel kavrayışı olarak inanç; yalnızca, ruhaniyet olan Tanrı’nın ruhsal Düzenleyici mevcudiyeti ile faninin sahip olduğu kişisel deneyim tarafından kesin bir biçimde yükseltilebilen bir biçimde, yalnızca açığa çıkarılış tarafından eğitilebilir.
\vs p103 7:2 Gerçek günahtan arınış; madde özdeşleşiminden başlayarak, madde ve ruhaniyet arasındaki morontia irtibat âlemleri boyunca bu ikisi arasındaki nihai ruhsal ilişkilendirimin yüksek evren düzeyine olan fani aklın kutsal evrimi yöntemidir. Ve, maddi nitelikli hissel içgüdü, karasal evrimde nedensel temele oturtturulmuş bilginin ortaya çıkışından önce gelirken; ruhsal nitelikli hissel kavrayışın dışavurumu, geçici olan insanın potansiyellerini bir Cennet kesinlik unsuru olarak ebedi olan insanın gerçekliğine ve kutsallığına dönüştürme işleyişi niteliğindeki göksel evrimin ulvi düzeni içinde, morontia ve ruhaniyet nedenselliğinin ve deneyiminin daha sonraki ortaya çıkışının habercisi olmaktadır.
\vs p103 7:3 Ancak, yükseliş halindeki insan Tanrı deneyimi için içe doğru ve Cennet\hyp{}yolunda ilerlerken, benzer bir biçimde, maddi kâinatın bir enerji anlayışı için dışa doğru ve uzay doğrultusunda ilerleyecektir. Bilimin ilerleyişi, inanın yeryüzüsel yaşamıyla kısıtlı değildir; onun evren ve aşkın\hyp{}evren yükseliş deneyimi, enerji dönüşümü ve madde başkalaşımının çalışmasından hiçbir biçimde daha az olmayacaktır. Tanrı ruhaniyettir, ancak İlahiyat birlikteliktir; ve, İlahiyat’ın birlikteliği yalnızca Kâinatın Yaratıcısı ve Ebedi Evlat’ın ruhsal değerlerini içine almaz, aynı zamanda Kâinatsal Düzenleyici ve Cennet Adası’nın enerji gerçekliklerinin bilincine sahipken, kâinatsal gerçekliğin bu iki fazı Bütünleştirici Bünye’nin akıl ilişkilerinde kusursuz bir biçimde ilişkilendirilmiş olup, Yüce Varlık’ın ortaya çıkan İlahiyatı’nda sınırlı düzeyde birleştirilir.
\vs p103 7:4 Deneyimsel felsefenin aracılığıyla bilimsel tutum ve dinsel kavrayışın birlikteliği, insanın uzun Cennet\hyp{}yükseliş deneyiminin bir parçasıdır. Matematiğin tahminleri ve kavrayışın kesinlikleri her zaman, Yüce’nin olası en yüksek erişimi yoksunluğunda deneyimin tüm aşamaları üzerinde akılsal mantığın uyumlaştırıcı faaliyetini gerektirecektir.
\vs p103 7:5 Ancak, mantık hiçbir zaman; bir kişiliğin bilimsel ve dini yönleri, sonunda kendisini bulacağı hangi sonuca götüreceğinden bağımsız olarak gerçekliğin peşine düşmenin içten arzusunu duyan bir biçimde gerçekliğin baskınlığı altında olmadıkça, bilimin bulguları ile dinin kavrayışlarını uyumlu hale getirmede başarılı olamaz.
\vs p103 7:6 Mantık, sahip olduğu ifade yöntemi olarak felsefenin işleyiş biçimidir. Gerçek bilimin nüfuz alanı içinde, nedensellik her zaman özgün mantığın etkisi altındadır; gerçek dinin nüfuz alanı içinde inanç her zaman, her ne kadar bilimsel yaklaşımın dışarıdan bakışıyla oldukça temelsiz görünebilse de, içsel bir bakış açısı temelinde her zaman mantıksaldır. İçe doğru bir biçimde dışarıdan bakıldığında evren maddi olarak görünebilir; ancak, dışa doğru bir biçimde içten bakıldığında aynı evren tamamiyle ruhsal olarak görünmektedir. Nedensellik maddi farkındalıktan, inanç ruhsal farkındalıktan doğmaktadır; ancak, açığa çıkarılış tarafından güçlendirilmiş bir felsefenin aracılığıyla mantık, aracılığıyla hem bilim hem de dinin istikrarını gerçekleştiren bir biçimde hem içsel hem de dışsal olan bakışı doğrulayabilir. Böylelikle, felsefenin mantığı ile olan ortak ilişki vasıtasıyla hem bilim hem de din; gittikçe azalan bir biçimde kuşkucu olarak birbirine karşı artan bir biçimde hoşgörülü hale gelebilir.
\vs p103 7:7 Gelişen dinin ve bilimin ikisinin de ihtiyaç duyduğu şey, evrimsel düzeydeki tamamlanmamışlığın daha büyük bir farkındalığı olarak daha fazla arayış içindeki ve korkusuz özeleştiridir. Hem dinin hem de bilimin öğretmenlerinin tümü çoğu zaman, aşırı bir biçimde kendinden emin ve dogmatiktir. Bilim ve din yalnızca, sahip oldukları \bibemph{bilgi\hyp{}gerçekliklerine} karşı özeleştirel tutum besleyebilirler. Bilgi\hyp{}gerçeklikleri aşaması terk edildiği anda, nedensellik sorumluluğu üstlenmede başarısız olur veya bunun yerine hızlı bir biçimde sahte mantığın bir birlikteliğine doğru yozlaşır.
\vs p103 7:8 Kâinatsal ilişkilerin, evrensel bilgi\hyp{}gerçekliklerinin ve ruhsal değerlerin bir anlayışı olarak --- gerçekliğe en iyi Gerçekliğin Ruhaniyeti’nin hizmeti vasıtasıyla elde edilebilir olup, en iyi bir biçimde\bibemph{ açığa çıkarılış} tarafından eleştirilir. Ancak, açığa çıkarılış ne bir bilimi ne de bir dini meydana getirmektedir; onun faaliyeti, hem bilim hem de dini gerçeğin gerçekliği ile eşgüdümsel hale getirmektir. Her zaman, açığa çıkarılışın yoksunluğunda yahut onu kabul etmedeki veya kavramadaki başarısızlıkta fani insan; gerçekliğin açığa çıkarılışının veya morontia kişililiğinin motasının tek insan eşleniği olarak faydasız metafiziksel ifadesine başvurmuştur.
\vs p103 7:9 Maddi dünyanın bilimi, insanı; fiziksel çevresini denetlemesi ve belli bir ölçüde onun üzerinde üstünlük kurmasına yetkin hale getirmektedir. Ruhsal deneyimin dini, bir bilimsel çağın medeniyetine ait karmaşık koşullarda insanları beraber yaşamaya yetkin hale getiren birliktelik uyarımının kökenidir. Metafizik, ancak daha kesin bir biçimde açığa çıkarılış; hem bilim hem de dinin keşifleri için ortak bir buluşma noktası sağlamakta, ve, bu ayrı fakat bağımsız düşünce alanlarını insanın mantıksal bir biçimde bilimsel istikrarın ve ruhsal kesinliğin oldukça dengeli bir felsefesine doğru ilişkilendirmesini mümkün kılmaktadır.
\vs p103 7:10 Fani düzey içinde hiçbir şey mutlak bir biçimde kanıtlanamaz; hem bilim hem de din, varsayımlara dayanmaktadır. Morontia düzeyinde, hem bilimin hem de dinin olası kabulleri, mota mantığı tarafından kısmi bir biçimde doğrulanmaya yetkindir. Olası en yüksek düzeyin ruhsal seviyesinde sınırlı kanıt için duyulan ihtiyaç, gerçeklik ve onunla birlikte olan mevcut deneyim karşısında kademeli olarak ortadan kalmaktadır; ancak, orada bile, hala kanıtlanmamış konumda bulunan sınırlı olanın ötesinde birçok şey bulunmaktadır.
\vs p103 7:11 İnsan düşüncesinin tüm birimleri, insanın akıl kazanımına ait yapıcı gerçeklik duyarlılığı tarafından kabul edilen, ancak kanıtlanmamış, belirli varsayımlara dayanmaktadır. Bilim, gurur duyduğu nedensellik sürecine şu üç şeyin gerçekliğini \bibemph{varsayarak} başlar: madde, hareket ve yaşam. Din, şu üç şeyin doğruluğunun varsayımıyla başlar: Yüce Varlık olarak --- akıl, ruhaniyet ve evren.
\vs p103 7:12 Bilim, zamanın mekân içindeki enerjisi ve maddesine ait olarak matematiğin düşünce alanı haline gelir. Din, sadece sınırlı ve geçici olan ruhaniyetle değil aynı zamanda ebediyet ve üstünlüğün ruhaniyeti ile ilgilenme sorumluluğunu da üstlenir. Yalnızca, motadaki uzun bir deneyim aracılığıyla evren algısının bu iki aşırı ucu; kökenlerin, işlevlerin, ilişkilerin, gerçekliklerin ve nihai sonların karşılaştırılabilir nitelikteki benzer yorumlarını üretecek konuma getirebilir. Enerji\hyp{}ruhaniyet ayrımının olası en yüksek uyumu, Yedi Üstün Ruhaniyet’in döngüleşmesi içinde gerçekleşir; onun ilk birleşimi Yüce’nin İlahiyatı’nda gerçekleşir; onun son birlikteliği, BEN olan İlk Kaynak ve Merkez’in sınırsızlığında ortaya çıkar.
\vs p103 7:13 \bibemph{Nedensellik}, enerji ve maddenin fiziksel dünyasında ve onunla birlikte gerçekleşen deneyime dair bilincin yargılarını tanıma eylemidir. \bibemph{İnanç}, geride kalan herhangi bir fani kanıtın erişemeyeceği bir şey olarak --- ruhsal bilincin doğruluğunu tanıma eylemidir. \bibemph{Mantık}; inanç ve nedenselliğin birlikteliğine ait birleşimsel gerçeği\hyp{}arama ilerleyişi olup, maddelerin, anlamların ve değerlerin içkin farkındalığı olarak fani varlıkların yapıcı akıl kazanımları üzerine inşa edilmiştir.
\vs p103 7:14 Düşümce Düzenleyicisi’nin mevcudiyeti içinde ruhsal gerçekliğin gerçek bir kanıtı bulunmaktadır; ancak bu mevcudiyetin doğruluğu, dışsal dünya için gösterilebilir konumda değildir, ancak Tanrı’nın ikametini bu şekilde deneyimleyen kişi için bu gerçekliktedir. Düzenleyici’nin bilinci; gerçekliğin ussal algısına, iyiliğin akıl\hyp{}ötesi kavrayışına ve sevmek için var olan kişilik güdüsüne bağlıdır.
\vs p103 7:15 Bilim, maddi dünyayı keşfeder; din, onu değerlendirir; ve, felsefe, onun anlamlarını yorumlamaya çalışırken, bilimsel olan maddi bakış açısını dini nitelikteki ruhsal kavramsallaşmayla eşgüdümsel hale getirir. Ancak, tarih, bilim ve dinin hiçbir zaman bütünüyle anlaşamayabileceği bir alandır.
\usection{8.\bibnobreakspace Felsefe ve Din}
\vs p103 8:1 Her ne kadar hem bilim hem de felsefe Tanrı’nın olasılığını nedensellikleriyle ve mantıklarıyla varsayabilse de, yalnızca, ruhaniyet tarafından yönlendirilmekte olan bir insanın kişisel dini deneyimi bu türden yüce ve kişisel İlahiyat’ın kesinliğini olumlayabilir. Yaşayan gerçekliğin bu türden bir doğumunun yöntemi vasıtasıyla, Tanrı’nın olasılığına dair felsefi varsayım dini bir gerçeklik haline gelebilir.
\vs p103 8:2 Tanrı’nın kesinliğinin deneyimlenişine dair kafa karışıklığı, farklı bireylere ek olarak insanların farklı ırkları tarafından sahip olunan bahse konu deneyime ait birbirine benzemeyen yorumlardan ve ilişkilerden doğmaktadır. Tanrı’yı deneyimleme tamamen gerçek olabilir; ancak, ussal ve felsefi nitelikteki Tanrı \bibemph{hakkında} yürütülen söylem ayrık ve sıklıkla kafa karıştıran bir biçimde gerçek dışıdır.
\vs p103 8:3 İyi ve soylu bir insan en bütüncül evlilik ölçütlerinde eşine âşık olabilir, ancak, evlilik aşkına dair yazılı psikoloji testi başarılıyla geçmede hiçbir şekilde yetkin olmayabilir. Karısı için neredeyse hiç sevgi beslemeyen başka bir insan, bu türden bir sınavı olabilecek en yüksek düzeyde geçebilir. Kendisine sevgi beslenenin gerçek doğasına dair sevgi duyanın kavrayışındaki tamamlanmamışlık, hiçbir biçimde ne onun sevgisinin gerçekliğini ne de içtenliğini en düşük biçimde bile gerçek dışı kılmaz.
\vs p103 8:4 İnançla onu bilip, ona derin sevgi duyan bir biçimde --- Tanrı’ya gerçekten inanıyorsanız; bilimin şüpheci imalarının, mantığın gereksiz eleştirilerinin, felsefenin varsayımsal düşüncelerinin veya Tanrı olmadan bir din yaratmak isteyen iyi niyetli ruhların zeki önderlerinin herhangi bir biçimde böyle bir deneyimin gerçekliğini azaltmasına veya onun verdiği neşeyi kaçırmasına izin vermeyin.
\vs p103 8:5 Tanrı\hyp{}bilen dindarın kesinliği, şüphe duyan maddeci bireyin belirsizliği tarafından rahatsız edilmemelidir; bunun yerine, inanmayan kişinin belirsizliği, deneyimsel inananın derin inancı ve sarsılmaz kendinden eminliği tarafından güçlü bir biçimde sınanmalıdır.
\vs p103 8:6 Felsefe, hem bilime hem de dine en büyük hizmeti sağlayanlardan biri olarak, hem maddeciliğin hem de her şeyin özünde Tanrı’nın bulunduğu görüşünün aşırı uçlarından kaçınmalıdır. Yalnızca, kişilik gerçekliğini tanıyan bir felsefe --- değişimin mevcudiyeti içinde kalıcılık olarak --- maddi bilim ve ruhsal dinin kuramları arasında bir irtibatı aracı olarak hizmet edebilir. Açığa çıkarılış, evrimleşen felsefenin zafiyetleri için bir telafidir.
\usection{9.\bibnobreakspace Din’in Özü}
\vs p103 9:1 Din\hyp{}kuramı dinin ussal içeriğiyle ilgilenirken, metafizik (açığa çıkarılış) felsefi özellikleri ile ilgilenir. Dini deneyim, dinin ruhsal içeriğinin \bibemph{tam da kendisidir}. Dinin ussal içeriği içerisinde bulunan engel olunamaz nitelikteki mitsel değişikliklere ve psikolojik yanılsamalara, hatalı metafiziksel varsayımlara ve bireyin kendisini yanıltmasının yöntemlerine, dinin felsefi içeriğinin siyasi çarpıtmalarına ve sosyoekonomik bozulmalara rağmen, kişisel dinin ruhsal deneyimi içten ve geçerli niteliğini korumaya devam etmektedir.
\vs p103 9:2 Din; hissetmekle, eylemde bulunmakla ve yaşamakla ilgilidir, sadece düşünmekle değil. Düşünme yakın bir biçimde maddi yaşamla ilgili olup, gerçeklik aracılığıyla ruhsal âlemlere gerçekleştirdiği maddi\hyp{}olmayan yaklaşımları, başlıca, ancak bütünüyle değil, nedenselliğin ve bilimin bilgi\hyp{}gerçekliklerinin üstünlüğünde olmalıdır. Bir kişinin din\hyp{}kuramı ne kadar gerçeğin dışında ve ne kadar hatalı olursa olsun, bir kişinin dini tamamiyle içten ve sonsuza kadar doğru olabilir.
\vs p103 9:3 Özgün halinde Budizm; her ne kadar gelişme sürecinde tanrısız niteliğinde artık bulunmasa da, Urantia’nın tüm evrimsel tarihi boyunca bir Tanrı olmadan doğmuş en iyi dinlerden biridir. İnanç olmadan din, bir çelişkidir; Tanrı olmadan, bir felsefi tutarsızlık ve bir ussal mantıksızlık durumudur.
\vs p103 9:4 Doğal dinin büyüsel ve mitsel doğum kökeni, daha sonraki açığa çıkarımsal dinlerin mevcudiyetine ve gerçekliğine ek olarak İsa’nın dininin tamamlayıcı konumdaki kurtarıcı müjdesini geçersiz kılmamaktadır. İsa’nın yaşamı ve öğretileri nihai olarak; büyünün hurafelerini, mitolojinin hayali ürünlerini ve geleneksel dogmacılığın esaretini bünyesinden uzaklaştırmıştır. Ancak, bu öncül büyü ve mitoloji oldukça etkili bir biçimde; madde\hyp{}ötesi değerler ve varlıkların mevcudiyetini ve gerçekliğini varsayarak daha sonraki ve üstün dinin ortaya çıkmasına zemin hazırlamıştı.
\vs p103 9:5 Her ne kadar dini deneyim tamamiyle ruhsal nitelikli öznel bir olgu olsa da, bu türden bir deneyim evrenin nesnel gerçekliğinin en yüksek düzeylerine doğru olumlu ve yaşayan bir inanç tutumunu bünyesinde barındırır. Dini felsefenin ideali, kâinat evrenlerinin tümünün sınırsız Yaratıcısı’na dair mutlak sevgiye koşulsuz bir biçimde bağlı kılacak türde bir inanç\hyp{}güvenidir. Bu gibi içten bir dini deneyim, idealist arzunun felsefi nesneleştirişinin çok ötesindedir; o, gerçekte, kurtuluştan tamamen emin olup, kendisini yalnızca, Cennet içindeki Yaratıcı’nın iradesini öğrenmekle ve onu yerine getirmekle ilgilenen konuma getirir. Bu türden bir dinin ayırt edici niteliği: bir yüce İlahiyat’a beslenen inanç, ebedi kurtuluşun ümidi ve, özellikle kişinin kendi akranlarına beslediği, sevgidir.
\vs p103 9:6 Din\hyp{}kuramı din üzerinde üstünlük kurduğu anda, din ölür; o, bunun yerine, bir yaşamın öğretisi haline gelir. Din\hyp{}kuramının amacı sadece, kişisel nitelikli ruhsal deneyimin öz\hyp{}bilincini kolaylaştırmaktır. Din\hyp{}kuramı; son kertede yalnızca yaşayan inanç tarafından olumlanabilecek dinin deneyimsel nitelikli savlarına açıklık getirme, açıklama ve meşruiyet kazandırmanın dini çabasını oluşturur. Evrenin daha yüksek felsefesi içinde bilgelik, tıpkı nedensellik gibi, inanç ile aynı safta yer alan konuma gelir. Nedensellik, bilgelik ve inanç insanın insan düzeyinde gerçekleştirdiği en yüksek kazanımlarıdır. Nedensellik insana, nesneler olarak bilgi\hyp{}gerçekliklerin dünyasını tanıştırır; bilgelik insana, ilişkiler olarak gerçekliğin bir dünyasını tanıştırır; inanç insanı, ruhsal deneyim olarak kutsallığın bir dünyasına kabul eder.
\vs p103 9:7 İnanç; nedensellik tek başına ilk önce kendi arayışına çıkıp, daha sonra ihtiyaç hissederek bilgeliği yanına alıp felsefenin tüm sınırlarına ulaştıkça, nedenselliği beraberinde taşır; ve, bunun sonrasında, yalnızca GERÇEKLİĞİN eşliğinde sınırları olmayan ve bitmek tükenmek bilmeyen evren yolculuğuna girişmeye cesaret eder.
\vs p103 9:8 Bilim (bilgi); evrenin kavranabilir nitelikte, nedenselliğin geçerli olduğuna dair içkin (emir\hyp{}yardımcı ruhaniyetinin) varsayımı üzerine inşa edilmiştir. Felsefe (eşgüdüm halindeki kavrayış); maddi evrenin ruhsal olan ile eşgüdümsel hale getirilebileceği nitelikte, bilgeliğin geçerli olduğuna dair içkin (bilgeliğin ruhaniyetinin) varsayımı üzerine inşa edilmiştir. Din (kişisel olan ruhsal deneyimin gerçekliği); Tanrı’nın bilinebileceği ve ona erişebileceği nitelikte, inancın geçerli olduğuna dair içkin (Düşünce Düzenleyicisi’nin) varsayımı üzerine inşa edilmiştir.
\vs p103 9:9 Fani yaşamın gerçekliğinin bütüncül gerçekleşimi; nedensellik, bilgelik ve inancın bu varsayımlarına inanmadaki ilerleyici bir gönüllülükten meydana gelmektedir. Bu türden bir yaşam, gerçeklikle güdülenen ve sevginin egemen olduğu bir yaşamdır; ve, bunlar, sahip olduğu mevcudiyeti maddi bir biçimde gösterilemeyecek olan nesnel kâinatsal gerçekliğin idealleridir.
\vs p103 9:10 Nedensellik doğru ve yanlışı bir kez tanıdığında, bilgeliği sergilemiş olur; bilgelik, gerçeklik ve yanılgı olarak doğru ve yanlış arasında tercihte bulunduğunda, ruhsal rehberliği göstermiş olur. Ve bu nedenle, aklın, ruhun ve ruhaniyetin işlevleri en başından beri birbirlerine yakın bir biçimde iç içe ve birbirleriyle işlevsel olarak ilişkilidir. Nedensellik, bilgisel gerçekliklerin bilgisiyle ilgilenir; bilgelik, felsefe ve açığa çıkarılışla; inanç, yaşayan ruhsal deneyimle. Gerçeklikle insan güzelliğe erişip, ruhsal sevgi ile iyiliğe yükselir.
\vs p103 9:11 İnanç, Tanrı’yı\hyp{}bilmeye götürür, yalnızca kutsal mevcudiyetin gizemli bir hissine değil. İnanç, duygusal sonuçlarından haddinden fazla etkilenmemelidir. Gerçek din, inanma ve bilmenin bir deneyimine ek olarak hissel bir tatmindir.
\vs p103 9:12 Dini deneyim içerisinde ruhsal içeriğiyle orantılı olan bir gerçeklik bulunmaktadır; ve, bu türden bir gerçeklik nedensellik, bilim, felsefe, bilgelik ve tüm diğer insan kazanımlarının ötesindedir. Bu türden bir deneyimin yargıları sorgulanamaz niteliktedir; dini yaşamın mantığı, yadsınamaz gerçekliktedir; bu türden bilginin kesinliği insan\hyp{}ötesindedir; onun tatminleri olağanüstü derecede kutsal, cesaretleri yenilmez, bağlılıkları sorgulanmaz, sadakatleri yüce ve nihai sonları --- ebedi, nihai ve kâinatsal olarak --- kesindir.
\vs p103 9:13 [Nebadon’un bir Melçizedek unsuru tarafından sunulmuştur.]
