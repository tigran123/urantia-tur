\upaper{8}{Sınırsız Ruhaniyet}
\vs p008 0:1 Ebediyete dönüldüğünde, Kâinatın Yaratıcısı’nın “ilk” sınırsız ve mutlak düşüncesi Ebedi Evlat’ta onun kutsal nitelendirmesi için bir tür kusursuz ve yeterli kelime bulur, orada bu karşılığın bulunuşundan sonra Düşünce\hyp{}Tanrısı ve Sözün\hyp{}Tanrısı’nın yüce istenci karşılıklı bir tanımlama ve birleşik eylemin evrensel ve sınırsız bir görevlisi için onun arkasından gelir.
\vs p008 0:2 Ebediyetin ilk ortaya çıkışında, Yaratıcı ve Evlat karşılıklı olarak birbirlerine bağlı olduklarının ve ebedi ve mutlak tek oluşlarının sınırsız bir biçimde farkına varırlar; ve bu sebeple kutsal birlikteliğin sonsuza kadar sürecek ve sınırsız olacak olan anlaşmasına varırlar. Bu hiçbir zaman sona ermeyecek olan sözleşme ebediyetin tüm döngüleri boyunca bütünleşmiş kavramsallaşmalarının uygulanması için yapılır, ve bu ebediyet hadisesinden itibaren Yaratıcı ve Evlat kutsal birleşmelerini sürdürür.
\vs p008 0:3 Bu bakımdan biz, İlahiyatın Üçüncül Bireyi olan Sınırsız Ruhaniyet’in ebediyet kökeniyle karşı karşıya geliriz. Yaratıcı olan Tanrı ve Evlat olan Tanrı’nın mutlak bir düşünce\hyp{}tasarısını uygulamasından doğan özdeş ve sınırsız bir eylemini birliktelikle kabul ettikleri bu esnada, Sınırsız Ruhaniyet kesin bir biçiminde eksiksiz olarak var olur.
\vs p008 0:4 Bu nedenle, İlahiyatlar’ın kökeninin bahsi geçen düzeninde ben sadece sizi onların bu ilişkisini düşünmeye sevk ediyorum. Gerçekte onlar üçü olarak ebediyetten gelmektedir ve onlar varoluş halindedir. Onlar günlerin başlangıcından veya bitişinden bağımsız; ve onlar eş güdüm halinde, yüce, nihai, mutlak ve sınırsızdır. Onlar her zaman bu niteliklerde şimdiki gibi olup, gelecekte de böyle olmayı süreceklerdir. Bununla birlikte, bu üç farklı biçimde bireyleşmiş ama ebedi bir biçimde birlikte olan bireyler; Yaratıcı olan Tanrı, Evlat olan Tanrı ve Ruhaniyet olan Tanrı’dır.
\usection{1.\bibnobreakspace Eylem olan Tanrı}
\vs p008 1:1 Geçmişin ebediyetinde, Sınırsız Ruhaniyet’in tüm kişileştirmeleri üzerinde kutsal kişilik çevrimi kusursuz ve tamamlanmış bir hale gelir. Eylemin Tanrı’sı mevcuttur, ve mekânın çok büyük bir kısmı ebedi çağların kutsal farklılaşan görünü olan evrensel serüven biçimindeki yaratılmışın fazlasıyla göz alıcı olan olaylarına sahne olması için oluşturulmuştur.
\vs p008 1:2 Sınırsız Ruhaniyet’in ilk eylemi onun kutsal ebeveynleri olan Yaratıcı\hyp{}Baba ve Ana\hyp{}Evlat’ın farkına varması ve onları incelemesidir. O Ruhaniyet olarak koşulsuz bir biçimde ikisini de kimlikleri bakımından tespit eder. Onların farklı olan kişiliklerinin, sınırsız özelliklerinin ve onların bütünleşen doğaları ve birleşen faaliyetlerinin tamamen bilincine varır. Öte yandan, gönüllü olarak, aşkın istençlilikle ve ilham verici kendiliğinden hareketle, ama İlk ve İkincil Bireyler’le eşit olan bir konumda, Üçüncül İlahiyat Bireyi Yaratıcı olan Tanrı’ya ebedi bir sadakatle bağlı bulunacağının sözünü verir ve Evlat olan Tanrı’ya sonsuza kadar sürecek olan bağlılığını kabul eder.
\vs p008 1:3 Bu karşılıklı iletişimin doğasında ve uygulayıcı birliğin üç biriminden her birinin kişilik bağımsızlığının karşılıklı tanınmasında, ebediyetin çevrimi kurulmuş olur. Cennet Kutsal Üçlemesi böylelikle mevcut hale gelir. Evrensel mekânın sahnesi, Ebedi Evlat’ın kişiliği boyunca ve Yaratıcı\hyp{}Evlat yaratan birlikteliğinin gerçeklik dışavurumları için yönetici bünye biçimindeki Eylem olan Tanrı’nın idaresi tarafından Kâinatın Yaratıcısı’nın sürekli kendisini açığa çıkaran niyetinin farklılaşan ve hiçbir zaman sona ermeyen uygulamaları için kurulmuş olur.
\vs p008 1:4 Eylem olan Tanrı faaliyetlerde bulunur ve mekânın sınırlı olan ve onları bir arada tutan çatıları hareket halindedir. Bir milyar kusursuz âlem böylece varlığın aydınlığına kavuşur. Bu varsayılan ebediyet anının öncesinde, Cennet’in doğasında bulunan mekân\hyp{}enerjileri varoluş içindedir ve potansiyel olarak faaliyete hazırdır; fakat onlar ne varlığın gerçekliğine sahiptir ne de fiziksel çekimleri maddi gerçekliklerin durmaksızın devam eden çekişine verilen tepki dışında ölçülebilir. Bu varsayılan, ebedi bir biçimde uzak olan anda maddi bir evren bulunmaz; fakat yine bu an bir milyar dünyanın gerçekleşmesine sebep olur, bu durumun kanıtı Cennet’in sonsuza kadar süren algısında onları bir arada da tutabilecek yeterli ve yetkin çekimin varlığıdır.
\vs p008 1:5 Bunun sonucunda, Evlatlar’ın yaratımı boyunca enerjinin ikinci biçimi açığa çıkar, ve bu ortaya çıkan ruhaniyet Ebedi Evlat’ın ruhsal çekimi tarafından aynı esnada algılanır. Bu nedenle, çift katmanlı çekimle yüzleşen evren sınırsızlığın enerjisiyle etkileşime girer ve kutsallığın ruhaniyetinde bir araya gelir. Bununla birlikte yaşam toprağı, Sınırsız Ruhaniyet’in birliktelikte bulunduğu akli döngülerde dışa vurulmuş aklın bilinci için hazırlanmış olur.
\vs p008 1:6 Olanaklı varoluşun bu tohumları üzerinde Tanrılar’ın merkezi yaratımı boyunca yayılmasıyla, Yaratıcı faaliyette bulunur ve yaratılmış kişilik açığa çıkar. Bunun ardından Cennet İlahiyatları’nın mevcudiyeti tüm düzenlenmiş mekânı doldurur, bununla birlikte her şeyi ve Cennet huzuru varlıklarını etkin bir biçimde yanına çekmeye başlar.
\vs p008 1:7 Sınırsız Ruhaniyet Havona dünyalarının doğuşuyla aynı anda olan bir biçimde ebedileşir, bu merkezi evren onun tarafından, onunla ve onun içinde Yaratıcı ve Evlat’ın bütünleşmiş iradelerine ve birliktelik kurmuş kavramlarına bağlı kalınarak yaratılmıştır. Üçüncül Birey bu birliktelikle oluşmuş yaratımın eylemini ilahileştirir ve böylelikle kendisi sonsuza kadar Bütünleştirici Yaratan haline gelir.
\vs p008 1:8 Yaratıcı, Evlat ve onların birlikteliğinin yardımcı ve ayrıcalıklı idarecisi olan Üçüncül Kaynak ve Merkez tarafından ve onların içinde oluşan bu anlar yaratılmışın gelişiminin bu büyük ve merak uyandırıcı anlarıdır. Bu heyecan uyandırıcı anların hiçbir kaydı yoktur. Biz bu konuda sadece Sınırsız Ruhaniyet’in sınırlı dışavurumlarına bu üstün etkileşimi ete kemiğe büründürmek için sahibiz. Buna ek olarak kendisi sadece merkezi evren ve orada onunla alakalı onun kişiliğiyle ve bilinç varlığıyla aynı esnada ebedileşen her şeyin varlığını sadece doğrular.
\vs p008 1:9 Kısacası, Sınırsız Ruhaniyet kendisi ebedi olduğu için merkezi evrenin aynı zamanda ebedi olduğunu doğrular. Ve bununla beraber bu durumun kendisi kâinat âlemlerinin tümünün tarihinin geleneksel olarak başladığı yerdir. Mutlak olarak; her şeyin merkezinde var olan, oldukça seçkin biçimde faaliyette bulunan, çok geniş evreni belirginleştiren idari bilgelik ve yaratıcı enerjinin bu görkemli açığa çıkışına öncül olan herhangi bir olay veya etkileşim hakkında hiçbir kayıt yoktur ve buna dair hiçbir bilgi bulunmamaktadır. Bu durumun ötesinde bu olay, mutlak gizem olan sınırsızlığın derinlerinde ve ebediyetin araştırılamaz etkileşimlerinde yatmaktadır.
\vs p008 1:10 Ve bu sebeple Üçüncül Kaynak ve Merkez’in ardıl kökenini fani yaratılmışların zaman tarafından bağlı ve mekân tarafından şartlanmış akıllarına irdeleyici bir lütuf olarak tasvir ediyoruz. İnsan aklı evren tarihini gözlerinin önüne getirmesi için bir başlangıç noktasına ihtiyaç duyar, ve ben ebediyetin tarihsel kavramına bu yakınlaşma biçimini size tedarik etmek için yönlendirildim. Fani akılda, tutarlılık İlk Sebep’e ihtiyaç duyar; bu sebeple Kâinatın Yaratıcısı’nı tüm yaratılmışların İlk Kaynak ver Mutlak Merkez’i olarak temel alıyoruz. Aynı zamanda, biz tüm yaratılmışların akıllarına Evlat ve Ruhaniyet’i, evren tarihinin bütün fazlarında ve yaratılma faaliyetinin tüm âlemlerinde Yaratıcı’yla birlikte eş ebedi olduğunu öğretiyoruz. Bunu yaparken biz; İlahi, Koşulsuz, Kâinatsal Mutlak’ın ve Cennet Adası’nın ebediyeti ve gerçekliği hakkında hiçbir biçimde saygısızlıkta bulunmuyoruz.
\vs p008 1:11 Zamanın çocuklarının maddi aklının ebediyet içindeki Yaratıcı’yı algılamaları onların yeteri kadar erişebilecekleri bir noktadadır. Biz herhangi bir evladının gerçekle kendisini, ilk olarak çocuk\hyp{}ebeveyn ilişkilerini anlamasıyla ve daha sonra bu kavramın genişleyerek bir bütünlük biçiminde aileyle kucaklaşması tarafından gerçekle en iyi bir biçimde bağdaştırabileceğinin bilincindeyiz. Bunun hemen ardından, evladın gelişen aklı aile ilişkilerinin kavramsallaşmasını toplum, ırk, dünya ve daha sonra evren, üstün evren ve hatta kâinatın âlemlerinin tümüne uygun bir biçimde aktarmaya ve onları bunun doğrultusunda düzenlemeye yetkin hale gelecektir.
\usection{2.\bibnobreakspace Sınırsız Ruhaniyet’in Doğası}
\vs p008 2:1 Bütünleştirici Yaratan ebediyetten gelmiştir ve tamamiyle hiçbir koşullamadan bağımsız olarak Kâinatın Yaratıcısı ve Ebedi Evlat’la birlikte tek’tir. Sınırsız Ruhaniyet kusursuzlukla sadece Cennet Yaratıcısı’nın doğasını dışa vurmaz, aynı zamanda Özgün Evlat’ın doğasını da yansıtır.
\vs p008 2:2 Üçüncül Kaynak ve Merkez birçok başlık altında bilinir. Evren Ruhaniyeti, Yüce Rehber, Bütünleştirici Yaratan, Kutsal Uygulayıcı, Sınırsız Akıl, Ruhaniyetlerin Ruhu, Cennet Ana Ruhaniyeti, Bütünleştirici Bünye, Nihai Yardımcı, Her Zaman Her Yerde Bulunan Ruhaniyet, Mutlak Akıl, ve Kutsal Eylem bunlardan birkaçıdır. Bununla birlikte kendisi zaman zaman Urantia üzerinde kâinat aklıyla karıştırılır.
\vs p008 2:3 İlahiyatın Üçüncül Bireyi’ni Sınırsız Ruhaniyet olarak atfetmek tamamiyle uygundur, çünkü Tanrı ruhaniyettir. Fakat maddi yaratılmışlar maddeyi sadece yavan gerçeklik ve akıl olarak değerlendirme hatasına düşmeye eğilimlidirler. Eğer kendisi Sınırsız Gerçeklik, Kâinatsal Düzenleyici veya Kişilik Yardımcısı olarak çağrılırsa, maddenin kökeninde hüküm sürdüğü biçimde maddeyi ruhaniyet ile birlikte görmek Üçüncül Kaynak ve Merkez’in daha iyi bir şekilde kavranmasının önünü açacaktır.
\vs p008 2:4 Sınırsız Ruhaniyet, kutsallığın bir evren açığa çıkarılışı olarak araştırılamaz olup bütünüyle insan kavramsallığının dışındadır. Ruhaniyet’in mutlaklığını hissetmek için sadece Kâinatın Yaratıcısı’nın sınırsızlığı üzerinde düşünmeniz ve Özgün Evlat’ın ebediyetinin merak dolu saygısının huzurunda olmanız gerekir.
\vs p008 2:5 Sınırsızlığın Ruhaniyeti’nin kişiliğinde gerçekten bir gizem bulunur, fakat bu gizem Evlat ve Yaratıcı’da olan gizem kadar fazla değildir. Yaratıcı’nın doğasının tüm yönlerinde Bütünleştirici Yaratan kendi sınırsızlığını en etkili bir biçimde ortaya çıkarır. Üstün evren nihai olarak sınırsızlığa genişlese de; Bütünleştirici Bünye’nin ruhani mevcudiyeti, enerji düzenlemesi ve akli potansiyeli bu tür sınırı olmayan bir yaratılmışın ihtiyaçlarına cevap vermek için yeterli bulunacaktır.
\vs p008 2:6 Kâinatın Yaratıcısı’nın sevgisinin, doğruluğunun ve kusursuzluğunun her türlü paylaşımında Sınırsız Ruhaniyet Ebedi Evlat’ın bağışlama özelliklerine benzer bir biçimde yakınlaşsa da ve bunun sonucunda Cennet İlahiyatları’nın muhteşem kâinata olan bağışlama hizmeti haline gelse de, kutsal Evlatlar Tanrı’nın sevgisini açığa çıkardıkları için ve kutsal Ruhaniyet Tanrı’nın bağışlamasını temsil ettiği için başından beri evrensel ve ebedi olarak her zaman Ruhaniyet bir bağışlama hizmetkârıdır.
\vs p008 2:7 Ruhaniyet’in Tanrı’nın iyiliğinden daha fazlasına sahip olması mümkün değildir, çünkü iyiliğin tümü kökenini Tanrı’dan alır. Fakat Ruhaniyet’in eylemleri içerisinde böyle bir iyiliği biz daha iyi kavrayabiliriz. Yaratıcı’nın inançlılığı ve Evlat’ın sürekliliği âlemlerin maddi yaratılmışları ve ruhani varlıklarına Sınırsız Ruhaniyet’in kişiliklerinin sonu gelmez hizmetiyle ve sevgi dolu yardımıyla sahip olabilecekleri gerçeklik haline getirildi.
\vs p008 2:8 Bütünleştirici Yaratan, Yaratıcı’nın sahip olduğu gerçeğin güzelliğinin ve karakterinin tümünü ondan alır. Bununla beraber, kutsallığın tüm bu ulvi özellikleri Üçüncül Kaynak ve Merkez’in sınırı ve koşulu olmayan aklının ebedi ve sınırsız bilgeliği altında ona bağlı olarak kâinat aklının üstünlüğe yakın düzeylerinde eş güdüm halindedir.
\usection{3.\bibnobreakspace Ruhaniyet’in Yaratıcı ve Evlat’la olan İlişkisi}
\vs p008 3:1 Ebedi Evlat’ın Kâinatın Yaratıcısı’nın “ilk” mutlak ve sınırsız düşüncesinin kelimesel olarak dışavurumu olmasından dolayı, böylelikle mutlak düşünce\hyp{}kelime birliğinin Yaratıcı\hyp{}Evlat kişilik ilişkisi tarafından bütünlükçü bir faaliyet için “başat” olarak tamamlanmış yaratıcı kavramsallaşması veya tasarısının kusursuz uygulayıcısı Bütünleştirici Bünye’dir. Üçüncül Kaynak ve Merkez merkezi veya emredilen yaratımla birlikte aynı anda ebedileşir, ve sadece bu merkezi yaratım âlemler arasında mevcut bir biçimde ebedidir.
\vs p008 3:2 Üçüncül Kaynak’ın kişilikleştirilmesinden itibaren, İlk Kaynak artık bireysel olarak evren yaratımına katılmaz. Kâinatın Yaratımı bu bağlamda tüm görevlerini onun Ebedi Evlat’ına devreder; benzer bir biçimde Ebedi Evlat tüm olası gücü ve otoriteyi Bütünleştirici Yaratan’a bahşeder.
\vs p008 3:3 Ebedi Evlat ve Bütünleştirici Yaratan birliğin üyeleri olarak ve onların yardımcı kişilikleri vasıtasıyla varlığa kavuşturulan her Havona sonrası evreni tasarladı ve ona arzuladığı şekli verdi. Ruhaniyet, Evlat’ın Yaratıcı’yla ilk ve merkezi yaratımda elinde bulundurduğu ilişkinin aynısını Ruhaniyet tüm ardıl yaratımda Evlat’a karşı bulundurur.
\vs p008 3:4 Ebedi Evlat’ın bir Yaratan Evlat’ı ve Sınırsız Ruhaniyet’in bir Yaratıcı Evlat’ı sizi ve sizin bulunduğunuz evreni yarattı; ve Yaratıcı inançlılıkla onların oluşturduklarını korurken bu oluşum bu Evren Evladı’na ve bu Evren Ruhaniyeti’ne çalışmalarını desteklemek ve onları sürdürmek için, aynı zamanda onların kendi yaratım süreçlerinin yaratılmışlarına hizmet için devredildi.
\vs p008 3:5 Sınırsız Ruhaniyet, bütünüyle sevgi dolu Yaratıcı ve tamamiyle bağışlama dolu Evlat’ın zaman ve mekân üzerindeki tüm dünyalarda, gerçek\hyp{}aşkı içerisinde yanıp tutuşan tüm ruhları kendilerine doğru çekmesinin ortak tasarılarını uygulamak için kurdukları etkin kurumdur. Ebedi Evlat onun Yaratıcı’sının âlemlerin yaratılmışlarının kusursuzluğa ulaşma tasarısını kabul ettiği bu andan itibaren, yükselişin projesi bir Yaratıcı\hyp{}Evlat tasarısına dönüştü, ve aynı anda Sınırsız Ruhaniyet, Yaratıcı ve Evlat’ın birleşik ve ebedi olan niyetinin bütünleşmiş idarecisi haline geldi. Sınırsız Ruhaniyet bu görevi yerine getirerek tüm kaynaklarının kutsal mevcudiyetini ve ruhaniyet kişiliklerini Yaratıcı ve Evlat’a sunacağının teminatını verdi; ve bunun sonucunda kendisini yüceltilmiş varlığını sürdürmeye çalışan irade sahibi yaratılmışlarının göz kamaştırıcı tasarısını \bibemph{tamamiyle} Cennet kusursuzluğunun kutsal doruklarına adadı.
\vs p008 3:6 Sınırsız Ruhaniyet; Kâinatın Yaratıcısı ve onun Ebedi Evlat’ının tamamlanmışlığı, onların ayrıcalıklı doğası, ve onların kâinatsal açığa çıkarılışıdır. Yaratıcı\hyp{}Evlat birlikteliğinin tüm bilgisi kutsal düşünce\hyp{}söz birliğinin birleşik tanıtımcısı olarak Sınırsız Ruhaniyet vasıtasıyla sahip olunabilir.
\vs p008 3:7 Ebedi Evlat, Kâinatın Yaratıcısı’na erişimin ona bağlanan tek çıkar yoludur; ve Sınırsız Ruhaniyet, Ebedi Evlat’a erişimin tek vasıtasıdır. Sadece Ruhaniyet’in sabır dolu hizmetiyle zamanın yükselen varlıkları Evlat’ı keşfedebilir.
\vs p008 3:8 Kâinata dair her şeyin merkezinde, Sınırsız Ruhaniyet yükselen kutsal yolcular tarafından ulaşılabilecek Cennet İlahiyatları’nın ilkidir. Üçüncül Birey, İkincil ve Birincil Bireyleri kendi görüntüsüyle kaplayarak onları dışarıdan görünmez hale getirir, bu sebeple Evlat ve Yaratıcı’nın tanıtımını sağlayacak tüm adaylar tarafından her zaman ilk sırada kendisi ayırt edilmelidir.
\vs p008 3:9 Buna ek olarak, diğer birçok biçimde Ruhaniyet eşit bir şekilde Yaratıcı ve Evlat’ı tanıtır ve buna benzer olarak ona hizmet eder.
\usection{4.\bibnobreakspace Kutsal Hizmetin Ruhaniyeti}
\vs p008 4:1 Fiziksel evren içinde Cennet çekiminin her şeyi beraberinde tutmasına benzer bir biçimde, içinde Evlat’ın sözünün Tanrı düşüncesini ve “ete kemiğe büründürme zamanının” anlaşılacak bir biçimde aktardığı ruhsal evren, yardımcı Yaratanlar’ın bütünleşmiş doğasının sevgi dolu bağışlamasını gösterir. Fakat bu maddi ve ruhani yaratılmışlığın tümü içinde ve onun boyunca, Sınırsız Ruhaniyet ve onun ruhani doğumunun; onların eş güdüm halinde tasarımlarının ve yapımlarının akli yapıya sahip çocuklarına doğru kutsal ebeveynlerin bağışlama, sabır ve sonu gelmez şefkatinin hepsinin birleşmiş halini üzerinde gösterdiği bir büyük bir alan vardır. Akla yapılan sonsuza kadar sürecek olan bu hizmet Ruhaniyet’in kutsal karakterinin özüdür. Bununla beraber, Bütünleştirici Bünyenin ruhaniyet doğumu, hizmette bulunmak için olan bu kutsal istenç biçimindeki yardım etme arzusunun içinde yar alır.
\vs p008 4:2 Tanrı sevgi, Evlat bağışlama, Ruhaniyet ise tüm akli yapıya sahip yaratılmışların kutsal sevgisinin ve bitmek bilmeyen bağışlama yardımı olan hizmettir. Ruhaniyet, Yaratıcı’nın sevgisi ve Evlat’ın bağışlamasının birey haline gelişidir; onun içinde evrensel hizmet için tüm bunlar ebedi bir biçimde bütünleşir. Ruhaniyet yaratılmışın yaratılması üzerine \bibemph{sevginin gösterilişidir}, ve bu sevgi Yaratıcı ve Evlat’ın bir araya gelen sevgisidir.
\vs p008 4:3 Urantia üzerinde Sınırsız Ruhaniyet, kâinatsal bir mevcudiyet olarak onun her zaman her yerde birden bulunuşu olarak bilinir; fakat Havona üzerinde onu, asli hizmetin kişisel mevcudiyeti olarak bilmelisiniz. Burada Cennet Ruhaniyeti’nin hizmeti, zaman ve mekânın dünyaları üzerinde onun eş güdüm halinde bulunan Ruhaniyetler’inin ve yaratılmış varlıklara hizmet ederek ona bağlı olarak görev yapan kişiliklerin her biri için örnek alınacak ve ilham verici biçimidir. Bu kutsal evrende Sınırsız Ruhaniyet, Ebedi Evlat’ın yedi aşkın görünüşüne katılır; buna benzer bir biçimde benzersiz Mikâil Evladı’yla birlikte Havona’nın döngülerine yapılan yedi bahşedilmişliğe katılır, ve bunun sonucunda ruhaniyet hizmetinin bu kusursuz döngülerin zirvesinde seyahat halinde olan zamanın her kutsal yolcusuna duygudaşlığın ve anlayışın ruhaniyet hizmeti haline gelir.
\vs p008 4:4 Tanrı’nın bir Yaratan Evlat’ı gösterilen bir yerel evrende yaratma görevinin sorumluluğunu kabul ettiği zaman, Sınırsız Ruhaniyet’in kişilikleri Mikâil Evladı’nın yaratım serüveni görevinin peşinden gitmesinin yorulmak bilmeyen hizmetlerinde olduğu gibi sorumluluklarını yerine getireceklerine dair teminat verirler. Özellikle yerel evren Ana Ruhaniyetleri olan Yaratıcı Kız Evlatları’nın kişiliklerinde, ruhsal erişimin gittikçe yükselen seviyelerine maddi yaratılmışların yükselişini destelemek görevine Sınırsız Ruhaniyet’in kendisini adayışını buluruz. Buna ek olarak yaratım hizmetinin bu görevi, bahsi geçen niyetlerle kusursuz bir uyumda ve bu yerel evrenlerin Yaratan Evlatlar’ın kişilikleriyle yakın ilişkide yerine getirilir.
\vs p008 4:5 Tanrı'nın Evlatları’nın bir evrene Yaratıcı’nın kişiliğinin açığa çıkarılışının bu devasa görevine katılımına benzer bir biçimde, Sınırsız Ruhaniyet her evrenin tüm çocuklarının bireysel akıllarına Yaratıcı ve Evlat’ın birleşik sevgisini açığa çıkarışın bitmek tükenmek bilmeyen hizmetine kendisini adamıştır. Bu yerel yaratılmışlarda Ruhaniyet, Tanrı'nın Evlatları’nın kesin olarak yaptığı gibi fani bedene benzer bir yapı içerisinde maddi ırklara gelmezler, fakat Sınırsız Ruhaniyet ve onun eş güdüm halindeki Ruhaniyetler’i bulundukları göksel mekândan aşağıya inerler, ve sizin yanınızda duran melekler olarak görünene ve dünyevi varoluşun düşük seviyedeki yolları için size rehberlikte bulununcaya kadar memnuniyet içinde mükemmel bir dizi kutsallığından feragat etmenin deneyimini yaşarlar.
\vs p008 4:6 Sınırsız Ruhaniyet’in bu kutsallığının azalan dizisi sayesinde gerçekte bir birey olarak hayvan kökenli âlemlerin her varlığının yakınına gelebilir. Ve Ruhaniyet, tüm bunları her şeyin merkezinde olan İlahiyatın Üçüncül Birey varoluşu olarak kendisinin mevcudiyet değerini kesinlikle dışlamadan gerçekleştirir.
\vs p008 4:7 Bütünleştirici Yaratan, gerçek anlamıyla ve sonsuza kadar evrensel bağışlama hizmetkârı olarak hizmette bulunan muhteşem kişiliktir. Ruhaniyet’in hizmetini kavramak için, kendisinin Yaratıcı’nın bitmek tükenmek bilmeyen sevgisinin ve Evlat’ın ebedi bağışlamasının birleşik tasviri olduğu gerçeğini enine boyuna düşünün. Buna rağmen Ruhaniyet’in hizmeti Ebedi Evlat ve Kâinatın Yaratıcısı’nın sadece tanıtılmasıyla sınırlı değildir. Sınırsız Ruhaniyet aynı zamanda kendi ismi altında ve kendisinin hakkı olarak âlem yaratılmışlarına hizmet etme gücünü elinde bulundurur; Üçüncül Birey kutsal saygınlığın bir parçasıdır ve kendi adına evrensel hizmetin bağışlamasını bahşeder.
\vs p008 4:8 İnsan bu Sınırsız Ruhaniyet’in yaratılmış ailesinin daha düşük düzeyde bulunan sevgi dolu ve yorulmak nedir bilmeyen hizmetini daha fazla öğrendiğinde; Ebedi Evlat ve Kâinatın Yaratıcısı’nın bu bütünleşen Eylemi’nin benzersiz karakterini ve aşkın doğasını daha fazla bir biçimde takdir edecek ve onlara her zamankinden daha fazla hayranlık besleyecektir. Gerçekten, Ruhaniyet “Koruyucu’nun her zaman doğrunun üzerinde olan gözleri”dir ve “onların dualarına en başından beri açık olan kutsal kulaklar”ıdır.
\usection{5.\bibnobreakspace Tanrı'nın Mevcudiyeti}
\vs p008 5:1 Sınırsız Ruhaniyet’in sıra dışı özelliği onun her zaman her yerde aynı anda bulunuşudur. Kâinat âlemlerinin tümü boyunca her yerde mevcut olan kutsal ve evrensel bir aklın varlığına oldukça benzer biçimde her yeri saran ruhaniyet bulunmaktadır. İlahiyat’ın İkincil ve Üçüncül bireyleri onların başından beri mevcut olan ruhaniyetleri tarafından tüm dünyalarda temsil edilirler.
\vs p008 5:2 Yaratıcı \bibemph{sınırsızdır} ve bu sebepten dolayı sadece kendi iradesi tarafından sınırlıdır. Düzenleyiciler’in bahşedilişinde ve kişiliğin kendi içerisindeki döngüsünde Yaratıcı tek başına hareket eder, fakat akli yapı varlıklarıyla olan ruhaniyet güçlerinin ilişkisinde Sınırsız Ruhaniyet ve Ebedi Evlat’ın kişiliklerini ve ruhaniyetlerini kullanır. Kendisi sahip olduğu iradesinin tercihiyle ruhani olarak Evlat’la veya Bütünleştirici Bünye ile eşit bir biçimde varoluş içindedir; çünkü kendisi Evlat’la \bibemph{birlikte} ve Ruhaniyet’in \bibemph{içerisinde} var olur. Yaratıcı kesin bir biçimde her yerde mevcuttur, ve biz onun mevcudiyetini, tüm bu farklı görünen fakat özünde birleşik olan güçlerin, etkilerin ve varoluşlarının herhangi birisi veya hepsi tarafından ve onların vasıtasıyla algılayabiliriz.
\vs p008 5:3 Sizin kutsal yazılarınızda \bibemph{Tanrı'nın Ruhaniyeti} kavramı değişken bir biçimde Cennet üzerindeki Sınırsız Ruhaniyet ve yerel evreninizin Yaratıcı Ruhaniyet’ini tanımlamak için kullanılmış gözlenmektedir. Kutsal Ruhaniyet, Cennet Sınırsız Ruhaniyeti’nin bu Yaratıcı Kız Evlatları’nın döngüsüdür. Kutsal Ruhaniyet her yerel evren için onların yerel niteliklerine uyumlu olan bir döngüdür ve bu yaratımın ruhsal âlemiyle sınırlıdır. Fakat şu unutulmamalıdır ki Sınırsız Ruhaniyet her zaman her yerde aynı anda bulunur.
\vs p008 5:4 Birçok ruhsal etki bulunmaktadır, ve bunların tümü aslında bir \bibemph{tek’tir}. Düşünce Düzenleyicileri’nin çalışması bile, tüm diğer etkilerden bağımsız olmasına rağmen, değişmeyen bir biçimde bir yerel evren Ana Ruhaniyet’in ve Sınırsız Ruhaniyet’in birleşik etkilerinin ruhaniyet hizmetiyle örtüşür. Bu ruhsal mevcudiyetler Urantia sakinlerinin yaşamlarında işlevini yerine getirirlerken bu süreç içerisinde biri diğerinden ayrı tutulamaz. Sizin akıllarınızda ve ruhlarınız üzerinde farklı olan kökenlerine rağmen onlar bir ruhaniyet olarak faaliyette bulunurlar. Buna ek olarak, bu birleşik ruhsal hizmet deneyimlendiğinde, bu hizmet “başından beri sizi hataya düşmekten koruyan ve yüksekte bulunan Yaratıcı’nızdan önce sizlere masumiyeti sunan bünye” olarak Yücelik’in etkisi haline gelir.
\vs p008 5:5 Şunu unutmayınız ki: Sınırsız Ruhaniyet \bibemph{Bütünleştirici} Bünye’dir; Yaratıcı ve Evlat beraberce onun içinde ve onun vasıtasıyla faaliyetlerini gerçekleştirir; o sadece kendisi olarak varoluş içerisinde değildir, fakat aynı zamanda Yaratıcı, Evlat ve Yaratıcı\hyp{}Evlat olarak mevcuttur. Bunun farkındalığından ve birçok buna ilave sebeplerden dolayı Sınırsız Ruhaniyet’in ruhaniyet mevcudiyeti “Tanrı’nın ruhaniyeti” olarak sıklıkla atıfta bulunulur.
\vs p008 5:6 Tüm ruhsal hizmetin ilişkisini Tanrı’nın ruhaniyeti olarak adlandırmak aynı zamanda tutarlı olacaktır, böyle bir bağlantısal iletişim Yaratıcı olan Tanrı, Evlat olan Tanrı, Ruhaniyet olan Tanrı ve hatta Yücelik olan Tanrı’nın ruhaniyeti olarak Yedi Katmanlı olan Tanrı’nın ruhaniyetlerinin tam anlamıyla birliğidir.
\usection{6.\bibnobreakspace Sınırsız Ruhaniyet’in Kişiliği}
\vs p008 6:1 Üçüncül Kaynak ve Merkez’in uçsuz bucaksız dağıtımının ve çok geniş bir alana yayılan bahşedişinin onun kişiliğinin gerçeğinin apaçıklığını bulandırmasına veya onun değerini azaltmasına izin vermeyin. Sınırsız Ruhaniyet; bir evren varoluşu, bir ebedi eylem, bir kâinatsal güç, bir kutsal etki ve bir evren aklıdır. Kendisi tüm bunların hepsi ve onların sınırsız bir biçimde daha fazlasıdır, fakat tüm bu niteliklerinin yanı sıra o aynı zamanda gerçek ve kutsal bir kişiliktir.
\vs p008 6:2 Sınırsız Ruhaniyet tamamlanmış ve kusursuz bir kişiliktir, Kâinatın Yaratıcısı ve Ebedi Evlat’ın kutsal eşiti ve yardımcısıdır. Bütünleştirici Yaratan, tıpkı Yaratıcı ve Evlat gibi, âlemlerin daha yüksek olan akli yapılarına benzer bir biçimde gerçek ve gözle görülebilir olarak gelmektedir; gerçekte bunlardan daha fazlası olarak kendisi, tüm yükselenlerin Evlat vasıtasıyla Yaratıcı’ya yaklaşmadan önce erişmek durumunda oldukları bünyedir.
\vs p008 6:3 İlahiyatın Üçüncül Bireyi olan Sınırsız Ruhaniyet sizin karakterle birliktelik kurduğunuz tüm özelliklerden oluşmuştur. Ruhaniyet’e mutlak akıl kazandırılmıştır: “Ruhaniyet her şeyi ve hatta Tanrı’nın en derinliklerini irdeler.” Ruhaniyet sadece akılla değil aynı zamanda iradeyle de donatılmıştır. Kendisinin bu hediyelerinin bahşedilişi hususunda, “Fakat tüm bunlar tek ve özdeş olan Ruhaniyet’inin bir eseridir, bunu her insana ayrı ayrı ve kendi iradesi uyarınca paylaştırır” söylemi biçiminde kayıt altına alınmıştır.
\vs p008 6:4 “Ruhaniyet sevgisi” ve aynı zamanda onların ıstırapları gerçektir; bu sebeple “Tanrı'nın Ruhaniyeti’nden üzüntü çekmeyin.” Sınırsız Ruhaniyet’i Cennet İlahiyatı veya yerel bir evren Yaratıcı Ruhaniyet’i olarak gözlemleyelim veya gözlemlemeyelim Bütünleştirici Yaratan sadece Üçüncül Kaynak ve Merkez değil, fakat aynı zamanda kutsal bir kişiliktir. Bu kutsal kişilik aynı zamanda evrene bir kişilik olarak karşılık verir. Ruhaniyet sizinle konuşur, tıpkı şu sözde olduğu gibi: “O bir kulağı Ruhaniyet’in ne söylediğini duyması için yarattı”. “Ruhaniyet kendisi olarak sizin için ricalarda bulunur.” Ruhaniyet yaratılmış varlıklar üzerinde doğrudan ve kişisel bir etkiyi kullanır, “Tanrı'nın Ruhaniyeti tarafından olabildiği kadar fazla bir biçimde yönlendirildiği için onlar Tanrı’nın evlatlarıdır.”
\vs p008 6:5 Sınırsız Ruhaniyet’in kâinat âlemlerinin tümünün uzak dünyalarına olan hizmetinin olgusallığını dikkatle değerlendirmemize rağmen, Üçüncül Kaynak ve Merkez içinde kökenini alan çok çeşitli varlıkların söylenmeyen birliğinin içinde ve onun vasıtasıyla hareket eden bu aynı yardımcı İlahiyat’ı tasavvur etmemize rağmen, Ruhaniyet’in her zaman her yerde aynı anda bulunuşunu tanımamıza rağmen, bu aynı Üçüncül Kaynak ve Merkez’in her âlemin, her varlığın ve her şeyin Bütünleştirici Yaratan’ı olarak bir kişilik olduğunu yine de hala kabul ederiz.
\vs p008 6:6 Âlemlerin yönetiminde Yaratıcı, Evlat ve Ruhaniyet kusursuz ve ebedi bir biçimde birbirleriyle etkileşimde bulunarak ilişki halindedir. Her biri tüm yaratılmışlara bireysel bir hizmetin içinde olsa bile, tüm bu üçü kutsal ve mutlak olarak onları her zaman\bibemph{ tek bir bütün} yapan yaratımın ve hizmetin bir servisinde kenetlenmiştir.
\vs p008 6:7 Ruhaniyet; Yaratıcı, Evlat ve tıpkı aynı zamanda onların ikisinin sonsuza kadar tek bir bütün olması gibi benzer bir biçimde var olduğundan dolayı, Sınırsız Ruhaniyet’in kişiliğinde Yaratıcı ve Evlat karşılıklı olarak her zaman koşulsuz kusursuzluğun içinde mevcuttur.
\vs p008 6:8 [Zamanın Ataları tarafından Sınırsız Ruhaniyet’in doğasını ve eserini tasvir etmek için Uversa’nın bir Kutsal Danışmanı tarafından Urantia üzerinde sunulmuştur.]
