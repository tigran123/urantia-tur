\upaper{84}{Evlilik ve Aile Yaşamı}
\vs p084 0:1 Maddesel gereklilik evliliği kurmuş, cinsel açlık onu süslemiş, din ona izin vermiş ve yüceltmiş, devam onu talep edip düzenlemişken; sonraki dönemlerde evrimleşen aşk, ev olarak, medeniyetin en yararlı ve ulvi kurumunun atası ve yaratıcısı olarak evliliği gerekçelendirme ve onu yüksek derecede övmeye başlamaktadır. Ve ev inşası, tüm eğitim çabalarının merkezi ve özü olmalıdır.
\vs p084 0:2 Çiftleşme tamamen, bireyin kendisini tatmininin değişen dereceleriyle beraber gerçekleşen bir bireysel çoğalım eylemidir; ev inşası olarak evlilik büyük bir ölçüde bireyin yaşamını kendi kendisine bir idame ediş durumu olup, toplumunun evrimi anlamına gelmektedir. Toplumun kendisi, aile birimlerinin bir araya gelmiş yapısıdır. Bireyler gezegensel etmenler olarak oldukça geçicidir --- sadece aileler toplumsal evrimde devamlılığı olan birimlerdir. Aile, bir nesilden diğerine kültür ve bilgi nehrinin boyunca aktığı kanaldır.
\vs p084 0:3 Ev temel olarak toplumsal bir kurumdur. Aile; bireyin kendisini tatmin ediş etkisinin büyük ölçüde yan etken olarak kaldığı biçimde, bireyin kendi yaşamını idame ettirişi ile işbirliği içindeki kendi çoğalımı birlikteliğinden doğmuştu. Yine de ev kurumu, insan mevcudiyetinin bu temel üç işlevinin hepsini de içine almaktadır; yaşamın devamlılığı evi temel insan kurumu yaparken, cinsel ilişki onu tüm diğer toplumsal etkinliklerden ayırmaktadır.
\usection{1.\bibnobreakspace İlkel Çift Birliktelikleri}
\vs p084 1:1 Evlilik cinsel ilişkiler üzerine kurulmamıştı; onlar evliliğin yan etkenleriydiler. Evlilik; kadın, çocuk ve evin sorumluluklarıyla kendi özgürlüğünü kısıtlamadan cinsel arzusunu bağımsız bir biçimde tatmin eden ilkel insan tarafından ihtiyaç duyulmamaktaydı.
\vs p084 1:2 Kadın, doğumuna karşı olan fiziksel ve duygusal bağımsızlığından dolayı, erkek ile işbirliğine ihtiyaç duymaktadır; ve bu durum onu, evliliğin barınak sağlayan korumasına itmektedir. Ancak, --- bırakın onu içinde tutmaya neden olsun --- hiçbir biyolojik dürtü kendi başına erkeği evliliğe sürüklememiştir. Evliliği erkek için cazibeli kılan aşk değildi; ancak ilk olarak yiyecek açlığı, kadını ve onun çocukları ile paylaşılan barınağı ilkel insan için çekici kılmıştı.
\vs p084 1:3 Evlilik, cinsel ilişkilerin sonucundan doğan sorumlulukların bilinçli bir biçimde yerine getirilişi ile bile gelmemişti. İlkel insan, cinsel duyguların tatmini ile onun sonrasında gelen bir çocuğun doğumu arasında hiçbir bağ görmemekteydi. Bakir bir kadının hamile olabileceğine herkes tarafından inanılmaktaydı. İlkel insan öncül bir biçimde, bebeklerin ruhani yerleşkede yapıldığına dair inanışa sahipti; hamileliğin, evrimleşen bir hayalet olarak bir ruhaniyetin kadına girmesi sonucu gerçekleştiğine inanılmaktaydı. Hem yenilen şeylerin hem de nazarın da, bakir veya evlenmemiş bir kadının hamileliğine neden olabileceğine inanılmıştı; bunun karşısında yaşamın başlangıcı ile ilgili daha sonraki inanışlar nefes ve güneş ışığıyla ilişkilendirilmişti.
\vs p084 1:4 Birçok öncül topluluklar, hayaletleri denizler ile ilişkilendirdi; bu nedenle bakir kadınlar, yıkanma faaliyetlerinde fazlasıyla kısıtlandırılmışlardı; genç kadınlar, cinsel ilişkiye girme karşısında denizin yükseldiği dönemlerde banyo yapmaktan çok daha fazla korku duymaktalardı. Sakat doğan veya vaktinden önce dünyaya gelen bebekler, dikkatsizce yapılan banyonun bir sonucu olarak veya kötü emelli ruhaniyet eylemi vasıtasıyla kadının bedenine bir şekilde girebilmiş genç hayvanlar olarak görülmekteydiler. İlkel insan toplulukları, tabii ki, bu türden bir doğumu dünyaya geldiği an boğmakta hiçbir sakınca görmedi.
\vs p084 1:5 Aydınlanmada ilk adım, cinsel ilişkilerin hamile bırakan hayaletin kadın bedenine girmesinin yolunu açtığına dair inanışla geldi. İnsan bu dönemden itibaren; babasının ve annesinin, doğumu başlatan yaşam kalıtımının eşit katılımcıları olduğunu keşfetmiştir. Ancak yirminci yüzyılda bile birçok ebeveyn hala, insan yaşamının kökeni hususunda çocuklarını neredeyse bütüncül bir cehalet içerisinde tutmak için çaba göstermektedirler.
\vs p084 1:6 Olağan türden bir aile, anne\hyp{}çocuk ilişkisine zemin hazırlayacak çoğalım faaliyeti gerçekliğiyle teminat altına alınmıştı. Anne sevgisi içgüdüseldir; evliliğin doğduğu gibi adetlerden çıkmamıştır. Her memelinin anne sevgisi yerel evrenin emir\hyp{}yardımcı akıl\hyp{}ruhaniyetlerinin içkin kazanımı olup, yoğunluk ve bağlılık bakımından türlerin yardıma muhtaç bebeklik süresiyle her zaman doğru orantılıdır.
\vs p084 1:7 Anne ve çocuk ilişkisi doğal, güçlü ve içgüdüseldir; ve bu nitelik, böylelikle, ilkel kadınları birçok tuhaf koşulları kabul etmeye ve anlatılmaz zorluklara dayanmaya itmiştir. Bu baskıcı anne sevgisi, erkek ile olan tüm mücadelelerinde kadınları her zaman devasa bir eşitsizlik konumuna düşüren sınırlayıcı duygudur. Böyle bir durumda bile insan türleri içindeki annesel içgüdü, her zaman üstün gelen bir duygu değildir; bu içgüdü geleceğe dair arzular, bencillik ve dinsel inanışlar tarafından engellenebilir.
\vs p084 1:8 Her ne kadar anne\hyp{}çocuk ilişkisi her ne kadar ne evlilik ne de ev kurumu olsa da, ikisinin de üzerinden filizlendiği çekirdek olmuştu. Çiftleşmenin evriminde büyük ilerleme, bu geçici birlikteliklerin sonuçsal olarak yarattığı doğumu yetiştirecek kadar uzun bir süre dayanmasıyla gelmişti; çünkü bu durum evi inşa eden gelişmeydi.
\vs p084 1:9 Bu öncül çiftlerin karşıtlıklarına rağmen, bu birlikteliğin gevşekliğine rağmen, hayatta kalma şansı bu erkek\hyp{}kadın birliktelikleriyle fazlasıyla artmıştı. İş birliğinde bulunan bir kadın ve erkek, aile ve çocuktan bir bağımsız olarak, iki erkek veya iki kadına kıyasla birçok açıdan çok daha fazla üstündür. Cinslerin bu şekilde çift hale gelişi yaşam mücadelesini geliştirmiş ve insan toplumunun ilk başlangıcını oluşturmuştur. Emeğin cinslere göre bölünüşü aynı zamanda rahatlığı sağlayıp, mutluluğu arttırmıştır.
\usection{2.\bibnobreakspace Öncül Anne\hyp{}Ailesi}
\vs p084 2:1 Kadının deneyimlediği dönemsel kan çıkışı ve onun doğumdaki ilave kan kaybı öncül bir biçimde, çocuğu dünyaya getirenin (ruhun kökeninin bile) kan olduğuna işaret etmiş olup, insan ilişkilerinin kan bağına dayanan kavramsallaşmasına kaynaklık etti. Öncül dönemlerde soyun tamamı, kesin olarak belirlenebilen kalıtımın tek unsuru olarak, kadın kuşağı içinde tanınmaktaydı.
\vs p084 2:2 Anne ve çocuğun içgüdüsel kadın bağından türeyen ilkel aile, kaçınılmaz bir biçimde bir anne\hyp{}ailesiydi; ve birçok kabile bu düzene uzun bir süre bağlı kalmıştı. Anne\hyp{}ailesi, sürü içindeki topluluk ailesi düzeyinden çok eşli ve tekeşli baba\hyp{}ailesinin daha sonraki ve gelişmiş ev yaşamına olan tek olası geçiş aşamasıydı. Anne\hyp{}ailesi doğal ve biyolojikti; baba\hyp{}ailesi toplumsal, ekonomik ve siyasaldı. Kuzey Amerikalı kızıl insanlar arasındaki anne\hyp{}ailesinin mevcudiyeti, başka alanlarda ilerici olan Iroquois birlikteliğinin neden hiçbir zaman gerçek bir devlet olamamasının başlıca nedenlerinden bir tanesidir.
\vs p084 2:3 Anne\hyp{}ailesi adetleri uyarınca kadın eşin annesi, evde neredeyse en büyük yönetim gücünü memnuniyetle deneyimledi; kadın eşin erkek kardeşleri ve onların erkek çocukları bile kocaya kıyasla ailenin yüksek denetiminde daha etkinlerdi. Babaların isimleri sıklıkla gerçekleşen bir biçimde kendi çocuklarınınkilerle değiştirilirdi.
\vs p084 2:4 Öncül ırklar, çocuğu tamamen anneden gelen bir biçimde görerek, babaya çok az değer vermişi. Onlar çocukların, beraberliklerinin bir sonucu olarak babaya benzediğine; veya annenin çocukları babaya benzer bir biçimde görünmeleri için “işaretlediğine” inanmaktaydılar. Daha sonra, anne\hyp{}ailesinden baba\hyp{}ailesine olan dönüşüm gerçekleştiğinde, baba çocuk için tüm övgüyü aldı; ve hamile bir kadın üzerindeki tüm tabular daha sonra onun erkek eşini içine alan bir biçimde genişledi. Müstakbel baba, doğumun yaklaştığı zaman çalışmaya son verdi; ve doğum döneminde, üç gün ile sekiz gün arasında istirahatta kalan bir biçimde karısıyla birlikte yatağa yattı. Kadın ertesi gün kalkıp, ağır işlere girişebilirdi; ancak baba, tebrikleri kabul etmek için yatakta kalmaya devam etti; bu uygulama bütüncül olarak, babanın çocuk üzerindeki hakkını oluşturmak için tasarlanan öncül adetlerin bir parçasıydı.
\vs p084 2:5 İlk başta, erkeğin kadın eşinin topluluğuna gitmesi adetti; ancak daha sonraki dönemlerde, bir erkek gelin parasını verdiğinde veya bu maddi sorunu çözdüğünde kadın eşini ve çocuklarını kendi topluluğuna götürürdü. Anne\hyp{}ailesinden baba\hyp{}ailesine olan geçiş, eş kan bağından olanlar kabul edilirken kuzen evliliklerinin belirli türlerine dair manasız görünen nitelikteki diğer kısıtlamaları açıklamaktadır.
\vs p084 2:6 Avcılık adetlerinin geçmesiyle birlikte, sürüleri gütme erkeklere ana yiyecek arzının denetimini verdiğinde, anne\hyp{}ailesi hızlı bir sona yaklaştı. Bu aile yalın bir değişle, yeni baba\hyp{}ailesi ile başarılı bir biçimde rekabet etmede başarısız oldu. Annenin erkek akrabalarında toplanan güç, baba olan erkek eşte toplanan ile yarışamadı. Kadın, çocuk yetiştirmeye ek olarak devamlı yönetim gücünü uygulama ve evdeki gücünü arttırmanın bileşik sorumluluklarının üstesinden gelebilecek bir konumda değildi. Yaklaşmakta olan kadın eş kaçırılışı ve daha sonraki kadın eş satın alımı, anne\hyp{}ailesinin geçişini hızlandırdı.
\vs p084 2:7 Anne\hyp{}ailesinden baba\hyp{}ailesine olan olağanüstü derecedeki etkileyici değişim, insan ırkı tarafından şimdiye kadar uygulanmış en keskin ve bütüncül geri dönüşlerden bir tanesidir. Bu değişiklik doğrudan bir biçimde daha büyük toplumsal etkiye ve artan aile değişimine sebebiyet vermiştir.
\usection{3.\bibnobreakspace Baba Baskısı Altındaki Aile}
\vs p084 3:1 Anneliğin içgüdüsü kadını evliliğe itmiş olabilir; ancak, adetlerin etkisiyle birlikte, erkeğin üstün kuvveti onu neredeyse aile bağları içinde kalmaya mecbur bırakmıştı. Kırsal yaşam, aile yaşamının ata\hyp{}erkil türü olarak adetlerin yeni bir sisteminin yaratılmasına meyil vermişti; buna ek olarak, sürü sahipliği ve tarım adetleri altında aile birlikteliğinin temeli, babanın sorgulanamaz, keyfi yönetim gücüydü. İster ülkesel veya ailesel olsun toplumun tümü, bir ata\hyp{}erkil düzenin dikta yönetimi aşamasından geçmişti.
\vs p084 3:2 Eski Ahit dönemi boyunca kadın türüne gösterilen yetersiz nezaket, sürü sahiplerinin adetlerinin gerçek bir yansımasıdır. Musevi aile reislerinin hepsi, “Hâkim benim Çobanımdır” ifadesinde gözlendiği gibi sürü sahipleriydi.
\vs p084 3:3 Ancak erkek, bu geçmiş dönemler boyunca kadına gösterdiği az değer için ondan daha fazla suçlanacak bir konumda değildi. O, acil bir durumda faaliyet gösterememesi nedeniyle ilkçağ dönemleri boyunca toplumsal tanınmayı elde etmede başarısız oldu; kadın, olağanüstü bir durumun veya krizin kahramanı değildi. Annelik, varoluş mücadelesi içinde ayrı bir engeldi; anne sevgisi, kabilesel savunmada kadınları sınırlı bir konumda bıraktı.
\vs p084 3:4 İlkel kadınlar aynı zamanda, erkeğin kavgacılığına ve gürbüzlüğüne dair beğeni ve takdirleri tarafından kendilerini erkeklere bağımlı kıldılar. Kahramanlığın bu yüceltilişi erkek benliğini yükseltirken, kadınınkini alçaltıp onu daha bağımlı bir konuma getirdi; bir askeri üniforma, hala kadınlık duygularını harekete geçirmektedir.
\vs p084 3:5 Daha gelişmiş ırklar arasında kadınlar, erkekler kadar kalıplı ve onlar kadar kuvvetli değillerdir. Daha güçsüz olarak kadın böylelikle daha sezinçli hale geldi; öncül bir biçimde cinsel cazibelerinden faydalanmayı öğrendiler. Her ne kadar birazcık daha az derin olsa da, erkekten daha dikkatli ve daha muhafazakâr hale geldi. Erkek, savaş meydanında ve avcılıkta kadının üstüydü; ancak evde kadın çoğunlukla, erkeklerin en ilkel olanlarının bile nasıl üstesinden geleceğini bildi.
\vs p084 3:6 Sürü sahibi birey sürülerine yaşamının devamlılığı için baktı; ancak bu kırsal çağlar boyunca kadın hala bitkisel yiyeceği sağlamak zorundaydı. İlkel insan topraktan kaçınmıştı; toprak onun için haddinden fazla huzurlu, olması gerektiğinden fazla maceradan yoksundu. Orada kadınların bitkileri daha iyi yetiştirebildiklerine dair eski bir hurafede bulunmaktaydı. Bugünün birçok geri kabilesi içinde erkekler et, kadınlar sebze pişirmektedir; ve Avustralya’nın ilkel kabileleri yaşamlarını idame ettirirlerken kadınlar hiçbir zaman avcılık oyunlarına katılmaz, erkekler ise bir kök toplamak için eğilmezler.
\vs p084 3:7 Kadın her zaman çalışmak zorundaydı; en azından çağdaş dönemlere kadar kadın her zaman gerçek bir üretken olmuştur. Erkek çoğunlukla kolay yolu tercih etmiştir; ve bu eşitsizlik, insan ırkının bütüncül tarihi boyunca varlığını sürdürmüştür. Kadın her zaman; aile mülkiyetini taşıyan ve çocuklara bakan bir biçimde zorluklara göğüs geren olmuş, böylelikle erkeğin ellerini savaşmak ve avcılıkta bulunmak için serbest bırakmıştır.
\vs p084 3:8 Kadının ilk bağımsızlığı, o döneme kadar kadın işi olarak görülen şeyi yapmayı kabul eden bir biçimde, erkeğin toprağı işlemeye razı oluşuyla geldi; erkek esirlerin öldürülmeyip tarımla uğraşanlar olarak köleleştirilmeleri büyük bir ileri adımdı. Bu uygulama, evi idare etmesine ve çocukların yetiştirilmesine daha fazla zaman ayırabildiği bir biçimde kadının özgürlüğünü getirdi.
\vs p084 3:9 Gençlere sütün verilmesi, bebeklerin öncül bir biçimde sütten kesilebilmelerine neden olup, bunun sonucunda anneler tarafından daha fazla çocuğun bakılması onların zaman zaman geçici olarak deneyimlediği çocuksuzluk dönemlerinden kurtulmalarına zemin hazırlamıştı; bunun yanı sıra koyun ve keçi sütünün kullanılması, bebek ölümlerini fazlasıyla azalttı. Toplumun sürü aşamasından önce anneler, dört ve beş yaşına kadar bebeklerini emzirmektelerdi.
\vs p084 3:10 İlkel savaşların fazlasıyla azalması ile birlikte, cinse dayalı iş bölümü arasındaki uçurum azaldı. Ancak kadınlar hala zorlu görevleri gerçekleştirirken, erkekler ise gözcülük yapmaktaydılar. Hiçbir konak yeri veya köy, gündüz veya gece nöbetçisiz bırakılmamaktaydı; ancak bu görev bile, köpeklerin evcilleştirilmesiyle hafifleştirilmişti. Çoğunlukla tarımın gelişi, kadının saygınlığını ve toplumsal düzeyini arttırdı; en azından bu durum, erkeğin ziraatçılığa dönüş zamanına kadar doğruluk göstermekteydi. Ve erkekler kendilerini toprağın ekimi ile tanımlar tanımlamaz, ilerleyen nesillere kadar gelişen bir biçimde tarım yöntemleri içinde doğrudan bir biçimde gerçekleşen büyük gelişme teminat altına alındı. Erkek; avcılık ve savaşta örgütlenmenin değerini öğrenmiş bir halde bulunup, bu yöntemleri üretime getirip, daha sonra, kadının çok belirli olmayan yöntemlerini fazlasıyla geliştirerek onun işinin büyük bir kısmını almıştır.
\usection{4.\bibnobreakspace Öncül Toplumda Kadının Düzeyi}
\vs p084 4:1 Genel olarak herhangi bir çağda kadının değeri, bir toplumsal kurum olarak evliliğin evrimsel ilerleyişinin adil bir göstergesiyken; evliliğin ilerleyişinin kendisi, insan medeniyetinin ilerlemelerine ışık tutan kabul edilebilir derecede doğru ibredir.
\vs p084 4:2 Kadının düzeyi her zaman toplumsal bir çelişkidir; o her zaman erkeklerin zeki yöneticisidir; o her zaman, erkeklerin daha güçlü olan cinsel uyarımını kendi çıkarları ve gelişimi için kullanmıştır. Zeki bir biçimde cinsel cazibelerinden faydalanarak o sıklıkla, en alçak kölelikte bile erkek tarafından tutulurken, erkek üzerinde baskın bir gücü uygulamaya yetkin olmuştur.
\vs p084 4:3 Öncül kadın erkek için bir arkadaş, yar, sevgili ve eş değildi; bunun yerine o bir mülkiyet eşyası, bir hizmetçi veya köle daha sonra ise ekonomik bir eş, eğlence ve çocukları dünyaya getirendi; Yine de münasip ve tatmin edici cinsel ilişkiler her zaman, kadınların tercihine ve işbirliğine bağlı olmuştur; ve bu durum ussal kadınlara her zaman, bir cinsiyet olarak toplumsal düzeylerinden bağımsız olarak, doğrudan ve kişisel düzeyleri üzerinde ciddi bir etkide bulunma gücü vermiştir. Ancak erkeklerin güvensizliği ve şüphesi; kadınların en başından beri, esaretlerini hafifletmek amacıyla kurnazlığa başvurmak zorunda kalışları gerçeğiyle dindirilememişti.
\vs p084 4:4 Cinsiyetler, birbirlerini anlamada büyük zorluk çekmektedirler. Erkek; şüpheyle ve nefretle değilse bile, cahilce duyulan güvensizliğin ve korkulan büyülenmenin tuhaf bir karışımıyla onları görerek kadını anlamada zorlandı. Birçok kabilesel ve ırksal tarihi anlatımlar sorunu Havva’ya, Pandora’ya ve bazı diğer kadın türünün temsilcisine dayandırmaktaydı. Bu anlatımlar her zaman, kadının erkek üzerine kötülüğü getirdiği şeklinde göstermek için çarptırılmıştı; ve bunların hepsi, bir zamanlar var olmuş evrensel bir biçimde kadınlara karşı duyulan güvensizliğe işaret etmektedir. Yaşam boyu bekâr kalan bir din adamlığını desteklemek için bahsi geçen nedenler arasında onların en başında gelen sav, kadının bayalığıydı. Büyücü olarak addedilen bireylerin çoğunun kadın olması gerçeği, bu cinsin eski dönemden gelen itibarını arttırmamıştı.
\vs p084 4:5 Erkekler uzun bir süreden beri kadınları tuhaf, hatta olağanın dışındaki varlıklar olarak görmektedir. Onlar, kadınların ruhlara bile sahip olmadıklarına inanmışlardı; böylelikle kadınlara isim verilmesine karşı koymuşlardı. Öncül dönemlerde, bir kadın ile gerçekleştirecek ilk cinsel ilişkiden büyük bir korku bulunmaktaydı; bu nedenle, bir bakire ile bir din adamının öncül cinsel ilişkisi adet haline gelmişti. Bir kadının gölgesinin bile tehlikeli olduğu düşünülmekteydi.
\vs p084 4:6 Çocuk doğurma bir dönem çoğunlukla, bir kadını tehlikeli kılan ve temizlikten uzaklaştırılan şey olarak görülmekteydi. Ve birçok kabile âdeti; bir annenin, bir çocuğun doğumundan sonra geniş arınma törenlerinden geçmek zorunda olduğuna emretmişti. Erkek eşin yatma sürecine katıldıkları topluluklar dışında, anne olması beklenen birey yalnız bırakılarak ondan uzak durulurdu. Eski topluluklar, bir çocuğun evde doğumundan bile kaçınmışlardı. Nihayeten, kıdemli kadınların doğum sırasında anneye katılmalarına izin verilmişti; ve bu uygulama, ebelik mesleğinin doğumuna kaynaklık etmişti. Doğum sırasında, bir doğumu kolaylaştırma çabası içinde sayısız budalaca şey söylenir ve yapılırdı. Hayaletin müdahalesini engellemek için yeni doğan bebeğe kutsal su serpiştirmek adetti.
\vs p084 4:7 Saf kabileler arasında doğum, yalnızca iki veya üç saat süren bir biçimde göreceli olarak kolaydı; melez ırklar arasında nadiren bu kadar kolaydı. Eğer bir kadın, özellikle ikizlerin doğumu sırasında olmak üzere, doğum anında yaşamını yitirirse ruhaniyet zinasından suçlu olduğuna inanılırdı. Daha sonra daha yüksek düzeydeki kabileler doğum anındaki ölüme cennetin iradesi gözüyle baktılar; bu tür anneler, soylu bir amaç uğrunda yaşamlarını yitirmiş olarak değerlendirilirlerdi.
\vs p084 4:8 Giyim ve bireyin bedenini gösterişine dair tanımladığınız kadınların iffeti, bir adet döneminde görülmekten duydukları ölümcül korkudan doğmuştu. Bu şekilde görülmek, tabunun ihlal edilişi olarak oldukça ciddi bir günahtı. Eski dönemlerin adetleri uyarınca ergenliğinden hamilelik dönemine kadar her kadın, her ay bir tam hafta boyunca bütüncül aile ve toplum tecridine tabiiydi. Onun dokunduğu, oturduğu veya üzerine uzandığı her şey “kirletilmişti.” Bir genç kızı, bedeninden kötü ruhaniyeti uzaklaştırma çabası içinde her aylık dönemden sonra vahşice dövmek uzun bir süre adetti. Ancak bir kadın hamilelik yaşını geçtiği zaman kendisine çoğunlukla, daha fazla hak ve ayrıcalık tanınan bir biçimde daha nazik davranılırdı. Bütün bunların ışığı altında kadınlara küçük gözle bakılması tuhaf değildir. Yunanlılar bile adet halindeki kadını kirlenmenin üç büyük nedeninden biri olarak gördü; diğer ikisi domuz ve sarımsaktı.
\vs p084 4:9 Bu eski dönem görüşleri ne kadar aptalca olursa olsun; haddinden fazla çalışan kadınlara, en azından gençken, her ayda bir hafta memnuniyet verici rahatlama ve yararlı düşünce süreci vermesi bakımından birtakım iyi sonuçlara neden olmuştu. Bu süreçte böylece onlar, zamanın geri kalan kısmında erkek birliktelikleriyle nasıl başa çıkmaları gerektiği üstüne düşünebilmektelerdi. Kadınların bu tecridi aynı zamanda erkekleri haddinden fazla bir biçimde cinsel ilişkiye düşmekten korumuş, böylece nüfusun sınırlandırılışına ve öz\hyp{}denetiminin gelişimine dolaylı bir biçimde katkıda bulunmuştu.
\vs p084 4:10 Bir erkeğin kadın eşini istediği zaman öldürmesi hakkı kaldırıldığında büyük bir ilerleme gerçekleşmişti. Benzer bir biçimde, bir kadının düğün hediyelerine sahip olabilmesi ileri bir adımdı. Daha sonra kadın, özel mülkiyete sahip olma, onu denetleme ve hatta onu satma hakkı kazanmıştı; ancak o uzun bir süre boyunca, din veya devlet kademelerinde herhangi bir görevde bulunma hakkından mahrum bırakılmıştı. İsa’dan sonra yirminci yüzyıla kadar ve bu yüzyıl içinde kadına her zaman az veya çok bir mülkiyet olarak davranılmıştı. Kadın, erkeğin denetim tecridinden dünya genelinde yaygın bir özgürlüğü henüz elde edememiştir. Gelişmiş topluluklar arasında bile insanın kadını korumaya çabası bile her zaman, kendi üstünlüğünün üstü kapalı bir biçimde hatırlatıcısı olmuştur.
\vs p084 4:11 Ancak ilkel kadınlar, yakın zamanda özgürleştirilmiş kız kardeşlerinin alışkanlık haline getirdiği bir biçimde kendilerine acımamışlardı. Onlar, sonuç olarak, oldukça mutlu ve hallerinden memnunlardı; onlar, daha iyi ve farklı bir yaşam üzerinde düşünmeye cesaret etmemişlerdi.
\usection{5.\bibnobreakspace Gelişen Adetler Uyarınca Kadın}
\vs p084 5:1 Bireyin kendi devamlılığını sağlamasında kadın erkeğin eşitidir; ancak bireyin kendi yaşamını idame ettirişindeki birliktelik içerisinde kadın, kendisini geri konuma iten belirli bir eşitsizlik içerisinde emek vermektedir; ve zorunlu anneliğin bu sınırlılığı yalnızca, gelişen medeniyetin aydınlanmış adetleriyle ve erkeğin sonradan kazanmış olduğu hakkaniyet niteliğinin artan algısıyla telafi edilebilir.
\vs p084 5:2 Toplum evirildikçe kadınlar arasındaki cins ölçütleri yükselmişti, çünkü onlar cins adetlerinin ihlal edilişinden daha çok zarar görmüşlerdi. Erkeklerin cins ölçütleri, medeniyetin talep ettiği hakkaniyetin ortak duyuşunun bir sonucu olarak sadece yavaş bir biçimde gelişme göstermektedir. Doğa hakkaniyete dair hiçbir şey bilmemektedir --- doğum sancılarını sadece kadınlara çektirmektedir.
\vs p084 5:3 Cinsel eşitliğe dair çağdaş düşünce, güzel ve gelişen bir medeniyete layıktır; ancak bu düşünce doğa içinde bulunamaz. Güçlü olanın haklı olduğu zamanlarda erkek kadın üzerinde hâkimiyet kurmaktadır; daha fazla adalet, barış ve hakkaniyet egemen olduğunda kadın, kademeli bir biçimde kölelik ve belirsizlik düzeyinden kurtulmaktadır. Kadının toplumsal düzeyi çoğunlukla, herhangi bir milletteki veya çağdaki askeri zihniyet ölçüsüne zıt bir biçimde çeşitlilik göstermiştir.
\vs p084 5:4 Ancak erkek ne bilinçli bir biçimde ne de hesaplı olarak kadının haklarını gasp etmiş, sonrasında onları kendilerine istemeye istemeye geri vermiştir; bütün bunların hepsi, toplumsal evrimin bilinçsiz ve tasarlanmamış bir sürecidir. İlave haklara memnuniyetle sahip olma zamanı kadınlar için geldiğinde, onlar bunları elde ettiler; ve bunların hepsi erkeğin bilinçli tutumu olmadan gerçekleşti. Yavaş ama kesin bir biçimde adetler, medeniyetin devamlı evriminin bir sonucu olan bu toplum uyumunu sağlayacak bir biçimde değişme göstermektedir. Gelişen adetler yavaş bir biçimde, kadınlara artan bir biçimde iyi davranılmasını sağlamıştı; onlara kabalığı sürdüren kabileler varlıklarını sürdürememişlerdi.
\vs p084 5:5 Âdem ve Nod toplulukları, kadınlara artan bir tanınma verdiler; ve göç eden And unsurlarına karışan bu topluluklar zaman içerisinde, toplum içinde kadınların yerine dair Cennet Bahçesi öğretilerinden etkilenme eğilimi göstermişlerdir.
\vs p084 5:6 Öncül Çin ve Yunan toplulukları kadınlara çevre toplulukların çoğundan daha iyi davranmıştı. Ancak Museviler, aşırı bir biçimde onlara güvensizlik duymaktalardı. Batıda kadın; her ne kadar Hıristiyanlık erkekler üzerine daha sıkı cinsel yükümlülükler getiren adetleri geliştirse de, Hıristiyanlık ile bütünleşmiş hale gelen Pavlik savlar altında zor bir engele sahip olmuşlardı. Kadının durumu Muhammedizm içinde kendisine verilen tuhaf alt düzey altında biraz ümitsiz bir konumda olup, kendisi diğer birkaç Doğu dinin öğretilerinin etkisi altında daha bile kötü koşullarda yaşamını sürdürmektedir.
\vs p084 5:7 Din değil bilim gerçek anlamıyla kadınları özgürleştirmişti; bilim, ev sınırlarından onu büyük ölçüde özgürleştiren çağdaş fabrikaydı. Erkeğin fiziksel yetileri, makinelerin yeni idaresinde artık hayati bir etken değildi; bilim böylelikle, erkek gücünün kadın gücüne artık üstün olmamasını sağlayan bir biçimde yaşam koşullarını değiştirmiştir.
\vs p084 5:8 Bu değişiklikler; kadının ev içindeki köleliğinden özgürleşmesine doğru meyil etmekte olup, erkeğinkine neredeyse eşit seviyede bir kişisel özgürlük ve cinsel tercihi artık memnuniyetle deneyimleyebildiği türden bir değişikliği onun toplum düzeyine getirmiştir. Bir zamanlar bir kadının değeri onun yiyecek üretme yetisinden meydana gelmekteydi; ancak icat ve refah --- erdem ve cazibenin alanları olarak --- içinde faaliyet gösterebileceği yeni bir dünya yaratmasında onu yetkin kıldı. Böylelikle üretim, kadının toplumsal ve ekonomik özgürleşmesi için verdiği bilinçsiz ve tasarlanmamış kavgasından galibiyetle ayrıldı. Ve tekrar evrim, açığa çıkarışların bile yerine getirmede başarısız olduğu bir şeyi yapmada başarılı oldu.
\vs p084 5:9 Toplum içinde kadının konumunu belirleyen adil olmayan adetlerden gelen aydınlanmış toplulukların tepkisi gerçekten de aşırı uçlarında bir sarkaç görünümüne sahip olmuştur. Endüstriyelleşmiş ırklar arasında kadın neredeyse her hakka sahip ve askerlik gibi birçok zorunluluktan muaftır. Yaşam mücadelesi içinde elde edilen her kolaylık kadının özgürleşmesine fazlasıyla katkıda bulunmuştur; ve o doğrudan bir biçimde, tekeşliliğe doğru gerçekleştirilen her ilemeden doğrudan bir biçimde yarar sağlamıştır. Toplumun ilerleyici evrimi içerisinde adetlerin her uyumlu hale getirilişinde, zayıf olanlar her zaman ters orantılı kazançlar elde etmektedirler.
\vs p084 5:10 Çift evliliğinin nihai emelleri içinde kadın en sonunda tanınmayı, saygınlığı, bağımsızlığı, eşitliği ve eğitimi kazanmıştır; ancak o, bu yeni ve emsalsiz kazanıma layık olduğunu ispatlayacak mı? Çağdaş kadın toplumsal özgürlüğün bu büyük kazanımına tembellik, umursamazlık, çocuksuzluk ve sadakatsiz ile mi cevap verecek? Bugün, yirminci yüzyılda, kadın; uzun mevcudiyetinin hayati sınavını vermektedir.
\vs p084 5:11 Kadın ırk yetiştiriliminde erkeğin eşit eşidir; böylelikle ırksal evrimin gerçekleşmesinde erkek kadar öneme sahiptir; bu nedenle evrim artan bir biçimde, kadın haklarının gerçekleştirilmesine doğru meyil etmiştir. Ancak kadının hakları hiçbir biçimde erkeğin hakları değildir. Tıpkı erkek kadınların haklarıyla refah düzeyini yükseltemezse, kadın erkeğin hakları altında gelişmesini sağlayamaz.
\vs p084 5:12 Her cinsiyet, içinde kendisine özel haklar ile birlikte kendisine özgü ayrışmış yaşam alanına sahiptir. Eğer kadın gerçekten erkeğin sahip olduğu hakların tümünden memnuniyet duyuyorsa; ileride, er veya geç sonunda, birçok kadının şimdilerde memnuniyetle sahip olduğu ve oldukça yakın bir geçmişte erkeklerden kazandığı mertlik ve özel itibarın yerini acımasız ve duygusuz mücadele kesin bir biçimde alacaktır.
\vs p084 5:13 Medeniyet hiçbir zaman, cinsler arasında mevcut olan davranış uçurumunu hiçbir zaman ortadan kaldıramaz. Çağdan çağa, adetler değişmektedir; ancak içgüdü her zaman aynı kalmaktadır. İçkin anne sevgisi, özgürleşmiş kadının üretimde erkeğin ciddi rakibi olmasına hiçbir zaman izin vermeyecektir. Sonsuza kadar her cins, biyolojik farklılıkları ve düşünsel yetkinliklerinin benzersizlikleriyle belirlenmiş olan kendisine ait nüfuz alanında üstün olmaya devam edecektir.
\vs p084 5:14 Her cins, her ne kadar ileride sürekli yaşanacak bir biçimde arada sırada çakışma gösterse de, her zaman kendisine ait özel bir alana sahip olmaya devam edecektir. Sadece toplumsal olarak erkek ve kadın eşit şartlar altında rekabet edecektir.
\usection{6.\bibnobreakspace Kadın ve Erkeğin Birlikteliği}
\vs p084 6:1 Çoğalım uyarımı kesin ve sürekli bir biçimde, bireyin kendi devamlılığı için erkekler ve kadınları bir araya getirmektedir; ancak tek başına, --- bir evin inşası olarak --- ortak bir işbirliği içerisinde birlikte kalmaya devam etmelerini garanti altına almamaktadır.
\vs p084 6:2 Her başarılı insan kurumu, ahenkli işlevsel çalışmaya uyarlanmış kişisel çıkarların zıtlıklarından meydana gelmektedir; ve ev işlerinin idaresi bu anlamda bir istisna değildir. Ev inşasının temeli olan evlilik, oldukça sık bir biçimde doğa ve toplum arasındaki karşıtlığı simgeleyen bu zıtsal işbirliğinin en yüksek dışavurumudur. Çatışma kaçınılmazdır. Çiftleşme içkindir; o doğaldır. Ancak evlilik biyolojik değildir; o toplumsal bir olgudur. Arzu insan ve kadının bir araya gelmesini teminat altına alacaktır; ancak daha zayıf düzeyde bulunan ebeveyn içgüdüsü ve toplumsal adetler onları bir arada tutacaktır.
\vs p084 6:3 Erkek ve kadın, genel olarak değerlendirildiğinde, yakın ve içten birliktelik içerisinde yaşayan aynı türün iki farklı çeşididir. Onların bakış açıları ve bütüncül yaşam tepkileri temel olarak farklıdır; onlar, birbirlerini bütüncül ve gerçek anlamıyla kavrama yetisinden tamamiyle yoksundurlar. Cinsler arasındaki bütüncül anlayış erişilemezdir.
\vs p084 6:4 Kadın, erkeklerden daha fazla içgüdüye sahip olan görünüme sahiplerdir; ancak onlar aynı zamanda, bir ölçüde daha az mantıksal olarak görülmektedirler. Kadın, buna rağmen, her zaman; ahlaki değerlerin öncüsü ve insan türünün ruhani önderi olmuştur. Beşiği sallayan kudretli el hala bireyleri nihai son ile bütünleştirmektedir.
\vs p084 6:5 Erkekler ve kadınlar arasındaki doğa, tepki, bakış açısı ve düşüncedeki farklılıklar, endişe yaratması söyle dursun, hem bireysel hem de bütünlüksel biçimde insan türü için çok faydalı olarak değerlendirilmelidir. Evren yaratılmışlarının birçok düzeyi, kişilik dışavurumunun çifte fazları içinde yaratılmışlardır. Faniler, Maddi Evlatlar ve yarı\hyp{}evlat\hyp{}unsurları arasında bu farklılık erkek ve kadın olarak tanımlanmaktadır; yüksek melekler, çocuksu melekler ve Morontia Refakatçileri arasında, artı veya girişken ve eksi veya çekingen olarak adlandırılmaktadır. Bu türden çifte birliktelikler, hatta Cennet\hyp{}Havona sistemi içindeki belirli üç katmanlı birlikteliklerde olduğu gibi, çok yönlülüğü fazlasıyla arttırmakta ve içkin sınırlılıkların üstesinden gelmektedir.
\vs p084 6:6 Erkek ve kadınlar, fani süreçlerine ek olarak morontiyal ve ruhsal yaşamlarında birbirlerine ihtiyaç duymaktadır. Erkek ve kadın arasında var olan bakış açılarındaki farklılık, ilk yaşamın da ötesinde yerel ve üstün evren yükselimleri boyunca varlığını devam ettirmektedir. Ve Havona’da bile, bir zamanlar erkek ve kadın haldeki kutsal yolcular hala Cennet yükselişinde birbirlerine yardım edeceklerdir. Kesinlik Birlikleri’nde bile yaratım başkalaşımı hiçbir zaman, insanların erkek ve kadın olarak adlandırdıkları kişilik eğilimlerini ortadan kaldırmayacaktır; insan türünün bu iki temel çeşitliliği her zaman birbirlerini olumlu yönde etkilemeye, harekete geçirmeye, desteklemeye ve birbirlerine yardım etmeye devam edecektir; her zaman onlar, karmaşık evren sorunlarının çözümünde ve çok katmanlı kâinat zorluklarının aşılmasında karşılıklı bir biçimde işbirliğine bağlı halde bulunacaklardır.
\vs p084 6:7 Her ne kadar cinsler birbirlerini bütünüyle anlamayı hayal dahi edemeseler de, onlar etkin bir biçimde birbirlerinin tamamlayıcılarıdır; ve her ne kadar işbirliği, sıklıkla, kişisel bir düzeyde az veya çok muhalif olsa da idare edilişe ve toplumun çoğalımına yetkindir. Evlilik, ırkların farklılıklarını bir araya getirmek için tasarlanmış bir kurum iken, diğer bir yandan da medeniyetin devamlılığını gerçekleştirmekte ve ırkın çoğalımını teminat altına almaktadır.
\vs p084 6:8 Evlilik, tüm insan kurumlarının anasıdır; çünkü o doğrudan bir biçimde, toplum yapısının temeli olan ev inşası ve idaresine yol açmaktadır. Aile hayati derecede, bireyin kendisini idame ettirme düzenine bağlıdır; çünkü o medeniyetin adetleri uyarınca ırkın devamlılığı için tek ümit kaynağı iken, aynı zamanda, birey tatmininin fazlasıyla memnuniyet verici belli başlı türlerini en etkin bir biçimde sağlamaktadır. Aile; erkek ve kadın eşin toplumsal ilişkilerine ek olarak onların biyolojik ilişkilerinin evrimini bir araya getiren bir biçimde, insanın kendi başına elde ettiği en büyük kazanımıdır.
\usection{7.\bibnobreakspace Aile Yaşamı’nın İdealleri}
\vs p084 7:1 Cinsel çiftleşme içgüdüsel, çocuklar doğal sonuçtur; ve aile böylelikle kendiliğinden var olmaktadır. Aileler ırkın veya milletin meydana getirirken, aynı şey onların yarattığı toplumu için de geçerlidir. Eğer aileler iyi olursa, toplum benzer bir biçimde iyi olacaktır. Musevi ve Çin topluluklarının sahip olduğu büyük kültürel istikrar, onların aile topluluklarındaki güce dayanmaktadır.
\vs p084 7:2 Kadının çocukları için sevme ve onlara bakma içgüdüsü, evliliğin ve ilkel aile yaşamının desteklenmesinde onu ilgili bir taraf haline getiren bir biçimde zorladı. Erkek evin inşasına yalnızca, daha sonraki adetler ve toplumsal kabuller tarafından itilmişti; evliliğin ve evin istikrara kavuşturulmasında ilgi duymada yavaş kalmıştı, çünkü cinsel ilişki eylemi biyolojik bakımdan onun yaşamında hiçbir sonuç doğurmamaktadır.
\vs p084 7:3 Cinsel birliktelik doğaldır; ancak evlilik toplumsal olup, her zaman adetler tarafından düzenlenmektedir. Özel mülkiyet, gurur ve mertlik ile birlikte adetler (dini, ahlaki ve etik kurallar olarak), evlilik ve aile kurumlarını istikrarlaştırmıştır. Ne zaman adetler dalgalanma gösterirse, ev\hyp{}evlilik kurumunun istikrarında dalgalanma gerçekleşmektedir. Evlilik mevcut an içerisinde özel mülkiyet aşamasından kişisel döneme geçmektedir. Eskiden erkek mülkiyeti olduğu için kadını korumuştu, ve o aynı sebepten kadın erkeğe itaat etmişti. Yararlarından bağımsız bir biçimde bu düzen istikrarı sağlamıştı. Şimdi kadın artık bir mülkiyet olarak görülmemekte, ve evlilik\hyp{}ev kurumunu istikrarlı hale getirmek için tasarlanan yeni adetler ortaya çıkmaktadır:
\vs p084 7:4 1.\bibnobreakspace Dinin yeni rolü --- çoğalım ayrıcalığına dair genişlemiş anlayış biçiminde, kâinatsal vatandaşları yaratma düşüncesi olarak, ebeveynsel deneyimin hayati oluşuna dair öğreti --- Yaratıcı’ya evlatlar verme.
\vs p084 7:5 2.\bibnobreakspace Bilimin yeni rolü --- insanın denetimine bağlı bir biçimde çoğalımın giderek daha fazla bir biçimde gönüllü hale gelişi. Eskiçağ dönemlerinde anlayış yoksunluğu, bu yönde arzunun bütüncül yoksunluğunda çocukların ortaya çıkışını teminat altına aldı.
\vs p084 7:6 3.\bibnobreakspace Haz cazibelerinin yeni işlevi --- bu ırksal varoluşta yeni bir etkiyi sunmaktadır; eskiçağ insanı istenmeyen çocukları terk etmekteydi; çağdaş insanlar ise onları tanışmayı reddetmektedirler.
\vs p084 7:7 4.\bibnobreakspace Ebeveynsel içgüdünün gelişimi --- her nesil şimdi, bir sonraki neslin müstakbel ebeveynleri olarak çocukların doğumunu etkin bir biçimde teminat altına alamayacak kadar güçsüz ebeveynsel içgüdüsü olan bireyleri ırkın üretmek akımından devre dışı bırakma eğilimi göstermektedirler.
\vs p084 7:8 Ancak bir kurum olarak ev, bir erkek ve kadının bir birlikteliği olarak, daha kesin bir değişle; çok uzun bir süre önce tek başlarına bırakılmış Andon ve onun doğrudan soylarının tekeşli uygulamalarının gerçekleştiği dönem olan yaklaşık yarım milyon yıl öncesinin Dalamatia zamanına kadar uzanmaktadır. Aile yaşamı, buna rağmen, Nod ve daha sonraki Âdem unsurları döneminin öncesinde fazlasıyla takdir edilecek bir konumda değildi. Âdem ve Havva, insan türünün tümü üzerinde devamlılığı olan bir etki bıraktı; dünya tarihi içerisinde ilk defa, erkek ve kadınlar Cennet Bahçesi içerisinde yan yana yürürken görüldüler. Bahçe üyeleri olarak bütüncül aile biçimindeki Cennet Bahçesi ideali, Urantia üzerinde yeni bir düşünceydi.
\vs p084 7:9 Öncül aile, köleleri de içine alan bir biçimde her bireyin tek çatı altında yaşadığı, birbirlerine bağlı bir çalışma topluluğundan oluştu. Evlilik ve aile yaşamı her zaman özdeş değildir; ancak ister istemez birbirleriyle yakından ilişki içerisindedir. Kadın her zaman bireysel aileyi istemişti, ve sonunda istediğini elde etti.
\vs p084 7:10 Dünyaya getirilen bireye duyulan sevgi neredeyse evrensel olup, ayrı bir var oluş değerine sahiptir. Eski çağ toplulukları her zaman, çocuğunun refahı için annenin rahatından feragat etmişlerdi; bir Eskimo annesi hala bile, çocuğunu yıkamak yerine hala diliyle temizlemektedir. Ancak ilkel anneler yalnızca çok gençken çocuklarını besleyip, onlara ilgi göstermişlerdi. Devamlılığı olan ve sürekli insan birliktelikleri hiçbir zaman tek başına biyolojik sevgi üzerine kurulmamıştır. Hayvanlar çocuklarını sevmektedir; insan --- medenileşmiş insan --- çocuklarının çocuklarını sevmektedir. Medeniyet daha yüksek düzeyde oldukça, ebeveynlerin çocukların gelişimi ve başarısında duyduğu sevinç artmaktadır; böylelikle \bibemph{isimden} duyulan gururun yeni ve daha yüksek düzeydeki gerçekleşimi açığa çıkmaktadır.
\vs p084 7:11 Eski çağ toplulukları arasındaki büyük ailelerin illa da sevgi üzerine kuruluş olmasına dair bir zorunluluk bulunmamaktaydı. Birçok çocuk şu nedenlerle istenmişti:
\vs p084 7:12 1.\bibnobreakspace İşçiler olarak değerliydiler.
\vs p084 7:13 2.\bibnobreakspace Eski\hyp{}çağ sigortalarıydılar.
\vs p084 7:14 3.\bibnobreakspace Kız çocuklar üzerinden para elde edilebilmekteydi.
\vs p084 7:15 4.\bibnobreakspace Aile gururu, ismin yayılmasını gerektirmekteydi.
\vs p084 7:16 5.\bibnobreakspace Çocuklar koruma ve savunma sağlamaktaydı.
\vs p084 7:17 6.\bibnobreakspace Hayalet korkusu, tek başına kalmaya dair derin bir korku yarattı.
\vs p084 7:18 7.\bibnobreakspace Belirli dinler doğumu zorunlu kıldı.
\vs p084 7:19 Eski din adamları erkek evlatlara sahip olmadaki başarısızlığı, tüm zamanlar ve ebediyetin tümü içindeki en büyük felaket olarak gördüler. Onlar, hayaletin ruhaniyet\hyp{}yerleşkeye olan ilerleyişi için gerekli kurbanlıkları veren bir biçimde ölüm sonrasındaki şöleni yönetmesi için her şeyden çok fazla bir şekilde erkek evlatlara sahip olmak istediler.
\vs p084 7:20 İlkçağ ilkel insanları içinde çocukların disiplini çok erken bir yaşta başladı; ve çocuk öncül bir biçimde, hayvanlara uygulandığı gibi, itaatsizliğin başarısızlık veya ölüm anlamına bile geleceğinin farkına vardı. Çağdaş itaatsizliğe oldukça fazla katkıda bulunan şey medeniyetin, budalaca davranışın doğal sonuçlarından çocuğu korumasıdır.
\vs p084 7:21 Eskimo çocukları çok az bir disiplin ve ceza altında gelişmektedir, çünkü onlar özü itibariyle uyumlu küçük hayvanlardır; kırmızı ve sarı insanların çocukları neredeyse eşit bir biçimde usludur. Ancak And kalıtımı taşıyan ırklardaki çocuklar bu kadar sakin değildir; bu daha fazla hayal gücüne sahip ve maceraperest olan gençler, daha fazla eğitim ve disipline ihtiyaç duymaktadır. Çocuk yetiştirilimindeki çağdaş sorunları, şu nedenler artan bir biçimde zor kılmaktadır:
\vs p084 7:22 1.\bibnobreakspace Irk karışımındaki büyük düzey.
\vs p084 7:23 2.\bibnobreakspace Yapay ve yüzeysel eğitim.
\vs p084 7:24 3.\bibnobreakspace Çocuğun ebeveynlerini taklit ederek kültürlü hale gelme olanaksızlığı --- ebeveynlerin aile yaşamından çok uzun bir süre ayrı kalması.
\vs p084 7:25 Aile disiplinine dair eski dönem düşünceleri, ebeveynlerin çocuğun mevcudiyetinin yaratıcısı olduklarına dair farkındalıktan kaynağını olan bir biçimde biyolojikti. Aile yaşamının gelişen nihai emelleri bir çocuğun dünyaya getirilişinin, belirli ebeveynsel hakları kazandırma yerine, insan mevcudiyetinin yüce sorumluluğu anlamına geldiği kavramsallaşmaya doğru ilerlemektedir.
\vs p084 7:26 Medeniyet ebeveynleri tüm sorumlulukları üzerine almış, çocukları ise tüm haklara sahip bir konumda görmektedir. Çocuğun ebeveynlerine olan saygısı, ebeveynsel korumada kastedilen zorunluluğa dair bilgiden değil, yaşam savaşından galibiyetle çıkmak için çocuğa verilen destekte sevgi dolu bir biçimde sergilenen ilginin, eğitimin ve şefkatin bir sonucu olarak doğal bir biçimde doğmaktadır. Gerçek ebeveyn, akıllı çocuğun ileride tanıyacağı ve takdir edeceğe devamlı bir hizmet\hyp{}vazifesi içindedir.
\vs p084 7:27 Mevcut üretim ve kent döneminde evlilik kurumu, yeni ekonomik doğrultularda evirilme göstermektedir. Aile yaşamı gittikçe artan bir biçimde pahalı hale gelmekte, geçmişte gelir olan çocuklar ekonomik giderler haline gelmişlerdir. Ancak medeniyetin güvencesi hala; bir neslin, bir sonraki ve gelecek nesillerin refahı üzerinde büyüyen yatırımda bulunma arzusuna dayanmaktadır. Ve ebeveyn sorumluluğunu devlet veya din kurumuna aktarmaya dair her çabanın medeniyetin refahı ve gelişiminin intiharı anlamına geleceği ortaya çıkacaktır.
\vs p084 7:28 Çocuklar ve onun sonucunda gelen aile yaşamıyla birlikte evlilik; insan doğası içindeki en yüksek potansiyelleri harekete geçirmekte, ve eş zamanlı olarak fani kişiliğin bu hareketlendirilmiş niteliklerinin dışavurumunda ideal ortamı sağlamaktadır. Aile, insan türlerinin biyolojik çoğalımı için zemin hazırlamaktadır. Ev, büyüyen çocuklar tarafından kan kardeşliğine dayalı etik kuralların içinde kavranabileceği doğal nitelikteki toplumsal ortamdır. Aile; ebeveyn ve çocukların, insanların tümü arasındaki kardeşliğin gelişmesi için oldukça temel nitelikte bulunan sabır, toplumsal fedakârlık, hoşgörü ve tahammülün derslerini öğrendikleri kardeşsel bütünlüğün ana birimidir.
\vs p084 7:29 Eğer medenileşmiş ırklar daha çoğunluksal bir biçimde And topluluklarının aile\hyp{}heyet uygulamalarına geri dönerse, insan toplumu fazlasıyla gelişebilir. Bu topluluklar, ata\hyp{}erkil veya baskıcı bir tür aile hükümetine sahip olmadılar. Onlar, bir aile hayatına dair her öneri ve düzenlenişi özgür ve içten tartışan bir biçimde, oldukça kardeşsel ve birliktelikseldi. Onlar tüm aile hükümet düzeni içinde ideal bir biçimde bütüncül bir haldedirler. Olası en yüksek düzeydeki bir aile içinde, evlatsal ve ebeveynsel sevginin ikisi de bütünlüksel bağlılıkla çoğalmıştır.
\vs p084 7:30 Aile yaşamı, göreve olan sadakat bilincinin atası biçiminde gerçek ahlakın kökenidir. Aile yaşamının bu zorunlu kılınan birliktelikleri; kişiliği istikrar altına almakta olup, onun büyümesini diğer ve farklı kişiliklere olan gerekli uyum zorunluluğu ile harekete geçirmektedir. Ancak buna bile ek olan bir biçimde, gerçek bir aile --- iyi bir aile --- ebeveynsel yaratıcılarına Yaratan’ın kendi çocuklarına olan tutumunu gösterirken; aynı zamanda bu tür ebeveynler, tüm evren çocuklarının Cennet ebeveyninin sahip olduğu derin sevginin yükselen dışavurumlarındaki uzun bir dizinin ilkini kendi çocuklarına sergilerler.
\usection{8.\bibnobreakspace Birey Tatmini’nin Tehlikeleri}
\vs p084 8:1 Aile yaşamıma karşı büyük tehdit, çağdaş haz çılgınlığı olan birey tatmininin baştan çıkarıcı artış dalgasıdır. Evlilik için başlıca teşvik unsuru eskiden ekonomikti; cinsel cazibe ikincil bir konumdaydı. Bireyin kendisini idare edişi üzerine inşa edilen evlilik, bireyin devamlılığını ve onunla birlikte birey tatmininin en arzu duyulan türlerinden bir tanesini sağladı. Bu kurum, yaşam için üç büyük teşvikin de hepsini içine alan tek oluşumdur.
\vs p084 8:2 Kökensel olarak mülkiyet, bireyin kendisini idare edişinin temel kurumuydu; bunun karşısında evlilik, bireyin devamlılığının benzersiz kurumu olarak faaliyet gösterdi. Her ne kadar yiyecek tatmini, oyun, mizah ve dönemsel cinsel ilişkilere dalma geçmişte birey tatmininin araçları olmuşken; evirilen adetlerin birey tatmininin ayrı herhangi bir kurumunu oluşturmada başarısız olduğu bir gerçek olarak varlığını sürdürmektedir. Ve haz verici mutluluğun özelleşmiş yöntemlerini geliştirmedeki bu başarısızlık nedeniyle insan kurumlarının tümü, bu haz arayışıyla oldukça bütüncül bir biçimde etkilenmektedir. Mülkiyet birikimi, birey tatminin tüm türlerini fazlalaştırmada bir araç haline gelmekteyken; evlilik sıklıkla, yalnızca bir haz aracı olarak görülmektedir. Ve bu haddinden fazla gerçekleşen tatmin, bu oldukça geniş bir biçimde yayılmış haz çılgınlığı, şimdi; ev kurumu olarak aile yaşamının toplumsal nitelikteki evrimsel oluşumunun şimdiye kadar karşısına dikilen en büyük tehdidi oluşturmaktadır.
\vs p084 8:3 Eflatun ırkı, insan türünün deneyimine yeni ve kusursuzca gerçekleştirilen tek bir niteliği tanıştırmışlardı --- mizah anlayışı ile bütünleşmiş oyun içgüdüsü. Bu nitelik, Sang ve Andon toplulukları içinde bir ölçüde bulunmaktaydı; ancak Âdemsel ırk kolu bu ilkel eğilimi, birey tatminin yeni ve yüceltilmiş bir türü olarak \bibemph{haz potansiyeline} yükseltti. Açlığın yatıştırılması dışında birey tatmininin temel türü cinsel tatmindir; ve bu tensel hazzın bu türü, Sang ve Ad topluluklarının karışımıyla devasa bir biçimde yoğunlaştı.
\vs p084 8:4 And\hyp{}sonrası ırkların huzursuzluk, meraklılık, serüvencilik ve sınırlandırılmamış haz niteliklerinin birleşiminde gerçek tehlike bulunmaktadır. Ruhun açlığı, fiziksel hazlar tarafından tatmin edilemez; ev ve çocuk sevgisi, hazzın akılsızca gerçekleştirilen arayışıyla çoğalmaz. Her ne kadar sanat, ses, ritim, müzik ve birey güzelliği kaynaklarını sonuna kadar kullansanız bile, böylece ruhu yüceltip ruhaniyeti besleyebileceğinizi hayal dahi edemezsiniz. Gösteriş ve moda, ev idaresine ve çocuk yetiştirilişine yardımcı olmamaktadır; gurur ve düşmanlık, ileriki nesillerin kurtuluş niteliklerini geliştirmede güçsüzlerdir.
\vs p084 8:5 İlerleyen göksel unsurların hepsi, dinlenceyi ve anımsama yöneticilerin hizmetini memnuniyetle deneyimlemektedirler. Sağlıklı eğlenceyi elde etmek ve neşelendirici oyuna katılmak için tüm çabalar güvenilirdir; tek düzeliğin sıkıcılığını engelleyen canlandırıcı uyku, dinlence, boş zaman etkinlikleri ve tüm hobiler değerlidir. Rekabete dayalı oyunlar, öykücülük ve iyi yemeği tatma bile birey tatmininin türleri olarak hizmet verebilir. (Yemeğinizin tadına varmak için tuzu kullandığınızda, yaklaşık bir milyon yıl boyunca insanın tuzu sadece yiyeceğini küle batırarak elde ettiğini bir düşünün.)
\vs p084 8:6 Bırakınız insan kendisini neşelendirsin; bırakınız insan ırkı bin bir şekilde hazzı yakalasın; bırakınız evrimsel insan türü, yukarı doğru uzun biyolojik mücadelenin meyveleri olarak yasal birey tatminin tüm türlerini keşfetsin. İnsan, bugünkü keyiflerinin ve hazlarının bazılarını fazlasıyla hak etmiştir. Ancak, nihai sonun amacına iyi bakın! Hazlar, bireyin kendisini idare edişinin kurumu olan mülkiyete zarar vermede başarılı olursa gerçekten de intiharsal konuma gelirler; ve birey tatminler --- insanın en üst düzeydeki evrimsel kazanımı ve medeniyetin tek kurtuluş ümidi olarak --- aile yaşamının çöküşü halindeki evliliğin yıkımını ve ev kurumunun yok oluşunu getirdiklerinde ölümcül bir bedele sahip olurlar.
\vs p084 8:7 [Urantia üzerinde konumlanan bir Yüksek Melek Önderi tarafından sunulmuştur.]
