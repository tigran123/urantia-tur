\upaper{94}{Batıdaki Melçizedek Öğretileri}
\vs p094 0:1 Salem dininin öncül öğretmenleri; kutsal lütfu elde etmenin tek bedeli olarak insanın tek bir evrensel Tanrı’ya beslediği inanca ve duyduğu güvene dair Maçiventa’nın müjdesini sürekli olarak duyuran bir biçimde, Afrika ve Avrasya’nın en uzak kabilelerine girmişti. Melçizedek’in İbrahim ile gerçekleştirdiği sözleşme, Salem ve diğer merkezlerden dışarı yayılan öncül tanıtım söyleminin tümü için bir yöntem teşkil etmekteydi. Urantia hiçbir zaman bu zamandan beri hiçbir zaman, Doğu Yarımküre’nin tamamı üzerinde Melçizedek’in bu öğretilerini taşımış olan bahse konu kutsal erkek ve kadınlardan daha fazla istekli ve daha fazla kararlı din yayıcılarına sahip olmamıştır. Bu din yayıcıları, birçok insan topluluğu ve ırkından toplanmıştı; ve onlar öğretilerini geniş ölçüde, yeni dini benimseyen yerel bireylerin aracılığı ile yaymışlardı. Onlar; Salem dininin yerlilerine öğretmiş oldukları yerler olan dünyanın farklı bölgelerinde eğitim merkezleri kurup, bunun sonrasında bu öğrencileri kendi insanları arasında öğretmenler olarak faaliyet göstermesi için görevlendirmişti.
\usection{1.\bibnobreakspace Vedik Hindistanı’nda Salem Öğretileri}
\vs p094 1:1 Melçizedek döneminde Hindistan, kuzeyden ve batıdan gelen Ari\hyp{}And istilacılarının siyasi ve dini egemenliği altına yakın bir zamanda gelmiş bir konumda bulunan çok uluslu bir ülkeydi. Bu dönemde yarımadanın sadece kuzey ve batı kısımları geniş bir biçimde Ari toplulukları tarafından kaplanmaktaydı. Bu yeni gelen Vedik unsurları kendileriyle birlikte birçok kabile ilahiyatını getirmişlerdi. Onların dini ibadet türleri; babanın bir din adamı ve annenin ise bir din kadını olarak faaliyet göstermeyi sürdürmesine ek olarak aile ocağının hala bir sunak olarak kullanılması bakımından öncül And atalarının törensel uygulamalarını yakın bir biçimde takip etti.
\vs p094 1:2 Vedik inancı bu dönemde; ibadetin genişleyen ayinselliği üzerinde kademeli olarak denetimi üstlenmekte olan, öğretmen\hyp{}din adamlarından meydana gelen Brahman toplumsal sınıfının yönetimi altında büyüme ve başkalaşım süreci içindeydi. Bir zamanlar otuz üç tane olan Ari ilahiyatının birleşimi, Salem din\hyp{}yayıcıları Hindistan’ın kuzeyine girdiğinde çoktan oluşum halindeydi.
\vs p094 1:3 Bu Ari toplulukların çoktanrılıcılığı; her kabilenin kendine ait saygı duyduğu tanrıya sahip olması biçiminde, kabilesel birimlere olan ayrılışları nedeniyle gerçekleşmiş öncül tek\hyp{}tanrıcılıklarının bir yozlaşmasını temsil etmişti. And Mezopotamyası’nın özgün tek\hyp{}tanrıcılığı ve kutsal\hyp{}üçleme\hyp{}inancının bu bozuluşu, İsa’dan önceki ikinci binyılın öncül çağlarında yeniden sentezlenme oluşumu içerisindeydi. Birçok tanrı, cennetin koruyucusu olan Dyaus pitarın kutsal üçleme önderliği altında bir tanrılar birliğine evirilmişti; atmosferin dalgalı koruyucusu olan İndra ve üç başlı ateş tanrısı olan Agni sırasıyla, dünyanın hâkimine ve öncül kutsal Üçleme kavramının sembolik düzeye inmiş simgesine evirilmişti.
\vs p094 1:4 Başkalarının tanrılarını reddetmeyen inanca ait belirli gelişmeler, evirilmiş bir tek\hyp{}tanrıcılığa ortam hazırlamaktaydı. En eski ilahiyat olan Agni sıklıkla, bütüncül tanrılar birliğinin baba\hyp{}önderine yükseltilmişti. Zaman zaman Prajapati olarak adlandırılan, zaman zaman Brahma olarak tanımlanan, ilahi\hyp{}baba kuramı; Brahman din\hyp{}adamlarının daha sonra Salem öğretmenleri ile verdikleri din\hyp{}kuramsal savaşta ortadan kaybolmuştu. \bibemph{Brahman}, Vedik tanrılar birliğinin tamamını etkinleştiren enerji\hyp{}kutsallığı olarak düşünülmüştü.
\vs p094 1:5 Salem din\hyp{}yayıcıları, cennetin En Yüksek Unsuru olan tek olan Melçizedek’in Tanrısı’nı duyurmuşlardı. Bu tasvir, tüm tanrıların kökeni olan Yaratıcı\hyp{}Brahma’ya dair ortaya çıkan kavramsallaşmayla bütünüyle uyumsuz bir konumda bulunmamaktaydı; ancak Salem inanışı ayinsel değildi, ve bu nedenle Brahman din adamlığının doğmalarına, geleneklerine ve öğretilerine doğrudan bir biçimde tezatlık oluşturdu. Brahman din adamları ayinsel adetler ve fedasal törenler dışında, Tanrı’nın lütfunu kazanma biçiminde, inanç ile kurtuluşun elde edilmesine dair Salem öğretisini hiçbir zaman kabul etmeyecekti.
\vs p094 1:6 Tanrı’ya olan güvene ve inanç yoluyla gelen kurtuluşa dair Melçizedek müjdesinin reddi, Hindistan için hayati bir dönüm noktasını simgelemişti. Salem din yayıcıları, ilkçağ Vedik tanrılarının tümüne olan inancın kaybolmasına fazlasıyla katkıda bulunmuş bir konumdaydı; ancak Vedizm’in din adamları olarak önderler, tek tanrı ve bir tek inanca dair Melçizedek öğretisini kabul etmeyi reddetmişti.
\vs p094 1:7 Brahman din adamları, Salem öğretmenleri ile başa çıkabilmek için dönemlerinin kutsal yazıtlarını bir araya topladı; ve bu derleme, daha sonra gözden geçirildiği gibi, kutsal kitapların en eskilerinden birisi olarak Rig\hyp{}Veda isminde çağdaş zamanlara kadar gelmiştir. Brahman din adamları, bu dönemlerin insanları üzerinde ibadet ve feda ayinlerini belirginleştirmeyi, şekillendirmeyi ve kesin bir temele oturtmayı amaçladıkça Veda yazıtlarının ikinci, üçüncüsü ve dördüncüsü birbirini takip etti. En iyi özellikleri göz önünde bulundurulduğunda bu yazılar, kavramdaki güzellik ve algıdaki gerçeklik bakımından benzer bir nitelikteki herhangi bir kutsal kitabın eşitidir. Ancak bu üstün din; güney Hindistan’ın hurafeleri, inanışları ve ayinlerinin binlercesi tarafından kirlenmiş hale geldiği için, fani insan tarafından şu ana kadar geliştirilen en alacalı sistem haline ilerleyen bir biçimde başkalaştı. Veda yazıtları üzerinde gerçekleştirecek bir inceleme, İlahiyat’a dair şimdiye kadar düşünülmüş en yüksek ve en alt düzeydeki kavramlarının bazılarını açığa çıkaracaktır.
\usection{2.\bibnobreakspace Brahmanlık}
\vs p094 2:1 Salem din yayıcıları Dravid Deccan yaylasına doğru güneye doğru giriş yaptıklarında, ikincil Sang topluluklarının artış gösteren bir dalgası karşısında ırksal kimliklerini kaybetmeyi engellemek için oluşturulmuş Ari unsurlarının düzeni olarak, yoğunlaşan bir toplumsal sınıf düzeniyle karşılaştılar. Brahan din adamlarının sınıfı bu sistemin tam da özü olduğu için, toplumsal düzen Salem öğretmenlerinin ilerleyişini fazlasıyla geriletmişti. Bu sınıf sistemi Ari ırkını kurtarmada başarısız olmuştu, ancak Brahmanların devamlılığını sağlamayı başarmıştı; onlar bunun karşılığında bu güne kadar, Hindistan’da dini egemenliklerini korumuşlardır.
\vs p094 2:2 Ve bu aşamada, daha yüksek gerçekliğin reddiyle gelen Vedizm’in zayıflamasıyla birlikte, Ari unsurların inanışı Deccan yaylasından gelen artış halindeki saldırılara maruz kalmıştı. Irksal yok oluşun ve dinsel tahribatın dalgasıyla baş etmenin çaresiz bir çabası içerisinde Brahman sınıfı kendisini herkesin üstünde bir yere çıkarmayı amaçladı. Onlar; ilahiyat için feda vermenin tek başına kesin bir biçimde etkili olduğunu, kudretiyle her zaman boyun eğdirici olduğunu öğrettiler. Onlar, evreninin iki temel kutsal kaynağından birinin ilahi Brahman, ötekisinin ise Brahman din adamlığı olduğunu duyurdu. Başka hiçbir Urantia topluluğu içinde din adamları, tanrılarına gösterilmesi beklenen onuru kendilerine indiren bir biçimde, tanrılarının bile üzerine yükseltmeye kalkışmadı. Ancak onlar bu haddinden fazla kendine güvenen söylemlerinde o kadar abes bir biçimde ileri gitmişlerdi ki, çevredeki ve daha az gelişmiş medeniyetlerden yağan alçaltıcı inançlar karşısında dengesiz duran bu bütüncül sistem çökmüştü. Geniş Vedik din adamlığının kendisi, bencil ve akılsız cüretkârlığın tüm Hindistan’a getirmiş olduğu eylemsizlik ve karamsarlığın kara seli altında çırpınıp battı.
\vs p094 2:3 Birey üzerinde haddinden fazla gerçekleştirilen yoğunlaşma kesin bir biçimde; insan, hayvan veya otlar olarak birbirini takip eden yeniden dünyaya gelişlerin sonu gelmez bir çevrimi içinde bireyin evrim\hyp{}dışı devamlılığına dair bir korkuya sebebiyet verdi. Ve ortaya çıkar haldeki bir tanrı inancına bağlı hale gelebilecek bu bozucu etkiye sahip inançların içinde hiçbiri, Dravid Deccan yaylasından gelmiş olan --- ruhların yeniden dünyaya geliş savı olarak --- ruhun göçüşüne dair bu inanç kadar körletici değildi. Tekrar eden ruh göçlerinin bahse konu bıktırıcı ve tekdüze çevrimine dair beslenen bu inanç; bir zamanlar öncül Vedik inancının bir parçası olmuş olan, çabalayan fanilerin sahip oldukları uzunca bir süredir sevgiyle beslemiş oldukları ölümde kurtuluş ve ruhsal ilerlemeyi bulma ümidini onlardan çalmıştı.
\vs p094 2:4 Bu felsefi olarak güçsüzleştirici öğretiyi yakın bir zaman içerisinde; tüm yaratımı içine alan üstün\hyp{}ruh olarak Brahman ile birlikte mutlak bütünlüğün evrensel huzuru ve barışına girerek bireyin edebi kaçışına dair inancın yaratımı takip etmiştir. Fani beklentisi ve insanın gelecek arzusu etkin bir biçimde zorla alı konulmuş ve neredeyse tamamen yok edilmişti. İki bin yıldan daha faza bir süredir Hindistan’ın daha iyi akılları arzunun her türünden kaçmayı amaçlamakta olup, böylece, ruhsal ümitsizliğin zincirlerlerine birçok Hindu insan topluluğunun ruhunu neredeyse mahkûm eden daha sonraki bu inanç ve öğretilere giriş için kapı ardına kadar açılmıştı.
\vs p094 2:5 Katı toplumsal tabakalaşma tek başına, Ari dini\hyp{}kültürel düzenin devamlılığını sağlayamadı; ve Deccan yaylasının alt düzey dinleri doğuya nüfuz ettiğinde, orada çaresizlik ve ümitsizliğin bir çağı gelişti. Bu karanlık günler boyunca yaşamın anlamsızlığına dair inanış ortaya çıkmış olup, bu dönemden itibaren varlığını sürdüregelmiştir. Bu yeni inanışların çoğu; geçmişte erişilebilen bir şey olarak düşünülmüş kurtuluşun yalnızca insanın aracısız gerçekleşen kendi çabaları sonucunda gelebileceğini ifade eden bir şekilde, aleni olarak tanrıtanımazdı. Ancak bütün bu talihsiz felsefenin büyük bir kısmı içinde, Melçizedek ve hatta Âdemsel öğretilerin bozulmuş kalıntıları bulunabilir.
\vs p094 2:6 Bu dönemler, Brahmana ve Upanişad kutsal kitapları olarak Hindu inancının daha sonraki yazıtlarının bir araya getirilme zamanlarıydı. Tek bir Tanrı ile yaşanılan kişisel inanç deneyimi vasıtasıyla gerçekleştirilen bireysel dinin öğretilerini reddetmiş, Deccan yaylasından gelen alçaltıcı ve güçsüzleştirici inanç ve öğretilerin seliyle kirlenmiş olarak, insan nitelikli tanrıları ve yeniden dünyaya gelişleriyle Brahman din adamlığı, bu bozucu inanışlara karşı şiddet dolu bir tepki deneyimledi; orada, \bibemph{doğru gerçekliği} arama ve bulmaya dair kesin bir çaba vardı. Brahmanlar, ilahiyatın Hint kavramını insani niteliklerden arınmaya girişti; ancak bunu yaparlarken, Tanrı’nın kişilik dışı hale getirilmesi biçiminde feci bir hataya düştüler; ve onlar, Cennet Yaratıcısı’na dair yüce ve ruhsal bir nihai düşünce ile değil, her şeyi içine alan bir Mutlak’a dair uzak ve metafizik bir düşünceyle ortaya çıktılar.
\vs p094 2:7 Mevcudiyetlerini korumadaki çabalarında Brahmanlar, Melçizedek’in tek Tanrısı’nı reddetmiş bir konumda bulunmaktalardı; ve bu aşamada onlar kendilerini, Hindistan’ın ruhsal yaşamını bu talihsiz günden yirminci yüzyıla kadar çaresiz ve halsiz bırakmış kişilik\hyp{}dışı ve herhangi bir güçten yoksun \bibemph{nesnel} o olan, belirsiz ve aldatıcı nitelikteki felsefi benlik biçimindeki Brahman varsayımıyla bulmuşlardı.
\vs p094 2:8 Upanişad yazıtlarının kaleme alınışı dönemlerinde Hindistan’da Budizm doğmuştu. Ancak onun bin yıllık başarısına rağmen, daha sonraki Hinduizm ile yarışamadı; daha yüksek bir ahlaka sahip olmasına rağmen onun öncül Tanrı tasviri, daha az ve daha kişisel ilahiyatları sunan Hinduizm’in sahip olduklarından daha da kısmi bir biçimde tanımlanmıştı. Budizm nihai olarak, evrenin en üstün Tanrı’sı olarak oldukça belirgin Allah kavramıyla birlikte savaşçı bir İslam’ın saldırısı karşısında kuzey Hindistan’da tutunamadı.
\usection{3.\bibnobreakspace Brahmansal Felsefe}
\vs p094 3:1 Brahmanizm’in en yüksek fazı neredeyse hiçbir biçimde din niteliğinde bulunmamış olsa da, fani insanın felsefe ve metafiziğin nüfuz alanına yaptığı en soylu girişlerden bir tanesiydi. Nihai gerçekliğin keşfine koyulan bir biçimde Hint aklı, dinin temel nitelikteki şu çifte kavramsallığı dışında din kuramının neredeyse her fazı hakkında fikir yürütene kadar durmamıştı: tüm evren yaratılmışlarının Kâinatsal Yaratıcısı’nın mevcudiyetine ek olarak kusursuz, hatta kendisi gibi olmalarını salık veren ebedi Yaratıcı’ya erişmeyi amaçlarlarken bahse konu bu yaratılmışların evreni içinde yükseliş halindeki deneyim gerçeği.
\vs p094 3:2 Brahman kavramı içinde bu dönemin akılları gerçekten de, her şeyi kapsayan belli bir Mutlak’a dair düşünceyi kavramışlardı; çünkü bu düşünce eş zamanlı olarak, yaratıcı enerji ve kâinatsal tepki olarak tanımlanmıştı. Brahman, yalnızca, tüm sınırlı niteliklerin birbirini takip eden reddi tarafından kavranılmaya yetkin bir biçimde tüm tanımların ötesinde olan bir konumda düşünülmüştü. O kesinlikle, bir mutlaklığa, hatta bir sınırsız, varlığa olan inanıştı; ancak bu kavram fazlasıyla, kişilik niteliklerden yoksun olup bu nedenle bireysel dindarlar tarafından deneyimlenemeyen bir öze sahipti.
\vs p094 3:3 Brahman\hyp{}Narayana; durağan mevcudiyette bulunan ve ebediyetin tümü boyunca potansiyel bir niteliğe sahip Kâinatsal Benlik olarak potansiyel kâinatın başat yaratıcı gücü şeklinde, sınırsız Nesnel O biçiminde Mutlak olarak düşünülmüştü. Bu dönemin filozofları Brahman’ı yaratılmış ve evrimleşen varlıklar tarafından erişilebilir bir kişilik niteliğinde ilişkilendirici ve yaratıcı olarak düşünebilen bir biçimde, ilahiyat kavramsallaşmasında bir sonraki aşamaya geçebilmiş olsalardı, bu türden bir öğreti böyle bir durumda Urantia üzerinde İlahiyat’ın en gelişmiş tasviri haline gelebilirdi; çünkü o, bütüncül ilahiyat faaliyetinin ilk beş düzeyini kapsayacak olup, geriye kalan ikisini de muhtemel bir biçimde tahayyül edebilen bir niteliğe sahip olacaktı.
\vs p094 3:4 Belirli fazlar içinde, tüm yaratılmış mevcudiyetinin toplamına ait bütünlük olarak Kâinat’ın Tek Üstün\hyp{}ruh kavramı Hintli filozofları Yüce Varlık’ın gerçekliğine oldukça yaklaştırdı; ancak bu gerçeklik kendilerine hiçbir şey sunmadı, çünkü onlar, Brahman\hyp{}Narayana’ya dair din\hyp{}kuramsal tek\hyp{}tanrı hedeflerinin erişilmesinde makul veya diğer bir değişle mantıklı bir kişisel yaklaşıma evirilmede başarısız oldu.
\vs p094 3:5 Nedensel ilişkiler ağının devamlılığına dayanan karma ilkesi, yine, Yücelik’in İlahiyat mevcudiyetindeki tüm zaman\hyp{}mekân eylemlerinin sonuçsal sentezine dayanan gerçekliğe oldukça yakındır; ancak bu düşünce hiçbir zaman, dindar birey tarafından gerçekleştirebilecek İlahiyat’ın eş\hyp{}güdümsel kişilik erişimini sunmamıştı; o yalnızca, tüm kişiliğin nihai olarak Kâinatın Üstün\hyp{}ruhu tarafından içine alınacağını sunmuştu.
\vs p094 3:6 Brahmanizm’in felsefesi aynı zamanda, Düşünce Düzenleyiciler’in ikametinin gerçekleşmesine oldukça yaklaşmıştı; o yalnızca, gerçekliğin yanlış kavranılışıyla ondan sapmış bir konuma gelmişti. Ruhun Brahman’ın ikameti olduğuna dair öğreti, bu kavram Kâinatsal Tek’in bahse konu ikametinden başka hiçbir insan kişiliğinin bulunmadığına dair inanç tarafından tamamiyle bozulmamış olsaydı, gelişmiş bir dinin zeminini hazırlamış olacaktı.
\vs p094 3:7 Birey\hyp{}ruhu ile Üstün\hyp{}ruhun birleşmesine dair inanç içinde Hindistan’ın din kuramcıları; insanın iradesi ile Tanrı’nın iradesinin bütünlüğünden doğan herhangi bir şey olarak, yeni ve benzersiz bir biçimde insan niteliğindeki bir şeyin varlığını devam ettirişini sunmada başarısız oldular. Ruhun Brahman’a olan geri dönüşüne dair öğreti, Kâinatın Yaratıcısı’nın bağrına olan Düzenleyici’nin dönüş gerçekliğine yakın bir biçimde benzerdir; ancak orada, fani kişiliğin morontia eşleniği biçiminde, Düzenleyici’den farklı olarak aynı zamanda varlığını sürdüren bir şey bulunmaktadır. Ve bu hayati kavram, Brahmansal felsefe içinde çok temel bir biçimde yoksundu.
\vs p094 3:8 Brahmansal felsefe evren ile ilgili birçok doğruyu yakın bir biçimde tahmin etmiş olup, sayısız kâinatsal gerçekliğe yaklaşmıştır; ancak o, mutlak, aşkın ve sınırlı gibi gerçekliğin çeşitli düzeylerini birbirinden ayırmada başarısız olma hatasına haddinden fazla bir biçimde düşmüştür. Bu felsefe, mutlak düzeyde sınırlı\hyp{}aldatıcı olan bir şeyin sınırlı düzeyde kesin bir biçimde gerçek olabileceğini hesaba katmada başarısız olmuştur. Ve o aynı zamanda; evrimsel yaratılmışın Tanrı ile olan sınırlı deneyiminden başlayarak, Cennet Yaratıcısı ile birlikte Ebedi Evlat’ın sınırsız deneyimine kadar tüm düzeylerde kişisel olarak ulaşılabilir nitelikteki Kâinatın Yaratıcısı’nın temel kişiliğini hiçbir biçimde tanımamıştır.
\usection{4.\bibnobreakspace Hindu Dini}
\vs p094 4:1 Hindistan’da geçen çağlar ile birlikte toplum büyük bir oranda, Melçizedek din\hyp{}yayıcılarının öğretileri ile değişikliğe uğratılmış ve daha sonraki Brahman din\hyp{}adamlığı tarafından kemikleştirilmiş Vedalar’ın ilkçağ ayinlerine geri dönmüştü. Dünya dinlerinin en eski ve en çok uluslusu olan bu din; Budizm ve Jainizm’e ek olarak Muhammed takipçilerinin dininin ve Hıristiyanlık’ın daha sonra ortaya çıkan etkilerine karşılık veren bir biçimde daha ileri değişikliklerden geçti. Ancak İsa’nın öğretilerinin ulaştığı zaman onlar çoktan, “beyaz insanın dinine” ait olamayacak kadar Doğululaşmışlardı; böylelikle bu öğretiler Hindu aklına garip ve yabancı geldi.
\vs p094 4:2 Mevcut an içerisinde Hindi din kuramı, ilahiyat ve kutsallığın dört azalan düzeyini tasvir etmektedir:
\vs p094 4:3 1.\bibnobreakspace \bibemph{Brahman}, Mutlak, Sınırsız Tek, Nesnel O.
\vs p094 4:4 2.\bibnobreakspace \bibemph{Trimurti}, Hinduizm’in yüce kutsal üçlemesi. Bu birliktelik içinde ilk üye olarak \bibemph{Brahman} --- sınırsız niteliğindeki --- Brahman’dan kendi kendisini yaratmış olarak düşünülmektedir. Her şeyin içinde olduğu varsayılan türe ait Sınırsız Tek ile yakın ilişkilendirilimi bulunmasaydı Brahma, Kâinatın Yaratıcısı’nın bir kavramsallaşmasının temelini oluşturabilirdi. Brahma aynı zamanda kader ile ilişkilendirilmektedir.
\vs p094 4:5 Şiva ve Vişnu olarak ibadetin ikinci ve üçüncü üyeleri, İsa’dan sonraki ilk bin yıl içinde doğmuştu. \bibemph{Şiva} yaşam ve ölümün koruyucusu, doğurganlığın tanrısı ve yok edişin hâkimidir. \bibemph{Vişnu}, insan türü içinde dönemsel olarak vücuda bürünüşüne dair inanç nedeniyle çok fazla bir biçimde sevilmektedir. Bu nedenle Vişnu, Hintliler’in düşüncelerinde gerçek ve yaşayan hale gelmektedir. Şiva ve Vişnu’nun her biri bazıları tarafından, her şeyin üstünde yüce olan konumda değerlendirilmektedir.
\vs p094 4:6 3.\bibnobreakspace \bibemph{Vedik ve Vedik\hyp{}sonrası ilahiyatları}. Agni, İndra ve Soma gibi Ari unsurlarının ilkçağ tanrılarının birçoğu, Trimurti’nin üç üyesi karşısında ikincil konumda varlıklarını sürdürmeye devam etti. Çok fazla sayıdaki ilave tanrı Vedik Hindistanı’nın öncül günlerinden beri doğuşlarını sürdürmüş olup, bunlar da aynı zamanda Hindu tanrılar birliğine eklemlendi.
\vs p094 4:7 4.\bibnobreakspace \bibemph{İnsansı\hyp{}tanrılar}: üstün\hyp{}insanlar, insansı\hyp{}tanrılar, kahramanlar, ecinniler, hayaletler, kötü ruhaniyetler, periler, canavarlar, gulyabaniler ve daha sonraki dönemin inançlarına ait azizler.
\vs p094 4:8 Her ne kadar Hinduizm Hint insanlarına hayat vermede uzunca bir süredir başarısız olmuşsa da, aynı zamanda genellikle hoşgörülü bir din olmuştur. Onun büyük gücü, kendisini ispatlamış bir biçimde Urantia üzerinde ortaya çıkan en uyumlu, başkalaşabilen din olduğu gerçeğinde yatmaktadır. O neredeyse sınırsız bir değişime yetkin olup, ussal Brahman’ın yüksek ve yarı\hyp{}tek\hyp{}tanrısal düşüncelerinden cahil inananlarının alt düzeydeki ve aşağı itilmiş sınıflarının sahip olduğu baştan aşağı kötü putlaştırışları ve ilkel inanç uygulamalarına kadar esnek uyumun olağandışı bir ölçeğine sahiptir.
\vs p094 4:9 Hinduizm, özünde Hindistan’ın temel toplumsal dokusunun tamamlayıcı bir parçası olduğu için varlığını sürdürmüştür. Rahatsız edilebilecek veya zarar verilebilecek hiçbir büyük ast\hyp{}üst ilişkisine sahip değildir; o, insanların yaşam biçimlerinin içine işlemiştir. Hinduizm; değişen şartlar karşısında tüm diğer inanışların üstünde yer alan bir uyuma sahip olup, birçok diğer dine karşı uyum sağlamanın hoşgörülü bir tutumunu sergilemektedir; öyle ki, Gotama Buda ve hatta İsa’nın kendisi Vişnu’nun bedene bürünmüş halleri olarak duyurulmuşlardır.
\vs p094 4:10 Bugün Hindistan’da --- sevgi dolu yardım ve toplumsal hizmet içinde kişisel olarak gerçekleştirilen, Tanrı’nın Yaratıcılığı, evlatlık ve bunun sonucu olarak tüm insanların kardeşliği biçiminde --- İsasal müjdenin tasvirine büyük bir ihtiyaç duyulmaktadır. Hindistan’da bahse konu felsefi düzen bulunmakta olup, bu inanış yapısı mevcuttur; tek ihtiyaç duyulan şey, Mikâil’in yaşam bahşedilişini bir beyaz insan dini yapma eğilimine sahip Doğulu dogma ve inanışlardan daha önce koparılmış olan, İnsan’ın Evladı’nın özgün müjdesi içinde tasvir edilen dinamik sevginin hayat verici kıvılcımıdır.
\usection{5.\bibnobreakspace Çin’de Gerçeklik İçin Verilen Mücadele}
\vs p094 5:1 Salem din\hyp{}yayıcıları En Yüksek Tanrı’ya dair inanışı ve inançla gerçekleşen kurtuluşu yayarak Asya boyunca geçerlerken, kat ettikleri çeşitli ülkelerin felsefesi ve dini düşüncesinin büyük bir kısmını özümsedi. Ancak Melçizedek tarafından görevlendirilen öğretmenler ve onların varisleri, görevlerinden sapmadılar; onlar, Avrasya kıtasının tüm insan topluluklarının içine girmiş olup, İsa’dan önceki ikinci bin yılın ortasında Çin’e vardılar. See Fuch’da bin yıldan daha fazla bir süre boyunca Salem unsurları, yönetim merkezlerini idare ettiler; burada onlar, sarı ırkın tüm kolları boyunca eğitimde bulunmuş Çinli öğretmenleri hazırladılar.
\vs p094 5:2 Bugün sahip olduğu ismi taşıyandan oldukça farklı bir din olarak Çin’de Taoizm’in en öncül biçiminin doğuşu bu öğretinin doğrudan bir sonucuydu. Öncül veya diğer bir değişle kökensel\hyp{}Taoizm, şu etkenlerin bir bileşeniydi:
\vs p094 5:3 1.\bibnobreakspace Cennet’in Tanrısı olan Şangti’nin kavramı içinde varlığını sürdürmüş Singlanton’un hala varlığını sürdüren öğretileri. Singlangton’un döneminde Çin insanları, neredeyse tek\hyp{}tanrısal hale geldi; onlar ibadetlerini, evren yöneticisi olarak Cennet’in Ruhaniyeti ismiyle daha sonra bilinen Tek Gerçeklik üzerinde yoğunlaştırdılar. Ve, ilerleyen çağlarda her ne kadar birçok bağlı tanrı ve ruhaniyet dinlerine sızmış olsa da, İlahiyat’ın bu öncül kavramı hiçbir zaman bütünüyle kaybolmamıştı.
\vs p094 5:4 2.\bibnobreakspace İnsanın inancına karşılık olarak insanlığa lütfunu bahşeden bir En Yüksek Yaratan İlahiyatı’nın Salem dini. Ancak, Melçizedek din\hyp{}yayıcıları sarı ırkın yerleşkelerine girmiş olduğunda özgün iletilerinin, Maçiventa dönemindeki Salem’in yalın öğretilerinden ciddi bir ölçüde değişmiş olduğu fazlasıyla gerçekti.
\vs p094 5:5 3.\bibnobreakspace Tüm kötülükten kaçma arzusu ile birleşmiş Hint filozoflarının Brahman\hyp{}Mutlak kavramsallaşması. Salem dininin doğuya doğru olan yayılımında en büyük dışsal etki, --- Mutlak olan --- Brahman kavramsallaşmalarını Salem unsurlarının kurtuluşsal düşüncesine aşılayan Vedik inancının Hintli öğretmenleri tarafından yaratılmıştı.
\vs p094 5:6 Bu bileşik inanç, dini\hyp{}felsefi nitelikli düşünce içerisindeki temel bir etki olarak sarı ve kahverengi ırkların karaları boyunca yayılmıştı. Japonya’da bu kökensel\hyp{}Taoizm Şinto olarak bilinmekteydi; ve Filistin’in Salemi’nden çok uzaklardaki bu ülkede insan toplulukları, Tanrı’nın isminin insanlık tarafından unutulmaması için dünya üzerinde ikamet etmiş olan Maçiventa Melçizedeği’nin vücuda bürünmesini öğrenmişlerdi.
\vs p094 5:7 Çin’de tüm bu inançların hepsi daha sonra, atalara olan ibadetin sürekli büyüyen inancı ile bozulmuş ve daha da kötüleşmişti. Ancak Singlangton’un döneminden beri Çinliler, din\hyp{}adamları mesleğine karşı çaresiz köleliğe hiçbir zaman düşmemişti. Sarı ırk, barbar kölelikten düzenli medeniyete doğru kurtuluşunu gerçekleştiren ilk ırktı; çünkü onlar, diğer ırkların korkmuş olduğu gibi ölülerin hayaletlerinden bile korkmayan bir biçimde tanrılara karşı beslenen sefil korkudan bir parça özgürlüğü elde eden ilk ırktı. Çin yenilgisiyle, din adamlarından öncül bağımsızlığının ötesine geçmede başarısız olması nedeniyle karşılaştı; o, atalara yapılan ibadet olarak neredeyse eşit düzeydeki vahim bir hataya düştü.
\vs p094 5:8 Ancak Salem unsurları boşu boşuna emek vermediler. Müjdelerinin temelleri üzerine altıncı yüzyılın büyük Çin filozofları öğretilerini inşa etti. Laozi ve Konfüçyüs dönemlerinin ahlaki atmosferi ve ruhsal düşünceleri, daha öncül bir çağın Salem din\hyp{}yayıcılarına ait öğretilerden gelişmişti
\usection{6.\bibnobreakspace Laozi ve Konfüçyüs}
\vs p094 6:1 Mikâil’in varışından yaklaşık altı yüz yıl önce, beden içinde ayrılmadan çok uzun bir süre önce Melçizedek; dünya üzerindeki öğretisinin saflığının, daha eski Urantia inanışlarının topluca karışımı tarafından aşırı bir biçimde tehlike altına girebileceğini düşündü. Bir süre boyunca, Mikâil’in bir müjdecisi olma görevinin başarısız olma tehlikesi altında bulunduğu göründü. Ve İsa’dan önce altıncı yüzyılda, ruhsal birimlerin olağandışı bir eş güdümü vasıtasıyla, ki bunların tamamı gezegensel yüksek\hyp{}denetimciler tarafından bile anlaşılamamaktadır, Urantia çok katmanlı dinsel gerçekliğin en olağan dışı sunumlarından birine şahit oldu. Birkaç insan öğretmenin aracılığı ile Salem müjdesi yeniden ifade edilip, yeniden canlandırıldı; ve bu haliyle bahse konu dönemlerde sunulduğu için, onların büyük bir kısmı bu yazının dönemine kadar varlığını sürdürmüştür.
\vs p094 6:2 Ruhsal ilerleyişin bu benzersiz çağı; medenileşmiş dünyanın tamamı üzerindeki büyük dini, ahlaki ve felsefi öğretmenler tarafından nitelenmektedir. Çin’de iki olağanüstü öğretmen Laozi ve Konfüçyüs’dü.
\vs p094 6:3 \bibemph{Laozi} Tao’yu tüm yaratımın Tek İlk Kökeni olarak duyurduğunda, onu doğrudan bir biçimde Salem’e ait tarihsel anlatımlardaki kavramlar üzerine kurmuştu. Lao, büyük ruhsal öngörüye sahip bir insandı. O insanın ebedi sonunun, “Yüce Tanrı ve Kâinatın Hükümdar’ı Tao ile olan sonsuza kadar sürecek bütünlük” olduğunu öğretti. Nihai kökene dair onun kavrayışı olabileceği en yüksek düzeyde sezgiye sahipti; çünkü o şunları yazmıştı: “Birliktelik, Mutlak Tao’dan doğmaktadır; ve Birliktelik’den kâinatsal Çiftelik ortaya çıkmaktadır; ve bu türden bir Çiftelik’den, Kutsal Üçleme mevcudiyetine kavuşmaktadır; ve Kutsal Üçleme, tüm gerçekliğin temel kökenidir.” “Gerçekliğin tümü her zaman, kâinatın potansiyelleri ve mevcudiyetleri arasındaki denge olmuştur; ve onlar ebedi bir biçimde kutsallığın ruhaniyeti tarafından ahenkli hale getirilmiştir.”
\vs p094 6:4 Laozi aynı zamanda, kötülüğe iyilikle karşılık verme öğretisinin en öncül sunumlarından bir tanesini gerçekleştirmişti: “İyilik iyiliği doğurur, ancak gerçekten iyi olan birisi için kötülük aynı zamanda iyiliği doğurur.”
\vs p094 6:5 O; yaratılmışın Yaratan’a olan dönüşünü öğretmiş olup, yaşamı kâinatsal potansiyellerden bir kişiliğin ortaya çıkışı olarak resmederken, ölüm bu yaratılmış kişilikten eve dönüş gibiydi. Onun gerçek inanç kavramı olağandışıydı; ve o da bu inancı “küçük bir çocuğun tutumuna” benzetti.
\vs p094 6:6 Tanrı’nın ebedi amacına dair onun anlayışı açıktı; çünkü o şunları söyledi: “Mutlak İlahiyat çabalamaz, bunun yerine o her zaman muzafferdir; o insan türünü zorlamaz, bunun yerine o her zaman onların gerçek arzularına karşılık vermede hazırdır; Tanrı’nın iradesi, sabırda ve onun dışavurumunun kaçınılmazlığında ebedidir.” Ve o gerçek dindar hakkında, vermenin almadan daha kutsanmış olduğuna dair gerçekliği ifade eden bir biçimde şunları söylemiştir: “İyi insan gerçekliği kendisi için elinde bulundurmayı amaçlamaz, bunun yerine bu zenginlikleri akranları üzerine bahşetmeye çabalar, çünkü bu gerçekliğin kendisini gerçekleştirişidir. Mutlak Tanrı’nın iradesi her zaman yarar sağlamaktadır, asla zarar vermemektedir; gerçek inananın amacı her zaman eylemde bulunmak olmalıdır, fakat hiçbir zaman zorlamak olmamalıdır.”
\vs p094 6:7 Lao’nun direnişte bulunmama öğretisi ve onun \bibemph{eylem} ve \bibemph{zorlama} arasında yaptığı ayrım daha sonra, “görme, yapma ve hiçbir şey düşünmeme” inançlarına doğru saptırılmış hale geldi. Ancak Lao, her ne kadar onun direnişte bulunmama sunumu Çin topluluklarının barışçıl eğilimlerinin daha da gelişiminde bir etken olmuş olsa da, hiçbir zaman bu türden hatayı öğretmedi.”
\vs p094 6:8 Ancak yirminci yüzyıl Urantiası'nın sahip olduğu yaygın Taoizm, algıladığı biçimde gerçekliği öğretmiş olan eski filozofun yüce düşünceleri ve kâinatsal kavramlarıyla çok ortak noktaya sahiptir; bu gerçeklik Mutlak Tanrı’ya olan inancın dünyayı yeniden yapacak olan kutsal enerjinin kökeni olduğu, ve bu inançla insanın Tao, Ebedi İlahiyat ve evrenlerin Yaratan Mutlak’ı ile birlikte ruhsal birlikteliğe yükseldiğiydi.
\vs p094 6:9 \bibemph{Konfüçyüs} (Kung Fu\hyp{}tze) altıncı yüzyıl Çin’inin daha genç bir çağdaşıydı. Konfüçyüs öğretilerini, sarı ırkın uzun tarihine ait daha iyi ahlaki gelenekler üzerine kurdu; ve o aynı zamanda, Salem din\hyp{}yayıcılarının hala varlığını sürdüren tarihi anlatımlarından bir parça etkilenmişti. Onun başlıca yaptığı şey, ilkçağ filozoflarının bilge sözlerinin derlemesi oluşmuştu. O, yaşam boyunca reddedilmiş bir öğretmendi; ancak yazıları ve öğretileri bu zamandan beri Çin ve Japonya’da büyük bir etki bırakmıştır. Konfüçyüs, ahlakı büyünün yerine getirerek şamanlar ile yeni bir rekabet alanı yaratmıştı. Ancak o bunu haddinden fazla bir başarıyla gerçekleştirdi; o \bibemph{düzenden} yeni bir put yaratmış olup, bu yazı döneminde bile Çin insanları tarafından hala derin bir biçimde hürmette bulunulan, atasal işleyiş karşısında bir saygı duygusunu oluşturdu.
\vs p094 6:10 Ahlakın Konfüçyüsçü duyurusu; dünyasal davranışın, cennetsel davranışın bozulmaya uğramış bir gölgesi olduğuna dair kurama dayanmıştı; geçici medeniyetin gerçek işleyişi, cennetin ebedi düzeninin birebir yansımasıydı. Konfüsyüsçülük içinde potansiyel Tanrı kavramı neredeyse tamamen, kâinatın işleyişi olarak Cennet’in Davranışı üzerine konumlanan vurguya bağlanmıştı.
\vs p094 6:11 Lao’nun öğretileri Doğu’daki bir kaçı dışında neredeyse tamamen kaybolmuştu; ancak Konfüçyüs’ün yazıları bu dönemden beri, Urantia unsurlarının neredeyse üçte birinin sahip olduğu kültürlerin ahlaki dokusunun temelini oluşturmuştur. Bu Konfüçyüs ilkeleri, geçmişin en iyi niteliklerinin devamlılığını sağlamış olsa da, geçmişte oldukça hürmet edilmiş kazanımları beraberinde getirmiş olan Çinli irdeleme ruhuna bir biçimde karşıt konumdaydı. Bu öğretilerin etkisi başarısız bir biçimde; hem Çin Şi Huang Ti’nin imparatorluksak çabaları hem de, sadece etiksel görev üzerine değil aynı zamanda Tanrı’nın derin sevgisi üzerine inşa edilmiş bir kardeşliği duyurmuş olan Mo Ti’nin öğretileri tarafından karşı konuldu. Mo Ti, yeni gerçekliğin tarihi arayışını yeniden yakmayı arzulamıştı; ancak onun öğretileri, Konfüçyüs’ün takipçilerinin sert karşıtlığı karşısında başarısız olmuştu.
\vs p094 6:12 Diğer birçok ruhsal ve ahlaki öğretmen gibi hem Konfüçyüs hem de Laozi nihai olarak, Taoist inancın çöküşü ve bozuluşu ile Hindistan’dan Budist din\hyp{}yayıcıların gelişi arasında gerçekleşen Çin’in bu ruhsal bakımdan karanlık çağları içerisinde takipçileri tarafından ilahlaştırıldı. Bu ruhsal bakımdan geri kalmış çağlar boyunca sarı ırkın dini; hepsinin, aydınlanmamış fani aklın korkularına olan geri dönüşü simgelediği bir biçimde içinin şeytanlar, ejderhalar ve kötü ruhaniyetler ile dolup taştığı acınası bir din kuramına doğru yozlaştı. Ve, bir zamanlar gelişmiş bir dine sahip olmasından dolayı insan toplumunun başını çekmiş olan Çin bu dönemde; yalnızca bireysel faninin değil aynı zamanda, zaman ve mekânın evrimsel bir gezegeni üzerinde kültür ve toplumun gelişimini niteleyen karmaşık ve katmanlaşmış medeniyetlerin de gerçek ilerleyişi için hayati derecede önemli olan Tanrı\hyp{}bilincinin gelişimine ait gerçek doğrultu üzerindeki ilerleyişte geçici başarısızlığı nedeniyle geride kaldı.
\usection{7.\bibnobreakspace Gotama Sidarta}
\vs p094 7:1 Çin’de Laozi ve Konfüçyüs’ün çağdaşı olarak, gerçekliğin bir diğer büyük öğretmeni Hindistan’da doğdu. Gotama Sidarta, İsa’dan önce altıncı yüzyılda kuzey Hindistan’ın Nepal bölgesinde dünyaya geldi. Onun takipçileri daha sonra kendisini, muazzam biçimde zengin bir idarecinin oğlu olarak resmettiler; ancak gerçekte o, güney Himalayalar’da küçük ve sapa bir dağ vadisi üzerinde istenmeden de olsa müsamaha gösterilerek idaresine izin verilmiş küçük bir kabile reisi hanedanının meşru mirasçısıydı.
\vs p094 7:2 Gotama, Yoga’nın altı yıllık faydasız uygulamasından sonra Budizm’in felsefesine doğru evirilmiş bu kuramları tasarlamıştı. Sidarta, büyüyen toplumsal tabaka düzenine karşı kararlı ancak yararsız bir savaşı gerçekleştirdi. Bu dönemin insanları için fazlasıyla çekici gelen bu genç tanrı\hyp{}elçisi prenste, yüce bir içtenlik ve benzersiz bir fedakârlık bulunmaktaydı. O, fiziksel rahatsızlık ve kişisel acıdan vasıtasıyla bireysel kurtuluşu arama uygulamasını yermişti.
\vs p094 7:3 Hindistan’ın kafa karışıklığı ve aşırı inanç uygulamaları arasında Gotama’nın daha mantıklı ve makul öğretileri, canlandırıcı bir rahatlama olarak geldi. O tanrıları, din\hyp{}adamlarını ve onların fedalarını kötümsedi; ancak o da, Kâinatın Tek Unsuru’nun \bibemph{kişiliğini} kavramada başarısız oldu. Bireysel insan ruhlarının mevcudiyetine inanmayarak Gotama, tabiî ki, ruhun göçüne olan gelenekselleşmiş inanca karşı gözü pek bir savaş verdi. O, koca evrende kendilerini rahat ve evinde hissetmeleri için insanları korkudan kurtarmak amacıyla soylu bir çabada bulundu; ancak o, --- Cennet olarak --- yükseliş halindeki fanilerin gerçek ve yüce evlerine götürecek yolu ve ebedi mevcudiyetin genişleyen hizmetini onlara göstermede başarısız oldu.
\vs p094 7:4 Gotama gerçek bir tanrı\hyp{}elçisiydi; ve keşiş Godad’ın tavsiyesini can kulağıyla dinleyip onu yerine getirseydi, Salem müjdesinin sunduğu inançla gelen kurtuluşun yeniden canlanışına ait ilhamla Hindistan’ın tümünü canlandırabilirdi. Godad, Melçizedek din\hyp{}yayıcılarının tarihsel anlatımlarını hiçbir zaman kaybetmemiş olan bir aileden gelmekteydi.
\vs p094 7:5 Benares’de Gotama okulunu kurdu; ve buradaki ikinci yılında Bautan isimli bir öğrencisi öğretmenine, İbrahim ile yapılmış Melçizedek sözleşmesi hakkında Salem din\hyp{}yayıcılarına ait tarihsel anlatımları aktardı; ve her ne kadar Sidarta, Kâinatın Yaratıcısı’na ait oldukça kesin bir kavrama sahip olmasa da, --- yalın inanış olan --- inanç vasıtasıyla erişilen kurtuluş üzerinde gelişmiş bir düşünceye ulaştı. O; “ister yüksek ister alçak mevkide olsun, tüm insanların kusursuz mutluluğa doğruluk ve adalete dair inanç ile ulaşabileceğini” takipçileri karşısında duyurup, öğrencilerini altmışarlı topluluklar halinde Hindistan’ın insanlarına “özgür kurtuluşun sevindirici haberlerini” duyurmaları için göndermeye başladı.
\vs p094 7:6 Gotama’nın eşi kocasının müjdesine inanmış olup, din\hyp{}kadınlarının bir düzeyinin kurucusu olmuştu. Onun oğlu Gotama’nın varisi haline gelmiş olup, bu inancı fazlasıyla geliştirdi; o inanç vasıtasıyla gelen kurtuluşa dair yeni düşünceyi kavramıştı, ancak ilerleyen yaşlarında sadece inançla elde edilen kutsal lütuf ile ilgili Salem müjdesinde bocaladı ve yaşlılığında son sözcükleri “kendi kurtuluşunuzu elde etmeye bakın” oldu.
\vs p094 7:7 En iyi olduğu aşamada duyurulduğunda; fedadan, işkenceden, ayinden ve din adamlarından uzak herkesi kapsayan kurtuluşa dair Gotama’nın müjdesi kendi dönemi için devrimsel ve hayrete düşürücü bir öğretiydi. Ve o şaşırtıcı bir biçimde, Salem müjdesinin bir yeniden canlanışına yaklaştı. O, çaresizlik çeken milyonlarca ruhun imdadına yetişti; ve, daha sonraki çağlarda gerçekleşen anlaşılamaz nitelikteki hoşnut olmayan bozuluşuna rağmen, hala milyonlarca insanın ümidi olmaya devam etmektedir.
\vs p094 7:8 Sidarta, ismini taşıyan çağdaş inançlar içinde varlığını sürdürebilmiş gerçeklerden çok daha fazlasını öğretti. Nasıralı İsa’nın öğretileri Hıristiyanlık için neyse, Gotama Sidarta’nın öğretileri Çağdaş Budizm için ondan daha az değildir.
\usection{8.\bibnobreakspace Budist İnanç}
\vs p094 8:1 Bir Budist olabilmek için bir kişi sadece, Sığınma’nın şu andını söyleyerek inancın toplum önündeki beyanında bulundu: “Ben Buda’ya sığınıyorum; ben Öğreti’ye sığınıyorum; ben Kardeşlik’e sığınıyorum.”
\vs p094 8:2 Budizm kökenini tarihsel bir kişiden aldı, bir mitten değil. Gotama’nın takipçileri kendisini, usta veya öğretmen anlamına gelen Sasta olarak çağırdı. Ne kendisi için ne de öğretileri için insan\hyp{}üstü niteliğinde hiçbir ifadede bulunmamış olsa da, onun takipçileri kendisini öncül bir biçimde \bibemph{aydınlanmış kişi}, Buda olarak çağırdı; daha sonra ise onlar kendisine, Sakyamuni Buda ismini verdi.
\vs p094 8:3 Gotama’nın özgün müjdesi şu dört soylu gerçeklik üzerine dayanmıştı
\vs p094 8:4 1.\bibnobreakspace Izdırabın soylu gerçekleri.
\vs p094 8:5 2.\bibnobreakspace Izdırabın kökenleri.
\vs p094 8:6 3.\bibnobreakspace Izdırabın yıkımı.
\vs p094 8:7 4.\bibnobreakspace Izdırabın yıkımına giden yol.
\vs p094 8:8 Izdırap ve ondan kaçışa dayanan savla yakından ilişkili olarak Sekizkatmanlı Yol’un felsefesi şuydu: doğru görüşler, geleceğe dair amaçlar, konuşma, davranış, geçim, çaba, farkındalık ve düşünmekti. Izdırapdan kaçınmak için çaba, arzu ve şefkatin tümünü yok etmeye girişmek Gotama’nın amacı değildi; onun öğretisi bunun yerine, geçici amaçlar ve maddi hedefler üzerine tüm umutların ve gelecek beklentilerin bağlanmasının faydasızlığını fani insana resmetmek için tasarlanmıştı. Birinin akranlarına duyacağı sevgiden kaçınması gerektiğinden çok, gerçek inananın bu dünyanın ilişkilerinin ötesinde aynı zamanda ebedi geleceğin gerçekliklerine bakması gerektiğini salık vermekteydi.
\vs p094 8:9 Gotama’nın duyurusuna ait ahlaki emirler sayıca şu beş tanesiydi.
\vs p094 8:10 1.\bibnobreakspace Öldürmemelisiniz.
\vs p094 8:11 2.\bibnobreakspace Çalmamalısınız.
\vs p094 8:12 3.\bibnobreakspace İffet sahibi olmalısınız.
\vs p094 8:13 4.\bibnobreakspace Yalan söylememelisiniz.
\vs p094 8:14 5.\bibnobreakspace Sarhoş edici alkollü içkileri içmemelisiniz.
\vs p094 8:15 Orada, inananlar için uygulaması tercihsel olan birkaç ek veya diğer bir değişle ikincil emir bulunmaktaydı.
\vs p094 8:16 Sidarta, insan kişiliğinin ölümsüzlüğüne neredeyse hiçbir biçimde inanmadı; onun felsefesi sadece, işlevsel devamlılığın bir türünü sağladı. O hiçbir zaman açık bir biçimde, Nirvana öğretisi içinde neyin bulunmuş olduğunu tanımlamadı. Fani mevcudiyeti içinde kuramsal olarak deneyimlenebileceği gerçeği, bütüncül bir yok oluş düzeyi olarak görülmediğine işaret etmişti. İçinde insanı maddi dünyaya bağlayan tüm zincirlerin kırılmış olduğu yüce bir aydınlanma ve göksel bir mutluluk durumu anlamına gelmişti; orada, fani yaşamın arzularından özgürlük ve sürekli deneyimlenen dönemsel yaşamın tüm tehlikelerinden kurtuluş bulunmaktaydı.
\vs p094 8:17 Gotama’nın özgün öğretilerine göre kurtuluş, kutsal yardımdan bağımsız olarak insan çabasıyla elde edilmektedir; orada inancı kurtarmaya veya insan\hyp{}üstü güçlere yapılan duaya yer bulunmamaktadır. Gotama, Hindistan hurafelerinin önemini azaltmak için insanları büyüsel kurtuluşun aleni söylemlerinden geri çevirmeye çabaladı. Ve bu çabada bulunurken varislerine, öğretilerini yanlış bir biçimde yorumlamaları ve kazanım için harcanan tüm insan emeğinin çirkin ve acı verici olduğunu duyurmaları için kapıyı ardına kadar araladı. Onun takipçileri; en yüksek mutluluğun değerli amaçların ussal ve hevesli arayışı ile ilişkili olduğunu, ve bu kazanımların bireyin kâinatsal düzeyde kendisini gerçekleştirişinde gerçek ilerleyişi meydana getirdiğini gözden kaçırdı.
\vs p094 8:18 Sidarta öğretisinin sunduğu büyük gerçeklik, mutlak adalete sahip bir evrene dair yapmış olduğu duyuruydu. O, fani insan tarafından bu zamana kadar yaratılmış en iyi tanrısız felsefeyi öğretti; o olası en yüksek insancılık olup, hurafe ve büyüsel ayinlere ek olarak hayaletler veya kötü emelli ruhlardan duyulan kurkunun tüm dayanaklarını en etkin bir biçimde ortadan kaldırmış felsefeydi.
\vs p094 8:19 Budizm’in özgün müjdesindeki büyük zayıflık, fedakâr toplumsal hizmetten meydana gelen bir dini üretmemiş olmasıydı. Budistik kardeşlik uzunca bir süre boyunca; inananların bir aidiyet birlikteliği değil, bunun yerine öğrenci öğretmenlerin bir cemiyetiydi. Gotama onların para almasını yasaklamış olup, böylece ast\hyp{}üst ilişkisine dayanan eğilimlerin büyümesini önlemeye çalıştı. Gotama’nın kendisi oldukça toplumsaldı; gerçekten de onun yaşamı, sunduğu şeylerden çok daha fazlasıydı.
\usection{9.\bibnobreakspace Budizm’in Yayılışı}
\vs p094 9:1 Budizm, aydınlanmış kişi olan Buda’ya olan inanç vasıtasıyla kurtuluşu sunduğu için gelişme gösterdi. O, doğu Asya boyunca bulunabilecek herhangi bir diğer dini sisteme kıyasla Melçizedek gerçekliklerini daha fazla temsil eder nitelikteydi. Ancak Budizm; Mısır’daki Akhenaton ile birlikte Melçizedek ve Mikâil dönemi arasındaki en dikkate değer toplum yöneticilerinden bir tanesi olan, alt toplumsal tabakadan gelmiş Asoka hükümdarı tarafından bireysel korunma içinde desteklenene kadar yaygın bir din haline gelmemişti. Asoka büyük bir Hint imparatorluğunu Budist din\hyp{}yayıcılarının düzenli tanıtımlarıyla inşa etti. Yirmi beş yıllın bir dönem boyunca o, tüm bilinen dünyanın en uzak sınırlarına on yedi bin din\hyp{}yayıcısından fazlasını eğitip göstermişti. Bir nesil içinde o Budizm’i, dünyanın yarısının baskın dini haline getirdi. Budizm yakın bir zaman içerisinde; Tibet, Keşmir, Sri Lanka, Burma, Cava Adası, Siam, Kore, Çin ve Japonya’da köklü hale geldi. Ve o genel bütünlüğü bakımından, yerini aldığı veya geliştirdiği dinlerden çok daha fazla üstün bir dindi.
\vs p094 9:2 Budizm’in Hindistan’daki anavatanından Asya’nın tümüne yayılışı, içten dindarların ruhsal bağlılığı ve din\hyp{}yayıcı kararlılığının heyecan verici hikâyelerinden bir tanesidir. Gotama müjdesinin öğretmenleri yalnızca kara\hyp{}üstündeki kervan yollarının tehlikelerine göğüs germediler, inançlarının iletisini tüm insan topluluklarına getiren bir biçimde Asya kıtası üzerindeki görevlerini amaçlarken Çin Deniz sularının tehlikeleriyle karşılaştılar. Ancak bu Budizm artık Gotama’nın yalın öğretisi değildi; o, kendisini bir tanrı yapan mucizeleştirilmiş müjdeydi. Ve Budizm Hindistan’da bulunan yükseltilerdeki evinden daha uzağa yayıldıkça, Gotama’nın öğretilerine daha az benzer hale geldi; ve o daha çok yerini aldığı dinler haline geldi, onlara doğru evirildi.
\vs p094 9:3 Budizm, daha sonra; Çin’deki Taoizm’den, Japonya’daki Şinto’dan ve Tibet’deki Hristiyanlık'dan çok fazla bir biçimde etkilenmişti. Bin yıllık bir süre zarfından sonra Hindistan’daki Budizm tamamiyle bozulup, miladını doldurdu. Bu Budizm Brahmanlaşan hale gelip, daha sonra perişan bir halde kendisini İslam’a teslim etti; buna ek olarak Doğu’nun geri kalan kısmında o, Gotama Sidarta’nın hiçbir zaman tanımayacağı bir ayine doğru evirilen biçimde yozlaştı.
\vs p094 9:4 Güneydeki Sidarta öğretilerinin köktenci katı düşüncesi Sri Lanka, Burma ve Hint\hyp{}Çin yarımadasında varlığını sürdürdü. Bahse konu bu düşünce, öncül veya diğer bir değişle toplumsal olmayan öğretiye bağlı kalan Budizm’in Hinayana farklılaşmasıdır.
\vs p094 9:5 Ancak Hindistan’daki çöküşünden bile önce, Gotama’nın takipçilerinin Çin ve kuzey Hint toplulukları; Hinayana veya diğer bir değişle “Küçük Yol” düşüncesine bağlı kalan güneydeki safçılara tezat bir biçimde, kurtuluş için Mahayana’nın “Büyük Yol” öğretisinin geliştirilmesine başlamış halde bulunmaktalardı. Ve bu Mahayanacılar, Budist öğreti içinde içkin olan toplumsal sınırlamaları gevşetmişlerdi; ve bu dönemden beri Budizm’in bu kuzey farklılaşması, Çin ve Japonya’da evirilmeye devam etti.
\vs p094 9:6 Budizm bugünün yaşayan, büyüyen bir dinidir; çünkü o, takipçilerinin en yüksek ahlaki değerlerinden çoğunu muhafaza etmede başarılı olmaktadır. O sakinliği ve bireyin\hyp{}öz\hyp{}denetimini teşvik etmekte olup, dinginliği ve mutluluğu arttırmaktadır; buna ek olarak o, üzüntü ve kederi engellemeye fazlasıyla katkı sağlamaktadır. Bu felsefeye inananlar, inanmayan birçoklarından daha iyi yaşamlarını daha iyi yaşamaktadırlar.
\usection{10.\bibnobreakspace Tibet’deki Din}
\vs p094 10:1 Tibet’de; Budizm, Hinduizm, Taoizm ve Hristiyanlık ile birleşmiş Melçizedek öğretilerinin en tuhaf birlikteliği bulunabilir. Budist din\hyp{}yayıcıları Tibet’e girdiğinde onlar, öncül Hıristiyan din\hyp{}yayıcıların Avrupa’nın kuzey kabileleri arasında bulmuş olduklarına çok benzer ilkel yabanlığın bir düzeyi ile karşılaşmışlardı.
\vs p094 10:2 Bu basit akıllı Tibetliler, ilkçağ büyü ve uğurlarından tamamiyle vazgeçmeyeceklerdi. Bugünün Tibet ayinlerine ait dini törenlerin irdelenmesi; çanları, zikirleri, kokuları, toplu ilahileri, tespihleri, imgeleri, uğurlu eşyaları, resimleri, kutsal suları, görkemli cüppeleri ve detaylı koroları içine alan ayrıntılı bir ayini uygulamakta olan kazınmış başlı din\hyp{}adamlar topluluğunun haddinden fazla büyümüş bir kardeşliğini ortaya çıkarır. Onlar; çok katı dogmalara ve esnekliğini kaybetmiş inanç ilkelerine, mistik törenlere ve özel oruç türlerine sahiptirler. Onların ast\hyp{}üst düzeni keşişlerden, rahibelerden, başrahiplerden ve Büyük Lama’dan meydana gelmektedir. Onlar meleklere, azizlere, bir Kutsal Anne’ye ve tanrılara dua etmektedir. Onlar günah çıkarmayı uygulamakta olup, arınılan arafa inanmaktadırlar. Onların manastırları oldukça geniş, katedralleri fazlasıyla görkemlidir. Onlar, sürekli bir biçimde, aralıksız tekrarlanan kutsal ayinlerini yerine getirmekte olup, bu tür törenlerin kurtuluşu getirdiğine inanmaktadır. Dualar bir çarka bağlanmakta olup, onun dönüşüyle birlikte bu arzuların yerine geldiğine inanmaktadırlar. Çağdaş dönemlerin başka hiçbir insan topluluğu içinde görülemeyecek derecede, birçok dinin birçok uygulaması bu dinde bulunabilir; ve, bu türden eklemlenmiş toplumsal ibadetin olağandışı bir biçimde hantal ve tahammül edilemez bir şekilde külfetli hale gelmiş olması kaçınılmazdır.
\vs p094 10:3 Tibetliler, İsasal müjdenin şu yalın öğretileri dışında başta gelen dünya dinlerinin tümüne ait birtakım şeyleri içinde barındırmaktadır: Tanrı ile olan evlatlık, insan ile olan kardeşlik ve ebedi evren içinde sürekli yükselmekte olan vatandaşlık.
\usection{11.\bibnobreakspace Budist Felsefe}
\vs p094 11:1 Budizm Çin’e İsa’dan sonraki ilk bin yıl içinde girmiş olup, sarı ırkın dini adetlerine oldukça iyi bir biçimde uyum sağlamıştır. Atasal ibadet içinde onlar uzunca bir süre boyunca ölüye dua eden bir konumda bulunmuşlardı; bu aşamada onlar aynı zamanda onlar içinde dua edebilmekteydi. Budizm yakın zaman içinde, dağılmakta olan Taoizm’in hala varlığını sürdüren ayinsel uygulamaları ile bütünleşmişti. Bu yeni birleşimsel din, ibadet tapınakları ve belirli dini törenleri ile birlikte yakın zaman içerisinde; Çin, Kore ve Japonya insanlarının genel olarak kabul görmüş inancı haline geldi.
\vs p094 11:2 Gotama takipçilerinin inanca ait tarihsel anlatımları ve öğretileri, onu kutsal bir varlık yapacak denli çarpıttıkları döneme kadar Budizm’in dünyaya yayılmamış olması bazı açılardan talihsiz olsa da, çok sayıdaki bir mucizeyle birlikte süslemiş olarak onun insan yaşamına ait bu mit, yine de, Budizm’in kuzey veya diğer bir değişle Mahayana müjdesinin dinleyicileri için onu oldukça ilgili çekici hale getirdi.
\vs p094 11:3 Daha sonraki takipçilerinden bazıları; Sakyamuni Buda’nın ruhaniyetinin dönemsel olarak yeryüzüne yaşan bir Buda olarak döndüğünü öğretip, böylece Buda imgeleri, tapınakları, ayinlerine ek olarak düzenbaz “yaşayan Budalar’ın” sınırsız bir çoğalımına yol açmıştı. Böylelikle büyük Hint karşıtının dini nihai olarak; oldukça korkusuz bir biçimde savaştığı ve oldukça gözü pek bir biçimde kötülediği aynı törensel uygulamaların ve ayinsel nakaratların kendisini zincirlenmiş bir halde buldu.
\vs p094 11:4 Budist felsefe içinde gerçekleşen büyük ilerleme, tüm gerçekliğin göreceliğindeki kavrayıştan meydana gelmişti. Bu savın işleyiş biçimi vasıtasıyla Budistler, kendi dinleri ile diğer birçoklarının arasındaki farklılığa ek olarak dini yazıtları içindeki ayrılıkları birleştirmeye ve ilişkilendirmeye yetkin olmuşlardır. Küçük gerçekliğin küçük akıllar için, derin gerçekliğin büyük akıllar için olduğu öğretilmişti.
\vs p094 11:5 Bu felsefe aynı zamanda, Buda (kutsal) doğasının insanların tümünde ikamet ettiğini düşünmekteydi; bu içsel kutsallığın gerçekleşmesine, insanın göstereceği çabalar vasıtasıyla erişebileceğine inanmaktaydı. Ve bu öğreti, ikamet eden Düzenleyiciler’in gerçekliğine dair bir Urantia dinin o zamana kadar yapmış olduğu en açık sunumlarından bir tanesidir.
\vs p094 11:6 Ancak Sidarta’nın özgün müjdesi içindeki büyük bir kısıtlılık, takipçileri tarafından yorumlandığı biçim itibariyle; objektif gerçeklikten benliği uzak tutma yöntemiyle fani doğanın sınırlılıklarının tümünden insan benliğinin bütüncül özgürlüğüne girişmesiydi. Bireyin gerçek olan kâinatsal kendisini gerçekleştirişi, kâinatsal gerçekliğe ek olarak mekân tarafından kısıtlanmakta ve zaman tarafından belirlenmekte olan enerji, akıl ve ruhaniyetin kısıtlı kâinatıyla özdeşleşmesinden doğmaktadır.
\vs p094 11:7 Ancak her ne kadar Budizm’in törenleri ve dışa\hyp{}doğru gerçekleşen adetleri seyahat ettikleri yerlerdekiler ile çok büyük oranda kirlenmişse de, bu yozlaşma, dönemsel olarak, bu düşünce ve inanış düzenini benimsemiş büyük düşünürlerin felsefi yaşamında hiç de bu şekilde ortaya çıkmamıştı. İki binden daha fazla yıllık bir süre boyunca Asya’nın en iyi akılları, mutlak gerçekliği ve Mutlak’ın gerçekliğini belirlemedeki sorunlar üzerine eğilmiştir.
\vs p094 11:8 Mutlak’a dair daha yüksek bir kavramsallaşmanın evrimi, düşüncenin birçok kanalı ve nedensellikçi düşüncenin dolambaçlı yolları aracılığıyla elde edilmişti. Sınırsızlığın bu kavramına dair bu olumlu ilerleme, Musevi din kuramı içindeki Tanrı kavramının evrimi kadar oldukça açık bir biçimde tanımlanmış değildi. Yine de, evrenlerin Başat Kaynağı’nı tahayyül edişlerindeki süreçleri boyunca Budist akıllarının ulaştığı, üzerinde ikamet ettiği ve geçtiği şu büyük aşamalar bulunmuştu:
\vs p094 11:9 1.\bibnobreakspace \bibemph{Gotama efsanesi:} Bu kavramın kökeni, Hindistan’ın tanrı\hyp{}elçisi prensi olan Sidarta’nın yaşam öğretilerine ait tarihsel gerçeklikti. Bu efsane mit olarak büyüdü, ve aydınlanmış kişi olarak Gotama düşüncesi düzeyine erişene ve ek nitelikler almaya başlayıncaya kadar çağlar ve Asya’nın geniş düzlükleri boyunca seyahat etti.
\vs p094 11:10 2.\bibnobreakspace \bibemph{Çok sayıdaki Buda:} Eğer Gotama Hindistan’ın topluluklarına geldiyse, insanlığın ırklarının yakın geçmişte ve yakın gelecekte gerçekliğin diğer öğretmenleri ile kutsanmış olabileceği ve kuşkusuz bir biçimde kutsanacak olduğu nedensel olarak düşünülmüştü. Bu düşünce; sonsuz ve sınırsız bir sayıda bulunan birçok Buda’nın geçmişte var olduğu, hatta herkesin --- bir Buda’nın kutsallığına erişmeyi amaçlayan biçimde --- onlardan biri olmayı arzulayabileceği düşüncesinin doğuşuna kaynaklık etti.
\vs p094 11:11 3.\bibnobreakspace \bibemph{Mutlak Buda.} Budalar’ın sayısı sonsuza yaklaşırken, bahse konu dönemin akıllarının bu karmaşıklaşmış kavramı yeniden toparlamaları gerekli haline geldi. Böylece tüm Budalar’ın; tüm gerçekliğin belli bir Mutlak Kaynağı olarak sonsuz ve sınırsız mevcudiyete ait bir Ebedi Tek biçiminde birtakım daha yüksek özün dışavurumundan başkası olmadığını düşünmeye başladılar. Bu noktadan itibaren Budizm’in İlahiyat kavramı, en yüksek türü içerisinde, Gotama Sidarta’nın insan kişisinden ayrılmış hale gelip, denetim altında tuttuğu insansı kısıtlılıktan kurtuldu. Ebedi Olan Buda’nın nihai kavramsallaşması tam olarak; Mutlak, hatta zaman zaman sınırsız BEN olarak bile tanımlanabilir.
\vs p094 11:12 Mutlak İlahiyat’a dair bu düşünce Asya’nın insan toplulukları arasında hiçbir zaman genele yayılan bir biçimde büyük çaplı bir onay görmemişse de, bu yerlerin düşünürlerini felsefelerini birleştirmeye ve evren bilimlerini uyumlu hale getirmeye yetkin kılmıştı. Mutlak olan Buda’nın kavramsallaşması; zaman zaman tümüyle kişilik\hyp{}dışı olan --- hatta sınırsız bir yaratıcı kuvvet --- biçiminde, bazı durumlarda kişisel görünümlüdür. Her ne kadar felsefe için yardımcı olsa da bu tür kavramsallaşmalar, dini gelişim için hayati değildir. İnsansı bir Yahveh bile, Budizm veya Brahmanizm’in sınırsız derecede uzak bir Mutlak’ından daha büyük dini değerdedir.
\vs p094 11:13 Zaman zaman Mutlak, sınırsız BEN içinde barınmakta olan bir biçimde bile düşünülmüştü. Ancak yürütülen bu fikirler; Tanrı’ya olan inancın kutsal lütfu ve ebedi kurtuluşu teminat alacağına dair Salem’in yalın müjdesini işitmeye, bu sözün cümlelerini duymaya can atan aç çoğunluklar için arzularını öldüren bir rahatlamayı getirmekteydi.
\usection{12.\bibnobreakspace Budizm’in Tanrı Kavramı}
\vs p094 12:1 Budizm’in evrenin bütünlüğe dair görüşü içindeki büyük zafiyet iki katmanlıydı: Hindistan ve Çin’in hurafelerinin birçoğu ile kirlenmiş olmasına ek olarak onun Gotama’yı ilk başta aydınlanmış biri ve daha sonra Ebedi Buda olarak yüceltilmesi. Tıpkı Hıristiyanlık’ın hatalı birçok insan felsefesinin özümsenişinden zarar görmesi gibi, Budizm de kendine ait doğum izine sahiptir. Ancak Gotama’nın öğretileri, geçmişte kalan iki buçuk bin yıl boyunca evirilmeye devam etmiştir. Buda’nın kavramsallaşması aydınlanmış bir Budist için, aydınlanmış bir Hıristiyan için Yehova’nın kavramsallaşmasının tamamiyle Horeb’in kötü ruhlu ruhaniyetinin kendisi olmasından daha az derecede Gotama’nın insan kişiliği değildi. Eski dönemin adlandırma düzenine gösterilen duygusal bağlılık ile birlikte terimlerin kıtlığı sıklıkla, dini kavramların evriminin gerçek önemini anlamadaki başarısızlığı tetiklemektedir.
\vs p094 12:2 Kademeli olarak Tanrı’nın kavramsallaşması, Mutlak’a tezat olan bir biçimde, Budizm’de ortaya çıkmaya başladı. Onun kökenleri, Küçük Yol ve Büyük Yol’un takipçilerinin bu farklılaşmasına ait öncül dönemlerde yatmaktadır. Budizm’in Büyük Yol farklılaşması içinde, Tanrı ve Mutlak’ın çifte kavramsallaşması nihai olarak olgunlaştı. Aşama aşama, çağ çağ Tanrı kavramsallaşması; Japonya’daki Ryonin, Honen, Shonin ve Shinran’nın öğretileri ile bu kavram nihai olarak Amida Buda’ya olan inanç şeklinde meyve verene kadar evirilmeye devam etti.
\vs p094 12:3 Bu inananlar arasında ölümü deneyimlemesi üzerine ruhun, mevcudiyetin nihai aşaması olan Nirvana’ya girmeden önce Cennet’te yapacağı kısa süreli bir ikameti tercih edebileceği öğretilmektedir. Bu yeni kurtuluşun, batıdaki Cennet’in Tanrı’sı olan, Amida’nın kutsal bağışlamalarına ve sevgi dolu ilgisine olan inanç vasıtasıyla erişilebileceği duyrulmaktadır. Felsefeleri içinde Amid unsurları, tüm sınırlı fani kavrayışın ötesindeki bir Sınırsız Gerçeklik’in varlığını düşünmektedirler; dinleri içinde onlar, gerçek inançla ve temiz bir kalp ile ismini çağıran bir faninin Cennet’e ait ulvi yüceliğin erişiminde başarısız olmasına izin vermeyecek bir biçimde dünyayı çok seven her şeyin bağışlayıcısı Amida’ya olan inanca bağlanmaktadırlar.
\vs p094 12:4 Budizm’in büyük gücü, destekleyicilerinin tüm dinlerin içindeki gerçekliği özgür bir biçimde seçebilmesine olanak sağlamasıdır; bu türden tercih özgürlüğü nadiren bir Urantialı inancı nitelemektedir. Bu açıdan Japonya’nın Şin mezhebi, dünyadaki en ilerleyici dini topluluklardan biri olmuştur; o, Gotama takipçilerinin tanrı\hyp{}elçisi olma ruhunu canlandırmış olup, diğer insan topluluklarına öğretmenler göndermeye başlamıştır. Her türlü kaynaktan gerçekliği kendisine katmanın bu istekliliği gerçekten de, İsa’dan sonraki yirminci yüzyılın ilk yarısı boyunca din inananları arasında ortaya çıkması övülür bir eğilimdir.
\vs p094 12:5 Budizm’in kendisi, bir yirminci yüzyıl rönesansı sürecinden geçmektedir. Hristiyanlık ile olan teması vasıtasıyla Budizm’in toplumsal nitelikleri fazlasıyla gelişmiştir. Öğrenme arzusu, kardeşliğin keşiş din\hyp{}adamı kalplerinde yeniden alevlendirilmiştir; ve eğitimin bu inanç vasıtasıyla yayılması kesin bir biçimde, dini evrim içinde yeni gelişmelerin tetikleyicisi olacaktır.
\vs p094 12:6 Bu yazının ortaya çıktığı dönemde, Asya’nın büyük bir kısmı ümidini Budizm’e bağlamaktadır. Geçmişin karanlık çağları boyunca oldukça gözü pek bir biçimde gerçekliği taşımış bu soylu inanç, tıpkı Hindistan’daki büyük öğretmenin takipçileri bir zamanlar onun yeni gerçekliğini dinlediği gibi genişlemiş kâinatsal gerçeklerin gerçekliğini bir kez daha teslim alacak mı? Bu ilkçağ inancı, oldukça uzun bir süre boyunca aramış olduğu Tanrı ve Mutlak’ın yeni kavramlarının sunumu için canlandırıcı uyarıma bir kez daha cevap verecek mi?
\vs p094 12:7 Urantia’nın tümü, evrimsel kökene ait dinler ile on dokuz çağlık temasın sonucunda birikmiş öğreti ve dogmalar tarafından etkilenmeyen Mikâil’in soylulaştırıcı iletisinin duyuruluşunu beklemektedir. Vakit; Budizm, Hristiyanlık, Hinduizm ve hatta tüm inançların topluluklarına İsa hakkında değil İsa’nın müjdesine ait yaşayan, ruhsal gerçekliğin sunumu için çatmaktadır.
\vs p094 12:8 [Nebadon’un bir Melçizedek unsuru tarafından sunulmuştur.]
