\upaper{58}{Urantia üzerinde Yaşamın Oluşumu}
\vs p058 0:1 Satania’nın tümü üzerinde, yaşamın\hyp{}değişikliğe\hyp{}uğratıldığı gezegenler olarak Urantia’ya benzeyen yalnızca altmış bir dünya bulunmaktadır. Yerleşik dünyaların büyük çoğunluğu, oluşturulan işleyiş biçimleri uyarınca insanlar tarafından yerleştirilmişlerdir; bu tür âlemler üzerinde Yaşam Taşıyıcıları, yaşam aktarımları için tasarımlarının yerine getirilmesinde çok az bir zaman kaybına sahiptirler. Ancak on dünyadan bir tanesi bir \bibemph{ondalık gezegeni} olarak adlandırılıp, Yaşam Taşıyıcıları’nın özel kaydına atanmaktadır; bu türden gezegenler üzerinde bizlerin, yaşayan varlıkların ortak evren türleri üzerinde değişiklikte bulunma ve muhtemel bir biçimde onları geliştirmeye dair bir çaba içerisinde belirli yaşam deneyimlerine girişmememize izin verilmektedir.
\usection{1.\bibnobreakspace Fiziksel Yaşam Şartları}
\vs p058 1:1 \bibemph{600.000.000} yıl önce Jerusem’den gönderilen Yaşam Taşıyıcıları’nın heyeti Urantia üzerine ulaşmış olup, Satania sisteminin 606’ncı dünyası üzerinde yaşamı faaliyete geçirmek için hazırlıksal nitelikteki fiziksel şartların çalışmasına başlamıştır. Bu etkinlik; Satania içindeki Nebadon yaşam işleyiş biçimlerinin başlatılması deneyimimizin altı yüz altıncısı olup, yerel evrenin temel ve ortak yaşam tasarımları içinde değişiklerde bulunma ve dönüşümleri gerçekleştirme imkânımızın altmışıncısıdır.
\vs p058 1:2 Bir âlemin evrimsel çevrimin başlatılması için olgun hale gelmesinden önce Yaşam Taşıyıcıları’nın yaşamı başlatmaya yetkin olmadıklarının altı çizilmelidir. Buna ek olarak ne de bizler, gezegenin fiziksel ilerleyişi tarafından desteklenebilecek ve yerine getirebilecek daha hızlı bir yaşam gelişimini sağlayabiliriz.
\vs p058 1:3 Satania Yaşam Taşıyıcıları, yaşamın bir sodyum klorür işleyiş biçimini önceden tasarlamış bulunmaktadırlar; bu nedenle, okyanus sularının yeterli bir biçimde tuzsal hale gelmesine kadar aktarım faaliyeti için hiçbir girişimde bulunulamaz. Urantia protoplazma türü, yalnızca elverişli bir tuz çözeltisi içerisinde faaliyet gösterebilir. Bitkisel ve hayvansal olarak atasal yaşamın tümü, bir tuz\hyp{}çözelti yaşam alanı içinde evrimleşmiştir. Ve daha yüksek bir biçimde düzenlenmiş kara hayvanları bile; “tuzlu derinlik” içine her küçük hücrenin gerçek anlamıyla daldığı şekilde onunla özgürce yıkandığı bir biçimde, kan akışı içerisinde bu aynı hayati tuz çözeltisi bedenleri boyunca dolaşmış olmasaydı, yaşamlarına devam edemezdi.
\vs p058 1:4 Sizin ilkel atalarınız özgür bir biçimde tuz okyanusu içinde dolaşmışlardır; mevcut zaman zarfında, bu benzer okyanussal tuz çözeltisi, tüm nitelikleri bakımından gezegen üzerinde faaliyet göstermesi için ilk yaşayan hücrelerin ilk protoplazmasal tepkilerini tetikleyen tuz suyuna benzer bir kimyasal sıvı ile her bireysel hücreyi yıkayan bir biçimde, özgürce bedenleriniz içinde dolaşmaktadır.
\vs p058 1:5 Ancak bu dönem başlarken Urantia, deniz yaşamının başlangıçsal türlerinin desteklenmesi için elverişli bir düzeye doğru her açıdan evirilmekteydi. Yavaş ama kesin adımlarla dünya ve onun komşu uzay bölgeleri üzerindeki fiziksel gelişmeler; karasal ve uzaysal olmak üzere fiziksel çevrenin kendisini gerçekleştirmesine en iyi uyumu sağlayacağına karar verdiğimiz, bu türden yaşam türlerinin oluşturulması için daha sonraki girişimlerin zemini hazırlamaktadır.
\vs p058 1:6 Bunun sonrasında Yaşam Taşıyıcıları’nın Satania heyeti; yaşam aktarımının mevcut olarak başlamasından önce, hala daha fazla iç denizi ve kapalı körfezi sağlayacak kıtasal kara kütlesinin ilave kırılışlarını beklemeyi tercih ederek Jerusem’e dönmüştür.
\vs p058 1:7 Bir deniz kökenine sahip bir gezegen üzerinde yaşam aktarımı için nihai en uygun koşullar, sığ suların ve kapalı körfezlerin geniş bir kıyı şeridi biçiminde iç denizlerinin geniş bir miktarı tarafından sağlanmaktadır; ve burada, dünyanın sularının tam da bu türden dağılımı hızlı bir biçimde gelişme göstermekteydi. Bu eski iç denizleri nadiren beş ila altı yüz fit derinliğindeydi; ve güneş ışığı, altı yüz fitten fazla okyanus suyu derinliğine girebilmektedir.
\vs p058 1:8 Ve daha sonraki bir çağın ılıman ve dengeli iklim koşullarına ait bu tür deniz kıyılarından, ilkel yaşam aktarımı kendisini karaya taşıyacak zemini bulmuştur. Atmosfer içinde karbonun yüksek düzeyi, hızlı ve gür büyüme için yaşam olanağının yeni kara çeşitliliğini sağlamıştır. Her ne kadar bu atmosfer bu zaman zarfında; bitki gelişimi için olası en iyi koşulu taşısa da, bu türden karbondioksit düzeyine sahip olarak bırakınız insanı hiçbir hayvanın dünya yüzeyinde yaşabilmesine imkân vermemekteydi.
\usection{2.\bibnobreakspace Urantia Atmosferi}
\vs p058 2:1 Gezegensel atmosfer, güneşin ışık emiliminin toplamının yaklaşık olarak iki milyarda birine kadar dünyaya ulaşan güneş ışıklarını süzmektedir. Eğer Kuzey Amerika’ya düşen ışık için saatteki kilovat başına iki sent ödenseydi, yıllık ışık faturası 800 katrilyon doların üstüne kadar çıkardı. Güneş ışığı için Chicago’nun faturası, günlük 100 milyon dolardan çok daha fazla bir fiyata denk gelirdi. Buna ek olarak, --- güneş ışığının atmosferinize ulaşan tek güneşsel dağıtım olmaması gerçeği biçiminde --- güneşten enerjinin diğer türlerini almakta olduğunuz hatırlanmalıdır. Çok geniş güneş enerjileri, insan görüşünün tanımlama kapsamının altında ve bütünde bir aralıktaki dalga boyları ile bütünleşen bir biçimde Urantia üzerine yayılmaktadır.
\vs p058 2:2 Dünyanın atmosferi, renk tayfının aşırı kızılötesi ucunda konumlanan güneş radyasyonunun birçoğu için neredeyse tamamen ışık geçirmez bir nitelikte bulunmaktadır. Bu türden dalga uzunluklarının birçoğu, dünya yüzeyinin yaklaşık olarak on mil yukarısında bulunan ve diğer bir on mil uzunluğunda mekân genişliği içinde enlemesine uzanan bir seviye boyunca var olan bir ozon tabakası tarafından emilmektedir. Dünyanın yüzeyi üzerinde hüküm süren koşullar içerisinde bu bölgeye nüfuz eden ozon, bir inçin onda biri kalınlıktaki bir tabakayı oluşturmaktadır; yine de ozonun bu göreceli olarak küçük ve ortaya çıktığı biçimiyle önemsiz büyüklükteki miktarı, güneş ışığı içinde mevcut olan tehlikeli ve yıkıcı kızılötesi radyasyonların fazlalığından Urantia sakinlerini korumaktadır. Ancak bu ozon tabakası çok az daha kalın olsaydı, mevcut an içerisinde dünyanın yüzeyine ulaşan ve sahip olduğunuz vitaminlerin en temel olanlarından bir tanesinin atası niteliğindeki, oldukça önemli ve hayat verici kızılötesi ışınlardan mahrum kalırdınız.
\vs p058 2:3 Ama yine de fani bilim adamlarınızdan daha az ufka sahip olanları, fani yaratımını ve insan evrimini bir kaza olarak görmede ısrar etmektedir. Urantia yarı\hyp{}ölümlüleri; kazasal talih ile uyumlu olmadığını gördükleri ve maddi yaratım içerisinde ussal amacın mevcudiyetini hataya yer bırakmayan bir biçimde göstermenin uğraşını verdikleri, fizik ve kimya bilimine ait elli bin gerçeği bir araya getirmiştir. Ve bütün bu gerçeklerin tümü; maddi kâinatın tasarlanması, yaratımı ve idaresi içinde aklın mevcudiyetinin var olduğunu savundukları fizik ve kimya biliminin nüfuz alanlarının dışında yüz bin bulgulunun varlığına dair fihristlerinin içinde değildir.
\vs p058 2:4 Güneşiniz, ölümcül ışınların ciddi bir taşkınını dışa doğru yaymaktadır; ve Urantia üzerindeki sizin keyifli yaşamınız, bu benzersiz ozon tabakasının faaliyetine benzer biçimde, görünüşte kırk kazasal nitelikteki koruyucu faaliyetten daha fazlasına ait “rastlantısal” etki altındadır.
\vs p058 2:5 Gece içerisinde atmosferin “yorgan” etkisi olmasaydı, radyasyon vasıtasıyla ısı o kadar hızlı bir biçimde kaybedilecekti ki dışsal destek olmadan yaşamın idaresi imkânsız hale gelecekti.
\vs p058 2:6 Dünya atmosferinin beş veya altı mil aşağısında troposfer bulunmaktadır; burası, hava olaylarını sağlayan rüzgârların ve hava akımlarının bölgesidir. Bu bölgenin üstü iç iyonosfer, ve onun üstündeki bölge ise stratosferdir. Dünyanın yüzeyinden yükseldikçe sıcaklık kademeli bir biçimde her altı veya sekiz milde bir düşmektedir; bu yükselişin zirve noktası yaklaşık olarak eksi 70 fahrenhayt derecesinde kaydedilmiştir. Eksi 65 ila 70 fahrenhayt derecesi arasında değişiklik gösteren bu sıcaklık, kırk milden daha yukarı olan yükseliş aşamasında değişmez bir nitelikte bulunmaktadır. Kırk beş veya elli mil yüksekliğinde sıcaklık artmaya başlamaktadır; ve bu sıcaklık, güneş doğum zamanları düzeyinde 1200 fahrenhayt derecedeki bir sıcaklığın erişimine kadar yükselmektedir; ve oksijeni iyonlaştıran etken bu yoğun ısıdır. Ancak bu türden bir seyrelmiş atmosfer içindeki sıcaklık, dünyanın yüzeyinde algılanan ısı ile neredeyse hiçbir biçimde karşılaştırılamaz. Atmosferinizin tamamının yüzde ellisinin yüzeyden ilk üç mil uzaklık içinde bulunabileceğini unutmayınız. Dünyanın sahip olduğu atmosferin yüksekliği, en yüksek güneş doğum akışlarının işaret ettiği biçimde yaklaşık olarak dört yüz mil uzunluğundadır.
\vs p058 2:7 Güneş doğum olguları; karasal konumda çok sıcak kasırgaları bile meydana getiren bir şekilde, güneş ekvatorunun üstü ve altı olmak üzere zıt yönde burgaç biçiminde dönüş halinde olan güneş tufanları biçiminde, doğrudan güneş lekeleri ile ilgilidir. Bu türden atmosfersel yıkıcı oluşumlar, ekvatorun üstünde veya altında oluşunca zıt yönlerde burgaç şeklinde dönmektedirler.
\vs p058 2:8 Işık dalgaları üzerinde değişiklik yaratmak için güneş lekelerinin sahip olduğu güç, bu güneş fırtına merkezlerinin devasa mıknatıslar olarak faaliyet gösterdiğini ortaya koymaktadır. Bu türden manyetik alanlar; güneş lekesi çukurlarından uzay boyunca, iyonlaştırma etkilerinin bu türden muhteşem güneş doğum oluşumlarını sergiledikleri yer olan dünyanın dışsal atmosferine kadar yüklenen parçacıkları savurmaya yetkindir. Bu nedenle siz; güneş lekelerinin daha genel bir biçimde ekvatorsal konumda yerleştikleri zaman zarfında gerçekleşen güneş lekelerinin zirve noktasında --- veya bunun hemen sonrasında --- en büyük güneş doğum olaylarına sahip olmaktasınız.
\vs p058 2:9 Pusula iğnesi bile, bu türden güneşsel etkiye karşılık vermektedir; çünkü iğne, güneş doğarken hafifçe doğuya güneşin batışına yakın ise yine az bir biçimde batıya doğru yönelerek tepki vermektedir. Bu durum her gün yaşanmaktadır; ancak güneş lekesi çevrimlerinin zirve noktası boyunca pusuladaki bu değişiklik iki kat daha büyüktür. Pusulanın bu günlük değişimleri, güneş ışığı tarafından üretilen üst atmosferin artan iyonlaşmasına gösterilen tepkidir.
\vs p058 2:10 Sahip olduğunuz uzun ve kısa dalga radyo yayınlarının uzak konumlara olan iletimini, yüksek stratosfer içinde elektrikle yüklenmiş iletken bölgelerin iki farklı düzeyinin varlığı açıklamaktadır. Sizin yayınlarınız zaman zaman, bu dışsal iyonosferin bölgeleri içinde ara sıra her şeyi birbirine katan devasa fırtınalar tarafından olumsuz yönde etkilenmektedir.
\usection{3.\bibnobreakspace Uzaysal Çevre}
\vs p058 3:1 Evrenin kendisini gerçekleştirme sürecinin öncül zamanları boyunca uzay bölgelerinin arasına, tıpkı bu türden gökbilimsel toz bulutlarının mevcut zaman zarfında uzayın derinlikleri boyunca birçok bölgeyi belirgin hale getirmesi gibi, çok geniş hidrojen bulutları girmektedir. Alevli güneşlerin ışıma enerjisi olarak kırdıkları ve dağıttıkları bu düzenlenmiş maddenin birçoğu kökensel olarak, uzayın öncül olarak ortaya çıkan hidrojen bulutlarından inşa edilmiştir. Olağanüstü nitelikte belirli şartlar altında atom parçalanması aynı zamanda, daha büyük hidrojen kütlelerinin çekirdeğinde meydana gelmektedir. Ve atom inşası ve parçalanmasının sahip olduğu bu olgularının tümüne, oldukça yüksek ısıya çıkarılmış nebulalarda olduğu gibi, ışıma enerjisine ait kısa uzay ışınlarının taşkın gel\hyp{}gitlerinin ortaya çıkışı eşlik eder. Eşlik eden bu çeşitli radyasyonlar, Urantia üzerinde bilinmeyen uzay\hyp{}enerji düzeyinin bir türüdür.
\vs p058 3:2 Evren uzayının bu kısa\hyp{}ışın enerji etkisi, düzenlenmiş mekân nüfuz alanlarında mevcut olan ışıma enerjisinin tüm diğer türlerinden dört yüz kat daha büyüktür. İster alevli nebulalardan, gergin elektrik alanlarından, uzayın derinliklerinden çıksın veya ister çok geniş hidrojen bulutlarından gelsin, kısa uzay ışınlarının üretimi; sıcaklık, çekim ve elektronik baskıların dalgalanmalarından veya onlar içindeki değişiklerden nicelik ve niteliksel olarak değişikliğe uğramaktadır.
\vs p058 3:3 Uzay ışınlarının kökeni içinde meydana gelen bu oluşumlar; değişikliğe uğramış döngülerden aşırı oval yörüngelere kadar çeşitlilik gösteren döngü halindeki maddenin yörüngelerine ek olarak birçok kâinatsal olay tarafından belirlenmektedir. Fiziksel şartlar aynı zamanda, büyük bir ölçüde değişikliğe uğrayabilir; çünkü elektron dönüşü zaman zaman, aynı fiziksel alan içerisinde bile, maddenin bütünsel davranışının tersi yönünde gerçekleşebilir.
\vs p058 3:4 Çok geniş hidrojen bulutları; evrim halindeki enerji ve başkalaşan maddenin tüm fazlarını içinde barındırarak, kâinatsal çapta dikkate değer kimyasal laboratuarlardır. Büyük enerji faaliyetleri aynı zamanda, oldukça sık gerçekleşen biçimde kesişen ve bu nedenle yaygın olarak birbirine eklemlenen büyük çifte yıldızlara ait azınlık gazları içinde meydana gelmektedir. Ancak uzayın bu devasa ve uçsuz bucaksız enerji etkinliklerinden hiçbiri, --- yaşayan maddelerin ve varlıkların çekirdek plazması olarak --- düzenlenmiş yaşam olgusu üzerinde en ufak bir etkide bulunmamaktadır. Uzayın bu enerji koşulları, yaşam oluşumuna ait temel çevre ile ilgidir; ancak onlar, ışıma enerjisinin daha uzun ışınlarının bazılarına ait olarak, çekirdek plazmasının miras etkenlerinin daha sonra gerçekleşen değişimi içerisinde etkin değillerdir. Yaşam Taşıyıcıları’nın aktarılan yaşamı bütünüyle, evren enerjisinin kısa uzay ışınlarına ait bu muhteşem taşkının tümüne karşı dirençli nitelikte bulunmaktadır.
\vs p058 3:5 Temel nitelikte kâinatsal koşulların tümü; Yaşam Taşıyıcıları’nın Urantia üzerinde yaşamın oluşumuna mevcut olarak başlayabilmelerinden önce, elverişli bir düzeye gelmek zorundaydı.
\usection{4.\bibnobreakspace Yaşam\hyp{}Doğuş Dönemi}
\vs p058 4:1 Yaşam Taşıyıcıları unsurları olarak adlandırılmamız sizlerin zihinlerini karıştırmamalıdır. Biz gezegenlere yaşam taşıyabilmekte olup, bunu hâlihazırda gerçekleştirmekteyiz; ancak biz Urantia’ya yaşam getirmedik. Urantia yaşamı, bu gezegene özgü bir biçimde benzersiz niteliktedir. Bu âlem, yaşamın\hyp{}değişikliğe\hyp{}uğratıldığı bir dünyadır; burada ortaya çıkmakta olan yaşamın tümü bu gezegen üzerinde bizler tarafından tasarlanmıştır; ve tüm Satania’da, hatta Nebadon’un hepsi içinde bile, başka hiçbir dünya Urantia’nın sahip olduğu yaşam deneyiminin aynısını taşımamaktadır.
\vs p058 4:2 \bibemph{550.000.000} yıl önce Yaşam Taşıyıcı birliği Urantia’ya dönmüşlerdir. Ruhsal ve aşkın fiziksel kuvvetler ile eş güdüm halinde bizler; bu dünyanın kökensel yaşam işleyiş biçimlerini başlatmış olup, onları âlemin konuksever sularına yerleştirdik. Gezegen\hyp{}ötesi kişilikler dışında, gezegensel yaşamın başlangıç zamanlarından Gezegensel Prens olarak Caligastia’ya dönemine kadar, Urantia kökenine; bizlerin üç özgün, özdeş ve eş zamanlı olarak gerçekleşen deniz\hyp{}yaşam aktarımında sahip olmuştur. Bu üç yaşam aktarımı; \bibemph{merkezi} veya diğer bir değişle Avrasya\hyp{}Afrika konumlu, \bibemph{doğusal} veya diğer bir değişle Avustralya konumlu ve Grönland ve Amerika kıtalarını içine alan konumda \bibemph{batısal} olarak adlandırılmıştır.
\vs p058 4:3 \bibemph{500.000.000} yıl önce, ilkel deniz\hyp{}bitkisel yaşamı Urantia üzerinde oldukça iyi bir biçimde oluşturulmuştu. Grönland ve kutup kara kütlesi, Kuzey ve Güney Amerika ile birlikte, kendilerinin kuzeye doğru uzun ve yavaş bir biçimde gerçekleşen ayrılış hareketlerine başlamaktaydı. Afrika, kendisi ile ana karası arasında Akdeniz havzası biçiminde doğu kuzey doğrulusunda bir hat yaratarak kısmi bir biçimde güneye doğru hareket etmiştir. Antarktika, Avustralya ve Büyük Okyanus adaları tarafından çevrelenen kara parçası; güneyden ve kuzeyden kopmuş olup, bu zaman zarfından beri ana karadan çok uzaklara doğru sürüklenmiştir.
\vs p058 4:4 Bizler, deniz yaşamının ilkel türünü; kıtasal kara kütlesinin ayrılmakta olan doğu\hyp{}batı uzantısındaki çatlağına ait merkezi denizlerin kapalı sıcak iklim körfezlerine aktarmış bir konumda bulunmaktaydık. Üç deniz\hyp{}yaşam aktarımını gerçekleştirmemizdeki amaç; karanın bir sonraki aşamada gerçekleşen ayrılışı birlikte, her büyük kara kütlesinin sahip olduğu sıcak\hyp{}su denizleri içinde bu yaşamı taşımasını teminat altına almak olmuştur. Bizler; kara yaşamının ortaya çıkışının daha sonraki dönemi içinde suyun geniş okyanuslarının, akıntı tarafından sürüklenmekte olan bu kıtasal kara kütlelerini birbirinden ayıracağını öngörmüş bir konumda bulunmaktaydık.
\usection{5.\bibnobreakspace Kıtasal Ayrılış}
\vs p058 5:1 Bu zaman zarfında kıtasal kara ayrılışları gerçekleşmeye devam etmiştir. Dünyanın çekirdeği, bir inç karede neredeyse 25.000 tonluk bir basınca maruz kalan bir biçimde çelik kadar yoğun ve sert bir hale gelmişti; buna ek olarak devasa çekim basıncı nedeniyle onun iç katmanları bu zaman zarfında ve hali hazırda şimdi bile oldukça sıcaktır. Dünya yüzeyinden itibaren sıcaklık, çekirdeğindeki ısının güneşin yüzey sıcaklığından biraz daha yüksek olduğu bir seviyeye kadar derinlere gidildikçe artış göstermektedir.
\vs p058 5:2 Dünya kütlesinin dış tabakasının bin mili başlıca olarak, farklı türdeki kayalardan meydana gelmiştir. Bu yüzeyin altında, daha yoğun ve daha ağır metal elementler bulunmaktadır. Bu öncül ve atmosfer\hyp{}öncesi çağlar boyunca dünya; eriyik ve çok yüksek sıcaklıkta bulunan düzeyinde neredeyse oldukça akışkan bir konumda bulunmaktadır ki, daha ağır metaller iç bölgelerin derinliklerine doğru ilerlemektedir. Bugün yüzeyin yakınında bulunan bu metaller; tarihi volkanların kalıntılarını, daha sonraki bir zaman zarfında gerçekleşen geniş çaplı lav akıntılarını ve yakın bir tarihte oluşan göktaşı birikintilerini yansıtmaktadır.
\vs p058 5:3 Dış kabuk, yaklaşık olarak kırk mil kalınlığında bulunmaktaydı. Bu dış tabaka; yüksek basınç altında tutulan, ama her zaman, değişen gezegensel basınçların dengelenmesi sürecinde etrafa doğru akış eğilimi göstererek böylelikle dünyanın kabuğunu istikrara kavuşturmaya yönelen eriyik haldeki lavın hareket halindeki bir tabakası olarak, değişen kalınlıktaki volkanik karataşın bir eriyiği tarafından desteklenip, doğrudan bir biçimde bunun üzerinde konumlanır.
\vs p058 5:4 Mevcut an içerisinde bile kıtalar, eriyik volkanik karataşın bu belirginleşmemiş yastıksal denizi üzerinde yüzmeye devam etmektedir. Bu koruyucu özellik bulunmamış olsaydı, daha ciddi depremler dünyayı gerçek anlamıyla parçalara ayırıncaya kadar sallarlardı. Depremlere, katı dış kabuğun yatay doğrultudaki kayışı ve onun dikey yöndeki hareketi neden olmaktadır; onlar volkanlar nedeniyle meydana gelmemektedir.
\vs p058 5:5 Dünya kabuğunun lav tabakaları, soğudukları zaman graniti meydana getirmektedir. Urantia’nın ortalama yoğunluğu, suyun sahip olduğu yoğunluğun beş buçuk katından biraz daha fazladır; granitin yoğunluğu, suyun yoğunluğunun üç katından azdır. Dünyanın çekirdeği, sudan on iki kat kadar yoğundur.
\vs p058 5:6 Deniz tabanları, kara kütlelerinden daha yoğundur; ve kıtaları suyun üstünde tutan bu özelliktir. Deniz tabanları; suyun üstüne çıkarıldığı zaman, kara kütlelerine ait granite kıyasla oldukça ağır olan bir lav türü biçimindeki geniş bir ölçüde volkanik karataşından meydana geldiği görülür. Ve yine benzer bir biçimde; eğer kıtalar okyanus tabanlarından daha hafif olmasaydı, çekim kuvveti okyanus kıyılarını karanın üstüne çıkacak bir şekilde yukarı doğru iterdi. Ancak bu türden oluşumlar gerçekleşmemektedir.
\vs p058 5:7 Okyanusların ağırlığı aynı zamanda, deniz tabanları üzerindeki basıncın artışında bir etkendir. Daha altta bulunan ancak göreceli olarak daha ağır okyanus tabaları, üzerinde bulunan suyun ağırlığı ile birlikte, daha yüksek fakat daha hafif kıtaların ağırlığına yaklaşmaktadır. Ancak kıtaların tümü, okyanuslara doğru sürünerek ilerleme eğilimi göstermektedir. Deniz\hyp{}taban seviyelerinde kıta basıncı, inç kare başına yaklaşık olarak 20.000 pauntluk basınca denk gelmektedir. Bu durum, okyanus tabanından 15.000 fit yukarıda yüzen bir kıta kütlesinin basıncıdır. Okyanus\hyp{}taban su basıncı yalnızca, inç kare başına yaklaşık olarak 5.000 paunttur. Bu basınç farkı, kıtaların okyanus tabanlarına doğru kayma eğilimi göstermesine neden olmaktadır.
\vs p058 5:8 Yaşam\hyp{}öncesi çağlar boyunca okyanus tabanının çöküşü, bağımsız bir kıta kütlesini öyle bir yüksekliğe çıkarmıştır ki; onun ensel basıncı, çevreleyen Büyük Okyanus’un suları istikametinde tabanda bulunan yarı\hyp{}akışmaz lav yataklarının üzerinden doğu, batı, güney kıyılarının aşağıya doğru kayış eğilimi göstermesine neden olmuştur. Bu durum kıtasal basıncın yarattığı basınç farkını o kadar bütüncül bir ölçüde telafi etmiştir ki, geniş bir kırılma bu eski Avrasya kıtasının doğu sahilinde ortaya çıkmamıştır; ancak bu zaman zarfından beri bahse konu bu doğu sahil şeridi, sudan meydana gelmiş bir ölüm bölgesine doğru kayma tehlikesi barındıran bir biçimde, kendisine ait bağımlı okyanus derinliklerinin uçurumlarında gezinmektedir.
\usection{6.\bibnobreakspace Geçiş Dönemi}
\vs p058 6:1 \bibemph{450.000.000} yıl önce, \bibemph{bitki yaşamından hayvan yaşamına olan geçiş} meydana gelmiştir. Bu başkalaşım, ayrılma sürecinde bulunan kıtaların geniş kıyı şeritlerine ait kapalı sıcak iklim körfezleri ve kıyı göllerinin sığ sularında gerçekleşmiştir. Ve özgün yaşam işleyiş biçimleri içinde içkin her şey içerisinde bulunan bu gelişim, kademeli olarak gerçekleşmiştir. Orada, öncül ilkel bitkisel yaşam türleri ve daha sonraki oldukça farklılaşmış belirgin hayvan organizmaları arasında birçok geçiş aşaması var olmuştur. Mevcut zaman içerisinde bile geçiş balçık kalıpları varlığını devam ettirmektedir; ve onlar neredeyse hiçbir biçimde, ne bitkiler ne de hayvanlar olarak tanımlanamamaktadır.
\vs p058 6:2 Her ne kadar bitkisel yaşamın evrimine ait izler hayvan yaşamı içinde sürülebilse de, ve her ne kadar en basitinden en karmaşık ve gelişmiş organizmalara kadar ilerleyen bir biçimde yükselen bitki ve hayvanların yetkin türleri bulunmuş olsa da; siz, ne hayvan krallığının içindeki büyük geniş türler arasında ne de insan\hyp{}öncesi hayvan türlerinin en yüksek canlıları ile insan ırklarının ilk canlıları arasında bu türden birleştirici halkaları bulamayacaksınız. Bu sözde “eksik halkalar” sonsuza kadar kayıp olarak kalmaya devam edecektir; çünkü geleceğe dair bu durumun basit nedeni, onların aslında hiçbir zaman var olmamasıdır.
\vs p058 6:3 Bir çağdan diğerine hayvan yaşamının yeni türleri köklü bir biçimde ortaya çıkmaktadır. Onlar, küçük farklılıkların kademeli olarak bir araya gelişinin sonucu olarak evirilmemektedir; onlar bütünüyle gelişimini tamamlamış ve tamamiyle yeni olan türleri olarak ortaya çıkmaktadır, onlar \bibemph{anlık olarak} belirmektedir.
\vs p058 6:4 Yaşayan organizmaların yeni türlerine ve çeşitlilik gösteren düzeylerine ait \bibemph{anlık} ortaya çıkış, kesinlikle doğal sürecin parçası olarak bütünüyle biyolojiktir. Bu anlık genetik değişimler ile ilgili doğa\hyp{}ötesi hiçbir şey mevcut bulunmamaktadır.
\vs p058 6:5 Tuzluluğun yeterli bir derecesinde okyanuslar içinde hayvan yaşamı evrimleşmiştir; ve bu durumda çok tuzlu suların, deniz yaşamının hayvan bünyeleri boyunca dolaşımda bulunmasına izin verilişi göreceli olarak kolay gerçekleşmiştir. Ancak okyanuslar daraldığında ve böylelikle tuz oranı büyük oranda arttığında bahse konu bu hayvanlar; tıpkı tatlı sularda yaşamak için tuz depolanışının oldukça hünerli işleyiş biçimleri vasıtasıyla beden sıvıları içindeki sodyum klorürün olması gereken düzeyini sağlama kabiliyetini elde eden canlılar gibi, beden sıvılarındaki tuzluluğu azaltma yetkinliğini evrimleştirmişlerdir.
\vs p058 6:6 Deniz yaşamının kaya ile bütünleşen fosilleri üzerindeki çalışmalar, bu ilkel organizmaların öncül uyum çabalarını ortaya çıkarmaktadır. Bitkiler ve hayvanlar, bu uyum deneyimlerini gerçekleştirmeyi hiçbir zaman ara vermediler. En başından beri çevre değişmekte, ve yaşayan organizmalar her zaman bu sonu gelmez değişiklikler karşısında kendi gereksinimlerini yerine getirmeye çabalamaktadır.
\vs p058 6:7 Yaşamın tümüyle yeni düzeylerine ait fizyolojik donanım ve anatomik düzen, fizik kanunlarının etkinliğine karşılık veren bir nitelikte bulunmaktadır; ancak aklın ilerideki kazanımı, içkin beyin yetkinliği uyarınca emir\hyp{}yardımcı akıl\hyp{}ruhaniyetlerinin bir bahşedilişidir. Akıl, her ne kadar bir fiziksel evrim olmasa da, tümüyle fiziksel ve evrimsel gelişmeler tarafından sağlanan beyin yetisine bütünüyle bağlıdır.
\vs p058 6:8 Uyum ve uyumsuzluk biçiminde kazanım ve kayıpların neredeyse sonsuz döngüleri boyunca tüm yaşayan organizmalar, çağdan çağa gelişme ve gerileme göstermektedir. Bazıları kâinatsal bütünlüğe erişmekte, bunun karşısında ise bu amacı yerine getirmede başarısız olanların mevcudiyetleri son bulmaktadır.
\usection{7.\bibnobreakspace Yeryüzü Tarih Kitabı}
\vs p058 7:1 Yaşam\hyp{}doğuş veya diğer bir değişle hayvan öncesi yaşam dönemi boyunca dünyanın dış kavuğunu oluşturan kaya sistemlerinin geniş topluluğu, dünya yüzeyi üzerinde birçok noktada şu an ortaya çıkmamaktadır. Daha sonraki çağların birikimlerinin tümünün altından bu topluluk ortaya çıktığında, orada bitkisel ve öncül ilkel hayvan yaşamının sadece fosil kalıntıları bulunacaktır. Bu daha eski su\hyp{}birikinti kayalarının bazıları, daha sonraki tabakalar ile birlikte iç içe geçmiştir; ve onların en yüksek tabakaları içinde genellikle öncül deniz\hyp{}hayvan organizmalarının daha ilkel türlerinin bazıları saptanırken, zaman zaman onlar bitkisel yaşamın daha önceki türlerinin herhangi birine ait fosil kalıntılarını sergileyebilirler. Birçok mekân içerisinde öncül deniz yaşamının fosillerine sahip olan bu tabakalaşmış en eski kaya katmanları, farklılaşmamış daha eski taşın en üstünde doğrudan bir biçimde bulunabilir.
\vs p058 7:2 Bu döneme ait fosiller; su yosunlarını, mercansı bitkileri, ilkel tek hücreli canlılarını ve süngersi geçiş organizmalarını ortaya çıkarmaktadır. Ancak öncül kaya tabakaları içinde bu türden fosillerin yokluğu doğrudan bir biçimde, yaşayan varlıkların bu fosillerin tortusal birikimleri zamanında hiçbir yerde mevcut olmadıklarının anlamına gelmemektedir. Yaşam, bu öncül zamanlar boyunca aralıklı ve dağılmış bir niteliğe sahipti; ve yalnızca yavaşça gelişen bir biçimde yaşam, dünya yüzeyi üzerinde kendi varlığını sağlamıştır.
\vs p058 7:3 Bu altın çağın kayaları mevcut an içerisinde; dünyanın yüzeyinde veya var olan kara alanının yaklaşık olarak sekizde birinden daha fazla olan bölgede yüzeye oldukça yakın bir konumda bulunmaktadır. Tabakalaşmış en eski kaya katmanları olarak bu geçiş kayasının ortalama kalınlığı, yaklaşık olarak bir buçuk mil kadardır. Bazı noktalarda bu eski kaya sistemleri, dört mil kalınlığa kadar ulaşmaktadır; ancak bu döneme kaynak olarak gösterilen tabakaların birçoğu gerçekte daha sonraki dönemlere aittir.
\vs p058 7:4 Kuzey Amerika içinde fosil taşıyan bu eski ve ilkel taş katmanı, Kanada’nın doğu, merkez ve kuzey bölgeleri üzerinde yüzeye çıkmıştır. Orada aynı zamanda, Pennsylvania ve eski Adirondack Dağları’ndan batı doğrultusunda Michigan, Wisconsin, ve Minnesota boyunca uzanan bu kayanın aralıklı doğu\hyp{}batı sıralı dağ sırtı mevcut bulunmaktadır. Diğer sıralı dağ sırtları, Newfoundland’den Alabama’ya ve Alaska’dan Meksika’ya uzanmaktadır.
\vs p058 7:5 Bu dönemin kayaları, tüm dünya üzerinde etrafa dağılmış bir biçimde ortaya çıkmaktadır; ancak bunların hiçbiri, birkaç tabaka içinde mevcut bulunan bu ilkel fosil taşıyan kayaların bahse konu çok uzak yaşam dönemlerinin kabuksal kabarışlarına ve yüzey dalgalanmalarına tanıklık ettikleri yer olan, Superior Gölü yakınlarına ek olarak Büyük Kanyon ve Colorado Irmağı içindeki örnekler bariz değildir.
\vs p058 7:6 Dünya kabuğu içinde en eski fosil taşıyan katman olarak bu taş tabakası, depremler ve öncül volkanların kabuksal kabarışlarının bir sonucu olarak buruşmuş, katlanmış ve garip bir şekilde bükülmeye uğramıştır. Bu çağın lav akıntıları, gezegensel yüzeyin yakınlarına daha fazla demir, bakır ve kurşunu getirmiştir.
\vs p058 7:7 Bu türden etkinliklerin Wisconsin’in St. Croix vadisi içindekilerden daha görsel olarak gösterilebildiği nadir mekân mevcut bulunmaktadır. Bu bölgede, kara üzerinde meydana gelen birini takip etmiş yüz yirmi yedi lav akışına ek olarak daha sonrasında ortaya çıkan su baskını ve bunun sonrasında oluşan kaya tabakalaşması gerçekleşmiştir. Her ne kadar üst kaya tortulaşması ve aralıklı lav akışı etkinliklerinin birçoğu mevcut an içerisinde gerçekleşmiyor olsa da, ve bu sistemin tabanı dünyanın derinine gömülmüş bir durumda bulunsa da; yine de geçmiş çağların bu tabakalaşmış kayıtlarının yaklaşık olarak yüzde atmış beşi ila yetmişi mevcut an içerisinde gözleme açıktır.
\vs p058 7:8 Birçok kara parçasının deniz seviyesinin yakınında bulunduğu bu öncül çağlar içinde, birbirini takip eden birçok su baskını ve su oluşumu meydana gelmiştir. Dünya kabuğu bu aşamada, göreceli istikrarının daha sonraki dönemine yeni giriş yapmaktadır. Öncül kıta ayrılışının çıkış ve iniş biçimindeki dalgalanmaları, büyük kara kütlelerinin dönemsel batışının sıklığını belirlemiştir.
\vs p058 7:9 İlkel deniz yaşamının bu zamanları boyunca kıtasal kıyıların geniş alanları, birkaç fitten yarım mile kadar denizlerin altına batmıştır. Daha eski kumtaşları ve çakıl kayaların birçoğu, bu eski kıyıların tortusal birikimlerini yansıtmaktadır. Bu öncül tabakalaşmaya ait olan tortusal kayalar; dünya çapındaki okyanusun ilk ortaya çıkışına kadar geri uzanan bir biçimde, yaşamın kökeninin çok ötesindeki zaman zarfına dayanan bu tabakaların tam üzerinde bulunmaktadır.
\vs p058 7:10 Bu geçiş kaya birikintilerine ait üst tabakaların bazıları; organik karbonun mevcudiyetine işaret eder ve bir sonraki Karboniferus veya diğer bir değişle kömür çağı boyunca dünyayı etkin bir biçimde saran bitki yaşamının bu türlerine ait ataların mevcudiyetine tanıklık eden bir biçimde, koyu renklerde bulunan killi yaprak taşının veya arduazın küçük bir parçasını taşımaktadır. Bu kaya tabakaları içindeki kurşunun birçoğu, su birikimi nedeniyle ortaya çıkmaktadır. Bazıları ise; daha eski kayaların çatlaklarında bulunmakta olup, birtakım eski kapalı kıyı şeridinin koyu bataklık suyunda yoğun olarak toplanmıştır. Kuzey Amerika ve Avrupa’nın demir madenleri, bir ölçüde daha eski tabakalaşmamış kayalara ek olarak yaşam oluşumunun bu geçiş dönemlerine ait daha sonraki tabakalaşmış kayalar içinde kısmen barınan birikimlerde ve kalıplar halindeki yüzeye çıkışlarda konumlanmıştır.
\vs p058 7:11 Bu dönem, dünya suları boyunca yaşamın bu dağılımına şahit olmuştur; deniz yaşamı Urantia üzerinde oldukça iyi bir biçimde yerleşmiş hale gelmiştir. Sığ ve geniş iç denizlerin tabanlarına, bitkisel yaşamın cömert ve zengin bir gelişimi kademeli olarak yayılırken; kıyı şerit suları hayvan yaşamının ilkel türleri ile dolup taşmaktadır.
\vs p058 7:12 Bu gelişim sürecinin öyküsünün tümü görsel bir biçimde, dünya kayıtlarının geniş “kaya kitabına” ait fosil sayfalarında anlatılmıştır. Ve eğer siz yorumlamak için gereken kabiliyete ayrıcalıklı olarak erişirseniz, bu devasa biyo\hyp{}yeryüzü kaydının sayfaları hatasız bir biçimde doğruyu anlatmaktadır. Bu tarihi deniz tabanlarının birçoğu, mevcut an içerisinde kara üstünde yükselmiş bir konumda bulunmaktadır; ve onların çağlar boyunca gerçekleşen birikimleri, bu öncül zamanların yaşam mücadelelerine dair süreci anlatmaktadır. “Üzerine bastığımız toz bir zamanlar canlıydı” biçimindeki sizin bir şairinizin ifadesi gerçek anlamıyla doğruluk teşkil etmektedir.
\vs p058 7:13 [Mevcut an içerisinde gezegen üzerinde ikamet etmekte olan, Urantia Yaşam Taşıyıcı Birliği’nin bir üyesi tarafından sunulmuştur.]
