\upaper{9}{Sınırsız Ruhaniyet’in Evren ile İlişkisi}
\vs p009 0:1 Cennet’in mevcudiyetinde Kâinatın Yaratıcısı ve Ebedi Evlat kendilerini kişileştirmek için bir araya geldiklerinde şaşırtıcı bir durum meydana geldi. Bu ebediyet durumunda hiçbir şey, Bütünleştirici Bünye’nin mutlak akılla eş güdüm halinde bulunan ve enerji işletiminin benzersiz ayrıcalıklarıyla bahşedilmiş bir sınırsız ruhaniyet olarak kişilikleştirmesinin önceden belirticisi olamaz. Onun varlığa bürünüşü, kişilik mutlaklığının engellerinden ve merkezileştirilmiş kusursuzluğun bağlarından Yaratıcı’nın kurtuluşunu sağlar ve böylelikle onun özgürleşmesini tamamlamış olur. Ve bu özgürleşme, daha sonra açığa çıkacak olan evrimleşen âlemlerin maddi yaratılmışlarına bile hizmet halinde bulunan ruhaniyetler olarak yardımda bulunmaya fazlasıyla uyum sağlamış varlıkları yaratmak amacıyla Bütünleştirici Bünye’nin muhteşem kudretinde açığa çıkarılmıştır.
\vs p009 0:2 Yaratıcı sevgi ve iradesini yerine getirmede, ruhsal fikir ve niyette sınırsızdır; o evrensel koruyucudur. Evlat doğrunun bilgisine sahiplikte ve bilgelikte, ruhsal dışavurumda ve onun anlamlandırılmasında sınırsızdır; o evrensel gerçeğin açığa çıkarıcısıdır. Cennet gücün bahşedilişinin potansiyelinde ve enerji üstünlüğünün büyüklük ölçeğinde sınırsızdır; o evrensel düzen sağlayıcısıdır. Bütünleştirici Bünye, tüm var olan evren enerjilerini, tüm mevcut evren ruhaniyetlerini ve tüm gerçek evren akli yapılarını düzenlemek için birleşimin benzersiz ayrıcalığı olan sınırsız yetiyi elinde bulundurur; Üçüncül Kaynak ve Merkez, Kâinatın Yaratıcısı’nın ebedi niyeti ve kutsal tasarısının sonucunda açığa çıktığı haliyle çeşitli yaratılmışların ve çok katmanlı enerjilerin evrensel birleştiricisidir.
\vs p009 0:3 Bütünleştirici Yaratan olarak Sınırsız Ruhaniyet evrensel ve kutsal bir yardımcıdır. Ruhaniyet sonu gelmez bir biçimde Evlat’ın bağışlamasına ve Yaratıcı’nın sevgisine hizmet eder, ve bunu Cennet Kutsal Üçlemesi’nin doğruluğu savunan, farklılıklar göstermeyen tutarlı ve değişmez adaletiyle bile uyumlu halde gerçekleştirir. Onun etkisi ve kişilikleri en başından beri sizin en yakınınızdadır; onlar tüm gerçekliğiyle sizi bilir ve içten bir biçimde sizi anlar.
\vs p009 0:4 Âlemler boyunca Bütünleştirici Bünye’nin kurumları durmak bilmeden tüm mekânın güçleri ve enerjilerinin işletilmesini sağlarlar. İlk Kaynak ve Merkez’in sahip olduğu niteliğe benzer bir biçimde, Üçüncül Kaynak ve Merkez ruhsallıkla ve maddeyle etkileşim halindedir. Bütünleştirici Bünye, anlamların ve maddelerin, buna ek olarak enerjiler, akıllar ve ruhaniyetler olarak değerlerin tümünün içinde bir araya geldiği Tanrı’nın bütünlüğünün bir açığa çıkışıdır.
\vs p009 0:5 Sınırsız Ruhaniyet tüm mekânı kapsamı altına alır ve içine nüfuz eder; o ebediyetin döngüsünde ikame eder; bununla birlikte Yaratıcı ve Evlat gibi Ruhaniyet mutlak bir biçimde kusursuz ve değişmezdir.
\usection{1.\bibnobreakspace Üçüncül Kaynak ve Merkez’in Özellikleri}
\vs p009 1:1 Üçüncül Kaynak ve Merkez, bahse konu ilişkiyi atfeden isimlerin tümünde ve onun hizmetinin tanınmasında “Ruhaniyet olarak Tanrı, Yaratıcı olan Tanrı’nın ve Evlat olan Tanrı’nın kutsal eşiti ve onların eş güdüm halindeki kişiliği olarak” birçok isim altında bilinir. Sınırsız Ruhaniyet olarak o, her zaman her yerde bulunan ruhsal bir etkidir. Evrensel İdareci olarak güç\hyp{}denetleyici yaratılmışların atası ve mekânın kâinatsal güçlerinin etkinleştiricisidir. Bütünleştirici Bünye olarak o, Yaratıcı\hyp{}Evlat’ın yönetici birlikteliğinin eş temsilcisidir. Mutlak Akıl olarak o, âlemler boyunca ussal bahşedilişin kaynağıdır. Eylem olan Tanrı olarak o, hareketin, değişimin ve ilişkinin görünürdeki atasıdır.
\vs p009 1:2 Üçüncül Kaynak ve Merkez’in bir takım özellikleri Yaratıcı’dan türemiş olup, bazıları ise kaynağını Evlat’dan almaktadır. Bunların dışında kalanlar ise etkin ve kişisel bir biçimde Yaratıcı veya Evlat’ın mevcudiyetinde gözlenemese de; bu özellikler, Üçüncül Kaynak ve Merkez’i ebedileştiren Yaratıcı\hyp{}Evlat birlikteliğinin Cennet’in mutlaklığının ebedi gerçekliğinin farkındalığında ve onunla uyum halinde devamlı bir biçimde faaliyette bulunuyor olmasının varsayımının dışında başka bir biçimde açıklanamazlar. Bütünleştirici Yaratan, Birincil ve İkincil İlahiyat Bireyleri’nin sınırsız ve bütünleşmiş kavramsallaşmasının tamamlanmışlığını mevcudiyetiyle somutlaştırır.
\vs p009 1:3 Yaratıcı’yı özgün bir yaratan ve Evlat’ı ruhani bir idareci olarak tahayyül ederken, Üçüncül Kaynak ve Merkez’i sınırsız eş güdümün bir hizmetkârı biçiminde olan evrensel bir denetleyici olarak düşünmeniz gerekir. Bütünleştirici Bünye tüm mevcut gerçekliğin bağdaştırıcısıdır; o Yaratıcı’nın düşüncesinin ve Evlat’ın sözünün İlahi muhafaza yeridir, ve eylemlerinde merkezi Ada’nın maddi mutlaklığının ebedi bir biçimde saygın bilinci içerisinde hareket eder. Cennet Kutsal Üçlemesi \bibemph{ilerleyişin} evrensel emrinin hükmünü verdi, ve bu bağlamda Tanrı’nın takdiri ilahisi Bütünleştirici Yaratan’ın ve evrimleşen Yüce Varlık’ın nüfuz alanını oluşturur. Hiçbir mevcut veya oluşum içerisinde olan gerçeklik Üçüncül Kaynak ve Merkez’in nihai ilişkisinden kaçamaz.
\vs p009 1:4 Kâinatın Yaratıcısı enerji öncesi, ruhaniyet öncesi ve kişiliğin âlemleri üzerinde hâkimiyet sahibidir; Ebedi Evlat ruhsal eylemlerin alanlarında üstündür; Cennet Adası’nın mevcudiyeti fiziksel enerjinin ve maddileştiren gücün nüfuz alanını birleştirir; Bütünleştirici Bünye, Evlat’ı temsil eden sadece sınırsız bir ruhaniyet olarak faaliyette bulunmaz, aynı zamanda Cennet’in enerjilerinin ve güçlerinin kâinatsal bir işletimcisi olarak görev yapar, böylelikle mutlak ve evrensel aklı mevcut hale getirir. Bütünleştirici Bünye, yapıcı ve benzersiz bir kişilik olarak özellikle ruhsal değerlerin daha yüksek alanlarında, fiziksel\hyp{}enerji ilişkilerinde ve gerçek akıl anlamlarında olmak üzere muhteşem kâinat boyunca faaliyette bulunur. Daha ayrıntılı olarak o, her ne zaman ve her nerde olursa olsun enerji ve ruhaniyet etkileşim ve birliktelik içinde girdiğinde faaliyette bulunur; aklıyla birlikte tüm tepkimelere üstünlük kurar, ruhsal dünyada muhteşem gücünü elinde barındırır, ve buna ek olarak enerji ve maddi üzerinde çok büyük bir etkiyi ortaya koyar. Her zaman Üçüncül Kaynak, İlk Kaynak ve Merkez’in doğasının dışavurumsal halidir.
\vs p009 1:5 Üçüncül Kaynak ve Merkez kusursuz bir biçimde ve herhangi bir koşuldan bağımsız olarak İlk Kaynak ve Merkez’in her zaman her yerde aynı anda bulunma özelliğini paylaşır, ve bu bakımdan bazı durumlarda Her Zaman Her Yerde Bulunan Ruhaniyet olarak adlandırılır. Alışılmışın biraz dışında ve fazlasıyla kişisel bir biçimde aklın Tanrı’sı Kâinatın Yaratıcısı’nın ve onun Ebedi Evladı’nın her şeyin bilgisine sahip oluşuna dair niteliğini paylaşır; Ruhaniyet’in bilgisi bu bakımdan çok derin ve eksiksizdir. Bütünleştirici Yaratan, Kâinatın Yaratıcısı’nın her şeye gücünün yetme özelliğinin belirli fazlarını dışa vurur, fakat kendisi gerçekte sadece aklın nüfuz alanında her şeye kadirdir. Üçüncül İlahi Birey akıl âlemlerinin evrensel idarecisi ve ussal merkezidir; burada onun egemenliği her hangi bir koşuldan bağımsız olduğu için o mutlaktır.
\vs p009 1:6 Bütünleştirici Bünye, Yaratan\hyp{}Evlat birlikteliği tarafından olumlu yönde yönlendiriliyormuş gibi görünebilir, fakat onun tüm eylemleri Yaratıcı\hyp{}Cennet ilişkisini tanımak için ortaya çıkar. Zaman zaman ve bazı belirli faaliyetlerinde o, Yüce olan Tanrı ve Nihai olan Tanrı gibi deneyimsel İlahiyatlar’ın tamamlanmamış gelişimini telafi ediyormuş gibi görünür.
\vs p009 1:7 Ve burada sınırsız bir gizem mevcuttur: Sınırsız’ın kendi sınırsızlığını Evlat içinde ve Cennet olarak eş zamanlı bir biçimde açığa çıkarışı, ve bunun sonucunda kutsallıkta, Evlat’ın ruhani doğasının yansımasında ve Cennet işleyişini etkinleştirmeye yetkinlikte Tanrı’ya eşit bir varlık olarak buradan mevcudiyete bürünür. Bu varlık, duruma bağlı bir biçimde egemenliğin emri altındadır, fakat birçok biçimde \bibemph{eylem bakımdan} bariz olarak çok yönlüdür. Ve eylem bakımından bu tür apaçık üstünlük, Cennet Adası’nın kâinatsal dışavurumu olan fiziksel çekiminden bile daha üstün olan Üçüncül Kaynak ve Merkez’in bir özelliğinde dışa vurulmuştur.
\vs p009 1:8 Enerji ve maddi şeyleri kapsamına alan bu fiziksel yüksek denetime ek olarak; Sınırsız Ruhaniyet üstün bir biçimde, onun ruhsal hizmetinde oldukça seçkin bir biçimde açığa çıkarılmış olan sabrın, bağışlamanın ve sevginin bu özellikleriyle bahşedilmiştir. Ruhaniyet yüce bir biçimde sevgiye hizmet etmeye ve bağışlamayla adaleti etkisi altına almaya yetkindir. Ruhaniyet olan Tanrı, Benzersiz ve Ebedi Evlat’ın bağışlayıcı sevgisinin ve tanrısal iyiliğinin tümünü elinde barındırır. Evreninizin kökeni adaletin örsü ve mahrumiyetin çekici arasında şekillenmiştir, fakat çekici elinde bulunduranlar Sınırsız Ruhaniyet’in ruhaniyet doğumu biçimindeki bağışlamanın çocuklarıdır.
\usection{2.\bibnobreakspace Her Zaman Her Yerde Aynı Anda Bulunan Ruhaniyet}
\vs p009 2:1 Tanrı ruhani üç katmanlıdır: Tanrı öncelikle kendisi olarak bir ruhaniyettir; buna ek olarak kendi Evlat’ının içinde herhangi bir koşuldan bağımsız olarak ruhaniyet biçimde ortaya çıkar; ve son olarak akılla birliktelik halinde bulunan ruhaniyet olarak Bütünleştirici Bünye’nin içindedir. Ve bu ruhsal gerçekliklere ek olarak, Yüce Varlık’ın, Nihai İlahiyat’ın ve İlahi Mutlaklık’ın ruhaniyetleri olarak deneyimsel ruhaniyet olgular bütününün düzeylerini algılayabildiğimizi düşünmekteyiz.
\vs p009 2:2 Sınırsız Ruhaniyet, tıpkı Evlat’ın Kâinatın Yaratıcısı’nın bir tamamlayıcısı olduğu gibi aynı ölçekte Ebedi Evlat’ın bir tamamlayıcısıdır. Ebedi Evlat, Yaratıcı’nın ruhsallaştırılmış bir kişilikleştirilmesidir; Sınırsız Ruhaniyet ise Ebedi Evlat ve Kâinatın Yaratıcısı’nın kişileşmiş bir ruhsallaştırılışıdır.
\vs p009 2:3 Urantia insanlarını doğrudan bir biçimde Cennet İlahiyatları’na bağlayan kısıtlanmamış birçok ruhsal güç hatları ve maddeler üstü güç kaynakları bulunmaktadır. Bunların arasında; Düşünce Denetleyicileri’nin Kâinatın Yaratıcısı’yla doğrudan ilişkisi, Ebedi Evlat’ın ısrarlı ruhsal\hyp{}çekim yönlendirmesinin çok geniş etkisi, ve Bütünleştirici Bünye’nin ruhsal varoluşu mevcuttur. Bu noktada, Ruhaniyet ve Evlat’ın ruhaniyetleri arasında yükümlü oldukları faaliyetleri bakımından bir fark bulunmaktadır. Üçüncül Birey kendi ruhsal hizmeti içinde akıl ve ruhaniyetle beraber veya yalnız ruhaniyet olarak faaliyette bulunabilir.
\vs p009 2:4 Bu Cennet mevcudiyetlerine ek olarak, Urantialı unsurlar yüce kusursuzluk hedefine ve kutsallığın nihai amaçlarına yukarı ve derinlemesine doğru, niyetin gerçekliğine ve kalbin dürüstlüğüne başından beri onları taşıyacak bu evrenlerin sevgi dolu kişiliklerinin neredeyse sonu olmayan dizilimleriyle beraber yerel ve aşkın\hyp{}evrenin ruhsal etkilerinden ve eylemlerinden yararlandılar.
\vs p009 2:5 Ebedi Evlat’ın kâinatsal ruhaniyetinin mevcudiyeti tamamiyle bizim \bibemph{bilgisine sahip olduğumuz} niteliktedir, biz hiçbir hataya yer bırakmadan onu tanırız. Üçüncül İlahi Birey’i olan Sınırsız Ruhaniyet’in mevcudiyetini, fani insanın bile bilebileceği bir biçimde, insan aklının ırklarına bahşedilmiş olan yerel evrenin Kutsal Ruhaniyet’i olarak faaliyet gösteren bu kutsal etkinin cömert ihsanlarını maddi yaratılmışlar gerçekte deneyimleyebilir. İnsanoğulları Kâinatın Yaratıcısı’nın birey dışı mevcudiyeti olan Düzenleyici’nin bilincine bir ölçüde varabilir. İnsan hizmetinde onun ruhsallaştırılması ve ahlaksal olarak yükseltilmesi için çalışan bu kutsal ruhaniyetlerin hepsi eş zamanlı olarak faaliyette bulunur ve onlar kusursuz bir eş güdüm halindedir. Fani yükselişin ve kusursuzluğa erişimin tasarılarının ruhsal işletim görevinde onlar bir tek bütün halindedirler.
\usection{3.\bibnobreakspace Evrensel İdareci}
\vs p009 3:1 Cennet Adası fiziksel çekimin özü ve kaynağıdır; ve bu durum çekimin fiziksel kâinat âlemlerinin tümünde \bibemph{en gerçek} ve ebedi olarak üzerinde emin olunabilecek şeylerden biri olduğu hususunda sizi yeterli bir ölçüde bilgilendirmeye yetkindir. Kendisine emanet edilen ve faaliyetlerinde birliktelik halinde olduğu Üçüncül Kaynak ve Merkez’in kişiliğiyle birlikte Yaratıcı ve Evlat tarafından bütünleşmiş bir biçimde sağlanan güçler ve enerjilerin dışında, bu çekim hiçbir biçimde değiştirilemez veya ortadan kaldırılamaz.
\vs p009 3:2 Sınırsız Ruhaniyet benzersiz ve muhteşem bir gücü elinde barındırır, bu ise \bibemph{karşı\hyp{}çekimdir}. Bu güç gözle görülen bir biçimde işlevsel olarak ne Yaratıcı’da ne de Evlat’da mevcut değildir. Maddi çekimin etkisine karşı koyma yetisi Üçüncül Kaynak’ın doğasındadır, ve bu yeti Bütünleştirici Bünye’nin evren ilişkilerinin belirli fazlarında sergilenen bireysel tepkimelerinde açığa çıkarılmıştır. Ve bu özgün özellik Sınırsız Ruhaniyet’in daha yüksekte bulunan belirli kişiliklerine iletilebilir bir niteliktedir.
\vs p009 3:3 Karşı\hyp{}çekim yerel bir çerçevelenmişlik içerisinde çekimi ortadan kaldırabilir; bu eylemi eşit güç mevcudiyetinin uygulanması sayesinde yapar. Bu yapı görevini sadece maddi çekimle olan kaynaksal ilişkisi sayesinde yerine getirir, bu bakımdan aklın bir eylemi değildir. Bir denge çarkı olan jireskop’un çekim\hyp{}karşıtlığı ve ona dayanıklılığı bahse konu karşı\hyp{}çekimin \bibemph{etkisinin} uygun bir örneğidir, fakat bu durum karşı\hyp{}çekimin \bibemph{sebebini} sergilemede hiçbir öneme sahip değildir.
\vs p009 3:4 Bütünleştirici Bünye daha ileri bir biçimde kuvveti aşan ve enerjiyi etkisiz hale getiren güçleri ortaya koyar. Bu tür güçler maddileştirme seviyesine kadar enerjiyi düşürme vasıtasıyla ve daha sizin için bilinmez olan diğer yöntemler tarafından görevlerini yerine getirir.
\vs p009 3:5 Bütünleştirici Bünye ne bir enerji, ne bir enerji kaynağı, ne de enerjinin nihai sonudur; bunun yerine o enerjinin \bibemph{işletimcisidir}. Bütünleştirici Bünye; devinim, değişim, değişiklik, eş güdüm, sabitleştirme ve denkleştirmeden oluşan bir eylemdir. Cennet’in doğrudan veya dolaylı denetimine bağımlı olan bu enerjiler, doğası gereği Üçüncül Kaynak ve Merkez ve onun çok katmanlı kurumlarının eylemleriyle karşılıklı etkileşim içerisindedir.
\vs p009 3:6 Kâinat âlemlerinin tümü, Üçüncül Kaynak ve Merkez’in güç\hyp{}denetim yaratılmışları tarafından çevrelenmiştir: bunlar fiziksel denetimciler, güç yöneticileri, güç merkezleri ve fiziksel enerjilerin istikrara kavuşturulması ve düzenlenmesiyle ilgilenen Eylem olan Tanrı’nın diğer temsilcileridir. Fiziksel işlevin bu benzersiz yaratılmışlıklarının hepsi, muhteşem kâinatın enerjilerini ve maddenin fiziksel dengesini kendi çabalarında oluşturmak için kullanılan karşı\hyp{}çekim gibi güç düzenlemesinin çeşitli özelliklerinin tümünü elinde bulundurur.
\vs p009 3:7 Eylem olan Tanrı’nın tüm maddi eylemleri onun Cennet Adası’yla olan faaliyetiyle ilişkili olmak üzere ortaya çıkar, ve gücün bu kurumları ebedi Ada’nın mutlaklığının tümüyle bilincinde ve hatta ona bağımlıdır. Fakat Bütünleştirici Bünye Cennet’e karşılık olarak veya onun için hareket içinde bulunmaz. O bireysel olarak Yaratıcı ve Evlat için eylemlerini gerçekleştirir. Cennet bir birey değildir. Üçüncül Kaynak ve Merkez’in birey olmayan, birey dışı ve ayrıca kişisel olmayan faaliyetleri Bütünleştirici Bünye’nin tamamiyle iradesi dâhilinde gerçekleşen eylemleridir; onlar herhangi bir şeyin veya birinin yansıması, türemesi veya sonuçları değildir.
\vs p009 3:8 Cennet sınırsızlığın bir işleyişidir; Eylem olan Tanrı ise bu işleyişi etkinleştirendir. Cennet sınırsızlığın maddi dayanak noktasıdır; Üçüncül Kaynak ve Merkez’in kurumları, maddi seviyeyi yönlendiren ve fiziksel yaratılışın işleyişine kendiliğinden kendisini gerçekleştirmeyi aşılayan akli yapı kaldıraçlarıdır.
\usection{4.\bibnobreakspace Mutlak Akıl}
\vs p009 4:1 Üçüncül Kaynak ve Merkez’in fiziksel ve ruhsal özelliklerinden farklı olarak onun bir akli doğası bulunur. Böyle bir doğa ilişki halinde olamaz, fakat kişisel olmayan bir biçimde ussal olarak birliktelik içerisindedir. Bu durum Üçüncül Birey’in ruhsal karakteri ve fiziksel özelliklerinden faaliyetin akıl seviyeleri üzerinde ayırt edilebilir, fakat kişiliklerin kavranması için bu doğa hiçbir zaman fiziksel veya ruhsal dışavurumlardan bağımsız bir biçimde faaliyette bulunmaz.
\vs p009 4:2 Mutlak akıl Üçüncül Birey’in aklıdır; bu akıl Ruhaniyet olan Tanrı’nın kişiliğinden ayrılamaz. Varlıklar içerisinde faaliyette bulunan biçimiyle akıl enerjiden, ruhaniyetten veya her ikisinden de ayrık değildir. Akıl enerjinin doğasında bulunmaz; bunun yerine enerji akıl karşısında algılayıcı bir konumda ve onunla ilişkilidir; akıl enerji üzerine konumlandırılabilir, fakat bilinç saf bir biçimde maddi düzeyin doğasında bulunmaz. Akıl katışıksız ruhaniyete eklenmek zorunda değildir, çünkü ruhaniyet kendiliğinden bilinç sahibi ve kimliğini tamamiyle açığa çıkarıcıdır. Ruhaniyet her zaman akıl sahibidir, ve belirli bir biçimde \bibemph{mantık doğrultusunda hareket eder}. Onun ussal nitelikleri gözlemlediğiniz bu veya şu akıllar, hatta tabirinize göre akıl öncesi veya yüksek akıl olabilir, fakat kesin olan bir şey vardır ki bu akıl düşünmenin ve bilgiye sahipliğin eşleniği olan bir akıldır. Ruhaniyetin içeriksel derinliği aklın bilincini aşar, onu dönüşüme uğratır ve yapısal olarak ona öncüllük eder.
\vs p009 4:3 Bütünleştirici Yaratan, kâinatsal akli yapıların âlemleri olarak sadece aklın nüfuz alanında mutlaktır. Üçüncül Kaynak ve Merkez’in aklı sınırsızdır; ve bu akıl bütünüyle kâinat âlemlerinin tümünün etken ve işleyen akıl döngüleri karşısında aşkın bir konumdadır. Yedi aşkın\hyp{}evrenin akıl bahşedilmişlikleri, Bütünleştirici Yaratan’ın öncül kişilikleri olan Yedi Üstün Ruhaniyet’den türemiştir. Bu Üstün Ruhaniyetler kâinat aklı olarak aklı muhteşem kâinata dağıtır, ve sizin yerel evreniniz Orvonton kökenli olan kâinatsal aklın Nebadon’a özgü değişik bir biçimi tarafından etki altına alınmıştır. qw
\vs p009 4:4 Sınırsız akıl zamanı göz ardı eder, nihai akıl zaman karşısında aşkınlaşır ve onun üzerine geçer, kâinat aklı ise zaman tarafından koşullanır. Ve böylelikle mekânla birlikte: Sınırsız Akıl mekândan bağımsızdır, fakat köken bakımından sınırsızlıktan aklın emir\hyp{}yardımcı düzeylerine kadar oluşturulurken, akli yapı mekânın kısıtlanması ve onun göz ardı edilemeyecek gerçekliğiyle artan bir biçimde yüzleşmesi gerekir.
\vs p009 4:5 Kâinatsal kuvvet, tıpkı kâinatsal aklın ruhaniyet ile karşılıklı etkileşim halinde bulunması gibi akılla karşılık bir ilişki halindedir. Ruhaniyet kutsal bir niyettir, ve ruhani akıl eylem bakımından bu nedenle aynı ölçekte kutsal bir niyettir. Enerji bir meta, akıl ise anlam, ruhaniyet bir değerdir. Zaman ve mekân içinde bile; ebediyet bakımından fikir veren, birliktelik içindeki enerji ve ruhaniyet arasında bu göreceli ilişkileri akıl oluşturur.
\vs p009 4:6 Akıl ruhaniyetin değerlerini ussal anlamlara dönüştürür; iradeyi kullanma durumu, aklın anlamlarına maddi ve ruhani nüfuz alanlarında hayat kazandıracak güce sahiptir. Cennet yükselişi; ruhaniyette, akılda ve enerjide göreceli ve farklılaşan bir büyümeyle iç içedir. Kişilik, deneyimsel bireyselliğin bu bileşenlerinin bütünleştiricisidir.
\usection{5.\bibnobreakspace Aklın Hizmeti}
\vs p009 5:1 Üçüncül Kaynak ve Merkez akıl bakımından sınırsızdır. Evren eğer sınırsızlığa doğru bir biçimde büyüme içerisinde ilerlemesini sürdürecekse; onun akli kapasitesi, sınırı olmayan sayıda yaratılmışlara uygun akıl ve buna imkân sağlayacak ussal diğer koşulları onlara ihsan edilmeye yetkin halde olacaktır.
\vs p009 5:2 Üçüncül Birey, \bibemph{yaratılmış aklın} nüfuz alanında, onunla eş güdüm halinde olan ve onun emri altında görev yapan yardımcılarıyla birlikte yüceliğin idareciliğini yapar. Yaratılmış aklın âlemleri Üçüncül Kaynak ve Evren’de ayrıcalıklı kökenine sahiptir; o aklın bahşedicisidir. Yaratıcı nüveleri bile, Sınırsız Ruhaniyet’in ruhsal faaliyetleri ve akli eylemleri tarafından onlar için buna uygun yol düzgün bir biçimde hazırlanana kadar insanların akıllarında ikamet etmeyi imkânsız bulurlar.
\vs p009 5:3 Aklın benzersiz özelliği onun bu kadar geniş bir yaşam üzerinde bahşedilebilir olmasıdır. Onun yaratıcı ve yaratılmış birlikteliğiyle Üçüncül Kaynak ve Merkez, tüm âlemler üzerindeki bütün akıllara hizmet eder. Yerel evrenlerin emir\hyp{}yardımcıları vasıtasıyla tüm insan ve alt\hyp{}insan akli yapılarına hizmet eder, fiziksel denetleyicilerin kurumsallığı vasıtasıyla yaşayan şeylerin en ilkel çeşitleri olan en düşük düzeyde deneyimleme yetisinden yoksun varlıklara bile hizmet eder. Bununla beraber her zaman aklın doğrultusunda bir akıl\hyp{}ruhaniyet veya akıl\hyp{}enerji kişiliklerinin hizmetkârıdır.
\vs p009 5:4 Üçüncül İlahiyat Bireyi aklın kökeni olduğu için, evrimsel irade yaratılmışlarının Sınırsız Ruhaniyet hakkında algılanabilir kavramsallaştırmalar oluşturmayı Kâinatın Yaratıcısı veya Ebedi Evlat’dan herhangi biri hakkında bunu yapmaktan daha kolay bulmaları fazlasıyla olağandır. Bütünleştirici Yaratan’ın gerçekliği insan aklının varoluşunun tam da kendisinde kusursuz olmayan bir biçimde açığa çıkarılmıştır. Bütünleştirici Yaratan kâinatsal aklın atasıdır, buna ek olarak insan aklı bireyselleştirilmiş bir döngü ve yerel evrende Üçüncül Kaynak ve Merkez’in bir Yaratıcı Kız’ı tarafından bahşedildiği şekliyle kâinatsal aklın bir birey dışı bölümüdür.
\vs p009 5:5 Üçüncül Birey aklın kökeni olduğu için, aklın tüm olgularının kutsal olmak zorunda olacağı gibi bir varsayımda bulunmayın. İnsanın sahip olduğu akli yapısı hayvan ırklarının maddi kökeninden gelmektedir. Evren akli yapısı ise, Cennet’in uyumu ve güzelliğinin gerçek bir açığa çıkarılışı olan fiziksel doğadan farklı olarak akıl olan Tanrı’nın gerçek bir açığa çıkarılışından başka bir şey değildir. Kusursuzluk doğanın içindedir, fakat doğa başlı başına kusursuz değildir. Bütünleştirici Yaratan aklın kökenidir, fakat akıl yine bu bağlamda başlı başına Bütünleştirici Yaratan değildir.
\vs p009 5:6 Urantia üzerinde akıl, sizin olgunlaşmamış insan doğanızın evrimleşen mantıksallığı ve düşünce kusursuzluğunun temeli arasındaki bir uzlaşmanın ürünüdür. Sizin ussal evriminiz için yaratılan tasarı gerçekte ilahi olan kusursuzluklardan bir tanesidir, fakat bedeninizin sizlere ev sahipliği yaptığı bünye içerisinde yaşamaya devam ettikçe bu kutsal hedefin gerisinde kalmaya devam edeceksiniz. Akıl bütünüyle kutsal köken içindedir, ve o kutsal nihai bir sona sahiptir; fakat sizin fani akıllarınız bu yapı içerisinde henüz bu kutsal soylulukta içinde barınmazlar.
\vs p009 5:7 Çok fazla sıklıkla ve bunların hepsinde akıllarınıza samimi olmayan bir biçimde zarar verip, onları doğruluk dışına çıkarak dağlamaktasınız; onların hayvanlara özgü olan bir korku duymalarına sebep olmakta ve onları gereksiz olan endişelerle işlemez hale getirmektesiniz. Bu nedenle aklın kökeni kutsal olmasına rağmen; akıl, yükselişte olan dünyanızda sizin bildiğiniz gibi bırakınız kendisine duyulabilecek hayranlık veya ona yapılacak ibadet şöyle dursun, büyük bir beğeninin öznesi olması bile beklenemez. Olgunlaşmamış ve etkin bir durumda olmayan insanın akli yapısı hakkındaki düşünceler sadece teması alçak gönüllülük olan tepkimelere yol açmalıdır.
\usection{6.\bibnobreakspace Akıl\hyp{}Çekim Döngüsü}
\vs p009 6:1 Kâinatsal akli yapı olan Üçüncül Kaynak ve Merkez; tüm yaratılmışlardaki \bibemph{her aklın} ve her usun kişisel olarak bilincine sahiptir. Buna ek olarak, uçsuz bucaksız âlemlerde akıl ihsan edilen ruhsal, morontial ve fiziksel bu tüm yaratılmışlarla bireysel ve kusursuz bir ilişkiyi sürdürür. Aklın tüm bu eylemleri Üçüncül Kaynak ve Merkez’de odaklanan akıl\hyp{}çekim döngüsünde algılanabilir ve bu çekim Sınırsız Ruhaniyet’in bireysel bilincinin bir parçasıdır.
\vs p009 6:2 Yaratıcı tüm kişilik kavramsallaşmasını kendisine doğru çektiği ve Evlat bütün ruhsal gerçekliği üzerinde topladığı, bununla beraber Bütünleştirici Bünye bir çekim kuvvetini tüm akıllar üzerinde uyguladığı kadar; aynı ölçüde o, koşulsuz olarak evren aklı döngüsünü denetler ve onun üzerinde üstünlük kurar. Tüm gerçek ve samimi ussal değerler, kutsal düşüncelerin ve kusursuz fikirlerin bütünü hataya yer bırakmayacak bir biçimde aklın bu mutlak döngüsüne doğru çekilir.
\vs p009 6:3 Akıl çekimi maddi ve ruhsal çekimden bağımsız olarak işleyişte bulunabilir, fakat her nerede ve her ne zaman olursa olsun ruhsal çekim bir etkiye sahip olduğunda akıl çekimi her zaman faaliyet içinde bulunmaya devam eder. Tüm bu üç çekim bir araya gelip etkileşim haline geçtiklerinde, kişilik çekimi fiziksel veya morontial, sınırlı veya absonit seviyede olan maddi yaratılmışla bütünleşebilir. Fakat bu durumdan bağımsız olarak, aklın ihsanı birey dışı varlıklarda bile onları düşünmeye yetkin kılar ve onların bilinç edinimini kişiliğin bütüncül yokluğuna rağmen sağlar.
\vs p009 6:4 İnsan veya kutsal, ölümsüz veya ölümsüzlük olanağına sahip kişilik saygınlığının şahsiyeti kaynağını ne akıldan ne de maddeden alır; bunun yerine gerçekte o Kâinatın Yaratıcısı’nın bir bahşedişidir. Bununla iniltili olarak, ne de ruhaniyetin, aklın ve maddenin çekiminin etkisi kişiliğin çekiminin ortaya çıkması için bir ön koşuldur. Yaratıcı’nın döngüsü ruhaniyet çekimine karşılık vermeyen akli\hyp{}maddi varlıkla bile bütünleşebilir veya maddi çekimle etkileşime girmeyen bir akıl\hyp{}ruhaniyet varlığını kapsamı içine alabilir. Kişilik çekiminin işleyişi her zaman Kâinatın Yaratıcısı’nın iradesi dâhilinde gerçekleşen bir eylemdir.
\vs p009 6:5 Akıl saf bir biçimde maddi varlıklar içinde enerji birlikteliğinde ve katışıksız ruhsal kişiliklerinde ruhaniyet birlikteliğindeyken; içine insanı da alacak bir biçimde kişiliğin sayılamayacak kadar çok olan düzeyleri, enerji ve ruhaniyet ile birliktelik halinde olan aklı ellerinde bulundurur. Yaratılmış aklın ruhsal nitelikleri hatasız bir biçimde Ebedi Evlat’ın ruhaniyet\hyp{}çekiminin etkisine karşılık gösterir; maddi özellikler ise maddi evrenin çekim istenciyle etkileşim halindedir.
\vs p009 6:6 Kâinatsal akıl, ne enerjiyle ne de ruhaniyetle etkileşim halinde olduğunda, yine bu bağlamda ne maddenin ne de ruhsallığın döngüsünün çekim gücüne bağlıdır. Saf akıl sadece Bütünleştirici Bünye’nin kâinatsal çekim algısına bağımlıdır. Saf akıl, sınırsız aklın geldiği kökene yakın bir soydandır, ve sınırsız akıl (ruhaniyet ve enerjinin mutlaklıklarının yapısal yardımcısı olduğu için) açık bir biçimde kendi başına bir işleyiş yasasına sahiptir.
\vs p009 6:7 Daha büyük bir ölçekte gerçekleşen ruhaniyet\hyp{}enerji ayrımı daha yakından gözlemlenebilecek aklın faaliyetine yol açar; daha düşük düzeyde olan enerji ruhaniyet farklılaşması ise bu bakımdan aklın hizmetini daha az görünmez kılar. Açık bir biçimde, kâinat aklının en yüksek sınırda olan faaliyeti mekânın zaman âlemlerindedir. Burada akıl, enerji ve ruhaniyet arasında bir orta alanda faaliyette bulunuyormuş gibi görünür, fakat bu aklın daha yüksek seviyeleri için doğru değildir; Cennet üzerinde, enerji ve ruhaniyet temel olarak bir bütündür.
\vs p009 6:8 Akıl\hyp{}çekim döngüsü güvenilirdir; o bu niteliğini Cennet üzerindeki Üçüncül İlahiyat Bireyi’nden alır, fakat aklın tüm gözle görülebilir faaliyeti tahmine açık değildir. Tüm bilinen yaratılmışlar boyunca, aklın bu döngüsüne benzerlik teşkil eden, faaliyetleri tahmin edilemeyen, hakkında çok az şey anlaşılabilmiş mevcudiyetler bulunur. İnanıyoruz ki bu tahmin edilemezlik Kâinatsal Mutlaklık’ın faaliyetiyle kısmen ilişkilendirilebilir. Bu faaliyetin ne olduğunun bilgisine tam anlamıyla sahip değiliz; onun neyi belirli bir yönde harekete geçirdiği hususunda sadece varsayımda bulunup; onun yaratılmışlar ile olan ilişkisi konusunda ise yalnızca doğruluğu kanıtlanmamış tahminlerde bulunabiliyoruz.
\vs p009 6:9 Sınırlı aklın tahmin edemeyişinin belirli aşamaları Üstün Varlık’ın tamamlanmamışlığından kaynaklanabilir, buna ek olarak Bütünleştirici Bünye ve Kâinatsal Mutlaklık’ın muhtemelen birbirlerine teğet oluşturdukları burada eylemlere sahiplik eden geniş bir alan bulunmaktadır. Akıl hakkında bilinmeyen birçok şey mevcuttur, fakat bizim emin olduğumuz gerçek: Sınırsız Ruhaniyet’in, Yaratan’ın aklının tüm yaratılmışlara olan kusursuz dışavurumu; Yüce Varlık’ın, onların Yaratan’ının tüm yaratılmışlarının akıllarındaki evrimleşen ifadesidir.
\usection{7.\bibnobreakspace Evren Yansıması}
\vs p009 7:1 Bütünleştirici Bünye; akılsal, maddi ve ruhsal olanın eş zamanlı olarak tanınmasını mümkün kılan bir biçimde evren mevcudiyetinin tüm seviyelerini eş güdümsel hale getirmeye yetkindir. Bu benzersiz ve açıklanamaz gücün görülmesi, duyulması, hissedilmesi ve her şeyin aşkın bir evren boyunca gerçekleştiği biçimde bilinmesi, buna ek olarak yansıma tarafından tüm bu bilginin ve bilgisel edinimin herhangi bir arzu edilen noktada üzerine odaklanılması olarak bu olgular bütünü \bibemph{evren yansımasının} tam da kendisidir. Yansımanın eylemi yedi aşkın\hyp{}evrenin her bir yönetim merkezi olan dünyalarında kusursuzluk içinde gösterilmiştir. Bu durum yerel evrenlerin sınırları içerisinde ve aşkın\hyp{}evrenlerin tüm bölümleri boyunca aynı zamanda işleyiş halindedir. Yansıma böylece en son aşamada Cennet üzerinde bir merkezde toplanır.
\vs p009 7:2 Yansımanın olgusallığı, aşkın\hyp{}evren idari merkezlerinde orada yerleştiği ve muhteşem dışavurumlarda ortaya çıktığı haliyle, her yaratılmışta bulunabilecek varoluşun tüm fazlarının en girişik karşılıklı birlikteliğini yansıtır. Ruhaniyetin bağlantıları Evlat’a kadar, fiziksel enerjinin Cennet’e kadar ve aklınkiler Üçüncül Kaynak’a kadar uzanır; fakat evren yansımasının sıra dışı olgusallığında tüm bu üçünün benzersiz ve eşine az rastlanır bir birlikteliği vardır. Bu bahsi geçen birlikteliğin üç unsuru, uzak koşulları anında ve eş zamanlı olarak onların oluşlarıyla birlikte, evren idarecilerini bunlardan haberdar olmaya yetkin kılmak için birliktelik halindedir.
\vs p009 7:3 Yansımanın işleyiş biçiminin büyük bir kısmını her ne kadar kavrasak da orada bizi kafa karışıklığına iten birçok faz bulunmaktadır. Bütünleştirici Bünye’nin akıl döngüsünün evren merkezi olduğunun, onun kâinatsal aklın atası olduğunun, kâinatsal aklın Üçüncül Kaynak ve Merkez’in mutlak akıl çekiminin baskınlığı altında işleyişini gerçekleştirdiğinin bilgisine sahibiz. Buna ek olarak, kâinatsal aklın döngülerinin tüm bilinen varoluşların ussal düzeylerini etkilediğini; bu döngülerin evrensel mekân raporlarını taşıdığını ve kesin bir biçimde Yedi Üstün Ruhaniyet’e odaklandığını ve Üçüncül Kaynak ve Merkez’de bir araya geldiğini biliyoruz.
\vs p009 7:4 Sınırlı kâinatsal akıl ve kutsal mutlak akıl arasındaki ilişki Yücelik’in deneyimsel aklında evrimleşme biçiminde ortaya çıkar. Zamanın ilk ortaya çıktığı anda, bu deneyimsel aklın Yücelik’e Sınırsız Ruhaniyet tarafından bahşedildiğinin bilgisi bize öğretildi ve biz bu bakımdan yansımanın olgusallığının belirli niteliklerinin sadece Yüce Akıl’ın eylemlerinin varsayımı tarafından açıklanabileceğini tasavvur ediyoruz. Yücelik yansımayla ilişki halinde olmasaydı, kâinatın bu bilincinin hataya yer bırakmayan faaliyetlerini ve karmaşık işlemlerini açıklamada ne yapacağını bilmez bir hale düşerdik.
\vs p009 7:5 Yansıma, deneyimsel sınırlılığın sınırları dâhilinde her şeyin bilgisine sahip olma biçiminde ortaya çıkar ve yansıma Yüce Varlık’ın mevcudiyet\hyp{}bilincinin oluşumunu yansıtır. Eğer bu varsayım doğruysa, yansımanın ona ait herhangi bir fazda kullanılması Yücelik’in bilinciyle birlikte kısmı iletişime denk bir durumda olacaktır.
\usection{8.\bibnobreakspace Sınırsız Ruhaniyet’in Kişilikleri}
\vs p009 8:1 Sınırsız Ruhaniyet güçlerinin ve ayrıcalıklarının birçoğunu kendi yardımcılarına ve emri altındaki kişiliklere ve kurumlara iletmenin bütüncül gücünü elinde bulundurur.
\vs p009 8:2 Sınırsız Ruhaniyet’in ilk İlahiyat\hyp{}yaratıcı eylemi, Kutsal Üçleme’den ayrı bir biçimde faaliyet gösterip yine de Yaratıcı ve Evlat ile birlikte bazı açığa çıkarılmamış birliktelik içerisinde kalarak, Sınırsız Ruhaniyet’in âlemlere dağıtıcısı olan Cennetin Yedi Üstün Ruhaniyet’in mevcudiyetinde kişileşmiştir.
\vs p009 8:3 Bir aşkın\hyp{}evren yönetim merkezinde Üçüncül Kaynak ve Merkez’in doğrudan bir temsilcisi bulunmamaktadır. Bu yedi yaratılmışlığın her biri, aşkın\hyp{}evrenin başkentinde yerleşik bir durumda olan yedi Yansıtıcı Ruhaniyetler vasıtasıyla faaliyette bulunan Cennetin Üstün Ruhaniyetleri’nden birine bağlıdır.
\vs p009 8:4 Sınırsız Ruhaniyet’in bir diğer ve devamlılık içinde olan yaratıcı eylemi zaman zaman Yaratıcı Ruhaniyetler’in üretiminde ortaya çıkmaktadır. Kâinatın Yaratıcısı ve Ebedi Evlat bir Yaratan Evlat’a ebeveyn haline geldikleri her zaman; tüm bunları takip eden evren deneyiminde Sınırsız Ruhaniyet, Yaratan Evlat’ın yakın birlikteliğinde olacak yerel bir evrenin Yaratıcı Ruhaniyet’inin atası haline gelir.
\vs p009 8:5 Ebedi Evlat’ı Yaratan Evlatlar’dan ayırmak her ne kadar gerekliyse aynı ölçüde Sınırsız Ruhaniyet ve Yaratan Evlatlar’ın yerel evren düzenleyicileri olan Yaratıcı Ruhaniyetler’i birbirinden ayırt etmek gereklidir. Bütünsel yaratımda Sınırsız Ruhaniyet ne ifade ediyorsa Yaratıcı Ruhaniyet ise yerel bir evren için o anlama gelmektedir.
\vs p009 8:6 Üçüncül Kaynak ve Merkez, muhteşem kâinatta geniş bir dizilimde hizmet eden ruhaniyetler, haber taşıyıcıları, öğretmenler, yargıçlar, yardımcılar, ve danışmanlarla birlikte fiziksel, morontial ve ruhsal doğanın belirli döngülerinin denetçileriyle temsil edilir. Bu varlıkların tümü kişiliğin katı kavramsallaşması dâhilinde değildir. Sınırlı\hyp{}yaratılmışın çeşitliliğinin kişiliği şu biçimlerde tanımlanır:
\vs p009 8:7 1.\bibnobreakspace Öznel birey\hyp{}bilinci.
\vs p009 8:8 2.\bibnobreakspace Tanrı’nın birey döngüsüne verilen nesnel karşılık.
\vs p009 8:9 Yaratan ve yaratılmış kişiliklerine ek olarak bu iki temel çeşitlerin dışında, Sınırsız Ruhaniyet’e kişisel olarak bağlı bulunan\bibemph{ Üçüncül Kaynak ve Merkez’in kişilikleri} bulunmaktadır; fakat bu kişilikler koşulsuz olarak yaratılmış varlıklara kişisel olarak bağlı değillerdir. Bu Üçüncül Kaynak kişilikleri Yaratıcı’nın kişilik döngüsünün bir parçası değildir. İlk Kaynak ve Üçüncül Kaynak kişilikleri karşılıklı olarak iletişim halinde olmasına ek olarak tüm kişilikler iletişim halindedir.
\vs p009 8:10 Yaratıcı, kendi kişisel özgür iradesi vasıtasıyla kişilik bahşedişinde bulunur. Onun bunu neden yaptığına dair sadece tasavvurda bulunabiliriz; fakat onun bunu nasıl gerçekleştirdiğine dair bir bilgiye sahip değiliz. Buna ek olarak Üçüncül Kaynak’ın Yaratıcı\hyp{}dışı kişilik bahşetmesinin nedenini bilmiyoruz, fakat bu konuda bildiğimiz bir şey var ki o da Sınırsız Yaratıcı’nın bu bahşedişini yaratım bakımından Ebedi Evlat’la birlikte ve sayılamayacak biçimlerde sizin tarafınızdan bilinemeyecek bir şekilde kendi adına gerçekleştirdiğidir. Sınırsız Ruhaniyet aynı zamanda İlk Kaynak kişiliğinin bahşedilişinde Yaratıcı için eylemlerde bulunur.
\vs p009 8:11 Üçüncül Kaynak kişiliklerinin birçok çeşidi bulunmaktadır. Sınırsız Ruhaniyet; Üçüncül Kaynak kişiliğini, Yaratıcı’nın kişilik döngüsüne dâhil olmayan belli başlı güç yöneticileri gibi birçok birim üzerinde bahşeder. Buna benzer bir biçimde Sınırsız Ruhaniyet; Yaratıcı Ruhaniyetler gibi, Yaratıcı’nın döngüsü içine alınmış yaratılmışlarıyla ilişkilerinde kendilerinin oluşturduğu birliktelikte sayılamayacak kadar çok olan varlık birimlerini kişilikler gibi idare eder.
\vs p009 8:12 İlk Kaynak ve Üçüncül Kaynak kişilikleri, insanın kişilik kavramsallaşmasıyla olan birlikteliğinden daha fazlası olarak bu kavramsallaşmanın bütünüyle ihsan edilmiştir; onlar hafızayla, nedensellikle, yargılamayla, yaratıcı hayal gücüyle, fikir birlikteliğiyle, tercihsellikle ve faniler tarafından hiçbir biçimde bilinemeyecek sayısız ilave güçleri içine alan akla sahiplerdir. Az sayıda birkaç istisna dışında, sizin algınızda açığa çıkarılan düzeyler yapısal biçimselliği ve benzerlerinden farklı olan bireyselliği elinde bulundurur; çünkü onlar gerçek varlıklardır. Onların büyük bir çoğunluğu ruhani mevcudiyetin tüm düzeylerine görünebilir bir haldedir.
\vs p009 8:13 Mevcut maddi gözlerinizin sınırlı bakış açısından kurtarıldığınız anda ve morontia bünyesi ile beraberinde gelen ruhsal unsurların gerçekliğine karşı genişleyen hassaslıkla birlikte, siz daha düşük düzeydeki ruhsal birliktelik içinde bulunduklarınızı bile görmeye yetkin hale geleceksiniz.
\vs p009 8:14 \bibemph{Üçüncül Kaynak ve Merkez’in işlevsel ailesi}, bu anlatımda ortaya çıkarıldığı biçimiyle üç muhteşem birim altında toplanır:
\vs p009 8:15 I.\bibnobreakspace \bibemph{Yüce Ruhaniyetler}. Birleşik kökenin bir birimi olarak diğerleri arasında aşağıda bahsi geçen düzeylerle birlikte bütünleşir:
\vs p009 8:16 1.\bibnobreakspace Cennetin Yedi Üstün Ruhaniyeti.
\vs p009 8:17 2.\bibnobreakspace Yüce Evrenlerin Yansıtıcı Ruhaniyetleri.
\vs p009 8:18 3.\bibnobreakspace Yerel Evrenlerin Yaratıcı Ruhaniyetleri.
\vs p009 8:19 II.\bibnobreakspace \bibemph{Güç Yöneticileri}. Denetleyici yaratılmışların ve kuruluşların bir birimi olarak, işleyiş halinde olan tüm mekân boyunca faaliyette bulunur.
\vs p009 8:20 III.\bibnobreakspace \bibemph{Sınırsız Ruhaniyetin Kişilikleri}. Bu adlandırma, bahse konu varlıkların bazılarının irade sahibi yaratılmışlar gibi benzersiz olmasına rağmen Üçüncül Kaynak kişilikleri olduğu anlamına gelmez. Bu kişilikler genel olarak üç büyük sınırlandırma altında toplanır.
\vs p009 8:21 1.\bibnobreakspace Sınırsız Ruhaniyet’in Daha Yüksek Kişilikleri.
\vs p009 8:22 2.\bibnobreakspace Mekân’ın İletici Ev Sahipleri.
\vs p009 8:23 3.\bibnobreakspace Zamanın Hizmetkâr Ruhaniyetleri.
\vs p009 8:24 Bu birimler üstün evrenlerde merkezi ve yerleşik evren olan Cennet üzerinde hizmet eder; buna ek olarak yerel evrenlerde hatta yıldız takımlarına, sistemlere ve gezegenlere faaliyette bulunan düzeylerle bütünleşir.
\vs p009 8:25 Kutsal ve Sınırsız Ruhaniyet’in geniş ailesinin ruhaniyet kişilikleri, zaman ve mekânın evrimsel dünyalarının akıl sahibi yaratılmışlarının tümüne ve Tanrı’nın sevgisinin ve Evlat’ın bağışlamasının yardımının hizmetine sonsuza kadar kendilerini adamıştır. Bu ruhaniyet varlıkları, fani insanın kargaşadan ihtişama ulaşacakları yaşayan bir merdiveni onlar için oluşturur.
\vs p009 8:26 [Zamanın Ataları tarafından Sınırsız Ruhaniyet’in doğasını ve eserini tasvir etmek için Uversa’nın bir Kutsal Danışmanı tarafından Urantia üzerinde sunulmuştur.]
