\upaper{119}{Mesih Mikâil’in Bahşedilişleri}
\vs p119 0:1 Nebadon’un Akşam Yıldızları’nın başı olarak ben, Cebrail tarafından Nebadon’un Mikâil’i olan Kâinat Egemeni’nin yedi bahşedilişine ait hikâyeyi açığa çıkarma görevi için Urantia’ya atandım; benim ismim Gavalia’dır. Bu sunumda bulunurken, görevlendirilmem tarafından konulmuş sınırlandırılmışlıklara katı bir biçimde bağlı kalacağım.
\vs p119 0:2 Bahşedilmenin niteliği, Kâinatın Yaratıcısı’nın Cennet Evlatları’nda içkin konumdadır. Kendilerinin alt düzeylerindeki yaşayan varlıkların yaşam deneyimlerine yaklaşma arzusu içinde, Cennet Evlatları’nın çeşitli düzeyleri; sahip oldukları Cennet ebeveynlerinin kutsal doğasını yansıtmaktadır. Cennet Kutsal Üçlemesi’nin Ebedi Evladı; Grandfanda’nınkine ek olarak zaman ve mekândan gelen kutsal yolcularının yükselişi dönemleri boyunca, kendilerini Havona’nın yedi döngüsü üzerinde kendisini yedi defa bahşetmiş olarak, bu uygulamaya öncü olmuştur. Ve, Ebedi Evlat; Mikâil ve Avonal Evlatları olarak kendi temsilcilerinin bireylerinde mekânın yerel evrenlerine kendisini bahşetmeye devam etmektedir.
\vs p119 0:3 Ebedi Evlat, önceden kararlanmış bir yerel evrene bir Yaratan Evladı’nı bahşettiği zaman, bu Yaratan Evlat; kendi yedi yaratılmış bahşedilişi başarılı bir biçimde tamamlana ve aşkın\hyp{}evrenin yönetim yetkisinde bulunan Zamanın Ataları tarafından onaylanana kadar, yeni yaratımın bütüncül egemenliğini üstlenmeyeceğine dair ebedi Kutsal Üçleme’nin çok önemli törensel andını da içine alan bir biçimde, bu yeni evrenin tamamlanışına, denetimine ve bütünlüğüne dair bütüncül sorumluluğu üstlenir. Bu sorumluluk, evren örgütlenişi ve yaratımına katılmak için Cennet’den ayrılmaya gönüllü olan her Mikâil Evladı tarafından üstlenilir.
\vs p119 0:4 Bu yaratılmış vücutlaştırılışlarının amacı; bu türden Yaratanlar’ın bilge, duygudaş ve anlayışlı egemenler haline gelmelerini sağlamaktır. Bu kutsal Evlatlar, içkin bir biçimde adillerdir; ancak, onlar, bu takip eden bahşedilme deneyimlerinin bir sonucu olarak anlayışlı biçimde bağışlayıcı hale gelir; onlar, doğalarından gelen bir biçimde bağışlayıcıdır; ancak, bu deneyimler kendilerini yeni ve ilave biçimlerde bağışlayıcı hale getirir. Bu bahşedilmeler, kutsal doğruluk içinde ve adil yargıyla yerel evrenleri yönetmenin ulvi görevleri için gördükleri eğitimleri ve hazırlanmalarının son adımlardır.
\vs p119 0:5 Her ne kadar önceden hesaba katılmamış sayısız fayda çeşitli dünyalarınkine, sistemlerinkine ve takımyıldızlarınınkine ek olarak bu bahşedilmelerden etkilenmiş ve yararlanmış evren uslarının farklı düzeyleri için açığa çıksa da, onlar hâlihazırda başat olarak; bir Yaratan Evlat’ın kişisel hazırlanışı ve evren eğitimini tamamlamak için tasarlanmıştır. Bu bahşedilmeler, bir yerel evrenin bilge, adil ve etkin idaresi için olmazsa olmaz derecede önemli değildir; ancak, onlar, çeşitli yaşam türleriyle ve usun binlercesiyle, ancak kusursuz olmayan yaratılmışlarıyla, dolu olan bir yaratımın adil, bağışlayıcı ve anlayışlı bir idaresi için mutlak bir biçimde gereklidir.
\vs p119 0:6 Mikâil Evlatları; yaratmış oldukları varlıkların çeşitli düzeyleri için bütüncül ve adil bir duygudaşlıkla, evren düzenlenişi olan görevlerine başlamaktadırlar. Onlar; tüm bu farklılık gösteren yaratılmışlar için bağışlamanın engin sınırına, hatta, kendi yarattıkları bencillik bataklığı içinde doğru yoldan sapmış bir biçimde nereye gittiğini bilmeyenler ve çırpınanlar için bile acımaya sahiptir. Ancak, adaletin ve doğruluğun bu türden donatılmışlıkları, Zamanın Ataları’nın kabulünden geçmekte yeterli olmayacaktır. Aşkın\hyp{}evrenlerin bu üçlü\hyp{}birlik yöneticileri; bir Yaratan Evlat’ı, kendi yaratılmışlarının içinde bulunduğu çevre içinde tam da bu yaratılmışların kendileri olarak mevcut deneyimde bulunan bir biçimde gerçek anlamda onların bakış açısını elde edinceye kadar, Evren Egemeni olarak onaylamayacaktır. Bu şekilde bu tür Evlatlar, us sahibi ve anlayışlı yöneticiler haline gelmektedirler; onlar, üzerinde idarede bulundukları ve evren yönetim gücünü uyguladıkları çeşitli toplulukları \bibemph{bilir} hale gelmektedirler. Yaşam deneyimi vasıtasıyla onlar kendilerinde, deneyimsel yaratılmış mevcudiyetinden doğan uygulamasal bağışlamaya, adil yargıya ve sabra sahip olurlar.
\vs p119 0:7 Nebadon yerel evreni, şimdi, bahşedilme hizmetini tamamlamış olan bir Yaratan Evlat tarafından yönetilmektedir; o, evrimleşen ve kusursuzlaşan evreninin engin âlemleri üzerinde adil ve bağışlayıcı yücelik içerisinde egemenliğine sahiptir. Nebadon’un Mikâili; Ebedi Evlat’ın zaman ve mekânın evrenleri üzerindeki 611.121’inci bahşedilişi olup, dört milyar yıl önce yerel evreninin düzenlenişine başlamıştır. Mikâil, bir milyar yıl önce gerçekleşen biçimde, yaklaşık olarak Urantia’nın bugünkü şeklini almakta olduğu zaman zarfında ilk bahşedilme serüveni için hazırlandı. Onun bahşedilişleri; yüz elli milyon yıl aralıklarla, sonuncusu bin dokuz yüz yıl önce Urantia üzerinde gerçekleşen biçimde, ortaya çıktı. Ben, şimdi, görevlendirilmemin izin verdiği bütünlüğüyle bu bahşedilmelerin doğası ve karakterini açıklayacağım.
\usection{1.\bibnobreakspace İlk Bahşedilme}
\vs p119 1:1 Nebadon evreninin bir araya getirilmiş yöneticilerinin ve baş idarecilerinin; Mikâil’in, büyük kardeşi Emanuel’in yakın bir zaman içinde Nebadon’da yönetimi eline alacağını, ve, bu arada kendisinin, açıklanmamış bir görev nedeniyle görevinden ayrı konumda bulunacağını duyurusunu duyduklarında, bu, bir milyar yıl önce Salvington üzerinde çok dikkate değer bir gelişmeydi. Takımyıldız Yaratıcıları’na yapılan bir elveda yayını dışında bu etkileşim ile ilgili hiçbir başka duyuruda bulunulmamıştı; diğer yönergelere ek olarak, bu yayın şunu söylemekteydi: “Ve, bu süreçte sizleri Emanuel’in koruması ve kollamasına emanet ederken, ben, Cennet Yaratıcım’ın emrini gerçekleştirmeye gidiyorum.”
\vs p119 1:2 Bu elveda yayınını gönderdikten sonra Mikâil; bu sefer kendisinin tek başına olan gelişi dışında, tıpkı Uversa’ya veya Cennet’e olan ayrılığı için hazırlandığındaki öncül birçok durumda olduğu gibi, Salvington’un gönderim alanı üzerinde ortaya çıktı. O, ayrılığına dair ifadesini şu sözlerle tamamladı: “Ben sizlerden ayrılıyorum, ancak kısa bir dönem için. Birçoğunuz, biliyorum, benimle gelmek isterdiniz; ancak, gittiğim yere sizler gelemezsiniz. Yapmaya hazırlandığım şeyi, sizler yapamazsınız. Ben, Cennet İlahiyatları’nın iradesini gerçekleştirmek için gidiyorum, ve, sahip olduğum görevi tamamladığımda ve bu deneyimi elde ettiğimde, sizler arasındaki yerime geri döneceğim.” Ve, bu şekilde konuşarak, Nebadon’un Mikâili; bu toplananların görüşünden kaybolmuş olup, ortak zamanın yirmi yılı boyunca tekrar ortaya çıkmamıştır. Tüm Salvington içinde; yalnızca Kutsal Hizmetkâr ve Emanuel neyin gerçekleşmekte olduğunu bilir konumda bulunup, Zamanın Ataları onun sırrını yalnızca Parlak ve Sabah Yıldızı, Cebrail olan evrenin baş uygulayıcısı ile paylaşmıştı.
\vs p119 1:3 Salvington’un tüm sakinlerine ek olarak takımyıldız ve sistem yönetim merkez dünyalarında ikamet etmekte olanlar; Yaratan Evlat’ın görevine ve nerede bulunduğuna dair bir kaç bilgi alabilme umuduyla, evren usu için ilgili ana alış merkezleri etrafında toplanmışlardı. Mikâil’in ayrılığından sonraki üçüncü güne kadar, muhtemel öneme sahip hiçbir ileti alınmamıştı. Bu günde, yalın bir ifadeyle şu olağanüstü ve daha\hyp{}önce\hyp{}hiç\hyp{}duyulmamış\hyp{}nitelikteki etkileşimi kaydeden, Nebadon içindeki bu düzeyin ana merkezi olan Melçizedek âleminden gelen bir iletişim Salvington üzerinde kaydedilmişti: “Bugün öğlen, bizlerin nüfusuna ait olmayan ancak tamamiyle bizim düzenimize benzeyen nitelikteki, garip bir Melçizedek Evladı bu dünyanın alış alanında ortaya çıktı. O; bu yeni Melçizedek Evladı’nın bizlerin düzeyine kabul edilmesini ve Nebadon Melçizedekleri’nin acil durum hizmetine verilmesini emreden biçimde, Zamanın Ataları’ndan alınmış ve Salvington’un Emanuel’i tarafından uygun görülmüş hükümleri Uversa’dan taşıyan ve bizlerin baş yöneticisine sunan yalnız bir dördüncül\hyp{}hizmetkâr\hyp{}ruhaniyeti tarafından eşlik edilmekteydi. Ve, bu hüküm uyarınca, emir verildi; bu hüküm, yerine getirildi.”
\vs p119 1:4 Ve, bu, ilk Mikâil bahşedilişi ile ilgili Salvington kayıtlarında neredeyse ortaya çıkan her şeydir. Mikâil’in dönüşünün ve, harfi harfine ifade edilmesi gerekli görülmemiş nitelikte bulunan, evren olaylarının idaresini tekrar eline alacağının gerçekliğinin kaydedildiği yüz yıllık Urantia zamanına kadar başka hiçbir şey görünmemektedir. Ancak, bahse konu çağın acil durum birliklerine ait bu benzersiz Melçizedek Evladı’nın hizmetine dair bir anlatım olarak, Melçizedek dünyası üzerinde tuhaf bir kayıt bulunmaktadır. Bu kayıt, Yaratıcı Melçizedek’in evinin önünü kaplayan basit bir mabet içinde korunmaktadır; ve, o, evren acil durumuna ait yirmi dört göreve olan onun görevlendirilişi ile ilgili, bu geçici Melçizedek Evladı’nın hizmetine dair anlatımdan oluşmaktadır. Ve, oldukça yakın bir zaman içinde gözden geçirmiş olduğum, bu kayıt, şu cümleler ile tamamlanmaktadır:
\vs p119 1:5 “Ve, bu gün öğlen vakti, duyurulmadan ve yalnızca kardeşliğimizin üçü tarafından gözlenen bir biçimde, düzeyimize ait bu ziyaret eden Evlat, sadece yalnız bir dördüncül\hyp{}hizmetkâr\hyp{}ruhaniyeti eşliğinde, geldiği gibi dünyamızdan ayrıldı; ve, bu kayıt şimdi, bu ziyaretçinin bir Melçizedek olarak yaşadığına, bir Melçizedek’in suretinde bir Melçizedek olarak görevde bulunduğuna, ve, düzeyimize ait bir acil durum Evladı olarak görevlerinin tümünü aslına uygun olarak yerine getirdiğine dair onayla sonlanmaktadır. Herkesin rızasıyla o; benzersiz bilgeliği, yüce derin sevgisi ve görevine olan muhteşem bağlılığıyla sevgimizi ve hayranlığımızı kazanmış bir biçimde, Melçizedekler’in başı haline gelmiştir. O, bizleri derinden sevdi, bizleri anladı ve bizlerle birlikte hizmet verdi; ve, sonsuza kadar bizler, kendisinin sadık ve adanmış akran Melçizedek unsurlarıyız; zira, dünyamız üzerindeki bu yabancı şimdi ebedi olarak, Melçizedek doğasının bir evren hizmetkârı haline gelmiştir.
\vs p119 1:6 Ve, bu, Mikâil’in ilk bahşedilişi hakkında sizlere söylemeye izin verilen her şeydir. Bizler, tabi ki; Melçizedekler ile birlikte bir milyar yıl önce oldukça gizemli bir biçimde hizmet vermiş olan bu tuhaf Melçizedek’in, ilk bahşedilişi görevindeki vücutlaştırılmış Mikâil’den başkası olmadığını tamamiyle anlamaktayız. Kayıtlar, özel bir biçimde, bu benzersiz ve verimli Melçizedek’in Mikâil olduğunu ifade etmemektedir; ancak, bu Melçizedek’in zamanında kendisi olduğuna herkes tarafından inanılmaktadır. Muhtemelen, bu gerçekliğe dair mevcut ifade, Sonarington’ın kayıtlarının dışına bulunabilen nitelikte değildir; ve, bu gizli dünyanın kayıtları, bizlere açık değildir. Kutsal Evlatlar’ın yalnızca bu kutsal dünyası üzerinde, vücutlaştırılışın ve bahşedilişin gizemleri tamamiyle bilinmektedir. Bizlerin hepsi, Mikâil bahşedilmelerine ait gerçeklikleri bilmekteyiz; ancak, bizler, onların nasıl gerçekleştirildiklerini anlamamaktayız. Bizler; Melçizedekler’in yaratanı olarak bir evrenin yöneticisinin nasıl, bu kadar aniden ve gizemli bir biçimde onların nüfuslarından bir tanesi haline gelebileceğini, ve, onlardan biri olarak, aralarında yaşayıp, yüz yıl boyunca bir Melçizedek Evladı olarak görev yapabileceğini anlamamaktayız. Ancak, o, tam da bu şekilde gerçekleşmişti.
\usection{2.\bibnobreakspace İkinci Bahşediliş}
\vs p119 2:1 Mikâil’in Melçizedek bahşedilişinden sonraki yaklaşık yüz elli milyon yıl boyunca, 37 numaralı takımyıldıza ait 11 numaralı sistemde kargaşa baş göstermeye başlayana kadar, Nebadon evreni içinde her şey yolunda gitmişti. Bu kargaşa, bir Sistem Egemeni olarak bir Lanonandek Evladı’nın merkezinde bulunduğu, Takımyıldız Yaratıcıları tarafından karara varılmış ve bu takımyıldıza Cennet danışmanı olarak görev yapan Zamanın İnançlıları tarafından onaylanmış bir yanlış anlamadan oluşmaktaydı; ancak, itiraz eden Sistem Egemeni, karardan bütünüyle hoşnut değildi. Yüz yıllık süren hoşnutsuzluğundan sonra, o; Uversa üzerinde Zamanın Ataları’nın eylemi tarafından uzun zamandan beri kararına varılmış ve sonlandırılmış nitelikteki bir isyan olarak, Yaratan Evlat’ın egemenliğine karşı Nebadon evreninde çıkarılmış en geniş kapsamlı ve yıkıcı isyanlardan birine birliktelik unsurlarını sürüklemişti.
\vs p119 2:2 Lutentia ismindeki bu isyankâr Sistem Egemeni, ortak Nebadon zamanına göre yirmi yıldan daha fazla bir süre boyunca kendi yönetim\hyp{}merkez gezegeninde en yüksek konumda idaresinde bulunmuştu; bunun üzerine, Uversa’dan gelen onayla birlikte, En Yüksek Unsurlar, Lutentia’nin görevden alınmasını emredip, ikamet edilen dünyaların çatışmayla parçalanmış ve kafası karışmış sisteminin yönetimini üstlenmek amacıyla yeni bir Sistem Egemeni’nin atanması için Salvington yöneticilerini görevlendirdi.
\vs p119 2:3 Salvington üzerinde bu talebin alınmasıyla eş zamanlı olarak, Mikâil; “zamanı gelince geri dönmenin” sözünü vererek ve yönetim yetkisinin tamamını Zamanın Birliktelikleri, Emanuel olan kendi Cennet kardeşinin elinde toplayarak, “Cennet Yaratıcım’ın emrini yerine getirme” amacıyla evren yönetim merkezinde bulunmama arzusuna dair bu olağanüstü nitelikteki duyurularının ikincisini başlattı.
\vs p119 2:4 Ve, bunun sonrasında, Melçizedek bahşedilişi ile ilişkili olarak onun ayrılışı zamanında gözlenmiş olan aynı yöntemle, Mikâil tekrar, kendi yönetim\hyp{}merkez âleminden ayrıldı. Bu açıklanmamış elvedadan üç gün sonra, orada, yeni ve bilinmez bir üye olarak Nebadon’un birinci\hyp{}derece Lanonandek Evlatları’nın yedek birlikleri arasında bir şey gelişti. Bu yeni Evlat; kendisinin, 37 numaralı takımyıldızın 11 numaralı sistemine, görevden alınmış Lutentia’nın varisi konumunda, ve, yeni bir egemenin atanışına kadar vekâleten bu görevi gerçekleştirecek Sistem Egemeni olarak bütüncül yetkiyle görevlendirilmesini emreden, Salvington’un Emanuel’i tarafından uygun görülmüş nitelikte bulunan Uversa Zamanın Ataları’ndan hükümleri taşıyan yalnız bir üçüncül\hyp{}hizmetkâr\hyp{}ruhaniyeti tarafından eşliğinde, önceden duyurulmadan öğlen vakti ortaya çıktı.
\vs p119 2:5 Evren zamanının on yedi yılından fazla bir süre boyunca, bu tuhaf ve bilinmeyen geçici yönetici; bahse konu kafası karışmış ve istikrarını yitirmiş yerel sistemin olaylarını idare edip, onun zorlukları üzerinde yargıda bulundu. Hiçbir Sistem Egemeni daha öncesinde hiç bu kadar derinden sevilmemiş veya herkes tarafından bu düzeyde onurlandırılmamış ve saygı duyulmamıştı. Adalet ve bağışlama içinde, bu yeni yönetici; kargaşa içindeki sistemi istikrara kavuştururken, sadece yanlış kararları için Emanuel’den özür dilemesi karşılığında yönetim yetkisinin sistem koltuğunu paylaşma ayrıcalığını bile kendi isyankâr selefine önererek, tüm özneleri üzerinde zorlayıcı hizmetinde bulunmuştur. Ancak, Lutentia; oldukça yakın bir süre önce karşı geldiği evren yöneticisinin tam da kendisi olarak, bu yeni ve tuhaf Sistem Egemeni’nin Mikâil’den başkası olmadığını oldukça iyi bilen bir biçimde, bağışlamanın bu tekliflerini elinin tersiyle geri çevirmişti. Ancak, onun yanlış yönlendirilmiş ve gerçeklikten saptırılmış takipçilerinin milyonlarcası, bu dönemdeki bilinen adıyla Palonia sisteminin Kurtarıcı Egemeni olarak, bu yeni yöneticinin bağışlamasını kabul etmişti.
\vs p119 2:6 Ve, bunun sonrasında, görevden alınmış Lutentia’nın kalıcı varisi olarak evren yönetim makamları tarafından belirlenmiş, yeni atanan Sistem Egemeni’nin ortaya çıktığı o dikkate değer gün gelmişti; ve, tüm Palonia, Nebadon’un o zamana kadar tanıdığı en soylu ve en iyi niyetli sistem yöneticisinin ayrılışı için yas tuttu. O; tüm sistem tarafından derinden sevilmiş olup, Lanonandek Evlatları’nın tüm topluluklarına ait akranları tarafından kendisine hayranlık beslenmekteydi. Onun ayrılışı, törensel nitelikte değildi; sistem yönetim\hyp{}merkezinden ayrıldığında, büyük bir kutlama düzenlenmişti. Ve, onun yanlış yapmakta olan selefi bile şu iletiyi yollamıştı: “Sen, her biçimde adil ve doğrusun. Her ne kadar Cennet yönetimini reddetmeye devam etsem de, senin adil ve bağışlayıcı bir idareci olduğunu itiraf etmek zorundayım.”
\vs p119 2:7 Ve, bunun sonrasında, bir isyankâr sistemin bu geçici yöneticisi geçici nitelikteki kısa idari ikametine ait gezegene elveda ederken, ondan sonraki üçüncü günde Mikâil, Salvington’da ortaya çıkmış olup, Nebadon evreninin yönetimine devam etmişti. Orada yakın bir zaman içerisinde, Mikâil’in egemenliğinin ve yönetim yetkisinin artan karar gücüne dair üçüncü Uversa duyuruşu izledi. İlk duyuru Nebadon’a onun varışı zamanında gerçekleştirilmiş olup, ikincisi Melçizedek bahşedilişinin tamamlanışından yakın bir süre sonra yürürlüğe konulmuş bulunup, bu aşamada üçüncüsü, ikinci veya diğer bir değişle Lanonandek görevinin sonlanmasını takip etmektedir.
\usection{3.\bibnobreakspace Üçüncü Bahşediliş}
\vs p119 3:1 Salvington üzerindeki yüce heyet; bir Maddi Evlat’a gerçekleştirecekleri yardımları için gönderilmek üzere, 61 numaralı takımyıldız içindeki 87 numaralı sistemin 217 numaralı gezegeni üzerinde Yaşam Taşıyıcıları’nın çağrısının değerlendirilişini tam da yeni bitirmiş konumda bulunmaktaydı. Bu aşamada gezegen; bu ana kadar tüm Nebadon içindeki bu türden ikinci isyan olarak, başka bir Sistem Egemeni’nin içinde doğru yoldan ayrıldığı yer olan yerleşik dünyalarının bir sistemi içinde konumlanmıştı.
\vs p119 3:2 Mikâil’in isteği üzerine, bu gezegenin Yaşam Taşıyıcıları’nın talebine karşılık veren eylem, Emanuel tarafından incelene ve bunun üzerine durum değerlendirilişinde bulunana kadar ertelenmişti. Bu, olağanın dışında bir işleyişti; ve, ben, hepimizin nasıl da görülmemiş bir şeyin gerçekleşeceğini öngörmüş olduğumuzu, ve, bu belirsizlikte fazla bırakılmadığımız, çok iyi hatırlamaktayım. Mikâil, Emanuel’in ellerine kâinat yönetimini aktarırken, göksel kuvvetleri Cebrail’in emrine emanet etmişti; ve, idari sorumluluklarından bu şekilde kurtulan biçimde, o, Evren Ana Ruhaniyeti’ne elveda deyip, bundan önceki iki seferde gerçekleştirdiğinin tamamiyle aynısını yaparak, Salvington’un ayrılış alanı konumunda gözden kayboldu.
\vs p119 3:3 Ve, beklenilebileceği gibi; tuhaf bir Hizmetkâr Evladı, 61 numaralı takımyıldızının 87 numaralı sistemine ait yönetim\hyp{}merkez dünyası üzerinde, Uversa Zamanın Ataları tarafından görevlendirilmiş, ve Salvington’un Emanuel’i tarafından onaylanmış, bir yalnız birincil\hyp{}hizmetkâr\hyp{}ruhaniyeti eşliğinde, duyurulmamış bir biçimde üçüncü gün sonrasındaki öğlen vaktinde ortaya çıktı. Vekâlet etmekteki Sistem Egemeni bu yeni ve gizemli Maddi Evladı derhal 217 numaralı dünyanın vekâlet edecek olan Gezegensel Prensi olarak atadı; ve, bu atama, 61 numaralı takımyıldızın En Yüksek Unsurları tarafından anında onaylandı.
\vs p119 3:4 Böylece, bu benzersiz Maddi Evlat; gezegensel zamana göre bir tam nesil boyunca yalnız başına görev yaparak, dış evren ile hiçbir doğrudan iletişimi bulunmadan kuşatılmış bir sistemde konumlanmış olarak ayrılığın ve isyanın tecrit altına alınmış bir dünyası üzerinde zorlu sürecine başlamıştı. Bu acil durum Maddi Evladı; görevini yerine getirmemekte olan Gezegensel Prens ve onun bütün çalışanlarının pişmanlığını ve geri dönüşünü gerçekleştirmiş, ve, yerel evrenler içinde kurulu konumda bulunan Cennet yönetiminin sadık hizmetine olan gezegenin geri kazanımına şahit olmuştur. Olması gereken bir süreç içinde bir Maddi Erkek ve Kız Evlat, onun yeniden canlandırdığı ve kurtardığı bu dünya üzerine varmıştır; ve, onlar, görünen gezegensel yöneticiler olarak yerinde bir biçimde göreve getirildiklerinde, geçişsel veya diğer bir değişle acil durum Gezegen Prensi, bir gün öğle vakti ortadan kaybolan bir biçimde resmi olarak elvedasında bulundu. Ondan sonraki üçüncü gün, Mikâil, Salvington üzerindeki olağan konumunda tekrar ortaya çıktı; ve, yakın bir zaman içinde, aşkın\hyp{}evren yayınları, Nebadon içinde Mikâil’in egemenliğinin ilave bir biçimde gelişimini açıklayan Zamanın Ataları’nın resmi dördüncü duyurusunu taşıdı.
\vs p119 3:5 Bu kafası karışmış gezegen üzerinde bahse konu Maddi Evlat’ın; sabırla, soğukkanlılıkla ve beceriyle zorlayıcı durumlarının hangileriyle yüzleştiğini anlatma iznine sahip olmadığımdan dolayı pişmanım. Bu tecrit edilmiş dünyanın eski haline dönüşü, Nebadon boyunca kurtuluşun yıllıkları içinde en güzel biçimde etkileyici bölümlerden bir tanesidir. Bu görevin sonlanışı sürecinde, sevgili yöneticilerinin, ussal varlığa ait bir alt unsur düzeyinin suretinde bu tekrar eden bahşedilmişliklere katılmayı tercih etme nedeni tüm Nebadon için oldukça anlaşılır hale gelmiş konumdaydı.
\vs p119 3:6 Bir Melçizedek Evladı, sonra bir Lanonandek Evladı ve onun da sonrasında bir Maddi Evlat olarak Mikâil’in bahşedilmişliklerinin tümü, eşit düzeyde gizemli ve açıklanabilirliliğin ötesindedir. Her birinde o, \bibemph{ansızın} ortaya çıkmış olup, bahşedilme topluluğunun tamamiyle gelişmiş bireyi olarak ortaya çıkmıştır. Bu tür vücutlaşmalarının gizemi, Sonarington’un gizli âlemi üzerindeki kayıtların az sayıdaki denetimcisine ulaşabilenlerin haricinde, hiçbir zaman bilinemeyecektir.
\vs p119 3:7 Tecrit ve isyan içindeki bir dünyanın Gezegensel Prensi olarak bu muhteşem bahşedilişten beri, hiçbir zaman, Maddi Erkek veya Kız Evlatları’nın herhangi biri, görevlendirilmelerinden şikâyetçi olma veya gezegensel görevlerinin zorluklarında kusur bulma çekiciliğine kapılmamıştır. Her zaman için Maddi Evlatlar; tıpkı kendilerinin denenmek ve sınanmak zorunda oluşları gibi “her açıdan denenmiş ve sınanmış olan” bir unsur olarak, evrenin sahip olduğu bir Yaratan Evlat düzeyi içinde, anlayışlı bir egemen ve duygudaş bir arkadaşa sahip olduklarını bilmektedir.
\vs p119 3:8 Bu görevlerin her birini, evren kökenindeki tüm göksel usları arasında gerçekleştirilmiş artan hizmet ve sadakatin bir çağı izlemişken, her takip eden bahşedilme çağı; evren idaresindeki tüm yöntemlerde ve hükümetin tüm işleyiş biçimlerinde ilerleme ve gelişim tarafından nitelenmişti. Bu bahşedilmeden beri, hiçbir Maddi Erkek veya Kız Evlat, hiçbir zaman, Mikâil’e karşı isyana kasıtlı olarak katılmamıştır; onlar kendisine çok sadık bir biçimde, herhangi bir şekilde bilinçli olarak onu reddetmeyecek kadar çok sadık biçimde, kendisini derinden sevmekte ve onu onurlandırmaktadırlar. Yalnızca aldanma ve temelsiz inanışlar sebebiyle, yakın dönemlerin Âdemleri, isyankâr kişiliklerin daha yüksek türleri tarafından doğru yoldan saptırıldılar.
\usection{4.\bibnobreakspace Dördüncü Bahşediliş}
\vs p119 4:1 Uversa’nın dördüncü bin\hyp{}yıllık çağrı listesinin sonunda, Mikâil, Nebadon hükümetinin yönetimini Emanuel ve Cebrail’in ellerine teslim edişini gerçekleştirdi; ve, tabi ki, bu türden eylemi takiben geçmişte neyin gerçekleşmiş olduğunu hatırlayan biçimde, hepimiz, bahşedilmenin dördüncü görevi içinde Mikâil’in ortadan kayboluşuna şahit olmaya hazırlandık; ve, bizler, uzun bir boyunca bekler konumda bırakılmadık, zira, o yakın bir süre içinde, Salvington ayrılış alanına hareket edip, görüşümüzden kayboldu.
\vs p119 4:2 Bu bahşedilme ortadan\hyp{}kayboluşundan sonraki üçüncü günde, bizler, Uversa’ya olan evren yayınlarında, Nebadon’un yüksek melek yönetim\hyp{}merkezinden yapılmış olan şu önemli haber girişini gözlemledik: “Yalnız bir birincil\hyp{}hizmetkâr\hyp{}ruhaniyeti ve Salvington’un Cebrail’i tarafından eşlik edilmiş, bilinmeyen bir yüksek meleğin duyurulmamış varışını bildirmekteyiz. Bu kaydedilmemiş yüksek melek; Nebadon’un düzeyine ait konumda olup, Salvington’un Emanuel’i tarafından uygun görülmüş olan, Uversa Zamanın Ataları’nın hükümlerini taşımaktadır. Bu yüksek melek; yerel bir evrenin sahip olduğu meleklerin en yüksek düzeye ait olarak deneyimine başlamış olup, çoktan, eğitmen danışmanların birliğine atanmış konumda bulunmaktadır.”
\vs p119 4:3 Mikâil, ortak evren zamanının kırk yıllından fazla bir dönem boyunca, yüksek melek bahşedilişi olarak, bu süreç boyunca Salvington’dan uzak kalmıştı. Bu zaman zarfın süresince, o; yirmi iki farklı dünya üzerinde faaliyet gösteren bir biçimde, yirmi altı farklı üstün eğitmene, özel bir üst düzey rehberi olarak adlandırabileceğiniz, yüksek meleksel bir eğitmen danışmanı olarak görevlendirilmişti. Onun son veya diğer bir değişle tamamlayıcı görevi, Nebadon evreninin 3 numaralı takımyıldızının 84 numaralı sistemi içindeki 462 numaralı dünya üzerinde, bir Kutsal Üçleme Evladı’nın bir bahşedilme görevine danışman ve yardımcı olarak görevlendirilişiydi.
\vs p119 4:4 Yedi yıllık bu görevlendirme boyunca, bu Kutsal Üçleme Eğitmen Evladı, hiçbir zaman, yüksek meleksel birlikteliğine kimliğini açıklamaya bütünüyle ikna olmadı. Tüm yüksek meleklerin bu çağ boyunca özel bir ilgi ve incelemeyle değerlendirildikleri doğrudur. Hepimiz, bir yüksek melek kimliği altında, sevgili Egemenimiz’in dışarıda evren içinde bir yerde olduğunu çok iyi bilmekteydik; ancak, hiçbir zaman bizler, onun kimliğinden de emin olamadık. Hiçbir zaman, onun kimliği; bu Kutsal Üçleme Eğitmen Evladı’nın bahşedilme görevine verilişi zamanına kadar, kesin bir biçimde tespit edilemedi. Ancak, bu dönem boyunca, yüce yüksek melekler her zaman; yaratılmış bahşedilişinin bir görevinde evren Egemeni’nin aniden misafiri olmamızın sonradan farkındalığını önlemek için, özel ilgiyle değerlendirilmişlerdi. Ve, böylece, melekler ile ilgili olarak; onların Yaratan ve Yöneticisi’nin, yüksek meleksel kişiliğin sureti içinde her açıdan denendiği ve sınandığı” sonsuza kadar doğru hale gelmişti.
\vs p119 4:5 Bu takip eden bahşedilmeler artan bir biçimde evren yaşamının daha alt türlerine ait doğayı almaya başlayınca, Cebrail, gittikçe daha çok; bahşedilmiş Mikâil ve Emanuel olan vekâlet halindeki evren yöneticisi arasında evren irtibatı olarak faaliyet gösteren bir biçimde, bu vücutlaşma serüvenlerinin bir birlikteliği haline geldi.
\vs p119 4:6 Bu aşamaya kadar Mikâil, kendi yarattığı evren Evlatları’nın şu üç düzeyine ait bahşedilme deneyiminden geçmiş bulunmaktaydı: Melçizedekler, Lanonandekler ve Maddi Evlatlar. Bunu takiben, o; zaman ve mekânın evrimsel fanileri olarak, sahip olduğu en alt düzeydeki irade sahibi yaratılmışlarının yükseliş süreçlerinin çeşitli fazlarına ilgisini yöneltmeden önce, bir yüce yüksek melek olarak meleksel yaşamın suretinde kişilikleşmek için alçalmaktadır.
\usection{5.\bibnobreakspace Beşinci Bahşediliş}
\vs p119 5:1 Urantia üzerinde zamanın takip edildiği biçimiyle, üç yüz milyon yıldan biraz daha fazla süre önce, bizler; Emanuel’e yapılan evren yönetiminin aktarımlarının bir başkasına daha şahit olup, ayrılık için Mikâil’in hazırlıklarını gözlemledik. Bu durum; Orvonton aşkın\hyp{}evreninin yönetim\hyp{}merkezi olan, doğrultusunun Uversa olduğunu duyurmasın bakımından öncekilerinden farklıydı. Zaman içinde, bizlerin Egemeni ayrıldı; ancak, aşkın\hyp{}evren yayınları hiçbir zaman, Zamanın Ataları’nın mahkemelerine olan Mikâil’in varışından bahsetmedi. Salvington’dan olan ayrılışından kısa bir süre sonra, Uversa yayınlarında üzerinde şu önemli ifade ortaya çıktı: “Oraya bugün; Salvington’un Emanuel’i tarafından onay alan ve Nebadon’un Cebrail’i tarafından eşlik edilen bir biçimde, Nebadon evreninden gelmekte olan fani kökendeki duyulmamış ve nüfusa kayıtlı olmayan bir yükseliş kutsal yolcusu ulaştı. Bu tanımlanamamış varlık; gerçek bir ruhaniyetin düzeyini temsil etmekte olup, birlikteliğimize kabul edilmiştir.”
\vs p119 5:2 Eğer siz bugün Uversa’ya gidecek olursanız, Uversa üzerinde bu isimle bilinen zaman ve mekânın bahse konu özel ve bilinmeyen kutsal yolcusu olarak, Eventod’un orada kısa süreli ikameti dönemine ait anlatımları duyarsınız. Ve, en aşağı, yükseliş fanilerinin ruhani düzeyinin tıpatıp suretindeki yüce bir kişilik olarak, bu yükseliş fanisi, Orvonton ortak zamanına göre on bir yıllık bir süreç boyunca Uversa’da yaşayıp, faaliyette bulundu. Bu varlık; görevlendirmeler alıp, Orvonton’un çeşitli yerel evrenlerinden gelen akranlarıyla ortak bir biçimde, bir ruhaniyet fanisinin sorumluluklarını yerine getirdi. “Tıpkı akranları gibi, her açıdan denenmiş ve sınanmış”, her durumda üstlerinin güvenine ve emanetine layık olduğunu kanıtlamıştı; bunu gerçekleştirirken, o, hatasız bir biçimde, akran ruhaniyetlerinin saygısını ve sadık hayranlığını hak etmişti.
\vs p119 5:3 Salvington üzerinde, bizler; bu dikkat çekmeyen ve nüfusa ait olmayan kutsal yolcu ruhunun, yerel evrenimizin bahşedilmiş yöneticisinden başkası olmadığını, Cebrail’in mevcudiyetine bakarak oldukça iyi bilen bir biçimde, bütün ilgimizle bu ruhaniyet yolcusunun sürecini takip ettik. Fani evrime ait bir aşamanın rolünde vücutlaştırılmış olarak Mikâil’in bu ilk ortaya çıkışı, tüm Nebadon’u büyük heyecana iten ve onları büyüleyen bir olaydı. Bizler; böyle şeyleri duymuştuk, ama bu aşamada onları gözlerimizle görüyorduk. O, Uversa üzerinde tamamiyle gelişmiş ve kusursuz bir biçimde eğitilmiş bir ruhani fani olarak ortaya çıktı; ve, bu bütünlük içerisinde, Havona’ya olan yükseliş fanilerinin bir topluluğunun ilerleyişine kadar sürecine devam etti; bu noktada, Zamanın Ataları ile görüşmede bulunup, doğrudan bir biçimde, Cebrail’in eşliğinde, Salvington üzerindeki her zamanki yerinde yakın bir süre sonra ortaya çıkan bir biçimde, Uversa’dan ansızın ve törensiz gerçekleşen elvedasında bulundu.
\vs p119 5:4 Bu bahşedilişin tamamlanışına kadar, Mikâil’in; en yüksek Melçizedek unsurlarından en aşağı zaman ve mekânın evrimsel dünyaları üzerindeki beden ve kan fanilerine kadar, sahip olduğu evren kişiliklerine ait çeşitli düzeylerin sureti içinde muhtemel bir biçimde vücutlaşımı, bizler tarafından kesin bir biçimde düşünülmemekteydi. Yaklaşık olarak bu zaman zarfında Melçizedek üniversiteleri, Mikâil’in ileride bir dönemde bedenin bir fanisi olarak vücutlaşabilme olasılığını öğretmeye başladılar; ve, bu türden açıklanamaz bahşedilişin olası yöntemi hakkında fazlasıyla düşünce ortaya atıldı. Mikâil’in bizzat, bir yükseliş fanisi rolünde bulunmuş olması; yaratılmış ilerleyişinin hem yerel evren hem de aşkın\hyp{}evren boyunca gerçekleşen en uç noktasına kadar uzanan bütüncül düzenine yeni ve ilave ilgi getirdi.
\vs p119 5:5 Hâlihazırda, bu takip eden bahşedilişlerinin yöntemi bir gizem olarak kaldı. Cebrail bile; aracılığıyla, bu Cennet Evladı ve evren Yaratanı’nın, istediğinde, kendisine ait alt düzey yaratılmışlarından birinin kişiliğini üstlenip, onun yaşamını yaşadığı yöntemi kavrayamadığını itiraf etmektedir.
\usection{6.\bibnobreakspace Altıncı Bahşediliş}
\vs p119 6:1 Bu aşamada, Salvington’un tümü; Mikâil’in, beşinci takımyıldızın yönetim\hyp{}merkez gezegeni üzerinde En Yüksek Yaratıcılar’ın mahkemelerinde bir morontia fanisinin sürecini üstlenme amacıyla Salvington’dan yakın zamanda ayrılacağını açıklayarak, yönetim\hyp{}merkez gezegeni üzerindeki geçici sakinleri bir araya topladığı ve, ilk kez, vücutlaşma tasarımının geri kalanını açıkladığı biçimde, gerçekleşmeyi beklemekte olan bir bahşedilişin hazırlıklarından haberdardı. Ve, bunun sonrasında, bizler; kendisinin yedinci ve son bahşedilişinin, fani bedenin suretinde bir evrimsel dünya üzerinde gerçekleşecek oluşuna dair açıklamayı ilk kez duymuş olduk.
\vs p119 6:2 Altıncı bahşediliş için Salvington’dan ayrılmadan önce, Mikâil; âlemin tüm bir araya toplanmış sakinlerine seslenip, bir yalnız yüksek meleğe ek olarak Nebadon’un Berrak ve Sabah Yıldızı eşliğinde, herkesin tüm görüş alanından ayrıldı. Evrenin idaresi tekrar Emanuel’e emanet edilmişken, orada, idari sorumluluklarının daha geniş bir dağılımı söz konusuydu.
\vs p119 6:3 Mikâil, yükseliş düzeyine ait tamamiyle gelişmiş bir morontia fanisi olarak beş numaralı takımyıldızın yönetim\hyp{}merkezinde ortaya çıktı. Çok üzülerek ifade etmek isterim ki, bu kayıt\hyp{}dışı morontia fanisinin yükselişine ait detayları açığa çıkarmam yasaklanmıştır; zira, o, Urantia üzerindeki çarpıcı ve trajik konukluğunu bile içine alan bir biçimde, Mikâil’in bahşedilme deneyiminde en olağanüstü ve derinden etkileyici aşamalardan bir tanesiydi. Ancak, bu görevi kabul etmemle birlikte bana getirilmiş sınırlılıkların birçoğu arasında, Endantum’un morontia fanisi olarak Mikâil’in bu muhteşem sürecinin detaylarını açığa çıkarmaya kalkışmamı yasaklayan bir maddenin bulunmaktadır.
\vs p119 6:4 Hepimiz için, Mikâil, bu morontia bahşedilişinden geri döndüğünde, bizlerin Yaratanı’nın; Evren Egemeni’nin aynı zamanda, âlemleri içindeki yaratılmış usun en alt düzeydeki türünün bile arkadaşı ve duygudaş yardımcısı olduğu biçiminde, bir akran yaratılmışı haline geldiği, bariz hale geldi. Bizler, bu bahşedilmeden önce, evren iradesi içinde yaratılmış bakış açısının bu ilerleyici erişiminin farkına varmış halde bulunmaktaydık; zira, o, kademeli bir biçimde ortaya çıkan haldeydi; ancak, morontia fani bahşedilişinin tamamlanışından sonra daha, hatta Urantia üzerinde marangozun evladının sürecinden geri dönüşünden sonra daha da bile fazla olan bir biçimde, bariz hale geldi.
\vs p119 6:5 Bizler; morontia bahşedilmesinden Mikâil’in salıverilme vaktinin geldiği hususunda Cebrail tarafından önceden bilgilendirilmiş olup, Salvington üzerinde yakışır bir karşılama töreni düzenledik. Milyonlarca varlık, Nebadon’un takımyıldız yönetim\hyp{}merkez dünyalarından bir araya geldi; ve, Salvington’a komşu dünyalardaki konukların büyük çoğunluğu, evrenin idaresine olan geri dönüşünde onu karşılamak için toplandı. Birçok hoş\hyp{}geldin dileklerimize, ve, yaratılmışları ile ilgili bu kadar şevkle ilgili olan bir Egemene hayranlığımızın ifadelerine karşılık olarak, o, sadece şunları söyledi: “Ben, sadece benim Yaratıcım’ın görevini yerine getirmekteydim. Ben, yalnızca, kendi yaratılmışlarını derinden seven ve onları anlamaya can atan Cennet Evlatları’nın zevkini yerine getirmekteyim.
\vs p119 6:6 Ancak, bu günden, Mikâil’in İnsan’ın Evladı olarak kendi Urantia serüvenine çıktığı ana kadar, tüm Nebadon; konukluğunun tüm takımyıldızına ait maddi dünyalardan bir araya gelmiş akranları gibi her açıdan sınanmış bir varlık olarak, evrimsel yükselişin bir morontia fanisinin bahşedilme vücutlaşımı halinde Endantum üzerinde faaliyet gösterirken Egemen Yöneticileri’nin yaptığı birçok şey üzerinde konuşmaya devam etti.
\usection{7.\bibnobreakspace Yedinci ve Son Bahşediliş}
\vs p119 7:1 Binlerce yıl boyunca, hepimiz, Mikâil’in yedinci ve son bahşedilişini dört gözle bekledik. Cebrail bizlere, bu sonuçlayıcı bahşedilişin fani bedenin suretinde gerçekleşeceğini bilgilendirmiş konumdaydı; ancak, bizler, bu tamamlayıcı serüvenin zamanı, mekânı ve biçimi hakkında tamamiyle bilgisizdik.
\vs p119 7:2 Mikâil’in Urantia’yı nihai bahşedilişinin sahnesi olarak seçmiş olduğuna dair herkese gerçekleştirilen duyurusu; bizlerin, Âdem ve Havva’nın görevlerindeki başarısızlıklarını öğrenmesinden yakın bir süre sonra yapıldı. Ve, böylece, otuz beş bin yıldan daha fazla bir süre boyunca, dünyanız, evrenin bütününün sahip olduğu heyetlerde oldukça bariz bir yer işgal etti. Orada, Urantia bahşedilişindeki herhangi bir aşama ile ilgili (vücutlaştırma gizemi dışında) hiçbir sır bulunmamaktaydı. İlkinden, Mikâil’in Salvington’a en yüksek Evren Egemeni olarak nihai ve utkulu dönüşünü olan, sonuncusuna kadar, küçük ancak oldukça onurlandırılmış dünyanız üzerinde gerçekleşmiş her şey ile ilgili bütüncül evren ilgisi bulunmaktaydı.
\vs p119 7:3 Her ne kadar bizler bu yönteme benzer bir biçimde gerçekleşeceğine inanmış olsak da, bu olayın bizzat gerçekleştiği ana kadar, Mikâil’in; âlemin korumasız bir bebeği olarak dünya üzerinde ortaya çıkacağını hiçbir şekilde bilmemekteydik. Bu zamana kadar o her seferinde, kendi bahşediliş tercihinin kişilik topluluğuna ait tümüyle gelişmiş bir birey olarak ortaya çıkmış konumdaydı; ve, Beytüllahim bebeğinin Urantia üzerinde doğmuş olduğunu söyleyen Salvington yayını, çok heyecan verici bir duyuruydu.
\vs p119 7:4 Bizler, bunun sonrasında; yalnızca, Yaratanımızın ve arkadaşımızın, korumasız bir bebek olarak bu bahşedilişte konumunu ve yönetimini görünüşte tehlikeye attığı bir biçimde, tüm süreci içindeki sonu en belirsiz olanını üstlenmekte olduğunun farkına varmamıştık, aynı zamanda, bu nihai ve fani bahşedilişteki deneyiminin, ebedi bir biçimde onu, Nebadon evreninin tartışmasız ve yüce egemeni tahtına oturtturacağını anlamıştık. Dünya zamanına göre bir çağın üçte biri boyunca, bu yerel evrenin tüm kısımları içindeki tüm gözler, Urantia’da odaklanmıştı. Tüm uslar, son bahşedilişin gerçekleşmekte olduğunun farkına varmıştı; ve, Satania içindeki Lucifer isyanı ve Urantia üzerindeki Caligastia sadakatsizliği hakkında uzun bir süredir bilgisi sahibi olan konumda bulunduğumuz için, yöneticimiz, maddi bedenin alçak gönüllü bütünlüğü ve sureti içinde Urantia’da vücutlaşmak için alçaldığında ortaya çıkabilecek mücadelenin yoğunluğunu çok iyi anlamıştık.
\vs p119 7:5 Musevi bebek olarak Yeşu bin Yusuf; tıpkı daha önceki ve o zamana kadar ki tüm diğer bebekler gibi, ancak bu bahse konu bebeğin Cennet’in bir kutsal Evladı’na ilaveten nesnelerden ve varlıklardan oluşan bu yerel evrenin tamamının yaratıcısı olarak Nebadon’un Mikâili’nin vücutlaşımı oluşu \bibemph{dışında}, gebe kalınıp, dünyaya gelmişti. Ve, dünya üzerindeki doğal kökenden başka yapıya ait olarak, İsa’nın insan bütünlüğü içerisindeki İlahiyat’ın vücutlaşımına ait bu gizem, sonsuza kadar çözülemez nitelikte kalacaktır. Ebediyet içinde bile sizler hiçbir zaman, sahip olduğu yaratılmışların bütünlüğünde ve suretinde Yaratan’ın vücutlaşımına ait işleyiş biçimi ve yöntemi bilmeyeceksiniz. Bu, Sonarington’un sırrıdır; ve, bu türden gizemler, bahşedilme deneyiminden geçmiş bu kutsal Evlatlar’ın ayrıcalıklı iyeliğidir.
\vs p119 7:6 Mikâil’in ortaya çıkacak olan varışını, dünyanın belirli bilge kişilileri bilmekteydi. Bir dünyanın diğeriyle olan iletişimleri vasıtasıyla, ruhsal kavrayışa sahip bu bilge insanlar; Urantia üzerinde Mikâil’in bahşedilişinin yaklaşmakta olduğunu öğrendiler. Ve, yüksek melekler, yarı\hyp{}ölümlü yaratılmışlar vasıtasıyla, önderi Ardnon olan Keldani dinadamlarının bir topluluğuna duyuruda bulundu. Tanrı’nın bu insanları, yeni doğmuş çocuğu, bulunduğu ahır yemliği içinde ziyaret etti. İsa’nın doğumu ile ilgili tek doğa\hyp{}üstü olay, Ardnon’a ve onun birlikteliklerine, ilk bahçede Âdem ve Havva’ya atanmış öncül yüksek melek tarafından bu duyurunun yapılışıydı.
\vs p119 7:7 İsa’nın insan ebeveynleri, dönemlerinin ve nesillerinin ortalama insanlarıydı; ve, bu vücutlaştırılmış Tanrı’nın Evladı böylece kadından dünyaya gelmiş nitelikte bulunup, bu ırka ve çağa ait çocukların olağan biçiminde yetiştirilmişti.
\vs p119 7:8 Dünyanız üzerindeki Yaratan Evlat’ın fani bahşedilişine dair anlatı olarak Mikâil’in Urantia üzerindeki konukluğuna ait hikâye, bu makalenin amacı ve kapsamının ötesindeki bir konudur.
\usection{8.\bibnobreakspace Mikâil’in Bahşediliş\hyp{}Sonrası Düzeyi}
\vs p119 8:1 Mikâil’in Urantia üzerindeki son ve başarılı bahşedilişinin ardından, o; sadece Nebadon’un egemen yöneticisi olarak Zamanın Ataları tarafından kabul edilmedi, aynı zamanda, kendi yaratımı olan yerel evrenine kendisini ispatlamış yöneticisi olarak Kâinatın Yaratıcısı tarafından tanındı. Salvington’a dönüşü üzerine, İnsanın Evladı ve Tanrının Evladı olarak bu Mikâil, Nebadon’un asaletini almış yöneticisi olarak duyuruldu. Uversa’dan Mikâil’in egemenliğine dair sekizinci duyuru gelirken, Cennet’den; Tanrı ve insanın bu birlikteliğini evrenin tek başı haline getiren ve Salvington üzerinde konumlanmış olan Zamanın Birliği’nden Cennet’e olan çekişilişini belirtmesini isteyen, Kâinatın Yaratıcısı ve Ebedi Evlat’ın ortak emir duyurusu geldi. Takımyıldız yönetim\hyp{}merkezi üzerinde Zamanın İnançlıları’na, aynı zamanda, En Yüksek Unsurlar’ın heyetlerindeki görevlerini tamamlamalarını isteyen emir verildi. Ancak, Mikâil, danışma ve eşgüdümün Kutsal Üçleme Evlatları’nın çekilmesine razı olmazdı. O; Salvington üzerinde onları bir araya toplayıp, kişisel olarak, Nebadon içinde sonsuza kadar görevlerinde kalmalarını onlardan kişisel olarak talep etti. Onlar, Cennet üzerindeki kendi yöneticilerine, taraflarına yapılmış bu talebe uymayı arzuladıklarını işaret ettiler; ve, yakın bir zaman içerisinde, Nebadon’un Mikâili’nin sahip olduğu yönetime, merkezi evrenin bu Evlatları’nı sonsuza kadar bağlayan Cennet ayrılışının bu emirlerine hüküm veridi.
\vs p119 8:2 Mikâil’in bahşedilme sürecini tamamlamak ve sahip olduğu yaratıma ait evren içinde yüce yönetiminin nihai oluşumunu yerine getirmek, Urantia zamanının neredeyse bir milyar yılını gerektirdi. Mikâil bir yaratan olarak doğmuş, bir idareci olarak eğitilmiş ve bir yönetici olarak hazırlanmıştı; ancak, o egemenliğini, deneyimle kazanmak zorundaydı. Ve, böylece, sizin küçük dünyanız; içinde Mikâil’in, kendi yaratımı olan evrenin sınırsız denetimi ve yönetimi kendisine verilmeden önce her Cennet Yaratan Evladı için zorunlu nitelikte bulunan deneyimi tamamlamış olduğu mekân olarak, tüm Nebadon boyunca bilinir hale geldi. Sizler yerel evrene yükseldiğinizde, Mikâil’in daha önceki bahşedilişleri ile ilgili kişiliklerin ideallerine dair daha çok şey öğreneceksiniz.
\vs p119 8:3 Yaratılmış bahşedilişini tamamlarken, Mikâil; yalnızca kendi egemenliğini oluşturmuş olmuyordu, aynı zamanda, Yüce olan Tanrı’nın evrim halindeki egemenliğini arttırıyordu. Bu bahşedilişlerin gidişatı boyunca Yaratan Evlat; sadece, yaratılmış kişiliğin çeşitli doğalarına yönelik bir alçalış keşfine katılmış olmamıştı, fakat o aynı zamanda, Yüce Yaratanlar tarafından açığa çıkarıldığı biçimiyle, bileşimsel bütünlüğü Yüce Varlık’ın iradesinin açığa çıkarıcısı olan Cennet İlahiyatları’nın farklı şekillerde çeşitlenmiş iradelerinin açığa çıkarılışına da erişmiş olmuştu.
\vs p119 8:4 İlahiyatlar’ın bu çeşitli irade nitelikleri, ebedi bir biçimde, Yedi Üstün Ruhaniyet’in farklılık gösteren doğaları içinde kişileşmiş niteliktedir; ve, Mikâil’in bahşedilmelerinin her biri, bu kutsallık dışavurumlarının birinin özel bir biçimde açığa çıkarıcısı olmuştur. Kendi Melçizedek bahşedilişinde, o, Yaratıcı, Evlat ve Ruhaniyet’in bütünleşmiş iradesini dışa vururken, Lanonandek bahşedilişinde Yaratıcı ve Evlat’ın iradesini dışa vurmuştur; Âdemsel bahşedilişte Yaratıcı ve Ruhaniyet’in iradesini açığa çıkarmışken, yüksek melek bahşedilişinde Evlat ve Ruhaniyet’in iradesini açığa çıkarmıştır; Uversa fani bahşedilişinde Bütünleştirici Bünye’nin iradesini sergilerken, morontia fani bahşedilişinde Ebedi Evlat’ın iradesini sergilemiştir; ve, Urantia maddi bahşedişinde, o, beden ve kanın bir fanisiyken, Kâinatın Yaratıcısı’nın iradesini yerine getirmiştir.
\vs p119 8:5 Bu yedi bahşedilişin tamamlanışı; Mikâil’in yüce egemenliğinin özgürleşimiyle, ve aynı zamanda, Nebadon içindeki Yüce’nin egemenliği için olasılığın yaratımıyla sonuçlandı. Mikâil’in bahşedilmelerinin hiçbirinde o, Yüce olan Tanrı’yı açığa çıkarmamıştır; ancak, yedi bahşedilişin tamamının toplam bütünlüğü, Yüce Varlık’a dair yeni bir Nebadon açığa çıkarılışıdır.
\vs p119 8:6 Tanrı’nın insana olan alçalışının deneyiminde, Mikâil; yüceliğin dışa vurulabilirliliğin sahip olduğu kısıtlılıktan, sınırlı eylemin yüceliğine ve absonit faaliyeti için sahip olduğu potansiyelinin özgürleştirilişine olan kesinliğe yükselmeyi eş zamanlı olarak deneyimlemekteydi. Bir Yaratan Evlat olarak Mikâil, bir zaman\hyp{}mekân yaratanıdır; ancak, yedi\hyp{}katmanlı bir Üstün Evlat olarak Mikâil, Kutsal Üçleme Nihayeti’ni oluşturan kutsal birliklerin birinin üyesidir.
\vs p119 8:7 Kutsal Üçleme’ye ait Yedi Üstün Ruhaniyet iradesini açığa çıkarmanın deneyiminden geçerek, Yaratan Evlat; Yüce’nin iradesini açığa çıkarmanın deneyiminden geçmiştir. Yüceliğin iradesinin bir açığa çıkaranı olarak faaliyet göstererek, Mikâil, tüm diğer Üstün Evlatlar ile birlikte; kendisini ebedi bir biçimde Yüce ile özdeşleştirmiştir. Bu evren çağında, o; Yüce’yi açığa çıkarmakta olup, Yücelik’in egemenliğinin gerçekleşimine katılmaktadır. Ancak, bir sonraki evren çağında, bizler; onun, dış uzayın evrenleri için ve onlar içinde, ilk deneyimsel Kutsal Üçleme içerisinde Yüce Varlık ile işbirliğinde bulunacağına inanmaktayız.
\vs p119 8:8 Urantia; Urantia’nın Gezegensel Prensi, fani bedenin suretinde bir İnsan Evladı, bir morontia ilerleyicisi, yükseliş ruhaniyetlerin bir birlikteliği, bir yüksek melek akran, bir Âdemsel özgürleştiricisi, bir sistem kurtarıcısı, âlemlerin bir Melçizedek hizmetkârı ve tüm Nebadon’un egemeni olarak Mesih Mikâil’in fani evi halindeki on milyon yerleşik dünyanın başında gelen bir biçimde, bütün Nebadon’un duygusal mabedidir. Ve, sizlerin kayıtları; bu aynı İsa’nın, Çarmıhın Dünyası olarak onun sonlandırıcı bahşediliş dünyasına belirli bir süre içinde tekrar geri döneceğine söz vermiş olduğunu ifade ettiğinde, gerçeği dile getirmektedirler.
\separatorline
\vs p119 8:9 [Mesih Mikâil’in yedi bahşedilişini tasvir eden bu makale; Urantia’nın tarihini, Mikâil’in fani beden sureti içinde yeryüzü üzerinde ortaya çıktığı zamana kadar gelen bir biçimde anlatan, sayısız kişilik tarafından sağlanmış, sunumların bir dizisinin altmış üçüncüsüdür. Bu makaleler; Mantutia Melçizedeği’nin yönetimi altında faaliyet gösteren on iki unsurun oluşturduğu bir Nebadon heyeti tarafından onaylanmıştır. Bizler; Urantia zamanının M.S. 1935 yılında, üstlerimiz tarafından onaylanmış bir yöntem vasıtasıyla, bu anlatımları kaleme alıp, İngilizce dilinde yazıya geçirdik.]
