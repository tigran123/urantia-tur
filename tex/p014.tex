\upaper{14}{Merkezi ve Kutsal Evren}
\vs p014 0:1 Kusursuz ve kutsal âlem tüm yaratımın merkezini kaplar; o, zaman ve mekânın çok geniş yaratılmışlarının etrafında döngüsü ebedi temel parçadır. Cennet, muhteşem ebedi evrenin tam da kalbinde hareketsiz bir biçimde ikamet eden devasa çekirdek Ada’nın mutlak sabitliğidir. Bu gezegensel merkezi aile Havona olarak adlandırılmakta olup, kendisi Nebadon’un yerel evrenin çok uzağındadır. O devasa boyutlardan oluşmakta olup, hayal dahi edilemeyecek güzelliğin ve aşkın ihtişamın sayısı bir milyar olan âlemlerinden ve bununla iniltili neredeyse inanılmayacak derece büyük olan kütleden meydana gelmiştir; fakat bu geniş yaratımın gerçek ölçeği, insan aklının anlamaya dönük kavrayışının kesin bir biçimde ötesindedir.
\vs p014 0:2 Bu düzen tek olup; sadece yerleşmiş, kusursuz ve oluşturulmuş dünyaların birlikteliğidir. O bütüncül olarak yaratılmış ve kusursuz olan bir evrendir; bu bakımdan o evrimsel olarak bir gelişim değildir. O; ebedi kusursuzluğun, nihai gerçekliğin, yüce kesinliğin ve kutsal tamamlanmışlığın en yüksek amacı olan yöntemsel evrenin mekân içinde yeniden üretilmesini ve zaman içinde çoğaltılmasını arzulayan Tanrı’nın Yaratan Evlatları’nın cüretkâr serüveni biçimindeki devasa evrimsel deneyimini oluşturan, âlemlerin sonsuz dönüşsel takibinin gerçekleştiği kusursuzluğun ebedi çekirdeğidir.
\usection{1.\bibnobreakspace Cennet\hyp{}Havona Sistemi}
\vs p014 1:1 Cennet’in çevresinden yedi aşkın evrenin içsel sınırlarına kadar, orada şu yedi mekân durumu ve hareketi bulunmaktadır:
\vs p014 1:2 1.\bibnobreakspace Cennet üzerinde etkisi olan hareketsiz orta\hyp{}mekân kısımları.
\vs p014 1:3 2.\bibnobreakspace Üç Cennet döngüsü ve yedi Havona döngüsüne ait saat yönündeki hareket yörüngesi.
\vs p014 1:4 3.\bibnobreakspace Havona döngülerini merkezi evrenin karanlık çekim bünyelerinden ayıran yarı hareketsiz mekân kısmı.
\vs p014 1:5 4.\bibnobreakspace Karanlık çekim bünyelerinin saat yönünün tersinde hareket eden içsel kemeri.
\vs p014 1:6 5.\bibnobreakspace Karanlık çekim bünyelerinin iki mekân doğrultusunu bölen ikincil özgün mekân kısmı.
\vs p014 1:7 6.\bibnobreakspace Cennet etrafında saat yönünde dönen karanlık çekim bünyelerinin dışsal kemeri.
\vs p014 1:8 7.\bibnobreakspace Karanlık çekim bünyelerinin dışsal kemerini yedi aşkın evrenin en iç döngülerinden ayıran, --- bir yarı hareketsiz mekân olarak --- bir üçüncül mekân kısmı.
\vs p014 1:9 Havona’nın milyarları bulan dünyası, Cennet uydularının bu üç döngüsünü eş zamanlı olarak çevreleyen yedi eş merkezli döngüler biçiminde düzenlenmiştir. Orantılı olan müdahil sayılarla birlikte, en iç Havona döngülerinde yirmi\hyp{}beş milyona kadar ve Havona’nın en dış kısım döngülerinde ise iki\hyp{}yüz\hyp{}kırk\hyp{}beş milyon’u aşan dünya mevcut bir halde bulunmaktadır. Her döngü farklılık göstermektedir, fakat bu döngülerin hepsi kusursuz bir biçimde dengelenmiş olup, onlar ender bir zarafetle düzenlenmiştir; buna ek olarak onların her biri, Döngülerin Yedi Ruhaniyeti’nden bir tanesi olan Sınırsız Ruhaniyet’in özelleştirilmiş bir temsillinin nüfuzu altındadır. Diğer işlevlerine ek olarak, bu kişilik dışı Ruhaniyet her döngü boyunca göksel olayların işletimsel işleyişini düzenler.
\vs p014 1:10 Havona gezegensel döngüleri konumsal bakımdan üst üste yerleşen bir mevcudiyeti kendi içinde barındırmaz; bunun yerine onların dünyaları, doğrusal olarak takip eden şekilde birbirlerini ardışık bir biçimde izler. Merkezi evren; Cennet alanlarının sayıca üç ve Havona dünyalarının sayıca yedi olan, sabitleştirilmiş on eş merkezli birimlerinden oluşan çok geniş bir tane düzlemde, hareketsiz Cennet Adası’nın etrafında dönüşünü gerçekleştirir. Fiziksel olarak düşünüldüğünde Havona ve Cennet döngüleri aynı ve tek bir bütün halindeki sistemdir; onların arasındaki ayrım, işlevsel farklılıklarının ve yönetimsel ayrılıklarının tanınmasından kaynaklanmaktadır.
\vs p014 1:11 Cennet üzerinde zaman ona dair varsayılan bir nitelik olarak karşımıza çıkmaz; çünkü birbirini takip eden olayların sırasal boyutu, merkezi Ada’ya özgü olan kişilerin kavramsallaşmasının doğasından kaynaklanmaktadır. Fakat Havona döngüleri ve onun üzerinde geçici bir süreyle ikamet eden göksel veya karasal kökenli sayısız varlık için zaman onların tabi oldukları bir olgudur. Her Havona dünyası bağlı olduğu döngüsü tarafından belirlenen kendisine ait yerel bir zamana sahiptir. Herhangi bir döngü içerisindeki tüm dünyalar eşit sürece haiz bir yıl aralığında sahiptir, çünkü onlar bütüncül olarak aynı döngü içerisinde Cennet etrafında dönmektedirler. Bu duruma ek olarak, bu gezegensel yılların uzunluğu en dışsal döngüden en içe doğru gittikçe azalır.
\vs p014 1:12 Havona\hyp{}döngü zamanının yanı sıra; Sınırsız Ruhaniyet’in yedi Cennet uydusundan gönderilen ve onlar üzerinde belirlenen, Cennet\hyp{}Havona’nın bir günlük zamanı ve bununla beraber diğer zaman kabulleri mevcuttur. Cennet\hyp{}Havona’nın bir günlük zamanı; ilk veya diğer bir değişle en içte bulunan Havona döngüsünün gezegensel yerleşkesinin, Cennet Adası’nın etrafında bir kez dönüşünü tamamlaması için geçen süre ölçüsünde belirlenmiştir. Buna ek olarak onun hızı, karanlık çekim bünyeleri ve devasa Cennet’in arasında olan konumu nedeniyle her ne kadar çok büyük olursa olsun, bu alanların kendi döngüsünü tamamlaması için neredeyse bin yıl geçmesi gerekir. “Tanrı ile bir günün bin yıl kadar sürmesine, ancak bu geçen sürenin bir gece seyri kadar kısa olmasına” dair satırlar üzerinde göz gezdirdiğinizde, farkında olmadan ona dair gerçeğin bilgisine erişeceksiniz. Cennet\hyp{}Havona’nın bir günü sadece yedi dakikadır; bu yedi dakika, Urantia’nın mevcut artık\hyp{}yıl takvimine göre bin yılının üç nokta on sekiz saniyesi kadar daha az olan bir süreçtir.
\vs p014 1:13 Her ne kadar yedi aşkın evrenin her biri kendisine ait yerel zaman birimlendirmelerini taşısa da, bu Cennet\hyp{}Havona günü onların bütünü için merkezi saat ölçümünü oluşturur.
\vs p014 1:14 Havona dünyalarının yedinci kemerinin oldukça uzağında olarak bu geniş merkezi evrenin dışsal kısımlarında, inanılması güç olan bir sayıdaki devasa karanlık çekim bünyeleri dönüşlerini gerçekleştirir. Bu sayıca çok olan karanlık kütleler, biçim olarak bile çok farklı olmalarına ek olarak birçok nitelik bakımından diğer mekân bünyelerinden oldukça ayrı bir konumda bulunur. Bu karanlık çekim bünyeleri ne ışığı yansıtır, ne de onu soğurur; fiziksel\hyp{}enerji ışığı karşısında tepkisel olmayan bir özellik sergilemelerine ek olarak onlar, zaman ve mekânın ikame edilmiş âlemlerin yakınından bile görünmesini engelleyecek şekilde Havona’yı oldukça bütüncül bir biçimde çevreleyip onu gizlemişlerdir.
\vs p014 1:15 Karanlık çekim bünyelerinin bu büyük kemeri, benzersiz olan bir mekân eklentisiyle iki eşit oval döngüye ayrılır. İçsel kemer saat yönü etrafında, dışsal olan ise saat yönüne ters istikamette dönüşünü gerçekleştirir. Karanlık bünyelerinin olağandışı kütlesiyle birleşen bu çok yönlü dönüşümsel hareket; merkezi evreni fiziksel olarak dengeli hale getiren ve onu kusursuz bir biçimde sabitleştirilen yaratımın açığa çıkması olarak gözlenen Havona çekiminin işlev hatlarını çok etkili bir biçiminde birbirine denkleştirir.
\vs p014 1:16 Karanlık çekim bünyelerinin içsel işleyiş düzeni, üç çevresel sınıflandırmadan oluşan düzenleme bakımından borusal bir nitelik gösterir. Bu döngünün bir kesişim noktası, eşit yoğunluk etrafındaki eş merkezli üç daireyi simgeler. Karanlık çekim bünyelerinin dışsal döngüsü, içsel döngüden on bin kez daha fazla olarak dik bir şekilde hizalanmıştır. Dışsal döngünün boylamasına olan çapı, enlemesine olan çapının elli bin katıdır.
\vs p014 1:17 Çekim bünyelerinin iki döngüsü arasında bulunan müdahil mekân, geniş âlemin tümü üzerinde başka hiçbir yerde bulunamayacak olması bakımından \bibemph{benzersizdir}. Bu özgün alan, boylamasına olan bir doğanın devasa dalga hareketleri tarafından tanımlanmasına ek olarak, bilinmeyen bir düzenin çok büyük enerji eylemleri vasıtasıyla nüfuz eder hale getirilir.
\vs p014 1:18 Kanımca merkezi evrenin karanlık çekim bünyeleri dışında ona benzer hiçbir şey, dış mekân düzeylerinin gelecek evrimini ondan daha iyi bir şekilde tanımlayamaz; bu nedenle biz üstün evrende, muhteşem çekim\hyp{}dengeleyici bünyelerin bu dönüşümlü takibini benzersiz olarak algılayıp onu bu biçimde tarif ediyoruz.
\usection{2.\bibnobreakspace Havona’nın Oluşumu}
\vs p014 2:1 Ruhani varlıklar ne belirsiz olan bir mekânda, ne de ruhani dünyalarda ikamet ederler; onlar bunun yerine, fanilerin yaşadığı ve onlar kadar gerçek dünyaların oluşturduğu maddi bir doğanın mevcut alanlarında yerleşiklerdir. Havona dünyaları, her ne kadar onların bütüncül özü yedi aşkın evrenin gezegenlerinin maddi düzenlenişinden farklılık gösterse de, mevcut ve bütünüyle varoluş halindedir.
\vs p014 2:2 Havona’nın fiziksel gerçeklikleri, mekânın evrimsel âlemlerinde egemen olan herhangi bir unsurdan temel olarak farklı bir biçimde, enerji düzenlenmesinin bir seviyesini temsil eder. Havona enerjileri üç katmanlı olup, enerjinin bir biçiminin olumlu ve olumsuz fazlarda mevcut olmasına rağmen enerji\hyp{}maddesinin aşkın evren birimleri iki katmanlı bir enerji etkisi taşır. Merkezi evrenin yaratımı Kutsal Üçleme biçiminde üç katmanlı, bir yerel evrenin yaratımı ise bir Yaratan Evlat ve bir Yaratıcı Ruhaniyet’in doğrudan olan katılımıyla iki katmanlıdır.
\vs p014 2:3 Havona’nın maddesi, Havona enerjisinin yedi biçiminin dengeli işlevi ve tamı tamına bin temel kimyasal yapıtaşının düzenlenmesiyle bir araya gelmiştir. Bu temel enerjilerin her biri, Havona yerlilerinin kırk dokuz uyarıcıyı algılayışına cevap vermesi için uyarımın dokuz düzeyinde açığa çıkar. Diğer bir değişle saf bir fiziksel açıdan bakıldığında, merkezi evrenin yerlileri kırk dokuz adet özelleşmiş duyuyu ellerinde bulundurur. Buna ek olarak morontial duyular yetmiş olup, daha yüksek düzeylerde bulunan unsurların tepkisel karşılıkları yetmişten iki yüz ona kadar değişen sayısal aralıkta bulunur.
\vs p014 2:4 Merkezi evrenin hiçbir fiziksel varlığı Urantialı unsurlar için görülebilir olma özelliği taşımaz. Bu uzak dünyaların herhangi bir fiziksel uyarıcısı, sizin hassaslaşmamış organlarınızda bir karşıt hisse yol açacak etkiyi harekete geçirmez. Eğer bir Urantia fanisi Havona’ya seyahat ettirilse, kendisi orada gözleri görmeyen, kulakları duymayan ve tamamiyle tüm diğer duyusal karşılıklardan yoksun bir durumda olacaktır; bunun karşısında ise orada bu kişi, tüm çevresel uyarıcılar ve onlardan kaynaklanacak bütüncül dışavurumsal karşılıklardan mahrum bir biçimde, sadece öz benliğe sahip sınırlı bir varlık olarak faaliyet gösterecekti.
\vs p014 2:5 Urantia gibi dünyalar üzerinde bilinmeyen, merkezi yaratım içinde gerçekleşen sayısız ruhsal tepki ve fiziksel olgu bulunmaktadır. Üç katmanlı bir yaratımın temel işleyiş düzeni, zaman ve mekânın yaratılmış âlemlerinin iki katmanlı oluşumunun bütünüyle dışındadır.
\vs p014 2:6 Doğa yasalarının bütünü evrimleşen yaratılmışların ikircikli enerji sistemleri bakımından bütünüyle farklı bir belirleyici temel üzerinde düzenlenir. Merkezi evrenin tamamı, kusursuz ve simetrik denetimin üç katmanlı işleyiş düzeniyle uyumlu olarak düzenlenir. Cennet\hyp{}Havona sisteminin bütünü boyunca, kâinatsal gerçekliklerin ve ruhsal kuvvetlerin hepsi arasında kusursuz bir denge sağlanır. Maddi yaratımın mutlak bir kavrayışıyla birlikte Cennet, bu merkezi evrenin fiziksel enerjilerini düzenler ve onların işleyişini sağlar; ruhaniyet kavrayışıyla tamamiyle bütünleşmenin bir parçası olarak Ebedi Evlat ise, Havona üzerinde ikamet eden unsurların ruhsal konumunu en kusursuz bir biçimde muhafaza eder. Cennet üzerinde hiçbir şey deneyimsel değildir, ve Cennet\hyp{}Havona sistemi yaratıcı kusursuzluğun bir birimidir.
\vs p014 2:7 Ebedi Evlat’ın kâinatsal olan ruhsal çekimi merkezi evren boyunca muhteşem bir biçimde etkindir. Tüm ruhaniyet değerleri ve ruhsal kişilikler, sonu gelmez bir biçimde Tanrı’nın yerleşkesine doğru çekim içerisindedirler. Bu Tanrı huzuruna olan ulaşma isteği yoğun ve kaçınılmazdır. Tanrı’ya erişme arzusu merkezi evrende daha güçlüdür; bunun olma sebebi ruhaniyet çekiminin dış evrenlerde daha güçlü olması değil, Havona’ya erişen bu varlıkların daha bütüncül olarak ruhanileştirildiği ve bu sebeple Ebedi Evlat’ın kâinatsal olan ruhaniyet çekiminin başından beri etkin olan eylemine daha baskın bir biçimde karşılık veriyor olmasıdır.
\vs p014 2:8 Benzer bir biçimde Sınırsız Ruhaniyet tüm ussal değerleri Cennet yolu huzuruna çeker. Merkezi evren boyunca Sınırsız Ruhaniyet’in akli çekimi; Ebedi Evlat’ın ruhaniyet çekimiyle iniltili bir biçimde faaliyet içerisinde bulunur, ve birlikte onlar yükselen ruhların Tanrı’yı bulmak, İlahiyat’a erişmek, Cennet’e kavuşmak ve Yaratıcı’yı tanımak biçimindeki bütünleşen arzularını oluştururlar.
\vs p014 2:9 Havona ruhsal olarak kusursuz ve fiziksel olarak sabit bir evrendir. Merkezi evrenin denge ve denetim halindeki sabitliği kusursuz olarak açığa çıkar. Fiziksel veya ruhsal olan her şey tahmin edilebilir bir niteliktedir, fakat aklın olgusallığı ve kişiliğin benliksel iradesini yerine getirmesi böyle bir kalıtsal özelliği beraberinde taşımaz. Günahın imkânsızlığın meydana gelişi olarak tarif edilebilir olduğu kanısındayız; çünkü biz bu kanıya, Havona’nın yerel özgür irade sahiplerinin İlahiyat’ın iradesine karşı gelmedikleri için hiçbir zaman suçluluk duymadıkları temelinden hareketle varmaktayız. Tüm ebediyet boyunca bu tanrısal varlıklar, Zamanın Ebediyetleri’ne kararlı bir biçimde sadık kaldılar. Havona’ya kutsal bir yolcu olarak giriş yapan hiçbir yaratılmışta günahın işlenmesine rastlanmadı. Orada, merkezi Havona evrenine kabul edilen veya orada yaratılmış herhangi bir hiçbir kişilik topluluğun yaratılmışının yanlış bir uygulaması veya hareketi gerçekleşmemiştir. Zamanın âlemlerinde tercih edilen yöntem ve araçlar o kadar kusursuz ve o kadar kutsaldır ki, Havona’nın kayıtlarında ortaya çıkan herhangi bir hataya ve yapılan bir yanlışa rastlanılmadığı gibi yükseliş halindeki hiçbir ruh vaktinden önce bir biçimde merkezi evrene kabul edilmemiştir.
\usection{3.\bibnobreakspace Havona Dünyaları}
\vs p014 3:1 Merkezi evrenin yönetimi ile ilgili hiçbir şey mevcut bir biçimde bulunmamaktadır. Havona eşine az rastlanır bir biçimde o kadar kusursuzdur ki, hiçbir ussal yönetim sistemi onun iradesi için gereklilik arz etmez. Orada ne sık sık düzenli olarak toplanan mahkemeler ne de yasamanın organı olan meclisler bulunmaktadır; Havona sadece idaresel bir gidişata ihtiyaç duyar. Bu noktada gerçek \bibemph{öz yönetimin} nihai amacının en üst noktasına olan erişimi gözlenebilir.
\vs p014 3:2 Böyle bir kusursuz ve ona yakın olan bu ussal yapılar arasında hiçbir yönetime ihtiyaç yoktur. Onlar hiçbir düzenlemeye ihtiyaç duymaz; çünkü onlar evrimsel yaratılmışlara katılmış asli kusursuzluğun varlıkları olarak aşkın evrenlerin yüce yargısının denetiminden çoktan geçmişlerdir.
\vs p014 3:3 Havona’nın idaresi kendiliğinden gerçekleşmez, fakat o hayretler içinde bırakan bir biçimde kusursuz ve kutsal olarak etkilidir. Bu idare başlıca olarak gezegensel olup, yerleşik Zamanın Ebediyeti’nde yetkilendirilmiştir; bununla birlikte her Havona alanı, Kutsal Üçleme’nin kökensel kişiliklerinin biri tarafından idare edilmektedir. Zamanın Ebediyetleri yaratan değillerdir, ancak onlar kusursuz idarecilerdir. Yüce yetenekleriyle öğretimde bulunmalarına ek olarak onlar, gezegensel evlatlarını Mutlaklığa yakın bilgeliğin bir kusursuzluğuyla yönlendirirler.
\vs p014 3:4 Merkezi evrenin milyarı bulan alanları, Havona ve Cennet için yerel olan yüksek kişiliklerinin eğitim dünyalarını oluşturmasına ek olarak, zamanın evrimsel dünyalarından gelen yükseliş halinde olan yaratılmışlar için nihai “kendini kanıtlama” alanları olarak görev yapar. Kâinatın Yaratıcısı’nın yaratılmışın yükselişi için oluşturduğu muhteşem tasarısının uygulanmasında, zamanın kutsal yolcuları yedinci veya diğer bir değişle en dışta olan döngünün varış dünyaları üzerine iniş yapar; bununla birlikte artan öğrenimleri ve genişleyen deneyimlerini izleyen bir biçimde, gezegenden gezegene ve döngüden döngüye sonunda kesin olarak İlahiyatlar’a ulaşıncaya ve Cennet üzerinde yerleşime erişinceye kadar içe doğru ilerlerler.
\vs p014 3:5 Her ne kadar bu yedi döngü alanı hâlihazırda kendilerinin göksel ihtişamının bütününde yerine getiriliyor olsa da; tüm gezegensel kapasitenin sadece yüzde biri kadar olan bölümü Yaratıcı’nın kâinatsal tasarımı olan fani yükselimin ilerleyen safhalarının uygulanmasında kullanılır. Bu devasa dünyaların bahse konu yüzde biri olan kısmının yaklaşık olarak bir bölü onu; Havona dünyaları üzerinde sık sık geçici süreliğine ikamet eden ve orada hizmette bulunan, ebedi olarak yaşamda ve aydınlıkta ikamet eden varlıklar olarak Kesinleştirici Birlikler’in eylemleri ve yaşamına ayrılmıştır. Bu yüceltilmiş varlıklar kendi kişisel yerleşimlerine Cennet üzerinde sahiptirler.
\vs p014 3:6 Havona alanlarının gezegensel inşası, mekânın evrimsel dünyaları ve sistemlerinin yapısal oluşumunun bütünüyle dışındadır. Muhteşem kâinatta başka hiçbir yer, yerleşik dünyalar kadar devasa alanları kullanmaya elverişli değildir. Engin karanlık çekim bünyelerinin dengeleyici etkisiyle bütünleşen triata fiziksel oluşumu, bu olağanüstü derecede büyük olan yaratımın çok çeşitli çekimlerini ender bir biçimde dengelemeyi ve fiziksel kuvvetleri oldukça kusursuz bir biçimde birbirlerine denkleştirmeyi mümkün hale getirir. Karşı\hyp{}çekim aynı zamanda, bu devasa dünyaların ruhsal eylemlerinin ve maddi işleyişlerinin düzenlenmesinde uygulanır.
\vs p014 3:7 Biyolojik özelliklerine ek olarak Havona alanlarının sanatsal hatları biçiminde karşımıza çıkan onun mimarisi, aydınlatması ve ısıtması insan tahayyülünün en nihai sınırlarının bile çok ötesindedir. Havona hakkında hiçbir zaman yeterli bir bilgiye erişemezsiniz, çünkü onun güzelliğini ve ihtişamını anlamak için onu görmeniz gerekmektedir. Yine de belirtmek gerekir ki, bu kusursuz dünyalarda gerçek nehirler ve göller bulunmaktadır.
\vs p014 3:8 Ruhsal olarak bu dünyalar olası en yüksek düzeyde hazırlanmış olup, merkezi evrende faaliyet gösteren farklı varlıkların sayısız düzeylerinin sığınması amacını karşılayan bir biçimde uyarlanmıştır. Çok katmanlı eylemler, insan kavrayışının fazlasıyla ötesinde olan bu güzel dünyalar üzerinde gerçekleşir.
\usection{4.\bibnobreakspace Merkezi Evren’in Yaratılmışları}
\vs p014 4:1 Havona dünyaları üzerinde yaşayan yedi temel varlık ve unsur biçimi bulunmaktadır, bu temel çeşitlerden her biri üç farklı fazda mevcut bulunur. Bu üç fazın her biri yetmiş büyük bölüme ayrılmış olup, her büyük bölüm bin küçük bölümden, alt bölümlerden ve diğerlerinden meydana gelir. Bu temel yaşam birimleri şu biçimlerde sınıflandırılabilir:
\vs p014 4:2 1.\bibnobreakspace Maddi.
\vs p014 4:3 2.\bibnobreakspace Morontial.
\vs p014 4:4 3.\bibnobreakspace Ruhsal.
\vs p014 4:5 4.\bibnobreakspace Absonit.
\vs p014 4:6 5.\bibnobreakspace Nihai.
\vs p014 4:7 6.\bibnobreakspace Eş Mutlak.
\vs p014 4:8 7.\bibnobreakspace Mutlak.
\vs p014 4:9 Yıpranma ve yaşamın son bulması Havona dünyaları üzerinde yaşam çevriminin bir parçası değildir. Merkezi evrende yaşayan unsurların daha düşük düzeyde olanları dönüşümün gerçekleşmesi sürecinden geçerler. Onlar biçin ve görünüm değiştirirler, fakat onlar yıpranma süreci veya hücresel ölüm tarafından dönüşerek yok olmazlar.
\vs p014 4:10 Havona yerlileri tamamiyle Cennet Kutsal Üçleme’sinin doğumudur. Onlar yaratılmış ebeveynden yoksun olup, doğuma yetkin olmayan varlıklardır. Hiçbir zaman yapım süreci sonucunda yaratılmamış varlıklar olarak merkezi evrenin bu vatandaşlarının yaratımını sizlere tasvir edemeyiz. Havona’nın bütüncül geçmişi, fani insanın kavradığı zaman veya mekânla hiçbir ilgisi olmayan, zaman\hyp{}mekânın içinden ifade edilmeye çalışılan bir ebediyet gerçeğidir. Fakat insanın yaratımsal felsefesinin bir başlangıç kökenine sahip olduğunu kabul etmemiz gerekir; insan düzeyinin çok üzerinde olan kişilikler bile “başlangıcın” yapısal bir kavramsallaşmasına ihtiyaç duyarlar. Her ne kadar bu başlangıçsal kavramsallaşma bir etken olsa da, Cennet\hyp{}Havona sistemi ebedidir.
\vs p014 4:11 Havona’nın yerlileri; kendilerine ait yerelliğin alanlarında ikamet eden kalıcı vatandaşların diğer düzeyleri gibi merkezi evren üzerinde milyarı bulan alanlarda yaşarlar. Bir aşkın evrende bir milyar yerel sisteminin maddi, ussal ve ruhsal işleyişini yerine getiren evlatlığın maddi düzeyleri gibi, daha büyük bir kapsamda Havona yerlileri merkezi evrenin milyarı bulan dünyaları üzerinde yaşar ve faaliyette bulunur. Siz muhtemelen, “maddeselliği” kutsal âlemin fiziksel gerçekliklerini tarif edecek bir biçimde anlamsal olarak geniş haliyle kullanıp bahse konu Havonalılar’ı maddi yaratılmışlar şeklinde değerlendirebilirsiniz.
\vs p014 4:12 Orada Havona için yerel olan bir yaşam mevcut olup bu yaşam sadece kendisinde ve kendi içinde önem arz eder. Havonalılar birçok biçimde Cennet’in alçalanlarına ve aşkın evren yükselenlerine hizmet eder, fakat onlar aynı zamanda merkezi evrende benzersiz olan hayatlarını yaşarlar. Buna ek olarak onlar hem Cennet’den hem de aşkın evrenden fazlasıyla ayrık olan göreceli bir anlama sahiptirler.
\vs p014 4:13 Evrimsel dünyaların inanç evlatlarının ibadeti Kâinatın Yaratıcısı’nın sevgisinin memnuniyetini yerine getirirken, Havona yaratılmışlarının yüceltilmiş hayranlığı kutsal güzelliğin ve gerçeğin kusursuz nihai amaçlarını fazlasıyla tatmin eder. Fani insan Tanrı’nın iradesini yerine getirme arzusunu taşırken, merkezi evrenin bu varlıkları Cennet Kutsal Üçlemesi’nin nihai amaçlarını sağlamak için yaşar. Tam da onların sahip oldukları bu doğada, onlar Tanrı’nın iradesinin \bibemph{kendisidir}. Her ne kadar İnsan ve Havonalılar yaşayan gerçeğin bağımsız hizmetinin mevcudiyetinden hoşnut olsa da; İnsan, Tanrı’nın iyiliğinin mevcudiyetinde memnuniyeti, Havonalılar ise kutsal güzellikte çok büyük bir neşeyi bulur.
\vs p014 4:14 Havonalılar tercihlere açık olan mevcut ana ve kendileri için açığa çıkarılmamış gelecek nihai sonlarına sahiplerdir. Orada aynı zamanda sadece merkezi evren için ayrıcalıklı olan, yerli yaratılmışların ait olduğu bir ilerleme durumu söz konusudur; bu ilerleme ne Cennet’e doğru alçalmayla ilişkin olup, ne de aşkın evrenlere katılımla iniltilidir. Yüksek Havona düzeyine olan ilerleme şu biçimlerde gerçekleşebilir:
\vs p014 4:15 1.\bibnobreakspace Birinci döngüden yedinci döngüye olan dışsal biçimdeki deneyimsel ilerleme.
\vs p014 4:16 2.\bibnobreakspace Yedinci döngüden ilk döngüye olan içsel biçimdeki ilerleme.
\vs p014 4:17 3.\bibnobreakspace Herhangi bir döngü içerisinde gelişim biçimindeki döngü içi ilerleme.
\vs p014 4:18 Havona yerlilerine ek olarak, merkezi evrenin içinde ikamet edenler; yaratım boyunca kendi türündeki danışmanların, yöneticilerin ve öğreticilerinin olduğu birçok evren birimleri için sayısız yöntemsel işleyiş sınıfıyla birlikte bütünleşir. Tüm âlemler içindeki bütün varlıklar; Havona’nın milyarı bulan dünyalarının bazılarında, yöntemsel işleyişin içinden türeyen cana sahip olan yaratılmışın bir takım düzeylerinin belirleyicileri boyunca şekillenir. Zamanın fanileri bile, yücelikte olan bu yöntemsel işlev alanlarının dışsal döngüleri üzerinde, yaratılmış mevcudiyetinin kendi amacına ve nihai hedeflerine sahiptir.
\vs p014 4:19 Bunlara ek olarak orada; Kâinatın Yaratıcısı’na erişen, oraya giriş çıkış yetkisine sahip, özel hizmetin görevleri üzerinde âlemler içinde her alanda faaliyet göstermek için atanan bahse konu varlıklar bulunmaktadır. Buna ek olarak her Havona dünyasında, merkezi evrene fiziksel olarak ulaşan erişim adayları bulunabilir; fakat onlar, Cennet’de ikamet etme hakkını sağlamada onları yetkin hale getirecek ruhsal gelişimi henüz elde edememişlerdir.
\vs p014 4:20 Sınırsız Ruhaniyet, Havona dünyaları üzerinde; merkezi evrenin karmaşık ve ruhsal olan olaylarının ayrıntılarını idare eden, ihtişamın ve lütfün varlıkları olarak kişiliklerin bir ev sahipliği tarafından temsil edilir. Kutsal kusursuzluğun bu dünyaları üzerinde onlar, bu geniş yaratımın olağan işleyişine özgü olan sorumluluğu yerine getirirler; buna ek olarak onlar, mekânın karanlık dünyalarından ihtişama yükselen yükseliş yaratılmışlarının sayıca devasa düzeyde olan bireylerine eğitim, öğretim ve yardımdan oluşun çok katmanlı görevleri yerine getirir.
\vs p014 4:21 Orada, yaratılmışın kusursuzluğa erişiminin yükseliş taslağıyla doğrudan hiçbir biçimde birliktelik halinde bulunmayan, Cennet\hyp{}Havona sistemine özgün olan sayısız varlık birimleri mevcuttur; bu nedenle onlar, fani ırklara sunulan kişilik sınıflandırmalardan uzak tutulmuştur. Sadece insan\hyp{}üstü varlıklarının büyük toplulukları ve sizin varlığınızı devam ettirme deneyiminizle doğrudan bağlantılı olan bu düzeydekiler burada temsil edilirler.
\vs p014 4:22 Havona; çabalarında alçak döngülerden daha yüksek olan döngülere ilerlemeyi kutsallığın gerçekleşmesinin yukarı seviyelerine erişmek için, buna ek olarak mutlak gerçekliklerin, nihai değerlerin, yüce anlamların genişleyen takdirine sahip olmak için amaçlayan ussal varlıkların tüm fazları, yaşam ile birlikte bir bütünlük içerisinde birbiriyle etkileşir.
\usection{5.\bibnobreakspace Havona’da Yaşam}
\vs p014 5:1 Urantia üzerinde, maddi varoluşunuzun öncül yaşamında kısa fakat yoğun olan bir denemeden geçersiniz. Bulunduğunuz sistem, takımyıldızı ve yerel evren boyunca konaklama dünyalarınız üzerinde, yükselimin morontia fazlarında katedersiniz. Aşkın evrenin öğrenim dünyalarında ise, ilerlemenin gerçek ruhaniyet düzeyleri boyunca geçişinize ek olarak, Havona’ya nihai erişiminiz için hazır bir hale getirilirsiniz. Havona’nın yedi döngüsü üzerinde sizin erişiminiz ussal, ruhsal ve deneyimseldir. Buna ek olarak, bu döngülerin her birine ait dünya üzerinde erişilmesi gereken belirli bir görev bulunmaktadır.
\vs p014 5:2 Merkezi evrenin kutsal dünyalarında hayatın oldukça zengin ve tamamlanmış olmasının yanı sıra onların üzerinde yaşam o kadar eksiksiz ve doygundur ki, o yaratılmış bir varlığın muhtemel bir biçimde deneyimleyebileceği herhangi bir insan kavramsallaşmasını bütünüyle aşar. Bu ebedi yaratımın toplumsal ve mali eylemleri, Urantia’ya benzer olan evrimsel dünyalarda yaşayan maddi yaratılmışların sahip oldukları mesleklerle hiçbir biçimde benzerlik göstermez. Havona fikirlerinin üretim biçimi bile, Urantia üzerindeki düşünce süreçlerinden farklıdır.
\vs p014 5:3 Merkezi evrenin düzenlemeleri uyumlu ve kalıtsal bir biçimde doğaldır; işleyişin kuralları keyfi değildir. Havona’nın her gerekliliğinde, adaletin idaresi ve doğruluğun nedenselliği mevcut bir halde açığa çıkar. Buna ek olarak bu iki etmen birleşmiş bir halde Urantia üzerinde \bibemph{hakkaniyet} olarak atfedilen kavrama denk düşer. Havona’ya eriştiğinizde şu anki yaşamınızda yapmaktan zevk duyduğunuz şeyleri orada gerçekleştirmekten doğal olarak memnuniyet duyacaksınız.
\vs p014 5:4 Ussal varlıklar ilk olarak merkezi evrene ulaştıklarında, onlar yedinci Havona döngüsünün öncü dünyasında karşılandı ve oraya yerleştirildi. Varışlarını yeni gerçekleştirenler, kendileriyle ilişkili aşkın evren Üstün Ruhaniyeti’nin kimlik kavrayışına erişerek ruhsal olarak ilerlemeye devam ettiğinde, buradan altıncı döngüye ulaştırılır. (Merkezi evrendeki bu düzenlemelerden insan aklında olan ilerleme döngüleri tasarlanmıştır.) Yükseliş halinde olanlar Yücelik’in bir gerçekleştirilişine erişince ve oradan da İlahiyat serüveni için hazırlanınca beşinci döngüye götürülürler; buna ek olarak Sınırsız Ruhaniyet’e erişince dördüncü döngüye aktarılırlar. Ebedi Evlat’a erişimi takiben üçüncü bölüme; ve Kâinatın Yaratıcısı’nı tanımalarıyla birlikte, Cennet ev sahiplerine daha fazla aşina olacakları yer olan ikinci döngü dünyaları üzerinde geçici olarak ikamet etmek için oraya hareket ederler. Havona’nın ilk döngüsüne olan varış Cennet’in hizmetine zamanın bireylerinin kabul edilmesini simgeler. Süresi belirlenmemiş bir biçimde yaratılmışın yükselişinin uzunluğu ve doğasına bağlı olarak, ilerleyici ruhsal erişimin içsel döngüsü üzerinde onlar beklemeye devam eder. Yükseliş halindeki kutsal yolcular bu içsel döngüden Cennet yerleşkesinin içine doğru ve Kesinliğe Erişecek Olanların Birlikleri’nin kabulüne doğru ilerlerler.
\vs p014 5:5 Bir yükselimin kutsal yolcusu olarak Havona üzerinde geçici ikamenizde, tarafınıza atanan döngünün dünyaları arasında özgür bir biçimde seyahat etmenize izin verilecek. Aynı zamanda daha önceden katettiğiniz döngülerdeki gezegenlere geri dönmenize müsaade edilecek. Buna ek olarak, birincil hizmetkâr ruhaniyetleri olarak oluşturulmasına gerek olmaksızın Havona döngüleri üzerinde geçici olarak ikamet eden tüm unsurlar için bahse konu bütün bu durumlar muhtemeldir. Zamanın kutsal yolcuları, “erişilen” mekânı katetmek için kendilerini gerekli bir biçimde hazırlamaya yetkindirler; fakat bu durum, “erişilmemiş” mekân üzerinde yapılacak anlaşma için hükmedilen işleyiş biçimine bağlıdır. Bir kutsal yolcu ne Havona’yı terk edebilir, ne de birincil hizmetkâr ruhaniyetlerinin ulaştırma desteği olmadan kendisine atanan döngünün ötesine geçebilir.
\vs p014 5:6 Bu çok geniş olan merkezi yaratımda zindeliği sağlayan bir özgünlük bulunmaktadır. Ussal varlıkların veya diğer yaşayan unsurların en temel düzeylerinin asli oluşumu ve maddenin fiziksel düzenlenmesinin dışında, Havona’nın dünyaları arasında ortak hiçbir şey yoktur. Bu gezegenlerin her biri özgün, benzersiz ve ayrıcalıklı bir yaratımdır; her gezegen benzersiz, harikulade ve kusursuz bir yapımdır. Buna ek olarak bireyselliğin bu farklılığı, gezegensel mevcudiyetin ruhsal, fiziksel ve ussal olan yönlerinin tüm özelliklerine kadar uzanır. Bu milyarı bulan kusursuz alanlardan her biri, yerleşik Zamanın Ebediyeti’nin tasarıları uyarınca gelişmiş ve göz alıcı hale getirilmiştir. Ve bu durum onların arasından seçilen herhangi bir alanın diğerine neden benzemediğinin sebebidir.
\vs p014 5:7 Havona döngülerinin sonuna kadar seyahatinizi gerçekleştirmeden ve en sonunda arta kalan Havona dünyasını ziyaret etmeden önce, içinde bulunduğunuz maceranın canlandırıcılığı ve merakın teşviki sizin bu kutsal sürecinizde yok olmayacaktır. Ve bunun sonucunda, ebediyetin ileriye yönlendiren uyarıcısı olarak arzu, onun öncülü olan zamanın macera cezp edicisinin yerini alacaktır.
\vs p014 5:8 Tekdüzelik, yaratıcı düşüncenin olgunlaşmamışlığının ve ruhsal edinimle birlikte akli eş güdümün etkisizliğinin göstergesidir. Yükseliş içinde bulunan faninin bu cennetsel dünyaların keşfine başlamasından önce, kendisi ruhsal olmasa da duygusal, ussal, ve toplumsal olgunluğa çoktan erişmiş bir düzeyde olur.
\vs p014 5:9 Havona üzerinde bir döngüden diğerine ilerlerken siz sadece hayal edilmeyen değişiklerle karşılaşmayacaksınız, buna ek olarak sizin şaşkınlığınız her döngü içerisinde bir gezegenden diğerine ilerleyişinizde tarif edilemez bir boyutta olacaktır. Bu milyarı bulan öğrenim dünyalarının her biri şaşkınlıkların gerçek bir üniversitesidir. Bitip tükenmek bilmeyen gerçeği bilme arzusu olarak süregelen şaşkınlıklar, bu döngülerde seyahat eden ve bu devasa alanlarda dolaşan kişilerin deneyimidir. Bu bakımdan tekdüzelik Havona sürecinin bir parçası değildir.
\vs p014 5:10 Evrim halindeki insan doğasının özünde bulunan nitelikler olarak --- serüven aşkı, merak ve tekdüzeliğe karşı duyulan tiksinti; yeryüzü üzerindeki kısa süreli ikameniz süresince sadece sizi sinir etmek veya rahatsızlık duymanıza sebep olmak için oraya konulmamıştır, bunun yerine onlar ölümün, keşfin ebedi bir yolculuğu, beklentinin sonsuza kadar sürecek bir yaşamı biçiminde serüvenin sonu gelmez bir sürecinin sadece başlangıcı olduğunu sizlere önermek için verilmiştir.
\vs p014 5:11 Araştırmanın ruhu, keşfin dürtüsü ve incelemenin güdüsü olarak merak, evrimsel mekân yaratılmışlarının kutsal ve onların doğasından gelen edinimlerinin bir parçasıdır. Bu doğal dürtüler sadece bastırılması veya engellenilmesi için verilmemiştir. Bu arzu sahibi dürtülerin dünya üzerindeki kısa olan yaşamınız sürecinde sık sık kısıtlandığı ve hayal kırıklıklarının tekrar eden biçimlerde deneyimlendiği gerçektir, fakat onlar gelecek çok uzun asırlar boyunca bütüncül olarak gerçekleştirilip muhteşem bir biçimde tatmin edilecektir.
\usection{6.\bibnobreakspace Merkezi Evren’in Nihai Amacı}
\vs p014 6:1 Yedi döngüden oluşan Havona eylemlerinin kapsamı devasa bir büyüklüktedir. Genel olarak bu kapsam şu başlıklar altında tarif edilir.
\vs p014 6:2 1.\bibnobreakspace Havonasal.
\vs p014 6:3 2.\bibnobreakspace Cennetsel.
\vs p014 6:4 3.\bibnobreakspace Evrimsel Nihai\hyp{}Yücelik olarak sınırlı\hyp{}yükselen.
\vs p014 6:5 Aşkın olan birçok sınırlı eylem, ruhaniyet ve akıl işlevlerinin absonit ve diğer fazlarının içinde dile getirilmeyen farklılıklarına katılan Havona’nın mevcut evren çağında gerçekleşir. Merkezi evrenin yaratılmış aklın kavrayışının ötesinde birçok şekilde faaliyet göstermesi gibi, tıpkı onun benim için açığa çıkarılmayan birçok nihai amaca hizmet etmesi olanak dâhilindedir. Yine de bu kusursuz yaratımın, kâinat aklının yedi düzeyinin ihtiyaçlarını karşılamada nasıl hizmet ettiğini ve onların memnuniyetine hangi biçimlerde katkıda bulunduğunu tasvir etmeye çabalayacağım.
\vs p014 6:6 1.\bibnobreakspace \bibemph{Kâinatın Yaratıcısı} --- Birincil Kaynak ve Merkez. Yaratıcı olan Tanrı, yüce ebeveynsel memnuniyetini merkezi yaratımın kusursuzluğundan elde eder. Kendisi yakın\hyp{}eşitlik düzeylerinde sevgiye doyum deneyiminin hoşnutluğunu taşır. Kusursuz Yaratan kusursuz yaratımın hayranlığıyla kutsal olarak hoşnuttur.
\vs p014 6:7 Havona, Yaratıcı’nın yüce erişiminin memnuniyetini yerine getirir. Havona içinde kusursuzluğun yerine getirilmesi, sınırsız genişlemenin ebedi istenci içinde zaman\hyp{}mekân kısıtlanması dolayısıyla yaşanan gecikmeyi telafi eder.
\vs p014 6:8 Yaratıcı, kutsal güzellik karşısında Havona’nın onu tamamlayan karşılığından memnuniyet duyar. Bu durum, tüm evrimleşen âlemler için ender uyumun kusursuz bir işleyişsel yöntemini sağlamak için kutsal aklı hoşnut kılar.
\vs p014 6:9 Yaratıcımız merkezi evreni kusursuz bir memnuniyetle karşılar; çünkü o, kâinat âlemlerinin tümünün bütün kişilikleri için ruhaniyet gerçekliğinin bu değere layık bir açığa çıkarılmış halidir.
\vs p014 6:10 Âlemlerin Tanrı’sı, zaman ve mekân üzerinde birbirlerini peşi sıra takip eden evren genişlemesinin bütünü için, ebedi güç çekirdeği olarak Havona ve Cennet’i ayrı bir gözle değerlendirir.
\vs p014 6:11 Ebedi Yaratıcı, Yaratan\hyp{}Yaratıcı’nın ebedi evine erişen kendisinin mekâna ait fani torunları olarak zamanın yükseliş adayları için değerli ve cezp edici bir hedef olarak Havona yaratımını bitmek tükenmek bilmeyen bir hoşnutlukla değerlendirir. Buna ek olarak Tanrı, kutsal aile ve İlahiyat’ın ebedi evi olarak Cennet\hyp{}Havona’nın kendisinden haz duyar.
\vs p014 6:12 2.\bibnobreakspace \bibemph{Ebedi Evlat} --- İkincil Kaynak ve Merkez. Mümkün olan en üst düzeydeki merkezi yaratım Ebedi Evlat için; Yaratıcı, Evlat ve Ruhaniyet olarak kutsal ailenin birliksel işlevselliğinin ebedi kanıtını sağlar. Bu durum, Kâinatın Yaratıcısı’nın içindeki mutlak huzurun maddi ve ruhsal altyapısını oluşturur.
\vs p014 6:13 Havona, ruhaniyet gücünün en başından beri genişleyen kendisini gerçekleştirişinin neredeyse sınırsız olan kaynağını Ebedi Evlat için sağlar. Merkezi evren Ebedi Evlat’a bu alanı; onun içinde güvenli ve emin bir biçimde, birliktelik halinde bulunduğu Cennet Evlatları’nın eğitimi için ruhsal ve biçimsel bahşediş hizmetini öğretmesi amacıyla tahsis etti.
\vs p014 6:14 Havona, Ebedi Evlat’ın sahip olduğu kâinatın âlemlerinin tümünün ruhani\hyp{}çekim düzenlenmesinin özsel oluşumudur. Bu merkez, ruhsal çoğalma olan ebeveynsel arzunun tatminini sağlar.
\vs p014 6:15 Havona dünyaları ve onların kusursuz sakinleri, Yaratıcı Sözü olan Evlat’ın ilk ve nihai olarak son temsilidir. Bu bakımdan Evlat’ın bilinci, kendisinden kusursuz bir biçimde hoşnut olan Yaratıcı’nın sınırsız bir tamamlayıcısıdır.
\vs p014 6:16 Buna ek olarak bu âlem, Kâinatın Yaratıcısı ve Ebedi Evlat arasında bulunan eşitliğin birlikteliğinin içerisinde onun tamamlayıcı karşılığının gerçekleşmesi için imkân sağlar, ve bu durum her birinin sınırsız kişiliğinin sonsuza kadar sürecek olan kanıtını oluşturur.
\vs p014 6:17 3.\bibemph{ Sınırsız Ruhaniyet} --- Üçüncül Kaynak ve Merkez. Havona âlemi, bütünleşen Yaratıcı\hyp{}Evlat’ın sınırsız temsilcisi olarak Bütünleştirici Bünye’nin Sınırsız Ruhaniyet olduğu kanıtını sağlar. Sınırsız Ruhaniyet; bu kutsal meydana gelişin mutlak eş mevcudiyetinin yerine getirilmesinden hoşnut olurken, bir diğer yandan ise yaratıcı bir eylem olarak bütünleşen faaliyetsel tatmini Havona içinde elde eder.
\vs p014 6:18 Havona içinde Sınırsız Ruhaniyet, olası bağışlama yardımcısı olarak hizmet etmenin yetisini ve istekliliğini gösterebileceği bir alan bulmuştur. Bu kusursuz yaratımda Ruhaniyet, evrimsel âlemlerde hizmetin serüveni için alıştırmalarını gerçekleştirmiştir.
\vs p014 6:19 Bu kusursuz yaratım Sınırsız Ruhaniyet’e; yardımcı\hyp{}Yaratan doğumu olarak bir evreni idare etmek için kutsal ebeveynleriyle birlikte kâinat yönetimine katılma, ve bu bakımdan Yaratan Evlatlar’ın yardımcıları olarak Yaratıcı Ruhaniyet biçiminde yerel evrenlerin birleşik idaresinin hazırlanması imkânını sağlamıştır.
\vs p014 6:20 Havona dünyaları; mevcudiyet içerisinde bulunan her yaratılmış akıl için belirli yardımcılar olup, kâinatsal usun yaratanlarının akıl laboratuarıdır. Akıl her Havona dünyası üzerinde farklıdır, ve o tüm ruhsal ve maddi yaratılmış akli yapıları için bir işleyiş yöntemi olarak hizmet eder.
\vs p014 6:21 Bu kusursuz dünyalar, Cennet toplumuna erişim amacıyla nihai olarak belirlenmiş tüm varlıklar için akli yüksek\hyp{}lisans programlarıdır. Onlar Ruhaniyet’e, güvenli ve fikir veren kişilikler üzerinde akli hizmet biçiminin denenmesinin olanağını fazlasıyla sağladı.
\vs p014 6:22 Havona, mekânın âlemlerinde Sınırsız Ruhaniyet’in bencil olmayan ve yaygın çalışmaları için ona tahsis edilen bir yerdir. Aynı zamanda Havona, zaman ve mekânın yorulmak bilmez Akıl Hizmeti’nin kusursuz bir evi ve istirahat yerleşkesidir.
\vs p014 6:23 4.\bibnobreakspace \bibemph{Yüce Varlık} --- deneyimsel İlahiyat’ın evrimsel birleşimi. Havona yaratımı Yüce Varlık’ın ruhsal gerçekliğinin ebedi ve kusursuz kanıtıdır. Bu kusursuz yaratım; zaman ve mekânın deneyimsel âlemleri içinde Cennet İlahiyatları’nın sınırlı yansımalarının güç\hyp{}kişilik bütünleşmesinin başlangıcından önce, Yüce olan Tanrı’nın simetrik ve kusursuz olan ruhaniyet doğasının bir açığa çıkarılışıdır.
\vs p014 6:24 Havona içinde Her Şeye Gücü Yeten’in güç olanakları Yücelik’in ruhsal doğasıyla birlikte bütünleşmiş bir haldedir. Bu merkezi yaratım, Yücelik’in gelecek\hyp{}ebediyet birlikteliğinin bir örneğidir.
\vs p014 6:25 Havona, Yücelik’in kâinatsal potansiyelinin kusursuz bir işleyişsel yöntemidir. Bu evren, Yücelik’in gelecek kusursuzluğunun tamamlanmış bir tasviri olup Nihayet’in potansiyeli mevcudiyetinin hatırlatıcısıdır.
\vs p014 6:26 Havona; kusursuz öz benlik denetimi ve yüceliğin yaşayan irade sahibi yaratılmışlarının varoluşu, ruhaniyetin nihai bir biçimde dengi olan aklın mevcudiyeti, ve sınırsız bir olanaklılıkla aklın gerçekliği ve birliği olarak ruhani değerlerin kesinliğini sergiler.
\vs p014 6:27 5.\bibnobreakspace \bibemph{Eş Güdüm Halindeki Yaratan Evlatlar}. Havona, Cennet Mikâilleri’nin evren yaratımındaki bir sonraki serüvenleri için hazır hale getirildikleri eğitimsel hazırlama merkezidir. Bu kutsal ve kusursuz yaratım, her Yaratan Evlat için işleyişsel bir yöntemdir. Kendisi evreninin, kusursuzluğun bu Cennet\hyp{}Havona düzeylerine nihai olarak erişimini sağlamak için çaba gösterir.
\vs p014 6:28 Bir Yaratan Evlat, kendi fani evlatları ve ruhaniyet varlıkları için Havona yaratılmışlarını kişiliğin\hyp{}işleyişsel yöntemi olarak kullanır. Mikâil ve diğer Cennet Evlatları, Cennet ve Havona’yı zamanın evlatlarının kutsal nihai sonu olarak görürler.
\vs p014 6:29 Yaratan Evlatlar, yerel evrenlerini birleştiren ve istikrarlı hale getiren hayati evren üst denetiminin gerçek kaynağının merkezi yaratım olduğunun bilincindedirler. Onlar aynı zamanda, Yücelik’in ezelden beri var olan etkisinin kişisel mevcudiyetinin ve Nihayet’in Havona’nın içinde mevcut olduğunun bilgisine sahiptirler.
\vs p014 6:30 Havona ve Cennet, bir Mikâil Evladı’nın yaratıcı gücünün kaynağıdır. Burada onunla evren yaratımında eş güdüm halinde olan varlıklar ikamet eder. Evren Ana Ruhaniyetleri olarak bu yerel âlemlerin eş yaratanları Cennet’den gelmektedirler.
\vs p014 6:31 Cennet Evlatları merkezi yaratımı kutsal ebeveynlerinin evi olarak kendi yerleşkeleri biçiminde görürler. Burası onların zaman geri döndükleri ve ziyaret etmekten hoşnut kaldıkları bir yerdir.
\vs p014 6:32 6.\bibnobreakspace \bibemph{Eş Güdüm Halinde olan Yardımcı Kız Evlatlar}. Yerel evrenlerin eş yaratanları olarak Evren Ana Ruhaniyetleri, Döngülerin Ruhaniyetleri’yle yakın bir birliktelikte Havona dünyaları üzerinde kendi kişilik öncesi eğitimini tamamlarlar. Bu merkezi evrende yerel evrenlerin Ruhaniyet Kız Evlatları, Yaratıcı’nın iradesine her zaman tabi olan Cennet Evlatları’yla birlikte eş güdüm yöntemleri içinde uygun bir biçimde eğitilmişlerdir.
\vs p014 6:33 Havona dünyaları üzerinde Ruhaniyet ve Ruhaniyet’in Kız Evlatları, kendi topluluklarının tümünün ruhsal ve maddi akli yapılarının ussal işleyiş yöntemleriyle karşılaşır; buna ek olarak bu merkezi evren, bir Evren Ana Ruhaniyeti’nin birleşik bir biçimde ilişkili bir Yaratan Evlat’la desteklediği bu yaratılmışların gelecekteki nihai sonudur.
\vs p014 6:34 Evren Ana Yaratanı, Cennet ve Havona’yı kendi kökensel yeri ve Sınırsız Akıl’ın kişilik mevcudiyetinin yerleşkesi olan Sınırsız Ana Ruhaniyet’in evi olarak hatırlar.
\vs p014 6:35 Bir Evren Kutsal Hizmetkârı’nın yaşayan irade sahibi yaratılmışların oluşturulması görevinde, bir Yaratan Evlat için tamamlayıcı olarak uyguladığı yaratma yetisinin kişisel ayrıcalıklarının bahşedişi bu merkezi evrenden gelmiştir.
\vs p014 6:36 Ve son olarak, Sınırsız Ana Ruhaniyet’in bu Kız Evlat Ruhaniyetleri kendilerine ait Cennet evine muhtemelen bir daha geri dönmeyecekleri için, Cennet üzerinde Majeston’da kişileşen ve Havona’da Yüce Varlık ile birliktelik halinde bulunan kâinatsal yansımanın olgular bütününden büyük bir haz duyarlar.
\vs p014 6:37 7.\bibnobreakspace \bibemph{Yükselim Sürecinin Evrimsel Fanileri}. Havona, her fani türünün işleyişe dayanan yöntemsel kişiliğinin evi olup, zamanın yaratılmışlarına özgü olmayan fani birlikteliğinin insan\hyp{}üstü kişiliklerinin bütünün ana yerleşkesidir.
\vs p014 6:38 Bu dünyalar, algılanabilecek en yüksek gerçeklik düzeyleri üzerinde gerçek ruhaniyet değerlerinin erişimine karşı tüm insan duyularının uyarıcılığını sağlar. Havona, yükseliş halinde bulunan her fani için Cennet öncesi eğitim amacının hayata geçtiği yerdir. Havona, Cennet ve Tanrı’ya olan erişim öncesinde bir ana kapı olarak her irade sahibi yaratılmış için mevcut bir halde bulunur.
\vs p014 6:39 Cennet bir ev, Havona ise kesinliğe erişecek olanlar için bir atölye ve keyif veren mutluluk yeridir. Bununla birlikte, Tanrı’yı tanıyan herkes kesinliğe erişeceklerden biri olmayı çok derin bir biçimde arzular.
\vs p014 6:40 Merkezi evren, insan için oluşturulmuş sadece nihai bir son değildir; burası aynı zamanda, Kâinatın Yaratıcısı’nın sınırsızlığını keşfetmenin deneyiminde açığa çıkmamış ve evrensel olan serüven içinde kesinliğe erişeceklerin ebedi sürecinin gelecekte yola koyulacakları başlangıç yeridir.
\vs p014 6:41 Havona, aşkın sınırlı düzeyler üzerinde Tanrı’yı bulmak için çabalayan mekânın kutsal yolcularına şahit olacak gelecek evren çağları üzerinde bile, absonitsel önemiyle birlikte kuşkusuz olarak faaliyet göstermeye devam edecektir. Havona, absonit varlıklar için eğitim evreni olarak hizmet verme yetkinliğine sahiptir. Dış uzayın ilkokulunun mezunları için ortaokul olarak faaliyet gösteren yedi aşkın evrenin mevcudiyetsel varlığı göz önüne alındığında, Havona bu benzetme içerisinde muhtemelen eğitimin tamamlandığı okul olacaktır. Bu bakımdan, Havona’nın ebedi olanaklarının gerçek bir biçimde sınırsız olduğuna; ve merkezi evrenin, yaratılmış varlıkların geçmiş, mevcut ve gelecek biçimlerinin tümü için deneyimsel eğitim evreni olarak hizmet etmesinin ebedi yetisine sahip olduğu fikrini onaylıyoruz.
\vs p014 6:42 [Uversa üzerindeki Zamanın Ataları tarafından bu bağlamda faaliyet göstermesi için görevlendirilen bir Bilgelik Kusursuzlaştırıcısı tarafından sunulmuştur.]
