\upaper{42}{Enerji --- Akıl ve Madde}
\vs p042 0:1 Evrenin oluşumu; mevcudiyetin tümünün temelinde enerjinin bulunması bakımından maddi olup, saf enerji Kâinatın Yaratıcısı tarafından düzenlenmektedir. Enerji olarak kuvvet, Kâinatsal Mutlak’ın mevcudiyeti ve varlığını gösteren ve onu doğrulayan bir biçimde sonsuza kadar ayakta kalan bir abidedir. Cennet Mevcudiyetleri’nden ilerleyen enerjinin bu engin akımı, hiçbir zaman başarısız olan bir biçimde doğrultusundan uzaklaşmamıştır; sınırsız tedarik içerisinde hiçbir zaman bir kesinti gerçekleşmemiştir.
\vs p042 0:2 Evren enerjisinin düzenlenmesi her zaman, Kâinatın Yaratıcısı’nın kişisel iradesi ve onun bütünüyle\hyp{}bilge olan emirleri uyarınca gerçekleşmektedir. Açığa çıkan gücün ve döngü halindeki enerjinin bu kişisel düzenlenişi; Bütünleştirici Bünye tarafından idare edilen Evlat ve Yaratıcı’nın bütüncül amaçlarına ek olarak Ebedi Evlat’ın eş güdümsel eylemleri ve kararları tarafından değişikliğe uğramaktadır. Bu kutsal varlıklar kişisel ve bireysel olarak hareket etmektedirler; onlar aynı zamanda, kâinat âlemlerinin tümü içinde ebedi ve kutsal amacın her birini değişken bir biçimde yansıtan, neredeyse sınırsız bir sayıda bulunan altlarındaki görevlilerinin kişileri ve güçleri içerisinde faaliyet göstermektedir. Fakat kutsal gücün bu işlevsel ve geçici değişimleri veya dönüşümleri hiçbir biçimde, tüm güç\hyp{}enerjisinin her şeyin merkezinde ikamet eden kişisel bir Tanrı’nın nihai denetiminde olduğuna dair gerçeği dışlamamaktadır.
\usection{1.\bibnobreakspace Cennet Kuvvetleri ve Enerjileri}
\vs p042 1:1 Evrenin oluşumu maddidir, ancak yaşamın özü ruhanidir. Ruhaniyetlerin Yaratıcısı aynı zamanda evrenlerin atasıdır; Özgün Evlat’ın ebedi Yaratıcısı aynı zamanda, Cennet Adası olarak özgün yaratım biçiminin ebediyet kökenidir.
\vs p042 1:2 Enerji olarak madde; aynı kâinatsal gerçekliğin farklılaşan dışavurumları olduğu için, bir evren olgusu olarak Kâinatın Yaratıcısı’nın içerisinde içkin bir şekilde bulunmaktadır. “Onun içinde mevcut hale gelmektedir.” Madde içkin enerjiyi dışa vuran bir biçimde ortaya çıkıp, kendi içinde taşıdığı güçleri sergileyebilir; ancak bahse konu bu tüm fiziksel olgular bütününe katılan enerjileri etkileyen çekimin hatları, Cennet’ten kökenini almakta olup ona bağımlıdır. Enerjinin ilk ölçülebilen türü olan ultimaton, çekirdeği olarak Cennet’e sahiptir.
\vs p042 1:3 Madde içerisinde içkin bir biçimde var olan ve kâinatsal uzay içerisinde mevcut olan Urantia üzerinde bilinmeyen enerjinin bir türü bulunmaktadır. Bu keşif bütünüyle yapıldığında, bunun sonrasında fizikçiler en azından maddenin gizemini çözdüklerini hissedeceklerdir. Ve böylelikle onlar Yaratan’a bir adım daha yaklaşacaklardır; ve onlar kutsal işleyiş biçiminin bir fazını daha üstünlükle geçeceklerdir; fakat onlar ne herhangi bir biçimde Tanrı’yı bulmuş olacaklar, ne de maddenin mevcudiyetini ve ne de Cennet’in kâinatsal işleyiş biçiminden ve Kâinatın Yaratıcısı’nın devinimsel amacından kökenini alan doğal yasaların işleyişini oluşturacaklardır.
\vs p042 1:4 Daha büyük ilerlemelerin ve daha ileri keşiflerin ardından, mevcut bilgiyle kıyaslanamayacak bir biçimde Urantia ilerleme gösterdikten sonra bile, maddenin elektriksel birimlerine ait enerji döngülerinin fiziksel dışavurumlarının değişikliğe uğratılması bakımından düzenlenmesine erişmiş bir düzeye gelecek olmanıza rağmen --- bu türden olası ilerlemelerin sonrasında bile ---, bilim adamları maddenin bir atomunu yaratmada, enerjinin bir ışıltısını oluşturmada veya yaşam olarak adlandırdığımız hayata bir madde bile katmada sonsuza kadar yetersiz olacaklardır.
\vs p042 1:5 Enerjinin yaratımı ve yaşamın bahşedilmesi, Kâinatın Yaratıcısı’nın ve onun birliktelik halindeki Yaratan kişiliklerinin ayrıcalığıdır. Enerji ve yaşamın ırmağı; tüm mekâna yayılan Cennet kuvvetinin kâinatsal ve bütünleşmiş akıntısı biçiminde, İlahiyatlar’dan boşalan devamlı bir nehirdir. Bu kutsal enerji tüm yaratımı içine almaktadır. Kuvvet düzenleyicileri bu değişimleri başlatıp, enerji içinde mevcut kılınan mekân\hyp{}kuvvetin bu dönüşümlerini oluşturmaktadır; güç idarecileri, enerjiyi maddeye aktarırlar; böylelikle maddi dünyalar doğmuş olur. Yaşam Taşıyıcıları, maddi hayat biçiminde yaşam olarak adlandırdığımız ölü madde içerisinde bu süreçleri başlatırlar. Morontia Güç Yüksek Denetimcileri buna benzer bir biçimde, maddi ve ruhsal dünyalar arasındaki geçiş âlemleri boyunca faaliyet göstermektedir. Daha yüksek ruhaniyet Yaratanları enerjinin kutsal türleri içinde benzer süreçleri hayata geçirir, ve orada ussal yaşamın daha yüksek ruhani türleri ortaya çıkar.
\vs p042 1:6 Kutsal düzey sonrasında şekillenmiş olduğu biçiminde, enerji Cennet’ten yayılır. Saf enerji olarak enerji, kutsal düzenin doğasından kaynağını almaktadır; enerji, üç Tanrı kâinat âlemlerinin yönetim merkezinde faaliyet gösterirken, onların benzeşimi sonrasında şekillenir, ve bir bütün haline gelir. Ve kuvvetin tümü; Cennet Mevcudiyetleri’nden gelen ve tekrar oraya dönen bir biçimde Cennet içerisinde döngüsel hale getirilmektedir, ve onun kökeni Kâinatın Yaratıcısı olarak sebepsiz Sebep’in bir dışavurumunun özünde bulunmaktadır; ve Yaratıcı olmadan ne herhangi bir şey mevcudiyet kazandı, ve ne de o olmadan herhangi bir şey ne mevcut bir hale gelirdi.
\vs p042 1:7 Kendiliğinden mukabil olan İlahiyat’dan elde edilen kuvvet, kendi başına hiçbir zaman mevcut değildir. Kuvvet\hyp{}enerji yok edilemez bir biçimde ortadan kaldırılamaz; Sınırsız’ın bu dışavurumları sınırsız değişimler, sonu olmayan dönüşümler ve ebedi başkalaşımlara bağlı olabilir; fakat hiçbir biçimde veya düzeyde, hayal edilebilecek en ufak bir ölçekte bile, onlar en başından beri bütünlüklerinin kaybına uğramamıştır ve uğramayacaktır. Fakat her ne kadar enerji kaynağını Sınırsız’dan alsa da, sınırsız bir biçimde açığa çıkmamaktadır; mevcut haliyle oluşturulmuş üstün evrenin dışsal sınırları bulunmaktadır.
\vs p042 1:8 Enerji ebedidir fakat sınırsız değildir; o hiçbir zaman Sınırsız’ın bütüncül her şeyi kavrayışına karşılık göstermemektedir. Sonsuza kadar kuvvet ve enerji varlığını sürdürür; Cennet’ten kaynağını aldıkları için onlar, emredilen döngünün tamamlanması için çağlar gerekse bile, buraya tekrar geri dönmek zorundadır. Cennet İlahiyat kökenine ait onlar unsurlar, yalnızca bir Cennet istikametine veya bir İlahiyat nihai sonuna sahiptir.
\vs p042 1:9 Tüm bu bilgilerin tümü; kâinat âlemlerinin tümünün döngüsel ve bir anlamda sınırlı, ancak düzenli bir biçimde ve uçsuz bucaksız nitelikte bulunduğuna dair bizim inancımızı desteklemektedir. Eğer bu yargı doğru olmasaydı, enerji tükenişine dair kanıt er veya geç en sonunda ortaya çıkardı. Tüm yasalar, düzenlenmeler, idareler ve evren kâşiflerinin tanıklığı olarak her şey; sınırsız bir Tanrı’nın mevcudiyetini işaret etmekte olup, ancak yine de bu sınırsızlığın karşıtı olarak, neredeyse sonu olmayan bir şekilde sonsuz mevcudiyetinin bir döngüsü olarak sınırlı bir evreni göstermektedir.
\usection{2.\bibnobreakspace Kâinatsal Ruhsal Olmayan Enerji Sistemleri\\(Fiziksel Enerjiler)}
\vs p042 2:1 Fiziksel, akılsal veya ruhsal olarak kuvvet ve enerjinin çeşitli düzeylerini İngilizce dili vasıtasıyla adlandırmak ve onun içerisinde tanımlamak için yerinde kelimeler bulmak gerçek anlamıyla zordur. Bu anlatımlar bütüncül olarak, sizin kabul ettiğiniz kuvvet, enerji ve güç tanımlamalarını birebir karşılamamaktadır. Dil içerisinde mevcut bulunan kısıtlılıklardan dolayı bu terimleri çoklu anlamlar altında kullanmak zorundayız. Bu makalede, örnek olarak, \bibemph{enerji} kelimesi olgusal hareketin eylemin ve potansiyelin türlerini ve tüm fazlarını adlandırmak için kullanılmıştır; bunun karşısında \bibemph{kuvvet} enerjinin çekim öncesi ve \bibemph{güç} ise enerjinin çekim sonrası aşamaları anlamında kullanılmaktadır.
\vs p042 2:2 Buna rağmen ben, fiziksel enerji biçimindeki kâinatsal kuvvet, ortaya çıkan enerji ve evren gücü için şu sınıflandırılmayı kullanmayı önererek kavramsal karışıklığı azaltmaya çalışacağım:
\vs p042 2:3 1.\bibnobreakspace \bibemph{Mekân Gücü}. Bu güç, Koşulsuz Mutlaklık’ın sorgulanmayan özgür mekân mevcudiyetidir. Bu kavramın uzantısı, Koşulsuz Mutlaklık’ın işlevsel bütünlüğü içinde evren kuvvet\hyp{}mekân potansiyelini çağrıştırır. Bunun karşısında bu kavramın içeriği; başlangıcı ve sonu olmayan, hiçbir şekilde hareket etmeyen ve değişikliğe uğramayan Cennet Adası’ndan ebediyet bilgeliğini alan evrenler biçimindeki kâinatsal gerçekliğin bütünlüğünü karşılamaktadır.
\vs p042 2:4 Bahse konu bu olgular bütünlüğü; Cennet’in alt tarafına özgü olup, muhtemel bir biçimde mutlak kuvvet mevcudiyeti ve dışavurumunun üç bölgesi ile bütünleşir. Bu bölgeler; Koşulsuz Mutlaklık’ın temel bölgesi, Cennet Adası’nın kendi bölgesi, ve belirli bir takım tanımlanmamış eşitleyici ve telafi edici birimler veya faaliyetlerin ara bölgesidir. Bu üçlü eş merkezsel bölgeler, kâinatsal gerçekliğe ait Cennet döngüsünün ana merkezidir.
\vs p042 2:5 Mekân gücü, gerçeklik öncesi bir birimdir; her ne kadar Öncül Üstün Kuvvet Düzenleyicileri’nin mevcudiyeti tarafından değişikliğe uğratılabilir olarak görünse de bu birim, Koşulsuz Mutlaklık’ın nüfuz alanına ait olup, yalnızca Kâinatın Yaratıcısı’nın kişisel kavrayışına karşılık göstermektedir.
\vs p042 2:6 Uversa üzerinde, mekân gücü ABSOLUTA olarak adlandırılmaktadır.
\vs p042 2:7 2.\bibnobreakspace \bibemph{Ezeli kuvvet}. Bu kuvvet; mekân gücü içerisinde ilk temel değişikliği temsil etmekte olup, Koşulsuz Mutlaklık’ın alt Cennet faaliyetlerinden bir tanesi olabilir. Alt Cennet’den dışarı çıkan mekân mevcudiyetinin, gelen mekân mevcudiyetine göre bir ölçüde değişikliğe uğradığının bilgisine sahibiz. Fakat bu olası ilişkilerden bağımsız olarak, mekân gücünün ezeli kuvvete olan açık bir biçimde tanınmakta dönüşümü; yaşayan Cennet kuvvet düzenleyicilerine ait gerilim\hyp{}mevcudiyetinin öncül farklılaşan faaliyetidir.
\vs p042 2:8 Etkisiz ve potansiyel kuvvet, Öncül Varedilmiş Üstün Kuvvet Düzenleyicileri’nin mekân mevcudiyeti tarafından sağlanan dirence karşılık olarak etkin ve ezeli bir hale gelir. Kuvvet mevcut an içerisinde; Eylem olarak Tanrı tarafından başlatılan belirli öncül hareketlere verilen karşılıklar ve bunun sonucundaki Kâinatsal Mutlak’dan doğan telafi edici belirli hareketlere verilen tepkiler biçiminde, çoklu tepkinin âlemlerine doğru Koşulsuz Mutlak’ın ayrıcalıklı nüfuz alanından ortaya çıkmaktadır. Ezeli kuvvet, mutlaklığın ölçüsünde aşkın nedenselliğe tepki gösteren bir görünüme sahiptir.
\vs p042 2:9 Ezeli kuvvet zaman zaman \bibemph{saf enerji} olarak adlandırılır; Uversa üzerinde biz onu SEGREGATA biçiminde adlandırmayı tercih etmekteyiz.
\vs p042 2:10 3.\bibemph{ Ortaya çıkış halindeki enerjiler}. Başat kuvvet düzenleyicilerinin etkisiz mevcudiyeti, mekân gücünü ezeli kuvvete dönüştürmek içinde yeterli bir niteliğe sahiptir; buna ek olarak bu mevcudiyet, bahse konu bu aynı kuvvet düzenleyicilerinin ilk ve etkin faaliyetlerine başladıkları bu türden etkinleştirilmiş bir mekân alanı üzerindedir. Ezeli kuvvet, evren gücü olarak ortaya çıkmadan önce enerji dışavurumunun âlemleri içinde başkalaşımının iki farklı fazından geçmenin nihai sonuna sahip kılınmıştır. Ortaya çıkan enerjinin bu iki düzeyi şunlardır:
\vs p042 2:11 a.\bibnobreakspace \bibemph{Kudretli enerji}. Bu enerji; öncül kuvvet düzenleyicilerin etkinlikleri tarafından hareket kazandırılmış devasa enerji sistemleri olarak, güçlü\hyp{}yönelimsel, kütle hareketi kazandırılmış, büyük güç altında gerilim kazandırılmış ve kuvvet uygulanabilen\hyp{}tepki veren enerjidir. Bu başat veya kudretli enerji, her ne kadar Cennet’in alt tarafından idare edilen mutlak etkilerin bütüncül topluluğuna karşı gösterilen bir toplam\hyp{}kütle veya mekân\hyp{}yönelimsel tepkiye muhtemel bir biçimde sebep olsa da, ilk olarak açık bir biçimde Cennet\hyp{}çekim etkisine karşılık göstermemektedir. Enerji, Cennet’in döngüsel ve mutlak\hyp{}çekim etkisine karşı ilk karşılığın düzeyine eriştiğinde; öncül kuvvet düzenleyiciler ikincil birlikteliklerinin faaliyetlerine yol vermektedir.
\vs p042 2:12 b.\bibnobreakspace \bibemph{Çekim enerjisi}. Şu an açığa çıkmakta olan çekim\hyp{}tepki enerjisi; evren gücünün potansiyelini taşımakta olup, evren maddesinin tümünün etkin atası haline gelmektedir. Bu ikincil veya diğer bir değişle çekim enerjisi, Birliktelik Halindeki Aşkın Üstün Kuvvet Düzenleyicileri tarafından oluşturulan basınç\hyp{}mevcudiyeti ve gerilim\hyp{}eğilimlerinden kaynaklanan enerji detaylandırılımının ürünüdür. Bahse konu bu kuvvet düzenleyicilerinin görevlerine karşılık olarak, mekân\hyp{}enerjisi çabuk bir biçimde kudretli enerji düzeyinden çekim düzeyine geçiş yapar; ve böylece mekân enerjisi, enerji ve maddenin elektronsal ve elektron sonrası düzeylerinin yakın bir zamanda ortaya çıkmakta olan maddi kütlesi içinde içkin olan doğrusal\hyp{}çekim etkisi için duyarlılığın belirli bir potansiyelini ortaya çıkararak, Cennet (mutlak) çekimin döngüsel kavrayışına doğrudan bir biçimde karşılık verir hale gelir. Çekim karşılığının ortaya çıkması üzerine Birliktelik Halindeki Üstün Kuvvet Düzenleyicileri, Evren Güç Düzenleyicileri’nin eylemin bu alanı üzerine görevlendirilmeleri koşuluyla mekânın enerji siklonlarından ayrılabilir.
\vs p042 2:13 Biz, kuvvet evrimine ait öncül aşamaların gerçek nedenlerinden hiçbir biçimde emin olamamaktayız; ancak biz Nihayet’in ussal etkisini, ortaya çıkan\hyp{}enerji dışavurumlarının iki seviyesi içinde de tanımlayabilmekteyiz. Kudretli ve çekim enerjileri bütüncül olarak düşünüldüğünde, Uversa üzerinde ULTİMATA olarak adlandırılır.
\vs p042 2:14 4.\bibnobreakspace \bibemph{Evren gücü}. Mekân\hyp{}kuvveti, mekân\hyp{}enerjisine dönüşmüş olup bu düzeyden de çekim düzenlenmesinin enerjisi haline gelmiştir. Böylelikle fiziksel enerji; gücün kanallarına doğru yönlendirilebilir ve evren Yaratanları’nın çok katmanlı amaçlarına hizmet eder bir hale getirildiği yer olan düzeye erişir bir halde olgunlaşmıştır. Bu görev; düzenlenmiş ve yerleşik hale getirilmiş yaratılmışlar olarak asli evren içinde fiziksel enerjinin çok yönlü idarecileri, merkezleri ve düzenleyicileri tarafından yerine getirilir. Bu Evren Güç Yöneticileri, yedi aşkın\hyp{}evrene ait mevcut enerji sistemini oluşturan enerjinin otuz fazının yirmi birinin neredeyse bütüncül denetimini üstlenmektedir. Güç\hyp{}enerji\hyp{}maddenin bu nüfuz alanı, Yüce olan Tanrı’nın zaman\hyp{}mekân yüksek denetimi altında faaliyet gösteren Yedi Katmanlı Tanrı’nın ussal etkinliklerinin âlemidir.
\vs p042 2:15 Uversa üzerinde biz, evren gücünün âlemini GRAVİTA olarak adlandırmaktayız.
\vs p042 2:16 5.\bibemph{ Havona enerjisi}. Kavramsal olarak bu anlatım, dönüşüme uğrayan mekân\hyp{}kuvvetin zaman ve mekânın evrenlerine ait enerji\hyp{}gücün çalışma düzeyine doğru aşama aşama ilerleyişiyle, Cennet yolu huzuruna doğru akmaktadır. Cennet yolunu üzerinde devam ederken, merkezi evrenin niteliği olan enerjinin mevcudiyet\hyp{}öncesi bir fazıyla karşılaşılır. Burada evrimsel çevrim tekrar kendisine doğru dönüyormuş gibi görünmektedir; burada enerji\hyp{}gücün tekrar kuvvete doğru kaymakta olduğu gözlenmektedir, ancak bir doğanın kuvveti, mekân gücü ve ezeli kuvvetten oldukça farklılık taşımaktadır. Havona enerji sistemleri çifte bir doğaya sahip değildir; onlar üçleme bütünlüğü halindedir. Bu enerji, Cennet Kutsal Üçlemesi adına faaliyet gösteren Bütünleştirici Bünye’nin mevcut enerji nüfuz alanıdır.
\vs p042 2:17 Uversa üzerinde Havona’nın bu enerjileri TRİATA olarak bilinmektedir.
\vs p042 2:18 6.\bibnobreakspace \bibemph{Aşkın enerji}. Bu enerji sistemi; Cennet’in üst düzeyi üzerinde ve buradan faaliyet göstermekte olup, yalnızca absonit insanlar ile iletişim halindedir. Uversa üzerinde bu enerji TRANOSTA olarak adlandırılmaktadır.
\vs p042 2:19 7.\bibnobreakspace \bibemph{Monota}. Enerji, Cennet enerjisi olduğu durumda kutsallığa yakın bir niteliğe sahiptir. Biz, Özgün Evlat’ın ruhaniyet enerjisi olarak canlı bir ebedi akranı biçiminde, monotanın Cennet’in ruhani olmayan enerjisi olarak canlı bir nitelikte bulunduğuna ve böylece Kâinatın Yaratıcısı’nın ruhsal olmayan enerji sistemi olduğuna dair inanca eğilim göstermekteyiz.
\vs p042 2:20 Biz, Cennet ruhaniyeti ve Cennet monotasının \bibemph{doğasını} ayırt edemeyiz; onlar açık bir biçimde birbirlerine benzemektedir. Onlar farklı isimlere sahiptir, fakat ruhsal ve ruhsal olmayan dışavurumlarının yalnızca \bibemph{isimleri} tarafından ayırt edilebilmesine dair bir bilgi mevcut bulunmamaktadır.
\vs p042 2:21 Biz, sınırlı yaratılmışların Yedi Katmanlı Tanrı ve Düşünce Düzenleyicileri’nin hizmeti vasıtasıyla Kâinatın Yaratıcısı’nın ibadet deneyimine erişmeye yetkin olduklarının bilgisine sahibiz; ancak biz, güç yöneticilerini bile içine alan bir biçimde herhangi bir alt mutlak kişiliğinin, İlk Muhteşem Kaynak ve Merkez’in enerji sınırsızlığını kavrayabileceğinden kuşku duymaktayız. Bu hususta bir şey kesin bir doğruluğa sahiptir: Eğer güç yöneticileri mekân\hyp{}kuvvet başkalaşımına ait olan işleyiş biçimine aşina ise, onlar bu sırrı hiçbirimize açıklamamaktadırlar. Benim fikrime göre onlar, kuvvet düzenleyicilerin faaliyetini bütüncül bir biçimde kavramamaktadırlar.
\vs p042 2:22 Bahse konu bu güç yöneticileri kendi başlarına enerji katalizörleridir; şöyle ki, onlar mevcudiyetleri tarafından birim oluşumu içinde enerjiyi ayrıştırır, düzenler veya bir araya getirirler. Ve bunların tümü, bu güç birimlerin mevcudiyetinde böylece faaliyet göstermelerine yol açan enerji içinde içkin olan bir unsurun var olması zorunluluğu olduğu anlamına gelmektedir. Nebadon Melçizedekleri uzun bir zamandan bu yana, kâinatsal kuvvetin evren gücüne olan dönüşümüne ait olgular bütününü “kutsallığın yedi sınırsızlığından” biri olarak tanımlamıştır. Ve bu düzey, sizin yerel evren yükselişiniz boyunca erişebileceğiniz en uzak noktadır.
\vs p042 2:23 Her ne kadar kâinatsal kuvvetin kökeni, doğası ve dönüşümlerini bütünüyle kavramak bizim yetkinliğimiz dışında olsa da; biz bütüncül bir şekilde, aşkın\hyp{}evren güç yöneticilerinin faaliyetinin yaklaşık olarak başlangıcı biçiminde, açığa çıkmakta olan enerjinin Cennet çekimi eylemine olan doğrudan ve hatasız tepkisinin zamanlarından başlayarak onun davranışının tüm fazlarına aşina bir konumda bulunmaktayız.
\usection{3.\bibnobreakspace Madde’nin Sınıflandırılışı}
\vs p042 3:1 Merkezi evren dışında evrenlerin tümü içinde madde özdeştir. Fiziksel nitelikleri bakımından madde; bileşen sayılarının döngüsel hızlarına, döngü halindeki üyelerin sayısına ve büyüklüğüne, onların çekirdeğin bedeninden veya maddenin mekân içeriğinden uzaklığına ve Urantia üzerinde henüz keşfedilmemiş belirli kuvvetlerin mevcudiyetine bağımlıdır.
\vs p042 3:2 Çeşitli güneşler, gezegenler ve mekân bünyeleri içinde maddenin on büyük sınıflandırılışı bulunmaktadır:
\vs p042 3:3 1.\bibnobreakspace Ultimatonsal madde --- elektronları oluşturmak için hareket eden enerji parçacıkları olarak, maddi mevcudiyetin başat fiziksel birimleri.
\vs p042 3:4 2.\bibnobreakspace Alt elektronsal madde --- güneş aşkın gazlarının patlayıcı ve püskürtücü aşaması.
\vs p042 3:5 3.\bibnobreakspace Elektronsal madde --- elektronlar, protonlar ve elektronsal toplulukların çeşitli oluşumlarına katılan çeşitli diğer birimler olarak, maddi farklılaşmanın elektronsal düzeyi.
\vs p042 3:6 4.\bibnobreakspace Alt atomsal madde --- sıcak güneşlerin iç bölgesi içinde fazlasıyla mevcut bulunan madde.
\vs p042 3:7 5.\bibnobreakspace Parçalanmış atomlar --- soğuyan güneşler içinde ve uzay boyunca bulunan unsurlar.
\vs p042 3:8 6.\bibnobreakspace İyonlaşmış madde --- elektriksel, termal veya X\hyp{}ışın etkinlikleri ve çözücüler tarafından bireysel atomların dışsal (kimyasal olarak etkin) elektronlarından ayrılması.
\vs p042 3:9 7.\bibnobreakspace Atomsal madde --- moleküler veya görünen maddenin bileşen birimleri olarak, elementsel düzenlenişin kimyasal düzeyi.
\vs p042 3:10 8.\bibnobreakspace Maddenin moleküler aşaması --- olağan koşullar altında göreceli sabit maddeleşmesinin bir düzeyi içinde Urantia üzerinde mevcut bir durumda bulunduğu haliyle madde.
\vs p042 3:11 9.\bibnobreakspace Radyoaktif madde --- orta düzey ısı ve azaltılmış çekim basınç koşulları altında daha ağır elementlerin düzen bozucu eğilimi ve etkinliği.
\vs p042 3:12 10.\bibnobreakspace Çöküntüye Uğramış Madde --- soğuk veya ölü güneşlerin iç bölgelerinde bulunan göreceli sabit bir konumdaki madde. Maddenin bu türü gerçekte durağan değildir; orada hala bir takım ultimatonsal ve hatta elektronsal etkinlik bulunmaktadır, fakat bu birimler çok yakın bir uzaklığa sahiptir ve onların dönüş hızları büyük bir oranda azalmıştır.
\vs p042 3:13 Maddenin bahse konu bu sınıflandırılışı; yaratılmış varlıkların görünüşünün türleri yerine, onun düzenlenişi ile ilgilidir. Bu sınıflandırılış ne enerjinin ortaya çıkış aşamalarından önceki düzeylerini ne de Cennet üzerinde ve merkezi evren içinde ebedi maddileşmeleri hesaba katar.
\usection{4.\bibnobreakspace Enerji ve Madde Dönüşümleri}
\vs p042 4:1 Işık, ısı, elektrik, manyetizma, kimyasal faaliyet, enerji ve madde --- köken, doğa ve nihai son bakımından --- henüz Urantia üzerinde keşfedilmemiş diğer maddi gerçeklikler ile birlikte bir ve özdeştir.
\vs p042 4:2 Biz, fiziksel enerjinin tabi olabileceği neredeyse sınırsız sayıdaki değişikliği bütünüyle kavrayamamaktayız. Bir evren içerisinde fiziksel enerji ışık olarak ortaya çıkmakta, bir diğerinde ışık ve ısı yaymakta, ve bir diğerinde ise Urantia üzerinde olduğu gibi enerjinin bilinmeyen türlerinde gözlenmektedir; ifade edilmemiş milyon yıllar içinde fiziksel enerji, çalkantılı elektriksel enerji veya manyetik güç olarak sürekli değişen bir tür içerisinde tekrar ortaya çıkabilir; ve daha sonra fiziksel enerji bir sonraki evren içinde, âlemlerin birtakım büyük felaketleri içinde dışsal fiziksel kayboluşunu takiben, başkalaşımın bir serisi boyunca çeşitli maddenin bir türü içinde tekrar ortaya çıkabilir. Hesaba gelmez çağlar ve evrenler boyunca neredeyse sonu gelmeyen gezintisi sonrasında bu aynı enerji böylece tekrar ortaya çıkar ve birçok kez kendi türü ve potansiyelini değiştirir; ve benzer olarak bu dönüşümler, takip eden çağlar ve sayısız âlemler boyunca devam eder. Böylece madde, zamanın dönüşümleri sürecinden geçerek fakat aynı zamanda her zaman ebediyetin döngüsüne bağlı bir biçimde salınarak ufalanır; her ne kadar kaynağına geri dönmesi uzun bir süreliğine engellenmiş olursa olsun, o her zaman buraya geri dönecektir, ve onu gönderen Sınırsız Kişilik tarafından emredilen doğrultu boyunca o her zaman ilerleyecektir.
\vs p042 4:3 Güç merkezleri ve onların birliktelikleri, ultimatonun elektronların döngülerine ve dönüşlerine olan dönüşümleriyle oldukça ilgilidir. Bu benzersiz varlıklar, ultimatonlar olarak maddileşmiş enerjinin temel birimlerine ait maharetli dönüşümleri vasıtasıyla gücü denetim altına alır ve onu bir araya getirir. Onlar, bu ilkel düzey içerisinde döngü halinde olduğu gibi enerjinin üstünleridir. Fiziksel düzenleyiciler ile irtibat halinde onlar, elektronsal aşama olarak adlandırıldığı biçimiyle elektriksel seviyeye dönüşmesinden sonra bile etkin bir biçimde denetlemeye ve yönlendirmeye yetkindir. Fakat onların faaliyet kapsamı, elektronsal olarak düzenlenmiş enerji atomsal sistemlerin burgaçlarına doğru salındığı zaman devasa bir biçimde kısıtlanmış olur. Bu tür maddileşme üzerine bu enerjiler, doğrusal çekimin çekim gücünün bütüncül kavrayışı altına girer.
\vs p042 4:4 Çekim; güç hatları, güç merkezlerinin enerji kanalları ve fiziksel düzenleyiciler üzerinde olumlu bir biçimde hareket eder; ancak karşı çekim edinimlerinin uygulanması olarak, bu varlıklar yalnızca çekim karşısında olumsuz bir ilişkiye sahiptir.
\vs p042 4:5 Mekânın tümü boyunca soğukluk ve diğer etkiler, yaratıcı bir biçimde ultimatonların elektronlar halinde düzenlenmesinde görev başındadır. Isı, elektronsal etkinliğin ölçüsü iken; bunun karşısında soğukluk, göreceli enerji hareketsizliği olarak, ne ortaya çıkan enerjinin ne de düzenlenmiş maddenin mevcut bir halde bulunmaması ve çekim kuvvetine karşılık vermemesi şartıyla mekânın evrensel kuvvet\hyp{}etkisinin düzeyi olarak, ısının yokluğunu simgeler.
\vs p042 4:6 Çekim mevcudiyeti ve faaliyeti, kuramsal mutlak sıfırın ortaya çıkmasını engelleyen unsurdur; çünkü yıldızlar arası uzay, mutlak sıfırın sıcaklığına sahip değildir. Düzenlenmiş mekânın tümü boyunca çekime karşılık veren enerji akımları, güç döngüleri, ultimatonsal etkinlikler ve düzenlenmekte olan elektronsal enerjiler mevcut bulunmaktadır. İşlev bakımından bahsedilecek olursa, mekân boş değildir. Urantia’nın atmosferi bile, yaklaşık olarak üç bin milde evrenin bu bölümü içerisinde ortalama uzay maddesine dönüşmeye başlayıncaya kadar artan bir biçimde incelir. Nebadon içinde boş mekâna en yakın olarak bilinen yer bile, her biri bir inç küp hacmindeki --- bir elektrona denk olarak --- yaklaşık yüz ultimatonun varlığını sergileyecektir. Maddenin bu türden bir azlığı işlevsel olarak boş mekân olarak değerlendirilir.
\vs p042 4:7 Sıcaklık ve soğukluk olarak ısı, enerji ve madde evriminin âlemleri içinde sadece çekim karşısında ikincildir. Ultimatonlar alçak gönüllü bir biçimde sıcaklığın aşırı değerlerine itaat etmektedir. Düşük sıcaklıklar, elektronsal inşanın ve atomsal bir araya gelmenin belirli türleri için elverişlidir; bunun karşısında yüksek sıcaklıklar, atomsal parçalanışın ve maddi ayrışımın her türünü yerine getirmektedir.
\vs p042 4:8 Belirli içsel güneş düzeylerinin ısı ve basıncına maruz kaldığında maddenin en ilkel birlikteliklerinin neredeyse tamamı dağılabilir. Isı böylece geniş bir biçimde çekim dengesinin üstesinden gelebilir. Fakat bilinen hiçbir güneş ısısı veya basıncı, ultimatonları kudretli enerji haline geri döndüremez.
\vs p042 4:9 Yakıcı güneşler, maddeyi enerjinin çeşitli türlerine dönüştürebilir; fakat karanlık dünyalar ve tüm dışsal uzay, bu enerjilerin âlemlerin maddesine dönüştüğü noktaya kadar elektronsal ve ultimatonsal etkinliği yavaşlatabilir. Nükleer maddenin temel birlikteliklerinin birçoğuna ek olarak yakın bir doğanın belirli elektronsal birliktelikleri; maddileşen enerjinin daha geniş birikimlerinin birlikteliği vasıtasıyla daha sonra bir araya gelerek, açık mekânın oldukça düşük sıcaklıklarında oluşturulmuştur.
\vs p042 4:10 Enerji ve maddenin bu sonu olmayan başkalaşımının bütünü boyunca biz; çekim basıncının etkisi ile sıcaklık, hız ve dönüşün belirli koşulları altında ultimatonsal enerjilerin karşı çekim davranışını göz önünde bulundurmak zorundayız. Sıcaklık, enerji akımları, uzaklık ve yaşayan kuvvet düzenleyicilerinin mevcudiyetine ek olarak güç yöneticileri aynı zamanda enerji ve maddenin dönüşüm olgularının tümü üzerinde bir etkiye sahiptir.
\vs p042 4:11 Madde içindeki kütlenin artışı, ışığın hızının karesi tarafından bölünen enerjinin artışına eşittir. Dinamiksel bir anlamda, hareketsiz maddenin sergilediği iş; Cennet’ten kendi parçalarının bir araya gelmesiyle genişleyen enerjiden, geçiş halinde üstesinden gelinen kuvvetlerin direnci ve birbirleri üzerinde maddenin parçaları tarafından uygulanan etkisinin çıkarılmasına eşittir.
\vs p042 4:12 Maddenin elektronsal öncesi türlerinin mevcudiyeti, kurşunun iki atomsal ağırlığı tarafından belirtilmiştir. Özgün oluşumun kurşunu, radyum sızıntılarının vasıtasıyla uranyum parçalanması boyunca üretilenden biraz daha fazladır; ve atomsal ağırlık içerisindeki bu farklılık, atomsal parçalanma içinde enerjinin mevcut kaybını temsil etmektedir.
\vs p042 4:13 Maddenin göreceli bütünlüğü; Urantia bilim adamlarının kuantum olarak adlandırdıkları, enerjinin yalnızca bu kesin sayılarda emilebildiği veya salınabildiği gerçeği tarafından tasdik edilmiştir. Maddi âlemler içindeki bu bilge koşul, evrenleri süreklilik içerisinde bir arada tutmaya hizmet etmektedir.
\vs p042 4:14 Enerjinin niceliği, elektronsal veya diğer konumlar yön değiştirdiğinde, her zaman bir “kuantum” veya onun bir takım çoklu halidir; fakat enerjinin bu birimlerinin titreşimsel veya dalgasal davranışı bütünüyle, ilgili maddi yapıların ebatları tarafından belirlenir. Bu türden dalgasal enerji dalgacıkları; bu şekilde ortaya çıkan ultimatonların, elektronların, atomların veya diğer birimlerin çaplarının 860 katıdır. Kuantum davranışına ait dalga mekaniğinin gözlenmesine dair hiçbir zaman sonuçlanmamış olan kafa karışıklığı, enerji dalgalarının birbiri üstüne eklenmesi nedeniyle gerçekleşmektedir: Dalgaların iki tepesi iki kat bir tepeyi bir araya getirebilir, bunun karşısında bir tepe ve bir oluk birleşebilir ve böylece karşılıklı olarak birbirlerini sonlandırabilirler.
\usection{5.\bibnobreakspace Dalga\hyp{}Enerji Dışavurumları}
\vs p042 5:1 Orvonton’un aşkın evreni içerisinde, dalga enerjisinin yüz kadar oktavı bulunmaktadır. Enerji dışavurumlarının bu yüzlük topluluğu arasında atmış dördü bütünüyle veya kısmen Urantia üzerinde tanınmıştır. Güneş’in ışınları, bu diziler içinde kırk altıncısı olarak, gözle görülen ışınların tek bir oktav ile bütünleşmesi biçiminde, aşkın evren ölçeği içerisinde dört oktavı oluşturmaktadır. Ultraviyole topluluğu bu sıradan sonra gelmektedir, bunun karşısında bu dizilerin ten oktav yukarısında, radyumun gama ışınlarını takip eden bir biçimde, X\hyp{}ışınları bulunmaktadır. Güneşin görünen ışığının üstündeki otuz iki oktav, birliktelik halindeki oldukça yüksek bir düzeyde enerji kazandırılmış madenin ufak parçacıkları ile çok sık bir biçimde karışıma uğrayan dış uzay enerji ışınlarıdır. Görünen güneş ışının hemen altında infrared ışınları ortaya çıkmaktadır, ve onların otuz oktav altında ise radyo iletim toplulukları bulunmaktadır.
\vs p042 5:2 Dalgasal enerji dışavurumlar --- Urantia’nın yirminci yüzyıl bilimsel aydınlanmasının bakış açısına göre --- şu on topluluk içerisinde sınıflandırılabilir:
\vs p042 5:3 1.\bibnobreakspace \bibemph{İnfraultimatonsal ışınlar} --- onlar belirli bir şekil kazanırlarken ultimatonların sınır dönüşleridir. Bu ışınlar, dalgasal olgular bütününün içinde tespit edilebileceği ve ölçüleceği ortaya çıkmakta olan enerjinin ilk düzeyidir.
\vs p042 5:4 2.\bibnobreakspace \bibemph{Ultimatonsal ışınlar}. Enerjinin ultimatonların ufak parçacıkları ile bir araya gelmesi, fark edilebilir ve ölçülebilir mekânın içeriğinde titreşimlere neden olmaktadır. Ve fizikçilerin ultimatonu keşfedecekleri zamandan uzun bir süre önce onlar kuşkusuz bir biçimde, bu ışınların olgular bütününü Urantia üzerine yağdıklarında tespit edecekler. Bu kısa ve güçlü ışınlar, ultimatonların maddenin elektronsal düzenlenişini saptıracakları noktaya kadar yavaşlarken onların başlangıç etkinliğini temsil etmektedir. Ultimatonlar elektronlar ile bir araya gelirken, yoğunlaşma enerjinin sonuçsal bir birikimi ile ortaya çıkmaktadır.
\vs p042 5:5 3.\bibnobreakspace \bibemph{Kısa mekân ışınları}. Bu ışınlar bütünüyle saf elektronsal titreşimlerin en kısası olup, maddenin bu türünün atom öncesi düzeyini temsil etmektedir. Bu ışınlar, üretimleri için olağandışı oldukça yüksek veya oldukça düşük sıcaklıklara ihtiyaç duymaktadır. Burada bahse konu mekân ışınlarının iki türü bulunmaktadır: bunlardan bir tanesi atomların doğumuna katılır ve diğeri ise atomsal bozulmanın belirleyicisidir. Onlar, Samanyolu olarak aşkın evrenin en yoğun düzleminden --- aynı zamanda dışsal âlemlerin en yoğunu olarak --- geniş miktarlarda yayılırlar.
\vs p042 5:6 4.\bibnobreakspace \bibemph{Elektronsal düzey}. Enerjinin bu düzeyi, yedi aşkın evren içinde maddileşmenin tümünün temelidir. Elektronlar yörüngesel dönüşlerin yüksek enerji düzeylerinden alçak seviyelerine geçerlerken, kuantum her zaman elden bırakılır. Elektronların yörüngesel değişimleri, ışık\hyp{}enerjisinin oldukça belirli ve tek\hyp{}tip olan ölçülebilir parçacıklarının fırlatımı veya emilimiyle sonuçlanır; bunun karşısında bireysel elektron, çarpışmaya maruz kaldığında ışık\hyp{}enerjisinin bir parçasını her zaman geride bırakır. Dalgasal enerji dışavurumları aynı zamanda, pozitif bedenlerin ve elektronsal düzeyin diğer üyelerinin ortaya çıkmasına katılmaktadır.
\vs p042 5:7 5.\bibnobreakspace \bibemph{Gama ışınları} --- bu ışın yayılımları, atomsal maddenin eş zamanları ayrışmasını temsil etmektedir. Elektronsal etkinliğe ait olan bu türün en iyi temsili, radyum ayrışması ile ilişkilendirilen olgular bütünü içerisindedir.
\vs p042 5:8 6.\bibnobreakspace \bibemph{X\hyp{}ışını topluluğu}. Elektronun yavaşlamasında takip eden aşama, yapay olarak üretilmiş X\hyp{}ışınları ile birlikte güneş X\hyp{}ışınlarının çeşitli türlerini beraberinde getirir. Elektronsal etki, bir elektriksel alan yaratmaktadır; hareket, bir elektriksel akıma sebep olmaktadır; akım bir manyetik alan yaratmaktadır. Bir elektron ani bir biçimde durdurulduğunda, bunun sonrasında açığa çıkan elektromanyetik karışıklık X\hyp{}ışınlarını üretmektedir; X\hyp{}ışınları \bibemph{bu} karışıklıktır. Güneş X\hyp{}ışınları, insan bedenin içini keşfetmek için mekaniksel olarak üretilen ışınlar ile onlardan biraz daha uzun olması dışında özdeştir.
\vs p042 5:9 7.\bibnobreakspace Güneş ışınlarının \bibemph{ultraviyole} veya kimyasal ışınları ve diğer çeşitli mekaniksel üretimler.
\vs p042 5:10 8.\bibnobreakspace \bibemph{Beyaz ışık} --- güneşlerin bütünüyle görülebilir ışığıdır.
\vs p042 5:11 9.\bibemph{ İnfrared ışınları} --- elektronsal etkinliğin dayanılabilir ısının yakın düzeyine olan yavaşlaması.
\vs p042 5:12 10.\bibnobreakspace \bibemph{Hertzsel dalgalar} --- bu enerjiler Urantia üzerinde yayın amaçlı kullanılan dalgalardır.
\vs p042 5:13 Dalgasal enerji etkinliğinin bu on fazı arasında insan gözü sadece; olağan güneş ışığının bütüncül ışığı olarak, tek bir oktava karşılık vermektedir.
\vs p042 5:14 Vakum olarak adlandırılan yapı, mekân içinde ortaya çıkan kuvvet ve enerji etkinliklerinin bir topluluğunu temsil eden yalnızca bütüncül bir isimdir. Enerjinin ultimatonları, elektronları ve diğer kütle birleşimleri, maddenin tek\hyp{}tip parçacıklarıdır; ve mekân boyunca geçişlerinde onlar gerçek anlamda doğrudan hatlar boyunca hareket ederler. Tanınabilen enerji dışavurumlarına ait ışık ve tüm diğer türler; çekim ve diğer katılımcı kuvvetler tarafından değişikliğe uğraması dışında, doğrudan hatlar içinde ilerleyen belirli enerji parçacıklarının bir dizisinden oluşmaktadır. Enerji parçacıklarının bu ilerleyişlerinin belirli gözlemler altında dalgalar olarak ortaya çıkmasının sebebi; vakum olarak adlandırdığınız, mekânın tümünün farklılaşmamış kuvvet örtüsüne ve maddenin ilgili birleşimlerinin içsel çekim gerilimine olan direncinden kaynaklanmaktadır. Enerji huzmelerinin ilk hızları ile birlikte maddenin parçacık\hyp{}aralıklarının konumlanması, enerji\hyp{}maddeye ait birçok türlerin dalgasal görünüşlerini oluşturur.
\vs p042 5:15 Mekânın içeriğine ait hareketlenme, maddenin hızla hareket eden parçacıklarının geçişine olan dalgasal bir tepkiyi üretmektedir. Bu durum tıpkı, bir geminin su boyunca ilerleyişinin değişen şiddette ve aralıktaki dalgaları başlatmasına benzemektedir.
\vs p042 5:16 Ezeli\hyp{}kuvvet davranışı, sizin vakum olarak düşündüğünüz yapıya birçok yönden benzerlik teşkil eden olgular bütününe sebep olmaktadır. Mekân boş değildir; mekânın tümünün âlemleri burgaç gibi dönmekte ve yayılmış kuvvet\hyp{}enerjinin geniş bir okyanusu boyunca dalmaktadır; buna ek olarak ne de bir atomun mekân içeriği boştur. Yine de ne vakum ve bu varsayılan vakumun yokluğu, yerleşik gezegenin güneşe düşmesine ve çevreleyen elektronun çekirdeğe düşmesine karşı olan direncin ortadan kalkmasına sebep olur.
\usection{6.\bibnobreakspace Ultimatonlar, Elektronlar ve Atomlar}
\vs p042 6:1 Evrensel kuvvetin mekân etkisi türdeş ve farklılaşmamış olsa da, enerjinin maddeye olan evriminin düzenlenişi, hassas çekim tepkisi olarak enerjinin belirli ebatlar ve oluşturulmuş ağırlığın ayrı kütlelerine olan yoğunlaşmasına yol açar.
\vs p042 6:2 Yerel veya doğrusal çekim, maddenin atomsal düzenlenişinin ortaya çıkışı ile birlikte bütüncül bir biçimde işlevsel hale gelir. Atom öncesi madde, X\hyp{}ışınları ve benzer diğer enerjiler ile etkin hale getirildiği zaman bir miktar çekime karşılık verir bir hale gelir; fakat ölçülebilen hiçbir doğrusal çekim etkisi özgür, bağımsız ve yüksüz elektronsal\hyp{}enerji parçacıkları veya birliktelik halinde bulunmayan ultimatonlar üzerinde uygulanmamaktadır.
\vs p042 6:3 Ultimatonlar, yalnızca dairesel Cennet\hyp{}çekim etkisine cevap vererek, karşılık çekim vasıtasıyla faaliyet gösterir. Doğrusal\hyp{}çekim karşılığı olmadan onlar böylelikle, evrensel mekân eğilimi içinde tutulurlar. Ultimatonlar, kısmi karşı çekim davranışı noktasına kadar döngüsel hızı artırmaya yetkindir; fakat onlar kuvvet düzenleyicileri veya güç yöneticilerinden bağımsız olarak kudretli\hyp{}enerji düzeyine geri döndüren bir biçimde kimlik dışına çıkmanın sınırsal kaçış hızına erişemezler. İçkin olarak ultimatonlar, yalnızca soğuyan veya ölmekte olan bir güneşin geçici engellenmesine katıldıkları zaman fiziksel mevcudiyetin düzeyinden kaçarlar.
\vs p042 6:4 Urantia üzerinde bilinmeyen ultimatonlar, elektronsal düzenleniş için döngüsel\hyp{}enerji şartlarına erişmelerinden önce fiziksel etkinliğin birçok fazı boyunca yavaşlamaktadır. Ultimatonlar, hareketin üç çeşidine sahiptir: kâinatsal kuvvet karşısında karşılıklı direnç, karşı çekim potansiyelinin bireysel döngüleri, ve karşılıklı birliktelik halindeki yüz ultimatonun iç elektronsal konumları.
\vs p042 6:5 Karşılıklı çekim, elektronun oluşumu içinde yüz ultimatonu bir arada tutmaktadır; ve orada hiçbir zaman, bir tipik elektron içinde yüz ultimatondan daha az veya daha fazla ultimaton hiçbir zaman bulunmamaktadır. Bir veya daha fazla ultimatonun kaybı, tipik elektronsal kimliği yok ederek böylece elektrona ait dönüşüme uğramış on biçiminden bir tanesini meydana getirir.
\vs p042 6:6 Ultimatonlar, yörüngeleri tasvir etmez veya elektronlar içinde döngüler etrafında dönüşlerini gerçekleştirmezler; fakat onlar dönüş eksen hızları uyarınca yayılır veya kümelenir, ve böylece farklı elektronsal ebatları belirlerler. Eksen dönüşünün bu aynı ultimatonsal hızı aynı zamanda, elektronsal birimlerin birkaç türünün negatif veya pozitif tepkilerini belirlemektedir. Elektronsal maddenin ayrışması veya topluluk haline gelmesi, enerji\hyp{}maddesinin negatif veya pozitif bedenlerinin elektriksel farklılaşması ile birlikte, tamamlayıcı ultimatonsal karşılıklı birlikteliğin bu çeşitli faaliyetlerinden kaynaklanmaktadır.
\vs p042 6:7 Her atom çapı bakımından bir inçin 1.100.000.000’undan birazcık daha fazladır; bunun karşısında bir elektron hidrojen olan en küçük atomun 1.2.000’inden biraz daha fazla ağır gelmektedir. Atomsal çekirdeğin belirleyici olan pozitif proton, her ne kadar bir negatif elektrondan hacimsel olarak daha geniş olmasa da, ondan yaklaşık olarak iki yüz bin defa daha ağır gelmektedir.
\vs p042 6:8 Madenin kütlesi; eğer in onsun onda birine eşit olan bir elektrona kadar büyütülürse, ve bunun sonrasında onun hacmi uygun olarak büyütüldüğünde, bu türden bir elektronun hacmi dünya kadar büyük bir hale gelecektir. Bir proton’un hacmi; bir elektronun ağırlığından on sekiz bin kat daha büyük olarak, bir iğnenin başı ebatlarına kadar büyütüldüğünde, ve bunun sonrasında karşılaştırmalı olarak bir iğnenin başı güneş etrafında dünyanın yörüngesine eşit olan bir çapa erişecektir.
\usection{7.\bibnobreakspace Atomsal Madde}
\vs p042 7:1 Maddenin tümünün oluşumu, güneş sisteminin düzeni üzerindedir. Enerji evreninin her dakikasında göreceli olarak sabit, karşılaştırmalı olarak durağan bir biçimde maddi mevcudiyetin nükleer bir parçası bulunmaktadır. Bu merkezi birime, dışavurumun üç katmanlı bir olasılığı kazandırılmıştır. Bu enerji merkezini çevreleyen, dalgalanan döngüler içinde sonsuz bolluk içinde, sizin güneş sisteminize benzeyen birtakım yıldızsal toplulukların güneşini çevreleyen gezegenlerle uzaktan karşılaştırılabilecek, enerji birimlerinin döngüsel burgaçları bulunmaktadır.
\vs p042 7:2 Güneş sisteminin mekânı içinde gezegenlerin güneş etrafında döndükleri düzlem ile karşılaştırılabilecek bir genişlikte, atom içerisinde elektronlar merkezi proton etrafında dönerler. Mevcut ebatlar ile karşılaştırıldığında; iç gezegen Merkür ve sizin güneşiniz arasında mevcut olan uzaklık, atomsal çekirdek ve iç elektronsal döngüler arasında var olan uzaklık ile aynı göreceli uzaklığa sahiptir.
\vs p042 7:3 Elektronsal eksen dönüşlerinin ve onların atomsal çekirdeğin etrafında yörüngesel hızlarının ikisi de, onların tamamlayıcı ultimatonlarının hızlarına bile daha gelmeden, insan hayalinin ötesindedir. Radyumun pozitif parçacıkları, saniyede on bin mil düzeyinde mekâna yayılır; bunun karşısında ise negatif parçacıklar yaklaşık olarak ışık hızına yakın bir hıza erişirler.
\vs p042 7:4 Yerel evrenler, ondalık yaratımların bir parçasıdır. Bir çifte evren içinde mekân\hyp{}enerjinin ayırt edilebilen yalnızca yüz atomsal maddileşmesi bulunmaktadır; bu sayı Nebadon içinde maddenin olası en yüksek düzenlenişidir. Maddenin bu yüz türü, içinde merkezi ve göreceli olarak bir bütün olan çekirdeğin etrafında dönen bir veya yüz elektronun düzenli bir serisinden meydana gelmektedir. Maddeyi meydana getiren nitelik çeşitli enerjilerin bu düzensel ve bağımlı birlikteliğidir.
\vs p042 7:5 Her dünya, yüzeyi üzerinde tanınabilen yüz elementi göstermeyecektir; fakat onlar herhangi bir yer de, hep var olmuş bir biçimde, mevcuttur veya evrim sürecindedir. Bir gezegenin kökeni ve onu takip eden evrimini çevreleyen şartlar, yüz atomsal türün kaçının gözlemlenebilir olduğunu belirlemektedir. Daha ağır olan atomlar, birçok dünyanın yüzeyinde bulunmamaktadır. Urantia üzerinde bile, bilinen daha ağır atomlar; radyum davranışında gösterildiği gibi, parçacıklar halinde havaya uçma eğilimi sergilerler.
\vs p042 7:6 Atomun durağanlığı, merkezi beden içinde elektriksel olarak etkin olmayan nötronların sayısına bağlıdır. Kimyasal davranış bütünüyle, özgürce dönüş yapan elektronların etkinliğine bağlıdır.
\vs p042 7:7 Orvonton içinde, bir atomsal sistem içerisinde yüzden fazla yörüngesel elektronları bir araya getirmek doğal olarak hiçbir zaman mümkün olmamıştır. Yüz bir elektron yapay olarak yörüngesel alana sokulduğunda, sonuç her zaman; diğer özgürleşen enerjiler ve elektronların şiddetli ayrışması ile birlikte merkezi protonun eş zamanlı olarak bozulması olmuştur.
\vs p042 7:8 Atomlar birden yüze kadar yörüngesel elektron taşıyabilirken, daha büyük atomların yalnızca dış on elektronu, ayrıntılı ve belirli yörüngeler üzerinde bir arada ve bütünsel olarak salınarak, farklı ve ayrı bünyeler halinde merkezi çekirdek etrafında dönmektedir. Merkeze en yakın otuz elektron, ayrı ve düzenli bünyeler olarak gözlem veya ayırt etme bakımından zorluk teşkil etmektedir. Nükleer yakınlık ile ilişkili olarak elektronsal davranışın bu karşılaştırmalı aynı oranı, sayılarından bağımsız elektronlar ile bütünleşen atomların tümü içinde mevcut bulunmaktadır. Ne kadar çekirdeğe yakın olunursa, elektronsal bireysellik o derece daha az olacaktır. Bir elektronun dalgasal enerji uzantısı, daha alt atomsal yörüngelerin tümünü kaplayacak kadar dışarı yayılabilir; bu durum özellikle atomsal çekirdeğe en yakın olan elektronlar için doğruluk teşkil eder.
\vs p042 7:9 En içte bulunan otuz yörüngesel elektron, bireysel bir kimliğe sahiptir; ancak onların enerji sistemleri, elektrondan elektrona ve neredeyse yörüngeden yörüngeye doğru genişleyen bir biçimde birbirine karışabilir. Bir sonraki otuz elektron; ikinci aile veya enerji bölgesini oluşturmakta olup, ilgili enerji sistemleri üzerinde daha bütüncül bir denetim uygulayan madde bedenleri olarak bireyselliğin ilerleyişinin bir parçasıdır. Geride kalan on elektron; yalnızca on ağır element içinde mevcut olarak, bağımsızlığın soyluluğuna sahip olup, böylelikle ana çekirdeğin denetiminden neredeyse özgür bir biçimde kaçabilmeye yetkindir. Sıcaklık ve basınçtaki en ufak bir değişiklik ile birlikte, elektronların bu dördüncü ve en dışta bulunan topluluk üyeleri; uranyum ve onun benzer elementlerinin eş zamanlı olarak bozulması ile gösterildiği gibi, merkezi çekirdeğin kavrayışından kaçacaktır.
\vs p042 7:10 Birden başlayarak yirmi yedi yörüngesel elektronu taşıyan ilk yirmi\hyp{}yedi atom, diğerlerine kıyasla daha kolay kavranmaktadır. Yirmi sekizden yukarı doğru biz, Koşulsuz Mutlaklık’ın varsayılan mevcudiyetinin tahmin edilemezliği ile gittikçe daha fazla karşılaşmaktayız. Fakat bu elektronsal tahmin edilemezliğin bazısı, farklılaşan ultimatonsal eksen dönüş hızları ve ultimatonların açıklanmamış “toplanma” eğilimi sebebiyle gerçekleşmektedir. Atomlar böylelikle tahmin edilebilirlik bakımından insanlara benzemektedir. İstatistikçiler geniş sayıdaki ister atomu veya ister insanı idare eden kanunları açıklayabilirler, fakat onlar tek bir bireysel atomu veya insanı idare eden kanunları oluşturamazlar.
\usection{8.\bibnobreakspace Atomsal Birleşim}
\vs p042 8:1 Çekim her ne kadar bir küçük atomsal enerji sistemini bir arada tutma ile ilgili olan birkaç etmenden biri olsa da; Urantia üzerinde keşfedilmeyi bekleyen bir kuvvet olarak, temel fiziksel birimlerin oluşumu ve nihai davranışının sırrı biçiminde, onların arasında mevcut ve bilinmez bir enerji aynı zamanda var olmaktadır. Bu evrensel etki, bu küçük enerji düzenlenişi içinde bütünleşen mekânın tümünü kaplamaktadır.
\vs p042 8:2 Bir atomun karşılıklı elektronsal mekânı boş değildir. Bir atom boyunca bu karşılıklı elektronsal mekân, elektronsal hız ve ultimatonsal dönüşler ile kusursuz bir biçimde eş zamanlı hale getirilmiş dalgasal dışavurumlar tarafından etkinleştirilmektedir. Bu kuvvet, sizin pozitif veya negatif çekim biçimindeki tanıdığınız yasalar tarafından bütünüyle idare edilmemektedir; onun davranışı bu nedenle zaman zaman tahmin edilemez bir niteliğe sahiptir. Bu isimlendirilmemiş etki, Koşulsuz Mutlaklık’ın bir mekân\hyp{}kuvvet tepkisi görünümüne sahiptir.
\vs p042 8:3 Atoma ait çekirdeğin yüklü protonları ve yüksüz nötronları, elektronun ağırlığının 180 katı olan maddenin bir parçacığı olarak, mesotronun karşılık verici faaliyeti tarafından bir arada tutulmaktadır. Bu düzen olmadan proton tarafından taşınan elektrik yükü, atomsal çekirdeğin bozulmasına neden olacaktır.
\vs p042 8:4 Atomlar oluşturulduğunda, ne elektrik ne de çekimsel kuvvetler çekirdeği bir arada tutabilirler. Çekirdeğin bütünlüğü; yüklü ve yüksüz parçacıkları, üstün kuvvet\hyp{}kütle gücü ve protonların ve nötronların sürekli bir biçimde yer değiştirmelerine sebep olan daha ileri faaliyeti vasıtasıyla, bir arada tutmaya yetkin olan mesotronun karşılıklı bütünleştiren etkinliği tarafından sağlanır. Mesotron, nükleer parçacıkların elektriksel çekimlerinin sürekli bir biçimde proton ve nötronlar arasında ileri geri fırlatılmasına neden olmaktadır. Bir saniyenin çok küçük bir kısmında herhangi bir nükleer parçacığı, proton ile yüklenip diğerinde ise yüksüz nötron haline gelir. Ve enerji düzeyinin bu dönüşümleri, o kadar hızlı bir biçimde gerçekleşir ki; elektriksel yük, bozucu bir etki olarak faaliyet göstermenin tüm imkânlarından mahrum kalır. Bunun sonucunda mesotron faaliyeti, atomun nükleer istikrarına çok büyük bir oranda katkıda bulunan bir “enerji\hyp{}taşıyıcı” parçacığı olarak faaliyet gösterir.
\vs p042 8:5 Mesotron’un mevcudiyeti ve faaliyeti aynı zamanda diğer atomsal bilmeceyi açıklamaktadır. Atomlar radyoaktifsel olarak ortaya çıktıklarında, normalden çok daha büyük bir enerji yayarlar. Bu radyasyon fazlalığı, “enerji\hyp{}taşıyıcı” mesotronun, sonunca yalnızca bir elektron haline geldiği, parçalanışından elde edilmektedir. Mesotronik parçalanma aynı zamanda belirli küçük yüksüz parçacıkların emilimi ile beraber gerçekleşmektedir.
\vs p042 8:6 Mesotron, atomsal çekirdeğin belirli bütünleştirici niteliklerini açıklamaktadır; fakat o, ne protonun proton ile birleşmesine ne de nötrondan nötrona olan tutunuma açıklık getirmektedir. Atomsal birleşim bütünlüğüne ait paradoksal ve güçlü kuvvet, Urantia üzerinde henüz keşfedilmemiş bir enerji türüdür.
\vs p042 8:7 Bu mesotronlar, sizin gezegeninize oldukça aralıksız bir şekilde çarpan mekân ışınları içinde bolca bulunur.
\usection{9.\bibnobreakspace Doğal Felsefe}
\vs p042 9:1 Din tek başında dogmatik değildir; doğal felsefe özdeş bir biçimde her şeyi dogmalar haline getirmeye meyillidir. Meşhur bir din öğreticisi, insan kafasında yedi boşluk bulunduğu için yedi sayısının doğanın temeli olduğunu nedensel bir biçimde temellendirdiğinde, eğer kimyayı biraz daha fazla bilseydi bu türden bir inancın fiziksel dünyanın gerçek bir olgular bütünü olduğunu bile öne sürebilirdi. Enerjinin ondalık oluşumuna ait olan evrensel dışavurumdan bağımsız olarak, zaman ve mekânın fiziksel evrenlerinin tümü içerisinde; madde öncesi yedi katmanlı elektronsal düzenlenişin en başından beri mevcut olan gerçekliğinin hatırlatıcısı mevcuttur.
\vs p042 9:2 Yedi sayısı, merkezi evrenin ve karakterin içkin iletimine ait ruhsal sistemin temelidir; fakat ondalık sistem olarak on sayısı enerji, madde ve maddi yaratım içinde içkin bir konumdadır. Yine de atomsal dünya; onun çok uzakta bulunan ruhsal kökeninin simgesi olarak bu maddi dünya tarafından taşınan bir doğum izi biçiminde, yedi topluluk içinde yeniden ortaya çıkan belirli dönüşümsel belirlenmeyi sergilemektedir.
\vs p042 9:3 Yaratıcı oluşumun bu yedi katmanlı sürekliliği; temel elementlerin atomsal ağırlıkları içinde düzenlendiklerinde yedi ayrı periyot içinde benzer fiziksel ve kimyasal niteliklerin yeniden bir ortaya çıkışı olarak, kimyasal nüfuz alanları içinde sergilenir. Urantia kimyasal elementleri böylece bir yatay sıra üzerinde sıralandığında, herhangi bir nitelik veya özellik yedili diziler halinde yeniden ortaya çıkma eğilimi gösterir. Yedili diziler tarafından bu periyotsal değişim; ilk veya daha hafif atomsal topluluklar içinde en bariz biçimde gözlenebildiği haliyle kimyasal tablonun bütünü boyunca azalan bir biçimde ve değişimlere uğrayarak yeniden oraya çıkar. Herhangi bir elementten başlayarak, herhangi bir niteliğin belirlenmesinden sonra, bu türden bir nitelik birbirini takip eden altı element boyunca değişecektir; ancak sekizincisine ulaşıldığında bu nitelik yeniden ortaya çıkma eğilimi gösterecektir; şöyle ki, kimyasal olarak etkin olan sekizinci element ilk elemente, dokuzuncu element ikinci elemente benzemekte ve bu dizi böyle devam etmektedir. Fiziksel dünyanın bu türden bir gerçeği; hataya yer bırakmayan bir biçimde atasal enerjinin yedi katmanlı çeşitliliğini işaret etmekte olup, zaman ve mekânın yaratımlarına ait olan yedi katmanlı çeşitliliğin temel gerçekliğinin göstergesidir. İnsan, doğal ışık tayfında yedi rengin olduğunu aynı zamanda dikkate almaktadır.
\vs p042 9:4 Doğal felsefenin varsayımlarının tümü gerçek değildir; insanın mekân olgular bütünü ile ilgili kendisinin bilgisizliğini örtmek için onun hünerli bir girişimini yansıtan varsayımsal vakum buna örnek olarak gösterilebilir. Evrenin felsefesi, bilim olarak adlandırılan gözlemler üzerine dayandırılamaz. Eğer bu türden bir başkalaşım gözle görünemiyorsa, bir bilim adamı bir tırtıldan bir kelebeğin gelişmesinin olasılığını reddetme eğilimi gösterecektir.
\vs p042 9:5 Biyolojik esneklik ile birliktelik içerisinde bulunan fiziksel istikrar, yaratımın Üstün Mimarları tarafından sahip olunan neredeyse sınırsız bilgeliğin tek sebebiyle doğada mevcuttur. Aşkın bilgelikten daha az olan hiçbir şey, bu derece dengede ve oldukça etkin bir biçimde esnek olan maddenin birimlerini hiçbir biçimde tasarlayamazdı.
\usection{10.\bibnobreakspace Evrensel Ruhsal Olmayan Enerji Sistemleri\\(Maddi Akıl Sistemleri)}
\vs p042 10:1 Cennet montasının mutlaklığından mekân gücünün mutlaklığına kadar göreceli kâinatsal gerçekliğin sonsuz yayılımı; mekân gücü içinde saklanan, monota içinde açığa çıkarılan ve arada bulunan kâinatsal düzeyler üzerinde geçici olarak ifşa edilen gerçeklikler olarak, İlk Kaynak ve Merkez’in ruhsal olmayan gerçeklikleri içinde ilişkinin belirli evrimlerinin göstergesidir. Evrenlerin Yaratıcısı içinde döngüsel hale getirildiği biçimiyle enerjinin ebedi çevrimi mutlaktır, onun mutlaklığı ne gerçek ne de bir değer biçiminde irdelemez; yine de Öncül Yaratıcı şu an bile --- her zaman olduğu gibi --- zaman\hyp{}mekânın sürekli genişleyen alanının ve zaman\hyp{}mekân\hyp{}aşkınlaşmasının kendi kendine gerçekleşmesidir; bu nitelikler, içinde enerji\hyp{}maddesinin ilerleyici bir biçimde yaşam ve insan aklının deneyimsel arzuları boyunca yaşam ve kutsal ruhun yüksek denetimine ilerleyici bir biçimde bağlı olduğu değişen ilişkiler içinde bir alan anlamına gelmektedir.
\vs p042 10:2 Evrensel ruhsal olmayan enerjiler; çeşitli düzeyler üzerinde Yaratan olmayan akılların yaşam sistemleri içinde, bazılarının şu biçimlerde tasvir edileceği şekliyle yeniden birliktelik kazandırılır:
\vs p042 10:3 1.\bibnobreakspace \bibemph{Emir\hyp{}yardımcı\hyp{}ruhaniyet öncesi akıllar}. Aklın bu seviyesi deneyim dışı olup, yerleşik dünyalar üzerinde Üstün Fiziksel Düzenleyiciler tarafından hizmet edilir. Bu akıl, maddi yaşamın en ilkel türlerine ait öğretilemez us biçiminde mekanik akıldır; ancak öğretilemez akıl, ilkel gezegensel yaşam dışında birçok düzey üzerinde faaliyet gösterir.
\vs p042 10:4 2.\bibnobreakspace \bibemph{Emir\hyp{}yardımcı\hyp{}ruhaniyet akılları}. Bu akıllar, maddi aklın (mekanik olmayan) öğretilebilir düzeyi üzerinde onun yedi emir\hyp{}yardımcı\hyp{}ruhaniyeti boyunca faaliyet gösteren bir yerel evren Ana Ruhaniyeti’nin hizmetidir. Bu seviye üzerinde maddi akıl deneyimde bulunmaktadır: ilk beş emir\hyp{}yardımcısında alt insan (hayvan) ussu olarak; yedi emir\hyp{}yardımcısında insan (ahlaki) usu olarak; geride kalan iki emir\hyp{}yardımcısı içinde insan\hyp{}üstü (yarı\hyp{}ölümlü) usu olarak deneyimde bulunur.
\vs p042 10:5 3.\bibnobreakspace \bibemph{Evrimleşen morontia akılları} --- yerel evren yükseliş süreçleri içinde evrimleşen kişiliklerin genişleyen bilinçleri. Bu akıllar, Yaratan Evlat ile birliktelik halindeki yerel evren Ana Ruhaniyeti’nin bahşedilmişliğidir. Bu akıl düzeyi, yerel bir evrenin Morontia Güç Yüksek Denetimcileri tarafından etkili hale getirilen maddiyatın ve ruhsallığın bir sentezi olarak, yaşam vasıtasının morontia türünün düzenlenmesini çağrıştırmaktadır. Morontia aklı, erişimin daha yüksek seviyeleri üzerinde kâinatsal akıl ile birlikte artan birliktelik yetkinliğini ortaya çıkaran bir biçimde, morontia yaşamının 570 seviyesine verilen karşılıkta farklılaşan bir biçimde faaliyet gösterir. Bu durum fani yaratılmışların evrimsel doğrultusudur, ancak morontia olmayan bir düzeyin aklı aynı zamanda bir Evren Evladı ve bir Evren Ruhaniyeti tarafından yerel yaratımların morontia olmayan çocukları üzerine bahşedilmiştir.
\vs p042 10:6 \bibemph{Kâinatsal akıl}. Bu akıl, her bir fazı Yedi Üstün Ruhaniyetler’den bir tanesi tarafından yedi aşkın evrenin bir tanesi için hizmet edilen, zaman ve mekânın yedi katmanlı farklılaşan aklıdır. Kâinatsal akıl; sınırlı\hyp{}akıl düzeylerin tümünü içine alıp, deneyimsel bir şekilde Yüce Akıl’ın evrimsel\hyp{}ilahiyat seviyeleri ile, ve aşkın bir biçimde, Bütünleştirici Bünye’nin doğrudan döngüleri olarak, mutlak aklın varoluşsal seviyeleri ile birlikte eş güdümü yerine getirir.
\vs p042 10:7 Cennet üzerinde akıl mutlaktır; Havona içinde absonittir; Orvonton içinde sınırlıdır. Akıl her zaman, çeşitli enerji sistemlerine ek olarak yaşayan hizmete ait mevcudiyet\hyp{}etkinliği çağrıştırmaktadır; ve bu durum, aklın seviyelerinin tümü ve çeşitlerinin hepsi için doğruluk taşımaktadır. Fakat kâinatsal aklın ötesinde, ruhsal olmayan enerji ile aklın ilişkisini tasvir etmek gittikçe zorlaşmaktadır. Havona aklı alt mutlaktır ancak o aşkın evrimseldir; varoluşsal ve deneyimsel olarak o, sizin için açıklığa çıkarılan herhangi bir diğer kavrama kıyasla absonit düzeye daha yakındır. Cennet aklı insan anlayışının ötesindedir; o varoluşsal, mekân dışı ve zaman dışıdır. Yine de aklın bu seviyelerinin tümü, Cennet üzerinde aklın Tanrı’sının akıl\hyp{}çekim kavrayışı vasıtasıyla, Bütünleştirici Bünye’nin evrensel mevcudiyeti tarafından aşılmaktadır.
\usection{11.\bibnobreakspace Evren İşleyiş Düzenleri}
\vs p042 11:1 Aklın idrakı ve tanıyışı hususunda, evrenin ne mekaniksel ne de sihirsel olmadığı hatırlanmalıdır; evren, aklın bir yaratımı ve yasaların bir işleyiş düzenidir. Fakat, doğanın yasalarının işlevsel uygulanması içinde, fiziksel ve ruhsal bir biçimde çifte âlemler varmış gibi görünse de, gerçekte onlar bir bütündür. İlk Kaynak ve Merkez, maddileşmenin tümünün başat nedeni ve aynı zamanda o ruhaniyetlerin tümünün ilk ve nihai Yaratıcısı’dır. Cennet Yaratıcısı, Düşünce Düzenleyicileri ve diğer benzer nüveler biçiminde, yalnızca saf enerji ve saf ruhaniyet olarak Havona ötesi evrenler içinde kişisel olarak ortaya çıkar.
\vs p042 11:2 İşleyiş düzenleri, mutlak bir biçimde bütünsel yaratım üzerinde baskın değildir; kâinat âlemlerini tümü \bibemph{bütünsel bir biçimde} akıl tarafından tasarlanan, onun vasıtasıyla yaratılan ve yine onun mevcudiyeti aracılığıyla idare edilen bir niteliktedir. Fakat kâinat âlemlerinin tümünün kutsal işleyiş düzeni bütünüyle, insanın sınırlı aklına ait bilimsel yöntemlerin sınırsız aklın baskınlığına dair bir izi bile kavrayabilmesi bakımından haddinden fazla kusursuzdur. Bu yaratım, düzenleme ve devamlılığı sağlama aklı, ne maddi bir akıl ne de yaratılmış aklıdır; bu akıl, kutsal gerçekliğin yaratan düzeyleri üzerinde ve buradan faaliyet gösteren ruhaniyet\hyp{}akıl işlevidir.
\vs p042 11:3 Evren işleyiş biçimleri içinde bu aklı kavrayabilme ve onu keşfetme bütünüyle, gözlemin bu türden bir amacı içine katılan sorgulayan aklın yetkinliği, kapsamı ve yetisine bağlıdır. Zaman ve mekânın enerjilerinden düzenlenmiş olan zaman\hyp{}mekân akılları, zaman ve mekânın işleyiş biçimlerine tabidir.
\vs p042 11:4 Hareket ve evren çekimi, kâinat âlemlerin tümünün kişisel olmayan zaman\hyp{}mekân işleyiş biçimlerinin ikiz görünüşüdür. Ruhaniyet, akıl ve madde için çekim karşılığının düzeyleri zamandan oldukça bağımsızdır; ancak gerçekliğin yalnızca gerçek ruhaniyet düzeyleri (mekânsal olmayan bir biçimde) mekândan bağımsızdır. Ruhaniyet\hyp{}akıl düzeyleri olarak evrenin daha yüksek akıl seviyeleri aynı zamanda mekân dışı olabilir; fakat insan aklı gibi maddi aklın seviyeleri, yalnızca ruhaniyet kimlikleşmesi ölçüsünde bu karşılığı yitirerek evren çekiminin etkileşimlerine karşılık veren bir niteliğe sahiptir. Ruhaniyet\hyp{}gerçeklik seviyeleri; onların ruhaniyet içeriği tarafından tanımakta olup, zaman ve mekân içindeki ruhsallık doğrusal\hyp{}çekim karşılığının tersi oranında ölçülmektedir.
\vs p042 11:5 Doğrusal\hyp{}çekim karşılığı, ruhaniyet olmayan enerjinin niceliksel bir ölçüsüdür. Düzenlenmiş enerji olarak kütlenin tümü, hareket ve akıl onun üzerinde eylemde bulunmasının dışında, bu çekim kavrayışına tabidir. İç atomsal bütünlüğün kuvvetleri, mikro kâinatın dar kapsamı kuvvetleri iken; bir biçimde doğrusal çekim, makro kâinatın dar kapsamlı bütünleştirici kuvvetidir. Madde şeklinde adlandırılan fiziksel olarak maddileşmiş enerji, doğrusal\hyp{}çekim kuvvetini etkilemeden mekân üzerinde yol katedemez. Her ne kadar bu türden çekim karşılığı doğrusal olarak kütle ile orantılı olsa da; çekim arada kalan mekân tarafından oldukça değişikliğe uğramaktadır ki, çekimin sonucu uzaklığın karesinin tersi biçiminde kabataslak tahmin edilenden daha fazla olmamaktadır. Mekân, çekim eylemini ve ona karşı gösterilen tüm tepkileri sıfırlamak için faaliyet gösteren sayısız aşkın maddi kuvvetin karşı çekim etkilerinin onun içindeki mevcudiyeti sebebiyle nihai olarak doğrusal çekimin üstesinden gelmektedir.
\vs p042 11:6 Oldukça karmaşık ve yüksek bir biçimde kendiliğinden ortaya çıkan kâinatsal işleyiş biçimleri her zaman; doğanın evren seviyeleri ve işleyiş biçiminin yetisinin çok altında olan usların herhangi bir veya tümünden, özgün veya yaratıcı ikamet eden aklın mevcudiyetini saklama eğilimine sahiptir. Bu nedenle daha yüksek evren işleyiş biçimlerinin yaratılmışların daha alt düzeyleri için akıl dışı olarak görünme zorunluluğu kaçınılmazdır. Bu türden bir yargılamaya dair olası tek istisna, ancak mevcut deneyimin bir parçasından daha çok felsefenin bir konusu olarak, \bibemph{açık bir biçimde kendi kendisini idare eden bir evrenin} olağanüstü olgular bütünü içinde akıl dışılığın ima edilmesi olabilir.
\vs p042 11:7 Akıl evreni eş güdümsel hale getirdiği için, işleyiş biçimlerinin sabitliği söz konusu değildir. Kâinatsal bireysel\hyp{}idare ile birliktelik içerisinde bulunan ilerleyici evrimin olgular bütünü evrenseldir. Evrenin evrimsel yetkinliği, eş zamanlılığın sınırsızlığı içerisinde bitmez tükenmez bir yetkinliğe sahiptir. İlişkilerin gittikçe artan bir karmaşıklığı üzerine aşkın bir biçimde uygulanan büyüyen deneyimsel bir sentez olarak, uyumlu birliğe doğru olan ilerleme yalnızca amaç sahibi ve baskın bir akıl tarafından yerine getirilebilir.
\vs p042 11:8 Evren aklı herhangi bir evren olgular bütünü ile daha yüksek bir biçimde birliktelik içerisinde olursa, aklın alçak düzeyleri için onun keşfi gittikçe zorlaşmaktadır. Ve evren işleyiş biçiminin aklı (Sınırsız’ın aklı içinde) yaratıcı ruhaniyet\hyp{}aklı olduğu için, bu akıl evrenin daha alt düzey akılları tarafından hiçbir zaman keşfedilmez ve onun tarafından kavranılamaz, bu durum insan olarak her şeyin \bibemph{en alt düzey} aklına sahip varlıklar için daha çok geçerlidir. Evrimleşen insan aklı; doğal olarak Tanrı’yı ararken, yalnız değildir ve insan aklı içkin bir biçimde Tanrı’nın bilgisine sahiptir.
\usection{12.\bibnobreakspace Yöntem ve Biçim --- Akıl Baskınlığı}
\vs p042 12:1 İşleyiş biçimlerinin evrimi, yaratıcı aklın saklı mevcudiyeti ve baskınlığı ima eder ve onu işaret eder. Kendiliğinden hareket eden işleyiş biçimlerini kavramak, tasarlamak ve yaratmak için fani usun yetkinliği; gezegen üzerinde baskın etki olarak insan aklının üstün, yaratıcı ve amaç dolu niteliklerini göstermektedir. Akıl her zaman şu niteliklere erişmeyi arzular:
\vs p042 12:2 1.\bibnobreakspace Maddi işleyiş biçimlerinin yaratımı.
\vs p042 12:3 2.\bibnobreakspace Saklı gizemlerin keşfi.
\vs p042 12:4 3.\bibnobreakspace Uzakta bulunan durumların aranması.
\vs p042 12:5 4.\bibnobreakspace Akli sistemlerin oluşturulması.
\vs p042 12:6 5.\bibnobreakspace Bilge hedeflerin erişçi.
\vs p042 12:7 6.\bibnobreakspace Ruhaniyet düzeylerin kazanımı.
\vs p042 12:8 7.\bibnobreakspace Yücelik, nihayet ve mutlaklık biçimindeki kutsal nihai sonların kazanımı.
\vs p042 12:9 Akıl her zaman yaratıcıdır. Hayvansal, fani, morontiyal, ruhani yükseliş varlığı veya kesinliğe erişecek olan unsur biçimindeki bir bireyin akıl kazanımı her zaman, yaşayan yaratılmış kimliği için uygun ve hizmet verebilecek bir beden yaratmaya her zaman yetkindir. Fakat bir kişiliğin mevcudiyet olgular bütünü veya bir kimliğin işleyiş biçimi gibi şeyler, ne fiziksel, ne akılsal veya ruhsal biçimde enerjinin bir dışavurumu değildir. Kişilik biçimi, bir yaşayan varlığın \bibemph{yöntemsel} özelliğidir; bu kavram, enerjilerin \bibemph{düzenlenmesi} ve böylece yaşam ve hareket ek olarak yaratılmış mevcudiyetinin \bibemph{işleyiş biçimi} anlamına gelmektedir.
\vs p042 12:10 Ruhaniyet varlıkları bile biçimlere sahiptir, ve bu ruhaniyet biçimleri (yöntemleri) gerçektir. Ruhaniyet kişiliklerinin en yüksek türü bile, Urantia bedenleri için her bakımından karşılaştırılabilir olan kişilik mevcudiyetleri olarak, biçimlere sahiptir. Yedi aşkın evren içinde karşılaşılan neredeyse her varlık, biçimleri ellerinde bulundurmaktadır. Fakat bu genel kural bakımından bir kaç istisna bulunmaktadır: Düşünce Düzenleyicileri, kendilerinin fani birlikteliklerinin kurtuluş halindeki ruhları ile olan bütünleşmelerine kadar bahse konu bu biçimden yoksun bir görünüme sahiptir. Yalnız İleticiler, Muazzam Kutsal Üçleme Ruhaniyetleri, Sınırsız Ruhaniyet’in Kişisel Yardımcıları, Çekim İleticileri, Aşkın Kaydediciler ve belirli diğer unsurlar aynı zamanda gözlenebilen herhangi bir biçimden yoksun bulunmaktadırlar. Fakat bu durum, birkaç istisna unsurun tipik özelliğidir; unsurların büyük bir çoğunluğu, kişisel karakterlere sahip gerçek biçimlere sahip olup bu türler tanınabilir ve kişisel olarak ayırt edilebilir niteliktedir.
\vs p042 12:11 Kâinatsal akıl ve emir\hyp{}yardımcı\hyp{}ruhaniyetlerin hizmeti ile olan birliktelik, evrimleşen insan varlığı için uygun bir fiziksel bedene evirilmektedir. Buna benzer olarak morontia aklı, tüm fani kurtuluş unsurları için morontia bünyesini bireyselleştirmektedir. Fani benden kişisel ve her insan varlığı için belirleyici olurken, buna benzer bir şekilde morontia bünyesi oldukça bireysel ve onu baskın hale getiren yaratıcı aklın yeterli bir belirleyiciliğinde olacaktır. Hiçbir iki morontia bünyesi, iki insan bedeninden daha çok birbirine benzememektedir. Morontia Güç Yüksek Denetimcileri, ve katılan yüksek meleklerin sağladıkları biçimde, morontia yaşamının onunla birlikte hizmet vermeye başlayabildiği farklılaşmayan morontia maddesini tedarik etmektedir. Ve morontia yaşamının sonrasında ruhaniyet biçimlerinin, ilgili ruhaniyet\hyp{}akıl ikametlerinin eşit bir derecede farklı, kişisel ve belirleyici nitelikte olduğu görülecektir.
\vs p042 12:12 Maddi bir dünya üzerinde siz bir bedeni bir ruhaniyete sahip olarak düşünmektesiniz, fakat biz ruhaniyetin bir bedene sahip olduğunu tahayyül etmekteyiz. Maddi gözler gerçek anlamıyla, ruhaniyet\hyp{}doğumlu ruhun pencereleridir. Ruhaniyet mimar, akıl inşa edendir ve beden ise maddi inşadır.
\vs p042 12:13 Fiziksel, ruhsal ve akılsal enerjiler, saf düzeylerinde olduğu ve onların içinde bulunduğu bir biçimde, olgular evrenlerinin mevcutları olarak bütünüyle etkileşime girmemektedir. Cennet üzerinde üç enerji eş güdümsel, Havona içinde ise eş güdüm halindedir; bunun karşısında sınırlı etkinliklerin evren düzeyleri içinde maddi akılsal ve ruhsal baskınlığın tüm kapsamlarıyla karşılaşılma zorunluluğu bulunmaktadır. Zaman ve mekânın birey dışı durumlarında fiziksel enerji üstün bir durumda bulunuyormuş gibi görünmektedir; fakat fiziksel enerji aynı zamanda gözlenmektedir ki, ruhaniyet\hyp{}akıl faaliyeti amacın kutsallığına ve eylemin yüceliğine daha çok yaklaştığında, ruhaniyet daha fazla olarak baskın bir hale gelmektedir; nihai seviye üzerinde ruhaniyet\hyp{}akıl neredeyse tamamen baskın hale gelebilir. Mutlak seviye üzerinde ruhaniyet kesin bir biçimde baskındır. Ve buradan zaman ve mekânın âlemlerinin dışına doğru her ne zaman bir gerçek ruhaniyet\hyp{}aklı faaliyet gösterir biçimde, her ne zaman bir kutsal ruhaniyet gerçekliği mevcut olursa, orada her zaman bu ruhaniyet gerçekliğinin maddi veya fiziksel bir eşinin üretilmesi eğilimi bulunmaktadır.
\vs p042 12:14 Ruhaniyet, yaratıcı gerçekliktir; fiziksel eş, ruhaniyet\hyp{}aklın yaratıcı eylemine ait fiziksel sonuç biçiminde, ruhaniyet gerçekliğinin zaman\hyp{}mekân yansımasıdır.
\vs p042 12:15 Akıl evrensel bir biçimde maddeye baskın gelmektedir; bunun sonucunda akıl, ruhaniyetin nihai yüksek denetimine karşılık verse bile bu durum geçerliliğini korumaktadır. Ve fani insan ile birlikte, ruhani doğrultuya özgür bir biçimde kendisini teslim eden yalnızca bu akıl; Yücelik, Nihayet ve Mutlaklık olarak Sınırsız’ın ebedi ruhani dünyasına ait ölümsüz bir evlat olarak fani zaman\hyp{}mekân mevcudiyetinden kurtuluşa ermeyi ümit edebilir.
\vs p042 12:16 [Nebadon içinde görevli olan ve Cebrail’in talebini karşılayan bir Kudretli İletici tarafından sunulmuştur.]
