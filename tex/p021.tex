\upaper{21}{Cennet Yaratan Evlatları}
\vs p021 0:1 Yaratan Evlatlar, zaman ve mekânın yerel evrenlerinin yaratıcıları ve yöneticileridir. Bu evren yaratanları ve egemenleri, Yaratıcı olan Tanrı’nın ve Evlat olan Tanrı’nın somutlaşmış karakterleri biçimindeki çifte kökene aittirler. Fakat her Yaratan Evlat bir diğerinden farklıdır; onların her biri, doğalarına ek olarak kişilik bakımından benzersizdir; onların her biri, kendisine ait olan kökenin kusursuz ilahiyat nihai amacının “kendisinden türeyen tek Evladı”dır.
\vs p021 0:2 Düzenlenen, evirilen ve kusursuzlaşan bir yerel evrenin geniş olan görevi içinde bu yüksek Evlatlar her zaman, Kâinatın Yaratıcısı’nın sürekli bir biçimde sağlanan onaylayışından memnuniyet duyarlar. Yaratan Evlatlar’ın Cennet Yaratıcısı ile olan ilişkisi içten ve en üst düzeydedir. Şüphesiz olarak İlahiyat’ın kendilerine ait olan kutsal soyları için gösterdiği ebeveynsel derin sevgisi, fani ebeveynlerin kendi çocuklarını yetiştirirken dahi gösterdikleri güzel ve neredeyse kutsal olan sevginin bir kökenidir.
\vs p021 0:3 Bu temel Cennet Evlatları, Mikâiller olarak kişileştirilmişlerdir. Onlar Cennet’den kendi evrenlerini bulmak için ayrıldıklarında, Yaratan Mikâiller olarak tanınırlar. Yüce idare sağlandığında ise onlar, Üstün Mikâiller olarak adlandırılırlar. Nebadon içindeki evreninizin egemenini zaman zaman Hazreti Mikâil olarak ifade ederiz. Onlar her zaman ve sonsuza kadar, onların düzeyinin ve doğasının ilk Evladı’nın “Mikâil’in düzeyi” biçiminde tanımlanması ardından egemenliklerini sağlarlar.
\vs p021 0:4 Asli veya ilk\hyp{}doğan Mikâil, bir maddi varlık olarak ete kemiğe bürünmeyi hiçbir zaman deneyimlemememiştir; fakat dışta bulunan âlemlerden en içte bulunan merkezi yaratımın döngülerine ilerlerken, Havona’nın yedi döngüsü üzerinde yükseliş halinde olan ruhsal yaratımın deneyimi boyunca yedi kez geçiş halinde bulunmuştur. Mikâil’in düzeyi, muhteşem kâinatı bir sonundan diğerine kadar bilmektedir; Mikâiller’in bireysel olarak katılmadıkları, zaman ve mekânın çocuklarından herhangi birinin asli deneyimi bulunmamaktadır; gerçekte onlar, sadece kutsal doğanın bir parçasına sahip olmakla kalmaz, aynı zamanda en üst düzeyden en alt seviyede bulunan düzeye kadar tüm doğaları içine alan bir biçimde, sizin doğanızın da bir parçasıdır.
\vs p021 0:5 Özgün Mikâil, her şeyin merkezinde Cennet Evlatları büyük bir biçimde katılımsal olan toplantı için bir araya geldiklerinde onların idareci başkanlığını yapmaktadır. Kâinat âlemlerinin tümünün bütünleşmesinin ve düzeninin sağlanmasının ilerleyişi ile ilgili görüşmelere katılan ve ebeveynsel mevcudiyet içerisinde bir araya gelen yüz elli bin Yaratan Evlat’ının ebedi Adası üzerinde muhteşem bir biçimde olan bir özel bir kurulun kâinatsal bir yayınını Uversa üzerinde kaydetmemizden uzun bir zaman geçmemiştir. Geçmiş zamanda olan bahse konu bu unsur, yedi katmanlı bahşedilmiş Evlatlar biçimindeki Egemen Mikâiller’in seçilmiş bir topluluğudur.
\usection{1.\bibnobreakspace Yaratan Evlatlar’ın Kökeni ve Doğası}
\vs p021 1:1 Ebedi Evlat içerisinde mutlak ruhsallığın düşünce olarak oluşturulmasının tamamlanması, Kâinatın Yaratıcısı içinde mutlak kişilik kavramsallaşmasının bütünlüğüyle birleştiğinde; böyle bir yaratıcı birliktelik kesin ve bütüncül olarak erişildiğinde; ruhaniyetin böyle bir mutlak kimliği ve kişilik kavramsallaşmasının bu tür sınırsız tekliği ortaya çıktığında; bunun sonucunda, sınırsız İlahiyatlar’ın herhangi biri tarafından kişiliğin veya ayrıcalığın herhangi bir şeyi kaybolmadan, kusursuzluğun ve gücün bu yeni yaratan kişiliğini üreten kusursuz nihai amacın ve güçlü düşüncenin kendisinden türeyen tek Evlat’ı biçimindeki özgün ve yeni bir Yaratan Evlat’ın uçsuz bucaksız varlığı ortaya çıkar.
\vs p021 1:2 Her Yaratan Evlat, kâinat âlemlerinin tümünün ezelden beri mevcut olan Yaratanlar’ının kusursuz ve ebedi biçimindeki sayıca iki sınırsız akıllarının özgün kavramsallaşmalarının kusursuz birlikteliklerinin doğmuş olan ve doğrulabilen tek çocuğudur. Orada buna benzer başka herhangi bir Evlat bulunmamaktadır; çünkü her yaratan Evlat, bahse konu Mikâil Evladı’nı mevcut hale getirmek için bütünleşen bahse konu kutsal yaratıcı potansiyeller içinde tüm ebediyet boyunca en başından beri bulunan, onun tarafından ifade edilen veya ondan evirilen, her kutsal gerçekliğin her olanağının içinde her özelliğin her fazına ait bütünlüğün nihai bir temsili ve somutlaşmasına ek olarak tamamlanmışlığı ve koşulsuzluğudur. Her Yaratan Evlat, onun kutsal kökenini oluşturan bütünleşmiş ilahiyat kavramlaşmalarının mutlaklığıdır.
\vs p021 1:3 İlke olarak Yaratan Evlatlar’ın kutsal doğaları, Cennet ebeveynlerinin ikisinin niteliklerinden eşit bir biçimde elde edilir. Kâinatın Yaratıcısı’nın kutsal doğasının tamamlanmışlığı ve Ebedi Evlat’ın yaratıcı ayrıcalıklarının benzer olan unsurlarının tümü içinde, evrenler içinde Mikâil’in işlevlerinin uygulanan işleyişlerini gözlemlediğimiz zaman gözle görülür bir biçimde olan farklılıkları algılarız. Bazı Yaratan Evlatlar daha çok Yaratıcı olan Tanrı’ya benzer bir biçimde görünmekte olup; diğerleri ise Yaratıcı olan Tanrı’ya benzemektedir. Nebadon evreni içindeki idarenin eğiliminin, onun Yaratan’ı ve idare eden Evlat’ının Ebedi Ana Evlat’a sahip oldukları doğa ve karakter bakımından daha çok benzediğini öne sürmesi örnek olarak gösterilebilir. Buna ilaveten, Cennet Mikâilleri tarafından idare edilen evrenlerin Yaratıcı olan Tanrı ve Evlat olan Tanrı’ya eşit olarak benzeyen bir görünüme sahip oldukları ifade edilmelidir. Bu bakımdan bahse konu gözlemler hiçbir biçimde ima edilen eleştiriler olmayıp, basit bir biçimde gerçeğin kayıt altına alınmış bir ifadesidir.
\vs p021 1:4 Mevcut halde bulunan Yaratan Evlatlar’ın kesin sayısının bilgisine sahip değilim, fakat onların sayıca yedi yüz binden daha fazla olduğuna inanmak için yeterli nedenlere sahip bulunmaktayım. Şu an içerisinde, sayıca yedi yüz bin Zamanın Birliktelikleri olduğunu ve artık yeni hiçbir varlığın yaratılmadığını bilmekteyim. Bunun yanı sıra biz; mevcut evren çağının emredilen tasarılarının, Zamanın Birliktelikleri’nden birinin her yerel evren içinde Kutsal Üçleme’nin danışma halinde bulunan elçisi olarak konumlandığını belirten görünümü gözlemlemekteyiz. Buna ilaveten biz; Yaratan Evlatlar’ın sürekli bir biçimde artan sayısının, Zamanın birlikteliklerinin sayıca değişmeyen nüfusunu geçtiğini kaydetmekteyiz. Fakat Mikâiller’in sayıca yedi yüz bini geçen nüfusunun nihai sonlarıyla alakalı olarak biz hiçbir zaman bilgilendirilmedik.
\usection{2.\bibnobreakspace Yerel Evrenler’in Yaratanları}
\vs p021 2:1 Öncül düzeyin Cennet Evlatları; yedi evrimsel aşkın evrenin temel yaratıcı birimleri biçimindeki zaman ve mekânın yerel evrenleri olarak onların ilgili nüfuz alanlarının idarecileri, kurucuları, yaratanları ve tasarımcılarıdır. Bir Yaratan Evlat’ın kendisine ait olan gelecek kâinat etkinliğinin mekânsal alanını seçmesine izin verilmiştir; fakat kendisine ait âleminin fiziksel olan düzenlenmesine başlayabilmesinden önce bile o, onun hedeflenen eyleminin aşkın evreni içinde konumlanmış çeşitli yaratımlar arasındaki kendi büyük kardeşlerinin çabalarının çalışmalarına adanan gözlemin uzun bir süreci boyunca zaman harcaması gerekmektedir. Buna ek olarak tüm bahse konu bu olayların öncesinde Mikâil Evladı, Havona eğitiminin ve Cennet gözleminin kendisine ait olan uzun ve benzersiz deneyimini tamamlamış bir düzeyde olacaktır.
\vs p021 2:2 Bir Yaratan Evlat Cennet’den ayrılıp, neredeyse Tanrı biçiminde kendisine ait olan düzenlenmenin yerel evreninin başı haline gelmek için evren yapımının serüvenine giriştiğinde; ilk olarak kendisini, bazı açılardan bağımlı olan bir biçimde Üçüncül Kaynak ve Merkez ile içten bir ilişki içerisinde bulur. Her şeyin merkezinde Yaratıcı ve Evlat ile birlikte ikame etmesine rağmen Sınırsız Ruhaniyet, her Yaratan Evlat’ın etkin ve mevcut yardımcısı olarak faaliyet gösterme biçimindeki nihai son ile sonlandırılmıştır. Bu nedenden dolayı her Yaratan Evlat, yeni yerel evrenin Ana Ruhaniyet’i biçiminde Kutsal Hizmetkâr haline gelmekle nihai olarak sonlandırılmış varlık olarak Sınırsız Ruhaniyet’in bir Yaratıcı Kız Evladı tarafından eşlik edilmektedir.
\vs p021 2:3 Bir Mikâil Evladı’nın bu durumdaki ayrılışı; belirli diğer yükseliş güçleri ve mevcudiyetlerine ek olarak bahse konu Kaynaklar’ın ve Merkezler’in mevcudiyet öncesi konumlarının doğasında bulunan belirli kısıtlamalara yalnızca bağlı olan Cennet Kaynakları ve Merkezleri’nden onun yaratan ayrıcalıklarını sonsuza kadar özgürleştirir. Bu kısıtlamalar arasında bir yerel evren Yaratıcısı’nın her şeye gücü yeten diğer yaratan ayrıcalıkları için olan kısıtlamalar şunlardır:
\vs p021 2:4 1.\bibemph{ Enerji\hyp{}maddesi} Sınırsız Ruhaniyet’in idaresi altındadır. Büyük veya küçük unsurların herhangi yeni bir biçiminin yaratılmasından, enerji\hyp{}maddesinin herhangi yeni bir dönüşümü denenmesinden önce; bir Yaratan Evlat, Sınırsız Ruhaniyet’in rızasını ve işlevsel eş güdümünü sağlamak zorundadır.
\vs p021 2:5 2.\bibnobreakspace \bibemph{Yaratılmış tasarımları ve biçimleri} Ebedi Evlat tarafından denetlenir. Bir Yaratan Evlat’ın, yaratılmışın herhangi yeni bir tasarımı biçimindeki varlığın yeni bir türünün yaratılmasına katılmasından önce, o Ebedi ve Özgün Ana Evlat’ın rızasını sağlamak zorundadır.
\vs p021 2:6 3.\bibnobreakspace \bibemph{Kişilik}, Kâinatın Yaratıcısı tarafından tasarlanır ve bahşedilir.
\vs p021 2:7 \bibemph{Aklın} türleri ve işleyiş biçimleri, varlığın yaratılma öncesi etkenleri tarafından belirlenir. Bu etkenlerin, kişisel veya diğer türde olan bir yaratılmışı oluşturmak için bir araya gelmesinden sonra; akıl, Cennet Yaratanları’nın düzeyinin altındaki tüm varlıklar için ussal hizmetin kâinatsal kökeni biçimindeki Üçüncül Kaynak ve Merkez’in edinimidir.
\vs p021 2:8 \bibemph{Ruhaniyet} tasarımları ve türlerinin denetimi, onların dışavurumlarının düzeyine bağlıdır. Son kertede ruhsal tasarı; Kutsal Üçleme veya Yaratıcı, Evlat ve Ruhaniyet biçimindeki Kutsal Üçleme kişiliklerinin Kutsal Üçleme öncesi ruhaniyet edinimleri tarafından denetlenir.
\vs p021 2:9 Böyle bir kusursuz ve kutsal olan Evlat, kendi seçtiği âlemin mekân bütünlüğünü eline aldığı zaman; âlemin maddileşmesinin ve bütüncül dengenin öncül sorunları çözüldüğü zaman; o, Sınırsız Ruhaniyet’in tamamlayıcı Kız Evlat’ı ile birlikte eş güdümsel ve etkin olan bir işleyiş birlikteliğini oluşturduğu zaman; bunun sonucunda bahse konu Evren Evladı ve Evren Ruhaniyeti, onların yerel evren çocuklarının sayısız olan ev sahiplerine kökeni sağlamak için tasarlanan birleşimi başlatırlar. Bu durumla ilişki içerisinde Cennet Sınırsız Ruhaniyeti’nin Yaratıcı Ruhaniyet odaklanması, bir yerel evrenin Ana Ruhaniyeti’nin kişisel nitelikleri açısından doğası bakımından değişmeye uğrar.
\vs p021 2:10 Yaratan Evlatlar’ın tümü kutsal bir biçimde kendilerinin Cennet ebeveynlerine benzemesine rağmen, hiçbiri kesin bir biçimde diğerine benzememektedir; her biri \bibemph{doğasına} ek olarak kişilikleri bakımından benzersiz, çeşitlilik gösteren, ayrıcalıklı ve özgündür. Buna ek olarak onlar, kendileriyle ilgili olan âlemlerin yaşam tasarılarının yapıcıları ve mimarları oldukları için; bahse konu bu çeşitlilik, onların nüfuz alanlarının aynı zamanda, orada muhtemel bir biçimde yaratılacak veya takip eden bir biçimde evirilecek biçimdeki Mikâil’den türeyen yaşayan mevcudiyetinin her türü ve fazı içinde çeşitlilik gösterecektir. Bu nedenden dolayı, yerel evrenler için özgün olan yaratılmışların düzeyleri oldukça değişiklik gösterir. Her bakımdan özdeş olan çifte kökenli yerel varlıklar tarafından onlardan herhangi bir ikisi ne idare edilebilir ne de yerleşik hale getirilebilir. Herhangi bir aşkın evren içinde, Yaratıcı Ruhaniyetler’in bütünlenmiş mevcudiyetinden türeyen bir biçimde onların doğalarından gelen niteliklerin bir yarısı oldukça benzerlik gösterir; geride kalan, farklılaşan Yaratan Evlatlar tarafından türeyen bir biçimdeki diğer yarısı çeşitlilik arz eder. Fakat bu tür bir farklılaşma; ne Yaratıcı Ruhaniyet içindeki benzersiz kökenin bu yaratılmışlarını, ne de merkezi veya aşkın evrenlere özgü olan buraya getirilmiş bahse konu varlıkları tanımlar.
\vs p021 2:11 Bir Mikâil Evladı kendi evreni üzerinde bulunmadığı zaman, onun hükümeti yerel evren baş idarecisi olan Berrak ve Sabah Yıldızı biçimindeki ilk\hyp{}doğan yerel varlık tarafından yönetilir. Zamanın Birlikteliği’nin tavsiyesi ve danışması bu zamanlar için oldukça değerlidir. Onun bu yokluğunun zamanları boyunca bir Yaratan Evlat, yerleşik dünyalar üzerinde ve kendisinin fani çocuklarının kalplerinde onun ruhsal mevcudiyetinin üst denetimiyle birliktelik içinde bulunan Ana Ruhaniyet’i yetkilendirmeye yetkin haldedir. Buna ek olarak yerel bir evrenin Ana Ruhaniyet’i, yetiştirme için gösterdiği bakımı ve ruhsal hizmeti bu tür bir evrimsel nüfuz alanının en dış bölümlerine doğru genişleten bir biçimde, onun yönetim merkezinde kalmaya her zaman devam etmektedir.
\vs p021 2:12 Bir Yaratan Evlat’ın kendi yerel evreni içindeki kişisel mevcudiyeti, oluşturulan maddi yaratımın pürüzsüz bir biçimde işlemesi için gerekli değildir. Bu tür Evlatlar Cennet’e doğru yolculukta bulunabilirlerken, kendi evrenleri uzay üzerindeki döngülerine buna rağmen devam ederler. Onlar zamanın çocukları olarak ete kemiğe bürünmek için gücün kendi doğrultularını ellerinden bıraktığında bile; onların âlemleri, kendileriyle iniltili olan merkezler etrafında dairesel hareketlerine devam ederler. Hiçbir maddi düzenleme, Koşulsuz Mutlaklık’ın mekân mevcudiyetinin doğasında bulunan kâinatsal üst denetiminden veya Cennet mutlak\hyp{}çekim kavrayışından bağımsız bir konumda bulunmamaktadır.
\usection{3.\bibnobreakspace Yerel Evren Egemenliği}
\vs p021 3:1 Bir Yaratan Evlat’a, Cennet Kutsal Üçlemesi’nin rızası tarafından ve ilgili aşkın evrene ait olan üst denetimi sağlayan Üstün Ruhaniyet’in onaylaması ile birlikte bir evren alanı verilmiştir. Bu tür faaliyet, emanet edilmiş bir kâinatsal alan biçimindeki fiziksel iyeliğin kapsamını oluşturur. Fakat bir Mikâil Evladı’nın, idareciliğin bu başlangıçtaki ve kendiyle sınırlı olan düzeyinden kendisinin kazandığı egemenliğin deneyimsel üstünlüğüne doğru olan yükselimi; kâinat yaratımının ve ete kemiğe bürünmenin bahşedilmesinin görevi içinde ona ait olan kişisel deneyimlerin bir sonucu olarak gerçekleşir. Bahşedilmenin kazanılan egemenliğinin erişimine kadar o, Kâinatın Yaratıcısı’nın vekilleri olarak idaresini gerçekleştirir.
\vs p021 3:2 Bir Yaratan Evlat, herhangi bir zaman içinde kendi kişisel yaratımı üzerine bütüncül egemenliğini uygulayabilir; fakat o, bilge bir biçimde bunu yapmamayı tercih eder. Yaratılmış bahşedilmeleri sürecinin öncesinde eğer o, kazanılmamış bir yüce egemenliği edinmiş olsaydı, kendi yerel evreni içinde bulunan yerleşik Cennet kişilikleri bu âlemi terk ederlerdi. Fakat zaman ve mekânın tüm yaratılmışları boyunca böyle bir şey hiçbir zaman gerçekleşmemiştir.
\vs p021 3:3 Yaratanlığın gerçeği, egemenliğin tamamlanmışlığı anlamına gelmektedir; fakat Mikâiller bunu deneyimsel olarak \bibemph{kazanmayı} tercih eder, böylelikle yerel evren yönetimine bağlı olan tüm Cennet kişiliklerinin bütüncül eş güdümünü ellerinde bulundurmaya devam ederler. Biz, Mikâiller’in aksi bir biçimde bunu gerçekleştirdiklerinin bilgisine sahip değiliz; fakat onlar bütünüyle bunu yapmaya yetkindirler, çünkü onlar gerçek bir biçimde özgür iradeye sahip Evlatlar’dır.
\vs p021 3:4 Bir Yaratan Evlat’ın yerel bir evrendeki egemenliği, altı veya muhtemel bir biçimde yedi olan deneyimsel açığa vurumun düzeyinden geçer. Onlar şu düzeyler içinde ortaya çıkarlar:
\vs p021 3:5 1.\bibnobreakspace Öncül vekâlet egemenliği --- birliktelik içerisinde bulunan Yaratıcı Ruhaniyet tarafından kişisel niteliklerin elde edilmesinden önce, bir Yaratan Evlat tarafından uygulanan kalıcı olmayan yalnız hâkimiyeti.
\vs p021 3:6 2.\bibnobreakspace Bütünlük içerisindeki vekâlet egemenliği --- Evren Ana Ruhaniyeti’nin kişilik erişiminin ardından Cennet eşinin birleşik idaresi.
\vs p021 3:7 3.\bibnobreakspace Çoğalan vekâlet egemenliği --- onun yedi yaratım bahşedilmişlerinin süreci boyunca bir Yaratan Evlat’ın ilerleyici hâkimiyeti.
\vs p021 3:8 4.\bibnobreakspace Yücelik egemenliği --- yedinci bahşedilmişliğin tamamlanmasını takiben yerleştirilmiş hâkimiyeti. Nebadon içinde yücelik egemenliği tarihsel olarak, Urantia üzerinde Mikâil’in bahşedilmişliğinin tamamlanmasına dayanmaktadır. O sadece, sizin gezegensel zamanınıza göre bin dokuz yüz yılı biraz aşkın bir süredir mevcut bir haldedir.
\vs p021 3:9 5.\bibnobreakspace Çoğalan yücelik egemenliği --- ışık ve yaşam içinde yaratılmış nüfuz alanlarının bir çoğunluğunun yerleştirilmesinden doğan ilerlemiş ilişkisi. Bu düzey yerel evreninizin erişilmemiş geleceği ile ilgilidir.
\vs p021 3:10 6.\bibnobreakspace Kutsal Üçlemesel egemenlik --- ışık ve yaşam içindeki yerel evrenin bütününün yerleştirilmesinin ardından uygulanması.
\vs p021 3:11 7.\bibnobreakspace Açığa çıkarılmamış egemenlik --- gelecek bir evren çağının bilinmeyen ilişkileri.
\vs p021 3:12 Öngörülmüş bir yerel evrenin başlangıçsal vekâlet egemenliğin kabul edilmesinde bir Yaratan Mikâil, aşkın evren idarecileri tarafından yedi yaratılmış bahşedilmişliklerinin tamamlanmasına ve onaylanmasına kadar yüce egemenliği kullanmaya başlamayacağına dair Kutsal Üçleme’ye söz verir. Fakat eğer bir Mikâil Evladı bunu irade dâhilinde yerine getirmeseydi, ve böylece bu tür bir kazanılmamış egemenliğini açıklasaydı; aksi olan bu durumu gerçekleştirmek için söz vermenin hiçbir anlamı olmazdı.
\vs p021 3:13 Bahşedilmenin öncesinde olan çağlarda bile bir Yaratan Evlat, onun nüfuz alanlarının bölümlerinin herhangi birinde uyuşmazlığın bulunmadığı zamanlarda neredeyse yüce olan bir biçimde bu alanı idare eder. Eğer egemenlik hiçbir biçimde zorlanmasaydı sınırlı olan iradecilik herhangi bir biçimde açığa çıkmamış olurdu. Bir bahşedilme öncesi Yaratan Evlat tarafından bir evren içinde uygulanan egemenlik, kendisine karşı olan bir başkaldırının bulunmadığı zamanda olanına kıyasla başkaldırının olduğu zamanlardan daha büyük bir nitelikte değildir; fakat bu ilk durum için egemenliğin kısıtlılıkları gözle görülür bir biçimde olmayıp, ikincisinde ise bariz bir durumda bulunur.
\vs p021 3:14 Şayet nadiren de olsa bir Yaratan Evlat’ın idaresi veya hâkimiyeti sarsılırsa, saldırıya uğrarsa veya tehlikeye düşerse; onun ebedi bir biçimde kollanması, korunması ve savunulmasına ek olarak, eğer gerekli görülürse onun kişisel yaratımının eski haline getirilmesi taahhüt edilmiştir. Bu tür Evlatlar, yalnızca kendilerinin yaratmış oldukları yaratılmışlar veya kendilerinin seçmiş olduğu daha yüksek olan varlıklar tarafından zor durumda bırakılabilir veya sıkıntıya düşürülebilir. Bu ifade içerisinde, yerel bir evrenin üzerinde bulunan düzeyler üzerinden kökenini alan “daha yüksek olan varlıkların” pek olası olmayan bir biçimde bir Yaratan Evlat’ı sıkıntıya düşürmeyeceği anlaşılabilir; ve bu yorum doğru bir çıkarsamadır. Yine de eğer onlar bunu yapmayı tercih ederlerse bunu gerçekleştirmeye yetkin bir durumda bulunurlar. Erdem, kişilik ile birlikte irade dâhilinde olan bir niteliktir; doğruluk ise irade sahibi yaratılmışlar içinde kendiliğinden gerçekleşen bir doğaya sahip değildir.
\vs p021 3:15 Bahse konu bahşedilme sürecinin tamamlanmasından önce bir Yaratan Evlat, egemenliğin kendiliğinden konan belirli sınırlamalarıyla iradesini gerçekleştirmektedir; fakat bu tamamlanmış bahşedilme hizmetinin ardından o, kendisine ait olan çok katmanlı yaratılmışların türü ve benzerliği içinde kendisine ait olan mevcut deneyimin erdemi tarafından idaresini gerçekleştirecektir. Bir Yaratan, kendi yaratılmışları arasında yedi kez kısa süreli olan ikamesini gerçekleştirdiğinde, bir bahşedilme süreci tamamlandığında, bunun sonucunda o yüce bir biçimde evren hâkimiyeti içinde yerleşik bir konuma oturup; o, bir egemen ve yüce idareci biçiminde Üstün Evlat haline gelir.
\vs p021 3:16 Yerel bir evren üzerinde yücelik egemenliğini elde etmenin işleyiş biçimi şu yedi deneyimsel aşamayla gerektirmektedir:
\vs p021 3:17 1.\bibnobreakspace Deneyimsel olarak, ilgili düzey üzerinde yaratılmışların bahse konu benzerliği içinde ete kemiğe büründürülmüş bahşedilişin işleyiş biçimi boyunca varlığın yedi yaratılma düzeyine katılım.
\vs p021 3:18 2.\bibnobreakspace Yedi Üstün Ruhaniyet içinde kişilikleştirildiği biçimiyle, Cennet İlahiyatı’nın yedi katmanlı iradesinin her fazı için deneyimsel bir kutsamanın yapılması.
\vs p021 3:19 3.\bibnobreakspace Cennet İlahiyatı’nın iradesi için yedi kutsamadan birinin uygulanmasıyla birlikte eş zamanlı olarak yaratılma düzeyleri üzerinde yedi deneyimden her birinin kat edilmesi.
\vs p021 3:20 4.\bibnobreakspace Her yaratılmış düzeyi üzerinde, deneyimsel olarak Cennet İlahiyatı ve tüm kâinat usları için yaratılmış yaşamının doruk noktasının tasvir edilmesi.
\vs p021 3:21 5.\bibnobreakspace Her yaratılmış düzeyi üzerinde, deneyimsel olarak bahşedilme düzeyi ve kâinatın tümü için İlahiyat’ın yedi katmanlı olan iradesinin bir fazının açığa çıkarılması.
\vs p021 3:22 6.\bibnobreakspace Deneyimsel olarak, yedi katmanlı olan yaratılmış deneyiminin İlahiyat’ın idaresi ve doğasının açığa çıkarılması için kutsamanın yedi katmanlı deneyimiyle birlikte bütünleşmesi.
\vs p021 3:23 7.\bibnobreakspace Yüce Varlık ile yeni ve daha yüksek ilişkinin erişilmesi. Bu Yaratan\hyp{}yaratılmış deneyiminin bütünlüğünün etkisi; Yüce olan Tanrı’nın aşkın evren gerçekliğini ve Her Şeye Gücü Yeten Yücelik’in zaman\hyp{}mekân egemenliğini arttırıp, bir Cennet Mikâili’nin yüce yerel evren egemenliğini gerçek hale getirir.
\vs p021 3:24 Yerel bir evren içerisinde egemenliğin sorgusunun yerleşmesinde Yaratan Evlat sadece irade etmek için kendi becerisel liyakatini göstermekle kalmaz; o aynı zamanda Cennet İlahiyatları’nın doğasını ortaya çıkarıp, yedi katmanlı olan tutumunu tasvir eder. Yaratıcı’nın başatlığının yaratılmış takdiri ve onun sınırlı bir biçimde anlaşılması; bir Yaratan Evlat’ın alçak gönüllülükle kendisini yaratılmışlarının deneyimleri ve biçiminin yerine koyduğu, ona ait olan serüven ile alakalıdır. Bu ana Cennet Evlatları, Yaratıcı’nın sevgi dolu doğasının ve faydalı hâkimiyetinin gerçek açığa çıkarıcılarıdır. Bahse konu, Ruhaniyet ve Evlat ile birliktelik içerisinde bulunan bu Yaratıcı, kâinatsal âlemlerin tümü boyunca gücün, kişiliğin ve hükümetin tümünün kâinatsal başıdır.
\usection{4.\bibnobreakspace Mikâil Bahşedilmişleri}
\vs p021 4:1 Bahşedilmiş Yaratan Evlatları’nın yedi topluluğu bulunmakta olup; onlar tekrar eden sayıları uyarınca öyle bir biçimde sınıflandırılmışlardır ki, kendilerine ait olan âlemlerin yaratılmışları üzerine kendilerini bahşedebilirler. Onlar, yaratılmış\hyp{}Yaratan deneyiminin yedinci ve son bölümüne erişene kadar, başlangıçsal deneyimden ilerleyici bahşedilmişliğin beş ek alanına kadar uzanan bir kapsamda bulunmaktadırlar.
\vs p021 4:2 Avonal bahşedilmişlikleri her zaman fani bedenin benzerliği içinde bulunmaktadır; fakat bir Yaratan Evlat’ın yedi bahşedilmişliği, varlığın yedi yaratılmışlık düzeyi üzerindeki onun ortaya çıkışına katılır, ve İlahiyat’ın doğası ve iradesinin yedi başat dışavurumunun açığa çıkarılmasıyla ilişkilidir. İstisna olmaksızın tüm Yaratan Evlatlar; kendilerine ait olan yaratımın âlemleri üzerinde yerleşik ve yüce yetki alanını üzerlerine almadan önce, yaratmış oldukları çocuklarına kendilerini yedi kez bahşetmenin bu sürecinden geçerler.
\vs p021 4:3 Her ne kadar bu yedi bahşedilmişler farklı birimler ve evrenler için değişkenlik gösterse de, onlar her zaman fani\hyp{}bahşedilme serüveni ile bütünleşirler. Son bahşedilme içerisinde bir Yaratan Evlat, bazı yerleşik dünya üzerindeki daha yüksek olan fani ırkların birinin üyesi olarak ortaya çıkar. Genellikle o, hayvan kökenli insanların fiziksel durumunu yukarı bir düzeye taşımak için bir önceki dönemden aktarılan Âdemsel birikimin en büyük kalıtımsal mirasını taşıyan, ırksal topluluğun bir üyesi olarak açığa çıkar. Bir bahşedilmiş Evlat olarak onun yedi katmanlı süreci içinde sadece bir kez bir Cennet Mikâil’i, sizin Beytüllahim’in bebeği olarak kayıt altına aldığınız gibi, bir kadından dünyaya gelmiştir. Sadece bir kez o, evrimsel irade sahibi yaratılmışların en alt düzeyinin bir üyesi olarak yaşamış ve ölmüştür.
\vs p021 4:4 Onun bahşedilmişliklerinin her biri sonrasında bir Yaratan Evlat; orada Yaratıcı’nın bahşedilmişlik onayını elde etmek ve kâinat hizmetinin takip eden bölümüne hazırlanış öğrenimini almak için “Yaratıcı’nın sağ eline doğru” ilerler. Yedinci ve son olan bahşedilmişliği takiben bir Yaratan Evlat, kendi evreni üzerinde Kâinatın Yaratıcısı’ndan yüce hakimiyeti ve yetki alanını teslim alır.
\vs p021 4:5 Gezegeniniz üzerinde kutsal Evlat’ın son ortaya çıkışının, kendisine ait olan bahşedilmişlik sürecinin altı fazını tamamlamış olan bir Cennet Yaratan Evladı olduğu bu kayıtlara aittir. Bunun sonucunda o, Urantia üzerinde ete kemiğe bürünen yaşamının bilinçsel kavrayışını elden bıraktığında, gerçek anlamda “o tamamlandı” biçimindeki ifadesini söylemeye yetkin olan bir biçimde bunu gerçekleştirdiğinde, bu faz kelimenin tam anlamıyla tamamlanmış bir konumda bulunmaktaydı. Urantia üzerindeki onun ölümü, kendine ait olan bahşedilme sürecini tamamlamıştır; bu düzey bir Cennet Yaratan Evladı’nın kutsal sözünün yerine getirilmesindeki son aşamadır. Buna ek olarak, bu tür Evlatlar’ın yüce kâinat egemenleri biçimindeki bu deneyimi elde edildiğinde; onlar artık Yaratıcı’nın vekilleri olarak idarede bulunmamaktadırlar, bunun yerine onlar kendilerinin ayrıcalıkları hakları içerinde “Hâkimlerin Hâkimi ve Koruyucuların Koruyucusu” ismiyle hâkimiyetlerini gerçekleştirirler. Bahse konu belirli istisnalar dâhilinde, bu yedi katmanlı bahşedilme Evlatları koşulsuz bir biçimde ikamet ettikleri evrenler içinde yücedirler. Onun yerel evreni ile alakalı “cennet ve dünya üzerindeki tüm güçler” bu zafer sahibi olan tahta geçmiş Üstün Evlat’ın önüne serilmiştir.
\vs p021 4:6 Yaratan Evlatlar; kendilerine ait olan bahşedilme süreçlerinin tamamlanışını takiben, yedi katmanlı Üstün Evlatlar biçiminde ayrı bir düzen olarak ortaya çıkarlar. Kişisel bakımdan Üstün Evlatlar, Yaratan Evlatlar ile birlikte özdeşlik gösterir; fakat onlar, farklı bir düzen olarak ortaklaşa bir biçimde tanımlanan, bu tür bir benzersiz bahşedilme deneyiminden geçmişlerdir. Bir Yaratan bir bahşedilmeyi gerçekleştirmek için tasarımda bulunduğu zaman, gerçek ve kalıcı olan bir değişiklik gerçekleşmek için nihai bir son kazanır. Bahşedilmiş Evlat’ın hala ve yine de bir Yaratan olduğu doğrudur; fakat o, onun dünyalarını bütünüyle idare etmek ve onun âlemini yönetmek için gerekli olan hakkı bütünüyle kazanmış bir Üstün Evlat’ın deneyimsel düzlemine onu yükseltmiş olan ve bir Yaratan Evlat’ın kutsal seviyesinden onu sonsuza kadar ayıran bir yaratılmış deneyimini kendi doğasına eklemlemiştir. Bu tür varlıklar, kutsal ebeveynlikten kaynaklanan her şeyi içlerinde barındırıp, kusursuzlaştırılan yaratılmış deneyiminden türeyen her şey ile bütünleşirler. Bahse konu Tanrılar deneyimsel biçimde değerli ve muktedir olmalarına ek olarak kendi âlemlerinin nüfuz alanlarını idare etmek için nihai ve bütüncül bir biçimde sorumlu olmalarından önce insanınkine benzer olan eş değer bir deneyim sürecinden geçmeleri gerekirken, neden insan kendi alt düzey kökeninden ve ona uygulanan evrimsel sürecinden yakınmaktadır!
\usection{5.\bibnobreakspace Üstün Evlatlar’ın Kâinat ile İlişkisi}
\vs p021 5:1 Bir Üstün Mikâil’in gücü, Cennet Kutsal Üçlemesi ile olan deneyimlenmiş birliktelikten kaynaklandığı için sınırsızdır. Aynı zamanda bu güç, bahse konu yaratılmışların bu tür yönetime bağlı olmaları biçimindeki mevcut olan deneyimden kaynaklandığı için sorgulanmayan bir niteliktedir. Bir yedi katmanlı Yaratan Evlat’ın egemenliğinin doğası yücedir çünkü o:
\vs p021 5:2 1.\bibnobreakspace Cennet İlahiyatı’nın yedi katmanlı bakış açısıyla bütünleşir.
\vs p021 5:3 2.\bibnobreakspace Zaman\hyp{}mekân yaratılmışlarının bir yedi katmanlı tutumunu bünyesinde barındırır.
\vs p021 5:4 3.\bibnobreakspace Cennet tutumunu ve yaratılmış bakış açısını kusursuz bir olarak bütüncül bir biçimde birleştirir.
\vs p021 5:5 Bu deneyimsel egemenlik, bu nedenden dolayı Yüce Varlık içinde sonuçlanan Yedi Katmanlı Tanrı’nın kutsallığının her şeyi kapsamı dâhiline almış halidir. Buna ek olarak bir yedi katmanlı Evlat’ın kişisel egemenliği, ilgili zaman\hyp{}mekân sınırları içinde dışa vurulabilen bir biçimde Cennet Kutsal Üçlemesi’nin gücü ve idaresinin olası en bütüncül içeriğinin bütünleşmesi gibi Yüce Varlık’ın bir zaman içinde tamamlanacak olan gelecek egemenliğine benzer bir niteliktedir.
\vs p021 5:6 Yüce yerel evren egemenliğinin erişimiyle birlikte orada, mevcut evren çağı boyunca yaratılmış varlıkların bütünüyle yeni olan çeşitlerini yaratmak için bir Mikâil Evladı’ndan güç ve imkân aktarımı gerçekleşmektedir. Fakat bir Üstün Evlat’ın; varlıkların bütünüyle yeni olan düzeylerinin meydana gelmesi için sahip olduğu gücünün kaybı, açığa çıkışın süreci içinde ve çoktan oluşturulmuş hayat detaylandırılmasının görevi için hiçbir biçimde engel teşkil etmemektedir. Kâinat evriminin bu geniş işleyişsel düzeni, herhangi bir kesinti veya sınırlama olmadan devam etmektedir. Bir Üstün Evlat tarafından yüce egemenliğin erişimi, mevcut bir biçimde çoktan tasarlanmış ve yaratılmış olan ve onlar tarafından peşi sıra bir biçimde türeyeceklerin idaresi ve desteğine olan kişisel adanmışlığın sorumluluğu anlamına gelmektedir. Zaman içinde orada, çeşitli varlıkların neredeyse sonu gelmeyen bir evrimi gerçekleşebilir; fakat akli yaratılmışın bütünüyle yeni düzensel işleyişinden veya biçiminden hiçbiri, bu nedenden dolayı, bir Üstün Evlat’dan aracısız olan kaynağını almayacaktır. Bahse konu bu durum, herhangi bir yerel evren içinde yerleşik hale getirilmiş bir idarenin başlangıcına ait olan ilk aşamadır.
\vs p021 5:7 Yedi katmanlı bir bahşedilmiş Evlat’ın kendi âleminin sorgulanmayan egemenliğine olan yükselişi, göreceli olan kargaşanın ve çağlarca süren belirsizliğinin sona erişinin başlangıcı anlamına gelecektir. Bahse konu olan bu oluşumu takiben, herhangi bir zaman içinde ruhsallaştırılamayanlar nihai bir biçimde düzensiz hale geleceklerdir. Buna ek olarak herhangi bir zaman içinde kâinatsal gerçeklik ile birlikte eş güdüm haline getirilemeyenler ise nihai olarak yok edilecektir. Sonu olmayan bağışlama ve tarifi olmayan sabrın yükümlülükleri âlemlerin irade sahibi yaratılmışların sadakati ve bağlılığını kazanmak için bu doğrultuda olan bir çaba içinde yerine getirildiğinde, adalet ve doğruluk bu sürecin sonucundan her zaman üstün bir biçimde çıkacaktır. Bu bağlamda bağışlamanın iyileştiremediklerini adalet nihai olarak ortadan kaldıracaktır.
\vs p021 5:8 Üstün Mikâiller, egemen yöneticiler olarak konumsal bir biçimde bir kez görevlendirildikleri zaman kendilerine ait olan yerel evrenler içerisinde yüce bir niteliktedirler. Onların idaresi üzerinde olan bir kaç kısıtlama, belirli kuvvetlerin ve kişiliklerin kâinatsal öncül\hyp{}mevcudiyetin doğasında bulunanlardır. Bunların dışında kalan durumlarda ise Üstün Evlatlar, kendilerinin ilgili evrenlerinde yönetim, sorumluluk ve idari güç bakımından yüce bir konumdadırlar; onlar Yaratanlar ve Tanrılar gibi neredeyse her bakımından yüce bir niteliğe sahiptirler. Orada, bahse konu bir evrenin işleyişi ile alakalı olarak onların bilgeliğinin ötesinde hiçbir etki söz konusu değildir.
\vs p021 5:9 Yerel bir evren içerisinde oluşturulan egemenliğine olan onun yükseliminin ardından bir Cennet Mikâili, kendi nüfuz alanı içerisinde faaliyet halinde olan Tanrı’nın diğer Evlatları’nın tümünün bütüncül denetimine sahiptir; buna ek olarak o, kendisine ait olan âlemlerin ihtiyaçlarının kavramsallaşması uyarınca özgür bir yönetime sahip olabilir. Bir Üstün Evlat irade dâhilinde, yerleşik gezegenlerin evrimsel uyumlaştırılmasının ve ruhsal yargılamasının düzeninde çeşitlik gösterebilir. Buna ek olarak bu tür Evlatlar, özel gezegensel ihtiyaçların, bilhassa fani bedenin benzerliği içinde ete kemiğe bürünmenin gezegeni olan kısa süreli bahşedilmenin âlemiyle daha çok bir biçimde ilgili ve onların yaratılmış kısa süreli ikamesinin dünyalarıyla alakalı tüm konularında, kendilerine ait olan tercihlerin tasarılarını yapar ve onları harekete geçirir.
\vs p021 5:10 Üstün Evlatlar; sadece kendilerinin kısa süreli olan kişisel ikamesinin dünyalarında değil, fakat bir Hakimane Evlat’ın kendisini üzerinde bahşettiği tüm dünyalar üzerinde olan bir biçimde, onlara ait olan bahşedilmiş dünyalarıyla birlikte kusursuz bir iletişim içinde olan bir görünüme sahiptir. Bu iletişim, onların “tüm beden üzerine yaymaya” yetkin oldukları Gerçekliğin Ruhaniyeti biçimindeki kendilerine ait olan ruhsal mevcudiyet tarafından sağlanır. Bu Üstün Evlatlar aynı zamanda her şeyin merkezinde olan Ebedi Ana Evladı ile birlikte koparılması mümkün olmayan bir iletişimi sürdürürler. Onlar, en yüksek düzeyde bulunan Kâinatın Yaratıcısı’ndan zamanın âlemleri içinde gezegensel hayatın alt düzey ırklarına kadar uzanan bir duygudaş erişimi ellerinde bulundurur.
\usection{6.\bibnobreakspace Üstün Mikâiller’in Nihai Sonu}
\vs p021 6:1 Hiçbir kimse yönetimin kesinliğiyle birlikte ne yerel âlemlerin yedi katmanlı Üstün Egemenleri’nin nihai sonlarını ne de doğalarını tartışmak amacıyla tahmin yürütmeyebilir; yine de hepimiz, bu konularla ilgili olarak varsayımda bulunmaya devam ederiz. Biz; her Cennet Mikâili’nin sahip olduğu kökenin çift ilahiyat kavramsallaşmasının \bibemph{mutlaklığı} olduğu, ve bu nedenle onun Kâinatın Yaratıcısı ve Ebedi Evlat’ın sınırsızlığının mevcut fazlarını içinde bulundurduğu konusunda bilgilendirilmiş olmamıza ek olarak bunun gerçekliğine inanmaktayız. Mikâiller bütüncül sınırsızlıkla ilişkin olarak tamamlanmamış bir niteliğe sahip olmalıdır; bunun karşısında ise onlar, kendilerine ait olan köken bakımından sınırsızlığın parçasıyla ilgili olarak muhtemel bir biçimde mutlaktırlar. Fakat mevcut kâinat çağı içinde onların faaliyetlerini gözlemlediğimizde ise biz, sınırlı olandan nitelik bakımından daha üstün olan bir eylem tespit etmemekteyiz; varsayılan herhangi bir sınırlılığı aşan bir aşkınlıkta bulunan yetkinlikler kendisinden müstakil fakat henüz açıklığa çıkarılmamış bir nitelikte olmalıdır.
\vs p021 6:2 Bahşedilmiş\hyp{}yaratılmış süreçlerin tamamlanışı ve yüce kâinat egemenliğine olan yükselimi, sınırlı olan bir hizmetten daha fazlası için yetkinliğin görünüşü tarafından desteklenen bir Mikâil’in sınırlı\hyp{}eylem etkinliklerinin tamamlanmış özgürleşmesini temsil etmek zorundadır. Bu ilişki içerisinde bu tür Üstün Evlatlar’ın yaratılmış varlıklarının yeni biçimlerinin türetilmesi bakımından sınırlanmış olduklarını belirtmemiz hususunda, bir sınırlama kuşkuya yer bırakmayan bir biçimde onların sınırlılığı aşan olanaklarının özgürleşmesi tarafından gerekli hale getirilmiştir.
\vs p021 6:3 Bu açığa çıkarılmamış yaratan güçlerinin mevcut kâinat çağı boyunca kendinden müstakil bir halde kalacak olması bir hayli muhtemel bir durumdur. Fakat uzak bir gelecekte herhangi bir zaman zarfında dışsal uzayın şu an hareket halindeki evrenleri içinde yedi katmanlı bir Üstün Evlat ile yedinci bir düzeyde bulunan Yaratıcı Ruhaniyet arasındaki birlikteliksel bağlantı, nihai kâinat öneminin aşkın düzeyleri üzerindeki yeni unsurların, anlamların ve değerlerin görünümü tarafından katılan hizmetin absonit düzeylerine erişebilir.
\vs p021 6:4 Tıpkı Yüceliğin İlahiyatı’nın deneyimsel hizmetin erdemiyle gerçekleştirmesine benzer bir biçimde, Yaratan Evlatlar kendilerinin kavranılamaz doğaları içinde Cennet\hyp{}kutsallık potansiyellerinin kişisel gerçekleşmesine erişmektedir. Urantia üzerinde Hazreti Mikâil bir keresinde “Ben nihayete uzanan doğrultu, gerçeklik ve hayatının kendisiyim” demiştir. Buna ek olarak ebediyet içerisinde Mikâiller'in kelimenin tam anlamıyla, yüce kutsallıktan nihai absonit boyunca ebedi ilahiyat kesinliğine olan uzanımı sürecinde tüm kâinat kişilikleri için en başından beri göz alıcı olan yön biçimindeki, “nihayete uzanan doğrultu, gerçeklik ve hayatın kendisi” olma ile nihai olarak sonlandırıldıklarına inanmaktayız.
\vs p021 6:5 [Uversa’dan olan bir Bilgeliğin Kesinleştiricisi tarafından sunulmuştur.]
