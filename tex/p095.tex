\upaper{95}{Levant’da Melçizedek Öğretileri}
\vs p095 0:1 Hindistan doğu Asya’nın dinleri ve felsefelerinin çoğuna kaynaklık ederken, benzer bir biçimde Levant Batı dünyasının inanışlarının anavatanıydı. Salem din\hyp{}yayıcıları; Maçiventa Melçizedeği’inin müjdesine dair iyi haberleri her yerde duyuran bir biçimde Filistin, Mezopotamya, Mısır, İran ve Arabistan boyunca güneybatı Asya’nın tamamı üzerine yayılmıştı. Bu yerlerin bazılarında onların öğretileri meyve verdi; diğerlerinde onların başarıları çeşitlilik gösterdi. Zaman zaman onların başarısızlıkları bilgelik eksikliği, zaman zaman ise denetimlerinin ötesindeki koşullar nedeniyle ortaya çıkmıştı.
\usection{1.\bibnobreakspace Mezopotamya’daki Salem Dini}
\vs p095 1:1 M.Ö. 2000’li yıllarda Mezopotamya dinleri; Seth unsurlarının öğretilerini daha yeni kaybetmiş bir konumda olup, batıdaki çölden yağmış Bedevi Samileri ve kuzeyden gelmiş olan barbar atlılar olan iki istilacı topluluğun ilkel inanışlarının fazlasıyla etkisi altındaydı.
\vs p095 1:2 Ancak haftanın yedinci gününü onurlandırmaya dair öncül Âdem topluluklarının âdeti Mezopotamya’dan hiçbir zaman tamamen kaybolmamıştı. Sadece Mezopotamya dönemi boyunca yedinci gün, kötü şansların en kötüsü olarak değerlendirilmişti. O tabular sonucunda böyle bir konuma oturmuştu; şeytani olan yedinci günde seyahate çıkmak, yemek pişirmek veya ateş yapmak yasaya aykırıydı. Museviler; Şabattum olan yedinci güne dair Babil âdetinden kaynaklanan bir biçimde buldukları Mezopotamya tabularının çoğunu Filistin’e geri getirmişlerdi.
\vs p095 1:3 Her ne kadar Salem öğretmenleri Mezopotamya dinlerini geliştirmeye ve yükseltmeye katkıda bulunmuş olsalar da, çeşitli toplulukları tek Tanrı’nın kalıcı tanınmasına getirmede başarılı olamadılar. Bu türden öğreti yüz elli yıldan fazla bir süre boyunca yükseliş kazandı, ve daha sonra kademeli olarak çoklu ilahiyatların bir düzenine beslenen eski inanca yerini bıraktı.
\vs p095 1:4 Salem öğretmenleri; bir zamanlar ana ilahiyatların sayısını Bel, Şamaş, Nabu, Anu, Ea, Marduk ve Sin’den meydana gelen bir biçimde yediye indirmiş olarak, Mezopotamya tanrılarının sayısını fazlasıyla düşürmüşlerdi. Yeni öğretinin zirve noktasında onlar, Babil üçlüsü olan bu tanrılardan şu üç tanesini diğerleri karşısında daha yüce bir konuma getirdi: yer, deniz ve gökyüzünün tanrıları olan Bel, Ea ve Anu. Bunun yanı sıra başka üçlüler; hepsinin, And ve Sümer unsurlarının kutsal üçleme öğretilerin kalıntısı olduğu ve Melçizedek’in üç daireden oluşan nişanına olan Salem unsurlarının inancına dayandığı bir biçimde, farklı bölgelerde doğmuştu.
\vs p095 1:5 Salem öğretmenleri hiçbir zaman, tanrıların annesi ve cinsel doğurganlığın ruhaniyeti İştar’a olan rağbetin üstesinden bütünüyle gelemedi. Onlar, bu kadın tanrıya olan ibadetin geliştirilmesine fazlasıyla katkıda bulundu; ancak Babil unsurları ve onların komşuları, cinsel ibadetin bu kılık değiştirmiş türlerini hiçbir zaman gelişerek geride bırakamamışlardı. Daha önceden, Mezopotamya’nın tümünde her kadının, yaşamlarının başında en az bir kere, yabancıların idaresine kendilerini teslim etmesi herkes tarafından uygulanan bir adet haline gelmişti; bu uygulama, İştar tarafından zorunlu kılınmış bir bağlılık olarak düşünülmekteydi; ve doğurganlığın fazlasıyla, bu cinsel ilişki fedasına dayandığına inanılmaktaydı.
\vs p095 1:6 Melçizedek öğretisinin öncül ilerleyişi; Kiş’deki okulun önderi olan Nabodad’ın, tapınak fahişeliğinin yaygın uygulamalarına karşı önceden tasarlanmış, kararlı bir saldırıda bulunmaya karar vermesine kadar oldukça tatmin ediciydi. Ancak Salem din\hyp{}yayıcıları, bu toplumsal reformu getirmedeki çabalarında başarısız oldu; ve bu başarısızlığın enkazında, daha önemli nitelikteki ruhsal ve felsefi öğretilerinin tümü bu yenilgiye boyun eğdi.
\vs p095 1:7 Salem müjdesinin bu mağlubiyetini derhal; Filistin’i Astoreh, Mısır’ı İsis, Yunanistan’ı Afrodit ve kuzey kabilelerini Astarte olarak hali hazırda etkisi altına almış, İştar inanışına olan büyük artış izledi. Ve İştar’a yapılan bu ibadetin canlanmasıyla ilişkili bir biçimde Babil din adamları, geleceği yıldızlardan okuma uygulamalarına yeniden başladı; yıldızbilimi son büyük Mezopotamya canlanışını yaşadı, falcılık moda haline geldi, ve çağlar boyunca din\hyp{}adamlığı artan bir biçimde kötüleşti.
\vs p095 1:8 Melçizedek, her şeyin Yaratıcısı ve Yoktan\hyp{}Var\hyp{}Edicisi olarak tek Tanrı’yı öğretmeleri ve sadece ibadet vasıtasıyla gelen kutsal lütfun müjdesini duyurmaları hususunda takipçilerini uyarmıştı. Ancak, anlık devrimle yavaş evrimi yer değiştirmeye girişen bir biçimde haddinden fazla şeye kalkışmak, yeni gerçekliğin öğretmenlerinin sıklıkla yaptıkları hata olagelmiştir. Mezopotamya’daki Melçizedek din\hyp{}yayıcıları, buradaki insanlar için ortak ahlaki ölçütü haddinden fazla yükseğe çekmişti; onlar haddinden fazla şeye kalkışmış olup, soylu amaçları yenilgiye uğramıştı. Kâinatın Yaratıcısı’nın mevcudiyetine ait gerçekliği duyurma biçiminde belirli bir müjdeyi duyurmak için görevlendirilmiş bir konumda bulunmaktaydılar; ancak onlar, adetleri kökten değiştirme gibi dışarıdan değerli görünen amaca takılıp kaldılar; ve böylece onların büyük görevi amacından sapıp, hüsran ve yıkım içinde neredeyse tamamen kaybolmuşlardı.
\vs p095 1:9 Bir nesil içinde Kiş’de bulunan Salem yönetim merkezi sona erdi; ve tek Tanrı’ya olan inanışın örgütlenmiş duyurusu tüm Mezopotamya boyunca neredeyse tamamen durdu. Ancak Salem okullarının kalıntıları varlığını sürdürdü. Küçük topluluklar sağa sola dağılıp, buralarda tek Yaratan’a dair inanışlarını sürdürüp, Mezopotamya din\hyp{}adamlarının putperestliğine ve ahlaksızlığına karşı savaşmıştı.
\vs p095 1:10 Daha sonraki dönemin Musevi din\hyp{}adamlarının esaret altında buldukları ve bunun hemen sonrasında Musevi yazarlığına atfetmiş oldukları ilahilerin derlemesine ekledikleri, taşa işlenmiş bir biçimde Eski Ahit Mezmurları’nın çoğunu yazan kişiler; öğretilerinin reddi sonraki dönemin Salem din\hyp{}yayıcılarıydı. Babil’den gelen bu güzel mezmurlar, Bel\hyp{}Marduk’un tapınaklarında yazılmamıştı; onlar, öncül Salem din\hyp{}yayıcılarından gelen soyların çalışmasıydı; ve onlar, Babil din\hyp{}adamlarının büyüsel derlemelerine karşı çarpıcı bir tezatlık içindedir. Eyüp’ün Kitabı, Kiş’deki Salem okuluna ait ve Mezopotamya boyunca var olmuş öğretilere dair iyi bir temsildir.
\vs p095 1:11 Mezopotamyalı din kültürünün çoğu, Musevi edebiyatı ve ayin işleyiş düzenlerine olan girişine, Amenemope ve Akhenaton’nun çalışmaları vasıtasıyla ulaşmıştı. Mısırlılar dikkat çekici bir biçimde; öncül And Mezopotamyalılar’dan elde edilmiş ve Fırat vadisini işgal etmiş daha sonraki Babil unsurları tarafından oldukça geniş bir ölçüde yitirilmiş toplumsal ödevin öğretilerini muhafaza etmişlerdi.
\usection{2.\bibnobreakspace Öncül Mısır Dini}
\vs p095 2:1 Özgün Melçizedek öğretileri gerçekten de, buradan Avrupa’ya daha sonra yayıldıkları yer olan Mısır’da en güçlü kökleri salmışlardı. Nil vadisinin evrimsel dini dönemsel olarak; Fırat Vadisi’nin Nod, Âdem ve daha sonraki And topluluklarının üstün ırk kollarının varışı tarafıyla büyümüştü. Zaman zaman, Mısırlı toplum idarecilerin çoğu Sümerliler olmuştu. Bu dönemlerde Hindistan dünya ırklarının en yüksek karışımına ev sahipliği yaparken, benzer bir biçimde Mısır Urantia üzerinde bulunabilecek dinsel felsefenin en bütüncül karışımını teşvik etmişti; ve Nil vadisinden o, dünyanın birçok kısmına yayılmıştı. Museviler, dünyanın yaratımına dair fikirlerinden çoğunu Babil unsurlarından almıştı; ancak onlar kutsal Yazgı’nın kavramsallaşmasını Mısırlılar’dan elde etmişti.
\vs p095 2:2 Salem öğretisini Mezopotamya’dan ziyade Mısır için daha elverişli kılan eğilimler felsefi ve dinsel yerine siyasi ve ahlakiydi. Mısır’da her kabile önderi, tahta geçmek için verdiği mücadele sonrası, kendi kabile tanrısını özgün ilahiyat ve tüm diğer tanrıların yaratanı olarak duyurarak hanedanlığının devamlılığını amaçlamıştı. Bu şekilde Mısırlılar, daha sonraki evrensel bir yaratan İlahiyat öğretisi için bir basamak olarak, kademeli bir biçimde üstün bir tanrı düşüncesine alıştılar. Tek\hyp{}tanrı düşüncesi, tek Tanrı’ya olan inancın her zaman daha fazla zemin kazanması ancak hiçbir zaman çoktanrılıcılığın evrimleşen kavramsallaşmaları üzerinde tamamiyle baskın çıkamaması şeklinde birçok çağ boyunca gidip gidip geldi.
\vs p095 2:3 Çağlar boyunca Mısır toplulukları, doğa tanrılarına yapılan ibadete kendilerini vermişlerdi; daha özelden bahsedilecek olunursa, kırk ayrı kabilenin her biri, birinin boğaya, diğerinin aslana, üçüncüsünün koça ve böyle giden bir biçimde özel bir topluluk tanrısına sahipti.
\vs p095 2:4 Zaman içinde Mısırlılar; tuğla mahzenlerinde gömülmüş olanlar çürürken, tuğlasız mezarlar içine yerleştirilen ölü bedenlerin --- mumyalayarak --- sodayla birebir karıştırılan kumun etkisi vasıtasıyla korunduğunu gözlemledi. Mısırlılar, bedenin muhafaza edilişinin birinin gelecek yaşama doğru olan ilerleyişini kolaylaştırdığına inandı. Bedenin çürümesinden sonra uzak gelecekte bireyin düzgün bir biçimde tanınabilmesi içinde, tabuttakine benzer bir suretteki heykel olarak ceset ile birlikte lahitin içine bir defin başı yerleştirdiler. Bu türden heykel defin başlarının yapılması, Mısır sanatı içinde büyük gelişime sebebiyet verdi.
\vs p095 2:5 Çağlar boyunca Mısırlılar inançlarını, bedenin ve ölümden sonra sonuçsal bir biçimde ortaya çıkan memnuniyet verici kurtuluşun teminatı olarak mezarlara bağladılar. Her ne kadar beşikten mezara kadar yaşam için külfetli olsa da, büyüsel uygulamaların daha sonraki evrimi olabilecek en etkin biçimde, mezarların dininden onları kurtardı. Din\hyp{}adamları tabutlara, “alttaki dünyada insanın kalbinin ondan alınmasına” karşı koruyucu olduğuna inanılan uğur yazılarını kazırlardı. Yakın bir zaman içerisinde bu sihirli yazıların çeşitli derlemeleri, Ölüler Kitabı olarak bir araya getirilip, muhafaza altına alındı. Ancak Nil vadisi içinde büyüsel ayin düzeni öncül bir biçimde, bu dönemlerin ayinleri tarafından sıklıkla erişilemeyen bir düzeyde bilinç ve kişiliğin alanlarıyla ilişkili hale gelmişti. Ve ilerleyen zamanlarda detaylı mezarlardan ziyade bu etik ve ahlaki idealler kurtuluşa bağlanmıştı.
\vs p095 2:6 Bu dönemlerin hurafeleri en iyi biçimde, Mısır’da doğmuş ve buradan Arabistan ve Mezopotamya’ya yayılmış olan bir düşünce biçiminde bir iyileştirme aracı olarak tükürüğün etkinliğine dair genel inanış tarafından resmedilir. Horus’un Set ile olan efsanevi savaşında genç tanrı gözünü kaybetmişti; ancak Set alt edildikten sonra bu göz, yaraya tüküren ve onu iyileştiren bilge tanrı Thoth tarafından eski haline getirildi.
\vs p095 2:7 Mısırlılar uzunca bir süre boyunca; yıldızların gece vakti parıldamasının değerli ölülere ait ruhların kurtuluşunu temsil ettiğine inandı, diğer kurtulanların güneş tarafından kendi içine katıldıklarına inandılar. Belirli bir dönem boyunca güneşe olan derin saygı, atalara yapılan ibadetin bir türü haline geldi. Büyük piramidin eğilimli giriş geçidi; kralların varsayılan yerleşkesi olan sabit yıldızların hareketsiz ve yerleşmiş takımyıldızlarına kralın ruhunun doğrudan bir biçimde mezardan kalktığında gidebilmesi amacıyla, doğrudan doğruya Kutup Yıldızı’na bakmaktadır.
\vs p095 2:8 Güneşin eğimli ışınlarının bulutlar içindeki bir gedikten dünyaya doğru giriş yaptığı gözlemlendiğinde, onların, kral ve diğer dürüst ruhların üzerinde göksel bir merdivenden aşağı doğru yavaş yavaş inişlerinin işareti olduklarına inanılmıştı. “Kral Pepi, üzerinde annesine yükselmek için ışıltısını bir merdiven gibi altına serdi.”
\vs p095 2:9 Melçizedek beden içinde ortaya çıktığında Mısırlılar, çevre topluluklarınkinin çok üzerinde bir dine sahiplerdi. Onlar; büyü reçeteleriyle yerinde bir biçimde kuşandıklarında, araya giren kötü ruhaniyetlerinden kurtulup, “cinayet, soygun, yalancılık, evlilik\hyp{}dışı cinsel ilişki, hırsızlık ve bencillikte” bulunmayanların neşe âlemlerine kabul edildiği yer olan Osiris’in yargı binasına ilerleyebileceklerine inandılar.” Eğer bu ruh terazide tartılıp, aşağıya çekerse, Kadın Öğütücü olan cehenneme sevk edilir. Ve bu, göreceli olarak, çevredeki birçok topluluğun sahip olduğu inanışlara kıyasla bir gelecek yaşama dair gelişmiş bir kavramsallaşmaydı.
\vs p095 2:10 Birinin yeryüzünde beden içindeki yaşamına ait günahlar için daha sonra verilecek yargıya dair kavramsallaşma Mısır’dan Musevi din kuramına taşınmıştı. Yargı kelimesi Musevi Mezmurları Kitabı’nın tamamı içinde sadece bir kez geçmektedir; ve bahse konu bu mezmur, bir Mısırlı tarafından kaleme alınmıştı.
\usection{3.\bibnobreakspace Ahlaki Kavramların Evrimi}
\vs p095 3:1 Her ne kadar Mısır’ın kültür ve dini başlıca And Mezopotamyası’ndan elde edilmiş ve daha sonraki medeniyetlere geniş ölçüde Musevi ve Yunan unsurları vasıtasıyla aktarılmışsa da, Mısırlılar’ın toplumsal ve etik idealizminin büyük bir kısmı, hem de çok büyük bir kısmı, tamamiyle evrimsel bir gelişme olarak Nil vadisinde doğmuştu. Her ne kadar Andit kökenine ait gerçeklik ve kültürün büyük bir kısmının aktarılmış olmasına rağmen Mısır içinde; Mikâil’in bahşedilişinin öncesinde belirli her bir diğer alan içinde ortaya çıkmış benzer doğal yöntemlere kıyasla, tamamiyle bir insan gelişimi olarak ahlaki kültürün daha fazlası evirilmişti.
\vs p095 3:2 Ahlaki evrim tamamiyle açığa çıkarışlara bağlı değildir. Yüksek ahlaki kavramlar, insanın kendi deneyiminden elde edilebilir. İnsan; kutsal bir ruhaniyet kendi içerisinde ikamet ettiği için, ruhsal değerleri bile geliştirebilir, bireysel düzeydeki kendi deneyimsel yaşamından kâinatsal kavrayış elde edebilir. Bilinç ve kişiliğin bu türden doğal evrimi aynı zamanda; ilkçağ dönemlerinde ikinci Cennet Bahçesi’nden ve daha sonra Salem’deki Melçizedek yönetim merkezinden gelmiş gerçeklik öğretmenlerinin dönemsel varışları tarafından büyümüştü.
\vs p095 3:3 Salem müjdesinin Mısır’a giriş yapışından bin yıllık bir süre önce Mısır’ın ahlaki önderleri; adalet, hakkaniyet ve açgözlülükten kaçınmayı öğretmişlerdi. Üç bin yıl önce Musevi yazıtları kaleme alınmıştı, Mısırlılar’ın ortak ilkesi şuydu: “Kendisini ispat etmiş kişi ortak ölçütü doğruluk olandır; onun yolunda yürüyendir.” Onlar nezaket, ölçülülük ve sağduyuluğu öğretmişti. Bu çağın büyük öğretmenlerinden bir tanesinin iletisi şuydu: “Herkese karşı iyi davran, herkese karşı adil ol.” Bu çağın Mısırlı üçlüsü Gerçeklik\hyp{}Adalet\hyp{}Doğruluk’du. Urantia’nın tamamiyle insan kökenli dinleri arasında hiçbiri, Nil vadisinin bir zamanlar sahip olduğu insancılığının bu toplumsal ideallerinden ve ahlaki ihtişamından daha üstün bir konuma gelememişti.
\vs p095 3:4 Bu evrimleşen etiksel düşünceler ve ahlaki ideallerin toprağında Salem dininin varlık mücadelesi veren öğretileri yeşerdi. İyilik ve kötülüğün kavramları, “Yaşam uysal, ölüm suçlu olana verilmiştir” düşüncesine inanan bir topluluğun kalplerinde hazır karşılık buldu. “Uysal, derinden sevileni yapan; suçlu nefret edileni gerçekleştirendir.” Çağlar boyunca Nil vadisinin sakinleri, --- iyi ve kötü olarak --- doğru ve yanlışın daha sonraki kavramları üzerinde hiç kafa yormadan önce, ortaya çıkmakta olan bu etik ve toplumsal ortak ölçütlerle yaşamaktaydı.
\vs p095 3:5 Mısır ussal ve ahlaklıydı, ancak üstün bir biçimde fazla ruhsal değildi. Altı bin yıl boyunca sadece dört büyük tanrı\hyp{}elçisi Mısırlılar arasından çıkmıştı. Amenemope’yi bir mevsim boyunca takip ettiler; Okhban’ı öldürdüler; İkhnaton’u kabul ettiler, ancak bir kısa nesil boyunca şevksizce bunu gerçekleştirdiler; Musa’yı reddettiler. Ve tekrar altı çizilecek olursa, dini koşullar yerine siyasi olanlar tek Tanrı’ya dair Salem öğretileri adına Mısır oyunca büyük bir etkide bulunmayı İbrahim ve daha sonra Yusuf’un için kolaylaştırdı. Ancak Salem din\hyp{}yayıcıları ilk kez Mısır’a girdikleri zaman, Mezopotamyalı göçmenlerin değişikliğe uğramış ortak ahlaki ölçütleri ile karışmış olan evrimin bu oldukça yüksek etik kültürüyle karşılaşmışlardı. Bu öncül Nil vadisi öğretmenleri, İlahiyat’ın sesi biçiminde Tanrı’nın emri olarak vicdanı ilk kez duyuranlardandı.
\usection{4.\bibnobreakspace Amenemope’nin Öğretileri}
\vs p095 4:1 Zaman içinde Mısır’da, birçokları tarafından “insanın oğlu” ve diğerleri tarafından Amenemope olarak adlandırılan bir öğretmen doğdu. Bu kâhin; vicdanı doğru ve yanlış arasında karar vermenin en yüksek noktasına çıkarmış, günah için cezalandırmayı öğretmiş, ve güneş ilahiyatına başvurarak gerçekleşen kurtuluşu duyurmuştu.
\vs p095 4:2 Amenemope, zenginliklerin ve talihin Tanrı’nın hediyesi olduğunu öğretti; ve bu kavramsallaşma bütüncül bir biçimde, daha sonra ortaya çıkış halinde bulunan Musevi felsefesini renklendirdi. Bu soylu öğretmen, Tanrı\hyp{}bilincinin tüm davranışlarda belirleyici etken olduğuna inandı; her anın, Tanrı’nın mevcudiyetinin ve ona duyulan sorumluluğun gerçekleştirilişi içinde yaşanılması gerektiğini düşündü. Bu bilgenin öğretileri daha sonra Musevi diline çevrilmiş olup, Eski Ahit yazıya geçirilmeden uzun bir süre önce bu topluluğun kutsal kitabı olmuştu. Bu iyi insanın başlıca duyurusu, emanet edilen hükümet makamlarındaki doğruluk ve dürüstlük hususunda oğlunu eğitmesiyle ilgiliydi; ve çok öncesinin bu soylu eğilimleri, herhangi bir çağdaş devlet adamını oldukça memnun kılardı.
\vs p095 4:3 Nil’in bu bilge adamı, --- dünyasal olan her şeyin gelip geçici olması şeklinde --- “zenginliklerin kanat alıp uçtuklarını” öğretmişti. Onun büyük duası, “korkudan kurtarılmak” içindi. O, “insanların sözlerinden” “Tanrı’nın eylemlerine” doğru yüzlerini dönmeleri için herkesi ikna etmeye çabalamıştı. Öz bakımın o şunu öğretti: İnsan niyetini eder, ancak Tanrı onu yerine getirir. Musevi diline çevrilen onun öğretileri, Eski Ahit Özdeyişler Kitabı’nın felsefesini belirledi. Yunanca’ya çevrilerek, daha sonraki tüm Helenistik dini felsefeye renk kattı. Daha sonraki İskenderiyeli filozof Philon, Bilgeliğin Kitabı’nın bir nüshasını elinde buldurmuştu.
\vs p095 4:4 Amenemope; evrimin etik kurallarını ve açığa çıkarılışın ahlaki değerlerini muhafaza etmek için faaliyet göstermiş, yazılarıyla onları hem Museviler’e hem de Yunanlılar’a aktarmıştı. O, bu çağa ait dini öğretmenlerinin en büyüğü değildi; ancak o, ------ Batı’nın dini inancının en yüksek gelişimini evrimleştiren içindeki Museviler ve saf felsefi düşünceyi Avrupa’daki en yüksek düzeyine geliştirmiş Yunanlılar olarak --- Batı medeniyetin büyümesinde iki hayati halkanın daha sonraki düşüncesini renklendirmesi bakımından en etkili olanıydı.
\vs p095 4:5 Musevi Özdeyişler Kitabı’nda on beş, on yedi ve yirminci bölümlerin tamamına ek olarak yirmi ikinci bölümün on yedinci mısrasından yirmi dördüncü bölümün yirmi ikinci mısrasına kadarki yazıların tümü Amenemope’nin Bilgeliğin Kitabı’nın neredeyse harfi harfine aynısıdır. Musevi Mezmur Kitabı’nın ilk mezmuru Amenemope tarafından yazılmış olup, özünde İkhnaton’un öğretileridir.
\usection{5.\bibnobreakspace Dikkate Değer İkhnaton}
\vs p095 5:1 Kraliyet ailesinden bir kadın Mısırlı bir Salem doktorunun etkisiyle Melçizedek öğretilerini benimsediği zaman, Amenemope’nin öğretileri Mısır aklındaki yerini yavaşça kaybetmekteydi. Bu kadın, Mısır Hükümdarı İkhnaton olan oğlu üzerinde, Tek Tanrı’ya dair bu öğretileri kabul etmesi için baskın gelmişti.
\vs p095 5:2 Melçizedek’in beden içinde ortadan kayboluşundan itibaren bahse konu döneme kadar hiçbir insan varlığı, İkhnaton’unki kadar Salem’in açığa çıkarılmış dinine dair bu türden açık bir kavramsallaşmaya sahip olmadı. Bazı açılardan bu genç Mısır kralı, insan tarihi içinde en dikkate değer kişilerden bir tanesidir. Mezopotamya’da artış gösteren ruhsal bunalımın bu dönemi boyunca o Mısır’da, tek Tanrı olan El Elyon’un öğretisini canlı kıldı; böylece bu dönemlere göre gelecekte gerçekleşecek olan Mikâil’in bahşedilişinin dini temeli için hayati derecede önemli nitelikteki felsefi tek\hyp{}tanrı düşünüşünün sürekliliğini sağladı. Ve, diğer nedenlere ek olarak bu cesaretin tanınışı içinde İsa, İkhnaton’un ruhsal varislerinden bazılarının kendisini gördüğü ve onun Urantia’ya olan kutsal görevinin belirli fazlarını bir parça anladığı Mısır’a getirilmişti.
\vs p095 5:3 Melçizedek ve İsa arasında en büyük kişilik olan Musa, Musevi ırkı ve Mısır’ın kraliyet ailesi dünyasının ortak hediyesiydi; ve şayet İkhnaton Musa’nın çok yönlülüğünü ve yetkinliğini elinde bulundursaydı, şaşırtıcı dini önderliği ile eşleşebilecek siyasi bir dehayı sergilemiş olurdu; ve bunun sonrasında Mısır, bu çağın büyük tek\hyp{}tanrılı ülkesi haline gelirdi; ve eğer böyle bir şey gerçekleşmiş olsaydı, İsa’nın fani yaşamının büyük bir kısmını Mısır’da geçirebilecek olması az da olsa olası bir durumdur.
\vs p095 5:4 Tüm tarih boyunca hiçbir zaman herhangi bir kral bu olağanüstü İkhnaton’un gerçekleştirdiği gibi, bu derece tasarlanmış bir biçimde bir ülkenin tamamını çoktanrıcılıktan tek\hyp{}tanrıcılığa kaydırma girişiminde bulunmamıştı. Olası en yüksek derecedeki muhteşem kararlılıkla bu genç idareci; geçmişinden ayrılıp, ismini değiştirip, başkenti terk edip, yepyeni bir şehir inşa edip, insanlarının tümü için yeni bir sanat ve edebiyat yarattı. Ancak o haddinden fazla hızlı gitti; o, terk ettiğinde üzerinde durabileceğinden daha fazlası olarak, haddinden fazla şey inşa etti. Ve tekrar, düşmanlık ve baskının ileriki selleri Mısırlılar’ı arasına kattığı zaman her şeyin dini öğretilerine karşı gerçekleştiği bir biçimde, o insanlarının maddi istikrarı ve refahını sağlamada başarısız oldu.
\vs p095 5:5 Muhteşem bir biçimde kesin görüşe ve olağanüstü derecede kararlı amaca sahip bu insan Musa’nın siyasi bilgeliğine sahip olsaydı, Batı dünyasındaki dinin evriminin ve gerçekliğin açığa çıkarılışının bütüncül tarihini değiştirmiş olurdu. Yaşamı boyunca o, genel olarak itibarsızlaştırdığı din adamlarının etkinliklerine kısıtlılık getirmede yetkindi; ancak onlar inançlarını gizlice idare edip, genç kral güçten düşer düşmez faaliyete geçmeye başladılar; ve onlar, onun hükümdarlığı altında tek\hyp{}tanrılı dinin kurulmasıyla Mısır’ın daha sonraki sorunlarının tümünü ilişkilendirmede gecikmediler.
\vs p095 5:6 Oldukça bilge bir biçimde İkhnaton, güneş\hyp{}tanrısı kılıfı içinde tektanrıcılığı kurmaya çalıştı. Tanrıların tümünü güneşin ibadetine özümseyerek Kâinatın Yaratıcısı’na olan ibadete bu yaklaşım kararı, Salem doktorunun tavsiyesiyle gerçekleşmişti. İkhnaton; İlahiyat’ın babalığı ve anneliği ile ilgili bu dönemin mevcut Aton inancına ait yaygınlaşmış öğretileri alıp, insan ile Tanrı arasında içten bir ibadetsel ilişkiyi tanıyan bir din yarattı.
\vs p095 5:7 İkhnaton; Aton’un yaratanı ve her şeyin yüce Yaratıcısı olarak Tek Tanrı’ya yapılan kılık değiştirmiş ibadete birlikteliklerini yönlendirirken, güneş\hyp{}tanrısı olan Aton’a yapılan dışa doğru ibadeti koruyacak kadar bilgeydi. Bu genç öğretmen\hyp{}kral; yönetime geri geldiklerinde din\hyp{}adamlarının tamamiyle yok ettiği, otuz bir bölümden oluşan bir kitap halindeki “Tek Tanrı” ismindeki savın kalemi olarak çok üretken bir yazardı. İkhnaton aynı zamanda; Musevi yazarlığına atfedilmiş olan, Eski Ahit Mezmurları içinde şu an korunmaktaki yüz otuz yedi ilahiyi de yazmıştı.
\vs p095 5:8 Günlük yaşam içinde İkhnaton’un yüce kelimesi “doğruluktu”; ve o hızlı bir biçimde doğruluğun kavramsallaşmasını, milli etik kurallarına ek olarak uluslararasındaki kuralları da kapsayacak ölçüde genişletmişti. Bu topluluk; muhteşem düzeydeki kişisel dindarlığın bir nesli olup, Tanrı’yı bulma ve onu tanımak için daha ussal erkek ve kadınlar arasında mevcut bulunan içten bir arzu tarafından simgelenmişti. Bu dönemlerde toplumsal konum ve refah, kanun gözünde hiçbir Mısırlı’ya herhangi bir menfaat sağlamadı. Mısır’ın aile yaşamı; ahlaki kültürü korumak ve onu geliştirmek için fazlasıyla katkıda bulunmuş olup, Filistin’deki Museviler’in daha sonraki mükemmel aile yaşamı için ilham olmuştu.
\vs p095 5:9 İkhnaton’un müjdesinin ölümcül zaafı; Aton’un sadece Mısır’ın yaratanı olmadığı, “bu Mısır ülkesinin yanı sıra Suriye ve Kuş’u bile içine alan bir biçimde tüm yabancı ülkelere ek olarak insan ve hayvanlardan oluşan tüm dünyanın aynı zamanda yaratanı” olduğunu öne süren öğretisi biçiminde onun en büyük gerçekliğiydi. “O herkesi kendilere ait yere yerleştirir, tüm ihtiyaçlarını sağlar.” İlahiyatın bu kavramsallaşmaları yüksek ve yüceltilmişti; ancak onlar milli değildi. Dindeki ulusallığın bu eğilimleri, savaş meydanları üzerinde Mısır ordusunun coşkusunu arttırmada başarısız oldu; buna ek olarak onlar din\hyp{}adamlarına, genç kral ve onun yeni dinine karşı kullanması için etkin silahları vermiş oldu. İkhnaton, daha sonraki Musevi unsurlarınınkinden çok daha yukarıda bir İlahiyat kavramsallaşmasına sahipti; ancak o, bir ulus inşacısının amaçlarına hizmet etmek için haddinden fazla gelişmiş nitelikteydi.
\vs p095 5:10 Her ne kadar tek\hyp{}tanrı ideali İkhnaton’un ölümünden zarar görmüş olsa da, tek Tanrı’nın düşüncesi birçok topluluğun akıllarında varlığını sürdürdü. İkhnaton’un damadı, ismini Tutankhamen olarak değiştirerek eski tanrılara yapılan ibadete geri dönen bir biçimde din\hyp{}adamlarıyla beraber aynı yoldan ilerledi. Başkent Teb’e geri taşındı, ve din adamları nihai olarak tüm Mısır’ın yedide birinin iyeliğini kazanan bir biçimde arazilerle oburlaştı; ve yakın bir zaman içinde bahse konu din\hyp{}adamları düzeyine ait biri krallığı ele geçirmeye girişti.
\vs p095 5:11 Ancak din\hyp{}adamları, tektanrıcılık dalgasının tamamen üstesinden gelememişti; Artan bir biçimde onlar, tanrılarını bir araya getirmek ve birleştirmek zorunda bırakıldılar; her geçen gün tanrıların ailesi daha fazla küçülmekteydi. İkhnaton daha önceden, cennetlerin alevli yuvarlak düzlemini yaratan Tanrı ile özleştirmişti; ve bu düşünce, köklü değişiklikleri gerçekleştiren genç öldükten uzunca bir süre sonra insanların kalplerinde, hatta din\hyp{}adamlarınınkinde bile yanmaya devam etti. Tek\hyp{}tanrı kavramsallaşması, Mısır’da ve dünyada insanların kalplerinde hiçbir zaman ölmedi. Bu kavramsallaşma, İkhnaton’un oldukça büyük bir şevkle tüm Mısır’ın ibadeti için duyurmuş olduğu tek Tanrı olan bahse konu kutsal Yaratıcı’ya ait Yaratan Evladı’nın varışına kadar bile varlığını sürdürmüştü.
\vs p095 5:12 İkhnaton’un öğretisinin zayıflığı, sadece eğitimli Mısırlıların kendi öğretilerini tamamiyle kavrayabileceği türden gelişmiş bir dini sunması gerçeğinde yatmaktadır. Tarım işçilerinin olağan üyeleri hiçbir zaman onun müjdesini gerçek anlamıyla kavrayamamışlardı; ve onlar din\hyp{}adamlarına, bu nedenle, İsis’e ek olarak karanlık ve kötülüğün tanrısı olan Set’in ellerinde vahşi bir ölümden mucizevî bir biçimde yeniden dirildiği varsayılan eşi Osiris’e yapılan eski ibadet için geri dönmüşlerdi.
\vs p095 5:13 Tüm insanlar için ölümsüzlüğün öğretisi, Mısırlılar için haddinden çok daha fazla ileri nitelikteydi. Sadece krallar ve zenginlerin yeniden dirilişine söz verilmişti; böylelikle onlar oldukça dikkatli bir biçimde, karar gününe karşı tabutları içinde bedenlerini mumyalayıp, korumaktalardı. Ancak İkhnaton tarafından öğretildiği biçimiyle kurtuluş ve yeniden dirilişin demokrasisi nihai olarak, Mısırlılar’ın dilsiz hayvanların bile kurtuluşuna inandığı bir ölçüde, üstün geldi.
\vs p095 5:14 Her ne kadar tek Tanrı’ya olan ibadeti insanlarına uygulamak için bu Mısır yöneticisinin çabası görünürde başarısız olsa da, onun çalışmasının sonuçlarının hem Filistin’de ve hem de Yunanistan’da çağlar boyunca varlığını sürdürmüş olduğunun altı çizilmelidir; ve Mısır’ın böylece, Batı’nın daha sonraki insan topluluklarının tümüne Nil’in birleşik evrimsel kültürüne ek olarak Fırat’ın açığa çıkarımsal dinini aktarmada aracı haline gelmiş olduğu önemle belirtilmelidir.
\vs p095 5:15 Nil vadisindeki ahlaki gelişim ve ruhsal büyümenin bu büyük döneminin ihtişamı yaklaşık olarak, Museviler’in milli yaşamının başladığı dönemde hızlı bir biçimde ortadan kaybolmaktaydı; ve Mısır’daki kısa süreli ikametlerinin sonucunda bu Bedeviler, bahse konu öğretilerin büyük bir kısmını taşımış olup, ırksal dinlerinde barınan İkhnaton inanışlarının çoğunun devamlılığını sağlamışlardı.
\usection{6.\bibnobreakspace İran’da Salem Öğretileri}
\vs p095 6:1 Filistin’den Melçizedek din\hyp{}yayıcıların bazıları Mezopotamya boyunca büyük İran yaylasına geçmişti. Beş yüz yıldan fazla bir süre boyunca Salem öğretmenleri istikametlerini İran’a çevirdi; ve bütün ülke, bir yönetim değişikliği Salem inancının tek\hyp{}tanrı öğretilerini neredeyse tamamen sonlandırmış sert bir düşmanlığa zemin hazırladığında Melçizedek dinine doğru kaymaktaydı. İbrahimsel sözleşmenin öğretisi; İsa’dan önce altıncı yüzyıl olan ahlaki rönesansın bu büyük çağında Zerdüşt Salem müjdesinin için için yanan közlerini canlandırmak için ortaya çıktığında, İran’da neredeyse tamamen yok olmuş haldeydi.
\vs p095 6:2 Yeni bir dinin bu kurucusu; Mezopotamya’daki Ur’a yaptığı ilk din yolculuğunda, hepsinin dini doğasını güçlü bir biçimde etkilediği --- diğer tarihsel anlatımlarla birlikte --- Caligastia ve Lucifer isyanının tarihsel anlatımlarını hâlihazırda öğrenmiş olan gözü pek ve maceraperest bir gençti. Bunun uyarınca o; Ur’dayken gördüğü bir rüya sonucunda, insanlarının dinini yeniden şekillendirmeye girişmek için kuzeydeki evine geri dönüşün bir sürecine başladı. Zerdüşt, kutsallığın Musasal kavramsallaşması olarak bir adalet Tanrısı’na Musevi düşüncesini özümsedi. Yüce bir Tanrı’ya dair düşünce aklında kesindi; ve o tüm diğer tanrıları, Mezopotamya’da önceden duymuş olduğu kötü ruhaniyetlerin düzeyine sevk eden bir biçimde şeytanlar olarak alt bir konuma getirdi. O daha öncesinden, tarihsel anlatım Ur’da hala varlığını sürdürürken Yedi Üstün Ruhaniyet’e dair anlatıyı öğrenmişti; ve böylece o, Ahura\hyp{}Mazda’yı başına getiren bir biçimde yedi tanrıdan oluşan bir büyük yıldız sistemi yarattı. Bu alt tanrıları o; Doğru Hukuk, İyi Düşünce, Soylu Hükümet, Kutsal Kişilik, Sağlık ve Ölümsüzlük’ün olası en yüksek düşüncesiyle ilişkilendirmişti.
\vs p095 6:3 Ve bu yeni din, dualar ve ayinler değil --- çalışma olarak --- bir eylem inanışıydı. Onun Tanrı’sı, yüce bilgeliğin bir varlığı ve medeniyetin hamisiydi; o kötülük, eylemsizlik ve gericilikle savaşmayı göze almış savaşçıl bir dini felsefeydi.
\vs p095 6:4 Zerdüşt; ateşe olan ibadeti öğretmedi, ancak alevi, saf ve bilge Ruhaniyet’in evrensel ve yüce hâkimiyetinin bir simgesi olarak kullanmaya amaçladı. (Onun daha sonraki takipçilerinin bu simgesel ateşe hem derin saygı besleyip hem de ibadet ettiği tamamiyle doğrudur.) Nihai olarak, bir İran prensiyle olan konuşması üzerine bu yeni din, kılıçla yayıldı. Ve Zerdüşt kahramansı bir biçimde, “ışığın Koruyucusu’nun gerçekliğine” beslediği inancı için verdiği savaşta öldü.
\vs p095 6:5 Zerdüştlük, Yedi Üstün Ruhaniyet’e dair Dalamatiasal ve Cennet Bahçesel öğretileri devam ettiren tek Urantia’lı inanıştır. Kutsal Üçleme kavramsallaşmasını evrimleştirmede başarısız olsa da, bir biçimde Yedi Katmanlı Tanrı’nınkine yaklaşmıştır. Özgün Zerdüştlük, saf bir ikircikli düşünce düzeni değildi; her ne kadar öncül öğretiler kötülüğü iyiliğin zamandaki bir eş\hyp{}güdümü olarak resmetmişseler de, o, kesin bir biçimde, iyiliğin nihai gerçekliği içine ebediyetsel olarak gömülüydü. Sadece daha sonraki dönemlerde, iyilik ve kötülüğün eşit düzeyde birbirleriyle mücadele ettiği inancı yaygın olarak destek gördü.
\vs p095 6:6 Musevi yazıtlarında kayda geçirildiği gibi, cennet ve cehenneme ek olarak şeytanların savına dair Musevi tarih anlatımları başat olarak; her ne kadar Lucifer ve Caligastia’nın hala varlığını sürdüren tarih anlatımları üzerine inşa edilmiş olsa da, Museviler’in İranlılar’ın siyasi ve kültürel egemenliği altında bulunduğu dönemlerde Zerdüştlerin inançlarından elde edilmişti. Mısırlılar gibi Zerdüşt “karar gününü” öğretmişti, ancak o bu etkinliği dünyanın sonuyla ilişkilendirmişti.
\vs p095 6:7 İran’da Zerdüştlük’den sonra gelen din bile ondan dikkate değer bir biçimde etkilenmişti. İranlı din adamları Zerdüşt’ün öğretilerini güçle yerinden etmeyi amaçlarken, Mitra’ya olan ilkçağ inancını yeniden dirilttiler. Ve Mitraizm, bir süreliğine hem Musevilik ve hem de Hıristiyanlık’ın çağdaşı olarak Levant ve Akdeniz bölgeleri boyunca yayılmıştı. Zerdüşt’ün öğretileri böylelikle, şu üç büyük dini ardı ardına etkilemek için gelmişti: Musevilik, Hristiyanlık ve onlar ile birlikte Muhammed takipçilerinin dini.
\vs p095 6:8 Ancak Zerdüşt’ün yüce öğretileri ve soylu mezmurlarından; Zerdüşt’ün hiçbir zaman bir kez olsun desteklemediği safsatalara olan inanışların yerine getirilmesiyle bütünleşen, ölülerden duyulan büyük korkularıyla birlikte İranlılar tarafından gerçekleştirilen Zerdüşt’ün müjdesine ait çağdaş sapmalar çok büyük bir değişimdi.
\vs p095 6:9 Bu büyük insan; kararmış dünyasında sonsuz yaşama götüren ışığın patikasını insana göstermek için yanan oldukça sönük kalmış bu Salem ışığının tamamen ve kesin bir biçimde sönüşüne engel olmak amacıyla İsa’dan önceki altıncı yüzyılda birden bire ansızın ve hızlıca ortaya çıkan bu benzersiz topluluğun üyelerinden biriydi.
\usection{7.\bibnobreakspace Arabistan’da Salem Öğretileri}
\vs p095 7:1 Tek Tanrı’ya dair Melçizedek öğretileri, görece yakın bir döneme kadar Arap çölünde yerleşmiş hale geldi. Yunanistan’da olduğu gibi Arabistan’da da Salem din\hyp{}yayıcıları, Maçiventa’nın haddinden fazla örgütlenme ile ilgili yönergelerini yanlış anlamaları sonucu başarısız oldu. Ancak onlar; müjdenin askeri kuvvet veya toplumsal zorlamayla yayılması için gösterilecek her tür çabaya karşı uyarıda bulunan Melçizedek ihtarını bu şekilde yorumlamalarına rağmen geri adım atmamışlardı.
\vs p095 7:2 Ne Çin’de ne de Roma’da Melçizedek öğretileri, Salem’in kendisine oldukça yakın olan bu çöl bölgesinde daha bütüncül bir biçimde başarısızlığa uğramamıştı. Doğu ve Batı topluluklarının çoğunluğunun sırasıyla Budist ve Hıristiyan haline gelmelerinden uzunca bir süre sonra Arabistan çölü, geçmişte bulunduğu konumda binlerce yıl boyunca varlığını sürdürdü. Her kabile kendisine ait eski putlaştırılmış şeye ibadet etti; ve birçok bireysel aile, kendisine ait aile tanrılarına sahipti. Babilli İştar, Musevi Yahveh, İranlı Ahura ve Koruyucu Hazreti İsa’nın Hıristiyan Yaratıcısı aralarındaki uzun mücadele devamlılığını korudu. Hiçbir zaman bir kavramsallaşma diğerinin yerini tamamen elde etmeye yetkin olamamıştı.
\vs p095 7:3 Arabistan boyunca tek tük yayılmış bir biçimde aileler ve kavimler, tek Tanrı’ya dair çok kesin olmayan bir düşünceye sahip olmuşlardı. Bu türden topluluklar; Melçizedek, İbrahim, Musa ve Zerdüşt’e dair tarihsel anlatımlara fazlasıyla değer verdiler. Orada, İsasal müjdeye karşılık verebilecek çok sayıda merkez bulunmaktaydı; ancak çöl arazilerinin Hıristiyan din\hyp{}yayıcıları, Akdeniz ülkelerindeki din\hyp{}yayıcılar olarak faaliyet gösteren uzlaşmacılar ve yenilikçilere kıyasla katı ve taviz vermez bir topluluktu. İsa’nın takipçileri “dünyanın her yerine gidin ve müjdeyi duyurun” şeklindeki kesin emrini daha ciddiye almış olsalardı, buna ek olarak kendi belirledikleri koşulsal toplumsal gereksinimlerde daha az katı, duyurularında daha merhametli olsalardı, bunun sonucunda birçok yer mutlulukla marangozun oğlunun bu yalın müjdesini almış oldurdu, buna Arabistan da dâhil.
\vs p095 7:4 Büyük Levantlı tektanrıcılığın Arabistan’da kök salışı başarısız olsa da bu çöl arazisi, toplumsal zorunluluklarında daha az talepkar, ama yine de tek\hyp{}tanrılı olan bir inancı üretmeye yetkindi.
\vs p095 7:5 Çölün ilkel ve düzensiz inanışlarına dair bir kabile, ırk veya milli kökenli tek bir etken bulunmaktaydı; o, neredeyse tüm Arap kabilelerinin Mekke’deki belirli bir tapınaktaki belirli bir siyah taş putlaşmasına göstermeye gönüllü olduğu tuhaf ve yaygın nitelikli saygıydı. Ortak iletişim ve derin saygının bu noktası ileride İslam dininin kurulmasına yol açmıştır. Volkan ruhaniyeti olan Yahveh Musevi Sami toplulukları için ne anlam taşıdıysa, Kâbe taşı Arap kuzenleri için aynı anlama gelmişti.
\vs p095 7:6 İslam’ın gücü; tek bir İlahiyat olarak oldukça belirgin ve çok iyi bir biçimde tanımlanmış Allah’ın sunumu olmuştur; onun zayıf noktası, kadınların alt bir düzeye indirilişiyle beraber duyurusuyla askeri kuvveti ilişkilendirmesi olmuştur. Ancak o kararlı bir biçimde, “görülen ve görülmeyen her şeyi bilen”, her şeyin Tek Kâinatsal İlahiyatı’nın sunumuna bağlı kalmıştır. “O bağışlayıcı ve merhamet sahibidir.” “Gerçekten de Tanrı, insanların tümüne olan iyilikte cömerttir.” “Ve ben hasta olduğumda, beni iyileştiren O’dur.” “Çünkü her ne zaman üç kişi kadar çok kişi konuşsa, Tanrı dördüncüsü olarak oradadır”, çünkü o “ilk ve son, aynı zamanda görülen ve görülmeyen değil midir?
\vs p095 7:7 [Nebadon’un bir Melçizedek unsuru tarafından sunulmuştur.]
