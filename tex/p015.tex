\upaper{15}{Yedi Aşkın Evren}
\vs p015 0:1 Bir Yaratıcı olarak Kâinatın Yaratıcısı’nın ilişkisi bakımından âlemler neredeyse tamamen mevcudiyet dışı bir konumdadır; kendisi sadece kişilikler ile ilişki halindedir; o kişiliklerin Yaratıcısı’dır. Yaratan birliktelikler olarak Ebedi Evlat ve Sınırsız Ruhaniyet bakımından âlemler, Yaratan Evlatlar’ın ve Yaratıcı Ruhaniyetler’in birleşik idaresi altında sınırlanmış ve bireysel bir karakter kazandırılmıştır. Cennet Kutsal Üçlemesi bakımından Havona dışında, Havona sonrası mekân düzeyinin ilk döngüsü üzerinde idare yetkisini elinde barındıran sadece yedi aşın evren olarak yedi yerleşik evren bulunmaktadır. Yedi Üstün Ruhaniyet, merkezi Ada üzerinden kendi etkisini yaymakta olup; bu nedenle, muhteşem Kâinatın dış bölgelerinin çevresinden, Yedi Üstün Ruhaniyet’in yayılımının yedi merkezi bağlantı noktasından, ebedi Cennet Adası’nın merkezinden oluşan bir devasa burgaç olarak bu uçsuz bucaksız yaratımı meydana getirmiş olur.
\vs p015 0:2 Kâinatsal yaratımın maddileşmesinin öncül safhalarında, aşkın evrenin işleyişsel idaresinin ve yönetiminin taslağı oluşturuldu. İlk Havona sonrası yaratım yedi muazzam birimlere ayrıldı, ve bu yerel evren hükümetlerinin merkezi yönetim dünyaları tasarlandı ve inşa edildi. Bahse konu yönetimin mevcut olan taslağı yakın ebediyetten kaynaklanan bir biçimde varoluş içindedir, buna ek olarak bu yedi aşkın evrenin idarecileri Zamanın Ataları biçiminde doğru bir şekilde ifade edilir.
\vs p015 0:3 Yedi aşkın evren ile ilgili olarak uçsuz bucaksız olan bilgiler bütünü içerisinde, size sadece ona dair çok az şey söyleyebilmenin ümidi içerisindeyim; fakat görkemli gücün ve kusursuz uyumun içerisinde orada faaliyet gösteren Kâinatsal çekim mevcudiyetlerine ek olarak fiziksel ve ruhsal kuvvetlerinin akli denetiminin bir yöntemi olan bu alanlar boyunca bu bilgiler bütünü faaliyet içerisindedir. İlk olarak aşkın evren nüfuz alanlarının maddi düzenlenmesi ve fiziksel oluşumuna dair tatmin edici bir fikre sahip olmak önemlidir; çünkü bunun sonucunda siz, bu yedi aşkın evren boyunca sayısız yerleşik gezegenin üzerinde etrafa dağılmış olan irade sahibi yaratılmışların akıl gelişimleri ve ruhsal yönetimi için sağlanan muhteşem düzenlemenin önemini daha iyi kavramak için hazır bir konuma geleceksiniz.
\usection{1.\bibnobreakspace Yedi Aşkın Evren Mekân Düzeyi}
\vs p015 1:1 Kısa olan yıllarınızın bir milyonunun veya bir milyarının içinde yaşamış kuşaklarınızın bellekleri, gözlemleri ve kayıtlarının sınırlı kapsamı içinde; Urantia’ya ve bahse konu âlemlere ait olan tüm işlevsel niyetler ve amaçlar yeni mekâna keşfedilmemiş erişimin uzun bir serüvenini deneyimlemektedir. Fakat Uversa kayıtlarına göre, daha önce yapılan gözlemler ışığında, bulunduğumuz düzeyin çıkarsamaları ve daha kapsamlı olan deneyimiyle uyumlu olarak, ilaveten bu ve diğer bulguların sonuçlarına dayanan yargıların neticesinde biz; evrenlerin oldukça iyi idrak edilen bir düzen içinde bulunmasına ek olarak, İlk Muhteşem Kaynak ve Merkez ve onun ikamet ettiği evrenin etrafında görkemli bir büyüklük içerisinde dönerek kusursuz bir biçimde biçimde denetlenen hareket yörüngesinin bilgisine sahibiz.
\vs p015 1:2 Uzunca bir zamandır yedi aşkın evrenin, devasa ve genişliğine kıyasla boyunun çok uzun olduğu bir döngü olan muhteşem bir elips üzerinde katettiğini gözlemlemekteyiz. Güneş sisteminiz ve zamanın diğer dünyaları, sınırları açık bir biçimde belirtilmemiş mekâna doğru yönergesiz ve pusulası olmayan bir şekilde dönüşsel hareketini gerçekleştirmemektedir. Sizin içinde bulunduğunuz sistemin ait olduğu yerel evren, merkezi evreni çevreleyen çok geniş olan döngü etrafında saat yönünün tersi istikamette kesin ve açık olarak gözlenen biçimiyle dönüşünü gerçekleştirir. Bu Kâinatsal işleyiş düzeni çok mükemmel bir biçimde belirlenmiş olup; tıpkı güneş sisteminizi oluşturan gezegenlerin yörüngelerinin Urantia gök bilimcileri tarafından bilinmesine benzer bir biçimde bu belirleniş, aşkın evren yıldız gözlemcileri tarafından bilinmektedir.
\vs p015 1:3 Urantia, bütünüyle düzenlenmemiş aşkın bir evrenin yerel evreni içerisinde konumlanmıştır; buna ek olarak yerel evreniniz, kısmi biçimde tamamlanmış sayısız fiziksel yaratımlarıyla doğrudan doğruya yakınlık içerisindedir. Bu bakımdan siz, görece yeni olan âlemlerden birine ait bir konumda bulunmaktasınız. Fakat diğer bir yandan şu an itibariyle siz, ne bilinmeyen bölgelere doğru körü körüne dönüşünüzü gerçekleştirmektesiniz, ne de belirlenmemiş bir mekân üzerinde ne yaptığınızı bilmez bir biçimde seyir halindesiniz. Mevcut an içerisinde siz, gezegensel sisteminiz veya ondan öncekilerinin çok uzun bir zaman önce katettiği aynı mekân boyunca seyrinizi gerçekleştirmektesiniz; buna ek olarak uzak gelecek içerisinde bir gün sisteminiz veya onu takip edenler, şu an çok çabuk bir biçimde hareket eden seyriniz boyunca aynı mekân üzerinde bu döngüyü tekrar katedecektir.
\vs p015 1:4 Bu çağda ve Urantia üzerinde yönsel kavramın algılandığı biçimiyle ilk aşkın evren kuzeye doğru, Muhteşem Kaynaklar ve Merkezler’in ikamet ettiği ve Cennet ve Havona’nın merkezi evrenine göre ise yaklaşık olarak ters istikamette bulunan daha doğuya yakın bir doğrultuda dönüşünü gerçekleştirir. Batıya doğru ilişkin olan bu konum, ebedi Ada’ya göre zamanın alanlarının en yakın fiziksel yaklaşımını temsil eder. İkinci aşkın evren kuzeyde bulunan konumundan batıya doğru olan dönüşüne hazırlanırken; üçüncü aşkın evren bu anda, güneye doğru olan eğilimine çoktan yönelişiyle bu büyük mekân yörüngesinin en kuzey birimini elinde bulundurur. Dördüncü aşkın evren bahse konu an içinde, Muhteşem Çevreler’e karşıt yönde ileri bölgelere olan yaklaşımı biçiminde güneysel hareketin görece dosdoğru ilerleyişi üzerindedir. Beşinci aşkın evren, doğuya doğru dönüşü takip eden doğrusal güney ilerleyişi üzerinde hareketini sürdürürken, Eş Merkezler’e karşıt bir istikamette kendi konumunu yaklaşık olarak terk eder. Altıncı aşkın evren, bulunduğunuz evrenin kısmi bir biçimde yakınından geçtiği birim olan güney kavisinin büyük bir kısmını oluşturur.
\vs p015 1:5 Yerel evreniniz Nebadon, yedinci aşkın evren olan ve bizim zaman algımıza göre uzun zamandan beri aşkın evren mekân düzeyinin güneydoğu eğilimine yönelmemiş birinci ve altıncı yerel evren arasında döngüsünü sürdüren Orvonton’a bağlıdır. Günümüzde Urantia’nın ait olduğu güneş sistemi, birkaç milyar yılı aşkın bir süredir güney eğrisi etrafında dönmektedir; ve bu dönüş sayesinde siz şu an güneydoğu eğiliminin ötesine doğru ilerlemekte olup görece doğrusal olan kuzey yörüngesi boyunca hızla hareketinize devam etmektesiniz. Bahsi geçmeyen gelecek çağlar boyunca Orvonton, neredeyse doğrusal olan bu kuzey istikametinde seyrini sürdürecektir.
\vs p015 1:6 Urantia, yerel evreninizin sınırlarının çok dışına doğru olan bir sisteme aittir; buna ek olarak yerel evreniniz mevcut haliyle Orvonton’un çevresinde seyir halindedir. Sizin varlığınızın ötesinde hala mevcut olan diğer unsurlar bulunmaktadır; fakat siz, İlk Kaynak ve Merkez’in görece yakınlığındaki muhteşem döngü etrafında dönüşünü gerçekleştiren bu fiziksel sistemlerden mekân bakımından çok büyük bir ölçekte uzaklaştırılmış bir konumda bulunmaktasınız.
\usection{2.\bibnobreakspace Aşkın Evrenlerin Düzenlenmesi}
\vs p015 2:1 Sadece Kâinatın Yaratıcısı, mekân içinde yerleşik dünyaların mevcut olan sayısının ve konumunun bilgisine sahiptir; kendisi onların tamamını isimleri ve sayısal değerleriyle tek tek tanımlar. Ben size, yerleşik veya yerleşime elverişli gezegenlerin sadece yaklaşık olarak rakamsal değerini sunabilirim; çünkü bazı yerel evrenler, ussal yaşamlar için onlar dışında kalanlara kıyasla elverişli olan daha fazla dünyaya sahiptir. Bu nedenle, benim sizlere sunabileceğim yaklaşık değerler yalnızca maddesel yaratımın enginliğine dair bazı fikirlerin oluşması amacına hizmet etmesi içindir.
\vs p015 2:2 Muhteşem kâinat içerisinde yedi aşkın evren bulunmaktadır, ve onlar yaklaşık olarak bahse konu şu birimler tarafından meydana gelmektedir:
\vs p015 2:3 1.\bibnobreakspace \bibemph{Sistem}. Yaklaşık olarak bin adet yerleşik veya yerleşmeye açık olan dünya tarafından oluşan aşkın yönetimin temel birimidir. Kavurucu güneşler, soğuk dünyalar, sıcaklık yayan güneşlerin çok yakınında bulunan gezegenler ve yaratılmışların ikamesi için elverişli olmayan diğer alanlar bu topluluğa dâhil değildir. Yaşamı sağlamak için uyumlu hale getirilen bu bin dünya bir sistem olarak adlandırılır, fakat yaş bakımından daha genç olan sistemlerde bu dünyaların görece az sayıda olanları ikame edilebilir bir niteliktedir. Her yerleşik gezegen bir Gezegensel Emir’in hâkimiyeti altındadır; ve her yerel evren yönetim merkezi olarak bir mimari alana sahip olup burası bir Sistem Egemeni tarafından yönetilir.
\vs p015 2:4 2.\bibnobreakspace \bibemph{Takımyıldızı}. Yaklaşık olarak 100.000 tane yerleşime açık olan gezegen biçiminde yüz sistem bir takımyıldızını meydana getirir. Her takımyıldızı bir tane mimari yönetim merkezi alanına sahip olup burası En Yüksektekiler olan üç Vorondadek Evlatları’nın yönetimi altındadır. Her takımyıldızı aynı zamanda, Cennet Kutsal Üçlemesi’nin bir elçisi olan gözlemle görevli bir Zamanın İnançlısı’na sahiptir.
\vs p015 2:5 3.\bibnobreakspace \bibemph{Yerel Evren}. Yaklaşık olarak 10.000.000 tane yerleşime açık olan gezegenden meydana gelmekte olan yüz takımyıldızı bir yerel evreni oluşturur. Her yerel evren muhteşem bir yönetim merkez dünyasına sahiptir, ve burası Mikâil’in düzeyi olan eş güdüm halindeki Tanrı'nın Yaratan Evlatları’nın bir tanesi tarafından idare edilir. Her evren, Cennet Kutsal Üçlemesi’nin bir temsilcisi olan bir Zamanın Birlikteliği mevcudiyeti tarafından kutsanmıştır.
\vs p015 2:6 4.\bibnobreakspace \bibemph{Azınlık Birimi}. Yaklaşık olarak 1.000.000.000 tane yerleşime açık olan gezegenden meydana gelmekte olan bin yerel evrenin oluşturduğu aşkın evren yönetiminin bir azınlık birimini oluşturur; burası muazzam bir yönetim merkezi dünyasına sahip olup, Zamanın Geçmişleri olarak onun yöneticileri azınlık biriminde gerçekleşen olayları idare ederler. Her azınlık birimi merkezleri üzerinde, Yüceliğin Kutsal Üçleme Kişilikleri olan üç Zamanın Geçmişi bulunmaktadır.
\vs p015 2:7 5.\bibnobreakspace \bibemph{Çoğunluk Birimi}. Yaklaşık olarak 100.000.000.000 tane yerleşime açık olan gezegenden meydana gelmekte olan bin azınlık birimi bir tane çoğunluk birimini meydana getirmektedir. Her çoğunluk birimi, muhteşem bir yönetim merkezine sahip olup, burası Yüceliğin Kutsal Üçleme Kişilikleri olan Zamanın Kusursuzlukları’nın üçünün idaresi altındadır.
\vs p015 2:8 6.\bibnobreakspace \bibemph{Aşkın Evren}. Yaklaşık olarak 1.000.000.000.000 tane yerleşime açık olan gezegenden meydana gelmekte olan on çoğunluk birimi bir aşkın evreni meydana getirir. Her aşkın evren, devasa ve muazzam yönetim merkezleriyle donatılmış olup buralar Zamanın Ataları’nın üçü tarafından yönetilir.
\vs p015 2:9 7.\bibnobreakspace \bibemph{Muhteşem Kâinat}. Yedi aşkın evren, yaklaşık olarak yedi trilyon yerleşime açık dünyaya ek olarak Havona’nın bir milyar yerleşim bölgesinden oluşan mevcut haldeki düzenlenmiş muhteşem Kâinatı meydana getirir. Aşkın evrenler, Yedi Üstün Ruhaniyet tarafından Cennet’in konumsal yerleşkesinden dolaylı ve aracısal bir biçimde yönetilip hükmedilir. Havona’nın milyarı aşan dünyaları doğrudan bir biçimde, bu tür bir Yüceliğin Kutsal Üçleme Kişiliklerinin bahse konu kusursuz alanlarının her birine hükmetmesi biçiminde, Zamanın Ebediyetleri tarafından idare edilir.
\vs p015 2:10 Cennet\hyp{}Havona âleminin dışında kalan kâinat düzeninin tasarımı aşağıdaki şu verileri sağlamaktadır:
\vs p015 2:11 Aşkın Evrenler \bibdf 7
\vs p015 2:12 Çoğunluk Birimleri \bibdf 70
\vs p015 2:13 Azınlık Birimleri \bibdf 7,000
\vs p015 2:14 Yerel Evrenler \bibdf 700,000
\vs p015 2:15 Yıldız Takımları \bibdf 70,000,000
\vs p015 2:16 Yerel Sistemler \bibdf 7,000,000,000
\vs p015 2:17 Yerleşime Elverişli Gezegenler \bibdf 7,000,000,000,000
\vs p015 2:18 Yedi aşkın evrenden her biri yaklaşık olarak şu oluşumlardan meydana gelmiştir:
\vs p015 2:19 Bir sistemin oluşumu yaklaşık olarak \bibdf 1,000 dünyadan
\vs p015 2:20 Yüz sistemden meydana gelen bir takımyıldızının oluşumu \bibdf 100,000 dünyadan
\vs p015 2:21 Yüz takımyıldızından meydana gelen bir evren oluşumu \bibdf 10,000,000 dünyadan
\vs p015 2:22 Yüz evrenden meydana gelen bir azınlık birimi oluşumu \bibdf 1,000,000,000 dünyadan
\vs p015 2:23 Yüz azınlık biriminden meydana gelen bir çoğunluk birimi oluşumu \bibdf 100,000,000,000 dünyadan
\vs p015 2:24 On çoğunluk biriminden meydana gelen bir aşkın evren oluşumu \bibdf 1,000,000,000,000 dünyadan
\vs p015 2:25 Tüm bu tahmini ölçümler yaklaşık olarak bu bağlamda yapılabileceklerinin en iyisini yansıtır; diğer düzensel oluşumlar geçici bir süreliğine maddi mevcudiyetin yok oluşuyla karşılaşırken, bunun karşısında yeni sistemler aralıksız bir biçimde evirilmeye devam etmektedir.
\usection{3.\bibnobreakspace Orvonton’un Aşkın Evreni}
\vs p015 3:1 İşlevsel olarak yıldızsal âlemlerin tümü, Orvonton’un aşkın evreni olarak muhteşem Kâinatın yedinci bölümüne ait olan Urantia üzerinde çıplak gözle görünebilir durumdadır. Geniş Samanyolu yıldız sistemi, yerel sisteminizin sınırlarının oldukça ötesinde olan Orvonton’un merkezi çekirdeğini temsil eder. Bu muhteşem güneşlerin, mekânın karanlık adalarının, çifte yıldızların, küresel kümelerin, yıldız bulutlarının, buna ek olarak sarmal ve diğer nebulaların bir araya gelişi sayısız bireysel gezegenlerle birlikte; yerleşik evrimsel âlemlerin yaklaşık olarak bir bölü yedisinin uzatılmış\hyp{}çevrimsel toplulukları biçiminde bir saat gibi işleyişini oluşturur.
\vs p015 3:2 Urantia’nın gök bilimsel konumundan yakınınızda bulunan sistemlerin kesişim noktası boyunca muhteşem Samanyolu’na doğru baktığınızda, genişliği kalınlığından çok daha fazla olan ve uzunluğu genişliğinden bile daha büyük olan uçsuz bucaksız bir uzatılmış düzlem üzerinde seyahat ettiğinizi gözlemlersiniz.
\vs p015 3:3 Samanyolu olarak tarif ettiğiniz yapının gözlemi; cennetler tek bir yönden gözlendiğinde Orvonton üzerindeki yıldızsal yoğunluğun karşılaştırmalı olarak artışını ve aynı zamanda herhangi farklı bir taraftan bakıldığında ise onun yoğunluğun azalışını açığa çıkarır. İçinde bulunduğumuz maddesel aşkın evrenin merkezi düzleminden ise yıldızların ve diğer alanların sayılarının azalmakta olduğu gözlemlenir. En yüksek yoğunluğun bu alanının ana bedensel bütünlüğüne doğru dikkatlice baktığınızda, gözlem açısı müsait olduğu takdirde siz, her şeyin merkezi ve yerleşik evrene doğru bakmakta olduğunuzu fark edersiniz.
\vs p015 3:4 Orvonton’un on büyük kısmından sekizi Urantialı gökbilimcileri tarafından kabataslak bir biçimde saptanmıştır. Geride kalan iki büyük kısım ayrımsal fark ediliş bakımından zor bir konumda bulunmaktadır, çünkü bu olgular bütününü görebilmeniz için onları içeriden görebiliyor olmanız gerekmektedir. Orvonton’un aşkın evrenine uzayın çok uzak bir konumundan bakabilirseniz, yedinci galaksinin on büyük kısmını eş zamanlı olarak ayırt edebilirsiniz.
\vs p015 3:5 Sizin azınlık biriminizin dönüşümlü merkezi; yerel evreniniz ve onunla birliktelik içinde bulunan yaratılmışların hepsinin onun etrafında hareket ettiği Yay Takımyıldızı’nın devasa büyüklükte ve yoğun olan yıldız bulutunun çok uzağında konumlanmıştır. Buna ek olarak bu çok geniş olan Yay Takımyıldızı’nın alt galaktik sisteminin karşıt taraflarından, muhteşem yıldız sarmalının içinde ortaya çıkan bulutların iki büyük akımını gözlemleyebilirsiniz.
\vs p015 3:6 Güneşinizin ve onunla birliktelik halinde bulunan gezegenlerin ait olduğu fiziksel sistemin çekirdeği, eskinin Andronover nebulasının merkezidir. Bu eskinin sarmal nebulası, güneş sisteminizin doğuşu üzerine görevlendirilen olaylarla ilişkili çekim kesintileri tarafından hafif bir biçimde bozulmaya uğramıştır; buna ek olarak bu duruma, geniş komşu bir nebulanın yakın biçimde yaklaşımı sebebiyet vermiştir. Neredeyse bütünsel olan bu çarpışma, Andronover’in küresel bir birlikteliğe benzer bir biçimde dönüşmesine sebebiyet vermiştir; fakat bu olay güneşlerin iki yönlü takibinin ve onların birliktelik içinde bulunduğu fiziksel topluluğun tamamiyle tahrip olmasıyla sonuçlanmamıştır. Şu an güneş sisteminiz, yaklaşık olarak merkezden yıldız sisteminizin uç noktasına kadar olan uzaklığın orta konumunda bulunan bu bozulmaya uğrayan sarmalın kollarının birinde oldukça merkezi bir konumu elinde bulundurur.
\vs p015 3:7 Yay Takımyıldızı birimi, tüm diğer birimler ve Orvonton’un bölümleri Uversa etrafında dönüşümlü olan döngüsel bir yapıdadır. Ve Urantialı yıldız gözlemcilerin sahip olduğu bir takım kafa karışıklıkları, aşağıda bahsi geçen şu çoklu döngüsel hareket tarafından üretilmiş görece bozulmaların ve yanılsamaların sonucunda açığa çıkar:
\vs p015 3:8 1.\bibnobreakspace Urantia’nın kendi güneş etrafında dönüşü.
\vs p015 3:9 2.\bibnobreakspace Eski Andronover nebulasının çekirdeği etrafında güneş sisteminizin dönüşü.
\vs p015 3:10 3.\bibnobreakspace Andronover yıldız ailesinin ve onunla birliktelik içerisinde bulunan kümenin, Nebadon’un yıldız bulutunun bütüncül döngüsel\hyp{}çekim merkezinin etrafında dönüşü.
\vs p015 3:11 4.\bibnobreakspace Nebadon’un yerel yıldız bulutunun ve onun birliktelik içerisinde bulunan yaratılmışlarının Yay Takımyıldızı’nın sahip olduğu azınlık birimi etrafında dönüşü.
\vs p015 3:12 5.\bibnobreakspace Yay Takımyıldızı’na ek olarak yüz azınlık biriminin kendilerinin çoğunluk birimi etrafında dönüşü.
\vs p015 3:13 6.\bibnobreakspace Yıldız akıntısı olarak da adlandırılan on çoğunluk biriminin Orvonton’un Uversa yönetim merkezleri etrafında dönüşü.
\vs p015 3:14 7.\bibnobreakspace Aşkın evren mekân düzeyinin saat yönü tersine olan hareket yörüngesi biçimindeki, Cennet ve Havona etrafında Orvonton’un ve onun birliktelik içerisinde bulunduğu altı aşkın evreninin hareketi.
\vs p015 3:15 Bu birkaç düzenin çoklu hareketleri hususunda gezegeninizin uzay yörüngesi ve güneş sisteminiz doğalarından gelen köken bakımından kalıtsaldır. Aynı zamanda Orvonton’un mutlak saat yönünün tersi istikametteki hareketi, üstün evrenin mimari tasarılarının doğasından kaynaklanan bir biçimde kalıtsaldır. Fakat bu düzene katılan hareketler; kısmen Cennet’in kuvvet düzenleyicilerin ussal ve bilinçli eylemi tarafından üretilen ve kısmen bütünleştirici madde\hyp{}enerji bölünmesinden aşkın evrenlere kadar elde edilen bütüncül kaynağa aittir.
\vs p015 3:16 Yerel evrenler Havona’ya yaklaştıklarında birbirlerine daha yakın bir konuma gelirler; döngüler sayı bakımından fazlalaşır ve orada katman üzerine katman biçimindeki üst üste ekleniş artar. Fakat ebedi merkezin dışına doğru gidildikçe daha az sayıda sistem, katman, döngü ve âlem bulunmaktadır.
\usection{4.\bibnobreakspace Âlemlerin Ataları olarak Nebula}
\vs p015 4:1 Yaratım ve evren düzenlemesi sonsuza kadar, sınırsız Yaratıcılar ve onlarla birliktelik içinde bulunanların denetimi altında kalacak olsa da; bu düzenin olgusal bütünlüğü, emredilen bir işleyiş yöntemi uyarınca ve maddenin, enerjinin ve kuvvetin çekim yasalarına bağlı olarak gerçekleşir. Fakat mekânın Kâinatsal kuvvet\hyp{}etkisiyle ilgili olarak gizemsel birtakım unsurlar bulunmaktadır. Biz maddesel yaratımların düzenlenmesini onların ultimatonik düzeyinden başlayarak sonuna kadar bütünüyle kavramaktayız, fakat ultimatonların Kâinatsal kökenlerinin tümüyle bilincine sahip değiliz. Biz bu kökensel kuvvetlerin bir Cennet özüne sahip olduğundan eminiz, çünkü onlar Cennet’in kesin olan devasa hatları içinde yerleşime açık mekân boyunca sonsuza kadar dönmektedirler. Her ne kadar Cennet çekimine bağlı olmasa da, tüm maddeleşme sürecinin kökeni olan mekânın bu kuvvet\hyp{}etkisi; Alt Cennet merkezi içinde ve dışında açık bir biçimde döngü halinde olan alt Cennet’in mevcudiyetine her zaman tabidir.
\vs p015 4:2 Cennet kuvvet düzenleyicileri, mekân gücünü ezeli kuvvete dönüştürür; ve bununla birlikte bu madde öncesi potansiyelin, fiziksel gerçekliğin birincil ve ikincil enerji dışavurumlarına evirilmesini sağlar. Bu enerji çekim\hyp{}karşılık düzeylerine erişince; güç yöneticileri ve onların aşkın evren düzen yardımcıları sahneye çıkar, buna ek olarak mekânın ve zamanın âlemlerinin enerji kanallarını ve çok katmanlı güç döngülerini oluşturması için tasarlanan, kendilerine ait sonu olmayan etkilerini uygulamaya koyulurlar. Bu nedenle fiziksel madde mekân içinde açığa çıkar, ve evrenin işleyişsel düzenlenmesinin faaliyetine başlaması böylelikle hazırlanmış olur.
\vs p015 4:3 Enerjinin bu bölünmesi, Nebadon fizikçileri tarafından hiçbir zaman açığa çıkarılamamış bir olgudur. Onların çektiği başlıca zorluk, Cennet kuvvet düzenleyicilerinin görece erişilemez oluşunda yatmaktadır; çünkü yaşayan güç yöneticileri her ne kadar mekân enerjisiyle başa çıkmakta yetkin olsalar da, oldukça yetkin ve ussal bakımından son derece etkin bir biçimde yönlendirebilecekleri enerjilerin kaynaksal kavramsallaşmasına sahip değillerdir.
\vs p015 4:4 Cennet kuvvet düzenleyicileri nebula yaratıcılarıdır; onlar, kâinat maddesinin ultimatonik birimlerinin nihai açığa çıkışları için yayılımcı kuvvetlerin tümünün harekete geçirilmesine kadar bir kere başladıktan sonra hiçbir zaman kısıtlanamayacak veya sonlandırılamayacak olan devasa kuvvet tufanının kendi mekân mevcudiyetini sunmaya yetkindirler. Bu nedenle, sarmal ve diğer nebulalar, doğrusal\hyp{}köken güneşin ve onların çeşitli sistemlerine ait ana burgaçları mevcut hale getirilmiştir. Dış uzayda öncül Kâinatsal evrimin fazları olan nebulanın farklı biçimlerde gözlenen on değişik bütünlüğü bulunabilir, ve bu geniş enerji burgaçları yedi aşkın evrende tıpkı ait olduğu gibi aynı kökene sahip olmuştur.
\vs p015 4:5 Nebulalar ölçek bakımından ve onların yıldızsal ve gezegensel doğumlarının bir araya getirdiği kütleler ve bunun sonucundaki sayısal topluluk bakımından fazlasıyla çeşitlilik gösterir. Orvonton’un hemen kuzey sınırında bulunan fakat aşkın evren mekân düzeyinin içinde kalan bir güneş\hyp{}meydana getiren nebula, yaklaşık olarak kırk bin güneşin oluşmasına çoktan kaynaklık eder. Buna ek olarak bu ana burgaç güneşleri dışarı doğru fırlatarak oluşturmaya devam eder, ve bu üretimin sonucunda ortaya çıkan güneşlerin büyük bir çoğunluğu sizin güneşinizden kat be kat daha büyüktür. Dış uzayın daha büyük olan bir takım nebulaları, yüz milyon kadar güneş oluşumuna kaynaklık etmektedir.
\vs p015 4:6 Her ne kadar bazı yerel evrenler tek bir nebulanın üretimiyle düzenlenmiş olsa da, nebulalar doğrudan bir biçimde azınlık birimleri veya yerel evrenler olan herhangi bir idari birimleriyle alakalı değildir. Her yerel evren, nebulasal ilişkiden bağımsız olarak bir aşkın evrenin bütünsel enerji etkisinin tamı tamına on\hyp{}bin parçasından biriyle bütünleşir; çünkü enerji nebulalar tarafından düzenlenmemekte olup o tamamiyle Kâinatsal olarak dağıtılmaktadır.
\vs p015 4:7 Sarmal nebulaların hepsi güneş yapımına katılmaz. Bazıları onların bölünmüş yıldızsal doğumlarının bir çoğunundaki denetimlerini elinde bulundurur; buna ek olarak onların sarmal görünüşü, güneşlerinin nebula kolundan benzer bütünlük içerisinde dışarı çıkması fakat farklı doğrultularda geri dönmesi sebebiyle belirlenmiştir. Bu sebeple onların bu durumu bir noktadan onların gözlenebilmesini kolay hale getirmektedir, fakat nebulasal kolundan dışarı ve onun uzağından gerçekleşen dönüş doğrultuları üzerinde güneşler geniş bir biçimde dağıldıklarında onları görmek daha zor bir hale gelir. Mevcut an içerisinde Orvonton üzerinde etkin olan fazla sayıda bir güneş\hyp{}oluşturan nebula bulunmamaktadır, fakat yerleşik aşkın evrenin dışında olan Andromeda nebulaları bu bağlamda fazlasıyla etkindir. Bu çok uzakta bulunan nebula çıplak göz için görülebilir bir niteliktedir; ve siz onu gördüğünüzde, güzünüze yansıyan ışığın uzak güneşlerden bir milyar yıl önce salındığını bir durup düşünün.
\vs p015 4:8 Samanyolu galaksisi, geçmiş sarmal ve diğer nebulaların çok geniş sayıdaki birlikteliklerinden meydana gelmiştir; ve onların birçoğu hala, kökensel oluşumlarını kendi içlerinde barındırmaya devam etmektedir. Fakat içsel felaketler ve dışsal çekim sonucunda onların birçoğu, Macellansal Bulut’a benzer yakıcı güneşlerin devasa derecedeki aydınlık kütleleri olarak bu çok büyük kümelenmelerin sebebini oluşturan bu tür bozulmalara ve yeniden düzenlenmelere maruz kalmıştır. Küresel biçimdeki yıldız kümelenmeleri, Orvonton’un dışsal uçlarının yakınında baskın bir konumdadır.
\vs p015 4:9 Orvonton’un çok geniş yıldız bulutları maddenin bireysel bir araya gelişleri olarak değerlendirilebilir, ve bu durum Samanyolu galaksisine dışsal olan mekân bölgelerinde gözlemlenen ayrık nebulalarla karşılaştırılabilir bir niteliktedir. Mekânın yıldız bulutları olarak adlandırılan unsurların birçoğu, aslında sadece gazsal maddeden oluşmuştur. Bu yıldızsal gaz bulutlarının enerji potansiyeli çok şaşırtıcı bir biçimde devasadır, buna ek olarak onlardan bazıları her yakın olan güneşten diğerlerine doğru kabul edilip, uzay içerisinde güneş salınımı olarak tekrar çevrime katılır.
\usection{5.\bibnobreakspace Mekân Bedenlerinin Kökeni}
\vs p015 5:1 Kütle yığını, nebulasal burgaçlardan kaynağını alan bir aşkın evrenin güneşi ve gezegenleri içinde taşınır. Maddenin sürekli farklılık gösteren miktarı açık uzay içinde oluşsa da; aşkın evren kütlesinin çok azı, mimari alanları inşasında olduğu gibi güç yöneticilerinin doğrusal eylemi tarafından düzenlenir.
\vs p015 5:2 Kaynak bakımından güneşlerin, gezegenlerin ve diğer alanların çok büyük bir kısmı şu on topluluklardan biri olarak tanımlanabilir:
\vs p015 5:3 1.\bibemph{ Eş Merkezli Daralım Halkaları}. Her nebula sarmal bir halde değildir. Engin bir nebulanın büyük bir kısmı, çifte bir yıldız sisteminin içine doğru dağılması veya sarmal olarak evrimleşmesi yerine, çoklu\hyp{}halka dizilişi tarafından yoğunlaşmaya uğrar. Uzun süreçler boyunca bu tür bir nebula, maddenin halka şeklinde görünen dizilişi biçimindeki sayısız miktardaki çevreleyici devasa bulutlar tarafından sarılan çok büyük bir güneş olarak görünür.
\vs p015 5:4 2.\bibnobreakspace \bibemph{Sarmal Biçiminde Dönen Yıldızlar}, çok yüksek ısılarda ısıtılmış gazların muazzam ana burgaçları tarafından oluşturulan bu güneşlerle bütünleşir. Onlar halkalar şeklinde dışarıya salınmamaktadır, fakat bunun yerine onlar sağlı\hyp{}sollu olan seyirleri biçiminde oluşturulmuşlardır.
\vs p015 5:5 3.\bibnobreakspace \bibemph{Çekim\hyp{}Patlama Gezegenleri}. Bir güneş bir sarmaldan veya bir sarmal nebulasından doğduğunda, nadir olmayan bir biçimde o çok büyük olan bir uzaklığa salınır. Böyle bir güneş yüksek bir oranda gazsal olup, bu hali takiben o bir biçimde soğuyunca veya yoğunlaşınca, uzayın karanlık bir adası veya devasa bir güneş biçimindeki maddenin çok büyük ölçekteki bir takım kütlesinin yakınında dönme şansına sahip olabilir. Böyle bir yaklaşım çarpışmaya sebebiyet verecek kadar yakınlıkta olmayabilir; fakat bu yaklaşım aynı zamanda, daha büyük olan bünyenin çekim etkisinin daha küçük olan parçasında gelgitsel kasılmaların oluşturmaya başlamasına izin verecek kadar yakındır. Bu nedenle sürecin kendisi, kasılmış güneşin karşıt tarafları üzerinde eş zamanlı olarak ortaya çıkan bir gelgitsel ani değişimi harekete geçirmiş olur. Onların en üst noktasında patlamanın bu açığa çıkışı, püsküren güneşin çekimin\hyp{}yeniden kazanımı bölgesinin ötesinde yönlendirilebilecek maddenin değişen boyutlarının bir araya gelmesinin bir dizisini üretir. Bu durumun sonucunda bahse konu bölüm ile alakalı, iki bedenden birinin etrafında kendisine ait yörüngelerde sabit bir hale gelir. Bu süreçlerden daha sonra, maddenin daha büyük olan toplulukları bir bütün biçiminde birleşip, daha küçük olan bedenleri gittikçe kendilerine doğru çekerler. Böylece daha küçük olan sistemlerin cisimsel gezegenlerinden birçoğu mevcudiyete kavuşturulur. Kendi güneş sisteminiz tıpkı böyle bir kökene sahiptir.
\vs p015 5:6 4.\bibnobreakspace \bibemph{Merkezden Uzaklaşan Gezegensel Kız Evlatlar}. Devasa güneşler büyümenin belirli aşamalarında, ve onların döngüsel hızları fazlasıyla artış gösterdiğinde, birliktelik halinde bulunduğu güneşi çevrelemeye devam eden küçük dünyaları oluşturmak için daha sonradan bir araya getirilebilecek maddenin geniş niceliklerinin salınmasına başlarlar.
\vs p015 5:7 5.\bibnobreakspace \bibemph{Yetersiz Çekim Alanları}. Bireysel yıldızların büyüklüğü için önemli bir sınır bulunmaktadır. Bir güneş bu sınıra ulaşınca ve çevrimsel hızında her hangi bir yavaşlama olmazsa, mecbur olarak bu yapı ayrışmaya uğrayacaktır. Güneşin bölünmesi bu aşamada meydana gelip yeni bir çifte yıldızın bu çeşitliliği hayata geçer. Sayısız küçük gezegen bu oluşumun sonrasında bahse konu devasa parçalanmanın bir yan ürünü olarak oluşabilir.
\vs p015 5:8 6.\bibnobreakspace \bibemph{Kasılım Yıldızları}. Daha küçük olan sistemlerde en geniş dışsal gezegen bazen kendisine komşu olan dünyaları, güneş yakınında bu dünyalar kendilerinin oluşumsal atılımlarına başlarlarken, kendisine doğru çeker. Sizin güneş sisteminiz bakımından böyle bir son, dört içsel gezegenin güneş tarafından çekilmesi anlamına gelebilir. Böyle bir durum karşısında ise, güneş sisteminin büyük bir gezegeni olan Jüpiter, dört gezegenden arta kalan dünyaları içine alarak fazlasıyla genişlemiş olacaktır. Bir güneş sisteminin böyle bir sonu, çifte yıldız oluşumunun bir çeşidi biçimindeki bitişik olan fakat eşit olmayan iki güneşin çoğalımıyla sonuçlanacaktır. Bu tür yıkımsal olaylar, aşkın evren yıldızsal bütünleşmesinin sınırları üzerinin dış bölgesi haricinde nadiren gerçekleşen bir niteliğe sahiptir.
\vs p015 5:9 7.\bibnobreakspace \bibemph{Toplu Alanlar}. Mekânda çevrim içindeki maddenin çok büyük miktarından küçük gezegenler yavaş bir biçimde toplu olarak birikebilir. Onlar çok hızlı bir biçimde birikerek ve küçük çaplı çarpışmalar sonucunda büyüyebilir. Uzayın belirli birimlerinde, var olan şartlar gezegensel doğuşun bu biçimlerini destekler. Birçok yerleşik halde bulunan bir dünya böyle bir kökene sahip olmuştur.
\vs p015 5:10 Bazı yoğun karanlık adalar, uzay içinde enerji dönüşümünün birikimlerinin doğrudan sonucu olarak karşımıza çıkar. Bu karanlık adaların bir diğer topluluğu; uzay içinde döngü halinde olan, yalnızca parçacıklardan ve göktaşlarından oluşan soğuk maddenin devasa ölçekteki miktarlarının bir araya gelmesiyle var olmuştur. Maddenin bu tür bütünleşmesi hiçbir zaman sıcak bir içeriğe sahip olmamıştır, buna ek olarak yoğunluk olarak farklılık göstermesi haricinde bu bütünleşme oluşum bakımından Urantia’ya oldukça benzerlik gösterir.
\vs p015 5:11 8.\bibnobreakspace \bibemph{Sönmüş Güneşler}. Uzayın bazı karanlık adaları sönmüş olan yalıtılmış güneşler olup onların sahip olduğu mevcut olan tüm mekân\hyp{}enerjisi harcanmıştır. Maddenin düzenlenen birimleri varsayılan bütüncül yoğuşma olan tam yoğuşuma yaklaşır; buna ek olarak, uzayın döngüleri içinde bir hayli yoğuşma içinde olan maddenin bu tür devasa kütlelerinin yeniden yüklenmesi için süre bakımından birçok çağın geçmesi gerekir. Böylece bu süreç sonunda, bir çarpışmanın veya aynı dirilten etkiye sahip bazı Kâinatsal oluşumlar sonucunda Kâinatsal faaliyetin yeni döngüleri için onlar hazırlanmış olurlar.
\vs p015 5:12 9.\bibnobreakspace \bibemph{Çarpışma Alanları}. Daha kalın olan kümelenmenin bu bölümlerinde çarpışmalar sıra dışı bir nitelik taşımaz. Böyle bir gökbilimsel yeniden düzenleme, madde dönüşümleri ve muazzam enerji değişimleri tarafından eşlik edilir. Sönmüş güneşlerle olan çarpışmalar, geniş enerji dalgalanmalarının yaratılmasında özel bir etkiye sahiptir. Çarpışmanın sonucunda ortaya çıkan enkaz, fani yerleşimine uyumlu hale getirilen gezegensel bedenlerin daha sonraki oluşumu için zaman zaman maddi çekirdekleri oluşturur.
\vs p015 5:13 10.\bibnobreakspace \bibemph{Mimari Dünyalar}. Bazı özel amaçlar için tasarılar ve belirtilen özellikler uyarınca inşa edilen dünyalar bulunmaktadır. Yerel evreninizin yönetim merkezi olan Salvington ve bizim içinde bulunduğumuz aşkın evreninin idari mevkisi olan Uversa bunlardan biridir.
\vs p015 5:14 Evrim halindeki güneşler ve ayrım içinde bulunan gezegenler için sayısız derecek çok olan diğer yöntemler bulunmaktadır; fakat sözü geçen usullerin sunduğu yöntemler vasıtasıyla, yıldız sistemlerinin ve gezegensel ailelerin bu çok büyük çoğunluğu mevcut hale getirilir. Yıldızsal dönüşüme ve gezegensel evirilişe katılan birçok işleyiş biçiminin tümünü tanımlamak amacıyla böyle bir girişimi üstlenmek, gezegensel kökenin ve güneş oluşumunun yaklaşık yüz farklı biçiminin anlatımına ihtiyaç duymaktadır. Sizin yıldız öğrencileriniz cennetleri incelediği zaman, onlar yıldız evriminin bahse konu bütün biçimlerinin belirtilerinden oluşan olgular bütününü gözlemleyecektir; fakat onlar çok seyrek olarak, geniş maddi yaratımların en önemli kısmı olan yerleşik gezegenlerde hizmet halindeki, maddenin aydınlık olmayan bir araya gelişlerinden meydana gelen bu küçük oluşumların kanıtını tespit edeceklerdir.
\usection{6.\bibnobreakspace Uzay Mekânları}
\vs p015 6:1 Kökeninden bağımsız olarak uzayın birçok alanı aşağıda bahsi geçen şu büyük farklılaşmalar altında sınıflandırılabilir:
\vs p015 6:2 1.\bibnobreakspace Uzayın yıldızları olarak güneşler.
\vs p015 6:3 2.\bibnobreakspace Uzayın karanlık adaları.
\vs p015 6:4 3.\bibnobreakspace Kuyruklu yıldızlar, göktaşları ve gezegenimsi alanlar olarak ölçek bakımından küçük uzay bedenleri.
\vs p015 6:5 4.\bibnobreakspace Yerleşik dünyaları içine alan gezegenler.
\vs p015 6:6 5.\bibnobreakspace Düzenleme amacıyla yaratılan dünyalar olarak mimari alanlar.
\vs p015 6:7 Mimari alanların istisnai durumuyla birlikte tüm bu uzay bedenleri evrimsel bir kökene sahiptir. İlahiyat’ın iradesi tarafından mevcudiyet haline getirilmemiş olması bakımından, buna ek olarak İlahiyat’ın yaratılan ve var edilen akli yapılarının birçoğunun işleyişi boyunca bir zaman\hyp{}mekân faaliyet biçimi tarafından Tanrı’nın yaratıcı eylemlerinin kendisini açığa çıkarışı bakımından evrimsel bir niteliğe sahiptir.
\vs p015 6:8 \bibemph{Güneşler}. Bu unsurlar, onların mevcudiyetinin çok çeşitli düzeylerinin tümü bakımından uzayın yıldızlarıdır. Onlardan bazıları kendi başına evrim içinde olan uzay sistemleridir; bunların dışında kalan diğerleri ise, kasılan veya genişleyen gezegensel sistemleri biçimindeki çifte yıldızlardır. Uzayın yıldızları, sayıca binden az olmayan farklı durumlarda ve düzeylerde mevcut bir halde bulunmaktadır. Siz güneşlerin ısı yardımıyla ışık yayması durumuna aşina bir durumdasınız; fakat ısı olmadan sadece ışık yayan güneşlerde bulunmaktadır.
\vs p015 6:9 Olağan bir güneş trilyonlarca yıl boyunca, maddenin her biriminin taşındığı enerjinin çok geniş muhafaza alanlarını çok iyi bir biçimde temsil eden ışı ve ışık yayışına devam edecektir. Fiziksel maddenin bu görünmeyen parçacıklarında muhafaza edilen mevcut enerji neredeyse hayal edilemeyecek kadar büyüktür. Buna ek olarak, yakıcı güneşlerlerin içinde hüküm süren ilgili enerji faaliyetlerine ve devasa ısı basıncına maruz kaldığı zaman bu enerji neredeyse ışık kadar erişilebilir hale gelir. Bunun yanı sıra, oluşturulan uzay döngülerinde kendi yörüngelerine oturan uzay enerjilerinin büyük bir çoğunluğunun dönüşümü ve yayılımı için diğer koşullar da güneşleri onların bahse konu bu faaliyetlerinde etkin hale getirir. Fiziksel enerjinin birçok fazı ve maddenin tüm biçimleri güneş dinamolarına bağlanmış bir konumda olup, onlar bu mekanizmalar tarafından dağıtılır. Böylelikle güneşler, kendiliğinden faaliyet gösteren güç\hyp{}denetim istasyonları biçiminde hareket eden enerji çevriminin yerel hareketlendiricileri olarak görev yaparlar.
\vs p015 6:10 Orvonton’un aşkın evreni, sayıca on trilyondan daha fazla bir miktardan oluşan yakıcı güneşler tarafından aydınlanır ve ısıtılır. Bu güneşler, sizin gözlenebilir gökbilimsel sisteminizin yıldızlarıdır. Onların arasında iki milyondan daha fazla sayıda olan güneş, Urantia’dan hiçbir zaman gözlenemeyecek kadar uzaklıkta ve küçüklüktedir. Fakat üstün evrende, dünyanızın içinde var olan okyanusların taşıdığı bardaklarca su kadar çok güneş bulunmaktadır.
\vs p015 6:11 \bibemph{Uzayın Karanlık Adaları}. Bu unsurlar, ışıktan ve ısıdan mahrum olan, maddenin ölü güneşlerinden ve diğer birlikteliklerinden oluşan yapılardır. Bahse konu karanlık adalar kütle bakımından bazı durumlarda devasa boyutlarda karşımıza çıkar, ve onlar evren dengesinde ve enerji yönetiminde çok güçlü bir etki bırakırlar. Bu geniş kütlelerin bazılarının yoğunluğu neredeyse hayal dahi edilemeyecek bir ölçeğe dayanır. Buna ek olarak kütlenin bu devasa yoğunluğu, etkin bağlayıcılıkta geniş komşu sistemleri bir arada tutan çok kuvvetli denge burgaçları olarak faaliyet göstermesi amacıyla bu karanlık adaları etkin hale getirir. Onlar birçok takımyıldızında gücün çekim dengesini bir arada tutar; bahse konu koruyucu karanlık adalarının çekim kavrayışı tarafından güvenli bir biçimde bir arada tutulan birçok fiziksel sistemler, bu düzenin yoksunluğu halinde yakın güneşlere hızla bir biçimde yönelerek çok büyük bir zarara sebep olurdu. Bu işleyiş nedeniyle, biz onların yerini bir arada olarak doğru bir biçimde tespit etmekteyiz. Biz ışık saçan bu bünyelerin çekim etkisini ölçmüş bir durumda bulunmaktayız, bu nedenle herhangi bir sistemde sabit bir biçimde birlikteliğin muhafazası şeklinde çok etkin olarak faaliyet gösteren uzayın karanlık adalarının ölçüsel olarak tam bir boyutunu ve konumunu hesaplayabiliriz.
\vs p015 6:12 \bibemph{Küçük Uzay Bedenleri}. Uzay içinde evrimleşen ve çevrim halinde bulunan göktaşları ve maddenin diğer parçacıkları, enerjinin ve maddesel özün çok devasa bir birlikteliğini meydana getirirler.
\vs p015 6:13 Birçok kuyruklu yıldız, merkezi yönetici güneş tarafından kademeli olarak denetim altına alınan ana güneş burgaçlarının irade dâhilinde oluşturulmamış yabani doğumlarıdır. Buna ek olarak kuyruklu yıldızlar sayısız derecede farklı kökenlere sahiplerdir. Bir kuyruklu yıldızın kuyruğu, güneşten veya bağlı bulunduğu bedeninden uzağa doğru yönlenir; bunun nedeni, onun sahip olduğu genleşme bakımından bir hayli yüksek gazların elektriksel tepkimesine ek olarak güneşten kaynaklığını alan diğer enerjilerin ve ışığın mevcut basıncıdır. Bu olgular bütünü, ışığın gerçekliğinin olumlu kanıtlarından birini ve onunla ilişkili enerjilerini meydana getirir; bahse konu bu olgular aynı zamanda ışığın da fiziksel bir ağırlığı olduğunu ortaya koyar. Bu bakımdan ışık gerçek bir öz olup, basit bir biçimde varsayımlara dayanan havanın bir bölümündeki dalgalardan ibaret değildir.
\vs p015 6:14 \bibemph{Gezegenler}. Bu yapılar bir güneş veya diğer uzay bedeni etrafında bir yörüngeyi takip eden maddenin daha geniş topluluklarıdır; onlar büyüklük bakımından, gezegenimsi yapılardan devasa gazlara, sıvılara veya katı alanlara varan geniş bir çerçevede mevcut bir halde bulunabilirler. Yüzen uzay maddesinin birlikteliği tarafından inşa edilmiş katı dünyalar, yakın bir güneş ile elverişli bir ilişki içerisinde bulundukları zaman, akli yapılara sahip sakinlerin sığınması için daha uygun gezegenler haline gelirler. Bir prensip olarak ölü güneşler yaşam için elverişli değillerdir; onlar genellikle, yaşamı etrafında barındıran yakıcı bir güneşten çok daha uzakta bulunur, ve buna ek olarak onlar bütünsel bir biçimde kütlesel bakımından çok büyük olup, yüzeylerinde bulunan çekim elverişli olmayan bir şekilde devasa boyuttadır.
\vs p015 6:15 Aşkın evreninizde kırk gezegen arasında sadece tek bir ılıman gezegen düzeyinizde bulunan varlıklar için yerleşime elverişli değildir. Buna ek olarak çok aşırı derece sıcak olan güneşler ve uzak olan dondurucu derecedeki soğuk dünyalar daha üstün yaşamın sığınması için tabii ki uygun değildir. Güneş sisteminizde mevcut haliyle sadece üç gezegen yaşam için elverişlidir. Mevcut haliyle Urantia; büyüklük, yoğunluk ve konum itibariyle insan yerleşkesi için birçok bakımdan en uygunudur.
\vs p015 6:16 Fiziksel\hyp{}enerji tutumunun kanunları temel olarak evrenseldir; fakat yerel etkiler, daha çok bireysel gezegenler üzerindeki ve yerel sistemler içinde etkin olan fiziksel şartlar ile alakalıdır. Yaratılmış yaşamın neredeyse sonsuz olan bir çeşitliliği ve diğer yaşam dışavurumları, uzayın sayısız dünyalarını tanımlar. Fakat ussal yaşamın ayrıca Kâinatsal bir işleyiş biçimi olmasına rağmen, herhangi bir sistem içerisinde ilişki içerisinde bulunan dünyaların bir topluluğu içinde benzerliğin belirli ortak noktaları bulunmaktadır. Aynı fiziksel döngüye ait olan bu gezegensel sistemler arasında fiziksel ilişkiler bulunmaktadır; buna ek olarak onlar, âlemlerin çevresi etrafında sayısız olan dönüşlerinde birbirlerini yakın olarak takip ederler.
\usection{7.\bibnobreakspace Mimari Alanlar}
\vs p015 7:1 Her aşkın evren hükümeti onların uzay bölümünün evrimsel âlemlerinin merkezi yakında yönetimini gerçekleştirirken; aynı zamanda bu oluşum, idare etmek için yaratılan bir dünyayı elinde bulundurup, bu dünya liyakat sahibi kişilikler tarafından toplanır. Bu merkezi yönetim alanları, onların belirli yaratılma amacı için özellikle yaratılmış mekân bedenlerinin mimari alanlarıdır. Yakın güneşlerin ışığını paylaşırken bu alanlar, bağımsız olarak ısı ve ışık alırlar. Her biri tıpkı Cennet’in uyduları gibi ısı olmadan ışık veren bir güneşe sahip olurken, aynı zamanda alanlarının yüzeyine yakın olan belirli enerji akımlarının çevrimi tarafından ısıyla beslenirler. Bu yönetim merkezleri dünyaları, onların ilişkili oldukları aşkın evrenlerinin gök bilimsel merkezinin yakınında konumlanmış daha büyük sistemlerden birine aittir.
\vs p015 7:2 Aşkın evrenlerin yönetim merkezleri üzerinde zaman, her biçimde aynı olarak tespit edilebilecek bir ölçüm haline getirilmiştir. Orvonton’un aşkın evreninde bir ortak gün, Urantia zamanına göre neredeyse otuz güne karşılık gelmektedir; buna ek olarak bir Orvonton yılı, yüz kadar Urantia gününe eşittir. Bu Uversa yılı, yedinci aşkın evrende tek bir ölçüm süresi olarak ortaktır, ve bu sayısal değer sizin yıllarınızın yaklaşık olarak sekiz nokta iki katı olarak Urantia zamanının üç bin gününden yirmi iki dakika eksiktir.
\vs p015 7:3 Yedi aşkın evrenin merkezi yönetim dünyaları, kusursuzluğun merkezi işleyişsel yöntemi bakımından Cennet’in doğasını ve ihtişamını alır. Gerçekte tüm merkezi yönetim dünyaları cennetseldir. Onlar tam anlamıyla cennetsel biçimdeki yerleşim yerleri olup, Jerusem’den merkezi Ada’ya gidildikçe maddesel büyüklük, morontial güzellik ve ruhaniyet ihtişamı bakımından artış gösterirler. Buna ek olarak, bu yönetim merkezleri dünyalarının tüm uyduları aynı zamanda mimari alanlardır.
\vs p015 7:4 Yönetim merkezlerinin birçok çeşidi, maddi ve ruhsal yaratımın her aşamasıyla donatılmıştır. Maddi, morontial ve ruhsal varlıkların tüm türleri, âlemlerin bu buluşma yerlerini kendi evleri olarak hisseder. Fani yaratılmışlar, maddi ve ruhsal alanlardan geçerek bu âleme yükseldiklerinde; kendi mevcudiyet düzeylerinin önceki âlemlere ilişkin takdirini ve hoşnutiyetini hiçbir zaman yitirmezler.
\vs p015 7:5 Satania’ya ait olan yerel sisteminizin yönetim merkezi \bibemph{Jerusem}, kendisine ait geçiş kültürünün yedi dünyasına sahiptir; bu dünyalardan her biri yedi uydu tarafından çevrilmiş olup, onların arasında insanın fani yaşam sonrası ilk yerleşkesi olan morontia gözetim alanının yedi malikâne dünyaları bulunmaktadır. Urantia üzerinde cennet kavramı kullanılırken, bu kavramsallaşma zaman zaman bu yedi malikâne dünyaları anlamına gelmektedir; bu dünyalar arasında ilk malikâne dünyası birincil cennet olarak adlandırılırken, aynı sıralandırma içerisinde yedinci dünya ise yedinci cennete karşılık gelir.
\vs p015 7:6 Norlatiadek’e ait olan sizin takımyıldızınızın yönetim merkezi \bibemph{Edentia}, sosyalleştiren kültüre ve öğretime kaynaklık eden yetmiş uyduya sahiptir. Yükseliş içinde olanlar bu uyduların üzerinde; kişiliğin taşınmasının, birliğinin ve onun gerçekleştirilmesinin düzeni olan Jerusem’in tamamlanışı ardından kısa bir süreliğine ikamet ederler.
\vs p015 7:7 Yerel evreniniz Nebadon’un başkenti \bibemph{Salvington}, her biri kırk dokuz alandan oluşan on üniversite topluluğu tarafından çevrelenmiştir. Yıldız takımının sosyalleştirme sürecinin ardından, bahse konu bu yerleşkelerin üzerinde insan ruhanileştirilme sürecine girer.
\vs p015 7:8 İçinde bulunduğunuz azınlık birimi olan Ensa’nın yönetim merkezi \bibemph{Uversa azınlık üçüncü birimi}, yükselim yaşamının daha yüksek olan fiziksel çalışmalarının yedi alanı tarafından çevrelenmiştir.
\vs p015 7:9 Sizin çoğunluk biriminiz olan Splandon’un yönetim merkezi \bibemph{Uversa çoğunluk beşinci birimi}, aşkın evrenin ilerleyici ussal eğitiminin yetmiş alanı tarafından çevrelenmiştir.
\vs p015 7:10 Sizin aşkın evreniniz olan Orvonton’un yönetim merkezi \bibemph{Uversa}, yükseliş içerisinde olan irade sahibi tüm yaratılmışlar için, ilerici ruhsal eğitiminin daha üst düzeyde bulunan yedi üniversitesi tarafından eş zamanlı olarak çevrelenmiştir. Bahse konu başyapıt alanlarının bu yedi topluluğunun her biri; zamanın kutsal yolcuların Havona’ya olan uzun yolculuğuna hazırlık için üzerinde yeniden eğitildiği ve değerlendirildiği, kâinat eğitimi ve ruhsal kültüre adanan dopdolu kurumların ve düzenlenmelerin binlercesini taşıyan sayıca yetmiş tane olan özelleşmiş dünyadan meydana gelir. Zamanın kutsal yolcularının varışları her zaman bu birliktelik içerisinde bulunan dünyalar üzerinde karşılanır, fakat buralardan Havona için ayrılacak olan mezunlar her koşulda Uversa’nın sınırlarından doğrudan bir biçimde yönlendirilir.
\vs p015 7:11 Uversa, sayıca yaklaşık olarak bir trilyonu bulan yerleşik veya yerleşime açık olan dünyalar için ruhsal ve idari yönetim merkezi konumundadır. Orvonton’un başkentinin ihtişamı, görkemi ve kusursuzluğu zaman\hyp{}mekân yaratılmışlarının herhangi bir başyapıtını aşan derecede üstün olan bir niteliğe sahiptir.
\vs p015 7:12 Eğer tasarlanan yerel evrenler ve onları bir araya getiren kısımların tümü oluşturulmuş olsalardı, yedi aşkın evren üzerinde beş yüz milyar mimari dünyadan sayıca biraz daha az olan yapıtlar karşımıza çıkacaktı.
\usection{8.\bibnobreakspace Enerji Denetimi ve Düzenlenmesi}
\vs p015 8:1 Aşkın evrenlere ait olan alanların yönetim merkezleri öyle bir biçimde inşa edilmişlerdir ki; onlar, yerel evren oluşum unsurlarına enerjinin yönlendirilmesi amacıyla merkezi noktalar olarak hizmet eden çok çeşitli birimleri için etkin güç\hyp{}enerji düzenleyicileri olarak faaliyet göstermeye yetkin hale gelmişlerdir. Onlar, düzenlenmiş mekân boyunca döngü içerisinde bulunan fiziksel enerjilerin denetimi ve dengesi üzerine çok güçlü bir etkide bulunurlar.
\vs p015 8:2 Bu ifade edilen niyet amacıyla oluşturulan tam zamanlı veya yarı zamanlı yaşayan ussal unsurlar biçimindeki aşkın evren güç merkezleri ve fiziksel denetleyicileri tarafından, daha ileri düzeydeki düzenleyici faaliyetler uygulanır. Bu kuvvet merkezleri ve denetleyicileri anlaşılması zor olan bir niteliği içerisinde barındırır; onların düşük olan düzeyleri öz iradeleri dâhilinde hareket etmez, bu nedenle onlar ne bir iradeye sahiptirler ne de tercih etme konumunda bulunurlar. Onların faaliyetleri son derece ussaldır; fakat bu nitelik açık bir biçimde istemsiz olarak, onların çok yüksek derece özelleşen düzenlenmesinin doğasından kaynaklanır. Aşkın evrenin kuvvet merkezleri ve fiziksel denetleyicileri, yerçekimi nüfuz alanını oluşturan otuz enerji sisteminin kısmi denetimini ve yönlendirmesini yerine getirir. Uversa’nın kuvvet merkezleri tarafından yönetilen fiziksel\hyp{}enerji döngüleri, aşkın evrenin dolanımını tamamlaması için 968 milyondan sayıca biraz daha fazla olan yıla ihtiyaç duyar.
\vs p015 8:3 Evrimleşen enerji bir öze sahiptir; onun fiziksel ağırlığı olup, bu ağırlık her zaman, döngüsel hıza, kütleye ve karşı\hyp{}çekime bağlı olarak, göreceli bir niteliğe sahiptir. Madde içerisindeki kütle, enerji içindeki hızı yavaşlatma eğilimine sahiptir. Buna ek olarak enerjinin herhangi bir yerinde mevcut olan hız; çok yüksek miktarda ısıtılan veya baskın olarak etkilenen bünyelerin yakınındaki fiziksel etkinin ve aşkın evrenin yaşayan enerji denetleyicilerinin düzenleyici faaliyetinin karşısında, taşınma esnasında karşılaşılan kütle tarafından yavaşlama sonucunda açığa çıkan hızın öncül edinimini temsil eder.
\vs p015 8:4 Madde ve enerji arasındaki dengenin sağlanması için Kâinatsal tasarı, daha küçük düzeydeki maddi birimlerin sonsuza kadar sürecek olan bozulması ve yeniden yapılmasını gerektirir. Kâinat Güç Yöneticileri enerjinin değişen miktarları olarak yoğunlaştırmanın ve muhafazanın veya genişlemenin ve salınımın yetisine sahiptir.
\vs p015 8:5 Yavaşlayan etkinin herhangi yeterli bir süreci içerisinde, çekim nihai olarak tüm enerjiyi madde biçimine iki unsurun mevcudiyeti sebebiyle dönüştürmez. Bu unsurlardan birincisi enerji denetleyicilerinin karşı\hyp{}çekim etkileri olup; ikinci olarak ise düzenlenen maddenin, çok yüksek miktarda enerji kazandırılmış yoğunlaşan maddenin soğuk bedenleri yakınında belirli ayrıcalıklı koşullar altında ve çok fazla sıcak olan yıldızlarda bulunan belirlenmiş şartlar altında çözülme eğilimi göstermesidir.
\vs p015 8:6 Kütle, aşırı bir biçimde bütünleşmesinin sonucunda büyüdüğünde ve enerjinin dengesini tehdit eder hale geldiğinde; çekimin enerjiyi haddinden fazla bir biçimde maddileştirmesinin ileriki aşamalarda ortaya çıkan kendi eğilimi mekânın ölü olan devasa bedenleri arasındaki bir çarpışmanın ortaya çıkması tarafından bertaraf edilmediği takdirde, uzay fiziksel güç döngülerini boşaltmak için fiziksel denetleyiciler çekimin eklemsel bir araya gelişlerini bu nedenle anlık bir tepkime biçimiyle tamamiyle yok ederek müdahale eder. Bu çarpışmasal olaylarda maddenin devasa kütleleri aniden enerjinin en az rastlanan biçimlerine çevrilir, buna ek olarak Kâinatsal denge için verilen mücadele böylece yeniden başlamış olur. Nihai olarak daha büyük fiziksel sistemler fiziksel sabitlenme biçiminde dengesel konuma gelir; ve onlar, aşkın evrenlerin oluşturulan ve dengelenen döngülerine doğru olan dönüş seyirlerine yerleştirilirler. Bu oluşumun sonrasında bu tür inşa edilmiş sistemlerde artık hiçbir çarpışma veya diğer yıkımsal afet gerçekleşmez.
\vs p015 8:7 Artan enerjinin zamanları boyunca, elektriksel dışavurumlar tarafından etkileşimsel olarak açığa çıkan güçsel bozukluklar ve ısısal dalgalanmalar mevcuttur. Azalan enerjinin zamanlarında ise; çevrim içinde olan enerjinin ve aslına daha uygun olarak sabitlenen maddenin arasındaki dengenin hızla yerine getirilmesini sağlayan gelgitsel veya çarpışmasal düzenlenmelerin sonucuyla birlikte, daha hassas bir biçimde dengeye oturtturulmuş döngülerde maddenin bütünleşmesi, yoğunlaşması ve denetimden çıkması için artış gösteren eğilimler mevcuttur. Uzayın karanlık adalarının ve yakıcı güneşlerinin buna benzer bir davranışını tahmin etmek ve ayrıca onu anlamak göksel yıldız gözlemleyicilerin görevlerinden biridir.
\vs p015 8:8 Kâinatsal dengeyi idare eden yasaların birçoğunu tanımaya ve evren dengesiyle alakalı birçok unsuru öngörmeye yetkin bir durumdayız. Gerçeğe uygulanabilir olarak bizim tahminlerimiz güvenilirdir, fakat biz, tarafımızca bilinen maddenin davranışı ve enerjinin denetimi yasalarına bütünüyle tabi olmayan belirli kuvvetler tarafından karşılaşmaktayız. Tüm fiziksel olgular bütününün tahmin edilebilirliği, Cennet’in âlemlerinden dışa doğru olarak ilerlediğimizde artan bir biçimde zor bir hale gelir. Cennet Yöneticiler’in kişisel idaresinin sınırları dışına geçtiğimizde ise, oluşturulan ortak ölçütlerle ve yanı başında bulunan gökbilimsel sistemlerin fiziksel olgular bütünüyle ayrıcalıklı gözlemlerimizle iniltili bir biçimde kazanılan deneyime bağlı olarak ayırt etmede yaşanan artan acziyetimizle karşılaşmaktayız. Yedi aşkın evrenin alanlarında bile biz, dışsal uzayın tüm bölgelerinin üzeri boyunca bütünleşen denge içinde, nüfuz alanlarımızın tümü üzerine yayılan ve genişleyen enerji tepkimelerinin ve kuvvet eylemlerinin ortasında yaşamaktayız.
\vs p015 8:9 Bulunduğumuz bölgeden daha dışarı doğru hareket ettiğimizde, deneyimsel İlahiyatlar’ın ve Mutlaklıklar’ın kavranılamaz mevcudiyet\hyp{}uygulamalarının hataya hiçbir biçimde yer bırakmayan niteliği biçimindeki bu tahmin edilemez ve çeşitlilik arz eden olgular bütünüyle daha kesin bir biçimde karşılaşmaktayız. Buna ek olarak bu olgular bütünü, kuvvetle muhtemel bir biçimde her şeye ait olan bazı Kâinatsal üst denetimlerin belirleyicisi olmalıdır.
\vs p015 8:10 Şu an itibariyle Orvonton’un aşkın evreni açıkça gözlenen bir biçimde çöküş eğilimindedir; dışsal evrenler ise benzeri olmayan gelecek faaliyetleri için oluşum ve toparlanma aşamasında olduğu gözlenmektedir; nihai olarak ise merkezi Havona evreni ebedi bir biçimde sabitleştirilmiştir. Çekim ve soğukluk biçimindeki ısının yoksunluğu maddeyi bir arada tutup onu düzenler; ısı ve karşı\hyp{}çekim maddede bozulmaya sebep olup onun enerjisini dağıtır. Yaşayan güç denetleyicileri ve kuvvet düzenleyicileri, Kâinatsal inşanın, yıkımın ve yeniden yapımın sonu gelmeyen başkalaşımlarının ussal yönlendirilişinin ve özel denetiminin sırrıdır. Nebulalar ortalığa saçılabilir, güneşler sönebilir, sistemler ortadan kalkabilir ve planetler bile tamamen yok olabilir; fakat âlemler hiçbir zaman yıkım süreci içerisine girmezler.
\usection{9.\bibnobreakspace Aşkın Evrenlerin Döngüleri}
\vs p015 9:1 Cennet’in Kâinatsal döngüleri gerçekte yedi aşkın evrenin alanlarını tümüyle kuşatır. Bu mevcudiyet döngüleri şu çekimler tarafından meydana gelir: Kâinatın Yaratıcısı’nın kişilik çekimi, Ebedi Evlat’ın ruhsal çekimi, Bütünleştirici Bünye’nin ussal çekimi ve ebedi Ada’nın maddi çekimi.
\vs p015 9:2 Kâinatsal Cennet döngülerine ve deneyimsel İlahiyatlar’ın ve Mutlaklıklar’ın mevcudiyet\hyp{}uygulamalarına ek olarak, aşkın evren uzay düzeyinde faaliyet gösteren güç ayrımının ve bölünmesinin sadece iki enerji döngüsü bulunmaktadır. Bunlar aşkın evren döngüleri ve yerel evren döngüleridir.
\vs p015 9:3 \bibemph{Aşkın Evren Döngüleri}:
\vs p015 9:4 1.\bibnobreakspace Cennetin Yedi Üstün Ruhaniyet’in birinin birleştiren ussal döngüsüdür. Bu tür bir Kâinatsal\hyp{}akıl döngüsü tek bir aşkın evrenle kısıtlıdır.
\vs p015 9:5 2.\bibnobreakspace Yedi Yansıtıcı Ruhaniyetler’in her aşkın evrendeki yansıtıcı\hyp{}hizmet döngüsü.
\vs p015 9:6 3.\bibnobreakspace Cennet üzerinde Kâinatın Yaratıcısı’na doğru olan, Divinington tarafından birtakım biçimler dâhilinde gerçekleştirilen yönlendiriliş ve karşılıklı etkileşim biçimindeki Gizem Görüntüleyicileri’nin gizli döngüleri.
\vs p015 9:7 4.\bibnobreakspace Ebedi Evlat’ın Cennet Evlatları ile olan karşılıklı ilişkisinin döngüsü.
\vs p015 9:8 5.\bibnobreakspace Sınırsız Ruhaniyet’in ışıltılı olan mevcudiyeti.
\vs p015 9:9 6.\bibnobreakspace Havona’nın uzay bildirimleri olarak Cennet’in yayımlayıcıları.
\vs p015 9:10 7.\bibnobreakspace Kuvvet merkezlerinin enerji döngüleri ve fiziksel denetleyicileri.
\vs p015 9:11 \bibemph{Yerel Evren Döngüleri}:
\vs p015 9:12 1.\bibnobreakspace Bahşedilmiş dünyaların Huzur Sağlayıcısı olarak Cennet Evlatları’nın ruhaniyet bahşedişi. Urantia üzerinde Mikâil’in ruhaniyeti olarak Gerçekliğin Ruhaniyeti.
\vs p015 9:13 2.\bibnobreakspace Dünyanızın Kutsal Ruhaniyet’i olarak Ana Ruhaniyetler’in yerel evreni biçimindeki Kutsal Hizmetkârlar’ın döngüsü.
\vs p015 9:14 3.\bibnobreakspace Emir\hyp{}yardımcı akıl\hyp{}ruhaniyetlerin çok farklı alanlarda faaliyet içerisindeki mevcudiyetine ek olarak bir yerel evrenin us\hyp{}hizmet döngüsü.
\vs p015 9:15 Aşkın evrenlerin herhangi birinden ayırt edilebilir hale gelen bu tür ruhsal uyumun bireysel birleşik döngüleri olan yerel bir evrende onun gelişimi olduğu, faaliyetin ve hizmetin birliğinin bu tür bir kimliği gerçekte hüküm sürdüğü zaman; yerel evren, aşkın yaratımın kusursuzlaştırılan birliğinin ruhsal konfederasyonuna giriş için aynı anda uygun hale gelerek eş zamanlı olarak ışığın ve yaşamın yerleştirilen döngüleri içinde dönüş seyrine geçer. Aşkın evren konfederasyonunun üyeliği biçimindeki Zamanın Ataları’nın kurullarına katılım için şartlar şunlardır:
\vs p015 9:16 1.\bibnobreakspace \bibemph{Fiziksel Denge}. Yerel bir evrenin yıldızları ve gezegenleri denge içinde olmak durumunda olup; dolaysız yıldızsal başkalaşımının süreçleri tamamlanmış olmalıdır. Kâinat pürüzsüz bir işleyiş içerisinde işleyişine devam etmek zorunda olup; onun yörüngesi güvenli ve nihai bir biçimde tamamen yerleştirilmiş olmalıdır.
\vs p015 9:17 2.\bibnobreakspace \bibemph{Ruhsal Sadakat}. Yerel bir evrenin olayları üzerinde hâkimiyete sahip olan Tanrı’nın Egemen Evladı’nın kâinatsal bir biçimde tanınmasının ve ona bağlılığın bir durumu mevcut bulunmak zorundadır. Buna ek olarak orada, yerel evrenin tamamının takımyıldızlarının, bireysel gezegenleri ve sistemlerinin arasında uyumlu eş güdümün varlığı oluşmalıdır.
\vs p015 9:18 Yerel evreniniz, aşkın hükümetin tanınan ruhsal ailesinde bir üyeliğe sahip olması şöyle dursun aşkın evrenin yerleştirilen fiziksel düzenine ait olarak bile gösterilemez. Her ne kadar Nebadon, Uversa üzerinde henüz bir temsilciliğe sahip olmasa da; aşkın evren hükümetinin görevlileri olan bizler, tıpkı Uversa üzerinden doğrudan olarak Urantia’ya doğru olan benim hareketimde olduğu gibi, zaman zaman özel görevler için ona ait dünyalara gönderiliriz. Biz, bu dünyaların zor olan sorunlarının çözümü amacıyla sizin yöneticileriniz ve idarecileriniz için olası her yardımı sunmakla mükellefiz; aynı zamanda biz, evreninizin aşkın evren ailesinin birliktelik içerisinde bulunan yaratılmışlarına bütüncül katılımı için yetkin hale geldiğini görmeyi arzuluyoruz.
\usection{10.\bibnobreakspace Aşkın Evren’in Yöneticileri}
\vs p015 10:1 Aşkın evrenin yönetim merkezleri, zaman\hyp{}mekân nüfuz alanlarının en yüksekte bulunan ruhsal hükümetinin mevkileridir. Kutsal Üçlemenin Kurulları’ndan kaynağını alan aşkın hükümetin yönetim kolu; Cennet’in en dış uyduları olan Sınırsız Ruhaniyet’in yedi özel dünyası üzerine konumlanan Yedi Yüce İdarecileri vasıtasıyla aşkın evrenleri idare eden ve Cennet otoritesinin mevkisi üzerinde oturan varlıklar olarak, yüce denetimin Yedi Üstün Ruhaniyet’in bir tanesi tarafından eş zamanlı olarak yönetilir.
\vs p015 10:2 Aşkın evren yönetim merkezleri, Yansıtıcı Ruhaniyetler ve Yansıtıcı Görüntü Yardımcıları’nın yerleşim mekânlarıdır. Bu arasal konumdan, bahse konu olağanüstü varlıklar kendilerinin mükemmel olan yansıma faaliyetlerini yerine getirip, aşağıda olan yerel evrenlere ve yukarıda bulunan merkezi evrene yardımcı olurlar.
\vs p015 10:3 Her aşkın evren, aşkın hükümetin birleşik ana yöneticileri olan Zamanın Ataları’nın üçü tarafından idare edilir. Ona ait olan yönetim kolunda, aşkın evren hükümetinin çalışanları yedi farklı topluluk tarafından meydana gelmiştir.
\vs p015 10:4 1.\bibnobreakspace Zamanın Ataları.
\vs p015 10:5 2.\bibnobreakspace Bilgeliğin Kusursuzlaştırıcıları.
\vs p015 10:6 3.\bibnobreakspace Kutsal Danışmanlar.
\vs p015 10:7 4.\bibnobreakspace Kâinatsal Denetimciler.
\vs p015 10:8 5.\bibnobreakspace Kudretli Haberciler.
\vs p015 10:9 6.\bibnobreakspace Yetkide Yüksek Olanlar.
\vs p015 10:10 7.\bibnobreakspace İsme ve Sayıya Sahip Olmayanlar.
\vs p015 10:11 Zamanın Ataları’nın üçü eş zamanlı olarak, üç milyar Kutsal Danışmanla beraberlik içinde bulunduğu Bilgeliğin Kusursuzlaştırıcıların bir milyarının bir birimi tarafından desteklenir. Kâinatsal Denetimcilerin sayıca bir milyar kadarı her aşkın evren idaresine bağlanmıştır. Bu üç topluluk, kaynağını doğrusal ve kutsal olarak Cennet Kutsal Üçlemesi’nden alan Eş Güdüm Kutsal Üçleme Kişilikleridir.
\vs p015 10:12 Kudretli Haberciler, Yetkide Yüksek Olanlar ve İsme ve Sayıya Sahip Olmayanlar biçimindeki geriye kalan üç ayrı düzeyde bulunanlar fanilerin yüceltilmiş yükselenleridir. Bu düzeydekilerden ilki, Grandfanda zamanı içinde yükseliş düzeni boyunca gelip, Havona’ya doğru geçiş yapar. Cennet’e erişimle birlikte onlar, Cennet Kutsal Üçlemesi vasıtasıyla bütünleşen Kesinliğe Erişecek Olanların Birlikleri’nde toplanırlar; ve bunun sonucunda ise Zamanın Ataları’nın tanrısal hizmetine atanırlar. Bir sınıf olarak bu üç ayrı düzey, mevcudiyet bakımından Kutsal Üçleme hizmeti olarak bilinen fakat ikircikli bir kökene sahip olarak Erişimin Kutsal Üçleme Evlatları olarak bilinirler. Bu nedenle aşkın evrenin yönetim kolu, evrimsel dünyaların kusursuzlaştırılan ve yüceltilen evlatlarını içine alacak şekilde genişletilmiştir.
\vs p015 10:13 Aşkın evrenin eş güdüm halindeki kurulu, önceki adlandırılışlarıyla, ve bunları takip eden birim yöneticileri ve diğer bölgesel denetçileri olarak yedi idare birimlerinden meydana gelmiştir.
\vs p015 10:14 1.\bibnobreakspace Zamanın Ataları --- aşkın evren çoğunluk birimlerinin yöneticileri.
\vs p015 10:15 2.\bibnobreakspace Zamanın Geçmişleri --- aşkın evrenin azınlık birimlerinin yönlendiricileri.
\vs p015 10:16 3.\bibnobreakspace Zamanın Birliktelikleri --- yerel evrenlerin idarecileri için Cennet danışmanları.
\vs p015 10:17 4.\bibnobreakspace Zamanın İnançlıları --- Yıldız takımı hükümetlerinin En Yüksek idarecileri için Cennet rehberleri.
\vs p015 10:18 4.\bibnobreakspace Zamanın İnançlıları --- Yıldız takımı hükümetlerinin En Yüksek idarecileri için Cennet rehberleri.
\vs p015 10:19 6.\bibnobreakspace Aşkın evren yönetim merkezlerinde mevcut olma yetkinliğine sahip olan Zamanın Ebedileri.
\vs p015 10:20 7.\bibnobreakspace Yedi Yansıtıcı Görüntü Yardımcıları --- yedi Yansıtıcı Ruhaniyetler ve onlar vasıtasıyla Cennetin Yedi Üstün Ruhaniyet’in temsilcilerinin sözcüleri.
\vs p015 10:21 Yansıtıcı Görüntü Yardımcıları aynı zamanda, aşkın evren hükümetlerinde etkin bir konuma sahip olan varlıkların sayısız topluluklarının temsilcileri olarak da faaliyet gösterir; fakat onlar şu mevcut an içerisinde birçok farklı nedenden dolayı bireysel yetilerinde tamamen etkin değillerdir. Bu toplulukla bütünleşenler: Yüce Varlık’ın evrim içinde bulunan aşkın evren kişilik dışavurumları, Yücelik’in Koşulsuz Yüksek Denetimcileri, Nihayet’in Yetkin Vicegerentleri, Majeston’un isim verilmemiş ilişki yansıtıcıları, ve Ebedi Evlat’ın üstün kişilik ruhaniyet temsilcileridir.
\vs p015 10:22 Neredeyse her zaman, aşkın evrenlerin yönetim merkezi dünyaları üzerinde yaratılmış varlıkların tüm topluluklarının temsilcilerini bulmak mümkündür. Aşkın evrenin olağan yardımcı olarak çalışması, kudretli ikincil hizmetkâr ruhaniyetleri ve Sınırsız Ruhaniyet’in çok geniş ailesinin diğer bireyleri tarafından gerçekleştirilir. Aşkın evren idaresi, denetimi, hizmeti, ve yönetimsel yargısının işlevinde; Kâinatsal yaşamın her alanının akli yapıları, etkin hizmetle, mahir yönetimle, sevgi dolu yardımla ve adil yargıyla kucaklaşır.
\vs p015 10:23 Aşkın evrenler elçisel temsilin hiçbir biçimini taşımazlar; onlar bütünüyle birbirlerinden ayrık bir konumdadır. Onlar kendilerine ait ortak olayları sadece Yedi Üstün Ruhaniyet tarafından yerine getirilen Cennet değişim merkezi vasıtasıyla öğrenebilirler. Onların idarecileri, Kâinatsal yaratımın diğer bölümlerinde hangi koşulda oluşum gösterdiğinden bağımsız olarak kendi aşkın evrenlerinin refahı için kutsal bilgeliğin kurullarında çalışırlar. Aşkın evrenlerin bu yalıtımı; onların eş güdümünün, evrim içerisinde olan deneyimsel Yüce Varlık’ın kişilik\hyp{}egemenliğinin daha fazla kesinleşmiş gerçekleşmesi tarafından erişilinceye kadar devam edecektir.
\usection{11.\bibnobreakspace Katılımcı Meclis}
\vs p015 11:1 Uversa olarak bu tür dünyalar üzerinde, kusursuzluğun mutlakıyeti ve evrimin demokrasisinin temsilci varlıkları yüz yüze görüşmek için buluşurlar. Aşkın hükümetin yönetim kolu, kusursuzluğun âlemleri içinde bir kökensel oluşum olarak meydana gelir; yasama kolu ise evrim içindeki âlemlerin yükselişinden kaynağını alır ve buradan yayılır.
\vs p015 11:2 Aşkın evrenin katılımcı meclisi yerleşke olarak yönetim merkezi dünyasıyla sınırlıdır. Bu yasama veya danışma kurulu yedi yasama organından meydana gelmektedir, aşkın evren kurullarına kabul edilen her yerel evrenden biri bağımsız bir temsilci seçer. Onların temsilcileri, Havona’ya erişim hakkını kazanmış Uversa üzerinde beklenenden daha uzun süre kalan, Orvonton’un yükseliş içinde olan kutsal yolcu mezunları arasından bu tür yerel evrenlerin yüksek kurulları tarafından seçilir. Onların ortalama görev süresi yaklaşık olarak aşkın evren ortak zamanının yüz yılı kadardır.
\vs p015 11:3 Orvonton yöneticileri ve Uversa meclisi arasında hiçbir bir anlaşmazlığın yaşanmadığının bilgisine sahibim. Yine de buna ek olarak ifade etmek gerekir ki; içinde bulunduğumuz aşkın evren tarihinde, aşkın hükümetin yürütme bölümü katılımcı meclis bünyesinin yürürlüğe koyduğu bir tavsiye kararını uygulamakta tereddüt dahi etmemiştir. Bu iki yapı arasında her zaman en kusursuz uyum ve işlevsel anlaşma hüküm sürmüştür. Bu ifadeler bütünü göstermektedir ki; evrimsel varlıklar, kutsal doğa ve kusursuz kaynağın kişilikleriyle birlikte onları yakınlık içerisine sokmak için yetkin hale getiren kusursuzlaştırılmış bilgeliğin yükseklerine gerçekten erişebilirler. Aşkın evren yönetim merkezleri üzerindeki katılımcı meclisler, Kâinatın Yaratıcısı ve onun Ebedi Evladı’nın çok geniş olan bütüncül evrimsel kavramsallaşmasının bilgeliğini açığa çıkarır ve onun nihai zaferinin haberciliğini yapar.
\usection{12.\bibnobreakspace Yüce Yargı Organları}
\vs p015 12:1 Uversa hükümetinin yürütücü ve karar alıcı kolları hakkında bilgilerimizi paylaşırken, Urantia’da olan sivil hükümetin belirli biçimlerinin karşılaştırılmasından bizim aynı zamanda üçüncü bir idare koluna veya tam adıyla yargı koluna sahip olmamız gerektiğinin çıkarsamasını yapabilirsiniz. Bu çıkarsamanız yerinde bir çıkarsamadır, fakat sahip olduğumuz yargı kolu ayrı bir görevlilerden oluşan bir yapıyı içinde barındırmaz. Mahkemelerimiz davanın çekimi ve doğası uyarınca bir Kutsal Danışman, bir Bilgeliğin Kusursuzlaştırıcısı ve bir Zamanın Ataları vasıtasıyla hükmedilen yapı tarafından meydana gelmektedir. Bir birey, bir gezegen, sistem, takımyıldızı veya evren için, yâda bunlara karşı sunulacak herhangi bir kanıt Denetimciler tarafından takdim edilip yorumlanır. Zamanın çocuklarının ve evrimsel gezegenlerin savunması, yerel evrenler ve sistemler için aşkın evren hükümetinin yasal gözlemcileri olarak Kudretli Haberciler tarafından verilir. Yüksek hükümetin tavrı Yetkide Yüksek Olanlar tarafından şekillenir. Buna ek olarak bu mahkeme davalarının sonucunda çıkacak olan karar genellikle, katılımcı meclis tarafından seçilen algılayış kişiliklerinin bir topluluğu ve İsme ve Sayıya Sahip Olmayanlar tarafından eşit olarak katılan değişik büyüklükteki bir kurul vasıtasıyla oluşturulur.
\vs p015 12:2 Zamanın Ataları’nın mahkemeleri, tüm katılımcı âlemlerin ruhsal hükmü için yüksek yargı mahkemeleridir. Yerel evrenlerin Egemen Evlatları kendilerine ait olan nüfuz alanlarında yüce bir konuma sahiptirler; onlar, irade sahibi yaratılmışların ortadan kaldırılması dışında kalan meselelerde Zamanın Ataları tarafından hüküm veya danışma için gönüllü olarak görüşlerini bildirdikleri ölçüde aşkın hükümete bağımlıdır. Yargının talimatları yerel evrenlerde oluşturulur, fakat irade sahibi yaratılmışların ortadan kaldırılmasıyla ilgili kararlar her zaman aşkın evrenin yönetim merkezlerinde alır ve onlar bu kurum tarafından uygulanır. Yerel evrenlerin Evlatlar’ı fani insanın varlığını sürdürmesine dair bir hükme varabilir, fakat sadece Zamanın Ataları ebedi yaşam ve ölüm hususlarında yönetici yargılayıcı mevkisinde oturabilir.
\vs p015 12:3 Dava açılmasını gerektirmeyen tüm meselelerde; kanıtın sunulması, Zamanın Ataları veya onların birlikteliklerinin aldıkları kararlar ve onların idareleri her zaman oybirliğiyle alınır. Burada biz tam anlamıyla kusursuzluğun kurullarından bahsetmekteyiz, bu bakımdan yüce ve en üstün nitelikte olan mahkemelerin herhangi bir hükmü için hiçbir bir biçimde anlaşmazlıklar veya azınlıkta kalan düşünceler bulunmamaktadır.
\vs p015 12:4 Belirli başlı küçük istisnalar dışında, aşkın hükümetler kendilerine ait nüfuz alanları üzerinde her unsur ve her varlık üzerinde yaptırım gücü uygularlar. Aşkın evren yönetimlerinin kararları ve idarelerinden kaynaklanan hiçbir temyiz talebi bulunmamaktadır, çünkü onlar Cennet’den gelenler olarak ilgili aşkın evrenin nihai sonu üzerinde hüküm sahibi biçimindeki Üstün Ruhaniyet ve Zamanın Ataları’nın çakışan düşüncelerini temsil ederler.
\usection{13.\bibnobreakspace Birim Hükümetleri}
\vs p015 13:1 Bir \bibemph{çoğunluk birimi}; yaklaşık olarak yüz milyar yerleşime açık dünyalar biçimindeki on bin yerel evrenin oluşturduğu yüz azınlık biriminden meydana gelmiş olup, bir aşkın evrenin takriben onda birini teşkil eder. Bu çoğunluk birimleri, Yüce Kutsal Üçleme Kişilikleri olan Zamanın Kusursuzlaştırıcıları’nın üçü tarafından idare edilir.
\vs p015 13:2 Zamanın Kusursuzlaştırıcıları’nın mahkemeleri her ne kadar Zamanın Ataları’nın bu kişileri tarafından oluşsa da, bu koşul onların âlemler üzerinde ruhsal karar mevkisinde oturmalarının dışında gerçekleşir. Bu çoğunluk birim hükümetlerinin işlevi başlıca uçsuz bucaksız bir yaratılmışın ussal düzeyiyle iniltilidir. Çoğunluk birimleri; Cennet İdarecileri’ne ait olan fani\hyp{}yükselim tasarılarının işleyişiyle veya âlemlerin ruhsal idaresiyle eş zamanlı biçimde ilgilisi bulunmayan, rutin ve idari doğanın aşkın evren önemine ilişkin tüm vakalar, Zamanın Ataları’nın mahkemelerine sunuş için el koyulabilir, hüküm verebilir ve sınıflandırabilir. Bir çoğunluk birimi hükümetinin görevlisi aşkın evrendeki eşleniğinden herhangi bir farklılığı içinde barındırmaz.
\vs p015 13:3 Uversa’nın muhteşem uyduları Havona’ya erişim için sizin nihai olan ruhsal hazırlığınızla ilgilidir; tıpkı bunun gibi Uversa çoğunluk beşinci biriminin yetmiş uydusu sizin aşkın evren ussal eğitimiz ve gelişiminiz için ayrılmıştır. Orvonton’un bütününden onlar, zamanın fanilerini ebediyetin sürecine doğru hazırlamak için yorulmak bilmeden çabalayan bilge varlıklarla birlikte burada toplanır. Yükseliş içinde bulunan fanilerin bu eğitiminin büyük bir kısmı yetmiş eğitim dünyası üzerinde gerçekleştirilir.
\vs p015 13:4 \bibemph{Azınlık birim} hükümetleri Zamanın Geçmişlerinin üçü tarafından idare edilir. Onların yönetimi, başlıca tamamlayıcı yerel evrenlerin idaresinin işleyişsel eş güdümü, fiziksel denetimi, birleşimi ve dengelenmesiyle iniltilidir. Her azınlık birimi, yaklaşık olarak sayıca bir milyar yerleşime açık dünyalardan veya bir milyon sistemlerden meydana gelen on bin takımyıldızının oluşturduğu yüz yerel evren kadarıyla bütünleşir.
\vs p015 13:5 Azınlık birimi yönetim merkez dünyaları, Üstün Fiziksel Denetleyiciler’in çok büyük buluşma yerleridir. Bu yönetim merkez dünyaları; aşkın evrenin giriş okullarını oluşturan yedi eğitim alanları tarafından çevrelenmiş olup, kâinat âlemlerinin tümüyle ilgili iradesel ve fiziksel bilgi için eğitim merkezleridir.
\vs p015 13:6 Azınlık birim hükümetlerinin yöneticileri, çoğunluk birim idarecilerinin eş zamanlı karar yetkisi altındadır. Zamanın Geçmişleri; takımyıldızlarının yönetim merkezlerindeki En Yüksekler’in kurullarına benzer bir biçimde bağlı olan Zamanın İnançlıları tarafından ve yerel evrenlerin yönetim merkezleri üzerindeki Kutsal Üçleme danışmanları ve gözlemcileri olarak konumlanan Zamanın Birliktelikleri’nden, bir aşkın evren için getirilen tüm tavsiyeleri eş güdüm haline getirmekten ve gözlemlerin raporlarının bütününü almakla sorumludur. Bu tür raporların tümü çoğunluk birimi üzerindeki Zamanın Kusursuzlaştırıcıları’na iletilmekte olup, bunun sonrasında ise onlar Zamanın Ataları’nın mahkemelerine ulaştırılır. Bu nedenle Kutsal Üçleme düzeni yerel evrenlerin takımyıldızlarından aşkın evren yönetim merkezlerine kadar olan bir çerçevede genişler. Yerel sistem yönetim merkezleri Kutsal Üçleme temsilcilerine sahip değillerdir.
\usection{14.\bibnobreakspace Yedi Aşkın Evren’in Amaçları}
\vs p015 14:1 Yedi aşkın evrenin evriminde kendini gerçekleştirmekte olan yedi ana amaç bulunmaktadır. Aşkın evren evriminde her ana amaç kendisinin bütüncül dışavurumunu sadece yedi evrenden birinde bulur, bu bakımdan her yerel evren kendisine özel bir işleve ve benzersiz olan bir doğaya sahiptir.
\vs p015 14:2 Yedinci aşkın evren olan ve sizi yerel evreninizin ait olduğu Orvonton, âlemlerin fanileri tarafından başlıca bağışlayıcı hizmetin fazlasıyla cömert ve devasa olan bahşedilmişliği sebebiyle tanınır. Zamanın özverisi, özgür olarak ebediyetin dengelenmesini sağlamak için yapılırken; o, sabır tarafından koşullanan bağışlama ve güç yöneticileri tarafından ölçülü hale getirilen adaletin hüküm sürdüğü durumlar için yenilenir. Orvonton, sevgi ve bağışlamanın Kâinatsal bir temsili örneğidir.
\vs p015 14:3 Fakat Orvonton üzerinde kendisini gerçekleştiren evrimsel niyetin gerçek doğasına dair bizim kavrayışımızı tarif etmek çok zordur. Yine de bütünlüğün tek bir anlamıyla karşılıklı olarak burada bütünleşen altı yardımcı aşkın yaratımların içinde dışa vurulduğu biçimiyle, bu aşkın yaratım üzerinde Kâinatsal evrimin benzersiz olan altı amacını hissettiğimizi söylemek yerine olabilir. Ve bu nedenlerden dolayı biz bazı zamanlar, Yüce olan Tanrı’nın tamamlanmış ve evrimini gerçekleştirmiş kişileşmesinin uzak bir zamanda ve Uversa’dan, kendisinin bütüncül deneyimsel büyüklüğünde ve bunun neticesinde her şeye gücü yeten egemen kuvvetinde kusursuzlaştırılmış yedi aşkın evreni idare edeceğinin varsayımında bulunmaktayız.
\vs p015 14:4 Tıpkı Orvonton’un doğası bakımından benzersiz ve nihai sonu itibariyle bireysel olması gibi, onun altı yardımcı aşkın evreninden her biri aynı niteliği taşır. Fakat bunun yanı sıra Orvonton’da gerçekleşen birçok oluşum sizin için açıklığa kavuşturulmamıştır; Orvonton yaşamının bu açığa çıkarılmamış özellikleri içinde onların birçoğunun bazı diğer aşkın evrenlerde çoğunlukla tamamlanmış dışavurumu bulunabilir. Aşkın evren evriminin yedi amacı tüm aşkın yedi aşkın evren boyunca işlev halindedir, fakat her aşkın yaratım bu amaçların en tamamlanmış dışavurumundan sadece bir tanesi açığa çıkaracaktır. Bu aşkın evren amaçlarını daha fazla kavramak hususunda, sizin algılayamadığınız birçok şeyden daha fazlası sizin için açığa çıkarılsa bile bunların sadece çok az bir kısmını kavrayabileceksiniz. Bu bütüncül anlatım bile sadece, sizin dünyanız ve yerel sisteminizin bir parçası olduğu engin yaratımın kısacık bir anlık bakışını yansıtır.
\vs p015 14:5 İçinde bulunduğunuz dünya Urantia olarak adlandırılır, ve onun numarası Satania sistemi olan gezegensel topluluğunun içinde 606’dır. Bu sistem 619 tane yerleşime açık dünyadan oluşur, ve bunlara ek olarak iki yüz gezegen gelecek zamanın birinde yerleşmeye açık dünyalar haline gelmeye uygun olarak evrim içerisindedir.
\vs p015 14:6 Satania, Jerusem olarak adlandırılan bir yönetim merkezine sahiptir, ve onun Norlatiadek takımyıldızının içerisindeki sistem numarası yirmi\hyp{}dörttür. Norlatiadek yüz yerel sistemden oluşmakta olup, o Edentia olarak adlandırılan bir yönetim merkez dünyasına sahiptir. Nortlatiadek’in Nebadon âlemi içerisindeki numarası yetmiştir. Nebadon’un yerel evreni yüz takımyıldızından oluşmakta olup, onun başkenti Salvington olarak bilinir. Nebadon’un âleminin Ensa’nın azınlık birimi içerisindeki numarası seksen\hyp{}dörttür.
\vs p015 14:7 Ensa’nın azınlık birimi yüz yerel evrenden oluşmakta olup, Uversa üçüncü azınlık birimi olarak adlandırılan bir başkente sahiptir. Splandon’un çoğunluk birimi içerisinde bu azınlık biriminin numarası üçtür. Splandon yüz azınlık birimlerinden oluşmuş olup, o Uversa beşinci çoğunluk birimi olarak adlandırılan bir yönetim merkez dünyasına sahiptir. Burası, muhteşem Kâinatın yedinci bölgesi olan Orvonton’un aşkın evreninin beşinci çoğunluk birimidir. Bu nedenle kendi gezegeninizi, kâinat âlemlerinin tümünün düzenlenmesi ve idaresinin oluşumsal yapısı içerisinde tespit edebilirsiniz.
\vs p015 14:8 Sizin dünyanız Urantia’nın muhteşem kâinat numarası 5,342,482,337,666. Bu kayıt numarası, Uversa ve Cennet üzerinde yerleşime açık dünyaların bulunduğu dizilim içindeki size ait numaralandırmadır. Buna ek olarak fiziksel\hyp{}alan kayıt numarasının bilgisine sahibim, fakat böyle bir olağandışı büyüklüğe sahip olan dizilim fani akıl için çok az bir işlevsel yere sahiptir.
\vs p015 14:9 İçinde bulunduğunuz gezegen devasa bir Kâinatın üyesidir; siz neredeyse sınırsız olan dünyalar ailesine aitsiniz. Fakat sizin âleminiz, tıpkı tüm varoluşun tek yerleşmeye açık dünyasıymışçasına eksiksiz bir biçimde idare edilip bütünsel bir sevgiyle beslenir.
\vs p015 14:10 [Yerleşkesi Uversa’dan gelen bir Kâinatsal Denetimci tarafından sunulmuştur.]
