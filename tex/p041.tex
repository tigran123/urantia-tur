\upaper{41}{Yerel Evren’in Fiziksel Özellikleri}
\vs p041 0:1 Her yerel evreni diğerinden ayıran mekân olgularının ayrıcı özelliği Yaratıcı Ruhaniyeti’nin mevcudiyetidir. Nebadon’un tümü, Salvington’un Kutsal Hizmetkârı’na ait olan mekân mevcudiyeti tarafından kesin bir biçimde kapsanmaktadır; ve bu türden bir mevcudiyet, yerel evrenimizin dışsal sınırlarında kesin olarak benzer bir biçimde sonlanmaktadır. Yerel evren Ana Ruhaniyeti’miz tarafından kapsanan yerleşke \bibemph{Nebadon’dur}; onun mekân mevcudiyetini aşan yerler ise, diğer yerel evrenler olarak Orvonton’un aşkın evrenine ait Nebadon dışı mekân bölgeleri biçimindeki dışsal Nebadon’dur.
\vs p041 0:2 Asli evrenin idari düzenlenişi merkezi, aşkın ve yerel evrenlerin hükümetleri arasında keskin bir iş bölümünü açığa çıkarsa da; ve bu bölümler gökbilimsel bir biçimde Havona ve yedi aşkın evrenin mekân bölünmeleri arasında paralel bir biçimde oluşturulmuş bir nitelikte bulunsa da; fiziksel sınırların açık hiçbir uzantısı yerel yaratımları birbirinden ayırmamaktadır. Orvonton’un çoğunluk ve azınlık birimleri bizler için apaçık bir biçimde ayırt edilen bir niteliğe sahiptir. Söz konusu bu durumun nedeni şudur: bu yerel yaratımlar idari bir biçimde, bir aşkın evrenin toplam enerji etkisinin birimlere ayrılışını idare eden belirli \bibemph{yaratıcı} esaslar uyarınca düzenlenmiştir; bunun karşısında güneşler, karanlık adalar, gezegenler ve benzerleri biçimindeki mekânın âlemleri olarak onların fiziksel bileşenleri kökenlerini nebulalardan almaktadır; ve bu bileşenler, Üstün Evren Mimarları’nın belirli bir \bibemph{yaratılış\hyp{}öncesi} (aşkın olarak) tasarımı uyarınca kendilerinin gökbilimsel görünüşlerini oluştururlar.
\vs p041 0:3 Bir veya daha fazla --- hatta sayıca birçokları olarak --- bu türden nebulalar; tıpkı Nebadon’un fiziksel olarak Andronover ve diğer nebulaların yıldızsal ve gezegensel doğum kökenlerinden bir araya getirilmesine benzer bir biçimde, tek bir yerel evrenin nüfuz alanı içinde çevrelenebilir. Nebadon’un bu âlemleri, çeşitli nebula kökenlerinin bir parçasıdır. Ancak onların tümü; aşkın evrenin yörüngeleri üzerinde komşu bir birim ile birlikte seyahat eden mekân bünyelerine ait bizim şu an bir araya gelişimizi üretmek için, güç yöneticilerinin ussal çabaları vasıtasıyla bu yönde düzenlenmiştir.
\vs p041 0:4 Söz konusu bu durum; bizim yerel yaratımımızın ait olduğu Orvonton’un azınlık biriminin Sagittarius merkezi yakınlarındaki gittikçe yerleşik bir konuma gelen bir yörünge içerisinde bugün dönüşünü yapan, Nebadon’un yerel yıldız bulutunun oluşumudur.
\usection{1.\bibnobreakspace Nebadon Güç Merkezleri}
\vs p041 1:1 Mekânın âlemlerinin ana burgaçları biçimindeki sarmal ve diğer nebulalar, Cennet kuvvet düzenleyicileri tarafından başlatılmıştır; ve çekim karşılığının nebulasal evrimini takiben, yıldızsal ve gezegensel doğumlara ait sonraki kuşakların fiziksel evrimini yönlendirmede bütüncül sorumluluğu bunun sonrasında üstlenen güç merkezleri ve fiziksel düzenleyicileri aşkın evren faaliyeti içerisinde onların yerini alır. Nebadon’un evren öncesi bütünlüğünün fiziksel yüksek denetimi Yaratan Evlat’ımızın gelmesi üzerine, evren düzenlenmesi için onun tasarımları ile eş zamanlı olarak uyumlu hale getirilmiştir. Tanrı’nın bu Cennet Evladı’nın nüfuz alanı içerisinde, Yüce Güç Merkezleri ve Üstün Fiziksek Düzenleyiciler, daha sonra ortaya çıkacak olan Morontia Güç Yüksek Denetimcileri ve diğerleri ile birlikte; Nebadon’un çok katmanlı mekân bünyelerini sıkı bir biçimde birleştirilmiş tek bir idari birime bağlayan iletişim hatları, enerji döngüleri ve güç hatlarının geniş yapısını inşa etmek için işbirliğinde bulunmuşlardır.
\vs p041 1:2 Dördüncü düzeyin sayıca yüz kadar Yüce Güç Merkezi, bizim yerel evrenimiz için kalıcı bir biçimde görevlendirilmiştir. Bu varlıklar; Uversa’nın üçüncü\hyp{}düzey merkezlerine ait gücün geliş hatlarını almakta olup, düzeyi düşürülmüş ve dönüştürülmüş döngüleri takımyıldızlarımızın ve sistemlerimizin güç merkezlerine yönlendirir. Bu güç merkezleri birliktelik içerisinde, aksi halde dalgalanma ve çeşitlilik gösteren enerjilerin dengesini ve dağıtımını idare etmek amacıyla işlerlik gösteren düzenlenmenin ve eşitlenmenin yaşayan düzenini üretmek için faaliyet gösterir. Güç merkezleri buna rağmen, güneş lekeleri veya sistem elektrik sorunları gibi geçici ve yerel enerji dengesizlikler ile ilgili değildir; ışık ve elektrik, mekânın temel enerjilerinden biri değildir; onlar ikincil ve alt dışavurumlardır.
\vs p041 1:3 Sayıca yüz kadar olan yerel evren merkezleri, bu evrenin tam enerji merkezinde onların faaliyet gösterdiği yer olan Salvington üzerinde konumlanmıştır. Salvingon, Edentia ve Jerusem gibi Mimari âlemler; onları mekânın güneşlerinden oldukça bağımsız bir hale getiren yöntemler vasıtasıyla aydınlatılmış, ısıtılmış ve enerji kazandırılmıştır. Bu âlemler; düzen kazandırılmak için güç merkezleri ve fiziksel düzenleyiciler tarafından inşa edilmiş olup, enerji dağıtımı üzerinde güçlü bir etki uygulamak için tasarlanmıştır. Güç merkezleri olarak enerji denetiminin bu türden odak noktaları üzerinde etkinliklerini konumlandırarak, yaşayan mevcudiyetleri vasıtasıyla mekânın fiziksel enerjilerini yönlendirir ve onları aktarır. Ve bu türden enerji döngüleri, fiziksel\hyp{}maddi ve morontia\hyp{}ruhsal olguların bütünü için temel bir niteliktedir.
\vs p041 1:4 Beşinci düzeyin sayıca on Yüce Güç Merkezi, yüz takımyıldızı biçimindeki Nebadon’un birinci derece alt birimlerinin her biri için görevlendirilmiştir. Sizin takımyıldızınız olan Norlatiadek içinde onlar, yönetim merkezi âlemi içinde konumlanmamıştır; ancak onlar, takımyıldızının fiziksel çekirdeğini oluşturan devasa yıldız sisteminin merkezinde konumlanmışlardır. Edentia üzerinde birliktelik halindeki mekanik düzenleyicilere ek olarak yakındaki güç merkezleri ile birlikte kusursuz bir biçimde ve sürekli olarak irtibatta bulunan frandalankların sayıca on tane unsuru mevcut bulunmaktadır.
\vs p041 1:5 Altıncı düzeyin bir Yüce Güç Merkezi, her yerel sistemin tam çekim odağında konumlanmıştır. Satania’nın sistemi içinde görevlendirilen güç merkezi, sistemin gökbilimsel merkezinde konumlanan uzayın karanlık bir adasında ikamet eder. Bu karanlık adaların birçoğu, belirli mekân\hyp{}enerjilerini harekete geçiren ve onları yönlendiren çok büyük dinamolar olup; bu doğal şartlar etkin bir biçimde, mekânın evrimsel gezegenleri üzerinde Üstün Fiziksel Düzenleyiciler için daha fazla maddileşen gücün akımlarını idare eden daha yüksek merkezlerle bir birliktelik olarak faaliyet gösteren yaşayan kütlelerin ait olduğu, Satania Güç Merkezi tarafından kullanılır.
\usection{2.\bibnobreakspace Satania Fiziksel Düzenleyicileri}
\vs p041 2:1 Üstün Fiziksel Düzenleyiciler, asli evren boyunca güç merkezleri ile birlikte hizmet verirken; Satania gibi yerel bir sistem içinde onların faaliyetlerinin kavranılması daha kolaydır. Satania; Sandmatia, Assuntia, Porogia, Sortoria, Rantulia, and Glantonia’nın sistemleri gibi doğrudan komşulara sahip olarak Norlatiadek’in takımyıldızının idari düzenini oluşturan yüz yerel sistemden bir tanesidir. Norlatiadek sistemleri birçok açıdan farklılık gösterir, fakat onların tümü tıpkı Satania gibi evrimsel ve gelişime açıktır.
\vs p041 2:2 Satania’nın kendisi; sizin güneş sisteminizin sahip olduğu kökene benzer bir niteliğe onların birkaçının sahip olduğu, sayıca yedi binden fazla gökbilimsel topluluktan veya diğer bir değişle fiziksel sistemlerden meydana gelmiştir. Satania’nın gökbilimsel merkezi, sistem hükümetinin yönetim merkezinden çok uzakta olmayan bir yerde katılımcı âlemleri ile birlikte konumlanmış olan uzayın devasa bir karanlık adasıdır.
\vs p041 2:3 Görevlendirilmiş güç merkezinin mevcudiyeti dışında Satania’nın fiziksel\hyp{}enerji sisteminin bütününün yüksek denetimi, Jerusem üzerinde merkezi bir biçimde konumlanmıştır. Bu yönetim merkezi âlemi üzerinde konumlanan bir Üstün Fiziksel Düzenleyici; Jerusem üzerinde yönetimsel olarak merkezileşmiş güç müfettişlerinin birliktelik baş idarecisi olarak hizmet ederek ve yerel sistem boyunca faaliyet göstererek, sistem güç merkezi ile birlikte eş güdüm içerisinde çalışmaktadır.
\vs p041 2:4 Enerjinin döngü içindeki dolanımı ve belirli bir doğrultuda aktarımı, Satania boyunca dağılmış olan yaşayan ve ussal enerji değiştiricilerinin sayıca elli yüz bin unsuru vasıtasıyla yüksek bir biçimde denetlenir. Bu türden fiziksel düzenleyicilerinin faaliyeti vasıtasıyla yüksek denetimde bulunan güç merkezleri; oldukça yüksek bir sıcaklıkta ısıtılmış kürelerin ve karanlık enerji\hyp{}etkisine maruz kalmış âlemlerin etkileşimsel yayılımlarını içine alan, mekânın temel enerjilerinin büyük bir çoğunluğunun bütüncül ve kusursuz denetimi içindedir. Yaşayan birimlerin bu topluluğu; düzenlenmiş mekânın fiziksel enerjilerinin neredeyse tümünü harekete geçirmeye, onları dönüştürmeye, değiştirmeye, idare etmeye ve aktarmaya yetkindir.
\vs p041 2:5 Yaşam, evrensel enerjinin devinimi ve dönüşümü için içkin yetkinliğe sahiptir. Siz; ışığın maddi enerjisinin bitkisel hükümranlığın çeşitli dışavurumlarına dönüştürülmesi bakımından, bitkisel yaşamın etkinliğine aşina bir durumda bulunmaktasınız. Siz aynı zamanda, bitkisel enerjinin hayvansal etkinliklerin olgularına dönüştürülebildiği işleyiş yönetimi hakkında birtakım niteliklerin bilgisine sahipsiniz. Ancak siz işlevsel olarak; mekânın çok katmanlı enerjilerini harekete geçirme, onları dönüştürme, yönlendirme ve yoğunlaştırma yetkinliği ile bahşedilen güç yöneticilerin ve fiziksel düzenleyicilerin işleyiş biçimi ile ilgili hiçbir şey bilmemektesiniz.
\vs p041 2:6 Enerji âlemlerin bu varlıkları kendilerini doğrudan bir biçimde, yaşayan varlıkların tamamlayıcı bir etkeni olarak enerji ile ilgili bir konuma sokmamaktadırlar; bu durum fizyolojik kimyanın nüfuz alanıyla bile söz konusu geçerliliği taşır. Onlar zaman zaman; başlangıç maddi organizmaların yaşayan enerjileri için fiziksel araçlar biçiminde hizmet edebilecek bu enerji sistemlerinin açıklığa kavuşturulması ile birlikte olarak, yaşamın fiziksel hazırlıkları ile ilgilidir. Bir açıdan fiziksel düzenleyiciler; emir\hyp{}yardımcı akıl\hyp{}ruhaniyetlerinin maddi aklın ruhaniyet öncesi faaliyetleri ile ilgili olduğu gibi, maddi enerjinin yaşam öncesi dışavurumları ile ilgilidir.
\vs p041 2:7 Güç denetiminin ve enerji yönlendirilişinin bu ussal yaratılmışları; ilgili gezegenin fiziksel oluşumu ve mimarisi doğrultusunda, her âlem üzerinde kendi işleyiş biçimlerini ayarlamak durumundadırlar. Onlar hataya yer bırakmayan bir biçimde; yüksek bir sıcaklıkta ısıtılan güneşlerin ve yüksek bir biçimde etkiye maruz kalan yıldızların diğer türlerinin yerel etkisi ile ilgili olarak, fizikçilerden ve diğer teknik danışmanlardan oluşan onların ilgili görevlilerinin hesaplamalarını ve çıkarımlarını kullanırlar. Uzayın devasa büyüklükteki soğuk ve karanlık gök cisimleri ve yıldız tozunun kümelenen bulutları bile bu durumla ilgili olarak tanınmalıdır; bahse konu bu maddi unsurların tümü, enerji idaresinin işlevsel sorunları ile ilgilidir.
\vs p041 2:8 Evrimsel dünyaların güç\hyp{}enerji yüksek denetimi, Üstün Fiziksel Düzenleyiciler’in sorumluluğudur; ancak bu varlıklar, Urantia üzerindeki enerjinin davranış bozuklularının tümünden sorumlu değildir. Bu tür enerji sorunlarının birçok sebebi bulunmaktadır; bu sorunlardan bazıları, fiziksel koruyucuların nüfuz alanı ve denetiminin ötesindedir. Urantia; devasa kütlelerin döngüsü içerisinde küçük bir gezegen olarak, muazzam büyüklükteki enerjilerin hatları üzerindedir; ve yerel düzenleyiciler zaman zaman, enerjinin bu hatlarının eşitlenmesine dair bir çaba içerisinde düzenlerinin devasa sayıdaki unsurlarını görevlendirir. Onlar, Satania’nın fiziksel döngüleri ile ilgili olarak bu işlemleri oldukça yerinde bir biçimde gerçekleştirir; fakat onlar, güçlü Norlatiadek akımlarının yalıtımları hususunda sorun yaşamaktadır.
\usection{3.\bibnobreakspace Bizim Yıldızsal Birlikteliklerimiz}
\vs p041 3:1 Satania içinde dışa doğru ışık ve enerji yayan sayıca iki binden fazla parlak güneş bulunmaktadır; ve sizin kendi güneşinizin ortama bir alev küresidir. Sizin güneşinize yakın olan otuz güneş içerisinde sadece üçü daha parlaktır. Evren Güç Yöneticileri, bireysel yıldızlar ve onların ilgili sistemleri arasında hareket eden enerjinin özelleşmiş akımlarını başlatır. Mekânın karanlık gök cisimleri ile birlikte bu güneş fırınları; maddi yaratımların enerji döngülerinin etkin yoğunlaştırılması ve yönlendirilmesi için yöntem istasyonları olarak, güç merkezlerine ve fiziksel düzenleyicilere hizmet eder.
\vs p041 3:2 Nebadon’un güneşleri, diğer evrenlerin bahse konu bu güneşlerine benzememektedir. Güneşlerin, karanlık adaların, gezegenlerin, uyduların ve hatta göktaşlarının maddi oluşumu oldukça özdeştir. Bu güneşler; sizin kendi güneşsel kürenizin sahip olduğu çaptan biraz daha az bir biçimde, yaklaşık olarak bir milyon milden oluşan ortama bir çapa sahiptir. Antares yıldızsal bulutu olarak evren içindeki en büyük yıldız; sizin güneşinizin çapının dört yüz elli kattı olup, hacimsel olarak onun altmış milyon katıdır. Ancak, bu devasa güneşlerin tümüne ev sahipliği yapacak çok yeterli bir miktarda mekânsal yerleşke mevcut bulunmaktadır. Onlar; tıpkı bir düzine portakalın Urantia’nın içi etrafında bir boş küre gezegeni gibi dönüş halinde bulunmasına benzer bir biçimde, mekân içinde göreceli olarak çok daha fazla hareket alanına sahiptir.
\vs p041 3:3 Gezenler haddinden fazla bir büyüklüğe sahip olduğunda, bir nebulasal ana burgaçtan atılırlar; ve onlar yakın bir zaman içinde parçalanır veya diğer bir değişle kendi bünyesinin parçalarıyla çifte yıldızları oluşturur. Her ne kadar güneşler daha sonra yarı akışkan bir düzeyde geçici bir süreliğine mevcut bir halde bulabilse de, güneşlerin tümü özgün bir biçimde gazsaldır. Sizin güneşiniz yüksek gaz basıncının bu yarı\hyp{}akışkan düzeyine ulaştığında, çifte yıldız oluşumunun bir türü olarak eşit bir biçimde ikiye ayrılmak için yeterli bir büyüklüğe sahip değildi.
\vs p041 3:4 Sizin güneşinizin büyüklüğünün onda birinden daha az olduğu zaman bu alev âlemleri çabuk bir biçimde büzüşür, yoğunlaşır ve soğur. Diğer bir değişle mevcut maddenin toplam kütlesinin otuz katı biçiminde, bu büyüklüğün otuz katından daha yüksek bir düzeyde bulunduklarında güneşler hazır bir biçimde; ya yeni sistemlerin merkezleri haline gelerek veya aynı sistem içinde bir diğerinin çekim etkisi içinde bulunmaya devam ederek ve çifte yıldızın bir türü biçiminde ortak bir merkez etrafında dönerek, iki ayrı bedene ayrılır.
\vs p041 3:5 Orvonton içinde en büyük kâinatsal patlamanın en son gerçekleşen etkinliği, Urantia içinde ışığının M.S. 1572 yılında ulaştığı olağanüstü çifte yıldız patlamasıydı. Bu alevlenme o kadar yoğundu ki, patlama açık bir biçimde geniş gün ışında bile görülebiliyordu.
\vs p041 3:6 Yıldızların tümü katı bir nitelikte değildir, ancak onların yaşlı olanların birçoğu bu halde bulunmaktadır. Zayıf bir biçimde parıldayan yıldızlar biçiminde kırmızımsı olanların bazıları; Urantia’da kullanılan ölçüm sistemine göre bu türden bir yıldızın bir inç küpünün sekiz bin pound ağırlığa denk geleceği biçiminde ifade edilebilecek, devasa kütlelerinin merkezinde bir yoğunluğu elde etmiştir. Isı ve döngü halindeki enerjinin kayboluşuyla eşlik edilen bu devasa basınç; elektronsal yoğunlaşmanın düzeyine onların şu an yakın bir biçimde yaklaştıkları ana kadar, temel maddi birimlerin yörüngelerini gittikçe birbirlerine yakınlaştırmak ile sonuçlanmıştır. Soğuma ve büzülmenin bu süreci, ültimatonik yoğunlaşmanın sınırlayıcı ve hassa patlama noktasına kadar devam edebilir.
\vs p041 3:7 Devasa güneşlerin büyük bir çoğunluğu göreceli olarak gençtir; küçük yıldızların birçoğu yaşlıdır, ancak bu durum hepsi için geçerlilik göstermez. Çarpışma sonucu oluşan küçük yıldızlar; oldukça genç bir durumda bulunabilir, ve onlar gençlik parıltısı olan başlangıçsal bir kırmızı ışık aşamasına hiçbir zaman görünmeden yoğun bir beyaz ışıkla parıldayabilir. Çok genç ve çok yaşlı yıldızların her ikisi de genellikle kırmızı bir ışık ile parıldar. Sarı renk izi, orta yaş veya bir yaşlılık aşamasını gösterir; ancak muazzam derecedeki parlak beyaz ışık, güçlü ve genişlemiş bir ergenlik yaşamının belirticisidir.
\vs p041 3:8 Ergenlik dönemindeki güneşlerin tümü, bir titreşim aşamasından en azından gözle görülür bir biçimde geçmeseler de; siz uzaya baktığınızda, devasa solunum kabartılarının bir dönüşümü tamamlaması için iki ila yedi güne ihtiyaç duyan bu geç yıldızların birçoğunu gözlemleyebilirsiniz. Sizin kendi güneşiniz hala, onun genç zamanlarından kalan kudretli kabartıların azalan bir etkisini taşımaktadır; ancak bu süreç, öncül üç buçuk gün titreşimlerinden mevcut haldeki on bir buçuk yıl güneş lekesi döngülerine doğru genişlemiştir.
\vs p041 3:9 Yıldızsal etmenler sayısız birçok kökene sahiptir. Birtakım çifte yıldızlarda gelgitler; yörüngeleri etrafında dönüş yapan iki bedene ek olarak aynı zamanda ışığın dönemsel dalgalanmaları durumu biçiminde, hızla değişsen uzaklıklar nedeniyle gerçekleşmiştir. Bu çekim çeşitliliği; tıpkı ilgili güneşin olağan aydınlığına hızlı bir biçimde geri dönecek olan enerji\hyp{}madde birikimi vasıtasıyla göktaşlarının içine çekildiği yüzeyde ışığın göreceli anlık bir ışıltısının oluşmasına benzer bir biçimde, düzenli ve tekrar açığa çıkan parıltılar üretmektedir. Zaman zaman bir güneş, etkisi azaltılmış çekim karşıtlığının bir hattı içinde göktaşlarının bir topluluğunu içine çekecektir; ve arada sıra çarpışmalar yıldızsal parıltılara sebep olacaktır; ancak bu gözlenen olguların büyük bir çoğunluğu bütünüyle içsel dalgalanmalar nedeniyle gerçekleşmektedir.
\vs p041 3:10 Değişken yıldızların bir topluluğu içinde ışık dalgalanmasının süreci, tamamiyle aydınlatma gücüne bağlıdır; ve bu gerçekliğin bilgisi, uzak yıldız bulutlarının ileri keşifleri için evren deniz fenerleri veya doğru ölçüm noktaları olarak bu güneşleri gökbilimcilerin kullanmalarını etkin bir hale getirmektedir. Bu işleyiş biçimi vasıtasıyla yıldızsal uzaklıkları tam olarak bir milyon ışık yılından daha fazla bir aralıkta ölçmek mümkündür. Mekân ölçümünün ve geliştirilmiş teleskopsal işleyiş biçiminin daha iyi yöntemleri ileride, Orvonton’un aşkın evrenine ait on büyük bölünmeyi bütünüyle ortaya çıkaracaktır; siz en azından, devasa ve simetrik yıldız kümelenmeleri olarak bu engin birimlerin sekizini tanımlayacaksınız.
\usection{4.\bibnobreakspace Güneş Yoğunluğu}
\vs p041 4:1 Sizin güneşinizin kütlesi; yaklaşık olarak iki oktilyon ($2\times 10^{27}$) ton olarak hesaplanmış biçimde, sizin fizikçilerinizin tahmininden biraz daha büyüktür. Güneşinizin yoğunluğu şu an içerisinde; suyun yoğunluğunun yaklaşık olarak bir buçuk katına sahip olarak, en yoğun ve en seyreltik yıldızların yaklaşık olarak ortasında bulunmaktadır. Fakat sizin güneşiniz gazsal bir biçimde olarak ne bir sıvı ne de bir katıdır; ve bu durum, gazsal maddenin bu yoğunluğa ve hatta bu yoğunluktan daha fazla olan düzeylere nasıl eriştiğine dair yapılacak olan bir açıklamanın zorluğuna rağmen, gerçektir.
\vs p041 4:2 Gaz, sıvı ve katı haller atomsal\hyp{}moleküler ilişkilerin olaylarıdır; fakat yoğunluk mekân ve kütlenin bir ilişkisidir. Yoğunluk, boşluk içindeki kütlenin niceliği ile doğru orantılı olarak çeşitlilik gösterir; bunun karşısında yoğunluk, kütle içindeki boşluğun niceliği ile ters orantılıdır. Bu boşluk, bu türden maddi parçacıkların içindeki boşluğa ek olarak bu merkezler etrafında dönen madde ve partiküllerin merkezi çekirdekleri arasında tanımlanmıştır.
\vs p041 4:3 Soğuyan yıldızlar fiziksel bir biçimde gaz halinde olup, aynı zamanda muhteşem bir derece yoğunluğa sahiptir. Siz, güneşsel\bibemph{ aşkın gazlara} aşina değilsiniz; fakat maddenin bu ve diğer olağanüstü türleri, katı olmayan güneşlerin nasıl --- Urantia’da olduğu gibi yaklaşık olarak --- bir demirin yoğunluğuna erişebildiklerini ve yine de yüksek bir düzeyde ısıtılmış haz halinde kalmaya devam edip güneşler olarak faaliyetlerini sürdürebildiklerini açıklamaktadır. Bahse konu bu yoğun aşkın gazlar içindeki atomlar olağandışı bir biçimde küçüktür; onlar, çok az elektron taşımaktadır. Bu türden güneşler aynı zamanda, enerjinin kendilerine ait özgür ültimatonik saklama alanlarını büyük bir oranda yitirmişlerdir.
\vs p041 4:4 Güneşiniz gibi yaklaşık olarak aynı kütlede yaşama başlayan sizin yakın güneşlerinizden bir tanesi mevcut an içerisinde; güneşinize kıyasla kırk bin kez daha fazla yoğunluğa sahip olarak, Urantia’nın büyüklüğü kadar büzülmüştür. Isı bakımından sıcak ve soğuk arasında ve hal durumu bakımından gaz ve katı arasında bulunan bu güneşin ağırlığı her bir inç küpte yaklaşık olarak bir tondur. Ve bu güneş hala; ışığın ölmekte olan hükümranlığının yaşlı parıltısı biçiminde, solgun bir kırmızımsı ışıltısıyla parıldamaktadır.
\vs p041 4:5 Güneşlerin birçoğu buna rağmen bu ölçüde yoğun değildir. Sizin yakın komşularınızdan bir tanesi, deniz seviyesindeki sizin atmosferinize tam olarak eşit bir yoğunluğa sahiptir. Ve sıcaklığın izin verdiği müddetçe siz; gece gökyüzü içerisinde anlık bir biçimde parıldayan ve dünyasal yaşam odalarınızın havası içinde algıladığınızdan daha fazlasını fark etmeyeceğiniz, güneşlerin büyük bir çoğunluğuna giriş yapabilirsiniz.
\vs p041 4:6 Orvonton içinde en büyük güneşlerden biri olan, Veluntia’nın devasa büyüklükteki güneşi; Urantia’nın atmosferinin yalnızca binde birine sahip bir yoğunluktadır. Bu güneş; oluşum bakımından sizin atmosferinize benzerlik gösterseydi ve yüksek bir biçimde ısıtılmasaydı, içlerinde insan varlıklarının yaşadığı bir durumda onların hızla boğulacakları bir havasız boşluktan ibaret olurdu.
\vs p041 4:7 Orvonton gök cisimlerinin diğerleri mevcut an içerisinde, sıcaklık bakımından derece üç binlerin altında önemsiz bir yüzey sıcaklığına sahiptir. Onun çapı; güneşinizi ve dünyanızın mevcut yörüngesini içine alabilecek kadar yeterli hareket alanına sahip olarak, üç yüz milyon milin üzerindedir. Ancak yine de, güneşinizin kırk milyon katından fazla olarak, bu devasa büyüklüğün tümü için onun kütlesi; yalnızca yaklaşık olarak otuz kat fazladır. Bu devasa güneşler, neredeyse birinden diğerine uzanan bir biçimde, genişleyen bir sınıra sahiptir.
\usection{5.\bibnobreakspace Güneş Radyasyonu}
\vs p041 5:1 Uzayın güneşlerinin çok yoğun olmaması, kaybolan ışık\hyp{}enerjilerinin hazır akımları tarafından ispatlanmıştır. Çok büyük bir yoğunluk opaklık vasıtasıyla ışığı, ışık\hyp{}enerji basıncının patlama noktasına gelişine kadar muhafaza edecektir. Bir güneş içerisinde; milyonlarca mil uzaklıktaki gezegenlere enerji sağlamak, onları aydınlatmak ve ısıtmak için uzaya giriş yaparak enerjinin bu türden bir akımının ileri doğru gönderilmesine sebep olan, devasa bir ışık ve gaz basıncı mevcut bir halde bulunmaktadır. Urantia’nın yoğunluğundaki yüzeyin on beş fitlik bölgesi etkin bir biçimde; devasa bir dışa doğru patlama ile birlikte çekim etkisinin üstesinden gelen atomsal bombardımanlardan kaynaklanan artış halindeki enerjilerin yükselen içsel basıncına kadar, bir güneş içerisinde tüm X\hyp{}ışınlarının ve ışık enerjilerin kaçmasını engellemektedir.
\vs p041 5:2 Sevk edici gazların mevcudiyetinde ışık, opak sıkıştırıcı duvarlar vasıtasıyla yüksek sıcaklıklarda taşındığında oldukça patlayıcıdır. Işık gerçektir. Dünyanız üzerinde sizin enerjiye ve güce paha biçişinize göre, güneş ışığının bir poundu bir milyon dolara karşılık gelebilecek bir değere sahiptir.
\vs p041 5:3 Sizin güneşinizin içi geniş bir X\hyp{}ışını üreticisidir. Güneşler içlerinde, bu kudretli ışın yayılımlarının aralıksız süren bombardımanları ile tedarik edilmişlerdir.
\vs p041 5:4 Bir X\hyp{}ışınıyla etkileştirilen elektronun ortalama bir güneşin tam merkezinden güneşsel yüzeye kadar olan doğrultusunda ilerlemesi için yarım milyon yıldan fazla bir süre gerekmektedir. Buradan elektron kendisinin uzay yolculuğuna; yerleşik bir gezeni ısıtmak, bir göktaşı tarafından alı konulmak, bir atomun doğumuna katılmak, mekânın yüksek bir biçimde etki altında bulunan adası tarafından çekilmek veya kökeninin güneşine benzer bir güneşin yüzeyine nihai bir biçimde dalış vasıtasıyla sonlanan kendi uzay yolculuğunu bulmak amacıyla başlayabilir.
\vs p041 5:5 Bir güneşin içine ait X\hyp{}ışınları, oldukça yüksek sıcaklıklarda ısıtılmış ve etkiye uğramış elektronların; katalizör maddeye ait alıkoyucu etkilerin ev sahiplerini geçerek ve çeşitli çekim etkinlerine rağmen, uzak sistemlerin ücra âlemlerine kadar uzay boyunca taşınması için yeterli enerjiyi sağlamaktadır. Bir güneşin çekim etkisinden kurtulmak için ihtiyaç duyulan büyük bir enerji; maddenin önemli derecedeki kütleleri ile karşılaşana kadar aşılamaz hızlarla hareket eden güneş ışınını sağlamaya yetkindir; bu karşılaşma üzerine güneş ışığı hemen, diğer enerjilerin açığa çıkmasıyla birlikte ısıya dönüşür.
\vs p041 5:6 Işık veya diğer türler içinde bulunmasından bağımsız bir biçimde enerji, mekân doğrultusunda olan hareketinde doğrusaldır. Maddi mevcudiyetin mevcut parçacıkları mekânı bir yaylım ateşi gibi kat eder. Onlar; daha yüksek kuvvetler tarafından harekete geçirilmediği ve madde kütlesi içinde içkin doğrusal çekim etkisine ek olarak Cennet Adası’nın çevresel\hyp{}çekim varlığına maruz kalmadığı müddetçe, doğrusal ve dağılmayan bir hat içinde veya hareket yörüngesinde ilerler.
\vs p041 5:7 Güneş enerjisi, dalgalar halinde ilerliyormuş gibi görünebilir; ancak bu durum, bağlı ve çeşitli etkilerin gerçekleşmesi sonucunda ortaya çıkar. Düzenlenmiş enerjinin ilgili herhangi bir türü, dalgalar halinde değil doğrusal hatlar boyunca ilerler. Kuvvet\hyp{}enerjinin ikincil veya üçüncül bir türünün mevcudiyeti, gözlem altında onların akımlarının dalgasal oluşumlar biçimde \bibemph{görülmesine} sebep olabilir. Bu durum, sert bir rüzgârla birlikte ortaya çıkan kör edici bir yağmur fırtınasında suyun zaman zaman yatay katmanlar veya dalgalar halinde yeryüzüne düşmesine oldukça benzemektedir. Aralıksız hareket yörüngesinin doğrusal bir hattı içinde yağmur damlaları yeryüzüne doğru hareket etmektedir; ancak rüzgârın hareketi, yağmur damlalarının yatay katmanlar veya dalgalar biçimindeki bu türden gözle görülen olgularını ortaya çıkarmaktadır.
\vs p041 5:8 Yerel evreninizin uzay bölgeleri içinde mevcut olan belirli ikincil ve diğer keşfedilmemiş enerjilerin etkinliği öyle bir oluşumdur ki güneş\hyp{}ışık yayılımı; birtakım dalgasal olgulara ek olarak belirli uzunluk ve ağırlığın ölçülemeyecek kadar küçük parçacıklarına kadar bölünüşünü uygular biçiminde görülmektedir. Ve işlevsel bir açıdan irdelendiğinde bu durum gerçekte yaşanılanların ta kendisidir. Nebadon’un mekân bölgeleri içerisinde işlerlik gösteren çeşitli güneş\hyp{}kuvvetlerine ek olarak güneş enerjilerinin karşılıklı etkileşimleri ve ilişkilerinin kavramına dair daha açık bir kavrayışa eriştiğiniz ana kadar siz, ışığın davranışının anlaşılmasına dair daha iyi bir kavrayışa ulaşmayı hayal dahi edemezsiniz. Sizin mevcut kafa karışıklığınız aynı zamanda; Bütünleştirici Bünye ve Koşulsuz Mutlak’ın mevcudiyetleri, etkinlikleri ve eş güdümü biçiminde üstün evrenin kişisel ve kişisel olmayan denetimlerine ait karşılıklı birliktelik içerisinde bulunan faaliyetleri ile ilgili olan, bu soruna dair eksik kavrayışınızdan kaynaklanmaktadır.
\usection{6.\bibnobreakspace Mekânın Gezgini olarak, Kalsiyum}
\vs p041 6:1 Işık tayfı olgusunun irdelenmesi içinde şu gerçek hep hatırlanmalıdır: uzay boşluktan ibaret değildir; uzayı kateden ışık zaman zaman, düzenlenmiş mekânın tümü içinde dolaşan bir biçimdeki enerji ve maddenin çeşitli türleri tarafından küçük bir derecede değişikliğe uğrar. Güneşinizin tayfı içerisinde ortaya çıkan bilinmeyen maddeyi işaret eden hatların bazıları; güneşsel element savaşlarının çetin çarpışmaları sonrasında oluşan ölü atomlar şeklinde, dağıtılmış bir konumda mekânın tümü boyunca uçmakta olan çok iyi bilinen elementlerin yapılarında meydana gelen değişiklerden kaynaklanmaktadır. Uzay; özellikle sodyum ve kalsiyum biçiminde bu gezinmekte olan sahipsiz cisimler tarafından kaplanmıştır.
\vs p041 6:2 Kalsiyum, gerçekte, Orvonton boyunca mekânın madde\hyp{}yayılımına ait başat elementtir. Bizim aşkın evrenimizin bütünü, özenle toz haline getirilen taşla serpilmiştir. Taş gerçek anlamıyla mekânın gezegenleri ve âlemleri için temel yapı maddesidir. Büyük mekân örtüsü olarak kâinatsal bulut, kalsiyumun değişikliğe uğramış olan atomlarının büyük bir kısmından meydana gelmiştir. Taş atomu, elementlerin en yaygın ve en dayanaklı olanların biridir. Onlar --- bölünme biçimiyle --- sadece güneşsel iyonlaşmaya karşı koymaz; onlar aynı zamanda, yıkıcı X\hyp{}ışınları tarafından yıpranmalarından ve yüksek güneş sıcakları altında dağılmalarından sonra bile bir birliktelik kimliği içerisinde kalmaya devam eder. Kalsiyum, maddenin daha fazla ortaklık gösteren türlerinin hepsinin de üstünde bir kişiliği ve dayanıklılığı elinde bulundurur.
\vs p041 6:3 Sizin fizikçilerinizin sezdikleri gibi, güneşsel kalsiyumun bu sakatlanmış kalıntıları kelimenin gerçek anlamıyla; çeşitli uzaklıklar için güneş ışınları üzerine binerler; ve böylece onların mekân boyunca geniş bir alana yayılan dağılımı, devasa bir biçimde gerçekleştirilmiş olur. Belirli dönüşümler altında sodyum atomu aynı zamanda ışık ve enerjinin gezinme hareketine yetkindir. Kalsiyum kazanımının bütünlüğü, daha önemli bir niteliğe sahiptir; çünkü bu element, sodyum kütlesinden yaklaşık olarak iki katı kütleye sahiptir. Kalsiyum tarafından yerel mekân\hyp{}yayılımı; değişikliğe uğramış tür içinde, kelimenin tam anlamıyla dışa doğru hareket eden güneş ışınlarına binerek, güneşsel fotosferden onların kaçmasıyla gerçekleşmektedir. Güneş elementleri arasında kalsiyum; yirmi dönüş elektronu taşıyan bir biçimde onun görece kütlesine rağmen, güneşin içinden mekânın âlemlerine doğru olan kaçışta en başarılı olanıdır. Bu durum; güneş üzerinde altı bin millik kalınlık biçimindeki bir gaz taşı yüzeyi olarak, bir kalsiyum tabakasının orada neden mevcut bulunduğunu açıklamaktadır; ve daha hafif olan diğer on dokuz elemente ve sayısız bir miktarda bulunan daha ağır olanlarına rağmen, güneşin tabanında yer alan element kalsiyumdur.
\vs p041 6:4 Kalsiyum, güneş sıcaklıklarında etkin ve çok yönlü bir elementtir. Taş atomu, birbirlerine çok yakın olan iki dış elektronsal döngüleri içerisinde çevik ve zayıf bir biçimde birbirlerine bağlanmış elektronlara sahiptir. Atomsal mücadelenin ilk aşamasında onlar dıştaki elektronunu yitirir; bu düzeyden itibaren taş atomu, elektronsal devrimin on dokuzuncu ve yirminci döngüleri arasında on dokuzuncu elektronun ileri geri bir biçimde hareket ederek gerçekleştirdiği üstün bir yer değiştirme etkinliği içine girer. Bu on dokuzuncu elektronun kendi yörüngesi ve kaybolan eşinin etrafından saniyede yirmi beş binden daha fazla bir biçimde ileri geri hareket edişi vasıtasıyla bir sakatlanmış taş atomu, yer çekimini kısmen devre dışı bırakmaya yetkindir; ve böylelikle bu atom başarılı bir biçimde, özgürlük ve serüven için güneş ışınları üzerinde ışığın ve enerjinin ortaya çıkan ırmakları içinde hareket etmeye yetkin hale gelir. Bu kalsiyum atomu, saniyede yaklaşık yirmi beş bin kere güneş ışığını tutarak ve onu bırakarak ileri doğru ivmelenmenin dönüşümlü kasılmaları vasıtasıyla dışa doğru hareket eder. Ve bu durum, bu taşın neden mekânın dünyalarının baş bileşeni olduğunu açıklamaktadır.
\vs p041 6:5 Bu cambaz kalsiyum elektronunun çevikliği, güneşsel kuvvetlerin X\hyp{}ışını sıcaklığı vasıtasıyla daha yüksek yörüngenin döngüsüne fırlatıldıklarında yalnızca bir saniyenin yaklaşık olarak bir milyonunda bu yörünge içine geçiş yapmaları gerçekliğiyle doğrulanmıştır; ancak atomsal çekirdek çekiminin elektriksel\hyp{}çekim gücünden önce bu elektron, atomsal sürecin etrafında bir milyon dönüşü tamamlamaya yetkin olarak eski yörüngesine geri döner.
\vs p041 6:6 Sizin güneşiniz, güneş sisteminin oluşuyla ilişkili olan kasılan patlamalarının zamanı boyunca devasa büyüklükteki kalsiyumunun çok büyük bir oranını yitirmiştir. Güneş kalsiyumunun birçoğu mevcut an içerisinde güneşin dışta bulunan tabakası içerisindedir.
\vs p041 6:7 Güneşin tayfsal irdelenmesinin yalnızca güneş\hyp{}yüzeyi oluşumunu gösterdiği unutulmamalıdır. Örnek olarak: Güneş tayfı demir hatlarını ortaya çıkarmaktadır, fakat demir güneş içinde temel bir element değildir. Bu olgu neredeyse tamamiyle; demir tayfının kaydı için oldukça elverişli sıcaklık olan 6.000’den biraz daha az derecedeki, güneşin mevcut sıcaklığından kaynaklanmaktadır.
\usection{7.\bibnobreakspace Güneş Enerjisi’nin Kaynakları}
\vs p041 7:1 Kendi güneşiniz dâhil olmak üzere, güneşlerin birçoğunun iç sıcaklığı ortak olarak inanılandan çok daha fazladır. Bir güneşin içinde işlevsel biçimde bütüncül hiç bir atom mevcut değildir; onlar az veya çok, bu tür sıcaklıklara özgü olan yoğun X\hyp{}ışını bombardımanı tarafından dağılmış bir haldedir. Bir güneşin dış tabakaları içinde hangi maddi elementlerin ortaya çıkabildiğinden bağımsız olarak, içte bulunan unsurlar çok benzer bir biçimde; yıkıcı X\hyp{}ışınlarının dağıtıcı etkisi tarafından mevcut hale gelir. X\hyp{}ışını, atomsal mevcudiyetin büyük eşitleyicisidir.
\vs p041 7:2 Güneşinizin yüzey sıcaklığı yaklaşık olarak 6.000 derecedir; ancak onun içine girildiğinde bu sıcaklık, merkezi bölgeleri içinde yaklaşık 35.000.000 derecelere varan inanılmaz bir yüksekliğe ulaşana kadar çabuk bir biçimde artış gösterir. (Bu sıcaklıkların tümü Fahrenhayt ölçeğine göredir.)
\vs p041 7:3 Bu olgular bütününün hepsi, devasa enerji çıkışının bir göstergesidir; önemlerine göre isimlendirilmiş bir biçimde, güneş enerjisinin kaynakları şunlardır:
\vs p041 7:4 1.\bibnobreakspace Atomlar ve nihai olarak elektronların yok olması.
\vs p041 7:5 2.\bibnobreakspace Böylece serbest bırakılan enerjilerin radyoaktif topluluklarını içine alarak, elementlerin dönüşümü.
\vs p041 7:6 3.\bibnobreakspace Belirli evrensel mekân\hyp{}enerjilerinin birleşimi ve aktarımı.
\vs p041 7:7 4.\bibnobreakspace Alev halindeki güneşlere doğru durmaksızın dalmakta olan uzay maddesi ve göktaşları.
\vs p041 7:8 5.\bibnobreakspace Güneş kasılması; bir güneşin soğuması ve bunun sonrasında meydana gelen büzülme, uzay maddesi tarafından tedarik edilenden zaman zaman daha fazla olan enerji ve sıcaklığı ortaya çıkarmaktadır.
\vs p041 7:9 6.\bibnobreakspace Yüksek sıcaklıklarda çekim etkinliği, döngüleştirilen gücü radyoaktif enerjilere dönüştürmektedir.
\vs p041 7:10 7.\bibnobreakspace Güneşi terk ettikten sonra tekrar onun içerisine çekilen tekrar alı konulan ışık ve diğer madde, diğer enerjiler ile birlikte daha fazla güneşsel kökene sahip olmaktadır.
\vs p041 7:11 Orada, zaman zaman milyonlarca derecede sıcaklıkta bulunan güneşi çevreleyen sıcak gazların düzenleyici bir örtüsü bulunmaktadır; bu örtü, ısı kaybını düzenlemek ve bunun geride kalan durumlarında ısı yayılımının yıkıcı dalgalanmalarını engellemek için faaliyet gösterir. Bir güneşin etkin yaşamı boyunca 35.000.000 derecedeki içsel sıcaklığı, dışsal sıcaklığın gittikçe artan düşüşünden oldukça bağımsız bir biçimde yaklaşık olarak aynı değerde kalmaya devam eder.
\vs p041 7:12 Siz sıcaklığın 35.000.000 derecelerini, elektronsal kaynama noktası olarak belirli çekim basınçları ile birliktelik halinde hayalinizde canlandırmaya çabalayabilirsiniz. Bu basınç ve sıcaklık altında atomların tümü, elektronsal ve diğer soy bileşenlerine ayrılmakta ve parçalanmaktadır; elektronlar ve ültimatonların diğer birliktelikleri parçalanabilir, ancak güneşler ültimatonlarına ayrılmaya yetkin değillerdir.
\vs p041 7:13 Bu güneş sıcaklıkları; bu koşullar altında mevcudiyetlerini en azından bahse konu bu son durumda olduğu gibi idare etmeye devam ederek, ültimatonları ve elektronları devasa bir biçimde hızlandırmak amacıyla işlev gösterir. Olağan bir su damlasının, atomların bir milyar trilyonundan fazla bir unsurunu beraberinde taşıdığını bir durup düşündüğünüzde; ültimatonsal ve elektron etkinliklerin hızlandırılması biçiminde bu yüksek sıcaklığın ne anlama geldiğini anlayacaksınız. Bu enerji, iki yıl boyunca sürekli bir biçimde yüz beygirden daha fazla gücünün ürettiği enerjiye karşılık gelmektedir. Her saniye içerisinde güneş sistemi tarafından şu an içinde salınan bütüncül ısı, yalnıza bir saniye içinde Urantia üzerindeki okyanusların tümündeki suyun hepsinin kaynamasına yetecek bir sıcaklığa sahiptir.
\vs p041 7:14 Evren enerjisinin ana akıntılarının doğrusal kanalları içerisinde faaliyet gösteren yalnızca bu güneşler, sonsuza kadar parıldamaya devam eder. Bu türden güneşsel fırınlar, mekân\hyp{}kuvvet ve benzer döngü halindeki enerjinin alınımı vasıtasıyla kendilerinin maddi kayıplarını gidermeye yetkin bir biçimde sonsuza kadar alevlenmeye devam ederler. Fakat yeniden yüklenmenin bu başlıca kanallarından sonsuza kadar ayrılmış olan yıldızlar; yavaşça soğuyup ve nihai olarak sönen bir biçimde, enerjinin bütüncül tüketiminin sürecinden geçmenin nihai sonuna sahip kılınmıştır.
\vs p041 7:15 Bu türden ölü veya ölmekte olan güneşler, çarpışma etkisi vasıtasıyla yeniden canlandırılabilir; veya onlar, parıldamayan belirli mekân enerji adaları tarafından veya yakında bulunan daha küçük güneşlerin veya sistemlerin çekim\hyp{}gaspı vasıtasıyla yeniden yüklü bir hale getirilebilir. Ölü güneşlerin büyük bir çoğunluğu, bu veya diğer evrimsel işleyiş biçimleri tarafından hayata tekrar dönüşü deneyimleyecektir. Böylelikle nihai olarak yeniden yüklü hale gelmeyen güneşlere; çekim yoğunlaşmasına ait enerji basıncının ültimatonsal yoğunlaşmasında eşik noktasına eriştiğinde, kütle patlayışı vasıtasıyla parçalanması sürecinden geçmenin nihai sonu kazandırılmıştır. Bu tür ortadan yok olan güneşler bunun sonucunda; daha elverişli bir konumda bulunan diğer güneşlere enerji kazandırmak amacıyla hayranlık verici bir biçimde kullanılan, enerjinin en nadir türlerinden biri haline gelir.
\usection{8.\bibnobreakspace Güneş\hyp{}Enerji Tepkileri}
\vs p041 8:1 Mekân\hyp{}enerji kanalları içinde döngüsel bir hale getirilen bu güneşler üzerinde güneş enerjisi; hidrojen\hyp{}karbon\hyp{}helyum tepkisinin en ortak unsuru olarak, çeşitli karmaşık nükleer\hyp{}tepki zincirleri tarafından serbest bırakılır. Bu başkalaşım içerisinde karbon, bir enerji katalizörü olarak hareket eder; çünkü karbon, hidrojenin helyuma dönüştürülmesinin bu süreci tarafından hiçbir biçimde mevcut bir halde değişikliğe uğramamaktadır. Yüksek ısıların belirli şartları altında hidrojen, karbon çekirdekleri içine nüfuz eder. Karbonun dört protondan daha fazlasını elinde bulundurmamasından dolayı, ve bu doygunluk düzeyi erişilince; karbon, yenileri gelir gelmez derhal protonları kendi içinden dışarıya doğru çıkarır. Bu tepki içinde içeriye nüfuz eden hidrojen parçacıkları, bir helyum atomunu ortaya çıkarır.
\vs p041 8:2 Hidrojen içeriğinin azalması, bir güneşin parlaklığını arttırır. Sönüşün nihai sonu kazandırılan güneşler içerisinde parlaklığın yüksekliği, hidrojenin bittiği noktada gerçekleşir. Bu noktayı takiben parlaklık, çekim büzülmesinin sonrasında ortaya çıkan işleyiş vasıtasıyla idare edilir. Nihai olarak bu türden bir yıldız, oldukça yoğun bir âlem biçiminde küçük bir beyaz yıldız olarak adlandırılan cisim haline gelir.
\vs p041 8:3 Küçük döngüsel nebulalar biçimindeki büyük güneşler içerisinde, hidrojen bittiğinde ve çekim büzülmesi bunun sonrasında gerçekleştiğinde, eğer bu türden bir bünye dışsal gaz bölgeleri için içsel basıncın sağlanmasını elinde bulundurmak için yeterince opak değilse, bunun sonrasında anlık bir çarpışma gerçekleşir. Çekim\hyp{}elektrik değişiklikleri, elektrik potansiyelinden mahrum olan küçük parçalıkların geniş sayılarının kökenini sağlamaktadır; ve bu türden parçacıklar hazır bir biçimde güneşin iç bölgesinden kaçıp, böylece birkaç gün içerisinde devasa bir güneşin çarpışması sonucunu beraberinde getirir. Elli yıl önce Andromeda nebulasının büyük nova çarpışmasına sebep olan “kaçak parçacıklarının” göçleri, bu türden göçlerden biridir. Urantia zamanına göre bu türden geniş bir yıldızsal bünye, kırk dakika içerisinde yıkıma uğramıştır.
\vs p041 8:4 Bir kural halinde maddenin geniş ihraçları; nebulasal gazların geniş bulutları biçiminde soğuyan artık güneşler olarak mevcut bir durumda bulunmaya devam eder. Ve bu anlatılan olayların tümü; yaklaşık dokuz yüz yıl önce kökenine sahip olduğu ve hala yalnız bir yıldız olarak bu düzensiz nebulasal kütlesinin merkezi yakınlarında ana âlem olarak ortaya çıkan, Yengeç nebulası gibi düzensiz nebulaların birçok türünün kökenini açıklamaktadır.
\usection{9.\bibnobreakspace Güneş’in Yerleşik Konumu}
\vs p041 9:1 Daha büyük güneşler; ışığın yalnızca güçlü X\hyp{}ışınlarının yardımı vasıtasıyla kaçtığı elektronları üzerinde, bu türden bir çekim denetimi sağlamaktadır. Bu yardımcı ışınlar, tüm uzaya nüfuz eder; ve onlar, enerjinin temel ültimatonsal birlikteliklerinin idaresi ile ilgidir. 35.000.000 derecelerinde üstünde bulunan en yüksek sıcaklığa olan erişimlerini takiben, bir güneşin öncül zamanlarındaki büyük enerji kayıpları, ültimatonsal sızıntı olarak ışığın kaçmasıyla çok ilgili değildir. Bu ültimaton enerjileri uzaya doğru; güneşin ergenlik dönemleri sürecinde gerçek bir enerji patlaması olarak, elektronsal birlikteliğin ve enerjinin maddi bir hale gelmesinin sürecine katılmak amacıyla kaçar.
\vs p041 9:2 Atomlar ve elektronlar çekim kuvvetine tabidir. Ültimatonlar, maddi çekimin karşılıklı ilişkisi olarak yerel çekime \bibemph{bağlı değildir}; ancak onlar bütünüyle, kâinat âlemlerinin tümünün evrensel ve ebedi döngüsüne ait dönüş olarak akım biçimindeki mutlak veya Cennet çekimine bağlıdır. Ültimatonsal enerji, yakın veya uzak maddi kütlelerin doğrusal veya doğrudan çekim etkisine uymamaktadır; fakat bu enerji her zaman, uçsuz bucaksız yaratımın büyük elips yörüngesinin döngüsü uyarınca dönüşünü gerçekleştirmektedir.
\vs p041 9:3 Sizin kendi güneş merkeziniz yıllık olarak, mevcut maddenin neredeyse yüz milyon tonunu merkezden yaymaktadır; bunun karşısında devasa güneşler, ilk bir milyar yıllık süreç olan öncül büyüme evreleri boyunca olağanüstü büyüklükte bir kütle yitirmiştir. Bir güneşin yaşamı, içsel ısısının en yüksek noktasına ulaşmasından sonra istikrara kavuşur; ve alt atomsal enerjiler serbest bırakılmaya başlar. Ve tam da bu eşik noktasında daha büyük olan güneşler kasılımsal titreşimleri almaktadır.
\vs p041 9:4 Güneşin yerleşik konuma gelmesi bütünüyle; inanılmaz sıcaklıklar tarafından karşılıklı olarak dengelenmiş devasa basınçlar biçiminde, çekim\hyp{}ısı mücadelesi arasındaki dengeye bağlıdır. Güneşlerin içsel gaz esnekliği, çeşitli maddelerin üst tabakaları idame ettirir; buna ek olarak çekim ve ısı dengede bulunduğu zaman, dışsal maddelerin ağırlığı temel ve içsel gazların sıcaklık basıncına tam anlamıyla eşittir. Daha genç olan yıldızların birçoğunda çekim yoğunlaşması sürekli artan içsel sıcaklıkları meydana getirir; ve içsel ısı yükseldikçe yüksek gaz rüzgârlarının iç X\hyp{}ışını basıncı o kadar yüksek bir hale gelir ki, merkezkaç etkisi ile birlikte bir güneş dış tabakalarını uzaya doğru atmaya başlayıp böylece çekim ve ısı arasındaki dengeyi yeniden kurar.
\vs p041 9:5 Sizin güneşiniz uzun bir zamandan beri; daha genç olan yıldızlarının birçoğu için devasa titreşimler oluşturan rahatsızlıklar biçimindeki, onun genişleme ve büzülme döngüleri arasında göreceli bir denge düzeyine erişmiştir. Sizin güneşiniz şu an, altıncı milyar yaşını geçmiştir. Güneşiniz, yirmi beş milyar yıldan daha fazla bir süreliğine mevcut etkinliği biçiminde ışımaya devam edecektir. Onun genç ve yerleşik konuma geçmiş faaliyetinin bir araya geldiği süreçler boyunca muhtemel bir biçimde, düşüşün kısmı bir etkin dönemini deneyimleyecektir.
\usection{10.\bibnobreakspace Yerleşik Dünyaların Kökeni}
\vs p041 10:1 En üst düzey titreşim durumunda veya onun yakınlarında, çeşitli yıldızlardan bazıları; birçoklarının nihai olarak sizin güneşinize ve onun etrafında dönen gezegenlerine benzeyeceği, alt sistemlere kaynaklık etme sürecindedir. Güneşiniz; Angona sistemi yörüngesi üzerinde güneşe yaklaşarak döndüğünde, güneşin dış yüzeyi maddenin sürekli tabakaları biçiminde gerçek akımları fışkırmaya başlamıştır. Bu süreç, bu sistemin güneşin en yakınına geldiği ana kadar olanca şiddetiyle devam etmiştir; güneş bütünlüğünün sınırlarına ulaşıldığında ve maddenin geniş bir zirvesine erişildiğinde, güneş sisteminin atası açığa çıkmıştır. Benzer koşullar altında etkileşimde bulunan bünyenin hareket yörüngesi içinde ona en yaklaştığı anda zaman zaman; bir güneşin dörtte biri veya üçte biri bile büyüklüğe sahip, bütüncül gezegenleri kendisine çeker. Bahse konu bu büyük ihraçlar, Jüpiter ve Satürn’e oldukça benzeyen âlemler olarak dünyaların alışılmamış belirli bulutsal türlerini oluştururlar.
\vs p041 10:2 Güneş sistemlerinin büyük bir çoğunluğu buna rağmen, sizinkinden oldukça farklı bir kökene sahip bulunmaktaydı; ve bu durum, gelgitsel\hyp{}çekim işleyiş biçimleri tarafından üretilmiş olanlar içinde doğruluk taşımaktadır. Fakat dünya inşasının hangi işleyiş biçiminden oluştuğundan bağımsız olarak, çekim her zaman yaratımın güneş sistem türünü üretmektedir. Bu üretim; gezegenler, uydular, alt uydular ve göktaşları ile birlikte merkezi bir güneş veya bir karanlık adanın oluşmasıdır.
\vs p041 10:3 Bireysel dünyaların fiziksel özellikleri büyük bir ölçüde; kökenin, gökbilimsel konumun ve fiziksel çevrenin türü tarafından belirlenmiştir. Uzay boyunca dönüşün yaşı, büyüklüğü ve sayısına ek olarak hız aynı zamanda belirleyici etmenlerdir. Gaz\hyp{}büzülmesi ve katı\hyp{}birikimi dünyaları dağlar tarafından belirlenmiş olup, öncül yaşamları boyunca çok küçük olmadıkları durumlarda su ve havadan oluşmuşlardır. Eriyik\hyp{}ayrışma ve çarpışmasal dünyalar zaman zaman geniş dağ sıralarına sahip değildir.
\vs p041 10:4 Bu yenidünyalarının hepsinin öncül çağları boyunca depremler sıklıkla görülmekte olup, onların hepsi büyük fiziksel rahatsızlıklar tarafından belirlenmektedir; bu durum özellikle, belirli bireysel güneşlerin öncül yoğunlaşma ve büzüşme zamanlarında geride bıraktıkları engin nebulasal halkalarından dünyaya gelmiştir. Urantia gibi çifte bir kökene sahip olan gezegenler, daha az şiddetli ve fırtınalı gençlik süreçlerinden geçerler. Böyle bir duruma rağmen sizin dünyanız; volkanlar, depremler, seller ve inanılmaz fırtınalar tarafından belirlenen kudretli karışıklıkların öncül bir fazını deneyimlemiştir.
\vs p041 10:5 Satania’nın kendisi Norlatiadek’in en dışta bulunan sisteminin yanında bulurken ve bu takımyıldızı mevcut an içinde Nebadon’un dış sınırlarını katetmekteyken; Urantia göreceli bir biçimde, Jerusem’den oldukça uzak bir biçimde ayrılmış halde tek bir istisnaya sahip olarak, güneş sisteminiz olan Satania’nın dış etekleri üzerinde tecrit edilmiş bir yerleşke konumlanmıştır. Mikâil’in bahşedilmişliğinin onurlu bir seviyeye ve asli evren amacına doğru olan yükselişine kadar, siz gerçek anlamıyla yaratılmışların en alt düzeyinde konumlanan unsurları arasında bulunmaktaydınız. En alt düzeyde bulunan gerçek anlamıyla en yüksek seviyeye erişirken, en sonda bulunan bünye ilk düzeye erişir.
\vs p041 10:6 [Nebadon Güç Merkezleri’nin Baş İdarecisi ile işbirliği halinde bir Baş Melek tarafından sunulmuştur.]
