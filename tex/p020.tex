\upaper{20}{Tanrı’nın Cennet Evlatları}
\vs p020 0:1 Tanri’nın Evlatları Orvonton’un aşkın evreninde faaliyet gösterirlerken, onlar şu üç genel başlık altında sınıflandırılır:
\vs p020 0:2 1.\bibnobreakspace Tanrı’nın Alçalış Halinde olan Evlatları.
\vs p020 0:3 2.\bibnobreakspace Tanrı’nın Yükseliş Halinde olan Evlatları.
\vs p020 0:4 3.\bibnobreakspace Tanrı’nın Kutsal Üçleme Haline Getirilmiş Evlatları.
\vs p020 0:5 Evlatlığın alçalan düzeyleri, kutsal ve doğrudan yaratıma ait olan kişilikleri içine alır. Fani yaratılmışlar biçiminde olduğu gibi yükseliş halinde olan evlatlar bu düzeye, evrim olarak bilinen yaratıcı işleyişi biçimi içine deneyimsel katılımlarıyla erişirler. Kutsal Üçleme Haline Getirilmiş Evlatlar, doğrudan Kutsal Üçleme kökenine ait olmasa da, Cennet Kutsal Üçlemesi tarafından bütünleşen tüm varlıkları içine alan birleşik kökenin bir topluluğunun üyesidir.
\usection{1.\bibnobreakspace Tanrı’nın Alçalış Halinde olan Evlatları}
\vs p020 1:1 Tanrı’nın Alçalış Halinde olan Evlatları’nın tümü yüksek ve kutsal kökenlere sahiplerdir. Onlar, zaman ve mekânın sistemleri ve dünyaları üzerinde hizmetin alçalış görevine adanmış olup; burada onlar, Tanrı’nın alçalış halinde olan evlatları biçimindeki, evrimsel kökenin alt düzey yaratılmışlarına ait olan Cennet erişiminin ilerleyişine olanak sağlarlar. Alçalış halinde olan Evlatlar’ın sayısız düzeyi içinde yedi tanesi bu anlatımın içinde tasvir edilecektir. Işığın ve Yaşamın Adası’nın merkezi üzerinde İlahiyatlar’dan kökenlerini alan bu Evlatlar, \bibemph{Tanrı’nın Cennet Evlatları} olarak adlandırılıp; onlar, bahse konu şu üç düzeyle bütünleşir:
\vs p020 1:2 1.\bibnobreakspace Yaratan Evlatlar olarak --- Mikâiller.
\vs p020 1:3 2.\bibnobreakspace Hakimane Evlatlar olarak --- Avonallar.
\vs p020 1:4 3.\bibnobreakspace Kutsal Üçleme Eğitmen Evlatları --- Daynallar.
\vs p020 1:5 Alçalan evlatlığın geride kalan şu dört düzeyi \bibemph{Tanrı’nın Yerel Evren Evlatları} olarak bilinir:
\vs p020 1:6 4.\bibnobreakspace Melçizedek Evlatları.
\vs p020 1:7 5.\bibnobreakspace Vorondadek Evlatları.
\vs p020 1:8 6.\bibnobreakspace Lanonandek Evlatları.
\vs p020 1:9 7.\bibnobreakspace Yaşam Taşıyıcıları.
\vs p020 1:10 Melçizedekler; bir yerel evren Yaratan Evlat’ın, Yaratıcı Ruhaniyet’in ve Yaratıcı Melçizedek’in ortak bir doğumudur. Vorondadekler ve Lanondadekler, bir Yaratan Evlat ve onun birliktelik halinde bulunduğu Yaratıcı Ruhaniyet tarafından mevcudiyet haline getirilmiştir. Vorondadekler en yaygın biçimiyle, Takımyıldızları Yaratıcıları olan En Yüksekler olarak bilinirler; Lanonandekler ise Düzen Egemenleri ve Gezegensel Prensler olarak adlandırılır. Yaşam Taşıyıcıları’nın üç katmanlı olan düzeyi, yetki alanının aşkın evrenine ait olan Zamanın Ataları’nın üçünden biriyle birliktelik halinde olan bir Yaratan Evlat ve Yaratıcı Ruhaniyet tarafından mevcut hale getirilmiştir. Fakat bu Tanrı’nın Yerel Evren Evlatları’nın doğaları ve etkinlikleri, yerel yaratımların olaylarıyla ilgili olan bu makaleler içinde daha yerinde bir biçimde tasvir edilmiştir.
\vs p020 1:11 Tanrı’nın Cennet Evlatları üç katmanlı kökene aittir: Öncül olan veya ismiyle adlandırıldığı gibi Yaratan Evlatlar, Kâinatın Yaratıcısı ve Ebedi Evlat tarafından mevcut hale getirilmiştir; ikincil olan veya ismiyle adlandırıldığı gibi Hakimane Evlatlar, Ebedi Evlat ve Sınırsız Ruhaniyet’in evlatlarıdır; Kutsal Üçleme Eğitmen Evlatları ise Yaratıcı, Evlat ve Ruhaniyet’in doğumudur. Hizmet, ibadet ve arzuyu dile getirme bakımından Cennet Evlatları bir bütündür; onların ruhaniyeti tek olup, onların faaliyetleri kesinlik ve nitelik bakımından özdeştir.
\vs p020 1:12 Zamanın Cennet düzeylerinin kutsal idareciler olarak onaylanmasıyla Cennet Evlatları’nın düzeyleri kendilerini; yaratanlar, hizmetkârlar, bahşediciler, hâkimler, eğitmenler ve gerçeği açığa çıkarıcılar biçimindeki kutsal yardımcılar olarak açığa çıkardılar. Onlar kâinat âlemlerinin tümü içinde; ebedi Ada’nın kıyılarından, bu anlatımlarda ortaya çıkarılmamış aşkın ve merkezi evrenlerde çok katmanlı hizmeti uygulayan zaman ve mekânın yerleşik dünyalarına kadar uzanan bir kapsamda hareket eder. Onlar, hizmetlerinin doğaları ve konumlarına bağlı olarak çeşitli biçimlerde düzenlenmişlerdir; fakat yerel bir evren içinde Hakimane ve Eğitmen Evlatlar, bu nüfuz alanı üzerinde idareyi sağlayan Yaratan Evlat’ın yönlendirmesi altında görevlerini yerine getirirler.
\vs p020 1:13 Yaratan Evlatlar, onların denetledikleri ve bahşedebildikleri kendi bireylerinin merkezinde olan ruhsal bir edinimi ellerinde bulundurur bir görünüme sahiptirler. Bu durum tıpkı Yaratan Evlat’ınızın kendi ruhaniyetini Urantia üzerindeki tüm fani beden üzerine yaymasıyla benzerlik gösterir. Her Yaratan Evlat, kendisine ait olan alan içindeki bahse konu bu ruhsal çekim gücüyle birlikte donatılmıştır; o, kendi nüfuz alanı içinde hizmet eden Tanrı’nın alçalış halinde olan her Evlat’ının her faaliyeti ve duygusunun bireysel olarak bilincine sahiptir. Bu durum; kâinat âlemlerinin tümü içerisinde nerede bulunurlarsa bulunsunlar, Ebedi Evlat’ın kendisini Cennet Evlatları’nın bütününe ulaştırmada, onlarla iletişimi geçirmede ve bu ilişkiyi sürdürmede yetkin hale getiren, onun mutlak olan ruhsal çekim gücünün yerel bir evren sureti biçimindeki kutsal bir yansımasıdır.
\vs p020 1:14 Cennet Yaratan Evlatları, hizmetin ve bahşedişin kendilerine ait alçalış görevlerinde sadece Evlatlar olarak hizmet etmemekte olup; onlar kendilerine ait olan bahşedişin süreçlerini tamamladıklarında, her biri kendi yaratımlarında bir evren Yaratıcısı olarak faaliyet gösterirlerken; Tanrı’nın Evlatları’nın diğerleri, Cennet Yaratıcısı’nın Yaratan Evlat’ının kâinat egemenliği için gezegensel bağlılığında ve onun iradesine adanan yaratım içinde sonuçlanan biçimiyle, Kâinatın Yaratıcısı’nın sevgi dolu yönetiminin istençli tanınması amacıyla gezegenleri bir bir kazanmak için tasarlanan bahşedici ve ruhsal canlanmanın hizmetine devam ederler.
\vs p020 1:15 Yedi Katmanlı bir Yaratan Evlat’da Yaratan ve yaratıcı, anlayışlı, cana yakın ve bağışlayıcı birliktelik içinde sonsuza kadar harmanlanmış bir haldedir. Yaratan Evlatlar biçimindeki Mikâil’in düzeyinin tümünün oldukça benzersiz bir biçimde olan onların doğalarının ve etkinliklerinin önemi, bu yazı dizisi içindeki diğer makalede ele alınacak iken; bu makaledeki anlatım başlıca, Hakimane Evlatlar ve Kutsal Üçleme Eğitmen Evlatları biçimindeki Cennet evlatlığının geride kalan iki düzeyiyle ilişkili biçimde olacaktır.
\usection{2.\bibnobreakspace Hakimane Evlatlar}
\vs p020 2:1 Her ne zaman Ebedi Evlat tarafından tasarlanmakta olan bir özgün ve mutlak kavramsallaşma Sınırsız Ruhaniyet tarafından algılanan sevgi dolu hizmetin yeni ve kutsal bir nihai amacıyla birlikte bütünleştiğinde, bir Cennet Hakimane Evladı biçimindeki Tanrı’nın özgün ve yepyeni bir Evlat’ı türetilmiş olur. Bu Evlatlar, Yaratan Evlatlar biçimindeki Mikâil düzeyine tezat bir biçimde Avonallar’ın düzeyini oluşturur. Bireysel biçimde yaratan olmasalar da onlar tüm görevlerinde Mikâiller ile yakın bir biçimde birliktelik halindedirler. Avonallar; tüm ırkların içinde, tüm dünyalar için ve tüm âlemler üzerinde zaman\hyp{}mekân alanlarının hâkimleri biçiminde gezegensel yardımcılar ve yargıçlardır.
\vs p020 2:2 Muhteşem kâinat içinde Hakimane Evlatlar’ın toplam sayısının yaklaşık bir milyar olduğuna inanmak için sebeplerimiz bulunmaktadır. Onlar; tüm âlemlerin hizmetlerinden seçilen deneyimli Avonallar tarafından oluşturulan Cennet üzerindeki yüce kurulları vasıtasıyla yönlendirilmekte olan bir özerk düzeydir. Fakat onlar yerel bir evrende görevlendirildiğinde ve onun üzerine atandığında, bu nüfuz alanının Yaratan Evladı’nın yönlendirmesi altında hizmet vermektedir.
\vs p020 2:3 Avonallar, yerel evrenlerin bireysel gezegenleri için hizmetin ve bahşedişin Cennet Evlatları’dır. Buna ek olarak her Avonal Evladı ayrıcalıklı bir kişiliğe sahip olduğundan ve onlardan herhangi birinin bile bir diğerine benzememesinden dolayı, onların görevi bireysel olarak kendilerine ait kısa süreli ikamet ettikleri âlemlerde benzersizdir. Bu âlemlerde onlar sık sık, fani bedene benzeyen bir suretin içinde ete kemiğe bürünüp, zaman zaman evrimsel dünyalar üzerinde dünyevi annelerden doğarlar.
\vs p020 2:4 Daha yüksek olan idari düzeyler üzerinde hizmetlerine ek olarak Avonallar, yerleşik dünyalar üzerinde şu üç katmanlı faaliyete sahiptir:
\vs p020 2:5 1.\bibemph{ Yargısal Faaliyetler}. Onlar gezegensel yazgı döneminin bitimine doğru faaliyet gösterirler. Zaman içinde bu tür görevlerin sayıca yüzlercesi her bireysel dünya üzerinde yerine getirilebilir; ve buna ek olarak onlar, uyku halinde bulunan varlığını sürdürmeye çalışanların özgürleştiricileri biçimindeki yazgı dönemi sonlandırıcıları olarak sayısız bir biçimde, aynı veya diğer dünyalara hareket edebilir.
\vs p020 2:6 2.\bibnobreakspace \bibemph{Hakimane Görevler}. Bu türün gezegensel bir teftişi genellikle bir bahşedici Evlat’ın varışının öncesinde ortaya çıkar. Böyle bir görev üzerinde bir Avonal, fani doğumla ilgisi olmayan ete kemiğe bürünmenin bir işleyişi tarafından âlemin bir erişkini olarak ortaya çıkar. Bu ilk ve olağan hakimane ziyaretini takiben Avonallar, bahşedilen Evlat’ın ortaya çıkışından önce ve sonra aynı gezegen üzerinde hakimane bir yetkinlik içerisinde tekrar eden biçimlerde hizmet halinde olabilir. Bu ek hakimane görevlerinin üzerine bir Avonal, görünebilen ve maddi bir biçimde açığa çıkabilir veya çıkmayabilir; fakat onların hiçbiri yardıma ihtiyaç duymayan bir bebek olarak dünyaya getirilmeyeceklerdir.
\vs p020 2:7 3.\bibnobreakspace \bibemph{Bahşetme Görevleri}. Avonal Evlatların hepsi, en azından bir kere, kendilerini bazı fani ırkı üzerinde bazı evrimsel dünyalar üzerinde bahşedebilirler. Yargısal ziyaretler sayısız derecede ve hakimane görevler çoklu olabilir, fakat her gezegen üzerinde tek bir bahşedilen Evlat bulunmaktadır. Bahşedilmiş Avonallar, Urantia üzerinde ete kemiğe bürünen Nebadonlu Mikâil olarak kadın cinsiyetinden doğmuştur.
\vs p020 2:8 Hakimane ve bahşedici vazifeleri üzerinde hizmet edebilecek Avonal Evlatları, görevleri bakımından sayıca kısıtlayacak hiçbir sınır bulunmamaktadır; fakat genellikle bahse konu deneyim yedi kez kat edildiğinde, bu tür hizmeti daha az yerine getirenleri gözeten bir askıya alma durumu söz konusudur. Çoklu bahşediş deneyiminin bu Evlatları bunun sonucunda bir Yaratan Evladı’nın yüksek bireysel kuruluna atanıp, bunun sonucunda kâinat olaylarının yönetiminde katılımcılar haline gelir.
\vs p020 2:9 Yerleşik dünyalar üzerinde ve onlar için kendilerine ait olan tüm görevlerinde Hakimane Evlatlar, Melçizedekler ve başmelekler biçimindeki yerel evren yaratımlarının iki düzeyi tarafından desteklenirken; bahşedici görevleri üzerinde ise onlar aynı zamanda, yerel yaratımlar içinde benzer bir kökene ait olan Berrak AkşamYıldızları tarafından eşlik edilir. Her gezegensel çabada Avonallar olarak ikincil Cennet Evlatları, hizmetin kendilerine ait yerel evreninde Yaratan Evlat biçimindeki öncül bir Cennet Evladı’nın bütüncül gücü ve yönetimi tarafından desteklenir. Yerleşik alanlar üzerinde tüm niyetler ve amaçlar için onların görevleri, bir Yaratan Evladı’nın hizmetinin fani yerleşimin bu tür dünyaları üzerinde olması gerektiği gibi etkin ve kabul edilebilir bir niteliktedir.
\usection{3.\bibnobreakspace Yargısal Faaliyetler}
\vs p020 3:1 Avonallar Hakimane Evlatlar olarak bilinirler, çünkü onlar zamanın dünyalarının ardışık yazgı dönemlerinin yargıçları biçiminde âlemlerin yüksek hâkimleridir. Onlar, uyku halinde bulunan varlığını devam ettirmeye çalışan varlıkların uyanması üzerinde iradeye sahip olup; bu alan üzerinde karar için yargıya varıp; askıya alınan adaletin bir yazgı dönemini tamamlayıp; şartlı bağışlamanın bir çağının emrini uygulayıp; yeni bir yazgı döneminin görevleri için gezegensel hizmetin mekân yaratılmışlarını yeniden görevlendirip; bunların tümü sonucunda ise, görevlerini tamamlamaları üzerine kendilerine ait yerel evrenin yönetim merkezlerine geri dönerler.
\vs p020 3:2 Onlar bir çağın nihai sonları üzerinde bir yargıya varmak için karara oturduklarında, Avonallar evrimsel ırkların nihai sonlarının hükmünü verdiklerinde; onlar her ne kadar bireysel yaratılmışların kimliğini ortadan kaldıran yargılara varsalar bile, bu kararları yine de bireysel olarak uygulamazlar. Bu doğanın kararları, bir aşkın evrenin yönetim mercileri dışında hiçbir kimse tarafından uygulanamaz.
\vs p020 3:3 Bir Cennet Avonalı’nın gezegensel devamlılığın yeni bir çağının başlatılması ve bir yazgı döneminin tamamlanması amacıyla bir evrimsel dünya üzerine varışı, bir bahşedilme veya hakimane görevi olmak zorunda değildir. Hakimane görevleri zaman zaman ve bahşedilme görevleri ise her zaman ete kemiğe bürünme şeklindedir; bu durum ise, her bu tür bir görevde Avonallar’ın bir gezegen üzerinde kelimenin tam anlamıyla maddi bir biçimde hizmet etmesidir. Onların diğer ziyaretleri “işleyişseldir,” ve bu yetkinlik içerisinde bir Avonal gezegensel hizmet için ete kemiğe büründürülmemiştir. Eğer bir Hakimane Evlat sadece bir yargı dönemi hâkimi olarak geliyorsa, o âlemin maddi yaratılmışları için görünebilen bir ruhsal varlık olarak bir gezegene ulaşır. Bu tür işleyişsel ziyaretler, bir yerleşik dünyanın uzun tarihi içerisinde tekrar eden biçimlerde ortaya çıkar.
\vs p020 3:4 Avonal Evlatları, hakimane ve bahşediş deneyimleri öncesinde gezegensel yargıçlar olarak hareket edebilir. Bu görevler üzerinde bununla birlikte, ete kemiğe bürünen Evlat geçmekte olan gezegensel çağının yargısına varacaktır; bu durum, fani bedenin sureti içerisinde bahşedilmişin bir görevi üzerinde bir Yaratan Evlat’ın ete kemiğe büründüğündeki faaliyetine benzerdir. Bir Cennet Evladı evrimsel bir dünyayı ziyaret ettiğinde, ve onun varlıklarından biri haline geldiğinde; onun mevcudiyeti, bir yazgı dönemini tamamlayıp, bir âlem yargısını oluşturur.
\usection{4.\bibnobreakspace Hakimane Görevler}
\vs p020 4:1 Bir bahşedilmiş Evlat’ın gezegensel ortaya çıkışının öncesinde bir yerleşik dünya, genellikle bir Cennet Avonalı tarafından hakimane bir görev dâhilinde ziyaret edilir. Eğer bu ziyaret öncül bir hakimane teftişi ise, Avonal her zaman bir maddi varlık olarak ete kemiğe bürünecektir. O, kendisine ait olan zamanın ve neslin fani yaratılmışları için bütünüyle görünebilen ve onlarla fiziksel ilişki halindeki bir varlık biçiminde fani ırkların uçsuz bucaksız bir erkek cinsiyeti olarak gezegensel görev üzerinde ortaya çıkar. Bir hakimane ete kemiğe bürünme süreci boyunca, Avonal Evlatları’nın yerel ve kâinatsal ruhsal kuvvetleriyle olan ilişkisi tamamlanmış ve bölünemez bir niteliktedir.
\vs p020 4:2 Bir gezegen, bir bahşedilmiş Evlat’ın ortaya çıkmasından önce ve sonra birçok hakimane ziyaretleri deneyimleyebilir. Burası, yazgı dönemi hâkimler olarak hareket eden aynı veya diğer Avonallar tarafından birçok kez ziyaret edilebilir; fakat, yargının bu tür işleyişsel görevleri ne bahşedişsel ne de hakimanedir, buna ek olarak Avonallar hiçbir zaman bu tür süreçler içerisinde ete kemiğe bürünmezler. Bir gezegen tekrar eden hakimane görevleri tarafından kutsanmış olsa bile, Avonallar her zaman fani özüne ait olan ete kemiğe bürünmeye başvurmazlar. Buna ek olarak onlar fani bendenin sureti içinde hizmet etse bile, her zaman âlemin erişkin varlıkları olarak ortaya çıkarlar; onlar bir kadın cinsiyetinden doğmamaktadırlar.
\vs p020 4:3 Hali hazırda Urantia, hakimane bir görev üzerinde ete kemiğe bürünmekle görevlendirilmiş bir Avonal tarafından ziyaret edilebilir; fakat Cennet Evlatları’nın gelecekteki ortaya çıkışları bakımından, “bu tür ziyaretlerin biçimini ve zamanlamasını cennette bulunan melekler dahi” bilmemektedir. Bu durumun sebebi, bir Mikâil\hyp{}bahşedilmiş dünyasının bir Üstün Evlat’ının kişisel ve bireysel vesayeti haline gelmesi; ve böylelikle bu dünyanın bütünüyle kendi tasarıları ve idaresine bağımlı bir nitelik kazanmasıdır. Buna ek olarak sizin dünyanızla birlikte bu durum, Mikâil’in geri dönmek için verdiği söz tarafından daha karmaşık bir hal almıştır. Nebadonlu Mikâil’in Urantia’daki kısa süreli ikamesiyle ilgili olan yanlış anlamalardan bağımsız olarak tek bir şey kesin olarak doğrudur; bu ise onun sizin dünyanıza geri dönmek için verdiği sözdür. Bu ihtimal bakımından sadece zaman, Urantia üzerinde Tanrı’nın Cennet Evlatları’nın ziyaretlerinin gelecek düzeylerini açığa çıkarabilir.
\vs p020 4:4 Urantia hiçbir zaman bir hakimane görevi üzerinde bir Avonal Evladı’na ev sahipliği yapmamıştır. Urantia yerleşik dünyaların genel tasarımını takip etmiş olsaydı, Âdem’in zamanı ve Hazreti Mikâil’in bahşedilmişi arasında bir zaman zarfında bir hakimane göreviyle kutsanmış olacaktı. Fakat Cennet Evlatları’nın düzenli takibi bütüncül olarak, bin dokuz yüz yıl öncesinde Yaratan Evlat’ınızın dönemsel bahşedilişi üzerinde onun ortaya çıkmasıyla birlikte dengesiz hale gelmiştir.
\vs p020 4:5 Hali hazırda Urantia, hakimane bir görev üzerinde ete kemiğe bürünmekle görevlendirilmiş bir Avonal tarafından ziyaret edilebilir; fakat Cennet Evlatları’nın gelecekteki ortaya çıkışları bakımından, “bu tür ziyaretlerin biçimini ve zamanlamasını cennette bulunan melekler dahi” bilmemektedir. Bu durumun sebebi, bir Mikâil\hyp{}bahşedilmiş dünyasının bir Üstün Evlat’ının kişisel ve bireysel vesayeti haline gelmesi; ve böylelikle bu dünyanın bütünüyle kendi tasarıları ve idaresine bağımlı bir nitelik kazanmasıdır. Buna ek olarak sizin dünyanızla birlikte bu durum, Mikâil’in geri dönmek için verdiği söz tarafından daha karmaşık bir hal almıştır. Nebadon’un Mikâili’nin Urantialı kısa süreli ikamesiyle ilgili olan yanlış anlamalardan bağımsız olarak tek bir şey kesin olarak doğrudur; bu ise onun sizin dünyanıza geri dönmek için verdiği sözdür. Bu ihtimal bakımından sadece zaman, Urantia üzerinde Tanrı’nın Cennet Evlatları’nın ziyaretlerinin gelecek düzeylerini açığa çıkarabilir.
\usection{5.\bibnobreakspace Tanrı’nın Cennet Evlatları’nın Bahşedilişi}
\vs p020 5:1 Ebedi Evlat Tanrı’nın Ebedi Sözü’dür. Ebedi Evlat, onun ebedi Yaratıcısı’nın sınırsız ve mutlak olan “öncül” düşüncesinin kusursuz bir dışavurumudur. Bu Özgün Evlat’ın bir kusursuz uzantısı veya bir kişisel sureti, fani ete kemiğe bürünmenin bir bahşedilme görevine başlarsa, kutsal olan “Söz’ün bedene bürünmesi” kelimenin tam anlamıyla gerçek haline gelir; ve böylece Söz hayvan kökenli alt düzey varlıkları arasında ikamet eder bir nitelik kazanır.
\vs p020 5:2 Urantia üzerinde, bir Evlat’ın bahşedilişinin amacının bir şekilde Kâinatın Yaratıcısı’nın tutumu üzerinde etkide bulunmak için olduğuna dair yaygın bir inanç bulunmaktadır. Fakat sizin aydınlanmanız bu yargının doğru olmadığını göstermelidir. Avonal ve Mikâil Evlatları’nın bahşedilişleri, bu Evlatları zaman ve mekânın gezegenleri ve insanlarının yöneticileri ve hâkimleri olarak güvenilir ve anlayışlı hale getirmek için tasarlanan deneyimsel sürecin gerekli bir parçasıdır. Yedi katmanlı bahşedişin varlıksal süreci, tüm Cennet Yaratan Evlatları’nın yüce hedefidir. Buna ek olarak tüm Hakimane Evlatlar, öncül Yaratan Evlatları’nı ve Cennetin Ebedi Evladı’nı fazlasıyla tanımlayan hizmetin bu aynı ruhaniyeti tarafından harekete geçirilmektedir.
\vs p020 5:3 Cennet Evladı’nın bazı düzeyleri, bahse konu alan üzerinde tüm olağan insan varlıklarının akıllarında Düşünce Düzenleyicileri’nin ikamesini mümkün kılmak amacıyla her fani yerleşik dünya üzerinde bahşedilmelidir; çünkü Düzenleyiciler, Gerçekliğin Ruhaniyeti’nin tüm beden üzerine yayılmasına kadar \bibemph{tüm} gerçek insan varlıkları için gelmemektedir; buna ek olarak Gerçekliğin Ruhaniyeti’nin bu gönderimi, bir evrimleşen dünya üzerinde fani bahşedişin görevini başarıyla uygulayan bir Cennet Evladı’nın evren yönetim merkezine dönmesine bağlıdır.
\vs p020 5:4 Bir yerleşik gezegenin uzun tarih süreci boyunca birçok yargı dönemi hâkimleri ortaya çıkacak olup, birden fazla hakimane görevi meydana gelecektir; fakat olağan bir biçimde, bir bahşedilmiş Evlat sadece bir kere bu alan içinde hizmet edecektir. Her yerleşik dünyanın, doğumundan ölümüne kadar bütüncül fani hayatı yaşamak için gelmekte olan bir bahşedilmiş Evlat’a sahip olması tek gerekli koşuldur. Elinde sonunda ruhsal düzeyden bağımsız bir biçimde her fani\hyp{}yerleşik dünya, bir Yaratan Evlat’ın kendi fani bahşedilişini üzerinde seçtiği her yerel evren içinde bir gezegen dışında, bir bahşedilmiş görevi üzerinde bir Hakimane Evladı’na ev sahipliği yapması bakımından nihai bir biçimde sonlandırılmıştır.
\vs p020 5:5 Bahşedilmiş Evlatlar ile ilgili daha geniş bir anlayışa sahip olmakla birlikte, Nebadon’un tarihi içinde Urantia için neden bu kadar önem atfedildiğini kavrayacaksınız. İçinde bulunduğunuz küçük ve önemsiz olan gezegen yerel evren öneminin bir parçasıdır, bu durumun nedeni yalın bir ifadeyle onun Nasıralı İsa’nın fani ana dünyası olmasıdır. Yaratan Evlat’ınızın kesin ve zafer dolu bahşedilişinin yarattığı imgelem, Mikâil’in Nebadon evreninin kişisel egemenliğini kazandığı mücadele alanını simgeler.
\vs p020 5:6 Bir Yaratan Evlat’ın yerel evreninin yönetim merkezinde, özellikle onun fani bahşedilişinin tamamlanmasından sonra, zamanının büyük bir kısmını Hakimane Evlatları ve diğerleri biçimindeki birliktelik içinde bulundukları Evlatlar’ın okullarında danışmanlık ve eğitmenlik görevlerinde geçirir. Sevgi ve sadakat içerisinde, şefkatli bağışlama ve sevecen düşünceyle birlikte bu Hakimane Evlatları kendilerini mekânın dünyaları üzerine bahşeder. Buna ek olarak bu gezegensel hizmetler hiçbir biçimde Mikâiller’in fani bahşedilmişlerinin altında bulunan bir konumda değildir. Yaratan Evlat’ınızın kendi nihai serüveni için seçmiş olduğu âlemin yaratım deneyimi bakımından olağan olmayan talihsizliklere sahip olduğu bir gerçektir. Fakat hiçbir gezegen bir Yaratan Evladı’nın bahşedilmesini onun ruhsal tedavisi üzerinde etkide bulunması amacıyla gerektirecek kadar böyle bir konumda bulunmamıştı. Bahşedilmişlerin topluluğu içerisinde herhangi bir Evlat eşit bir derecede bu görev için yeterli bir konumda olabilirdi; çünkü onların görevlerinin tümü bakımından yerel bir evrenin dünyaları üzerinde Hakimane Evlatlar, Yaratan Evlat biçimindeki onların Cennet kardeşinin olması gerektiği gibi, tıpkı onun kadar kutsal bir biçimde etkin ve bütünüyle akılcıdır.
\vs p020 5:7 Her ne kadar herhangi bir felaketin olasılığı, bu Cennet Evlatları’nın bahşedilmesinin ete kemiğe bürünme sürecinde onların başına gelse de; ben bahşedilmenin herhangi bir görevi üzerinde, bir Yaratan veya Hakimane Evladı’nın herhangi birinin başarısızlığı veya yükümlülüğünü yerine getirememesinin kaydına rastlamadım. Onların ikisinin de ait olduğu köken, başarısızlığa uğramayacak kadar mutlak kusursuzluğa yakındır. Gerçekte onlar, beden ve kanın fani yaratılmışları haline kelimenin tam anlamıyla gelme ve böylece benzersiz yaratılmış deneyimini etme biçimindeki tehlikenin farkınadırlar; fakat benim gözlemim kapsamında onlar, her zaman bu tür zorluğun üstesinden gelmeyi başarmış bir durumdadırlar. Onlar bahşedilmişliğin görevinin hedeflerini yerine getirmekte hiçbir zaman başarısız olmamışlardır. Nebadon boyunca onların bahşedilme ve gezegensel hizmetinin hikâyesi, yerel evreninizin tarihi içinde en soylu ve en büyüleyici bölümü oluşturur.
\usection{6.\bibnobreakspace Fani\hyp{}Bahşedilme Süreçleri}
\vs p020 6:1 Bir Cennet Evladı’nın, bahşedilmiş gezegen üzerinde bir anne tarafından dünyaya gelme biçimindeki, bir bahşedilmiş Evlat olarak fani ete kemiğe bürünmesi için hazır hale gelmesinin işleyiş biçimi kâinatsal bir gizemdir. Buna ek olarak bu Sonarington’un işleyiş biçimini tespit etmek için harcanacak herhangi bir çabanın kesin başarısızlıkla karşılaşacak olması eli mahkûm bir durumdur. Nasıralı İsa’nın fani yaşamının bu ulvi bilgisinin sizin ruhlarınıza doğru yayılmasına izin verin; fakat aynı zamanda, Nebadonlu Mikâil’in bu gizemli ete kemiğe bürünmesinin nasıl meydana geldiği hakkında gereksiz varsayımlar üzerinde herhangi bir düşünceyi boş yere harcamayın. İzin verin hepimiz bu bilgi ve güvence içinde, bu tür erişimlerin kutsal doğa için mümkün olduğunun memnuniyetini yaşayalım; ve böyle bir olgusallığı etkilemek için kutsal bilgelik tarafından işleyiş biçiminin nasıl uygulandığı hakkında faydasız olan varsayımlar üzerinde boş yere vakit harcamayalım.
\vs p020 6:2 Fani\hyp{}bahşedilişin bir görevi üzerinde bir Cennet Evladı her zaman bir kadından dünyaya gelmiş olup, Urantia üzerinde İsa’nın gerçekleştirdiği gibi, bu âlemin erkek bir çocuğu olarak büyümesini tamamlamıştır. Yüce hizmetin bu Evlatları’nın hepsi, tıpkı bir insanoğlunun yerine getirdiği gibi, bebeklikten gençlik boyunca erişkinliğe geçmişlerdir. Her açıdan onlar dünyaya geldikleri ırkın fanilerine benzer bir halde oluşum içerisindedirler. Onlar, hizmet ettikleri âlemlerin çocuklarının yaptıkları gibi Yaratıcı’ya yakarışta bulunurlar. Maddi bir bakış açısından, bu insan\hyp{}kutsal Evlatlar bir istisna dışında olağan bir biçimde yaşarlar. Onlar, kısa süreli olan ikamelerinin dünyaları üzerinde kendilerinden olan bir doğum bırakmamaktadırlar; gerçekte bu durum, Cennet bahşedilmiş Evlatları’nın tüm düzeyleri üzerinde uygulanan kâinatsal bir kısıtlamadır.
\vs p020 6:3 Sizin dünyanız üzerinde İsa bir marangozun oğlu olarak çalıştığı gibi, diğer Cennet Evlatları kendilerine ait olan bahşedilmiş gezegenleri üzerinde çeşitli yetkinlikler içinde çaba harcayacaklardır. Zamanın evrimsel gezegenlerinin herhangi biri üzerinde, kendi bahşedilmişliklerinin süreci içinde herhangi bir Cennet Evladı tarafından çalışılmamış bir mesleğin bulunduğuna dair düşünceye sahip olamazsınız.
\vs p020 6:4 Bir Bahşedilmiş Evlat fani hayatın yaşanmışlığını deneyimi üzerinde üstünlüğe sahip olduğu, kendisine ait olan ikamet halindeki Düzenleyici’si ile birlikte uyumun kusursuzluğuna eriştiği zaman; bunun üzerine o, beden içinde onun kardeşlerinin ruhları için ilhamda bulunmak ve onların akıllarını aydınlatmak amacıyla tasarlanan gezegensel görevinin bu kısmına başlamış olur. Eğitmenler olarak bu Evlatlar ayrıcalıklı bir biçimde, kendilerinin kısa süreli olan ikametlerinin dünyaları üzerinde fani ırkların ruhsal aydınlanmasına adanmışlardır.
\vs p020 6:5 Mikâil ve Avonallar’ın fani\hyp{}bahşedilmiş süreçleri birçok açıdan karşılaştırılabilir olsa da, onlar bütüncül olarak özdeş değillerdir. Hiçbir zaman bir Hakimane Evlat, “Her kim Evlat’ı görmüşse gerçekte Yaratıcı’nın kendisini gözlemlemiştir” biçiminde Yaratan Evladı’nızın beden halinde ve Urantia’da bulunduğu zaman zarfında yaptığı gibi bir bildiride bulunmaz. Fakat bir bahşedilmiş Avonal’ı; “Her kim beni görmüşse Tanrı’nın Ebedi Evladı’nı gözlemlemiştir” biçiminde ilanda bulunur. Hakimane Evlatlar, ve Kâinatın Yaratıcısı’nın doğrudan bir soyuna aittirler; ne de Yaratıcı’nın iradesine bağlı bir biçimde bireysel bir ete kemiğe büründürmeyi gerçekleştirirler; onlar her zaman kendilerini, Cennetin Ebedi Evladı’nın iradesine bağlı olan Cennet \bibemph{Evlatları} olarak bahşederler.
\vs p020 6:6 Yaratan veya Hakimane biçimindeki bahşedilmiş Evlatlar ölümün kapılarından içeriye girdiklerinde, üçüncü gün içerisinde tekrar ortaya çıkarlar. Fakat onların, bin dokuz yüz yıl önce sizin dünyanız üzerinde kısa süreli olarak ikamet eden Yaratan Evlat tarafından karşılaşılan acıklı son ile her zaman karşılaştıklarına dair bir fikri yürütmemelisiniz. Nasıralı İsa tarafından geçirilen bu olağandışı ve nadiren ortaya çıkan zalim deneyim, Urantia’nın yerel olarak “haçın dünyası” olarak bilinmesine sebep oldu. Böyle bir insanlık dışı davranışı bir Tanrı’nın Evladı’yla ilişkilendirmek gerekli bir neden değildir; gezegenlerin geniş bir çoğunluğu, onların kendilerine ait fani süreçlerini bitirmelerine izin vererek, çağlarını sonlandırarak, uyku halinde bulunan varlığını devam ettiren varlıkların yargısına vararak ve yeni bir yazgı dönemini şiddetli bir ölümle onlar üzerinde uygulamayarak başlatmasıyla, onların daha düşünceli bir biçimde karşılanmasını sağladı. Bir bahşedilmiş Evlat ölümle karşılaşmalı ve âlemlerin fanilerinin mevcut deneyimlerinin süreçsel bütünlüğünü takip etmelidir; fakat bu ölümün şiddetli veya olağanın dışında olması kutsal tasarının bir gerekliliği değildir.
\vs p020 6:7 Bahşedilmiş Evlatlar şiddet tarafından ölüme gönderilmediklerinde onlar gönüllü olarak hayatlarını bırakıp ölüm kapıları boyunca hareket ederler. Bu süreç “katı adaletin” veya “kutsal gazabın” gerekliliklerini tatmin etmek için değil; fakat bunun yerine, fani deneyimin gezegenleri üzerinde bir yaratılmışın yaşamının yaşanacağı gibi bunu oluşturan her şey içinde, kişisel deneyimin ve ete kemiğe bürünmenin sürecinin “kadehi içme” biçimindeki bahşedilişi tamamlaması içindir. Bahşedilme bir gezegensel ve kâinatsal gereklilik olup, fiziksel ölüm bir bahşediliş görevinin gerekli bir parçasından daha fazlası değildir.
\vs p020 6:8 Fani ete kemiğe bürünme süreci tamamlandığında, hizmetin Avonal’ı Cennet’e ilerleyip, Kâinatın Yaratıcısı tarafından kabul edilip, bunun sonrasında yerel evren görevine geri dönüp, ve son olarak ise Yaratan Evlat tarafından tanınır. Bunun üzerinde Bahşedilmiş Avonal ve Yaratan Evlat, kendilerine ait olan birleşik bir haldeki Gerçekliğin Ruhaniyeti’ni, bahşedilmiş dünya üzerinde ikamet eden fani ırkların kalpleri içinde faaliyet göstermesi için gönderir. Yerel bir evrenin bu egemenlik öncesi çağlarında, bahse konu iki Evlat’ın bu birleşik ruhaniyeti, Yaratıcı Ruhaniyet tarafından uygulanır. Bu durum, bir Mikâil’in yedinci bahşedişini takip eden yerel evren çağlarını tanımlayan Gerçekliğin Ruhaniyeti’nden bir şekilde farklılık gösterir.
\vs p020 6:9 Bir Yaratan Evlat’ının nihai bahşedilişinin tamamlanması üzerine, bahse konu yerel evrenin tüm Avonal bahşedilmiş dünyaları üzerine daha önceden gönderilen Gerçekliğin Ruhaniyeti, egemen Mikâil’in ruhaniyeti haline kelimenin tam anlamıyla daha fazla gelerek doğası bakımından değişikliğe uğrar. Bu olgular bütünü, Mikâil’in fani bahşedilmiş gezegeni üzerinde hizmet etmek için Gerçekliğin Ruhaniyeti’nin özgürleşmesiyle birlikte eş zamanlı olarak ortaya çıkar. Bunun sonucunda bir Hakimane bahşedilişi tarafından onurlandırılan her dünya, bahse konu Hakimane Evlat ile bütünlük halinde yedi katmanlı olan Yaratan Evlat tarafından aynı ruhaniyet Huzur Sağlayıcısı’na erişecektir. Eğer her yerel evren Egemeni kişisel olarak kendisinin bahşedilmiş Evladı biçiminde ete kemiğe büründürmeyi gerçekleştirmiş olsaydı, o kendi içinde bu erişime ulaşmış olacaktı.
\usection{7.\bibnobreakspace Kutsal Üçleme Eğitmen Evlatları}
\vs p020 7:1 Bu son derece kişisel ve ruhsal olan Cennet Evlatları, Cennet Kutsal Üçlemesi tarafından mevcudiyet haline getirilmişlerdir. Onlar Havona içinde Daynallar'ın düzeyi olarak bilinirler. Orvonton içinde ise onlar, soylarının ait olduğu kimlik sebebiyle adlandırıldıkları biçimiyle Kutsal Üçleme Eğitmen Evlatları olarak kayıt altına alınmışlardır. Salvington üzerinde onlar zaman zaman Cennet Ruhsal Evlatları olarak isimlendirilirler.
\vs p020 7:2 Sayıları bakımından Eğitmen Evlatlar sürekli bir biçimde artış halindedir. En son yayınlanan kâinatsal nüfus sayımı, merkezi ve aşkın evrenler içinde faaliyet gösteren Kutsal Üçleme Evlatları’nın bu sayısının yirmi bir milyardan biraz daha fazla olduğunu göstermiştir; buna ek olarak bu rakam, mevcut haldeki tüm Kutsal Üçleme Eğitmen Evlatları’nın üçte birinden daha fazlasını içine alan Cennet yedek birliklerini içine almamaktadır.
\vs p020 7:3 Evlatlığın Daynal düzeyi, yerel veya aşkın evren idarelerinin organik bir parçası değildir. Onların üyeleri ne yaratıcılar, ne düzelticiler; ne yargıçlar ne de idarecilerdir. Onlar, ahlaki aydınlanmanın ve ruhsal gelişimle iniltili olan evren idaresiyle yakın bir ilişki içinde değillerdir. Onlar, tüm âlemlerin ruhsal uyanışına ve ahlaki yönlendirilişine adanan varlıklar olarak kâinatsal eğitimcilerdir. Onların hizmeti, Sınırsız Ruhaniyet’in kişilikleriyle birlikte içten bir biçimde karşılıklı ilişki içinde olup; onlar, yaratılmış varlıkların Cennet yükselimiyle yakın bir biçimde birliktelik halindedir.
\vs p020 7:4 Kutsal Üçleme’nin bu Evlatları, üç Cennet İlahiyatı’nın bir araya gelmiş doğalarının bir parçasıdır; fakat Havona içinde onlar, Kâinatın Yaratıcısı’nın doğasını daha fazla bir biçimde yansıtan bir görünüme sahiptirler. Yerel yaratımlar içinde onlar, Sınırsız Ruhaniyet’in karakterini açıklığa kavuşturan bir biçimde ortaya çıkarlarken; aşkın\hyp{}evrenler içinde ise onlar, Ebedi Evlat’ın doğasını tasvir eden bir görünüme sahiptirler. Tüm evrenler içinde onlar, bilgeliğin hizmetinin ve sağduyusunun somutlaşmış halidir.
\vs p020 7:5 Onların Cennet kardeşleri olan Mikâiller ve Avonallar’ın aksine Kutsal Üçleme Eğitmen Evlatları, merkezi evren içinde öncül bir eğitim görmezler. Onlar doğrudan bir biçimde aşkın\hyp{}evrenlerin yönetim merkezlerine gönderirler; ve buradan onlar, birtakım yerel evrenler içinde hizmet vermek için görevlendirilirler. Bu evrimsel âlemler için hizmetlerinde onlar, bir Yaratan Evlat’ın ve onunla birliktelik içinde bulunan Hakimane Evlatlar’ın bir araya gelen ruhsal etkisini kullanırlar; çünkü Daynallar kendileri içinde ve kendilerine ait olan bir biçimde bir ruhsal çekim gücünü ellerinde barındırmazlar.
\usection{8.\bibnobreakspace Daynallar’ın Yerel Evren Hizmeti}
\vs p020 8:1 Cennet Ruhsal Evlatları eşi benzeri olmayan Kutsal Üçleme kökenli varlıklar olup; sadece Kutsal Üçleme yaratılmışları, çifte kökenli olan âlemlerin faaliyetiyle oldukça bütüncül bir biçimde birliktelik halindedir. Onlar sevgi dolu bir biçimde, ruhsal varlıkların alt düzeyleri ve fani yaratılmışları için eğitimsel hizmete adanmışlardır. Onlar, çalışmalarına yerel sistemlerde başlayıp; deneyimleri ve kazanımları uyarınca, yerel evrenin en yüksek görevine uzanan takımyıldız hizmeti boyunca içsel bir doğrultuda ilerlemektedirler. Yetkinlikleri üzerine onlar, yerel evren hizmetlerini temsil eden ruhsal elçiler haline gelebilirler.
\vs p020 8:2 Nebadon içindeki Eğitmen Evlatlar’ın kesin sayısını bilmemekteyim; fakat orada onların binlercesinin bulunduğunu ifade edebilirim. Melçizedek okulları içindeki bölüm başlarının birçoğu bu düzeye ait iken; Salvington Üniversitesi’ni düzenli bir biçimde oluşturan bir araya gelmiş bahse konu görevliler, bu Evlatlar’ın yüz bininden fazlasıyla bütünleşmektedir. Geniş sayıda unsur, çeşitli morontia\hyp{}eğitim dünyaları üzerinde konumlanmış bir haldedir; fakat bütüncül bir biçimde onlar, fani yaratımların ussal ve ruhsal erişimiyle birlikte donanmamışlardır. Onlar eşit bir biçimde, yerel yaratılmışların diğer yerellerinin ve yüksek meleksel varlıkların öğrenimleriyle birlikte ilişkilidir. Onların yardımcılarının birçoğu, yaratılmış biçimindeki kutsal üçleme haline getirilmiş olan varlıkların düzeylerinden seçilmiştir.
\vs p020 8:3 Eğitmen Evlatları; çevre merkez koruyucularının görevlerinden yıldız öğrencilerinin yükümlülüklerine kadar uzanan, kâinat hizmetinin tüm alt düzey fazlarının bütününün yetkinliği ve tasdiki için yapılan tüm sınavları uygulayan ve bütün sınamaları idare eden eğitim sorumlularını oluştururlar. Onlar, gezegensel derslerden Salvington üzerinde konumlanan Bilgeliğin yüksek Üniversitesi’ne kadar uzanan bir kapsamda eğitimin yüzyıllar süren akışını yerine getirirler. Çabanın ve kazanımın belirticisi olan tanınma, bilgelik ve doğruluk içinde bu serüvenleri tamamlayan yükseliş içindeki fani veya arzu dolu çocuksu melekler biçimindeki herkes için sağlanmıştır.
\vs p020 8:4 Tüm aşkın\hyp{}evrenler içinde Tanrı’nın Evlatları’nın hepsi, bu ezeli bir biçimde inançlı olan ve kâinatsal olarak etkin Cennet Eğitim Evlatları’na borçludur. Onlar, Tanrı’nın Evlatları’nın kendilerinin bile gerçek ve güvenilir eğitmenleri olarak, tüm ruhaniyet kişiliklerinin yüceltilmiş öğretmenleridir. Fakat Eğitmen Evlatları’nın faaliyetleri ve görevlerinin sonu gelmez ayrıntıları hakkında sizleri bilgilendirememekteyim. Daynal\hyp{}evlatlığının faaliyetlerinin geniş nüfuz alanı Urantia üzerinde; gezegeninizin ruhsal olan tecridi sona erdiği ve siz us bakımından daha ileri bir düzeye eriştiğiniz zaman, daha iyi bir biçimde anlaşılacaktır.
\usection{9.\bibnobreakspace Daynallar’ın Gezegensel Hizmeti}
\vs p020 9:1 Evrimsel bir dünya üzerindeki olayların ilerleyişi; zamanın ruhsal bir çağının başlatılması için olgun bir hale geldiğini gösterdiği an, Kutsal Üçleme Eğitmen Evlatları her zaman bu hizmet için gönüllükte bulunur. Siz, evlatlığın bu düzeyine aşina olan bir konumda bulunmamaktasınız; çünkü Urantia hiçbir biçimde, kâinatsal aydınlanmanın bir bin\hyp{}yıl süreci biçimindeki bir ruhsal çağı deneyimlememiştir. Fakat Eğitmen Evlatlar şu an bile, âleminiz üzerinde onların hedeflenen kısa süreli ikametleriyle iniltili olan tasarılarını oluşturma amacıyla dünyanızı ziyaret etmektedir. Onlar Urantia’da; buranın sakinlerinin hayvan olma durumunun zincirlerinden ve maddiyatın engellerinden göreceli olan kurtuluşlarını kazanmalarından sonra ortaya çıkacaklardır.
\vs p020 9:2 Kutsal Üçleme Eğitmen Evlatları’nın gezegensel yazgı dönemini nihayetlendirme ile ilgili hiçbir ilişkisi bulunmamaktadır. Onlar ne ölümün yargıçları ne de yaşamın dönüştürücüleridir; fakat her gezegensel görev üzerinde onlar, bu hizmetleri yerine getiren bir Hakimane Evlat ile birliktelik halindedirler. Eğitmen Evlatlar bütünüyle, evrimsel bir gezegen üzerinde ruhsal gerçekliklerin döneminin doğuşu biçimindeki bir ruhsal çağın başlangıcına ilişkindir. Onlar maddi bilginin ve dünyevi bilgeliğin ruhsal eşleniklerini gerçek hale getirirler.
\vs p020 9:3 Eğitmen Evlatlar genellikle, gezegensel zamana göre bin yıl süresince ziyaret ettikleri gezegen üzerinde kalmaya devam ederler. Bir Eğitmen Evlat, bin yıllık gezegensel egemenliğini idare edip, bu süreç içinde kendisine ait düzeyin sayıca yetmiş olan birlikteliği tarafından desteklenir. Daynallar ne ete kemiğe bürünür ne de başka bir biçimde fani varlıklar için görünebilir bir halde kendilerini maddileştirir; bu nedenle onların ziyaretlerinin ilişkisi, Kutsal Üçleme Evlatları’yla birliktelik halinde bulunan yerel evren kişilikleri biçimindeki Berrak Akşam Yıldızları’nın etkinlikleri tarafından sağlanır.
\vs p020 9:4 Daynallar bir yerleşik dünyaya birçok kez geri dönebilir; ve buna ek olarak onların nihai görevini takiben bahse konu gezegen, mevcut evren çağının fani\hyp{}yerleşim dünyalarının tümünün evrimsel amacı biçimindeki ışığın ve yaşamın bir âleminin sabitlenmiş düzeyine getirilmiş olur. Kesinliğe Erişecek Olanların Fani Birlikleri, ışık ve yaşam içinde sabitlenen âlemlerle yakın bir biçimde ilgilidir; bununla birlikle onların gezegensel etkinlikleri bahse konu bu Eğitmen Evlatları’yla temas halindedir. Gerçekte Daynal evlatlığının bütün düzeyi içten bir biçimde, zamanın ve mekânın evrimsel yaratılmışları içinde kesinliğe erişecek olanların etkinliklerinin tüm fazlarıyla ilişki halindedir.
\vs p020 9:5 Kutsal Üçleme Evlatları, evrimsel yükselişin öncül aşamaları vasıtasıyla fani ilerleyiş düzeyiyle bütüncül bir biçimde tanımlanmış bir görünüme sahiptir ki; sıklıkla biz, gelecek evrenlerin açığa çıkarılmamış süreci içerisinde onların kesinliğe erişecek olanlarla olası birlikteliği hakkında varsayımda bulunmaya yönelmekteyiz. Biz, aşkın evrenlerin idarecilerinin Kutsal Üçleme kökenli kişiliklerin ve Kutsal Üçleme ile bütünleşmiş yükseliş halindeki evrimsel yaratılmışların bir parçası olduğunu gözlemekteyiz. Biz kararlı bir biçimde; mevcut an içerisinde Eğitmen Evlatları’nın ve kesinliğe erişecek olanların, açıklanmayan gelecek bir son içerisinde daha yakın birliktelik amacıyla onları hazırlayacak olan öncül eğitim nitelikte olabilecek zaman\hyp{}birlikteliğinin deneyimini elde etmekle meşgul bir durumda olduklarına inanmaktayız. Aşkın evrenler nihai bir biçimde ışık ve yaşam içinde sabit bir konuma getirildikleri zaman; evrimsel dünyaların sorunlarıyla fazlasıyla aşina olmuş bir hale gelen ve evrimsel fanilerin süreçleriyle oldukça uzun bir süre boyunca birliktelik içinde olan bu Cennet Eğitmen Evlatları’nın, Kesinliğe Erişecek Olanların Cennet Birlikleri ile yaşanacak ebedi birlikteliğe muhtemel bir biçimde aktarılacaklarına dair inanca Uversa üzerinde sahip bulunmaktayız.
\usection{10.\bibnobreakspace Cennet Evlatları’nın Bütünleşmiş Hizmeti}
\vs p020 10:1 Tanrı’nın Cennet Evlatları’nın tümü kökeni ve doğası bakımından kutsaldır. Her dünya adına her bir Cennet Evladı’nın görevi, hizmetin Evladı’nın Tanrı’nın ilk ve tek Evladı olmuşçasına benzersiz bir öneme sahiptir.
\vs p020 10:2 Cennet Evlatları, zamanın ve mekânın nüfuz alanları için İlahiyat’ın üç bireyinin faaliyet içerisinde olan doğalarının kutsal temsilidir. Yaratan, Hakimane ve Eğitmen Evlatlar; insanlığın çocukları ve yükseliş olanağının tüm diğer kâinat yaratılmışları için ebedi İlahiyatlar’ın hediyeleridir. Tanrı’nın bahse konu bu Evlatları, ebediyetin yüksekte bulunan ruhaniyet hedefine erişen zamanın yaratılmışlarına yardım etme görevine durmak bilmeyen bir biçimde bağlı olan kutsal yardımcılardır.
\vs p020 10:3 Yaratan Evlatlar içinde, Kâinatın Yaratıcısı’nın sevgisi Ebedi Evlat’ın bağışlamasıyla harmanlanmış olup; bu nitelik yerel evrenler için Mikâiller’in yaratıcı gücü, sevgi dolu hizmeti ve anlayışlı egemenliği içerisinde açığa çıkarılmıştır. Hakimane Evlatlar içinde Sınırsız Ruhaniyet’in hizmeti ile bütünleşen Ebedi Evlat’ın bağışlaması; bahse konu Avonallar’ın yargısının, hizmetinin ve bahşedilmişliğinin süreci içinde evrimsel nüfuz alanları için açığa çıkarılmıştır. Kutsal Üçleme Eğitmen Evlatları içinde Cennet İlahiyatları’nın üçünün de sevgisi, bağışlaması ve hizmeti en yüksek zaman\hyp{}mekân değer\hyp{}düzeyleri üzerinde eş güdüm haline getirilmiş olup; bu nitelikler evrenlere yaşayan doğruluk, kutsal iyilik ve gerçek ruhsal güzellik olarak sunulmuştur.
\vs p020 10:4 Yerel evrenler üzerinde evlatlığın bu düzeyleri, mekânın yaratılmışları için Cennet’in İlahiyatları’nın açığa çıkarılmış gerçeklikleri üzerinde etkide bulunmak amacıyla işbirliği halindedirler. Bu işbirliği içerisinde yerel bir evrenin Yaratıcısı olarak bir Yaratan Evlat, Kâinatın Yaratıcısı’nın sınırsız olan karakterini temsil eder. Bağışlamanın bahşedilmiş Evlatları olarak Avonallar, sınırsız merhametin Ebedi Evladı’nın eşi benzeri olmayan doğasını açığa çıkarır. Yükseliş halinde bulunan kişiliklerin gerçek öğretmenleri olarak Kutsal Üçleme Daynal Evlatları, Sınırsız Ruhaniyet’in eğitmen kişiliğini ortaya koyar. Onların kutsal bir biçimde kusursuz olan eş güdümü içerisinde; Mikâiller, Avonallar ve Daynallar zaman\hyp{}mekân evrenleri için ve onun içerisinde Yüce olan Tanrı’nın kişiliğinin ve egemenliğinin gerçekleştirilmesine ek olarak açığa çıkarılmasına katkıda bulunur. Onların üçleme bütünlüğü etkinliklerinin uyumunda Tanrı’nın bu Cennet Evlatları; sonsuza kadar sürecek olan Cennet Adası’ndan mekânın bilinmeyen derinliklerine kadar olan İlk Muhteşem Kaynak ve Merkez’in kutsallığının hiçbir zaman sona ermeyecek olan büyümesini takip ederlerken, İlahiyat’ın kişiliklerinin öncülüğünde ezelden beri faaliyet içerisinde bulunurlar.
\vs p020 10:5 [Uversa’dan olan bir Bilgeliğin Kesinleştiricisi tarafından sunulmuştur.]
