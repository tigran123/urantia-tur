\upaper{97}{İbraniler Arasında Tanrı Kavramının Evrimi}
\vs p097 0:1 İbraniler’in ruhsal önderleri; --- Tanrı kavramsallaşmalarını, sadece filozoflar tarafından kavranılabilecek İlahiyat’ın soyut bir kavramına dönüştürmeyen bir şekilde insansı nitelikten çıkararak --- kendilerinden önce hiç kimsenin bu zamana kadar başaramadığını gerçekleştirmişlerdi. Genel halk bile; bir birey olmasa da en azından bir ırka ait biçimde, Yahveh’in olgunlaşmış kavramsallaşmasını bir Yaratıcı olarak değerlendirmeye yetkindi.
\vs p097 0:2 Tanrı’nın kişiliğine dair kavramsallaşma, her ne kadar açık açık Melçizedek döneminde Salem’de öğretilmiş olsa da, Mısır’dan kaçış zamanında belirsiz ve muğlaktı; ve yalnızca kademeli olarak, İbrani aklında nesilden nesile ruhsal önderlerin öğretilerine gösterilen karşılık sonucunda evirildi. Yahveh’in kişiliğine dair algı, İlahiyat’ın diğer birçok niteliğinkine kıyasla ilerleyici evrimi bakımından çok daha fazla devamlılık içindeydi. Musa’dan Malaçi’ye kadar İbrani aklında Tanrı’nın kişiliğine dair neredeyse aralıksız gerçekleşmiş bir düşünsel büyüme açığa çıktı; ve bu kavramsallaşma nihai olarak, gökteki Yaratıcı hakkındaki İsa öğretileri tarafından geliştirilmiş ve yüceltilmişti.
\usection{1.\bibnobreakspace Samuel --- İbrani Peygamberlerin İlki}
\vs p097 1:1 Filistin’deki çevre toplulukların düşmansı baskısı yakın zaman içerisinde İbrani şeyhlerine, kabile örgütlenmelerini merkezileşmiş bir hükümet haline getirerek konfederasyonlaşmada bulunmadıkları takdirde hayatta kalmayı ümit dahi edemeyeceklerini öğretmişti.
\vs p097 1:2 Samuel, ibadet türlerinin bir parçası olarak Melçizedek gerçekliklerini korumada kararlı olmuş Salem öğretmenlerinin eski bir kuşağından gelmişti. Bu öğretmen coşkulu ve kararlı bir kişiydi. Olağanüstü azmiyle birlikte yalnızca büyük sadakati, İsrail’in tümünü Musa dönemlerinin yüce Yahvehi’ne olan ibadete döndürmeye giriştiğinde karşılaştığı neredeyse her kesin gösterdiği tepkiye karşı gelmede onu yetkin kılmıştı. Ve bu dönemde bile o sadece kısmi bir biçimde başarılı olmuştu; o, Yahveh’in daha yüksek kavramsallaşmasına yapılacak hizmette İbraniler’in yalnızca daha ussal yarısını geri kazanmıştı; arta kalan diğer yarısı ise, ülkenin kabile tanrılarına ek olarak Yahveh’in daha alt düzeydeki kavramsallaşmasına yapılmakta olan ibadeti sürdürdü.
\vs p097 1:3 Samuel; bir günde birliktelikleri ile çıkıp, Baal’a inanan yerleşkelerinin yirmi kadarında ona yapılan ibadete son verebilen, uygulayıcı bir kökten değiştirici olarak sert\hyp{}ve\hyp{}hazır bir karakterdi. Onun gerçekleştirdiği ilerleme, tamamen zorlama etkisiyle sağlanmaktaydı; o çok az vaaz vermişti, bundan da azını öğretmişti, ancak yeterince faaliyette bulunmuştu. Bir gün Baal’in dinadamıyla alay ediyor; diğer bir gün esir bir kralı parçalarına ayırıyordu. O kendini adamış bir biçimde tek Tanrı’ya inanmıştı; ve o, bu tek Tanrı’nın yer ve gökyüzünün yaratanı olduğuna dair belirgin bir kavramsallaşmaya sahipti: “Dünyayı ayakta tutan sütunlar Koruyucu’ya ait olup, o dünyayı bunların üstüne yerleştirmiştir.”
\vs p097 1:4 Ancak Samuel’in İlahiyat’ın kavramsallaşmasının gelişimine yaptığı büyük katkı; hatasız kusursuzluk ve kutsallığın sonsuza kadarki aynı vücuda bürünüşü olarak Yahveh’in \bibemph{değişmez} olduğuna dair güçlü resmi duyurusuydu. Bu dönemlerde Yahveh, yaptığı her şeyden her zaman pişmanlık duyan, kıskanç huysuzluğun acınası bir Tanrı olduğu düşünülmekteydi; ancak bu aşamada, Mısır’dan dışa doğru yelken açtıkları zamandan beri İbraniler ilk kez şu şaşırtıcı kelimeleri duymuşlardı: “İsrail’in Gücü ne yalan söyler, ne de pişmanlık duyar; zira o bir insan değildir, insan bunu düşündüğü için tövbe etmelidir.” Kutsallık ile ilgili olarak istikrar duyuruldu. Samuel; Melçizedek’in Ephraim ile olan sözleşmesini vurgulayıp, İsrail’in Koruyucu Tanrısı’nın tüm gerçekliğin, istikrarın ve devamlılığın kaynağı olduğunu duyurdu. İbraniler öncesinde Tanrıları’nı her zaman, bilinmeyen kökenden gelen yüceltilmiş bir ruhaniyet olarak, üstün\hyp{}insan şeklinde bir insan olarak görmüşlerdi; ancak bu aşamada onlar, bir zamanların Horeb ruhaniyetinin yaratan kusursuzluğa ait değişmez bir Tanrı olarak yüceltildiğini duydular. Samuel; insan aklındaki değişikliklerden ve fani mevcudiyetin iniş\hyp{}çıkışlarından çok daha yukarılara çıkması için evirilmekte olan Tanrı kavramsallaşmasına destek vermekteydi. Öğretisi altında İbraniler’in Tanrısı; kabile tanrılarının düzeyindeki bir düşünceden, tüm yaratımın her\hyp{}şeye\hyp{}gücü\hyp{}yeten, değişmez Yaratanı’nı ve \bibemph{Yüksek\hyp{}Denetimcisi }idealine yükselmeye başlamaktaydı.
\vs p097 1:5 Ve o, sözleşmeye sadık kalan güvenilirliği olarak Tanrı’nın içtenliğine dair hikayeyi yeniden duyurdu. Samuel şunu söyledi: “Koruyucu insanlarını yalnız bırakmaz.” “O bizimle, her ayrıntısını belirleyip teminat altına aldığı, sonsuza kadar sürecek bir sözleşme yaptı.” Ve böylece, Filistin’in tümü boyunca, yüce Yahveh’in ibadetine doğru geri çağırım yankılandı. Bu coşkulu öğretmen “Sen büyüksün, Sen Koruyucu Tanrı’mız, zira ne senin gibi birisi var ne de senden başka herhangi bir Tanrı var.”
\vs p097 1:6 Bu döneme kadar İbraniler, Yahveh’in lütfunu başlıca olarak maddi refah bakımından değerlendirmişti. Samuel şunu duyurmaya cesaret ettiğinde, İsrail için büyük sarsıntı olup, neredeyse onun yaşamına mal oluyordu: “Koruyucu zenginleştirir ve fakirleştirir; o alçaltır ve yükseltir. O fakiri çöpten çıkarır, dilencileri ihtişamın tahtını alması için prenslerin arasına yerleştirir.” Musa’dan beri alt tabakada bulunanlar ve daha az talihsiz olanlar için bu tür teselli edici sözler duyurulmamıştı; ve fakirler arasında çaresizlerin binlercesi, ruhsal düzeylerini geliştirebilecekleri umudunu beslemeye başladılar.
\vs p097 1:7 Ancak Samuel, bir kabile tanrı kavramsallaşmasının çok ötesine geçen bir biçimde ilerleyememişti. O, her insanı yaratan bir Yahveh’i duyurmuştu; ancak o başlıca olarak, seçilmiş insanları biçiminde İbraniler ile meşgul olmuştu. Böyleyken bile, Musa dönemlerinde olduğu gibi, bir kez daha Tanrı kavramsallaşması kutsal ve doğru olan bir İlahiyat’ı tasvir etmişti. “Orada, Koruyucu kadar kutsal olan biri yoktur. Bu kutsal Koruyucu Tanrı’yla kim karşılaştırılabilir?”
\vs p097 1:8 Seneler ilerledikçe, saçlarına aklar düşmüş eski önder Tanrı’nın anlayışında ilerledi; zira o şunu duyurdu: “Koruyucu bir bilgi Tanrısı’dır, ve eylemler onun tarafından ölçülür. Koruyucu, merhametliye merhamet göstererek dünyanın sonunu yargılayacak, doğru insana o da doğru davranacaktır.” Burada bile, her ne kadar merhametli olan ile sınırlı olsa da, bağışlamanın başlangıcı söz konusudur. Daha sonra Samuel, fazlasıyla sıkıntı çektikleri bir durumda insanlarından şunu yapmaları için onları ikna etmeye çalıştığında bir adım daha atmıştı: “Tanrı’nın ellerine düşmemize izin verin, zira onun merhametleri büyüktür.” “Çok azını veya çok fazlasını kurtarması için Koruyucu’nun üzerinde hiçbir sınırlama yoktur.”
\vs p097 1:9 Ve Yahveh’in kişiliğinin kavramsallaşmasına dair bu kademeli gelişim, Samuel’in varislerinin hizmeti altında devam etti. Onlar Yahveh’i, sözleşme\hyp{}sadığı bir Tanrı olarak sunmaya giriştiler; ancak nadiren, Samuel tarafından kurulan barışı muhafaza edebildiler; onlar, Samuel’in daha sonra düşünmüş olduğu Tanrı’nın merhametine dair düşünceyi geliştirmede başarısız oldular. Yahveh’in tüm diğer tanrıların üstünde oluşunun muhafazasına rağmen, diğer tanrıların tanınmasına doğru düzenli bir kayış bulunmaktaydı. “Seninkisi krallıktır, Ey Koruyucu, sen her kesin başı olarak yüceltilmişsindir.”
\vs p097 1:10 Bu dönemin temel vurgusu kutsal güçtü; bu çağın tanrı\hyp{}elçileri İbrani tahtına gelecek kralı teşvik etmek için tasarlanmış bir dini duyurmaktaydı. “Seninkisi, Ey Koruyucu, büyüklük, güç, ihtişam, zafer ve görkemdir. Senin elinde güç ve kudret bulunup, sen muhteşem bir şeyi yapmaya ve her şeye güç vermeye muktedirsin.” Ve bu söylem, Samuel ve onun doğrudan varisleri dönemi boyunca Tanrı kavramsallaşmasının düzeyiydi.
\usection{2.\bibnobreakspace İlyas ve Elyasa}
\vs p097 2:1 İsa’dan önceki onuncu çağda İbrani milleti, iki krallığa ayrılmış hale geldi. Bu siyasi bölümlerin her ikisi içinde de, birçok gerçeklik öğretmeni; başlamış ve bölünme savaşından sonra oldukça zarar verici bir şekilde devam etmiş ruhsal yozlaşmanın gerici dalgasını durdurmak için çaba sarf etmişlerdi. Ancak İbrani dininin geliştirmek için bu çabalar, doğruluğun kararlı ve korkusuz savaşçısı olan İlyas öğretilerine başlayana kadar başarılı bir biçimde serpilmemişti. Elyasa, Samuel’in döneminde beslenen görüş ile karşılaştırabilir bir Tanrı kavramsallaşmasını kuzey krallığında eski haline getirdi. İlyas, Tanrı’ya dair gelişmiş bir kavramsallaşmayı sunmak için çok az olanağa sahipti; o, daha önce Samuel’in olduğu gibi, Baal’ın sunaklarını yerinden etmekle ve sahte tanrıların putlarını yıkmakla meşgul edilmişti. Ve o, putlara tapan bir kralın karşıtlığı karşısında köklü değişikliklerine devam etmişti; onun görevi, Samuel’in karşılaşmış olduğundan bile daha devasa ve zordu.
\vs p097 2:2 İlyas alıkonulduğunda onun sadık birlikteliği Elyasa görevini devralmıştı, ve çok az kişi tarafından tanınan Micaiah’ın kıymetli desteğiyle gerçekliğin ışığını Filistin’de canlı tuttu.
\vs p097 2:3 Ancak bu dönemler İlahiyat’ın kavramsallaşmasında ilerleyişin gerçekleştiği zamanlar değildi. Henüz İbraniler Musasal ideale bile yükselememişti. İlyas ve Elyasa’nın dönemi; yüce Yahveh’e olan ibadete daha iyi sınıfların dönmesiyle sonlanmış olup, Kainatın Yaratıcısı’na dair düşüncenin yaklaşık olarak Samuel’in bıraktığı yere doğru yeniden gelişini gözlemlemişti.
\usection{3.\bibnobreakspace Yahveh ve Baal}
\vs p097 3:1 Yahveh’e inananlar ve Baal’i takip edenler arasındaki çok uzun süren ihtilaf; dini inanışlar içindeki bir farklılıktan ziyade, ideolojilerin sosyo\hyp{}ekonomik nitelikli bir çatışmasıydı.
\vs p097 3:2 Filistin sakinleri, arazinin özel mülkiyetine dair tutumlarında farklılık gösterdiler. Güneydeki veya diğer bir değişle (Yahveh unsurları olan) gezgin Arap kabileleri araziyi, --- İlahiyat’ın kavime verdiği bir hediye biçiminde --- bir devredilemez olarak görmüşlerdi. Onlar, arazinin satılamayacağı veya rehin alınamayacağı görüşünü savundular. “Yahveh şunu ifade etmişti: ‘Arazi satılamaz, zira arazi benimdir.’”
\vs p097 3:3 Kuzeyde ve daha yerleşik hale gelmiş (Baal unsurları olarak) Kenani toplulukları sınırlandırılmamış bir biçimde topraklarını alıp, satıp, rehin verdiler. Baal inanışı, iki temel inanç savı üzerine inşa edilmişti: Birincisi, --- araziyi alma ve satma hakkı olarak --- özel mülkiyet değişimleri, anlaşmaları ve sözleşmelerinin geçerliliği; ikincisi, --- toprağın bereketliliğinin bir tanrısı olarak --- Baal’ın yağmur göndermesinin varsayılımıydı. İyi ekinler, Baal’ın lütfuna bağlıydı. Bu inanış büyük ölçüde, iyeliği ve bereketi olmak üzere araziyle ilgiliydi.
\vs p097 3:4 Genellikle Baal unsurları; evlere, arazilere ve kölelere sahip oldular. Onlar soylu arazi sahipleri olup, şehirlerde yaşadılar. Her Baal tanrısı bir mekana, bir din\hyp{}adamlığına ve ayinsel fahişeler olan “kutsal kadınlara” sahipti.
\vs p097 3:5 Arazinin değerlendirilişindeki bu temel farklılıktan; Kenani ve İbrani toplulukları tarafından sergilenmiş toplumsal, ekonomik, ahlaki ve dini tutumlardaki sert düşmanlıklar evirilmişti. Bu sosyo\hyp{}ekonomik ihtilaf, İlyas’ın dönemlerine kadar belirli bir dini mesele haline gelmemişti. Bu sinirli tanrı\hyp{}elçisinin dönemlerinden itibaren bahse konu mesele, --- Yahveh ve Baal karşıtlığı olarak --- daha keskince ayrılmış dini taraflar haline çatışılmıştı; ve bu ihtilaf Yahveh’in zaferine ek olarak ileride gerçekleşecek tektanrıcılığa olan yönelişle sonuçlanmıştı.
\vs p097 3:6 İlyas, Yahveh\hyp{}Baal ihtilafını arazi meselesinden İbrani ve Kenani dünya görüşlerinin dini niteliğine doğru kaydırdı. Ahav; Nabot unsurlarını arazilerini elde etmek için bir kumpas içinde öldürdüğünde, İlyas bu olayı, eskiden gelen arazi adetlerden doğan bir ahlaki sorun yapıp, Baal unsurlarına karşı sert bir harekatta bulundu. Bu durum aynı zamanda, şehirlerin egemenliğine karşı taşra halkının bir savaşıydı. Başlıca İlyas dönemi zamanında Yahveh Elohim haline geldi. Bu tanrı\hyp{}elçisi bir tarım ıslahatçısı olarak başlayıp, İlahiyat’ı yüceltici olarak son buldu. Baal tanrıları çoktu, Yahveh tanrısı \bibemph{tek} idi --- tektanrıcılık, çoktanrıcılık karşısında zafer elde etmişti.
\usection{4.\bibnobreakspace Amos ve Hosea}
\vs p097 4:1 Öncül İbraniler’in Yahveh’i olarak, fedalar ve törenler ile uzunca bir süre kendisine hizmet edilmiş biçimde --- kabile tanrısından, insanları arasında bile suçu ve ahlaksızlığı cezalandırabilecek bir Tanrı’ya olan geçiş içinde büyük bir adım; kuzey kabilelerin suçluluğunu, sarhoşluğunu, baskıcılığını ve ahlaksızlığını kötülemek için güney tepelerinden ortaya çıkan Amos tarafından atılmıştı. Musa’nın ortaya çıkışından beri, bu türden güçlü gerçeklikler Filistin’de duyurulmamıştı.
\vs p097 4:2 Amos yalnızca, var olan düzeni eskisine doğru düzeltmeye çalışan veya onda köklü yenilikler yapan bir kişi değildi; o, İlahiyat’a ait yeni kavramsallaşmaların bir kaşifiydi. O; kendisinden daha önce gelenler tarafından bildirilmiş Tanrı’ya dair şeylerin çoğunu herkese resmi olarak duyurmuş olup, sözde seçilmiş olan insanları arasında süregelen günaha göz yuman bir Kutsal Varlık’a duyulan inanca cesur bir biçimde karşı çıkmıştı. Melçizedek döneminden beri ilk kez insanın kulakları, milli adalet ve ahlakın ikiyüzlülüğünün kötülenişini duymuştu. Tarihlerinde ilk kez İbrani kulakları kendi tanrıları olan Yahveh’in, yaşamlarındaki suç ve günaha diğer insan topluluklarından daha fazla bir biçimde hoşgörüyle bakmayacağını işitmişlerdi. Amos, Samuel ve İlyas’ın sert ve katı Tanrısı’nı tahayyül etmişti; ancak o aynı zamanda, yanlışın cezalandırılışına geldiğinde İbraniler’i herhangi bir diğer ırktan daha farklı bir biçimde düşünmeyen bir Tanrı’yı gördü. Bu, “seçilmiş insanların” bencil inanç savına doğrudan bir saldırıydı; ve bahse konu dönemlerin birçok İbrani topluluk üyesi bu düşünceye karşı ciddi öfke besledi.
\vs p097 4:3 Amos şunu söyledi: “Dağları yükselteni ve rüzgarı yaratanı, yedi yıldızı ve Orion’u oluşturanı, ölümün gölgesini sabaha, gün karanlığını geceye çeviren O’nu arzulayın.” Ve, yarı\hyp{}dindar, fırsatçı ve zaman zaman ahlaksız olan akranlarını kötüleyen bir biçimde o, kötülük yapanlar hakkında şunları ifade ettiğinde, değişmez bir Yahveh’in önlenemez adaletini tasvir etmeyi amaçlamıştı: “Eğer cehenneme yolunda olursalar, o zaman ben onları buradan alacağım; eğer cehenneme çıkarsalar, buradan onları aşağıya indireceğim.” “Ve, eğer düşmanları karşısında esir düşerlerse, adaletin kılıcını üzerlerine tutacağım, ve bu kılıç onları yok edecektir.” Amos, azarlayıcı ve suçlayıcı bir parmağı onlara doğrultan bir biçimde Yahveh adına şunu duyurduğunda dinleyicilerini şaşırtmıştı: “Kesinlikle yaptıklarınızın hiçbirini hiçbir zaman unutmayacağım.” “Ve ben, buğdayın bir elekten geçirildiği gibi tüm milletler arasında İsrail yurdunu elekten geçireceğim.”
\vs p097 4:4 Amos; Yahveh’i tüm milletlerin tanrısı olarak duyurup, ayinin doğruluğun yerini almaması için İsrail topluluklarını uyardı. Ve bu cesur öğretmen ölene kadar taşlanmadan önce, yüce Yahveh’in inanç savını kurtarmak için yeteri kadar gerçeklik mayası çalmıştı; o, Melçizedek açığa çıkarılışının ileri evrimini teminat altına almıştı.
\vs p097 4:5 Hosea, Amos ve onun evrensel bir Tanrı öğretisini; derin sevgi olarak bir Tanrı’ya dair Musa kavramsallaşmasının yeniden dirilişiyle takip etti. Hosea bağışlanmanın tövbeyle elde edilişini duyurdu, fedayla değil. O şunları ifade ederek sevgi dolu iyiliğin ve kutsal merhametin bir müjdesini duyurdu: “Ben sizi sonsuza kadar kendime bağlayacağım; evet, ben sizi doğruluk ve adalette, sevgi dolu iyilikte ve bağışlamada kendime bağlayacağım.” “Ben onları sınırsızca seveceğim, zira artık onlara kızgın değilim.”
\vs p097 4:6 Hosea sadık bir biçimde, şunları ifade ederek Amos’un ahlaki uyarılarına devam etti: “Onları cezalandırmak benim arzumdur.” Ancak İsrail unsurları, Hosea şunları söylediğinde bunu millete ihanet derecesindeki saygısızlık olarak gördü: “Geçmişte benim insanlarım olmayanlara ‘siz benim insanlarımsınız diyeceğim.’; ve onlar ‘sen bizim Tanrımız’sın’ diyecekler.” O şunları söyleyerek tövbe ve bağışlamayı duyurmaya devam etti: “Ben onların yanlışa geri dönüşlerini iyileştireceğim; onları sınırsız bir biçimde seveceğim, zira artık onlara kızgın değilim.” Hosea her zaman ümit ve bağışlamayı duyurdu. Onun insanlığa duyurusunun ağır sorumluluğu en başından beri şuydu: “İnsanlarım karşısında bağışlayıcı olacağım. Onlar benden başka hiçbir Tanrı’yı tanımıyorlar, zira benden başka hiçbir kurtarıcı yoktur.”
\vs p097 4:7 Varsayıldığı biçimiyle sırf seçilmiş insanlar oldukları için Yahveh’in suç ve günahı kendi topluluğu içinde maruz görmeyeceğine dair tanınmaya Amos İbraniler’in milli bilincini hazırlarken, Hosea, İşaya ve onun birliktelikleri tarafından oldukça seçkin bir biçimde söylenen kutsal merhamet ve sevgi dolu iyiliğin daha sonraki bağışlayıcı akortlarının açılış notalarını çalmıştı.
\usection{5.\bibnobreakspace İlk İşaya}
\vs p097 5:1 Bu zamanlar; bazıları, kuzey kavimleri içinde kişisel günahlara ve milli suça karşı verilecek cezanın tehditlerini duyururken, diğerlerinin, güney krallığı içindeki ihlaller karşısında ortaya çıkacak felaketin tahminini yaptığı zamanlardı. İbrani milletleri içinde vicdan ve bilincin bu uyanışının başında ilk İşaya ortaya çıkışını gerçekleştirmişti.
\vs p097 5:2 İşaya, güvenirliğinin değişmez kusursuzluğu niteliğinde onun sınırsız bilgeliği olarak Tanrı’nın ebedi doğasını duyurmaya devam etti. O, İsrail’in Tanrısı’nı şunları söyleyerek temsil etti: “Adaleti aynı doğruluğun çekülüne koyacağım.” “Koruyucu sizi; içinde insanın hizmet vermeye hazır kılındığı, kederden, korkudan ve zor sorumluluktan huzura eriştirecektir.” “Ve sizin kulaklarınız arkanızda şunları söyleyen bir cümle duyacaktır: ‘doğru yol bu, buradan yürü.’” “İşte Tanrı benim kurtuluşum; ben ona güveneceğim, ondan korkmayacağım, zira Koruyucu benim gücüm, benim şarkımdır.” “‘Şimdi gel ve beraber düşünelim’ der Koruyucu, ‘her ne kadar günahlarınız şimdi al al olsa da, ileride kar kadar beyaz olacaklardır; her ne kadar kıpkırmızı olsalar da, yün kadar ak olacaklardır.”
\vs p097 5:3 Korku tarafından yönlendirilen ve ruhsal olarak açlık çeken İbraniler’e konuşarak bu tanrı\hyp{}elçisi şunları söyledi: “Doğ ve ışı, çünkü ışığın geldi, ve Tanrı’nın ihtişamı üzerine doğdu.” “Koruyucu’nun ruhaniyeti üzerimde, çünkü o güçsüzlere iyi haberleri duyurmakla beni kutsadı; o beni, kalbi kırık olanları bir araya getirmek, esirlerin özgürlüğünü ve zincirlenenlerin zindanının açılışını duyurmak için gönderdi.” “Ben Koruyucu’nun mevcudiyetinden fazlasıyla sevinç duyacağım, benim ruhum Tanrım’da mutlu olacaktır, zira o beni kurtuluşun elbiseleriyle örtüp, kendi doğruluk kaftanıyla kuşatmıştır.” “Çektikleri tüm acılarda o da acı çekmişti, ve mevcudiyetinin meleği onları kurtarmıştı. Derin sevgisiyle ve acıyışında onları günahlarından arındırmıştı.”
\vs p097 5:4 Bu İşaya, ruhu tatmin eden müjdesini onaylamış ve onu süslemiş olan Micah ve Obadiah tarafından takip edilmişti. Ve bu iki mert haberci; İbraniler’in, din adamları tarafından yönlendirilen ayin düzenini cesurca kötülemiş, feda sisteminin tamamına korkusuzca eleştirmişlerdi.
\vs p097 5:5 Micah “ödül için hüküm veren yöneticileri, özel ders için öğreten dinadamlarını ve para için kutsayan tanrı\hyp{}elçilerini” kötümsemişti. O, hurafelerden ve dinadamları yollarından bir günlük özgürlüğü öğretmişti: “Ancak her insan kendi sarmaşığı altında oturacak, ve bu hiç kimseyi korkutmayacaktır, zira tüm insanlar, her biri kendi Tanrı anlayışı uyarınca yaşayacaktır.”
\vs p097 5:6 En başından beri Micah’ın iletisinin sorumluluğu şuydu: “Tanrı’nın önüne yanmış sunularla mı çıkmalıyım? Koruyucu, bin koçla mı yoksa bin yağ ırmakla mı tatmin olacak? Benden ilk doğana yanlışımı mı vermeliyim, ruhumun günahı için bedenimin meyvesini mi? O bana, ey insanlar, iyinin ne olduğunu gösterdi; ve Koruyucu’nun yalnızca sizden, adil bir biçimde yaşamanızı, bağışlamayı sevmenizi ve Tanrınız ile birlikte alçakgönüllülükle yürümenizi istediğini gösterdi.” Ve bu dönem büyük bir çağdı; bunlar gerçekten de, bu tür özgürleştirici iletileri iki buçuk bin yıldan daha da öncesinde fani insanın duyduğu ve hatta bazılarının inandığı zamanlar olarak çok hareketli dönemlerdi. Ve eğer dinadamlarının inatçı karşıtlığı olmasaydı, bu öğretmenler İbrani ibadet ayini düzenine ait tüm kanlı törenselliği kaldırmış olacaktı.
\usection{6.\bibnobreakspace Korkusuz Jeremiah}
\vs p097 6:1 Her ne kadar birkaç öğretmen İşaya’nın müjdesini detaylı bir biçimde açıklamaya devam ettiyse de, İbraniler’in Tanrısı olan Yahveh’in uluslararası düzeye getirilmesinde bir diğer cesur adamı atmak Jeremiah’a kalmıştı.
\vs p097 6:2 Jeremiah kusursuz bir biçimde Yahveh’in, diğer milletler ile olan askeri mücadelelerinde İbraniler’in yanında olmadığını duyurmuştu. O, Yahveh’in; tüm yeryüzünün, tüm milletlerin ve tüm toplulukların Tanrısı olduğunu açık ve kararlı bir biçimde öne sürdü. Jeremiah’ın öğretileri, İsrail’in Tanrısı’nın uluslararası düzeye gelişinin gittikçe yükselen dalgasıydı; nihai olarak ve sonsuza kadar bu gözü pek hatip Yahveh’in tüm milletlerin Tanrısı olduğunu ve Mısırlılar için Osiris, Babilliler için Bel, Asuriler için Asur veya Filistinliler için Dagon’un bulunmadığını duyurmuştu. Ve böylece İbraniler’in dini, yaklaşık olarak bu zaman zarfında ve ondan sonra dünya boyunca gerçekleşmekte olan tektanrıcılığın yeniden\hyp{}doğuşunu paylaşmıştı; en sonunda Yahveh’in kavramsallaşması, gezegensel ve hatta kainatsal soylulukla bir İlahiyat düzeyine yükselmiş konuma gelmişti. Ancak Jeremiah’ın birlikteliklerden çoğu Yahveh’i İbrani milletinden ayrı olarak düşünmekte zorlanmıştı.
\vs p097 6:3 Jeremiah aynı zamanda İşaya tarafından şu ifadeler ile tasvir edilmiş adil ve sevgi dolu Tanrı’yı duyurmuştu: “Evet, ben sizi sonsuza kadar sürecek bir aşk ile sevmekteyim; bu nedenle sevgi dolu iyilikle sizleri yakınıma çekmiş bulunmaktayım.” “Zira o, insan çocuklarına kasıtlı bir biçimde acı çektirmez.”
\vs p097 6:4 Bu korkusuz tanrı\hyp{}elçisi şunu söyledi: “Doğruluk, tavsiyede yüce, işte kudretli olan Koruyucumuz’dur. Onun gözleri, her birine tercih ettiği yol ve yaptıklarının meyvesi uyarınca lütuf dağıtarak insan evlatlarının tümünün her türlü yoluna açıktır.” Ancak, Kudüs’ün kuşatıldığı dönem boyunca Jeremiah şunları söylediğinde ifadesi hakaret edici ihanet olarak değerlendirilmişti: “Ve şimdi ben bu toprakları, hizmetkarım olan Babil kralı Nebuchadnezzar’ın eline teslim ettim.” Ve Jeremiah şehrin teslimini tavsiye ettiğinde, dinadamları ve şehir idarecileri iç karartıcı bir zindanın bataklık içindeki kuytusuna atarak ondan kurtulmuşlardı.
\usection{7.\bibnobreakspace İkinci İşaya}
\vs p097 7:1 İbrani milletinin yıkımı ve onun Mezopotamya’daki esareti; sahip olduğu dinadamlık düzeninin korunması için kararlı faaliyete yönelmiş olmasaydı, genişleyen din\hyp{}kuramlarına büyük yararlar sağlayan sonuçları ortaya çıkarabilirdi. Onların milleti, Babil orduları gelmeden önce çökmüş bir konumdaydı; ve onların milli Yahveh’i ruhsal önderlerin uluslararası vaizlerinden zarar görmüş halde bulunmaktaydı. Milli tanrılarının kaybından doğan hınç; yeni ve genişlemiş düşünceye ait tüm milletlerin uluslararası hale getirilmiş bir Tanrısı’nın bile seçilmiş insanları olarak Museviler’i tekrar eski konumuna getirme çabası içinde, efsanevi öykülerin yaratılmasında ve İbrani tarihinde mucizevi olarak görünen olayların çoğaltılmasında Musevi dinadamlarının bu tür uzun uğraşlar vermelerine neden olmuştu.
\vs p097 7:2 Esaret boyunca Museviler; bir yandan, kendilerine uyarladıkları Keldani öykülerinin ahlaki vurgusunu ve ruhsal önemini kesin bir biçimde geliştirmiş olduklarının, ve her ne kadar bu efsaneleri İsrail’in kökeni ve tarihini onurlandıracak ve onu yüceltecek bir biçimde çeşitli şekillerde çarpıtmış olmalarının altının çizilmesi gerekse de, Babil’e dair tarihi anlatılardan ve onların efsanelerinden fazlasıyla etkilenmişlerdi.
\vs p097 7:3 Bu İbrani dinadamları ve katipleri, akıllarında tek bir düşünceye sahiplerdi; ve o da, Musevi milletinin düzeltilmesi, İbraniler’e ait tarihi anlatımların yüceltilmesi ve ırksal tarihlerinin üstün bir konuma çıkartılmasıydı. Eğer, Batı dünyasının büyük bir kısmına dair bu dinadamlarının sahip oldukları hatalı düşüncelerine bağlı kalışlarından bir öfke duyulacaksa, onların bunu bilinçli olarak yapmadıkları hatırlanmalıdır; onlar yazdıkları şeylerin vahiyle geldiğini öne sürmemişlerdi; onlar, kutsal bir kitap yazma gibi herhangi bir meslek yaratmamışlardı. Onlar yalnızca, esaret altındaki akranlarının azalan cesaretini güçlendirmek için tasarlanan bir ders kitabı hazırlamaktaydı. Onlar kesin bir biçimde, yurttaşlarının milli ruhunu ve kendine güvenini geliştirmeyi amaçlamaktalardı. Bu ve diğer yazıtları, sözde hatasız öğretilerin rehber bir kitabı yapacak şekilde bir araya getirmek daha sonraki insanlara kaldı.
\vs p097 7:4 Musevi dinadamlığı, esaret döneminden sonra bu yazıtları cömert bir biçimde kullandı; ancak akran esirleri üzerindeki etkileri, daha birinci İşaya’nın sunmuş olduğu, adaletin, derin sevginin, doğruluğun ve bağışlamanın Tanrısı’na, inancını bırakıp tamamiyle yönelmiş olan ikinci İşaya olarak genç ve boyun eğmez tanrı\hyp{}elçisinin varlığıyla fazlasıyla engellenmişti. O aynı zamanda, Yahveh’in tüm milletlerin Tanrı’sı haline önceden gelmiş olduğuna Jeremiah ile birlikte inanmıştı. O; Tanrı’nın doğasına dair bu savları öyle bir etkiyle söylemişti ki, Museviler ve onları esir alanlar arasında eşit sayıda dine kazandırma gerçekleştirmişti. Ve bu genç hatip; her ne kadar, güzellik ve ihtişam için ortak saygıları öncül İşaya’nın yazılarını kullanmalarına yol açtıysa da, düşmancıl ve acımasız dinadamları onunla bütünüyle ilişkilendirmekten kurtulmayı amaçlamışlardı. Ve bu nedenle, bahse konu ikinci İşaya’nın yazıları, kırktan başlayarak kendisi de dâhil olmak üzere kırk beşe kadar uzanan bölümlerden meydana gelen bir biçimde bu isimdeki kitabı içinde bulunabilir.
\vs p097 7:5 Maçiventa’dan başlayarak İsa dönemine kadar hiçbir tanrı\hyp{}elçisi veya dini öğretmen, esaretin bu döneminde duyurulmuş olan ikinci İşaya’nın yüksek Tanrı kavramsallaşmasına erişememişti. Bu ruhsal önderin duyurduğu Tanrı hiçbir şekilde küçük, insansı nitelikte insan tarafından yapılmış Tanrı değildi. “İşte bak, o adaları çok küçük şeylermiş gibi kaldırıyor.” “Ve cennetler nasıl yeryüzünden daha yüksekse, birçok yol sizin takip ettiğinizden daha yüksek ve birçok düşünce sahip olduklarınızdan daha yücedir.”
\vs p097 7:6 En sonunda Maçiventa Melçizedeği, gerçek bir Tanrı’yı fani insana duyurarak insan öğretmenlerine baktı. İşaya gibi ilk başta bu önder, evrensel yaratımın ve idarenin bir Tanrısı’nı duyurdu. “Yeryüzünü ben yaratıp, insanı içine yerleştirdim. Ben onu amaçsız yaratmadım; onu yerleşilmesi için oluşturdum.” “Ben ilk ve sonum; benden başka hiçbir Tanrı yoktur.” İsrail’in Koruyucu Tanrısı hakkında konuşurken bu yeni tanrı\hyp{}elçisi şunu söyledi: “Cennet ortadan kaybolabilir, ve dünya kocayıp yok olabilir; ancak benim doğruluğum sonsuza kadar ve benim kurtuluşum nesilden nesile dayanacaktır.” “Korkuya kapılmayın, çünkü ben sizlerleyim; endişelenmeyin, çünkü ben sizin Tanrınız’ım.” “Adil bir Tanrı ve bir Kurtarıcı olarak benden başka hiçbir Tanrı yoktur.”
\vs p097 7:7 Ve şu cümleleri duymak, tıpkı bu dönemden beri binlercesine olduğu gibi, Musevi esirlerine huzur vermişti: “Tanrı söyle söyler, ‘ben seni yarattım, ben seni günahlarından kurtardım, ben seni isminle çağırdım; sen benimsin’” “Sulardan geçtiğinizde, sizlerle birlikte olacağım, zira siz benim gözümde kıymetlisiniz.” “Bir kadın, oğluna merhamet göstermeyecek kadar ağzı memedeki çocuğunu unutabilir mi? Evet, o unutabilir, fakat ben çocuklarımı unutmayacağım, zira bakın, onları ben ellerimin avuçları içine işledim; onları ellerimin gölgesiyle bile kapladım.” “Ahlaksız olanın tercih ettiği yolu bırakmasına, doğru olmayanın düşüncelerinden vazgeçmesine izin verin, onun Koruyucu’ya dönmesine izin verin, Koruyucu ona merhamet gösterecektir, Tanrımız’a dönün, zira o cömertçe bağışlayacaktır.”
\vs p097 7:8 Salem’in Tanrısı’nın bu yeni açığa çıkarılışının müjdesini tekrar dinleyin: “O sürüsünü bir çoban gibi besleyecek; koyunlarını kolları altına toplayacak ve onları göğsünde taşıyacaktır. O güçsüze güç verir, dermanı olmayanlar için kuvvetlerini arttırır. Koruyucu için bekleyenlerin kuvvetleri yenilenir; kartallar gibi kanatlara sahip olurlar; koşar ama yorulmazlar; yürürler ama bitkin hale düşmezler.”
\vs p097 7:9 9 Bu İşaya, yüce bir Yahveh’in genişleyen kavramsallaşmasına ait müjdenin uçsuz bucaksız bir örgütlü duyuruşunu gerçekleştirdi. O Musa ile, Kainatın Yaratan’ı olarak İsrail’in Koruyucu Tanrısı’nı tasvir edişteki hitabet sanatında yarışmıştı. İkinci İşaya, Kainatın Yaratıcısı’nın sınırsız niteliklerinin betimlemesinde şairaneydi. Bu dönemden beri Gökteki Yaratıcı hakkında daha güzel duyurularda bulunulmamıştır. Mezmurlar gibi İşaya’nın yazıtları; Urantia’ya Mikâil’in gelişi öncesine kadar fani insanın kulaklarını selamlamış, Tanrı’nın ruhsal kavramsallaşmasının en yüce ve doğru sunumları arasındadır. Onun şu ilahiyat tasvirini dinleyin: “Ben, ebediyette ikamet eden yüksek ve yüce biriyim.” “Ben ilk ve sonum, ve benden başka hiçbir Tanrı yoktur.” “Ve Tanrı’nın eli ne kurtaramayacak kısa, ne de kulakları duyamayacak kadar az işitir.” Ve, bu yumuşak başlı ancak emredici tanrı\hyp{}elçisi Tanrı’nın sadakati olarak kutsal bağlılığın duyuruşunda ısrar ettiğinde, Musevi topluluğu içinde bu yeni bir inanç savı haline gelmişti. O şunu duyurdu: “Tanrı unutmaz, yalnız bırakmaz.”
\vs p097 7:10 Bu cesur öğretmen, insanın Tanrı ile oldukça yakın bir biçimde ilişkili olduğunu şöyle ifade ederek duyurdu: “Benim ismimle anılan herkesi ihtişamın için yarattım, ve onlar benim övgüme layık olduklarını göstermelidirler. Ben, ben bile, ihlallerinin üzerini kendim için örten biriyim, ve ben onların günahlarını hatırlamayacağım.”
\vs p097 7:11 Bu büyük İbrani’nin; şerefle Kainatın Yaratıcısı’nın kutsallığını herkese duyururken, onun hakkında şunları söyleyerek milli bir Tanrı’ya dair kavramsallaşmayı yıkışına kulak verin: “Gökler benim tahtım, yeryüzü ayağımı koyduğum taburemdir.” Buna ek olarak İşaya’nın Tanrısı yine de kutsal, görkemli ve anlaşılamazdı. Çöl Bedevileri’nin kızgın, intikamcı ve kıskanç Yahveh’e dair kavramsallaşması neredeyse ortadan kaybolmuştur. Yüce ve herkesi kapsayan Yahveh’e dair yeni bir kavramsallaşma fani insanın aklında bir daha insan gözünden kaybolmamacasına ortaya çıkmıştır. Kutsal adaletin gerçekleşmesi, ilkel büyü ve biyolojik korkunun yıkılışını başlatmıştır. En sonunda insan, bir kanun ve düzen evrenine ek olarak güvenilebilir ve nihai niteliklere sahip evrensel bir Tanrı ile tanıştırılmıştır.
\vs p097 7:12 Ve ulvi bir Tanrı’ya dair bu duyuru, bu \bibemph{derin sevgi olarak Tanrı’yı} duyurmaya hiçbir zaman son vermedi. “Ben, yüksek ve kutsal bir yerde ikamet etmekteyim, aynı zamanda pişman ve alçak gönüllü bir ruhaniyete sahip olanla.” Ve bu büyük öğretmen daha da fazla rahatlatıcı kelimelerle çağdaşlarına hitap etmişti: “Ve Koruyucu sürekli bir biçimde size rehberlik yapacak ve ruhunuzu tatmin edecektir. Siz sulanmış bir bahçe, suları verimsiz hale gelmeyen bir bahar olacaksınız. Ve şayet düşman bir sel gelecek olursa, Koruyucu’nun ruhaniyeti ona karşı bir savunma yükseltir.” Ve bir kez daha Melçizedek’in korku\hyp{}yıkıcı müjdesi ve Salem’in güven\hyp{}aşılayan dini insanlığın kutsanışı için ışıl ışıl parıldamıştı.
\vs p097 7:13 İleri görüşlü ve cesur İşaya; sevgi dolu Tanrı, evrenin yöneticisi ve tüm insanlığın şefkatli Yaratıcısı olarak yüce Yahveh’in ihtişamı ve evrensel muktedirliğine dair muhteşem tasviriyle milli Yahveh’i etkili bir biçimde gölgede bırakmıştı. Bu dikkate değer dönemden beri, Batı’daki en yüksek Tanrı kavramsallaşması evrensel adaleti, kutsal bağışlamayı ve ebedi doğruluğu içine almıştır. Harika dille ve benzersiz incelikle bu büyük öğretmen, her şeye derin sevgi besleyen Yaratıcı olarak her şeye gücü yeten Yaratan’ı betimlemişti.
\vs p097 7:14 Esaret döneminin bu tanrı\hyp{}elçisi; Babil’deki nehir kenarında kendisini dinlerlerken insanları ve birçoklarınınkine duyurusunu gerçekleştirmişti. Ve bu ikinci İşaya, söz verilen Mesih’in görevine ait birçok yanlış ve ırksal olarak bencil kavramsallaşmaya karşı gelmek için çok şey yapmıştı. Ancak bu verilen çaba içerisinde o, tamamiyle başarılı olmamıştı. Dinadamları yanlış algılanmakta olan bir milliyetçiliği inşa etme görevine kendilerini adamış olmasalardı, iki İşaya’nın da öğretileri söz verilen Mesih’in tanınışı ve kabulü için zemin hazırlamış olacaktı.
\usection{8.\bibnobreakspace Kutsal ve Dinsiz Tarih}
\vs p097 8:1 İbraniler’e ait deneyimlerin kaydını kutsal tarih olarak ve dünyanın geri kalanında ortaya çıkmış etkileşimleri dinsiz tarih olarak görme adeti, tarihin yorumlanışında insan aklında mevcut olan kafa karışıklığının çoğunun sorumlusudur. Ve bu zorluk, Museviler’in hiçbir din\hyp{}dışı tarihinin olmaması nedeniyle doğmaktadır. Babil sürgünü dönemindeki dinadamları; Eski Ahit içinde tasvir edildiği biçimiyle İsrail’in kutsal tarihi olarak Tanrı’nın İbraniler ile yaptığı varsayılan mucizevi etkileşimlerinin yeni bir kaydını hazırladıktan sonra, dikkatli biçimde --- “İsrail Kralları’nın Yaptıkları” ve “Yehud Kralları’nın Yaptıkları” ile beraber İbrani tarihinin belli bir ölçüde doğru kayıtları halinde --- İbrani olaylarının mevcut kayıtlarını tamamen ortadan kaldırmışlardı.
\vs p097 8:2 Din\hyp{}dışı tarihin yok edici baskısı ve kaçınılmaz zorlayışının; esaret altında ve yabancılar tarafından idare edilmekte olan Museviler’i, tarihlerini baştan aşağıya yeniden yazmaya ve yeniden sunmaya giriştirecek kadar nasıl dehşete düşürdüğünü anlamak için, kafa karışıklığına iten milli deneyimlerinin kaydını kısaca irdelememiz gerekmektedir. Museviler’in, din\hyp{}kuramı içermeyen yeterli bir yaşam felsefesi geliştirmede başarısız oldukları hatırlanmak zorundadır. Onlar, günah için çok sert cezalarla beraber doğruluk için kutsal ödüllere dair özgün ve Mısırlı kavramlaşmalarıyla mücadele vermişlerdi. Eyüp’ün yaşadıkları, bu yanlış felsefeye karşı bir çeşit itirazdı. Zebur’un içten karamsarlığı, Yazgı’ya yapılan bu haddinden fazla inanışlar karşısında dünyasal nitelikte bilge bir tepkiydi.
\vs p097 8:3 Ancak yabancı yöneticilerin aşırı hükümranlığı altında geçen beş yüz yıl, sabırlı ve uzunca bir süredir çile çeken Museviler için bile gerektiğinden fazlaydı. Tanrı\hyp{}elçileri ve dinadamları şöyle haykırmaya başlamışlardı: “Daha ne kadar, ey Koruyucu, daha ne kadar?” Dürüst bir Musevi birey Kutsal Yazıtları araştırırken, kafa karışıklığı daha içinden çıkılmaz bir hal almıştı. Eskinin bir kahini, Tanrı’nın “seçilmiş insanlarını” koruyacağı ve onları kurtaracağı sözünü vermişti. Amos, milli doğruluklarının ortak ölçütlerini yeniden inşa etmezlerse Tanrı’nın İsrail’i gözden çıkaracağıyla tehdit etmişti. Tevrat yazıtı --- kutsanma ve lanetlenme biçiminde iyilik ve kötülük arasında olmak üzere --- Büyük Tercih’i tasvir etmişti. Birinci İşaya, yardımsever bir kral\hyp{}kurtarıcısının duyuruşunda bulunmuştu. Jeremiah, --- kalbin levhaları üzerine yazılmış sözleşme biçiminde --- içsel doğruluğun bir dönemini duyurmuştu. İkinci İşaya, feda ve günahlardan arınmayla gelen kurtuluştan bahsetti. Zülkifi bağlılığın hizmetiyle gelen özgürleşmeyi duyurdu, ve Ezeyir kanuna olan bağlılıkla gelen refahın sözünü verdi. Ancak tüm bunlara rağmen onlar esaret altında kalmaya devam etmiş olup, özgürlük onlar için ertelenmişti. Bunun sonrasında Danyal, --- Mesihsel krallık olarak doğruluğun sonsuza kadar sürecek egemenliğine ait ihtişamın ve anlık yerleşiminin cezalandırışı biçimindeki --- yaklaşmakta olan “buhranı” altında yaşanacakları sundu.
\vs p097 8:4 Ve beslenen bu türden boş ümidin tümü öyle bir derecede ırksal hayal kırıklığına ve hüsrana neden oldu ki, Musevi önderleri; bir kutsal Cennet Evladı yakın bir zaman içerisinde fani beden suretinde --- İnsan Evladı olarak vücuda bürünmüş halde --- kendilerine geldiğinde onun görevi ve hizmetini tanımada başarısız olacak kadar kafa karışıklığına düşmüşlerdi.
\vs p097 8:5 Tüm çağdaş dinler, insanlık tarihinin belirli çağlarına mucizevi bir yorum getirme çabası hatasına ciddi bir biçimde düşmüştür. Her ne kadar Tanrı’nın birçok kez, insan olaylarının akışına Yaratıcı’nın bir yazgısal müdahale elini uzatmış olduğu doğru olsa da; din\hyp{}kuramsal dogmaları ve dini hurafeyi insanlık tarihinin bu akışı içinde mucizevi eylemle ortaya çıkan doğa\hyp{}üstü birikim olarak görmek yanlıştır. Gerçek, “İnsanların krallığındaki En Yüksek Unsurlar’ın” din\hyp{}dışı tarihi kutsal olarak varsaydığınız tarihe dönüştürmemesidir.
\vs p097 8:6 Eski Ahit eserinin sahipleri ve daha sonraki Hıristiyan yazarları, iyi niyetli olan Musevi tanrı\hyp{}elçilerini aşkınlaştırma çabalarıyla İbrani tarihinin bozuluşunu derinleştirdiler. Böylelikle İbrani tarihi, hem Musevi hem de Hıristiyan yazarlar tarafından oldukça yıkıcı bir biçimde sömürülmüştür. Din\hyp{}dışı İbrani tarihi bütünüyle dogmalaştırılmıştır. O; bir kutsal tarih kurgusuna dönüştürülmüş olup, sözde Hıristiyan milletlerinin ahlaki kavramsallaşmaları ve dini öğretiler ile ayrılmaz hale gelmiştir.
\vs p097 8:7 İbrani tarihindeki önemli olayların kısa bir anlatımı, insanlarının gündelik din\hyp{}dışı tarihini kurgusal ve kutsal bir tarihe dönüştürecek derecede Musevi dinadamları tarafından gerçeklerin kayıtlarının nasıl değiştirildiğini aydınlatacaktır.
\usection{9.\bibnobreakspace İbrani Tarihi}
\vs p097 9:1 Orada hiçbir zaman İsrail topluluklarının on iki kabilesi bulunmamaktaydı --- Filistin’de ikamet eden yalnızca üç veya dört kabile bulunmaktaydı. İbrani milleti varlığına, İsrailoğulları olarak adlandırılan topluluklar ile Kenani unsurlarının birleşiminin sonucunda kavuştu. “Ve İsrail’in çocukları Kenaniler arasında ikamet etti. Ve İsrail’in çocukları Kenaniler’in kızlarını eşleri olarak alıp, onların evlatlarına kızlarını verdi.” Her ne kadar dinadamlarının ilgili kayıtları tereddütsüz aksini söylese de, İbraniler Kenani topluluklarını hiçbir zaman Filistin’in dışına sürmemişlerdi.
\vs p097 9:2 İsrail topluluklarının bilincinin kökenini, Ephraim’in tepe ülkesinden aldı; daha sonraki Musevi bilinci Yehud’un güney kavminde doğdu. Museviler (Yehud toplulukları) her zaman, kuzey İsrailoğulları’nın (Ephraim topluluklarının) yazılı tarihini kötülemeyi ve lekelemeyi amaçlamışlardı.
\vs p097 9:3 Gösteriş içindeki İbrani tarihi; Ürdün’ün doğuluları --- Gilead toplulukları olarak --- akran kabile üyelerine karşı Ammon unsurlarının saldırısıyla başa çıkmak için Şaul’un kuzey kavimleri bir araya getirmesiyle başlar. O üç binden biraz daha fazla sayıdaki bir ordu ile düşmanı yenmiş olup, bu cesur hareket tepe kabilelerinin kendisini kral yapmalarına yol açtı. Sürgün edilmiş dinadamları bu hikayeyi yeniden yazdıklarında Şaul’un ordusunun sayısını 330.000’e çıkarmış olup, Yehud’u savaşa katılan kabilelerin listesine eklemişlerdi.
\vs p097 9:4 Ammon topluluklarının bozgunu hemen takiben Şaul, birlikleri tarafından genel seçimle kral yapılmıştı. Hiçbir dinadamı veya tanrı\hyp{}elçisi bu olaya katılmamıştı. Ancak dinadamları kayıtlara daha sonra, Şaul’un kutsal emirler uyarınca tanrı\hyp{}elçisi Samuel tarafından krallıkla taçlandırdığını yazdı. Onlar bunu, Davud’un Yehud krallığı için bir “kutsal soy kökeni” oluşturmak amacıyla gerçekleştirdi.
\vs p097 9:5 Musevi tarihinin tüm çarpıtmaları içinde en büyüğü Davut ile ilgiliydi. Şaul’un Ammon toplulukları karşısındaki (Yahveh’e atfetmiş olduğu) zaferinden sonra, Filistin toplulukları tetiğe geçip, kuzey kavimlere karşı saldırılara başladılar. Davud ve Şaul hiçbir zaman anlaşamadı. Altı yüz kişiyle Davud bir Filistin ittifakına adım atmış olup, Esdraelon sahiline orduyla yürüdü. Gath’da Filistin toplulukları, Davud’un topraklarını terk etmelerini emretti; onlar, Davud’un Şaul’a katılmak için fikir değiştirebileceğinden korktu. Davud geri çekildi; Filistin toplulukları Şaul’a saldırıp, onu yendiler. Onlar bunu, Davud İsrail’e bu kadar sadık olmasaydı gerçekleştiremezdi. Davud’un ordusu, büyük ölçüde topluma uyum sağlayamamışlardan ve adaletten kaçanlardan meydana gelen bir biçimde hoşnutsuzluk besleyenlerin çoklu bir topluluğuydu.
\vs p097 9:6 Gilboa’daki Şaul’un trajik yenilgisi, çevre Kenani topluluklarının gözünde Yahveh’i tanrılar arasındaki en alt konumuna getirdi. Genel olarak Şaul’un mağlubiyeti Yahveh’in doğrultusundan çıkmaya atfedilebilirdi; ancak bu sefer Yehud’un yazı düzenleyicileri bunu ayinde yapılan hatalara bağladı. Onlar, Şaul ve Samuel’in tarihsel anlatılarına Davud’un krallığı için bir arka planı oluşturması bakımından ihtiyaç duymuştular.
\vs p097 9:7 Küçük ordusuyla birlikte Davud yönetim merkezini, El Halil’in İbrani olmayan şehrine kurdu. Yakın bir zaman içinde onun yurttaşları kendisini Yehud’un yeni krallığının kralı olarak duyurdu. Yehud çoğunlukla, --- Ken, Caleb, Jebus ve diğer Kenani toplulukları olarak --- İbrani\hyp{}dışı unsurlardan meydana gelmişti. Onlar sürü sahipleri olarak göçebelerdi, ve bu nedenle toprak sahipliğine dair İbrani düşüncesine bağlıydılar. Onlar, çöl kavimlerinin dünya görüşlerini benimsemişlerdi.
\vs p097 9:8 Kutsal ve dinsiz tarih arasındaki fark en iyi, Eski Ahit içinde bulunan Davud’un kral yapılışı hakkındaki iki farklı anlatım tarafından sergilenmektedir. Onun doğrudan takipçilerinin (ordusunun) kendisini nasıl kral yaptığına dair din\hyp{}dışı anlatımın bir kısmı; tanrı\hyp{}elçisi Samuel’in, kutsal emirle, Davud’u kardeşleri arasından seçişine ve resmi olarak öne sürüşüne ek olarak, detaylı ve ciddiyetle yapılan ayinler ile İbraniler üzerine kendisine dini kral makamı verilişinin ve sonrasında kendisini Şaul’un varisi olarak duyuruluşunun içinde tasvir edildiği kutsal tarihin uzun ve yaratıcılıktan uzak anlatımını ileride hazırlamış olan dinadamları tarafından, tarihi kayıtlarda yanlışlıkla unutulmuştu.
\vs p097 9:9 Birçok kez dinadamları, Tanrı’nın İsrail ile olan mucizevi etkileşimlerinin kurgusal anlatımlarını hazırladıktan sonra, kayıtlarda halihazırda bulunan yalın ve duygusal olmayan ifadeleri tamamen silmede başarısız olmuşlardı.
\vs p097 9:10 Davud kendisini siyasi bir biçimde ilk kez Şaul’un kızı ile, daha sonra zengin Edom unsuru olan Nabal’ın dul kalmış eşiyle, ve daha da sonra ise Geşhur’un kralı olan Talmai’nin kızıyla evlenerek inşa etti. O Jebus kadınlarından altı eş aldı, Hitit topluluklarından gelen kadın eş olan Batşeba daha bu kadınların arasında bile değildir.
\vs p097 9:11 Ve, bu yöntemler aracılığıyla ve bu tür insanların yaratımıyla; Ephraim topluluklarının İsraili’ne ait ortadan kaybolmaktaki kuzey krallığın mirasının ve geleneklerinin varisi olarak Davud’un, Yehud’un kutsal bir krallığını inşa edişinin kurgusu ortaya çıktı. Davud’un Yehud’a ait çok uluslu kabilesi Museviler’den daha çok karışmış haldeydi; yine de Ephraim’in ezilmiş ataları gökten inip, onu “İrail’in kralı olarak taçlandırdı.” Askeri bir tehditten sonra Davud bunun uyarınca; Jebus topluluklarıyla bir anlaşma yapıp, Yehud ve İsrail arasında ortada konumlanan güçlü duvarlarla çevrilmiş bir şehir olan Jebus’da (Kudüs’de) birleşik hale gelmiş krallığa ait kendisinin yönettiği başkenti kurdu. Filistin toplulukları uyanıp, Davud’a saldırdı. Çetin bir savaş sonrasında onlar yenilgiye uğradı, ve bir kere daha Yahveh “Meleklerin Koruyucu Tanrısı” olarak yerleşti.
\vs p097 9:12 Ancak Yahveh, mecburen, bu ihtişamın bir kısmını Kenani tanrıları ile paylaşmak zorundadır, zira Davud’un ordusunun büyük bir kısmı İbrani olmayan kökenden gelmekteydi. Ve bu nedenle tarihi kayıtlarınızda (Yehud topluluklarından olan düzenleyiciler tarafından gözden kaçırılmış bir halde) gizi açığa çıkaran şu ifade görünmektedir: “Yahveh düşmanlarıma gözlerimin önünde karşı koydu. Bu nedenle o bu yere Baal\hyp{}Perazim adını verdi.” Ve onlar bunu, Davud’un askerilerinin yüzde sekseninin Baal toplulukları olması nedeniyle gerçekleştirmişlerdi.
\vs p097 9:13 Davud, Şaul’un Gilboa’daki yenilgisini; sakinlerinin Ephraim unsurları ile daha öncesinden bir barış anlaşması yaptığı bir Kenani şehri olan Gibeon’a Şaul’un saldırışını neden göstererek açıkladı. Bundan dolayı Yahveh onu yalnız bıraktı. Şaul’un döneminde bile Davud, Filistin topluluklarına karşı Keilah’ın Kenani şehrini savunmuştu; ve bunun sonrasında o başkentini bir Kenani şehrinde konumlandırmıştı. Kenani topluluklarıyla olan anlaşmanın bütünlüğüne sadık kalan bir şekilde Davud, Şaul’un soyundan gelen yedi kişiyi Gibeon unsurlarına asılmaları için teslim etti.
\vs p097 9:14 Filistin topluluklarının yenilgisinden sonra Davud, “Yahveh’in gemisine” sahip olup, onu Kudüs’e getirdi; ve o, Yahveh’e olan ibadeti krallığı için resmi düzeye taşıdı. Davud bunu takiben; --- Edom, Moab, Ammon ve Suriyeli topluluklar olarak ---komşu kabileleri yüklü haraçlara bağladı.
\vs p097 9:15 Davud’un yolsuz siyasi düzeni; İbrani adetlerine karşı gelerek kuzeydeki arazilerin kişisel iyeliğini elde etmeye başlayıp, yakın bir zaman içinde, Filistinliler tarafından öncesinden toplanmakta olan kervan gümrüklerinin denetimini ele geçirdi. Ve bunun sonrasında, Uriya’nın öldürülmesiyle zirve noktasına ulaşan bir dizi vahşet eylemi yaşandı. Tüm yargısal itirazlar Kudüs’de karara varıldı; artık “ihtiyar heyeti” adaleti dağıtamamaktaydı. İsyanın patlak verişi hiç de şaşırtıcı değildi. Abşalom halkın ortak arzularını kullanan bir yönetici olarak tanımlanabilir; onun annesi bir Kenani’ydi. Orada, --- Solomon --- olarak Batşeba’nın oğlunun yanı sıra tahta talip yarım düzine kişi bulunmaktaydı.
\vs p097 9:16 Davud’un ölümünden sonra Solomon, tüm kuzey etkilerinin siyasi düzenini temizlemişti; ancak o, babasının yönetimine ait zorba idare ve vergilendirmenin tümünü sürdürdü. Solomon, müsrif hanedanı ve detaylı imar izlencesiyle ülkeyi iflas ettirmişti. Orada; Lübnan’ın evi, Firavun’un kızının sarayı, Yahveh’in tapınağı, kralın sarayı ve birçok şehir duvarının yeniden yapılandırılışı bulunmaktaydı. Solomon, Suriye denizcileri tarafından çalıştırılan ve tüm dünya ile ticaret içerisinde bulunan büyük bir İbrani donanması kurdu. Onun hareminin üyeleri neredeyse bine varmaktaydı.
\vs p097 9:17 Bu dönemde, Şiloh’daki Yahveh tapınağı itibarsızlaştırılmıştı; ve ülke ibadetinin tümü Jebus’da muhteşem bir saltanat mabedinde merkezi olarak konumlandırılmıştı. Kuzey krallık daha fazla gerçekleşen bir biçimde, Elohim ibadetine geri döndü. Onlar, güney krallığını haraca bağlayan bir biçimde Yehud’u daha sonra köleleştirmiş olan Firavunlar’ın lütfunu memnuniyetle deneyimlemişlerdi.
\vs p097 9:18 Orada --- İsrail ile Yehud arasında gerçekleşen savalar olarak --- iniş çıkışlar bulunmaktaydı. Dört yıllık iç savaştan ve üç hanedandan sonra İsrail, arazi ticaretine başlamış şehir zorbalarının idaresine düşmüştü. Kral Omri bile, Shemer’in yerleşkesini almaya kalkışmıştı. Ancak III. Şalmanezer Akdeniz sahilini denetim altına almaya karar verince, bu olayın sonu çabucak yaklaşmıştı. Ephraim’in Kral Ahab’ı diğer on topluluğu da bir araya getirip, Karkar’da direniş gösterdi; savaş berabere sonlanmıştı. Bu Asuri durdurulmuştu, ancak müttefiklerden büyük sayıda kayıplar verilmişti. Bu büyük kavgadan, Eski Ahit’de bile bahsedilmemektedir.
\vs p097 9:19 Kral Ahab Naboth’dan arazi satın almaya çalışınca yeni sıkıntılar baş gösterdi. Onun Fenikeli eşi, “Elohim ve kralın” isimlerinin kutsallığını tanımaması suçu nedeniyle Naboth’un arazisine el konulmasına emredecek şekilde Ahab’ın ismini evraklarda kullanarak sahtecilikte bulunmuştu. Naboth ve onun oğulları vakit kaybetmeden idam edilmişti. Tez canlı İlyas olay yerinde ortaya çıkıp, Ahab’ı Nabothlar’ın ölümünü nedeniyle kınadı. Tanrı\hyp{}elçilerin en büyüklerinden biri olarak İlyas böylelikle öğretisine, şehirlerin ülkeyi egemenliği altına alma çabasına karşı olarak Baaller’in arazi\hyp{}satıcı tutumu karşısında eski toprak adetlerinin bir savunucusu şeklinde öğretisine başlamıştı. Ancak bu köklü değişiklik, ülke sahibi Jehu’nun, Samaria’daki Baal’in tanrı\hyp{}elçilerini (emlakçılarını) yok etmek için roman kabile reisi Jehonadab ile güçlerini birleştirmesine kadar başarılı olmamıştı.
\vs p097 9:20 Yeni yaşam Yoaş’da ortaya çıkmış, ve onun oğlu Yarovam İsrail’i düşmanlarından kurtarmıştı. Ancak bu zaman zarfında Samaria’da, hasarları eski dönemlerin Davudsal hanedan üyelerinkine zorluk çıkarmış çete vari bir soylu sınıf yönetimde bulunmuştu. Devlet ve din kurumu aynı elden yönetilmekteydi. İfade özgürlüğünü baskılama girişimi İlyas, Amos ve Hosea’nın gizli yazımlarına başlamalarına neden oldu; ve bu, Musevi ve Hıristiyan İncilleri’nin gerçek başlangıcıydı.
\vs p097 9:21 Ancak kuzey krallığı, İsrail kralı gizlice Mısır kralı ile kumpas kurmak için anlaşıncaya ve Asur’a daha fazla haraç vermeyi reddedinceye kadar ortadan kaybolmamıştı. Bunun sonrasında, kuzey krallığının bütüncül dağılımının takip ettiği üç yıllık kuşatma başlamıştı. Ephraim (İsrail) böylelikle ortadan kaybolmuştu. Yehud --- “İsrail’in kalıntıları”, Museviler olarak; İşaya’nın söylediği gibi, “Ev üstüne ev ve tarla üstüne tarla ekleyerek” birkaç kişinin ellerinde arazinin toplanmasına başlamıştı. Yakın bir zaman içinde Kudüs’de, Yahveh mabediyle birlikte Baal’in mabedi ortaya çıkmıştı. Dehşetin bu egemenliği, Yahveh için otuz beş yıl seferde bulunmuş çocuk kral Yoaş tarafından başı çekilen tek\hyp{}tanrısal bir başkaldırıyla sona ermişti.
\vs p097 9:22 Bir sonraki kral Amatzya, vergi veren Edom unsurlarının ve onların komşularının başkaldırışıyla sorun yaşamıştı. Olağanüstü bir zaferden sonra o, kuzey komşularına saldırmak için yönünü değiştirdi, ve aynı olağanüstü derece içerisinde mağlup oldu. Bunun sonrasında, kırsal kesimdeki halk başkaldırdı; onlar krala suikast düzenledi, ve onun on altı yaşındaki oğlunu tahta geçirdi. Bu kişi, İşaya tarafından Uziyahu olarak adlandırılmış Azariah’dı. Uziyahu’dan sonra, işler çok daha kötüleşti, ve Yehud yüz yıl boyunca Asur’un kralına haraç vererek varlığını sürdürdü. Birinci İşaya onlara, Yahveh’in şehri olarak Kudüs’ün hiçbir zaman düşmeyeceğini söyledi. Ancak Yeremya, onun çöküşünü duyurmada tereddüt etmedi.
\vs p097 9:23 Yehud’un gerçek çöküş nedeni, çocuk bir kralın idaresi altında faaliyet gösteren siyasilerden oluşan yolsuz ve zengin bir çevre tarafından ortaya çıkmıştı. Değişmekte olan ekonomi, özel arazi uygulamaları Yahveh’in dünya görüşüne karşı olan Baal’ın ibadetine dönüşü daha makul görmüştü. Asur’un çöküşü ve Mısır’ın üstün konuma gelişi, Yehud’a bir süreliğine özgürlüğü getirmişti; ve şehir halkı yönetimi eline almıştı. Yoşiya altında onlar, yolsuz siyasetçilerin Kudüs çevresini yok etmişlerdi.
\vs p097 9:24 Ancak bu dönem; Babil’e karşı Asur’a yardım etmek için Mısır’dan sahil boyunca kuzeye hareket eden Necho’nun kudretli ordusunu engellemek için Yoşiya gitme cüretini gösterdiğinde, trajik bir sona sahip oldu. O tamamen yok edildi, ve Yehud haraç için Mısır’a bağlandı. Baal siyasi partisi Kudüs’de gücü tekrar elde etti, ve böylece \bibemph{gerçek} Mısır köleliği başlamış oldu. Bunun sonrasında, süresince Baalli dinadamlarının hanedanlığı ve dinadamlığını denetim altında tuttuğu bir süreç ortaya çıktı. Baal ibadeti, toprak verimliliğine ek olarak mülkiyet haklarıyla ilgilenen ekonomik ve toplumsal bir düzendi.
\vs p097 9:25 Necho’nun Nebukadnezar tarafından tahtan indirilişiyle birlikte Yehud; Babil idaresi altına girmiş olup, kendisine on yıllık barış dönemi verilmişti; ancak yakın bir zaman içinde o başkaldırdı. Nebukadnezar buranın topluluklarının karşısına durduğunda, Yehud unsurları Yahveh’i etkilemek için köleleri serbest bırakma gibi köklü toplumsal değişiklikleri başlattı. Babil ordusu geçici bir süreyle çekildiğinde, İbraniler sihirli köklü değişikliklerinin kendilerini kurtardığından büyük memnuniyet duydular. Bu süreç içinde Jeremiah onlara yaklaşmakta olan yıkımdan bahsetti, ve yakın bir zaman içerisinde Nebukadnezar geri döndü.
\vs p097 9:26 Ve böylece Yehud’un sonu ansızın geldi. Şehir yıkıldı, ve insanlar Babil’e taşındı. Yahveh\hyp{}Baal mücadelesi esaret ile sonuçlandı. Ve bu esaret, İsrail’in geride kalanlarının tek\hyp{}tanrı inanışına uyandırdı.
\vs p097 9:27 Babil’de Museviler; tuhaf toplumsal ve ekonomik adetlere sahip olmalarından dolayı Filistin’de küçük bir topluluk halinde yaşayamayacakları, ve eğer dünya görüşleri olduğu gibi devam ederse Musevi olmayanları dinlerine dönüştürmek zorunda kalacakları çıkarımına vardılar. Böylelikle --- Museviler’in Yahveh’in seçilmiş köleleri haline gelme zorunluluğu düşüncesi olarak --- nihai sona dair yeni kavramsallaşmaları doğdu. Eski Ahit’in Musevi dini gerçekten de, esaret boyunca Babil’de evirildi.
\vs p097 9:28 Ölümsüzlük inanış savı aynı zamanda Babil’de bütünlük kazandı. Museviler; gelecek yaşam düşüncesinin, toplumsal adalete dair müjdelerindeki vurgudan alındığını düşünmektelerdi. Bu aşamada ilk kez din\hyp{}kuramı toplum ve ekonomi biliminin yerini aldı. Din; bir insan düşünce düzeni olarak bütünlük kazanmakta, ve davranış gittikçe artan bir biçimde siyaset, toplum bilimi ve ekonomiden ayrılır hale gelmekteydi.
\vs p097 9:29 Ve böylece, Musevi insanları hakkındaki gerçeklik; kutsal tarih olarak görülen şeylerin büyük bir kısmının, sıradan dinsiz tarihin tarihsel kayıtlarından biraz daha fazlası olduğunu açığa çıkarmaktadır. Yahudilik Hıristiyanlığın içinden büyümüş olduğu topraktı, ancak Museviler mucizevi bir topluluk değildi.
\usection{10.\bibnobreakspace İbrani Dini}
\vs p097 10:1 Onların önderleri, İsrail topluluklarına; fazladan müsamaha ve kutsal lütfun tekelini vermek için değil, ancak, her milletin tamamına tek Tanrı’nın gerçekliğini taşımanın özel hizmeti için, seçilmiş bir topluluk olduklarını öğretmişti. Ancak onlar Museviler’e; eğer bu nihai sonu gerçekleştirirlerse tüm toplulukların ruhsal önderleri haline gelebileceklerini ve yolda olan Mesih’in kendileri ve tüm dünya üzerinde Barışın Prensi olarak egemenliğine sahip olacağının sözünü vermişlerdi.
\vs p097 10:2 Museviler Farslılar tarafından serbest bırakıldıklarında, Filistin’e sadece; dinadamlarının üstünlüğünde hayata geçirilmiş yasaların, fedaların ve ayinlerin düzenine olan köleliğe düşmek için dönmüşlerdi. Ve İbrani kavimler feda ve kefarete dair Musa’nın elveda konuşmasında sunulan Tanrı’nın muhteşem hikayesini reddederlerken, benzer bir biçimde İbrani milletinin bu geride kalan unsurları; ikinci İşaya’nın, onların sahip oldukları büyümekte olan dinadamlığı kurumunun yöneticileri, düzenleyicileri ve ayinlerine dair olağanüstü kavramsallaşmayı reddetmişlerdi.
\vs p097 10:3 Milli bencillik, yanlış yorumlanan haldeki söz verilmiş bir Mesih’e duyulan inanç ve dinadamlığı kurumunun artan baskısı ve zorbalığı (Danyal, Ezekiel, Haggai ve Malaçi dışında) ruhsal önderlerinin seslerini sonsuza kadar kısmıştı; ve bu dönemden Yahya’nın zamanına kadara İsrail’in tümü, artış gösteren bir ruhsal gerilimi deneyimledi. Ancak Museviler hiçbir zaman Kainatın Yaratıcısı kavramsallaşmasını kaybetmedi; İsa’dan sonraki yirminci yüzyıla kadar bile onlar, bu İlahiyat kavramsallaşmasının takibine devam ettiler.
\vs p097 10:4 Musa’dan Yahya’ya kadar orada; hiç durmadan ahlaksız yöneticilerini azarlarken, ticaretleştiren dinadamlarını kınarken ve İsrail’in Koruyucu Tanrısı olan yüce Yahveh ibadetine bağlı kalmalarını en başından beri insanlarından ciddi bir şekilde talep ederlerken, bir nesilden diğerine ışığın tektanrılı meşalesini devreden sadık öğretmenlerinin aralıksız bir soyu uzanmıştı.
\vs p097 10:5 Bir millet olarak Museviler nihai bir biçimde, siyasi aidiyetlerini kaybettiler; ancak tek ve evrensel Tanrı’ya olan İbrani dininin içten inanışı, oradan oraya atılmış olan sürgüne gönderilmişlerin kalplerinde yaşamaya devam etmektedir. Ve bu din, takipçilerinin en yüksek değerlerini muhafaza etmek için etkin bir biçimde faaliyet göstermesi nedeniyle varlığını sürdürmektedir. Musevi dini, bir topluluğun olası en yüksek düşüncelerini korumuştu; ancak o, ilerlemeyi teşvik etmede ve gerçekliğin âlemlerindeki felsefi nitelikli yaratıcı keşfi cesaretlendirmede başarısız olmuştu. Musevi dini birçok yanlışta bulunmuştu --- felsefe bakımından yetersiz ve estetik niteliklerden neredeyse tamamen yoksundu; ancak o, ahlaki değerleri muhafaza etmişti; bu nedenle varlığını sürdürmüştü. İlahiyat’ın diğer kavramsallaşmalarıyla kıyaslandığında yüce Yahveh; sınırları belirgin, keskin, kişisel ve ahlakiydi.
\vs p097 10:6 Museviler, az sayıdaki topluluğun duyumsamış olduğu gibi, adaleti, bilgeliği, gerçekliği ve doğruluğu derinden sevmişlerdi; ancak onlar, bu kutsal niteliklerin ussal kavranışına ve ruhsal anlaşılmasına tüm topluluklar içinde en az katkıda bulunmuşlardı. Her ne kadar Musevi din\hyp{}kuramı genişlemeyi reddetmiş olsa da, Hristiyanlık ve Muhammed takipçilerinin inancı olarak diğer iki dünya dininin gelişiminde önemli bir rol oynadı.
\vs p097 10:7 Musevi dini aynı zamanda, kurumları nedeniyle varlığını sürdürdü. Birbirinden ayrışmış bireylerin özel ibadeti olarak varlığını sürdürmek din için zordur. Bu durum, dini önderlerin en başından beri hatası olmuştur: Kurumlaştırılmış dinin kötülüklerini görerek onlar, toplumsal faaliyet yönetimini ortadan kaldırmayı amaçlamaktadır. Ayinlerin hepsini yok etmek yerine onlarda köklü değişikliğe giderlerse daha iyi yapmış olurlar. Bu açıdan Ezekiel, çağdaşlarından daha bilgeydi; her ne kadar, kişisel düzeydeki ahlaki sorumlulukta ısrar etmede onlara katılmışsa da, üstün ve arındırılmış bir ayinin sadık uygulamasını birinci elden yapmaya başlamıştı.
\vs p097 10:8 Ve böylece İsrail’in birbirlerini takip eden önderleri, Urantia üzerinde din evrimi içinde bu döneme kadar yerine getirilmiş en büyük mahareti gerçekleştirmişlerdi: bu; şiddetli bir biçimde patlayan Sina volkanının kıskanç ve acımasız ruhaniyeti olan yabansı kötü ruh Yahveh’e ait barbarsı kavramsallaşmanın, her şeyin Yaratanı ve tüm insanlığın sevgi dolu ve bağışlayıcı Yaratıcısı olan yüce Yahveh’in daha sonraki yüceltilmiş ve ulvi kavramsallaşmasına olan kademeli ancak sürekli olan dönüşümdür. Ve Tanrı’nın bu İbrani kavramsallaşması; Nebadon’un Mikâil’i olan, onun Evladı’nın kişisel öğretileriyle ve yaşam örneğiyle daha fazla genişlediği ve oldukça seçkin bir biçimde güçlendiği vakte kadar, Kainatın Yaratıcısı’na dair en yüksek insan tahayyülüydü.
\vs p097 10:9 [Nebadon’un bir Melçizedek unsuru tarafından sunulmuştur.]
