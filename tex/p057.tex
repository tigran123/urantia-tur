\upaper{57}{Urantia’nın Kökeni}
\vs p057 0:1 Urantia’nın ataları ve onun öncül tarihi hakkındaki kayıtları için Jerusem’in arşivlerinden alıntıları sunarken, --- yılda 365¼ günün mevcut artık yıl takvimi biçiminde --- şimdiki var olan kullanımının ölçülerinde zamanı hesaplamamız konunda yönlendirilmekteyiz. Bir kural olarak, her ne kadar kayıtlara geçmişse de, kesin yıl sürelerinin belirtilmesine girişilmeyecektir. Biz, bu tarihsel gerçeklerin sunulması için daha iyi yöntem olarak yaklaşık olarak en yakın ondalık sayıları kullanacağız.
\vs p057 0:2 Bir veya iki milyon yıl önceki bir olayı belirtirken, biz bu türden bir oluşumun; Hıristiyan dünyalarının kullandığı takvim döneminin yirminci yüzyılının ilk on yıllarından başlayarak yıl sayılarının geriye doğru sayılmasına dair tarihi kastetmekteyiz. Biz oldukça uzak bir geçmişte gerçekleşen bu olayları; böylelikle binli, milyonlu ve trilyonlu dönemler halinde bile ortaya çıkan süreçler içinde gerçekleşmiş bir biçimde tasvir edeceğiz.
\usection{1.\bibnobreakspace Andronover Nebulası}
\vs p057 1:1 Urantia’nın kökeni güneşinizden kaynaklanmakta, güneşinizin kökeni ise; Nebadon yerel evreninin fiziksel kuvvet ve maddi oluşumunun bir bileşen parçası olarak bir süre zarfında bir kere düzenlenmiş olan Andronover nebulasının çok çeşitli doğumlarından biridir. Ve bu büyük nebulanın kökenini çok uzun seneler önce, Orvonton aşkın\hyp{}evreni içindeki mekânın evrensel kuvvet\hyp{}etkisinden kaynağını almıştır.
\vs p057 1:2 Bu oluşumun başladığı zaman zarfında Cennet’in Birincil Üstün Kuvvet Düzenleyicileri uzun süreçler boyunca; Andronover nebulası olarak daha sonra düzenlenmiş, mekân\hyp{}enerjilerinin bütüncül denetimi içinde bulunmaktaydılar.
\vs p057 1:3 \bibemph{987.000.000.000} yıl önce, birliktelik içerisindeki kuvvet düzenleyicisi ve bunun sonrasında Uversa’dan gelen Orvonton dizilerinin 811.307’inci faal denetim unsuru Zamanın Ataları’na; o zamana göre Orvonton’un doğu bölgesi olan, belirli bir birim içinde maddileşme olgusunun başlaması için mekân koşullarının uygun olduğunun bilgisini vermiştir.
\vs p057 1:4 \bibemph{900.000.000.000} yıl önce orada, Uversa arşivlerinin doğruladığı bir biçimde; 811.307’inci denetim unsuru tarafından daha önceden belirlenen bölgeye bir kuvvet düzenleyicisinin ve görevlisinin gönderilmesini resmen onaylaması için aşkın\hyp{}evren hükümetine, Uversa Eşitlik Heyeti tarafından bir iznin verildiği kaydedilmiştir. Orvonton makamları, bu olası evrenin ilk kâşifini; yeni bir maddi yaratımın düzenlenmesi için göreve çağıran Zamanın Ataları’nın hükmünün uygulaması için görevlendirilmiştir.
\vs p057 1:5 Bu iznin kaydı, kuvvet düzenleyicisi ve görevlisinin; Orvonton içinde yeni bir fiziksel yaratımın ortaya çıkmasıyla sonuçlanacak bu çok uzun süren etkinliklere daha sonrasından katılacakları yer olan, doğu mekân bölgesi doğrultusunda uzun süreli seyahatleri için Uversa’dan çoktan ayrılmış olduklarını belirtmektedir.
\vs p057 1:6 \bibemph{875.000.000.000} yıl önce, 876.926’ıncı devasa Androvener nebulası yetkin bir biçimde oluşturulmuştur. Yalnızca kuvvet düzenleyicisi ve birliktelik görevlisinin mevcudiyeti, mekânın bu geniş kasırgasına şeklinde nihai olarak büyüyecek enerji hortumunun başlatılması için yeterliydi. Bu türden nebulasal döngülerin başlatılmasından sonra yaşayan kuvvet düzenleyicileri bu oluşumu, döngüsel daire düzleminin doğru açılarında sadece bırakmış olup; bu zaman zarfından beri enerjinin içkin nitelikleri, bu türden yeni bir sistemin ilerleyici ve düzensel evrimini gerçekleştirmektedir.
\vs p057 1:7 Yaklaşık olarak yine bu zaman zarfında anlatım, aşkın\hyp{}evrenin kişiliklerinin faaliyetine kaymaktadır. Cennet kuvvet düzenleyicilerin, Orvonton aşkın\hyp{}evrenine ait güç yöneticileri ve fiziksel düzenleyicilerinin faaliyeti için mekân\hyp{}enerji şartlarını hazır hale getirmesi biçimde --- gerçekte bu tarihsel oluşum gerçek başlangıç kısmına bu noktadan itibaren giriş yapmaktadır.
\usection{2.\bibnobreakspace Öncül Nebulasal Aşama}
\vs p057 2:1 Evrimsel maddi yaratımlarının tümü, döngüsel ve gaz nebulalarından doğmuştur; ve bu türden öncül nebulaların tümü, gaz mevcudiyetlerinin ilk süreci boyunca döngüseldir. Onlar yaşlanınca genel olarak sarmalsal hale gelirler; ve güneş oluşum faaliyetleri kendi doğrultusu içinde ilerleyince, --- birçok açıdan ufak güneş sisteminize benzeyen --- gezegenlerin, uyduların ve maddenin küçük topluluklarının çeşitli sayılardaki yapıları tarafından çevrelenmiş yıldız kümeleri veya devasal güneşler olarak sıklıkla görevleri sonlanır.
\vs p057 2:2 \bibemph{800.000.000.000} yıl önce Andronover yaratımı, Orvonton’un muhteşem öncül nebulalarından biri olarak oldukça iyi bir biçimde oluşturulmuştu. Yakın evrenlerin gök bilimcileri mekânın bu yeni olgusuna baktıklarında, ilgilerini çeken çok az şey görmüşlerdir. Bitişik yaratımlarda ölçülen çekim tahmini hesaplamaları, mekân yaratımlarının Andronover bölgelerinde gerçekleştiğini göstermiştir; ve onların bilgileri bu bulgunun ötesine geçmemiştir.
\vs p057 2:3 \bibemph{700.000.000.000} yıl önce Andronover sistemi, devasa boyutlar almaktaydı; ve ilave fiziksel düzenleyiciler onu çevreleyen dokuz maddi yaratıma, oldukça hızlı bir biçimde evirilen bu yeni maddi sistemin güç merkezleri için destek sağlamak ve eş güdümü yerine getirmek amacıyla gönderilmiştir. Takip eden yaratılmışlara miras bırakılmış maddenin tümü bu uzak zaman içinde; çapının en yüksek derecesine ulaştıktan sonra yoğunlaşmaya ve büzüşmeye devam edene kadar gittikçe artan bir biçimde dönmesi için, döngüsel hareketine sürdüren bu devasa uzay burgacının sınırları içinde tutulmuştur.
\vs p057 2:4 \bibemph{600.000.000.000} yıl önce Andronover enerji\hyp{}yönlendiriliş döneminin doruk noktasına erişilmiştir; bu nebula, kendisinin olası en yüksek kütlesine ulaşmıştır. Bu zaman zarfında bahse konu nebula, düzleşmiş bir küreye bir ölçüde benzer şekil içerisinde devasa bir döngüsel gaz bulutu haline bulunmaktadır. Bu oluşum, farklılaşan kütle oluşumunun ve çeşitlilik gösteren döngüsel hızın öncül dönemidir. Çekim ve diğer etkiler, mekân gazlarını düzenlenmiş maddeye dönüştürme görevlerine başlama noktasında bulunmaktadırlar.
\usection{3.\bibnobreakspace İkincil Nebulasal Aşama}
\vs p057 3:1 Bu devasa nebula; bu aşamada kademeli olarak sarmal biçimini almaya başlamakta olup, uzak evrenlerin bile gök bilimcileri tarafından açık bir şekilde görülebilen bir hale gelmektedir. Bu oluşum, nebulaların birçoğunun olağan gelişim sürecidir; güneşleri serbest bırakma ve evren inşasının görevine başlamalarından önce bu ikincil mekân nebulaları, genellikle \bibemph{sarmal olgular} olarak gözlemlenir.
\vs p057 3:2 Bu tarihin çok uzak döneminde bulunan yakın yıldız öğrencileri, Andronover nebulasının bu başkalaşımını gözlemlediklerinde; yirminci yüzyıl gök bilimcilerinin teleskoplarını uzaya doğru çevirip komşu dışsal uzayın mevcut\hyp{}çağ sarmal nebulalarına baktığında gördükleri şeylerin tıpatıp aynısını deneyimlemişlerdir.
\vs p057 3:3 Kütlenin olası en yüksek düzeyine erişildiği zaman zarfında, gazsal içeriğin çekim denetimi azalmaya başlamıştır; ve orada bu durumu takiben, ana kütlenin iki karşı kutbundan kaynağını alan iki devasa ve farklı kol biçiminde gazların dışa doğru püskürüşü oluşumunda, gaz çıkış aşaması gerçekleşmiştir. Bu devasa merkezi çekirdeğin yüksek hızlardaki dönüşleri yakın bir zaman zarfı içerisinde, bu iki gaz akıntıları bakımından sarmalsal bir görünüşü açığa çıkarmıştır. Bu fışkıran akıntı kollarına ait çıkan parçacıkların soğuması ve bunun sonrasında gerçekleşen yoğunlaşmaları, nihai olarak onların iç içe geçmiş görünüşünü meydana getirmiştir. Bu yoğun parçacıklar, ana burgacın çekim etkisi içinde güvenli bir biçimde tutulurken nebulanın gaz bulutunun ortasında uzay boyunca dönen fiziksel maddenin çok geniş sistemleri ve alt sistemleridir.
\vs p057 3:4 Ancak nebula büzülmeye başlamış ve döngünün hız derecesindeki artış çekim denetimini daha fazla azaltmıştır; ve çok geçmeden dışsal gaz bölgeleri, düzensiz yörüngenin döngüleri içinde uzaya dağılıp, döngülerini tamamlamak için çekirdeksel bölgelere geri dönerek ve bu ilerleyişi sürekli olarak takip ederek, nebulasal çekirdeğin doğrusal bütünlüğünden mevcut bir biçimde ayrılmaya başlamışlardır. Ancak bu oluşum, nebulasal ilerleyişin yalnızca geçici bir aşamasıydı. Burgaç dönüşünün sürekli artan hızı yakın bir zaman içinde bağımsız döngüler üzerinde devasa güneşleri uzaya fırlatmaya başlayacaktır.
\vs p057 3:5 Bu olaylar çağlar öncesinden Andronover’in deneyimlediği oluşumlardır. Enerji burgacı, olası en yüksek genişlemesine erişinceye kadar büyümeye devam etmiştir; ve bunun sonrasında nihai olarak büzüşmenin başladığı zaman zarfında bu burgaç, nihai olarak eşiksel merkezkaç aşaması ulaşılana ve büyük kopuş meydana gelinceye kadar gittikçe artan bir hızda dönmeye devam etmiştir.
\vs p057 3:6 \bibemph{500.000.000.000} yıl önce, ilk Andronover güneşi doğmuştur. Bu parıldayan alevli şerit, ana çekim etkisini kırmış ve yaratımın kâinatı içinde bağımsız bir serüven içerisinde uzaya doğru ayrılmıştır. Onun yörüngesi, kaçış doğrultusu tarafından belirlenmiştir. Bu türden genç güneşler çabuk bir biçimde küresel bir hale gelmekte ve uzayın yıldızları olarak uzun ve ciddi süreçlerine başlamaktadır. Dönemsel nebulasal çekirdekleri dışında Orvonton güneşlerinin çok büyük bir kısmı, benzer bir doğuma sahip olmuştur. Bu türden ayrılan güneşler, evrimin çeşitli süreçleri ve takip eden evren hizmetleri boyunca ilerlemektedir.
\vs p057 3:7 \bibemph{400.000.000.000} yıl önce, Andronover nebulasının yeniden birleşim süreci başlamıştır. Yakında bulunan ve küçük güneşlerin çok büyük bir kısmı, ana çekirdeğin kademeli büyümesinin ve daha ileri yoğunlaşmasının bir sonucu olarak yeniden yakalanmıştır. Çok yakın bir zamanda orada; enerji ve maddenin bu engin uzay bütünleşmelerinin nihai ayrışmasından her zaman daha önce gerçekleşen süreç olarak, nebulasal yoğunlaşmanın dönemsel fazı başlatılır.
\vs p057 3:8 Cennet’in bir Yaratan Evladı olarak Nebadon Mikâili’nin evren inşası için kendi serüveninin yerleşkesi niteliğinde ayrışmakta olan bu nebulayı seçmesi bu aşamanın neredeyse bir milyon yıl sonrasında gerçekleşmiştir. Neredeyse eş zamanlı olarak Salvington’un mimari dünyalarının ve gezegenlerin yüz takımyıldız yönetim merkezinin inşası başlatılmıştır. Özel olarak yaratılan dünyaların bu topluluklarının tamamlanması yaklaşık olarak bir milyon yıl almıştır. Yerel sistem yönetim merkez gezegenleri, bu zaman sürecinden fazla olarak yaklaşık beş milyar yılı bulan bir süre zarfında inşa edilmiştir.
\vs p057 3:9 \bibemph{300.000.000.000} yıl önce Andronover güneş döngüleri, oldukça iyi bir biçimde oluşturulmuştur; ve bu süreç içinde nebulasal sistem, görece fiziksel istikrarın bir geçiş döneminden geçiş yapmaktadır. Bu zaman zarfında Mikâil’in yönetim görevlileri, Salvington’a gelmiş; ve Orvonton’un Uversa hükümeti fiziksel tanınmasını Nebadon yerel evrenine kadar genişletmiştir.
\vs p057 3:10 \bibemph{200.000.000.000} yıl öncesi, Andronover merkezi kümesi veya diğer bir değişle çekirdeksel kütlesi içinde devasa bir ısı yaratımı ile birlikte gerçekleşen büzülmenin ve yoğunlaşmanın gelişimine şahit olmuştur. Görece uzay, merkezi ana\hyp{}güneş burgacının yakınında bulunan bölgelerde bile ortaya çıkmıştır. Dışsal bölgeler daha istikrarlı ve daha iyi düzenlenmiş bir hale gelmektedir; yeni doğan güneşler etrafında dönen bazı gezegenler, yaşam aktarımı için uygun hale gelmesi bakımından yeterince soğumuştur. Nebadon’un en eski yerleşik gezegenlerinin oluşumu bu tarihe rastlamaktadır.
\vs p057 3:11 Bu aşamada Nebadon’un tamamlanmış evren işleyiş biçimleri faaliyet göstermeye başlamaktadır; ve Mikâiller’in yaratımları, Uversa üzerinde yerleşimin ve ilerleyici fani yükselişin bir evreni olarak kaydedilmiştir.
\vs p057 3:12 \bibemph{100.000.000.000} yıl önce, yoğunlaşma geriliminin nebulasal zirvesine ulaşılmıştır; olası en yüksek ısı gerilim noktası erişilmiştir. Çekim\hyp{}ısı geriliminin bu ciddi eşik aşaması zaman zaman çağlar boyunca sürmektedir; ancak nihai olarak ısı çekim ile girdiği çekişmeden galip çıkmakta olup, güneş dağılımının muhteşem dönemi başlar. Ve bu oluşum, bir uzay nebulasının ikincil sürecinin sonunu simgelemektedir.
\usection{4.\bibnobreakspace Üçüncül ve Dördüncül Aşamalar}
\vs p057 4:1 Bir nebulanın birincil düzeyi daireseldir; ikincil düzeyi, sarmal biçimindedir; üçüncül düzeyi ise ilk güneş dağılımının şeklindedir; bunun karşısında dördüncül düzey, ana çekirdeğin bir küresel bulut kümesi veya dönemsel bir güneş sisteminin merkezi şeklinde faaliyet gösteren yalnız bir güneş biçiminde sonlanmasıyla, ikinci ve son güneş dağılım çevrimi ile bütünleşir.
\vs p057 4:2 \bibemph{75.000.000.000} yıl önce bu nebula, güneş ailesi düzeyin doruk noktasına erişmiştir. Bu oluşum, güneş kayıplarının ilk döneminin zirve noktasıydı. Bu güneşlerin büyük bir çoğunluğu bu zamandan beri; gezegenlerin, uyduların, karanlık adaların, kuyruklu yıldızların, göktaşlarının ve kâinatsal toz bulutlarının geniş sistemlerine kendi içlerinde sahip oldular.
\vs p057 4:3 \bibemph{50.000.000.000} yıl önce, güneş dağılımının ilk dönemi tamamlanmıştır; bu nebula, 876.926 güneş sistemine kaynaklık eden bir süreç boyunca mevcudiyetinin dördüncü çevrimini hızlı bir biçimde tamamlamaktaydı.
\vs p057 4:4 \bibemph{25.000.000.000} yıl öncesi; nebulasal yaşamın üçüncü çevriminin tamamlanışını gözlemlemiş olup, bu ebeveynsel nebuladan elde edilen uçsuz bucaksız sistemlerin düzenlenişini ve görece istikrarını beraberinde getirmiştir. Ancak fiziksel büzülme ve artan ısı üretiminin süreci, nebulasal kalıntılının merkezi kütlesi içinde devam etmiştir.
\vs p057 4:5 \bibemph{10.000.000.000} yıl önce, Andronover’in dördüncü çevrimi başlamıştır. Çekirdek\hyp{}kütlenin olası en yüksek sıcaklığı erişilmiştir; yoğunlaşmanın önemli eşiği yaklaşmaktaydı. Kökensel ana çekirdek; kendisine ait içsel\hyp{}ısı yoğunlaşma gerilimine ek olarak özgürleştirilmiş güneş sistemlerinin çevreleyen kümelenişine ait aratan çekim\hyp{}gelgit etkisinin bir araya gelen basıncı altında, şiddetle sarsılmaktaydı. İkinci nebulasal güneş çevrimini başlatacak olan patlamalar oldukça yakındı. Nebulasal mevcudiyetin dördüncü çevrimi çok yakın bir zaman zarfında başlayacaktı.
\vs p057 4:6 \bibemph{8.000.000.000} yıl önce, şiddetli nitelikte dönemsel patlama başlamıştır. Yalnızca dışsal sistemler, kâinat kapsamındaki büyük ve ani bu türden değişiklerinin gerçekleştiği zaman zarfında güvenli bir konumda bulunmaktadır. Bu nihai güneş püskürüşleri, neredeyse iki milyon yıllık bir süreci aşkın zaman zarfına kadar devam etmiştir.
\vs p057 4:7 \bibemph{7.000.000.000} yıl öncesi, Andronover’in dönemsel kopuşunun doruk noktasına şahit olmuştur. Bu süreç; daha geniş dönemsel güneşlerin doğumu olup, şiddetli yerel fiziksel oluşumların en yüksek noktasıdır.
\vs p057 4:8 \bibemph{6.000.000.000} yıl öncesi; dönemsel kopuşun sonuna ek olarak, Andronover ikinci güneş ailesinin son biriminden meydana elen elli altıncı güneş biçiminde sizin güneşinin doğuşunu simgeler. Nebula çekirdeğinin bu nihai patlaması, birçoğunun yalnız gökcisimleri olduğu 136.702 güneşi meydana getirmiştir. Andronover nebulası içinden kaynağını olan güneşler ve güneş sistemlerinin toplam sayısı 1.013.628’di. Güneş sistemi güneşi, bu açığa çıkan güneşlerin 1.013.572’ncisiydi.
\vs p057 4:9 Ve bu aşamada büyük Andronover nebulası artık, uzayın bu ana bulutundan kaynağını alan birçok güneş ve onların gezegensel aileleri dışında bir yerde mevcut bulunmamaktadır. Bu muhteşem nebulanın nihai kalıntısı hala; kırmızı bir parıltı ile yanmaya devam etmekte olup, ışığın hükümdarlarına ait iki kudretli neslin bu saygın annesi etrafında dönmekte olan yüz altmış beş dünyadan oluşan gezegensel aileyi kendisi ile ısıtmaya devam etmektedir.
\usection{5.\bibnobreakspace Monmatia’nın Kökeni --- Urantia Güneş Sistemi}
\vs p057 5:1 \bibemph{5.000.000.000} yıl önce güneşiniz; doğumuna eşlik eden yakın zamanda gerçekleşmiş patlayışın kalıntıları olarak yakın konumdaki uzayın döngü halindeki maddelerin birçoğunu kendisinde toplayarak, göreceli nitelikte yalnızlaşmış bir alevli gökcismidir.
\vs p057 5:2 Mevcut an içerisinde güneşiniz, göreceli istikrarına kavuşmuş bir haldedir; ancak onun on bir buçuk yıl güneş lekesi çevrimi, gençliğinde değişkenlik gösteren bir yıldız olduğunu ortaya çıkarmaktadır. Güneşinizin ilk zamanlarında devam eden büzüşme ve bunun sonrasında gerçekleşen ısının kademeli yükselişi, yüzeyinde devasa kasılmaları başlatmıştır. Bu dev çırpınışlar, değişkenlik gösteren parlaklığın bir çevriminin tamamlanması için üç buçuk güne ihtiyaç duymuştur. Bu dönemsel kasılma niteliğindeki bahse konu değişkenlik gösteren düzey, kısa zamanda karşılaşılacak olan belirli dış etkenlere karşı güneşinizi oldukça tepkisel kılmıştır.
\vs p057 5:3 Böylelikle bu oluşum; dünyanızın ait olduğu yerel sistem niteliğindeki güneşinizin gezegensel ailesinin ismi olarak, \bibemph{Monmatia’nın} benzersiz kökeni için hazırlanmış yerel mekân aşamasıydı. Orvonton’un gezegensel sistemlerinin yüz de birinden daha azı bu türden benzer kökene sahiptir.
\vs p057 5:4 \bibemph{4.500.000.000} yıl önce devasa Angona sistemi, bu yalnız güneşin çevresine olan yaklaşımına başlamıştır. Bu büyük sistemin merkezi; yüksek bir biçimde etkiye maruz kalan katı nitelikte ve muhteşem çekim etkisine sahip, mekânın devasa karanlık yıldızıydı.
\vs p057 5:5 Güneşsel kasılmalar boyunca olası en yüksek genişleme zamanında Angona daha yakın bir biçimde güneşe yaklaştığında, gaz maddesinin akımları devasa güneş uzantıları olarak uzaya fırlatılmıştır. İlk başta bu yanan gaz uzantıları, güneşe sürekli olarak geri dönecekti; ancak Angona gittikçe yakına geldiğinde, devasa ziyaretçinin çekim etkisi o kadar büyük bir hale gelmiştir ki bu gaz uzantılarının dışarı atılımları güneşe tekrar dönerek belirli bir noktada kesilmiştir; bunun karşısında ise güneşin dışsal bölgeleri, kendilerine ait oval yörüngeleri içinde güneş etrafında eş zamanlı olarak dönmeye başlayan güneş göktaşları biçiminde, maddenin bağımsız bünyelerini oluşturmak için ondan ayrılmışlardır.
\vs p057 5:6 Angona sistemi yakınlaştıkça, güneşsel püskürme giderek büyümüştür; gittikçe artan bir nicelikte madde, çevreleyen uzay içinde döngü halindeki bağımsız bedenler haline gelmek için güneşten çekilmiştir. Bu durum, Angona güneşe olan en yakın konumuna gelene kadar yaklaşık olarak beş yüz bin yıl boyunca gelişme göstermiştir. Bunun üzerine, kendisine ait dönemsel içsel kasılmalarından biri ile eş zamanlı olarak, güneş kısmı bir parçalanma yaşamıştır; iki zıt kutbundan ve eş zamanlı olarak maddenin devasa hacimleri dışarıya atılmıştır. Angona tarafından ise; güneşin doğrudan çekim denetiminden kalıcı bir biçimde ayrılmış hale gelen nitelikteki, iki kutbu kaplayan ve merkezde önemli ölçüde toplanan güneş gazlarının geniş bir sütunu dışarı doğru çekilmiştir.
\vs p057 5:7 Güneşten böylelikle koparılan güneşsel gazların bu büyük sütunu daha sonra, güneş sisteminin on iki gezegenine evirilmiştir. Bu devasa güneş sistemi atasının patlamasıyla birlikte gel\hyp{}git içerisinde güneşin iki kutbundan gazın nihai çekimi bu zamandan beri; her ne kadar bu maddenin oldukça büyük bir kısmı Angona sistemi uzayın derinliklerine geri çekilirken güneşsel çekim tarafından daha sonra geri yakalansa da, güneş sisteminin göktaşlarına ve uzay tozuna gelecek bir biçimde yoğunlaşmıştır.
\vs p057 5:8 Her ne kadar Angona güneş sistem gezegenlerinin atasal maddesinin çıkarımında başarılı olmuş ve maddenin devasa büyüklükteki hacmi bu aşamada küçük gezegenler ve gök taşları olarak güneş etrafında dönse de; bu güneş maddesinin hiçbir parçasını kendi içine almamıştır. Ziyaret eden sistem, güneşin özünün herhangi bir parçasını gerçek anlamıyla çalacak kadar onun yakınına gelmemiştir; ancak bu sistem, mevcut güneş sistemini meydana getiren maddenin tümünü arada kalan uzaya kadar çekmek için yeterli bir biçimde ona yakın bir konumda dönmüştür.
\vs p057 5:9 Beş iç ve beş dış gezegenler yakın bir zaman içinde, Angona’nın güneşten koparmada başarılı olduğu devasa çekim kabartısının daha az büyüklükte ve sivrilen uç konumlarında soğuyan ve yoğunlaşan çekirdeklerden küçük oluşumlar halinde meydana geldiler; bunun karşısında ise Satürn ve Jüpiter, bu yapının daha büyük kütledeki ve sivri konumlarından meydana geldikler. Jüpiter ve Satürn’ün güçlü çekim etkisi, uyduların şahit olduğu belirli gerileme hareketi olarak Angona’dan çalınan maddenin birçoğunu önceden almıştır.
\vs p057 5:10 Aşkın ısıtılmış güneşsel gazların devasa sütununun tam merkezinden meydana gelen bir biçimde Jüpiter ve Satürn; muhteşem bir parlaklıkta ışık verecek ve ısının devasa hacimlerini yayacak kadar, oldukça yüksek düzeyde ısıtılmış güneş maddesini içinde barındırmaktaydı. Güneş sistem gezegenlerinin bu iki en büyük oluşumu bu zamana kadar, tamamlanmış yoğunlaşma veya katılaşma noktasına kadar henüz soğumamış bir biçimde, gazsal olarak kalmaya devam ettiler.
\vs p057 5:11 Diğer on gezegenin sahip olduğu çekirdeklerin gaz\hyp{}büzüşümü yakın bir zaman içerisinde; katılaşma aşamasına erişmiş olup, yakın uzay içinde döngü halindeki göktaşı cisminin artan sayılarını böylelikle kendilerine doğru çekmiştir. Güneş sisteminin dünyaları böylece, göktaşlarının devasa sayılarının yakalanması vasıtasıyla daha sonra büyüten gaz yoğunlaşmasının çekirdekleri olarak, çifte bir kökene sahip olmuştur. Gerçek anlamıyla onlar, azalan sayılarda göktaşlarını yakalamaya devam etmektedir.
\vs p057 5:12 Gezegenler, güneşsel annelerinin ekvatorsal düzlemi etrafında dönmemektedir; bunun aksi bir durum eğer güneşsel dönüşleri vasıtasıyla dışarı atılmaları durumunda gerçekleşirdi. Bunun yerine onlar; güneşin ekvator düzemine olan ciddi bir yakınlıkta meydana gelen açıda, Angona güneş püskürüşünün düzlemi içinde seyahat etmektedir.
\vs p057 5:13 Angona her ne kadar güneşsel kütlenin herhangi bir parçasını yakalayamamış olsa da, sizin güneşiniz; bu ziyaret eden sistemin döngü halindeki uzay maddesinin bir parçasını kendisine ait başkalaşan gezegensel ailesine eklemiştir. Angona’nın etkili çekim alanı nedeniyle onun bağlı gezegensel ailesi, devasa karanlık yıldızdan ciddi bir uzaklıkta bulanan yörüngeyi takip etmiştir. Güneş sisteminin atasal kütlesinin püskürüşünden sonra ve Angona hali hazırda güneşin yakınında bulunurken, Angona sisteminin üç büyük gezegeni; güneşin çekim etkisiyle birleşen güneş sisteminin çekim etkisinin Angona’nın çekim etkisi karşısında eşit bir çekimle karşı koymasına ve gezinti halinde bulunan gök cisimlerinin bu üç bağlı unsurundan kalıcı bir biçimde kendisini ayırmasına yetecek kadar, bu devasa güneş sisteminin yakınında hareket etmiştir.
\vs p057 5:14 Güneşten elde edilen güneş sistemi maddesinin tümüne kökensel olarak, yörüngesel dönüşün uyumlu doğrultusu kazandırılmıştır; bu yabancı mekân bünyelerinin müdahalesi olmasaydı, güneş sistem maddesinin tümü yörüngesel hareketinin aynı doğrultusunu hala korumaya devam ederlerdi. Bu durumun meydana gelmiş olması sebebiyle Angona’nın bağlı üç gök cisminin etkisi, ortaya çıkmakta olan güneş sistemine \bibemph{gerileme hareketinin} sonuçsal dışavurumu ile birlikte yeni ve yabancı yön kuvvetlerini eklemiştir. Herhangi bir gökbilim sistemi içinde gerileme hareketi; her zaman kazasal olup, her koşulda yabancı mekân bünyelerinin çarpışmasal etkisinin bir sonucu olarak ortaya çıkmaktadır. Bu türden çarpışmalar her zaman gerileme hareketi ortaya çıkarmayabilir, ancak çeşitli kökenlere sahip olan gök kütlelerini taşıyan bir sistem dışında herhangi bir gerileme hareketi hiçbir zaman ortaya çıkmamaktadır.
\usection{6.\bibnobreakspace Güneş Sistemi Aşaması --- Gezegen Oluşum Dönemi}
\vs p057 6:1 Güneş sisteminin doğumunu takiben, azalan niceliklerde bulunan güneş püskürmelerin bir süreci gerçekleşir. Sayıları azalan bir biçimde, bir diğer beş yüz bin yıllık süreç içerisinde, güneş; küçülen hacimlerdeki maddeyi çevreleyen uzaya atmaya devam eder. Ancak değişken yörüngelerin bu ilk zamanları boyunca, çevreleyen cisimler güneşe olan en yakın yaklaşımlarında bulundukları zaman, güneşsel ebeveyn bu göktaşı maddesinin büyük miktarlarını yeniden bünyesinde yakalamaya yetkin bir halde bulunmuştur.
\vs p057 6:2 Güneşe en yakın olan gezegenler, gel\hyp{}git etkisi sebebiyle dönüşleri yavaşlayacak olan ilk gök cisimleri olmuştur. Bu türden çekimsel etkiler aynı zamanda; Urantia’ya her zaman aynı yüzünü dönen Merkür gezegeni ve ay tarafından dışa vurulduğu gibi, gezegenin bir yarıküresini her zaman güneş tarafına dönen bir biçimde bırakarak, eksensel dönüş sonlanana kadar bir gezegenin dönüşünün giderek artan bir biçimde azalmasına sebep olan bir biçimde, gezegensel\hyp{}eksen dönüşünün hızı üzerinde bir fren olarak faaliyet göstererek, gezegensel yörüngelerin istikrarına katkıda bulunur.
\vs p057 6:3 Ay ve dünyanın gel\hyp{}git etkileri birbirine eşit hale geldiğinde, dünya her zaman aynı yarıküresini aya dönecektir; ve gün ve takvimsel ay süreci --- yaklaşık bir biçimde kırk beş gün uzunluğunda olarak --- birbirine benzer hale gelecektir. Yörüngelerin bu türden istikrarı erişildiğinde, gel\hyp{}git etkileri; ayı artık dünyadan daha uzağa doğru sürüklemeyecek bunun yerine kademeli olarak bu uyduyu gezegene doğru çekecek bir biçimde, tersi yönde hareketine geçecektir. Ve bunun sonrasında, uzak bir zaman içinde, ay dünyanın on bin mil yakınlarına geldiğinde dünyanın çekim etkisi ayı etkilemeye başlayacaktır; ve bu gel\hyp{}git çekim patlaması, Satürn’ün sahip olduğu halka maddesine benzer bir biçimde dünya etrafında toplanarak veya kademeli bir biçimde göktaşları olarak dünya tarafından çekilecek bir biçimde, ayı küçük parçacıklara ayıracaktır.
\vs p057 6:4 Eğer uzay bünyeleri birbirine benzer büyüklükte ve yoğunlukta ise, çarpışmalar gerçekleşebilir. Ancak benzer yoğunluğa sahip iki uzay cismi büyüklük bakımından göreceli olarak birbirine eşit değilse, eğer küçük olan artan bir biçimde daha büyük olana yaklaşıyorsa, küçük olan cismin patlaması; bu cismin yörüngesinin yarıçapı, büyük olan cismin yarıçapının iki buçuk katından daha az hale geldiğinde gerçekleşecektir. Mekânın devasa yıldızları arasındaki çarpışmalar gerçekte nadiren gerçekleşmektedir; ancak küçük cisimlerin çekim gel\hyp{}git patlamaları oldukça sıklıkla görünmektedir.
\vs p057 6:5 Düşen yıldızlar oldukça bol bir biçimde meydana gelmektedir, çünkü onlar; yakında bulunan ve daha büyük uzay cisimleri tarafından uygulanan gel\hyp{}git çekimi tarafından parçalanan maddenin büyük cisimlerine ait kalıntılardır. Satürn’ün halkaları, parçalanmış bir uydunun kalıntılarıdır. Jüpiter’in sahip olduğu uydulardan biri gel\hyp{}git etkisinin ciddi eşiğine mevcut an içerisinde tehlikeli bir biçimde yaklaşmaktadır; ve birkaç milyon yıl içerisinde bu cisim, ya bir gezegen tarafından yok olacaktır veya çekim gel\hyp{}git parçalanma sürecine girecektir. Güneş sisteminin beşinci gezegeni çok uzun bir süre önce; çekim gel\hyp{}git parçalanmasının ciddi eşiğine girene kadar dönemsel olarak Jüpiter’e giderek yaklaşma biçiminde düzensiz bir yörüngede hareket etmiş olup, bu eşik sonrasında aniden parçalarına ayrılmış ve mevcut an içerisindeki asteroit kümesi haline gelmiştir.
\vs p057 6:6 \bibemph{4.000.000.000} yıl öncesi; birkaç milyar yıl boyunca büyüklük bakımından artmaya devam etmiş uyduları dışında, bu günkü gibi görülen Jüpiter ve Satürn sistemlerinin düzenlemesine şahit olmuştur. Gerçekte güneş sistemine ait gezegenler ve uyduların tümü, devam eden göktaşların yakalanmalarının sonucu olarak hâlihazırda büyümeye devam etmektedir.
\vs p057 6:7 \bibemph{3.500.000.000} yıl önce, diğer on gezegenin yoğunlaşması çok yerinde bir biçimde gerçekleşmişti; ve her ne kadar küçük olan uyduların bazıları daha sonra mevcut anda bulunan daha büyük uydular ile bütünleşmiş olsa da, uyduların birçoğunun çekirdekleri bütüncül haldeydi.
\vs p057 6:8 \bibemph{3.000.000.000} yıl önce, güneş sistemi mevcut an içerisinde hareket ettiği gibi faaliyet göstermekteydi. Onun üyeleri, uzay göktaşlarının gezegenlere ve onların uydularına olan şaşılacak bir derecedeki akımları devam ederken büyüklük bakımından genişlemeye devam etmiştir.
\vs p057 6:9 Bu zaman zarfında güneş sisteminiz; Nebadon’un fiziksel kaydına girmiş olup, kendisine Monmatia ismi verilmiştir.
\vs p057 6:10 \bibemph{2.500.000.000} yıl önce gezegenler büyüklük bakımından çok devasa ölçülerde genişlemiştir. Urantia bu zaman zarfında; mevcut kütlesinin yaklaşık olarak on katı büyüklüğünde çok iyi gelişmiş bir âlem olup, büyüklüğü göktaşsal yığılımı vasıtasıyla hızla hala artmaktaydı.
\vs p057 6:11 Bu devasa etkinliğin tümü; Urantia’nın düzeyi üzerinde bir evrimsel dünyanın oluşumuna ait olağan bir süreç olup, zamanın yaşam serüvenlerini hazırlamasında mekânın bu türden dünyalarının fiziksel evriminin başlaması için gökbilimsel ön hazırlıkları meydana getirir.
\usection{7.\bibnobreakspace Göktaşsal Çağ --- Volkanik Dönem\\İlkel Gezegensel Atmosfer}
\vs p057 7:1 Bu öncül zamanlar boyunca güneş sisteminin mekân bölgeleri, küçük parçalayıcı ve yoğunlaşmış uzay cisimleri ile dolup taşmaktaydı; koruyucu bir yanma atmosferinin yokluğunda bu türden mekân cisimleri, doğrudan bir biçimde Urantia’nın yüzeyine çarpmıştır. Bu aralıksız gerçekleşen etkiler, gezegen yüzeyini nihai olarak sıcak tutmuş; ve âlemin çekim etkisinin artması ile birlikte bu durum, demir gibi daha ağır olan kimyasal elementlerin kademeli olarak gezegenin merkezine doğru daha çok hareket etmesine sebep olan bir biçimde bu değişimleri gerçekleştirmiştir.
\vs p057 7:2 \bibemph{2.000.000.000} yıl önce dünya, kesin bir biçimde büyüklük bakımından ayı geçmeye başlamıştır. Bu gezegen her zaman uydusundan daha fazla bir büyüklüğe sahip olmuştur; ancak, dünya tarafından devasa uzay cisimlerinin yakalandığı an olan bu zaman zarfına kadar büyüklük bakımından aralarında çok büyük bir fark bulunmamaktaydı. Urantia bu zaman zarfında; mevcut büyüklüğünün beşte biri kadar olup, ısıtılmış iç çekirdeği ve soğuyan kabuğu arasında gerçekleşen içsel nitelikteki temel karşıtlığın bir sonucu olarak ortaya çıkmaya başlayan ilkel atmosferi tutacak kadar büyük bir konuma gelmiştir.
\vs p057 7:3 Belirli volkanik faaliyetin gerçekleşme süreci bu zaman zarfına uzanmaktadır. Dünyanın iç ısısı, göktaşları olarak uzaydan alınan radyoaktif veya diğer bir değişle ağır kimyasal elementlerin daha derinlere doğru gömülümü vasıtasıyla artmaya devam etmiştir. Bu radyoaktif elementlerin çalışması, yüzeyi bakımından Urantia’nın bir milyar yıldan daha yaşlı olduğunu ortaya çıkaracaktır. Radyum saati, bu gezegenin yaşına dair bilimsel tahminlerin yerine getirilmesi için sahip olduğunuz en güvenilir zaman ölçerdir; ancak bu türden tahminlerin tümü haddinden daha fazla bir biçimde kısa bir süreci göstermektedir, çünkü incelemenize açık olan radyoaktif maddelerin tümü dünyanın yüzeyinden alınmış olup bu nedenle göreceli olarak yakın zamanda meydana gelen Urantia’nın bu elementlerin kazanımlarını temsil etmektedir.
\vs p057 7:4 \bibemph{1.500.000.000} yıl önce ay mevcut andaki kütlesine yaklaşırken, dünya şu anda sahip olduğu büyüklüğünün üçte ikisi kadardı. Büyüklük bakımından dünyanın aya kıyasla çok daha hızlı genişlemesi, ayın kökensel olarak sahip olduğu küçük atmosferin dünya tarafından yavaşça gerçekleşen çalınışına izin vermiştir.
\vs p057 7:5 Volkanik faaliyet bu aşamada zirve noktasında bulunmaktadır. Dünyanın bütünü, ağır metallerin merkeze doğru çekilmesinden önceki konumda bulunan yüzeyin öncül eriyik düzeyde varoluşuna benzeyen bir biçimde, gerçek anlamıyla alev dolu bir cehennemdir. \bibemph{Bu düzey, volkanik çağdır.} Yine de, göreceli olarak başlıca daha hafif granitten meydana gelen bir kabuk kademeli bir biçimde oluşmaya başlamaktadır. Bu aşama, ilerde bir zaman zarfında yaşamı destekleyebilecek olan bir gezegen için oluşturulmaktadır.
\vs p057 7:6 İlkel gezegensel atmosfer; bu aşamada bir miktar su buharı, karbon monoksit, karbondioksit ve hidrojen klorür taşıyarak, yavaşça evirilmektedir; ancak orada özgür nitrojen veya özgür oksijen neredeyse hiçbir biçimde bulunmamaktadır. Bir dünyanın atmosferi volkanik çağ içerisinde tuhaf bir görüntü yansıtmaktadır. Yukarıda sıralanan gazlara ek olarak atmosfer; sayısız volkanik gazlara ek olarak hava kemer oluşumları niteliğinde gezegensel yüzeye doğru sürekli bir biçimde fırlayan ağır göktaşı yağmurlarının patlama parçacıkları tarafından yoğun bir biçimde etkilenmektedir. Bu türden göktaşı patlamaları; atmosfersel oksijeni neredeyse yok olma düzeyine getirmekte olup, onların göktaşı bombardımanı hala devasa düzeylerde bulunmaktadır.
\vs p057 7:7 Bu aşamada atmosfer, gezegenin sıcak kayalık yüzeyi üzerinde yağmur yağışının başlaması için yeteri miktarda oluşumunu tamamlamış ve soğumuştur. Binlerce yıl boyunca Urantia, buharın engin ve devasa bir yorganı altında kaplanmış bulunmaktaydı. Ve bu çağlar boyunca güneş, dünyanın yüzeyi üzerinde hiçbir şekilde parıldamamıştır.
\vs p057 7:8 Atmosferin sahip olduğu karbonun birçoğu, gezegenin yüzeysel tabakaları içinde bolca bulunan çeşitli metallerin sahip olduğu karbonatlardan elde edilmiştir. Daha sonra bu karbon gazlarının çok daha büyük nicelikleri, öncül ve verimli bitki yaşamı tarafından tüketilmiştir.
\vs p057 7:9 Daha sonraki dönemler içinde bile devam eden lav akıntıları ve gelmekte olan göktaşları, havanın oksijeninin neredeyse bütününü kullanmıştır. Yakın bir zamanda meydana gelecek olan ilkel okyanusun ilk birikintileri bile renkli kayalara veya şistlere sahip olmamıştır. Ve uzun bir süre boyunca bu ilk okyanusun ortaya çıkmasından sonra, atmosfer içinde neredeyse hiçbir özgür oksijen bulunmamaktaydı; ve oksijen, su yosunları ve bitki yaşamının diğer türleri tarafından daha sonra üretilene kadar önemli miktarlarda ortaya çıkmamıştır.
\vs p057 7:10 Volkanik çağın ilkel gezegensel atmosferi, göktaşı yağmurlarının çarpışmasal etkilerinden çok az bir korunumu sunmaktaydı. Göktaşlarının milyonlarcası, katı cisimler olarak gezegensel kabuğa çarpmak için bu türden bir hava kemerinin içinde geçmeye yetkin bulunmaktaydı. Ancak zaman ilerledikçe, gittikçe azalan miktarlardaki büyük cisimler, daha sonraki dönemlerin oksijen bakımından zengin atmosferin sürekli güçlenen sürtünme kalkanına karşı koyar hale gelmiştir.
\usection{8.\bibnobreakspace Kabuksal İstikrar\\Depremler Çağı\\Dünya Okyanusu ve İlk Kıta}
\vs p057 8:1 \bibemph{1.000.000.000} yıl öncesi, Urantia tarihinin mevcut bir biçimde başlama zamanıdır. Bu gezegen, yaklaşık olarak mevcut büyüklüğüne erişmiştir. Ve bu zaman zarfında Nebadon’un fiziksel kayıtlarına girmiş olup, bu âleme \bibemph{Urantia} ismi verilmiştir.
\vs p057 8:2 Atmosfer, aralıksız meydana gelen nem yağışı bile birlikte, dünya kabuğunun soğumasını gerçekleştirmiştir. Volkanik hareket, öncül bir biçimde iç\hyp{}ısı basıncını ve kabuksal büzülüşü eşitlemiştir; ve volkanların sayısı hızlı bir biçimde azalırken, depremler kabuksal soğumanın ve onun uyumlu hale gelişinin bu çağı ilerledikçe dışavurumlarını sergilemişlerdir.
\vs p057 8:3 Urantia’nın gerçek yeryüzü tarihi, ilk okyanusun oluşmasına yetecek kadar dünya kabuğunun soğumasıyla başlamaktadır. Dünyanın soğuyan yüzeyi üzerinde su\hyp{}buharının yoğunlaşması bir kez başladığında, neredeyse tamamlanana kadar faaliyetine devam etmiştir. Bu sürecin sonucunda okyanus, bir milin üstünde ortalama bir derinlikte gezegenin tamamını kaplayan bir biçimde, dünya\hyp{}çapında bulunan bir nitelikteydi. Bunun sonrasında gel\hyp{}gitler mevcut an içerisinde gözlendiği gibi faaliyet içerisindeydi; ancak bu ilkel okyanus tuzlu değildi; o dünyayı neredeyse tamamen temiz suyla kaplamaktaydı. Bu zaman süreçleri içinde klorürün birçoğu çeşitli metaller ile birleşmişti; ancak orada bu suyu hafif bir biçimde asidik hale getirecek hidrojen birlikteliği mevcut bulunmaktaydı.
\vs p057 8:4 Bu uzak çağın açılmasıyla birlikte Urantia bu aşamada, su ile çevrilmiş bir gezegen olarak düşünülmelidir. Daha sonra, daha derin ve böylece daha yoğun lav akımları mevcut Büyük Okyanus’un tabanına eklenmekte olup, su ile kaplı yüzeyin bu kısmı ciddi bir biçimde aşağıya doğru itilmiş bir hale gelmiştir. İlk kıtasal kara kütlesi dünya okyanusundan, kademeli olarak dünya kabuğunun kalınlaşma dengesinin karşılıksal düzenlemeleri içinde ortaya çıkmıştır.
\vs p057 8:5 \bibemph{950.000.000} yıl önce Urantia, büyük bir kıta karasına ilaveten Büyük Okyanus olarak büyük bir su bedeninin görünümünü yansıtmaktadır. Volkanlar geniş çaplı bir biçimde görünmekte olup, depremler ile birlikte sık ve ciddi bir biçimde gerçekleşmektedir. Göktaşları dünyaya çarpmaya devam etmektedir; ancak onların bu bombardımanı, sıklık ve büyüklük bakımından azalma göstermektedir. Atmosfer açılmaya başlamaktadır; ancak karbondioksitin miktarı büyümeye devam etmektedir. Dünya kabuğu kademeli olarak istikrar kazanmaktır.
\vs p057 8:6 Urantia’nın Satania sistemine gezegensel idare için atanması ve onun Norlatiadek’in yaşam kaydına alınması bu zaman zarfında gerçekleşmiştir. Bunun sonrasında orada; üzerinde takip eden süre içinde Mikâil’in fani bahşedilişin muhteşem görevine katılacağı, ve bu andan itibaren “haçın dünyası” olarak yerel bir biçimde bilenen hale gelmesine neden olacak deneyimlerin içinde bulunacağı, küçük ve kayda değer ölçüde bulunmayan bu âlemin idari tanınması başlamıştır.
\vs p057 8:7 \bibemph{900.000.000} yıl öncesi, gezegende incelemelerde bulunması ve onun bir yaşam deneyim düzeyine olan uyumu hakkında durum değerlendirilmesinde bulunması için Jerusem’den gönderilen Satania’nın ilk keşif topluluğunun Urantia’ya ulaşmasına şahit olmuştur. Bu heyet; Yaşam Taşıyıcıları, Lanonandek Evlatları, Melçizedekler, yüksek meleklere ek olarak gezegensel düzenleme ve idarenin öncül zamanları ile ilgili göksel yaşamın diğer düzeyleriyle bütünleşen, yirmi dört üyeden meydana gelmiştir.
\vs p057 8:8 Gezegenin bu zorlu bir incelenişinde bulunduktan sonra bu heyet; Jerusem’e dönüp, Urantia’nın yaşam deneyim kaydına alınmasını tavsiye eden bir biçimde Sistem Egemeni’ne olumlu görüş bildiriminde bulunmuştur. Dünyanız bunun uyarınca Jerusem üzerinde bir ondalık gezegeni olarak kaydedilmiştir; ve Yaşam Taşıyıcıları, bunun takip eden süreç içerisinde yaşam nakli ve aktarımları için bu gezegene ulaştıklarında mekanik, kimyasal ve elektriksel yönlendirmelerin yeni yöntemlerini yerine getirmek için izne sahip oldukları hakkında bilgilendirilmişlerdir.
\vs p057 8:9 Bu zaman zarfında gezegensel oluşum için düzenlemeler; Jerusem’in on iki unsurundan oluşan karma heyet tarafından tamamlanmış, Edentia üzerindeki yetmiş unsurdan oluşan gezegensel heyet tarafından onaylanmıştır. Yaşam Taşıyıcıları’nın danışma karar yardımcıları tarafından önerilen bu tasarımlar, nihai olarak Salvington üzerinde kabul edilmiştir. Bunun sonrasında yakın bir zaman içerisinde Nebadon yayıncıları; Urantia’nın, Nebadon yaşam biçimlerinin Satania türünü genişletmek ve onu geliştirmek için tasarlanan altmışıncı Satania deneyimlerini Yaşam Taşıyıcıları’nın üzerinde yerine getireceği düzlem haline geleceğine dair duyuruyu taşımıştır.
\vs p057 8:10 Urantia’nın Nebadon’un tümü için evren yayınları içinde ilk kez tanınmasından kısa bir süre sonra, ona bütüncül evren düzeyi verilmiştir. Bunun sonrasında yakın bir zamanda gerçekleşen bir biçimde Urantia, aşkın\hyp{}evrenin azınlık ve çoğunluk birim yönetim merkezlerinin kayıtlarına alınmıştır; ve bu çağın sonlanmasından önce Urantia, Uversa’nın gezegensel yaşam kaydında kendine yer elde etmiştir.
\vs p057 8:11 Bu çağın bütünü, sık ve oldukça şiddetli fırtınalar tarafından nitelenmektedir. Dünyanın öncül kabuğu, sürekli gerçekleşen değişimin bir durumu içinde bulunmaktaydı. Dünyanın yüzeyi içerisinde bu kökensel gezegen kabuğuna dair hiçbir şey geride kalmamıştı. Bu kabuğun tümü, derin köklerden gelen lav fışkırmaları ile birçok kez karışmış ve dünya çapında mevcut olan öncül okyanusun daha sonraki birikintileri ile bütünleşmiştir.
\vs p057 8:12 Hudson körfezi etrafında Kanada’nın kuzeydoğusu haricinde, bu eski okyanus\hyp{}öncesi kayaların değişikliğe uğramış kalıntıları dünya yüzeyinin hiçbir yerinde artık bulunamamaktadır. Bu geniş granit yükselişi, okyanus\hyp{}öncesi çağlara ait olan kayalıktan meydana gelmiştir. Bu kayalık tabakaları ısıtılmış, eğilmiş, bükülmüş ve yukarı doğru büzülmüştür; ve bu oluşumlar, zarar verici başkalaşımsal deneyimler boyunca sürekli olarak tekrarlanmıştır.
\vs p057 8:13 Okyanussal çağlar boyunca, fosili barındırmayan tabakalaşmış kayanın devasa katmanları bu eski okyanus tabanında birikmiştir. (Kireçtaşı, kimyasal yağışın bir sonucu olarak oluşabilir; ancak daha eski olan kireç taşlarının tümü deniz\hyp{}yaşam birikimi vasıtasıyla üretilmemiştir.) Bu eski kayalık oluşumlarının hiçbiri içinde yaşam izleri bulamaz; onlar, su çağlarının daha sonraki birikimlerinin daha eski yaşam\hyp{}öncesi katmanlar ile bütünleşmesinin şans eseri gerçekleşmesi dışında, hiçbir fosili taşımamaktadırlar.
\vs p057 8:14 Dünyanın öncül kabuğu, oldukça istikrarsız bir konumdaydı; ancak bu süreç içerisinde dağlar, oluşum aşamasında bulunmamaktaydı. Gezegen, tıpkı oluşum sürecinde gerçekleştiği gibi, çekim basıncı altında büzüştü. Dağlar, büzüşen bir âlemin soğuyan kabuğunun çökmesi sonunca oluşmamaktadır; onlar daha sonra yağmur, çekim ve toprak kayması faaliyetinin bir sonucu olarak ortaya çıkmaktadır.
\vs p057 8:15 Bu çağın kıtasal kara kütlesi, dünya yüzeyinin neredeyse onda birini kaplayana kadar artmıştır. Ciddi depremler, karanın kıtasal kütlesinin denizden oldukça yukarı doğru çıkışına kadar başlamamıştır. Depremler bir kere ortaya çıkmaya başladığında, sıklık ve ciddilik bakımından çağlar boyunca artış göstermiştir. Milyonlarca yıl boyunca depremlerin sayısı azalmıştır, ancak Urantia hali hazırda ortalama olarak günde on beş depreme sahiptir.
\vs p057 8:16 \bibemph{850.000.000} yıl önce, dünya kabuğunun istikrarlı hale gelişinin ilk gerçek çağı başlamıştır. Daha ağır olan metallerin birçoğu, dünyanın merkezine doğru yerleşmiştir; soğuyan kabuk, geçmiş çağlarda olduğu gibi geniş bir ölçek içinde gerçekleşen çöküşüne son vermiştir. Kara çıkışı ve daha ağır okyanus tabanı arasında daha iyi kurulan bir denge oluşturulmuştur. Alt\hyp{}kabuk lav yatağının akışı neredeyse dünya çapında görülen bir hale gelmiştir; ve bu oluşum, soğuma, büzüşme ve yüzeysel kaymadan doğan dalgalanmaları telafi etmiş ve onları dengeli hale getirmiştir.
\vs p057 8:17 Volkanik patlamalar ve depremler, sıklık ve ciddilik bakımından azalmaya devam etmiştir. Atmosfer, volkanik gazlar ve su buharından arınmaktadır; ancak karbondioksitin oranı hala çok yüksek bir düzeyde bulunmaktadır.
\vs p057 8:18 Hava ve yeryüzü üzerindeki elektriksel rahatsızlıklar aynı zamanda azalma göstermektedir. Lav akışları, kabuğu çeşitli hale getiren ve onu belirli mekân\hyp{}enerjilerine karşı koruyan elementlerin bir karışımını yüzeye çıkarmıştır. Ve bütün bunların tümü; manyetik kutupların faaliyeti tarafından dışa vurulan bir biçimde, karasal enerjinin denetimini yerine getirmek ve onun akışını düzenlemekle ilgili olmuştur.
\vs p057 8:19 \bibemph{800.000.000} yıl öncesi, artan kıtasal ortaya çıkışın çağı olarak ilk büyük kıta çağının başlayışını deneyimlemiştir.
\vs p057 8:20 Dünyanın sahip olduğu hidrosferin yoğunlaşmasından beri, ilk olarak dünya okyanusun ve daha sonra Büyük Okyanus’a doğru, suyun bu geniş kütlesinin dünya yüzeyinin onda dokuzunu kaplamış olduğu bu aşama için düşünülmelidir. Denize düşen göktaşları okyanus tabanı içinde birikmektedir; ve göktaşları, genel olarak, ağır maddelerden meydana gelmektedir. Kara üzerine düşen göktaşları büyük ölçüde oksitlenmiş, bunun sonrasında toprak kayması tarafından ufalanmış ve okyanus havzasına dökülmüştür. Böylelikle okyanus tabanı artan bir biçimde ağır hale gelmiştir; ve bu duruma ek olarak, bazı yerlerde on mil derinliğini bulan bir su kütlesinin ağırlığı mevcut bulunmuştur.
\vs p057 8:21 Büyük Okyanus’un aşağı yönlü artan itiş etkisi, kıtasal kara kütlesinin daha fazla yukarı doğru çıkışını sağlamıştır. Büyük Okyanus’un yatağı, ilave bir karşılıksal batış düzenlemesi içine girerken; Avrupa ve Afrika’nın, Avustralya, Kuzey ve Güney Amerika ve Antarktika kıtası ismiyle mevcut an içerisinde adlandırılan bu kütleler ile birlikte Büyük Okyanus’un derinliklerinden yükselişi başlamıştır. Bu sürecin sonunda dünya yüzeyinin neredeyse üçte biri, tek bir kıtasal beden biçiminde karadan oluşmuş bir halde bulunmaktaydı.
\vs p057 8:22 Kara yükselişin bu artışıyla birlikte, gezegenin ilk mevsimsel farklılıkları ortaya çıkmaya başlamıştır. Kara yükselişi, kâinatsal bulutlar ve okyanussal etkiler mevsimsel dalgalanmaların baş etkenleridir. Asya kara kütlesinin omurgası, olası en yüksek kara ortaya çıkışı zamanı içinde neredeyse dokuz mil yüksekliğine ulaşmıştır. Bu rakımı büyük yükselti bölgesi üzerinde gezinen hava içerisinde daha fazla nem bulunsaydı, devasa buz yorganları ortaya çıkmış olurdu; bu çağı gerçekleştiğinden çok daha önce ortaya çıkardı. Bu türden karanın tekrar suyun üstüne çıkmasından önce birkaç yüz bin yıl geçmiştir.
\vs p057 8:23 \bibemph{750.000.000} yıl önce, kıtasal kara kütlesi içinde ilk kopuşlar büyük kuzey ve güney çatlakları olarak başlamıştır; bu çatlaklar daha sonra okyanus sularını içine almış olup, Grönland adası dâhil olmak üzere Kuzey ve Güney Afrika kıtalarının batı yönlü ayrılışının zeminin hazırlamıştır. Bu uzun doğu\hyp{}ve\hyp{}batı çatlağı; Afrika’yı Avrupa’dan koparmış olup, Avustralya, Büyük Okyanus Adaları ve Antarktika’nın kara kütlelerini Asya kıtasından ayırmıştır.
\vs p057 8:24 \bibemph{700.000.000} yıl önce Urantia, yaşam sağlamak için elverişli şartların olgunlaşması düzeyine yaklaşmaktaydı. Bu kıtasal kara çatlağı yarılışına devam etmiştir; artan bir biçimde okyanus, deniz hayatı için bir yaşam alanı olarak oldukça elverişli sığ sular ve kapalı körfezleri sağlayarak parmak uzunluğunda denizler biçiminde karaya doğru giriş yapmıştır.
\vs p057 8:25 \bibemph{650.000.000} yıl öncesi, kara kütlelerinin ilave ayrılışına ve bunun sonucunda gerçekleşen kıtasal denizlerin ek bir genişlemesine şahit olmuştur. Ve bu sular hızlı bir şekilde, Urantia yaşamı için hayati derecede önemli olan tuzluluk düzeyine erişmektedir.
\vs p057 8:26 Oldukça iyi korunmuş kaya tabakaları içinde, katman katman, bir dönemin diğerini takip ettiği bir çağın bir diğeri üzerine doğduğu biçimde daha sonradan keşfedilen şekliyle, Urantia’nın yaşam kayıtlarını bırakan bu denizler ve onları takip eden su kütleleridir. Eski zamanların bu iç denizleri gerçek anlamıyla evrimin beşiğiydi.
\vs p057 8:27 [Kökensel Urantia Birliği’nin bir üyesi konumunda bulunmuş ve mevcut an içerisinde bir yerleşik gözlemci olan, bir Yaşam Taşıyıcısı tarafından sunulmuştur.]
